\bookheader{Amos}
\labelbook{Amos}

\bookpretitle{The Book of the Prophet}
\booktitle{Amos}

\labelchapt{1}
\passage{Amos is Called to Prophesy}

\chapt{1}
\v{1}The words of Amos,\fnote{\fbackref{1:1} The Heb. name \fbib{Amos} means \fbib{burden}} who was among the sheep breeders of Tekoa, which he spoke\fnote{\fbackref{1:1} Lit. \fbib{saw}} concerning Israel during the reign of\fnote{\fbackref{1:1} Or \fbib{in the days of}} Uzziah, king of Judah and during the reign of\fnote{\fbackref{1:1} Or \fbib{in the days of}} Joash's son Jeroboam, king of Israel, two years before the earthquake.

\begin{poetry}
\poeml \v{2}He said, ``From Zion the \divine{Lord} roars, \\
\poemll    and from Jerusalem he shouts aloud. \\
\poeml The shepherds' pastures will languish, \\
\poemll    and Carmel's summit will wither.''
\passage{A Warning to Damascus}
\poeml \v{3}This is what the \divine{Lord} says: \\
\poeml ``For three transgressions of Damascus \\
\poemll    ---and now for a fourth--- \\
\poemlll       I will not turn away; \\
\poeml because they have trampled down\fnote{\fbackref{1:3} Or \fbib{have threshed}} Gilead \\
\poemll    with ironclad threshing sleds. \\
\poeml \v{4}So I will send down fire upon the house of Hazael, \\
\poemll    and it will devour the palaces of Ben-hadad. \\
\poeml \v{5}I will shatter the gate bars of Damascus, \\
\poemll    and I will cut off the residents of the Aven Valley, \\
\poeml along with the one who holds the scepter from Beth-eden; \\
\poemll    and the people of Aram will be exiled to Kir,'' \\
\poemlll       says the \divine{Lord}.
\end{poetry}
\passage{A Warning to Gaza}

\v{6}This is what the \divine{Lord} says:

\begin{poetry}
\poeml ``For three transgressions of Gaza \\
\poemll    ---and now for a fourth--- \\
\poemlll       I will not turn away; \\
\poeml because they exiled the entire population, \\
\poemll    delivering them to Edom. \\
\poeml \v{7}So I will send down fire upon the wall of Gaza, \\
\poemll    and it will devour their fortified citadels; \\
\poeml \v{8}and I will cut off the inhabitants of Ashdod, \\
\poemll    along with Ashkelon's ruler.\fnote{\fbackref{1:8} Lit. \fbib{with the one who holds the scepter of Ashkelon}} \\
\poeml I will turn to attack\fnote{\fbackref{1:8} Lit. \fbib{turn my hand against}} Ekron, \\
\poemll    and the rest of the Philistines will die,'' \\
\poemlll       says the Lord \divine{God}.
\end{poetry}
\passage{A Warning to Tyre}

\v{9}This is what the \divine{Lord} says:

\begin{poetry}
\poeml ``For three transgressions of Tyre \\
\poemll    ---and now for a fourth--- \\
\poemlll       I will not turn away; \\
\poeml because they delivered the entire population to Edom, \\
\poemll    and did not remember their covenant with their relatives.\fnote{\fbackref{1:9} Or \fbib{brothers}} \\
\poeml \v{10}So I will send down fire upon the wall of Tyre, \\
\poemll    and it will devour their fortified citadels.''
\end{poetry}
\passage{A Warning to Edom}

\v{11}This is what the \divine{Lord} says:

\begin{poetry}
\poeml ``For three transgressions of Edom \\
\poemll    ---and now for a fourth--- \\
\poemlll       I will not turn away; \\
\poeml because he\fnote{\fbackref{1:11} I.e. the nation personified as an individual} pursued his brother with a sword, \\
\poemll    refusing to be compassionate.\fnote{\fbackref{1:11} Lit. \fbib{sword, abandoning his compassion}} \\
\poeml His anger was raging\fnote{\fbackref{1:11} Lit. \fbib{anger was tearing away}} continuously; \\
\poemll    he kept up his unending wrath. \\
\poeml \v{12}So I will send down fire upon Teman, \\
\poemll    and it will devour the fortified citadels of Bozrah.''
\end{poetry}
\passage{A Warning to Ammon}

\v{13}This is what the \divine{Lord} says:

\begin{poetry}
\poeml ``For three transgressions of the Ammonites \\
\poemll    ---and now for a fourth--- \\
\poemlll       I will not turn away; \\
\poeml because they ripped open the pregnant women of Gilead \\
\poemll    in order to enlarge their national borders.\fnote{\fbackref{1:13} Or \fbib{their boundary}} \\
\poeml \v{14}So I will send down fire upon the wall of Rabbah, \\
\poemll    and it will devour their fortified citadels \\
\poemlll       with an alarm sounding in the time of battle, \\
\poemll    and with a whirlwind in the time of storm. \\
\poeml \v{15}Their king will go into captivity--- \\
\poemll    he and his princes together,'' \\
\poemlll       says the \divine{Lord}.
\end{poetry}
\labelchapt{2}
\passage{A Warning to Moab}

\chapt{2}
\v{1}This is what the \divine{Lord} says:

\begin{poetry}
\poeml ``For three transgressions of Moab \\
\poeml because they\fnote{\fbackref{2:1} Lit. \fbib{he}; i.e. the nation personified as an individual} cremated the bones of the king of Edom, \\
\poemll    burning them\fnote{\fbackref{2:1} The Heb. lacks \fbib{burning them}} to lime. \\
\poeml \v{2}So I will send down fire upon Moab, \\
\poemll    and it will devour the fortified citadels of Kerioth. \\
\poeml Moab will die in the uproar of battle,\fnote{\fbackref{2:2} The Heb. lacks \fbib{of battle}} \\
\poemll    with a war cry \\
\poemlll       and with the trumpeting of the ram's horn. \\
\poeml \v{3}I will execute their rulers among them, \\
\poemll    killing all of their officials as well,'' \\
\poemlll       says the \divine{Lord}.
\end{poetry}
\passage{A Warning to Judah}

\v{4}This is what the \divine{Lord} says:

\begin{poetry}
\poeml ``For three transgressions of Judah \\
\poemll    ---and now for a fourth--- \\
\poemlll       I will not turn away; \\
\poeml because they\fnote{\fbackref{2:4} Lit. \fbib{he}; i.e. the nation personified as an individual} rejected the Law of the \divine{Lord} \\
\poemll    and did not keep his statutes. \\
\poeml Their own lies made them wander off, \\
\poemll    following along the same path their ancestors walked. \\
\poeml \v{5}So I will send down fire upon Judah, \\
\poemll    and it will devour the fortified citadels of Jerusalem.''
\end{poetry}
\passage{A Warning to Israel}

\v{6}This is what the \divine{Lord} says:

\begin{poetry}
\poeml ``For three transgressions of Israel \\
\poemll    ---and now for a fourth--- \\
\poemlll       I will not turn away; \\
\poeml because they sold the righteous for money, \\
\poemll    and the poor for sandals, \\
\poeml \v{7}moving quickly\fnote{\fbackref{2:7} Lit. \fbib{They chase}} to rub the face\fnote{\fbackref{2:7} Lit. \fbib{head}} of the needy in the dirt. \\
\poeml Corrupting\fnote{\fbackref{2:7} Lit. \fbib{turning aside}} the ways of the humble, \\
\poemll    a man and his father go to the same woman, \\
\poemlll       deliberately defiling my holy name. \\
\poeml \v{8}They lay down beside every altar, \\
\poemll    on garments pledged as collateral,\fnote{\fbackref{2:8} I.e. violating Deut 24:10-13 in addition to idolatry} \\
\poeml drinking wine paid for through fines \\
\poemll    imposed by the temple of their gods. \\
\poeml \v{9}Yet it was I who destroyed the Amorites in front of them, \\
\poemll    though their height seemed like a cedar,\fnote{\fbackref{2:9} I.e. a genus of coniferous evergreen in the family \fbib{Pinaceae}} \\
\poemll    though their strength seemed like an oak, \\
\poeml but whose fruit I destroyed from above \\
\poemll    and the roots from beneath. \\
\poeml \v{10}Furthermore, I brought you up from the land of Egypt, \\
\poemll    leading you in the wilderness for 40 years, \\
\poemlll       to take possession of the land of the Amorites. \\
\poeml \v{11}I also raised up your sons to be prophets, \\
\poemll    and from your young men I raised up Nazirites.\fnote{\fbackref{2:11} I.e. men who make a special vow with God (cf. Num 6:1-21)} \\
\poeml Is this not true, people of Israel?'' \\
\poemll    declares the \divine{Lord}. \\
\poeml \v{12}``But you forced the Nazirites to drink wine, \\
\poemll    and commanded the prophets, \\
\poemlll       `You are not to prophesy!' \\
\poeml \v{13}``Oh, how I am burdened down with you, \\
\poemll    as a wagon is overloaded with harvested grain! \\
\poeml \v{14}So the swift runner will not escape,\fnote{\fbackref{2:14} Lit. \fbib{So flight will escape the swift runner}} \\
\poemll    the valiant will not fortify his strength, \\
\poemlll       and the mighty warrior will not save his life. \\
\poeml \v{15}The skilled archer will not be able to stand, \\
\poemll    the swift runner will not survive, \\
\poemlll       and the mounted rider will not preserve his own life. \\
\poeml \v{16}Even the bravest of elite troops will run away naked at that time,'' \\
\poemll    declares the \divine{Lord}.
\end{poetry}
\labelchapt{3}
\passage{A Higher Standard of Accountability}

\chapt{3}
\v{1}``Listen to this message that the \divine{Lord} has spoken about you, people of Israel. It concerns the entire family that I brought from the land of Egypt:

\begin{poetry}
\poeml \v{2}`You alone have I known from among all of the families of mankind; \\
\poemll    therefore I will hold you accountable for all your iniquities.'\,''
\passage{Seven Questions to Ponder}
\poeml \v{3}``Will a couple walk in unity \\
\poemll    without having met? \\
\poeml \v{4}Will a lion roar in the forest \\
\poemll    without having found its prey? \\
\poeml Will a young lion cry from its den \\
\poemll    without having caught anything? \\
\poeml \v{5}Does a bird fall into a snare on the ground \\
\poemll    without any bait in the trap? \\
\poeml Will a trap snap shut \\
\poemll    when there is nothing to catch? \\
\poeml \v{6}And when an alarm\fnote{\fbackref{3:6} Lit. \fbib{trumpet}} sounds in the city, \\
\poemll    the people will tremble, won't they? \\
\poeml If there is trouble in a city, \\
\poemll    the \divine{Lord} has brought it about, has he not?''
\passage{The \divine{Lord}'s Purposes}
\poeml \v{7}``Truly the Lord \divine{God} will do nothing he has mentioned \\
\poemll    without revealing his purposes to his servants the prophets. \\
\poeml \v{8}A lion has roared! \\
\poemll    Who will not fear? \\
\poeml The Lord \divine{God} has spoken! \\
\poemll    Who will not prophesy? \\
\poeml \v{9}Announce this\fnote{\fbackref{3:9} The Heb. lacks \fbib{this}} in the fortified citadels of Ashdod, \\
\poemll    and in the fortified citadels of the land of Egypt. \\
\poeml Tell them, `Gather together on the mountains of Samaria; \\
\poemll    look at the great misery among the citadels,\fnote{\fbackref{3:9} Lit. \fbib{them}} \\
\poemlll       along with the oppression within Egypt.'\fnote{\fbackref{3:9} Lit. \fbib{her}; i.e. Egypt personified as a woman} \\
\poeml \v{10}Because they do not know how to act right,'' \\
\poemll    declares the \divine{Lord}, \\
\poeml ``they are filling their strongholds with treasures \\
\poemll    that they took from others by violence into their fortified citadels.'' \\
\poeml \v{11}Therefore this is what the Lord \divine{God} says: \\
\poeml ``An enemy will surround the land. \\
\poemll    He\fnote{\fbackref{3:11} I.e. the invading forces personified as an individual} will pull down your defenses, \\
\poemlll       and plunder your fortified citadels.'' \\
\poeml \v{12}This is what the \divine{Lord} says: \\
\poeml ``Just as a shepherd might save from the lion's mouth \\
\poemll    only two leg bones or a scrap of an ear, \\
\poeml the Israelis will be saved in a similar manner--- \\
\poemll    those in Samaria who sit on the remains of their broken beds,\fnote{\fbackref{3:12} Lit. \fbib{on the corner of a bed}} \\
\poemlll       and those in Damascus who lie on the edge of their couches.'' \\
\poeml \v{13}``Listen and testify against the house of Jacob,'' \\
\poemll    declares the Lord \divine{God}, the God of the Heavenly Armies, \\
\poeml \v{14}``because on that day I will lay out the charges against Israel. \\
\poemll    I will also bring judgment upon the altars of Bethel; \\
\poeml the horns of the altar will be cut off \\
\poemll    and will fall to the ground. \\
\poeml \v{15}I will wreck both the winter house and the summer house, \\
\poemll    and the ivory houses will fall.\fnote{\fbackref{3:15} Lit. \fbib{perish}} \\
\poeml These palaces will surely fall,'' \\
\poemll    declares the \divine{Lord}.
\end{poetry}
\labelchapt{4}
\passage{Judgment on the Women of Israel}

\begin{poetry}
\poeml \chapt{4}
\v{1}``Listen to this message, you fat cows from Bashan, \\
\poeml who live on the Samaritan mountains, \\
\poemll    who oppress the poor, \\
\poeml who rob the needy, \\
\poemll    and who constantly ask your husbands for one more drink!'' \\
\poeml \v{2}The Lord \divine{God} has taken a sacred oath:\fnote{\fbackref{4:2} Lit. \fbib{oath on his holiness}} \\
\poeml ``The day is coming when they\fnote{\fbackref{4:2} Lit. \fbib{coming upon you, and he}; i.e. the invading forces personified as an individual} will take you away on fishhooks, \\
\poemll    every last one of you on fishhooks. \\
\poeml \v{3}Each of you will go out through the breaches of the walls\fnote{\fbackref{4:3} The Heb. lacks \fbib{of the walls}} \\
\poemll    straight to Mt. Hermon,''\fnote{\fbackref{4:3} Heb. \fbib{Harmonah}} \\
\poemlll       declares the \divine{Lord}.
\passage{The \divine{Lord}'s Rebuke to Israel}
\poeml \v{4}``Come to Bethel and sin, \\
\poemll    to Gilgal and sin even more! \\
\poeml Bring along your morning sacrifices, \\
\poemll    and pay your tithes every other day.\fnote{\fbackref{4:4} Lit. \fbib{tithes for the three days}} \\
\poeml \v{5}While you're at it,\fnote{\fbackref{4:5} Lit. \fbib{And}} present a thank offering with leaven, \\
\poemll    and publicize your freewill offerings, \\
\poeml letting everyone hear about it, \\
\poemll    because this is what you really love to do, you Israelis,'' \\
\poemlll       declares the Lord \divine{God}.
\passage{Israel's Refusal to Return to God}
\poeml \v{6}``I also have scheduled\fnote{\fbackref{4:6} Lit. \fbib{appointed}} food shortages\fnote{\fbackref{4:6} Lit. \fbib{appointed clean teeth}} for you in all of your cities, \\
\poemll    and lack of bread in all of your settlements, \\
\poeml but you haven't returned to me,'' \\
\poemll    declares the \divine{Lord}. \\
\poeml \v{7}``I therefore have withheld the rain from you \\
\poemll    three months before the harvest, \\
\poeml causing rain to come upon one city, \\
\poemll    but not upon another, \\
\poeml and upon one field \\
\poemll    but not upon another, \\
\poemlll       so that it would wither. \\
\poeml \v{8}So the people of\fnote{\fbackref{4:8} The Heb. lacks \fbib{the people of}} two or three cities staggered away to another\fnote{\fbackref{4:8} Lit. \fbib{to a single}} city \\
\poemll    in order to obtain drinking water, \\
\poeml but you have not returned to me,'' \\
\poemll    declares the \divine{Lord}. \\
\poeml \v{9}``I afflicted you with blight and fungus; \\
\poemll    and the locust swarm devoured the harvest \\
\poemlll       of your gardens, your vineyards, your fig trees, and your olive trees, \\
\poemll    but you have not returned to me,'' \\
\poemlll       declares the \divine{Lord}. \\
\poeml \v{10}``I sent plagues among you as I did with Egypt. \\
\poemll    I killed your choicest young men with the sword. \\
\poeml I took your horses away from you. \\
\poemll    I filled your noses with the stench of your encampments, \\
\poeml but you have not returned to me,'' \\
\poemll    declares the \divine{Lord}. \\
\poeml \v{11}``I overthrew your cities,\fnote{\fbackref{4:11} The Heb. lacks \fbib{cities}} \\
\poemll    as God overthrew Sodom and Gomorrah. \\
\poeml You've become like a burning ember, snatched from the fire, \\
\poemll    but you have not returned to me,'' \\
\poemlll       declares the \divine{Lord}. \\
\poeml \v{12}``Therefore this is what I will do to you, Israel. \\
\poemll    Because I am about to do this, \\
\poemlll       prepare to be summoned to your God, Israel!'' \\
\poeml \v{13}Look! The one who crafts mountains, \\
\poemll    who creates the wind, \\
\poeml who reveals what he is thinking to mankind, \\
\poemll    who darkens the morning light, \\
\poeml who tramples down the high places of the land--- \\
\poemll    the \divine{Lord}, the God of the Heavenly Armies is his name.
\end{poetry}
\labelchapt{5}
\passage{A Lament for Israel}

\begin{poetry}
\poeml \chapt{5}
\v{1}``Hear this accusation\fnote{\fbackref{5:1} Lit. \fbib{word}} that I am bringing against you:
\end{poetry}

\begin{poetry}
\poeml `A dirge, house of Israel: \\
\poeml \v{2}Fallen is Israel the virgin---never to rise again! \\
\poemll    She is abandoned on her own land, \\
\poemlll       with no one to raise her up.' \\
\poeml \v{3}``For this is what the Lord \divine{God} says: \\
\poeml `The city that is sending out a thousand \\
\poemll    will have a hundred left; \\
\poeml The city\fnote{\fbackref{5:3} The Heb. lacks \fbib{The city}} that is sending out a hundred \\
\poemll    will have ten left of the house of Israel.'\,''
\passage{Seek God, and Live}
\poeml \v{4}``For this is what the \divine{Lord} says to the house of Israel: \\
\poeml `Seek me and live, \\
\poeml \v{5}but don't seek Bethel. \\
\poeml Don't go to Gilgal, \\
\poemlll       and don't pass over to Beer-sheba. \\
\poeml Because Gilgal will surely go into captivity,\fnote{\fbackref{5:5} The root Heb. for \fbib{Gilgal} is a pun on the Heb. \fbib{go into captivity}} \\
\poemll    and Bethel will come to nothing. \\
\poeml \v{6}`Seek the \divine{Lord} and live! \\
\poemll    Otherwise, he may break out like a fire in the house of Joseph \\
\poemlll       and devour Bethel,\fnote{\fbackref{5:6} So MT; LXX reads \fbib{devour the House of Israel}} \\
\poemll    and there will be no one to extinguish it. \\
\poeml \v{7}Those of you who are making justice taste bitter,\fnote{\fbackref{5:7} Lit. \fbib{are turning justice into wormwood}} \\
\poemll    and who have thrown righteousness to the ground: \\
\poeml \v{8}Seek\fnote{\fbackref{5:8} The Heb. lacks \fbib{Seek}} the one who fashions the Pleiades and Orion, \\
\poemll    who turns the deep darkness\fnote{\fbackref{5:8} Or \fbib{the shadow of death}} into morning, \\
\poemll    who darkens day into night, \\
\poemll    who calls out to the waters of the sea, \\
\poemlll       pouring them out onto the surface of the earth--- \\
\poemll    the \divine{Lord} is his name. \\
\poeml \v{9}It is he who is raining sudden destruction \\
\poemll    upon the strong like lightning,\fnote{\fbackref{5:9} The Heb. lacks \fbib{like lightning}} \\
\poemlll       so that ruin comes upon the fortress. \\
\poeml \v{10}They have hated those who are presenting their cases in court,\fnote{\fbackref{5:10} Lit. \fbib{in the gate}} \\
\poemll    detesting the one who speaks truthfully. \\
\poeml \v{11}`Therefore, since you trample the poor continuously, \\
\poemll    taxing his grain, \\
\poemll    building houses of stone in which you won't live \\
\poemll    and planting fine vineyards from which you won't drink--- \\
\poeml \v{12}and because I know that your transgressions are many, \\
\poemll    and your sins are numerous \\
\poeml as you oppose the righteous, \\
\poemll    taking bribes as a ransom, \\
\poemlll       and turning away the poor in court\fnote{\fbackref{5:12} Lit. \fbib{in the gate}}--- \\
\poeml \v{13}therefore the prudent person remains silent at such a time, \\
\poemll    for the time is evil. \\
\poeml \v{14}`Pursue good and not evil, \\
\poemll    so that you may live, \\
\poeml and this is what will happen:\fnote{\fbackref{5:14} Lit. \fbib{And so it was}} \\
\poemll    The \divine{Lord} God of the Heavenly Armies will be with you, \\
\poemlll       as you have been claiming. \\
\poeml \v{15}Hate evil and love good, \\
\poemll    and establish justice in court---\fnote{\fbackref{5:15} Lit. \fbib{in the gates}} \\
\poeml perhaps the \divine{Lord}, the God of the Heavenly Armies, \\
\poeml will be gracious to the survivors of Joseph.'\,'' \\
\poeml \v{16}Therefore this is what the \divine{Lord}, the God of the Heavenly Armies, the Lord, says: \\
\poeml `There will be dirges in all of the streets; \\
\poemlll       and in all of the highways they will cry out in anguish.\fnote{\fbackref{5:16} Lit. \fbib{will say, ``Alas! Alas!''}} \\
\poeml They will call the farmer to mourning \\
\poemll    and those who lament\fnote{\fbackref{5:16} I.e. professional mourners} to grieve. \\
\poeml \v{17}And in all of the vineyards there will be mourning \\
\poemll    when I pass through your midst,' \\
\poemlll       says the \divine{Lord}.''
\passage{The Fearful Day of the \divine{Lord}}
\poeml \v{18}``Woe to those who are craving the Day of the \divine{Lord}! \\
\poemll    How is it to your benefit, this Day of the \divine{Lord}? \\
\poemlll       It's a day of\fnote{\fbackref{5:18} The Heb. lacks \fbib{a day of}} darkness to you, and not light. \\
\poeml \v{19}It will be like a man who runs from a lion, \\
\poemll    only to encounter a bear; \\
\poeml or who comes home, leans his hand against a wall, \\
\poemll    and a serpent bites him! \\
\poeml \v{20}Will not the Day of the \divine{Lord} be darkness, and not light--- \\
\poemll    pitch black at that, without a ray of sunshine?''
\passage{Let Justice Roll On}
\poeml \v{21}``I hate---I despise---your festival days, \\
\poemll    and your solemn convocations stink.\fnote{\fbackref{5:21} Lit. \fbib{and I smell no pleasant scent in your solemn assemblies}} \\
\poeml \v{22}And\fnote{\fbackref{5:22} Lit. \fbib{Because}} if you send up burnt offerings to me \\
\poemll    as well as your grain offerings, \\
\poeml I will not accept them, \\
\poemll    nor will I consider your peace offerings of fattened cattle. \\
\poeml \v{23}Spare me your noisy singing--- \\
\poemll    I will not listen to your musical instruments.\fnote{\fbackref{5:23} I.e. a stringed instrument such as a harp or lyre} \\
\poeml \v{24}``But let justice roll on like many\fnote{\fbackref{5:24} The Heb. lacks \fbib{many}} waters, \\
\poemll    and righteousness like an ever-flowing river. \\
\poeml \v{25}``Was it to me that you brought offerings and gifts \\
\poemll    in the desert for 40 years, house of Israel? \\
\poeml \v{26}And you carried the tent of your king\fnote{\fbackref{5:26} LXX reads \fbib{of Moloch}; MT reads \fbib{carried Sikkuth your king}}--- \\
\poemll    and Saturn,\fnote{\fbackref{5:26} Heb. \fbib{Kiyyun}} your star god idols\fnote{\fbackref{5:26} So MT; LXX reads \fbib{and the star of your God Raiphan, the images}} that you crafted for yourselves. \\
\poeml \v{27}So I will cause you to be taken captive beyond Damascus,'' \\
\poemll    says the \divine{Lord}, \\
\poemlll       whose name is God of the Heavenly Armies.
\end{poetry}
\labelchapt{6}
\passage{Mourning for the House of Israel}

\begin{poetry}
\poeml \chapt{6}
\v{1}``Woe to those who are at ease in Zion, \\
\poemll    to those who rest on the mountain of Samaria--- \\
\poeml the famous men of the nations \\
\poemll    to whom the house of Israel came! \\
\poeml \v{2}Cross over to Calneh\fnote{\fbackref{6:2} I.e. a Mesopotamian city} and look around, \\
\poemll    then go on to that great city of\fnote{\fbackref{6:2} The Heb. lacks \fbib{city of}} Hamath, \\
\poemlll       and from there go down to Gath of the Philistines. \\
\poeml Are you better than these kingdoms? \\
\poemll    Or is their territory more extensive than yours? \\
\poeml \v{3}``Disbelieving that a day of evil will come,\fnote{\fbackref{6:3} The Heb. lacks \fbib{will come}} \\
\poemll    embracing opportunities to commit violence,\fnote{\fbackref{6:3} Lit. \fbib{yet are pressing hard the seat of violence}} \\
\poeml \v{4}lying on ivory beds, \\
\poemll    stretching out on your couches, \\
\poeml eating lambs from the flock, \\
\poemll    and fattened calves from the stall, \\
\poeml \v{5}chanting to the sound of stringed instruments as if they were David, \\
\poemll    composing songs to themselves as if they were musicians, \\
\poeml \v{6}drinking wine from bowls, \\
\poemll    anointing themselves with the choicest of oils, \\
\poemlll       but not grieving on the occasion of Joseph's ruin--- \\
\poeml \v{7}therefore you will be the first to go into exile, \\
\poemll    and the celebrations of those who are lounging will end.''
\passage{The \divine{Lord} Swears an Oath}
\poeml \v{8}``The Lord \divine{God} has sworn by himself,'' \\
\poemll    declares the \divine{Lord}, the God of the Heavenly Armies, \\
\poeml ``I utterly detest the arrogance of Jacob; \\
\poemll    I hate his fortresses; \\
\poeml and I will deliver up the city, \\
\poemll    along with everyone in it. \\
\poeml \v{9}``And if there are ten men remaining in one house, \\
\poemll    they will die. \\
\poeml \v{10}One's relative will pick up the corpse\fnote{\fbackref{6:10} Lit. \fbib{bones}} \\
\poemll    to carry them from the house for burning,\fnote{\fbackref{6:10} Or \fbib{house to burn incense}} \\
\poeml saying to whomever remains inside the house, \\
\poemll    `Is there anyone still with you?' \\
\poeml And he will say, `No.' \\
\poemll    He will respond, `Be quiet, \\
\poemlll       because we do not mention the name ``\divine{Lord}''.' \\
\poeml \v{11}For indeed, the \divine{Lord} is giving the command--- \\
\poemll    and he will smash the large house to rubble \\
\poemlll       and the small house into bits. \\
\poeml \v{12}``Horses don't run over bare rock, do they? \\
\poemll    One doesn't plow rock\fnote{\fbackref{6:12} The Heb. lacks \fbib{rock}} with oxen, does he? \\
\poeml But you have turned justice to gall, \\
\poemll    and the fruit of righteousness into bitterness.\fnote{\fbackref{6:12} Lit. \fbib{wormwood}} \\
\poeml \v{13}You rejoice in nothing worth mentioning--- \\
\poemll    that is, you keep on saying, \\
\poeml `We captured Karnaim by our own strength of will \\
\poemll    and by our own effort, didn't we?' \\
\poeml \v{14}``So look, house of Israel! I will raise up a nation against you,'' \\
\poemll    declares the \divine{Lord}, the God of the Heavenly Armies, \\
\poeml ``and they will harass you from the entrance of Hamath \\
\poemll    to the wadi\fnote{\fbackref{6:14} I.e. perhaps the Wadi of Egypt, a seasonal stream or river that channels water during rain seasons but is dry at other times; ancient Israel's southwestern-most border} of the wilderness.''
\end{poetry}
\labelchapt{7}
\passage{The Vision of Locusts}

\chapt{7}
\v{1}This is what the Lord \divine{God} showed me: Look! He was forming locust swarms as the latter plantings were just beginning to sprout. Indeed, the king had just taken his first fruit tax.\fnote{\fbackref{7:1} So MT; LXX reads \fbib{the locusts have one king, Gog.}} \v{2}And so it came about that when the swarm\fnote{\fbackref{7:2} The Heb. lacks \fbib{the swarm}} had finished eating the grass of the land, I was saying,

\begin{poetry}
\poeml ``Lord \divine{God}, forgive---please! \\
\poemll    How will Jacob stand, since he is small?''
\end{poetry}

\v{3}So the \divine{Lord} relented from this. ``This will not happen,'' said the \divine{Lord}.
\passage{The Vision of Fire}

\v{4}This is what the Lord \divine{God} showed me: Look! The Lord \divine{God} was calling for judgment by fire, and it was drying up the great depths of the ocean\fnote{\fbackref{7:4} The Heb. lacks \fbib{of the ocean}} and consuming the land. \v{5}So I kept on saying,

\begin{poetry}
\poeml ``Lord \divine{God}, forgive---please! \\
\poemll    How will Jacob stand, since he is so small?''
\end{poetry}

\v{6}So the \divine{Lord} relented from this. ``This will not happen, either,'' said the Lord \divine{God}.
\passage{The Vision of the Plumb Line}

\v{7}This is what he showed me: Look! The Lord was standing upon a wall that stood straight and true, with a plumb line in his hand.\fnote{\fbackref{7:7} Lit. \fbib{wall by a plumb line}} \v{8}And the \divine{Lord} was asking me, ``What do you see, Amos?''

I replied, ``A plumb line.''

So the Lord said,

\begin{poetry}
\poeml ``Look, I have set a plumb line \\
\poemll    in the midst of my people Israel. \\
\poemlll       I will no longer spare them. \\
\poeml \v{9}Isaac's high places will be destroyed, \\
\poemll    and the sanctuaries of Israel will be ruined. \\
\poemlll       I will rise in opposition to the house of Jeroboam with my\fnote{\fbackref{7:9} The Heb. lacks \fbib{my}} sword.''
\end{poetry}
\passage{A Rebuke for Amaziah}

\v{10}So Amaziah priest of Bethel sent a message\fnote{\fbackref{7:10} The Heb. lacks \fbib{a message}} to Jeroboam king of Israel. It said, ``Amos has been conspiring against you in the very heart of the house of Israel! The land cannot bear everything he has to say, \v{11}because Amos is saying this:

\begin{poetry}
\poeml `By the sword will Jeroboam die, \\
\poemll    and Israel will surely go into exile \\
\poemlll       far from her homeland.'\,''
\end{poetry}

\v{12}So Amaziah kept saying to Amos, ``Get out of here, you seer! Go back to the land of Judah. Live\fnote{\fbackref{7:12} Lit. \fbib{Eat}} there and prophesy there. \v{13}Don't prophesy anymore at Bethel, because it's the king's sanctuary and a temple of the kingdom.''

\v{14}Amos replied in answer to Amaziah,

\begin{poetry}
\poeml ``I am no prophet, \\
\poemll    nor am I a prophet's son, \\
\poeml for I have been shepherding \\
\poemll    and picking the fruit of\fnote{\fbackref{7:14} The Heb. lacks \fbib{the fruit of}} sycamore\fnote{\fbackref{7:14} The sycamore fruit tree native to Israel bears figs} trees.
\end{poetry}

\v{15}But the \divine{Lord} took me from tending the flock and the \divine{Lord} kept saying to me, `Go, prophesy to my people Israel.'

\v{16}``Very well then, hear this message from the \divine{Lord}:

\begin{poetry}
\poeml `You are saying, \\
\poemll    ``Don't prophesy against Israel, \\
\poemlll       and don't preach against the house of Isaac.'' \\
\poeml \v{17}`Therefore this is what the \divine{Lord} says: \\
\poemll    ``Your wife will become a whore in the city, \\
\poemlll       and your sons and daughters will die by the sword. \\
\poemll    Your land will be divided and apportioned, \\
\poemlll       and you will die in a foreign\fnote{\fbackref{7:17} Lit. \fbib{in an unclean}} land. \\
\poemll    Israel will surely go into exile, \\
\poemlll       far from its homeland.''\,'\,''
\end{poetry}
\labelchapt{8}
\passage{The Vision of a Fruit Basket}

\chapt{8}
\v{1}This is what the Lord \divine{God} showed me: Look! A basket of summer fruit! \v{2}And he was asking, ``What do you see, Amos?''

I answered, ``A basket of summer fruit.''

Then the \divine{Lord} told me,

\begin{poetry}
\poeml ``The end\fnote{\fbackref{8:2} The Heb. \fbib{end} sounds like Heb. word \fbib{summer fruit}} approaches for my people Israel. \\
\poemll    I will no longer spare them.\fnote{\fbackref{8:2} Lit. \fbib{him}} \\
\poeml \v{3}At that time,'' \\
\poemll    declares the Lord \divine{God}, \\
\poeml ``the temple songs will be wailing. \\
\poemll    Many bodies will accumulate everywhere. \\
\poeml \v{4}``Hear this, you who are swallowing up the needy, \\
\poemll    who intend to make the poor of the land fail, \\
\poeml \v{5}and who are saying, \\
\poemll    `When will the New Moon fade \\
\poemlll       so we may sell grain, \\
\poemll    and the Sabbath conclude\fnote{\fbackref{8:5} The Heb. lacks \fbib{conclude}} \\
\poemlll       so we may market winnowed wheat?--- \\
\poeml shortchanging the measure,\fnote{\fbackref{8:5} The Heb. \fbib{ephah}} \\
\poemll    raising the price, \\
\poemlll       falsifying the scales by treachery, \\
\poeml \v{6}buying the poor for cash,\fnote{\fbackref{8:6} Lit. \fbib{silver}} \\
\poemll    and the needy for a pair of sandals, \\
\poemlll       selling chaff mixed in with the wheat.' \\
\poeml \v{7}``The \divine{Lord} has sworn by the pride of Jacob: \\
\poemll    I will never forget anything they have done. \\
\poeml \v{8}Surely the land will tremble because of this, won't it? \\
\poemll    And all who live in it will mourn, won't they? \\
\poeml The entire land will swell up like a flooded\fnote{\fbackref{8:8} The Heb. lacks \fbib{flooded}} river. \\
\poemll    It will be stirred up and then will sink \\
\poemlll       like the river of Egypt. \\
\poeml \v{9}It will come about at that time,'' declares the Lord \divine{God}, \\
\poemll    ``I will cause the sun to set at noon \\
\poemlll       and the earth to darken in the daylight. \\
\poeml \v{10}I will turn your festivals into mourning, \\
\poemll    and all of your songs to dirges. \\
\poeml I will cause all of you to put on sackcloth \\
\poemll    and to shave all of your heads. \\
\poeml I will make that time like mourning for an only son, \\
\poemll    and its conclusion will be like the end of\fnote{\fbackref{8:10} The Heb. lacks \fbib{the end of}} a bitter day.''
\passage{A Famine of the Word of God}
\poeml \v{11}``Look! The days are coming,'' \\
\poemll    declares the Lord \divine{God}, \\
\poeml ``when I will send a famine throughout the land--- \\
\poemll    not a famine of food or a thirst for water--- \\
\poemlll       but rather a famine of hearing the words of the \divine{Lord}. \\
\poeml \v{12}People\fnote{\fbackref{8:12} Lit. \fbib{They}} will stagger from sea to sea, \\
\poemll    from north to east. \\
\poeml They will run back and forth, \\
\poemll    searching for a message from the \divine{Lord}, \\
\poemlll       but they won't find it. \\
\poeml \v{13}At that time, \\
\poemll    the beautiful virgins will faint, \\
\poemlll       as will the strong young men---from thirst. \\
\poeml \v{14}Those who have been swearing oaths by the sin of Samaria, \\
\poemll    or who say, `As your god lives, Dan{\ldots}' \\
\poeml or who say, `As the way of Beer-sheba lives{\ldots}'--- \\
\poemll    will fall, and will never rise again.''
\end{poetry}
\labelchapt{9}
\passage{Israel to be Destroyed}

\chapt{9}
\v{1}I saw the Lord standing beside the altar as he was saying,

\begin{poetry}
\poeml ``Strike the doorposts \\
\poemll    so that the thresholds tremble, \\
\poemlll       bringing them down on the heads of all of them. \\
\poeml Those who survive I will kill with the sword. \\
\poemll    Those who flee will not escape. \\
\poemlll       There will be no deliverance for the fugitives. \\
\poeml \v{2}``Even if they burrow into Sheol,\fnote{\fbackref{9:2} I.e. the realm of the dead} \\
\poemll    from there my hand will find them. \\
\poeml Even if they ascend to the heavens, \\
\poemll    from there I will bring them down. \\
\poeml \v{3}Even if they hide at the top of Mount\fnote{\fbackref{9:3} The Heb. lacks \fbib{Mount}} Carmel, \\
\poemll    from there I will search and seize them. \\
\poeml Even if they hide from my sight in the depths of the sea, \\
\poemll    from there I will order the serpent to strike them. \\
\poeml \v{4}Even if they go into exile among their enemies, \\
\poemll    from there I will order the sword to kill them. \\
\poeml I will fix my gaze on them to inflict disaster, \\
\poemll    and not to do good.\fnote{\fbackref{9:4} Lit. \fbib{them for evil and not for good.}} \\
\poeml \v{5}``The Lord \divine{God} of the Heavenly Armies \\
\poeml who is touching the earth so that it melts \\
\poemll    and all of its inhabitants mourn there--- \\
\poemlll       the land rises like the Nile\fnote{\fbackref{9:5} The Heb. lacks \fbib{Nile}} River, \\
\poemlll       but sinks like the river of Egypt--- \\
\poeml \v{6}who is building his stairway to heaven \\
\poemll    and setting its foundation on earth; \\
\poeml who is calling for the waters of the sea \\
\poemll    and pouring them out over the surface of the land--- \\
\poemlll       the \divine{Lord} is his name! \\
\poeml \v{7}``Aren't you people of Israel like the people of Cush to me?'' \\
\poemll    declares the \divine{Lord}. \\
\poeml ``I brought Israel up from the land of Egypt, did I not, \\
\poemll    as well as the Philistines from Caphtor\fnote{\fbackref{9:7} I.e. possibly Crete} \\
\poemlll       and the Arameans from Kir? \\
\poeml \v{8}Look! The eyes of the Lord \divine{God} are on the sinful kingdom. \\
\poemll    I will destroy it from the face of the earth; \\
\poeml but I will not totally destroy the house of Jacob,'' \\
\poemll    declares the \divine{Lord}. \\
\poeml \v{9}``Look! I'm giving the order: \\
\poemll    I will sift the house of Israel throughout all the nations, \\
\poemlll       as one sifts with a sieve, \\
\poemll    yet not a single kernel will reach the ground! \\
\poeml \v{10}All sinners among my people will die by the sword, \\
\poemll    especially all who are saying, \\
\poemlll       `Disaster will not come upon or conquer us!'\,''
\passage{Israel to be Restored}
\poeml \v{11}``At that time I will restore David's fallen tent, \\
\poemll    restoring its torn places. \\
\poeml I will restore its ruins, \\
\poemll    rebuilding it as it was long ago, \\
\poeml \v{12}so my people\fnote{\fbackref{9:12} Lit. \fbib{so they}} may inherit the remnant of Edom \\
\poemll    and all of the nations that bear my name,'' \\
\poemlll       declares the \divine{Lord} who is bringing this about. \\
\poeml \v{13}``Look! The days are coming,'' \\
\poemll    declares the \divine{Lord}, \\
\poeml ``when the one who sows will overtake the harvester \\
\poemll    and the treader of grapes will overtake\fnote{\fbackref{9:13} The Heb. lacks \fbib{will overtake}} the planter. \\
\poeml Fresh wine will drip down from the mountains, \\
\poemll    cascading down from the hills. \\
\poeml \v{14}I will surely restore my people Israel; \\
\poemll    they will rebuild the ruined cities \\
\poemlll       and inhabit them.\fnote{\fbackref{9:14} The Heb. lacks \fbib{them}} \\
\poeml They will plant vineyards \\
\poemll    and drink the wine from them. \\
\poeml They will plant gardens \\
\poemll    and eat the fruit from them. \\
\poeml \v{15}I will plant the people of Israel\fnote{\fbackref{9:15} Lit. \fbib{plant them}} in their own land, \\
\poemll    never again to be torn out of their land \\
\poemlll       that I gave them,'' \\
\poemll    says the \divine{Lord} your God.\end{poetry}
