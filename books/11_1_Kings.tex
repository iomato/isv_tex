\bookheader{1 Kings}
\labelbook{1King}

\bookpretitle{The Book of}
\booktitle{First Kings}

\labelchapt{1}
\passage{Adonijah's Attempted Coup}

\chapt{1}
\v{1}When David had grown very old, they covered him with blankets, but he could not keep warm, \v{2}so his servants suggested to him, ``Let's look for a young virgin woman to take care of you, your majesty. She will be of use to you if you have her lie down near you\fnote{\fbackref{1:2} Lit. \fbib{lie in your lap}} so that your majesty may keep warm.'' \v{3}So they conducted a search throughout the territory of Israel for a beautiful young woman, and Abishag the Shunammite was located and brought to the king. \v{4}The young woman was absolutely beautiful. She served the king and was very useful to him. The king was not sexually involved with her.

\v{5}Meanwhile, about this time Haggith's son Adonijah began to seek a reputation for himself and decided,\fnote{\fbackref{1:5} Lit. \fbib{said}} ``I'm going to be king!'' So he prepared chariots, cavalry, and 50 soldiers to serve as a security detail to guard him.\fnote{\fbackref{1:5} Lit. \fbib{soldiers to run ahead of him}} \v{6}His father had never challenged him at any time during his life by asking him, ``Why are you acting like this?'' Adonijah\fnote{\fbackref{1:6} Lit. \fbib{He}} was very handsome and had been born after Absalom. \v{7}He had the support of Zeruiah's son Joab and of Abiathar the priest, who followed Adonijah\fnote{\fbackref{1:7} Lit. \fbib{him}} and assisted him, \v{8}but Zadok the priest, Jehoiada's son Benaiah, Nathan the prophet, Shimei, Rei, and David's personal elite forces would have nothing to do with Adonijah.

\v{9}Adonijah sacrificed sheep, oxen, and fatted cattle by the Serpent Stone\fnote{\fbackref{1:9} Or \fbib{the stone of Omelet}} near En-rogel,\fnote{\fbackref{1:9} Cf. Josh 15:7; 18:16; 2Sam 17:17} inviting all of his relatives, the king's sons, and all of the men of Judah who worked for the king, \v{10}but he did not invite Nathan the prophet, Benaiah, David's\fnote{\fbackref{1:10} Lit. \fbib{the}} personal elite forces, or his brother Solomon.
\passage{Nathan and Bathsheba Confer about Adonijah}

\v{11}``Haven't you heard?'' Nathan asked Solomon's mother Bathsheba. ``Haggith's son Adonijah has become king and David, our true king,\fnote{\fbackref{1:11} Lit. \fbib{our lord}} isn't aware of it. \v{12}If you listen to me, you'll save your life and the life of your son Solomon. \v{13}Go right now to King David and ask him, `Your majesty, you promised your servant that ``Your son Solomon will certainly become king after me and will sit on my throne,'' didn't you? So why has Adonijah become king?' \v{14}Then, while you are still talking to the king, I'll come in after you and verify your statement.''

\v{15}So Bathsheba went to the king in his private room. Now the king was very old, and Abishag the Shunammite was attending to him.\fnote{\fbackref{1:15} Lit. \fbib{to the king}} \v{16}Bathsheba knelt and bowed down to the king, and the king asked her, ``What do you wish?''

\v{17}``Your majesty,'' she replied, ``you promised your servant in the name of\fnote{\fbackref{1:17} The Heb. lacks \fbib{the name of}} the \divine{Lord} your God, `Your son Solomon will certainly become king after me and will sit on my throne.' \v{18}Now look, Adonijah has become king, and your majesty is not aware of it. \v{19}Adonijah\fnote{\fbackref{1:19} Lit. \fbib{He}} has sacrificed myriads of oxen, fattened cattle, and sheep, and he has invited all of the king's sons, Abiathar the priest, and Joab the commander of the army, but he has not invited your servant Solomon. \v{20}And as for you, your majesty, everyone in Israel is looking to you to tell them who will sit on your majesty's throne after you.\fnote{\fbackref{1:20} Lit. \fbib{him}} \v{21}Otherwise, as soon as your majesty is laid to rest with his ancestors, my son Solomon and I will be branded as traitors.''\fnote{\fbackref{1:21} Lit. \fbib{sinners}}

\v{22}While she was still talking to the king, Nathan the prophet arrived. \v{23}They informed the king, ``Nathan the prophet is here.''

When he had been ushered into the presence of the king, Nathan bowed low in front of the king with his face to the ground \v{24}and asked, ``Your majesty, did you say `Adonijah will be king after me and will sit on my throne'? \v{25}Well now, he went down today and sacrificed lots of oxen, fattened cattle, and sheep, and has invited all the king's sons, the army commanders, and Abiathar the priest. They're having a party together and saying, `Long live King Adonijah!' \v{26}Of course, he never invited me, Zadok the priest, Jehoiada's son Benaiah, nor your servant Solomon. \v{27}Were you behind this, your majesty, without letting your servants know who would sit on your majesty's throne after him?''
\passage{David Affirms Solomon as King}

\v{28}``Call Bathsheba for me,'' King David replied. So she came in and stood in front of the king. \v{29}``As the \divine{Lord} lives,'' the king said with an oath, ``who has redeemed me from all sorts of troubles, \v{30}I certainly did tell you in the name of\fnote{\fbackref{1:30} The Heb. lacks \fbib{the name of}} the \divine{Lord} God of Israel, `Your son Solomon will be king after me and will sit on my throne in my place.' I'm certainly going to make this happen today!''

\v{31}``King David,'' Bathsheba said as she bowed low in front of the king with her face to the ground, ``your majesty, may you live forever.''

\v{32}``Get me Zadok the priest,'' King David said, ``along with Nathan the prophet, and Jehoiada's son Benaiah.'' So they were ushered into the king's presence \v{33}and David addressed them. ``Take your lord's servants, have my son Solomon ride on my own mule, and take him down to Gihon. \v{34}Have Zadok the priest and Nathan the prophet anoint him there as king over Israel. Then sound a trumpet and declare `Long live King Solomon!' \v{35}After this, you are to follow him back here, and he is to come and sit on my throne and take my place as king, because I've appointed him to be Commander-in-Chief\fnote{\fbackref{1:35} Lit. \fbib{Nagid}; i.e. a senior officer entrusted with dual roles of operational oversight and administrative authority} over Israel and Judah.''

\v{36}``Amen!'' replied Jehoiada's son Benaiah to the king. ``May the \divine{Lord} God of your majesty make this happen! \v{37}As the \divine{Lord} has been with your majesty the king, so may he be with Solomon. May he make his throne greater than the throne of your majesty, King David.''
\passage{Solomon is Anointed King}

\v{38}So Zadok the priest, Nathan the prophet, Jehoiada's son Benaiah, the special forces\fnote{\fbackref{1:38} Lit. \fbib{Cherethites}; i.e. elite body guards} and mercenaries\fnote{\fbackref{1:38} Lit. \fbib{Pelethites}; i.e. special couriers} went out and had Solomon ride the king's mule all the way to Gihon. \v{39}Zadok the priest brought from his tent a horn filled with oil and anointed Solomon, a trumpet was sounded, and everybody yelled out, ``Long live King Solomon!'' \v{40}All the people followed after him, playing on wind pipes and so full of joy that the earth shook because of all the noise!

\v{41}Right about then, Adonijah and all of his guests were just finishing their meal when they heard all the noise. ``Why is the city in such an uproar?'' Joab asked as he heard the trumpet sounds.

\v{42}While he was still asking that question, Jonathan, the son of Abiathar the priest arrived, so Adonijah told him, ``Come on in, since you're a worthy man and are bringing us good news!''

\v{43}``No,'' Jonathan answered. ``Our lord King David has installed Solomon as king. \v{44}The king has sent Zadok the priest, Nathan the prophet, Jehoiada's son Benaiah, the special forces\fnote{\fbackref{1:44} Lit. \fbib{Cherethites}; i.e. elite body guards} and mercenaries,\fnote{\fbackref{1:44} Lit. \fbib{Pelethites}; i.e. special couriers} along with Solomon, who is riding the king's personal mule. \v{45}Zadok the priest and Nathan the prophet have anointed him in Gihon, and they just left from there rejoicing, and that's why the city is all in an uproar. That's the noise that you've been hearing! \v{46}Solomon now sits on the royal throne. \v{47}In addition to all of this, the king's servants have come along to congratulate our lord King David. They've been telling David `May your God make Solomon's reputation even more famous than yours, and may he make his throne greater than yours!' The king has himself bowed in worship on his own bed\fnote{\fbackref{1:47} I.e. a possible allusion to sacred oaths such as Joseph's promise to Jacob in Gen 47:31} \v{48}and said `Blessed be the \divine{Lord} God of Israel, who has provided someone to sit on my throne today. I've seen it with my own eyes!'\,''

\v{49}Terrified, all of Adonijah's guests jumped up and ran away. \v{50}Afraid of Solomon, Adonijah also jumped up and headed straight for the horns of the altar.\fnote{\fbackref{1:50} I.e. the altar associated with sacrifices in the tent}

\v{51}``Hey look!'' somebody informed Solomon. ``Adonijah is terrified of King Solomon! He's gone out, grabbed hold of the horns of the altar, and now he's begging King Solomon, `Swear to me that you won't put your servant to death with a sword!'\,''

\v{52}``If he's done nothing wrong, not a hair of his head will be harmed,'' Solomon replied. ``But if we find evil in him, he's a dead man.''

\v{53}So King Solomon sent for him, and he was brought down from the altar. When he had arrived, he fell on his face in front of King Solomon, so Solomon told him, ``Go home!''
\labelchapt{2}
\passage{David Instructs Solomon}

\chapt{2}
\v{1}As David's time to die approached, he addressed his son Solomon with these words:

\begin{poetry}
\poeml \v{2}``I'm headed down the road that everyone who lives on earth travels, so be strong and demonstrate that you're a grown man \v{3}by keeping the charge that the \divine{Lord} your God entrusted to you. Live life his way, keep his statutes, his commands, his ordinances, and his testimonies, just as they're written down in the Law of Moses, so that you may succeed in everything you do and wherever you go,\fnote{\fbackref{2:3} Lit. \fbib{turn}} \v{4}and so that the \divine{Lord} may fulfill his promise that he spoke about me when he said, `If your sons pay attention to how they live by walking truthfully in my presence with all their heart and with all their soul, you will never lack a man on the throne of Israel.' \\
\poeml \v{5}``Furthermore, you're aware of what Zeruiah's son Joab did to me and to those two commanders of the armies of Israel, Ner's son Abner and Jether's son Amasa, whom he killed, and how he shed the blood of wartime during times of peace, staining the very belt he wears around his waist and the sandals he wears on his feet. \v{6}So act consistently with your wisdom, and don't let him die as a peaceful old man.\fnote{\fbackref{2:6} Lit. \fbib{let his gray hair descend to Sheol in peace}} \v{7}Be gracious to the descendants of Barzillai the Gileadite, and provide for them in your household,\fnote{\fbackref{2:7} Lit. \fbib{them at your table}} because they helped me when I had to run from your brother Absalom. \\
\poeml \v{8}``Pay attention now! You have with you Gera's son Shimei the descendant of Benjamin from Bahurim. He cursed me violently that day when I had to leave for Mahanaim. When he visited me at the Jordan River,\fnote{\fbackref{2:8} The Heb. lacks \fbib{River}} I made an oath to the \divine{Lord} and told him, `I won't execute you with a sword.' \v{9}But don't let him off unpunished, since you're a wise man and you'll know what you need to do to him. Find a way that he dies in his old age\fnote{\fbackref{2:9} Lit. \fbib{Bring his gray hair down to Sheol}} by shedding his blood.''
\end{poetry}
\passage{David Dies and Solomon Consolidates His Reign}
\passageinfo{(1 Chronicles 3:4; 29:26-28)}

\v{10}After this, David died, as had\fnote{\fbackref{2:10} Lit. \fbib{David slept with}} his ancestors, and he was buried in the City of David. \v{11}David had reigned over Israel for 40 years. He reigned in Hebron for seven years and in Jerusalem for 33 years. \v{12}Solomon then assumed his father David's throne, and his kingdom was firmly established.
\passage{Adonijah asks for Abishag}

\v{13}Later, Haggith's son Adonijah approached Solomon's mother. ``Are you here on a peaceful mission?'' she asked.

``Yes,'' he replied. \v{14}``I have something to ask you about.''

``Talk,'' she told him.

\v{15}So he replied, ``You know that the kingdom should have come to me, and that everyone in Israel intended to place me as the next\fnote{\fbackref{2:15} The Heb. lacks \fbib{the next}} king. However, the kingdom has turned around and now belongs to my brother, because it went to him from the \divine{Lord}. \v{16}So now I'm asking one thing from you. Don't refuse me.''

``Talk,'' she told him.

\v{17}Then he asked her, ``Please talk to King Solomon for me, since he won't refuse you. Ask him to give me Abishag the Shunammite as a wife.''

\v{18}``Very well,'' Bathsheba replied. ``I'll talk to the king for you.'' \v{19}So Bathsheba went to talk to King Solomon for Adonijah. The king rose to meet her, bowed to her, and sat down on his throne. He ordered a throne be set in place for his mother. She sat on a throne to his right \v{20}and told him,\fnote{\fbackref{2:20} The Heb. lacks \fbib{to him}} ``I would like to make a minor request of you. Please don't refuse me.''

``What is your request, mother?'' the king asked her. ``I won't turn you down.''

\v{21}So she asked him, ``Give Abishag the Shunammite to your brother Adonijah as a wife.''

\v{22}But King Solomon replied to his mother, ``Why are you asking Abishag the Shunammite for Adonijah? Why not ask me to give up the kingdom for him, since he's my older brother, and why not ask\fnote{\fbackref{2:22} The Heb. lacks \fbib{why not ask}} for Abiathar the priest, and for Zeruiah's son Joab?''

\v{23}Then King Solomon took this oath in the name of the \divine{Lord}: ``May God do so to me, and more besides, if Adonijah hasn't endangered his life by bringing up this subject. \v{24}Now therefore, as the \divine{Lord} lives, who has established me and set me on the throne of my father David, and who has established a dynasty, just like he promised, Adonijah will surely be executed today.'' \v{25}So King Solomon sent for Jehoiada's son Benaiah, who attacked and killed Adonijah.\fnote{\fbackref{2:25} Lit. \fbib{him}}

\v{26}The king also told Abiathar the priest, ``Go home to Anathoth. You deserve to die, but I won't kill you today, because you carried the ark of the Lord \divine{God} before my father David and because you shared all the troubles that my father went through.'' \v{27}So Solomon fired Abiathar as the \divine{Lord}'s priest, thus fulfilling the promise that the \divine{Lord} had spoken in Shiloh concerning Eli's household.\fnote{\fbackref{2:27} Cf. 1Sam 2:27-36}
\passage{Joab is Executed}

\v{28}When Joab learned what had happened, he ran to the \divine{Lord}'s tent and grabbed hold of the horns of the altar, since Joab had supported Adonijah (though he had not supported Absalom). \v{29}Somebody informed King Solomon, ``Joab just ran to the \divine{Lord}'s tent and now he's standing beside the altar!''

But Solomon ordered Jehoiada's son Benaiah, ``Go kill him!''

\v{30}So Benaiah went into the \divine{Lord}'s tent and told Joab,\fnote{\fbackref{2:30} Lit. \fbib{him}} ``The king orders you to come out!''

``No,'' Joab said, ``I'd rather die here!''

So Benaiah went and informed the king, ``This is how Joab answered me.''

\v{31}The king replied to him, ``Do just what he asked. Kill him and bury him so that you may remove from me and from my father's household the guilt that Joab shed needlessly. \v{32}The \divine{Lord} will repay him for his bloodshed because, without my father David's consent he attacked and murdered two men more righteous and better than he, Ner's son Abner, the commander of Israel's army and Jether's son Amasa, commander of Judah's army. \v{33}May their blood be repaid to Joab and to his descendants forever, and may there be peace shown from the \divine{Lord} forever to David, to his descendants, to his household, and to his throne.''

\v{34}Jehoiada's son Benaiah then approached Joab, attacked him, killed him, and had him buried at Joab's\fnote{\fbackref{2:34} Lit. \fbib{his}} home in the wilderness. \v{35}The king appointed Jehoiada's son Benaiah in charge of the army to replace Joab and also appointed Zadok the priest to replace Abiathar.
\passage{Shimei is Executed}

\v{36}The king sent for Shimei and told him, ``Build yourself a house in Jerusalem and live there, but don't go anywhere from there. \v{37}If you ever leave and cross the Kidron Brook, you can be sure that you'll die. You'll be responsible for your own death.''

\v{38}Shimei replied to the king, ``What your majesty has decreed is acceptable to me. I'll do what you've said.'' So Shimei lived in Jerusalem for quite some time. \v{39}But three years later, two of Shimei's servants escaped to Maacah's son Achish, the king of Gath.

Somebody told Shimei, ``Look! Your servants went to Gath!'' \v{40}So Shimei got up, saddled a donkey, and traveled to Gath to find his servants. He found them and brought them back from Gath.

\v{41}Later, Solomon found out that Shimei had left Jerusalem, gone to Gath, and had returned, \v{42}so the king sent for Shimei and asked him, ``Didn't I make a promise to the \divine{Lord} and warn you, `The day you leave and go anywhere else, you can be sure you'll die'? And you told me, `What your majesty has decreed is acceptable to me.' \v{43}So why haven't you kept the oath you made to the \divine{Lord}, and why didn't you obey my personal order to you?''

\v{44}The king also reminded Shimei, ``You know all the evil things that you admit you did to my father David. Therefore the \divine{Lord} is going to repay you for\fnote{\fbackref{2:44} Lit. \fbib{repay on your head}} all of your evil. \v{45}But King Solomon will be blessed, and David's throne will be established in the presence of the \divine{Lord} forever.'' \v{46}So the king gave orders to Jehoiada's son Benaiah to go out, attack Shimei, and kill him. That is how the kingdom was established under Solomon's control.
\labelchapt{3}
\passage{Solomon Prays for Wisdom}
\passageinfo{(2 Chronicles 1:2-13)}

\chapt{3}
\v{1}Later, Solomon intermarried with the family of\fnote{\fbackref{3:1} Lit. The Heb. lacks \fbib{the family of}} Pharaoh, the king of Egypt by taking his daughter and bringing her to the City of David to live until he had completed building his own palace, the \divine{Lord}'s Temple, and the wall around Jerusalem. \v{2}The people were sacrificing at various high places because the Temple had not yet been built and dedicated to\fnote{\fbackref{3:2} Lit. \fbib{built for the name of}} the \divine{Lord}.

\v{3}Solomon loved the \divine{Lord}, and lived according to the statutes that his father David obeyed, except that he sacrificed and burned offerings at the high places. \v{4}The king used to go to Gibeon to sacrifice, since there was a famous high place there, where Solomon once offered 1,000 burnt offerings on that altar. \v{5}The \divine{Lord} appeared to Solomon one night in a dream and told him, ``Ask me for whatever you want and I'll give it to you.''

\v{6}So Solomon said:

\begin{poetry}
\poeml ``You have demonstrated abundant gracious love to your servant David, my father, as he lived in your presence truthfully, righteously, and uprightly in his heart. In addition, you have kept on showing this abundant gracious love by giving him a son to sit on his throne today. \v{7}Now, \divine{Lord} my God, you have set me as king to replace my father David, but I'm still young. I don't have any leadership skills.\fnote{\fbackref{3:7} Lit. \fbib{I'm} only \fbib{a youth and don't know how to come and go}} \v{8}Your servant lives in the midst of your people that you have chosen, a great people that is too numerous to be counted. \v{9}So give your servant an understanding mind to govern your people, so I can discern between good and evil. Otherwise, how will I be able to govern this great people of yours?''
\end{poetry}

\v{10}The \divine{Lord} was pleased that Solomon had asked for this, \v{11}so God told him:

\begin{poetry}
\poeml ``Because you asked for this, and you didn't ask for a long life for yourself, and you didn't ask for the lives of your enemies, but instead you've asked for discernment so you can understand how to govern, \v{12}look how I'm going to do precisely what you asked. I'm giving you a wise and discerning mind, so that there will have been no one like you before you and no one will arise after you like you. \v{13}I'm also giving you what you haven't requested: both riches and honor, so that no other king will be comparable to you during your lifetime. \v{14}If you will live life my way, keeping my statutes and my commands, just like your father David did, I'll also increase the length of your life.''
\end{poetry}

\v{15}Then Solomon woke up and realized that he had dreamed a dream. Then he went back to Jerusalem, stood before the ark of the \divine{Lord}'s covenant, offered burnt offerings and peace offerings, and threw a party for all of his servants.
\passage{Solomon's Wisdom is Tested}

\v{16}Right about then, two prostitutes approached the king and requested an audience with him. \v{17}One woman said, ``Your majesty, this woman and I live in the same house. I gave birth to a child while she was in the house. \v{18}Three days later, this woman also gave birth. We lived alone there. There was nobody else with us in the house. It was just the two of us. \v{19}This woman's son died overnight because she laid on top of him. \v{20}She got up in the middle of the night, took my son from me while your servant was asleep, and laid him to her breast after laying her dead son next to me. \v{21}The next morning, I got up to nurse my son, and he was dead. But when I examined him carefully in the light of day, he turned out not to be my son whom I had borne!''

\v{22}``Not so,'' claimed the other woman. ``The living child is my son, and the dead one is yours.''

But the first woman said, ``Not so! The dead child is your son and the living one is my son.'' This is what they testified before the king.

\v{23}The king said, ``One of them claims, `This living son is mine, and your son is the dead one' and the other claims `No. Your son is the dead one and my son is the living one.' \v{24}``Somebody get me a sword.'' So they brought a sword to the king. \v{25}``Divide the living child in two!'' he ordered. ``Give half to the one and half to the other.''

\v{26}The woman whose child was still alive cried out to the king, because her heart yearned for her son. ``Oh no, your majesty!'' she said. ``Give her the living child. Please don't kill him.''

But the other woman said, ``Cut him in half! That way, he'll belong to neither one of us.''

\v{27}The king announced his decision: ``Give the living child to the first woman. Don't kill him. She is his mother.'' \v{28}When this decision that the king had handed down was announced, everybody in Israel was amazed at\fnote{\fbackref{3:28} Lit. \fbib{Israel feared}} the king, because they all saw that God's wisdom was in him, enabling him to administer justice.
\labelchapt{4}
\passage{Solomon's Administration}

\chapt{4}
\v{1}And so King Solomon ruled over all of Israel. \v{2}Here's a list of his officials: Zadok's son Azariah was priest, \v{3}Shisha's sons Elihoreph and Ahijah were his secretaries, Ahilud's son Jehoshaphat was recorder, \v{4}Jehoiada's son Benaiah commanded the army, Zadok and Abiathar served as priests, \v{5}Nathan's son Azariah supervised the governors, Nathan's son Zabud the priest was the king's counselor, \v{6}Ahishar supervised palace matters, and Abda's son Adoniram supervised conscripted labor. \v{7}Solomon also appointed twelve governors over all of Israel, each of whom were responsible for providing one month's food provisions to the king and to his administration during each year.

\v{8}Here's a list of their names: Ben-hur from the hill country of Ephraim; \v{9}Ben-deker in Makaz, Shaalbim and Beth-shemesh and Elonbeth-hanan; \v{10}Ben-hesed served in Arubboth (where he supervised Socoh and all of the territory of Hepher); \v{11}Ben-abinadab supervised the Dor heights (Solomon's daughter Taphath was his wife); \v{12}Ahilud's son Baana served Taanach, Megiddo, and all of Beth-shean near Zarethan below Jezreel, including from Beth-shean to Abel-meholah as far as the other side of Jokmeam; \v{13}Ben-geber in Ramoth-gilead, including the towns that belonged to Manasseh's descendant Jair that are in Gilead; \v{14}Iddo's son Ahinadab served in Mahanaim; \v{15}Ahimaaz served in Naphtali (he was married to Solomon's daughter Basemath); \v{16}Hushai's son Baana served in Asher and Bealoth; \v{17}Paruah's son Jehoshaphat served in Issachar; \v{18}Ela's son Shimei served in Benjamin; \v{19}and Uri's son Geber served in the territory of Gilead, the territory formerly ruled by King Sihon of the Amorites and King Og of Bashan (he was the only governor over that territory).
\passage{Solomon's Magnificence}

\v{20}Judah and Israel became as numerous as the sand on the seashore. They enjoyed abundance, and ate, drank, and rejoiced regularly. \v{21}\fnote{\fbackref{4:21} This v. is 5:1 in MT, 4:22 is 5:2, and so on through 4:34}Solomon ruled over all the kingdoms from the Euphrates River\fnote{\fbackref{4:21} The Heb. lacks \fbib{River}} to the territory of the Philistines and south\fnote{\fbackref{4:21} The Heb. lacks \fbib{south}} to the border of Egypt. They brought tribute and served Solomon throughout his lifetime. \v{22}Solomon's daily provisions were 30 kors of fine flour, 60 kors of meal, \v{23}ten fattened oxen, 20 pasture-fed cattle, 100 sheep, as well as deer, gazelles, roebucks, and domestic poultry. \v{24}He ruled over everything west of the Euphrates\fnote{\fbackref{4:24} The Heb. lacks \fbib{Euphrates}} River from Tiphsah to Gaza, over all of the kings west of the Euphrates\fnote{\fbackref{4:24} The Heb. lacks \fbib{Euphrates}} River, and he enjoyed peace on all sides around him.

\v{25}Judah and Israel lived safely, and everyone enjoyed their own vine and fig tree from Dan to Beer-sheba through all of Solomon's life. \v{26}Solomon owned 40,000 stalls for the horses that drove his chariots, and he employed 12,000 men to drive them.\fnote{\fbackref{4:26} The Heb. lacks \fbib{to drive them}} \v{27}His officers supplied provisions for King Solomon and for everyone who visited King Solomon's palace,\fnote{\fbackref{4:27} Lit. \fbib{table}} each in their respective month of service responsibility.\fnote{\fbackref{4:27} The Heb. lacks \fbib{of service responsibility}} Nothing ever ran out. \v{28}They also provided barley and straw for the horses and camels to their respective locations, each consistent with their responsibilities.
\passage{Solomon's Fame}

\v{29}God gave Solomon wisdom and great discernment. His insights were as numerous as sand on the seashore. \v{30}Solomon was wiser than any of the eastern leaders and wiser than anyone in Egypt. \v{31}He was wiser than anyone of his day---wiser than Ethan the Ezrahite, Heman, and wiser than Mahol's sons Calcol and Darda.

His reputation was known throughout the surrounding nations. \v{32}Solomon wrote 3,000 proverbs and 1,005 songs. \v{33}He described trees---everything from cedars\fnote{\fbackref{4:33} I.e. a genus of coniferous evergreen in the family \fbib{Pinaceae}; and so throughout the book} that grow in Lebanon to hyssop that grows on a garden wall. He described animals, birds, reptiles, and fish. \v{34}People came from everywhere to hear Solomon's advice. Every king on the earth heard of his wisdom.
\labelchapt{5}
\passage{Preparations to Build the Temple}
\passageinfo{(2 Chronicles 2:1-18)}

\chapt{5}
\v{1}\fnote{\fbackref{5:1} This v. is 4:15 in MT, 5:2 is 4:16, and so on through 5:18, which is 4:32 in MT}King Hiram of Tyre sent his servants to Solomon when he learned that Solomon\fnote{\fbackref{5:1} Lit. \fbib{he}} had been anointed king to replace his father, because Hiram had been David's lifelong friend.\fnote{\fbackref{5:1} Lit. \fbib{David's friend all his days}} \v{2}Solomon sent this message to Hiram:

\begin{poetry}
\poeml \v{3}``You know that my father David was unable to build a temple dedicated to\fnote{\fbackref{5:3} Lit. \fbib{temple to the name of}} the \divine{Lord} his God because he was busy fighting wars all around him until the \divine{Lord} defeated his enemies. \v{4}But now the \divine{Lord} has given me rest all around, since I have neither foreign adversaries nor domestic crises. \v{5}So now I'm planning to build a temple dedicated to\fnote{\fbackref{5:5} Lit. \fbib{temple to the name of}} the \divine{Lord} my God, just as the \divine{Lord} told my father when he said, `Your son, whom I will set on your throne to replace you, will build the Temple dedicated to me.'\fnote{\fbackref{5:5} Lit. \fbib{to my name}} \v{6}Now therefore please order that cedars of Lebanon be cut for me. My servants will work with your servants, and I will pay your servants whatever wages you set, because you know there is no one among us who knows how to cut timber like the Sidonians do.''
\end{poetry}

\v{7}As soon as Hiram received the message from Solomon, he became so ecstatic that he exclaimed, ``Blessed be the \divine{Lord} today, who has given David a wise son to rule this great people!'' Then he sent this message to Solomon:

\begin{poetry}
\poeml \v{8}``I have read the letter that you sent me. I'll do what you've asked about the cedar and cypress timber. \v{9}My servants will transport them from Lebanon to the sea, where we'll make them into rafts and float them by sea to the port that you tell me to send them. We'll have them prepared for transport there and then you can carry them from there. You can meet my needs by providing provisions for my household.''
\end{poetry}

\v{10}That's how Hiram came to provide Solomon as much cedar and cypress timber as he needed. \v{11}In return, Solomon paid Hiram 20,000 kors of wheat as food for his household, and 20 kors of beaten oil. Solomon provided this amount every year during the construction.\fnote{\fbackref{5:11} The Heb. lacks \fbib{during the construction}}

\v{12}The \divine{Lord} continued giving Solomon wisdom, just as he had promised, and Hiram and Solomon entered into a peace treaty between themselves.
\passage{Conscripted Labor for the Building Program}

\v{13}King Solomon conscripted laborers from throughout Israel. The work force numbered 30,000 men. \v{14}He sent 10,000 men to Lebanon in shifts lasting one month. They worked one month in Lebanon for every two months they worked at home. Adoniram was placed in charge of the conscripted labor. \v{15}Solomon also employed 70,000 heavy-lift workers and 80,000 stonecutters in the hill country. \v{16}Solomon also employed 3,300 officials to supervise the work and to manage the people employed in the construction. \v{17}The king specified that large, expensive stones be quarried so the foundation of the Temple could be laid with cut stones. \v{18}As a result, Solomon's builders worked with Hiram's builders, accompanied by the Gebalites, to quarry the stone and to prepare the timber and other\fnote{\fbackref{5:18} The Heb. lacks \fbib{other}} stone for the Temple's construction.
\labelchapt{6}
\passage{Temple Construction Begins}
\passageinfo{(2 Chronicles 3:1-14)}

\chapt{6}
\v{1}During the month of Ziv, which was the second month of the fourth year of Solomon's reign over Israel, 480 years after the Israelis left the land of Egypt, Solomon began to build the \divine{Lord}'s Temple. \v{2}The Temple for the \divine{Lord} that Solomon was building was 60 cubits\fnote{\fbackref{6:2} I.e. about 90 feet; a cubit was about eighteen inches} long and 20 cubits\fnote{\fbackref{6:2} I.e. about 30 feet; a cubit was about eighteen inches} wide. \v{3}A portico extended in front of the Temple for 20 cubits\fnote{\fbackref{6:3} I.e. about 30 feet; a cubit was about eighteen inches} outward, corresponding to the width of the Temple. Along the front of the Temple its depth was ten cubits.\fnote{\fbackref{6:3} I.e. about fifteen feet; a cubit was about eighteen inches} \v{4}Solomon\fnote{\fbackref{6:4} Lit. \fbib{He}} also constructed windows in the Temple with specially designed\fnote{\fbackref{6:4} Or \fbib{latticed}} frames.

\v{5}Against the wall of the Temple he built a series of rooms that encompassed the exterior of the Temple walls around the inner sanctuary. He built these side chambers all around the building.\fnote{\fbackref{6:5} The Heb. lacks \fbib{the building}} \v{6}The lower structures were five cubits\fnote{\fbackref{6:6} I.e. about seven and a half feet; a cubit was about eighteen inches} wide, the middle structures were six cubits\fnote{\fbackref{6:6} I.e. about nine feet; a cubit was about eighteen inches} wide and the third structures were seven cubits\fnote{\fbackref{6:6} I.e. about ten and a half feet; a cubit was about eighteen inches} wide. Offsets were placed all around the Temple so that beams would not protrude through the walls of the Temple. \v{7}The Temple was constructed of stone precut at the quarry so that no hammer, axe, or any other iron implement would be heard in the Temple while it was being built. \v{8}A passageway to the side chamber was constructed on the south side of the Temple by which people\fnote{\fbackref{6:8} Lit. \fbib{they}} could ascend winding stairs to the middle story, then from there to the third story.
\passage{Interior Finishing with Gold and Cedar}

\v{9}After Solomon\fnote{\fbackref{6:9} Lit. \fbib{he}} built the Temple and finished it, he covered the Temple with beams and planks made of cedar. \v{10}He constructed this structure to adjoin the entire Temple, five cubits\fnote{\fbackref{6:10} I.e. about seven and a half feet; a cubit was about eighteen inches} high, and fastened it to the Temple with cedar timbers.

\v{11}Then this message from the \divine{Lord} came to Solomon: \v{12}``Concerning\fnote{\fbackref{6:12} The Heb. lacks \fbib{Concerning}} this Temple that you're building, if you live your life\fnote{\fbackref{6:12} Lit. \fbib{you walk}} according to my statutes, carry out my ordinances, and keep all of my commands, and live according to them, then I will do what I promised to your father David. \v{13}I will reside among the Israelis and will never abandon my people Israel.''

\v{14}So Solomon kept on building the Temple and finished it. \v{15}Then he built the inside walls of the Temple, lining them from floor to ceiling with cedar boards, and overlaying the Temple floor with boards made of cypress wood. \v{16}He lined 20 cubits\fnote{\fbackref{6:16} I.e. about 30 feet; a cubit was about eighteen inches} of the rear part of the Temple from floor to ceiling with cedar boards specially constructed for the inside to serve as the Most Holy Place. \v{17}The rest of the main nave in the front was 40 cubits\fnote{\fbackref{6:17} I.e. about 60 feet; a cubit was about eighteen inches} long. \v{18}Cedar\fnote{\fbackref{6:18} I.e. a genus of coniferous evergreen in the family \fbib{Pinaceae}; and so throughout the book} carvings in the form of gourds and blooming flowers covered the entire interior of the Temple so that no stone could be seen.

\v{19}Solomon\fnote{\fbackref{6:19} Lit. \fbib{He}} also prepared an inner sanctuary within the Temple where the \divine{Lord}'s Ark of the Covenant was placed. \v{20}The inner sanctuary was 20 cubits\fnote{\fbackref{6:20} I.e. about 30 feet; a cubit was about eighteen inches} long, 20 cubits\fnote{\fbackref{6:20} I.e. about 30 feet; a cubit was about eighteen inches} wide, and 20 cubits\fnote{\fbackref{6:20} I.e. about 30 feet; a cubit was about eighteen inches} high, and overlaid with pure gold. The altar was also overlaid with cedar. \v{21}Solomon overlaid the inside of the Temple with pure gold, fastened gold chains across the front of the inner sanctuary, and overlaid it with gold. \v{22}He finished the Temple by overlaying it entirely with gold, including overlaying with gold the whole altar that was by the inner sanctuary.
\passage{Temple Furnishings}
\passageinfo{(2 Chronicles 4:1-10, 19-22; 5:1)}

\v{23}Inside the inner sanctuary Solomon\fnote{\fbackref{6:23} Lit. \fbib{he}} placed two cherubim crafted from olive wood, each ten cubits\fnote{\fbackref{6:23} I.e. about fifteen feet; a cubit was about eighteen inches} high. \v{24}Each wing of one cherub was five cubits\fnote{\fbackref{6:24} I.e. about seven and a half feet; a cubit was about eighteen inches} long, and each wing of the other cherub was five cubits\fnote{\fbackref{6:24} I.e. about seven and a half feet; a cubit was about eighteen inches} long, so that the distance from the end of one wing to the end of the other wing was ten cubits.\fnote{\fbackref{6:24} I.e. about fifteen feet; a cubit was about eighteen inches} \v{25}Each cherub was ten cubits\fnote{\fbackref{6:25} I.e. about fifteen feet; a cubit was about eighteen inches} high, and both were of the same size and shape, \v{26}the height of one cherub being ten cubits,\fnote{\fbackref{6:26} I.e. about fifteen feet; a cubit was about eighteen inches} as was the height of the other.

\v{27}Solomon\fnote{\fbackref{6:27} Lit. \fbib{He}} placed the cherubim in the middle of the inner sanctuary, with their wings spread in such a way that the wing of one was touching the one wall and the opposite wing of the other cherub was touching the opposite wall. Furthermore, their wings in the center of the wall were touching each other wing-to-wing. \v{28}Each cherub was overlaid with gold.

\v{29}Solomon\fnote{\fbackref{6:29} Lit. \fbib{He}} also inlaid all the inner walls of the Temple---both the inner and outer sanctuaries---with carved engravings of cherubim, palm trees, and blooming flowers. \v{30}He also overlaid the floor of the Temple with gold in both the inner and outer sanctuaries.

\v{31}Solomon\fnote{\fbackref{6:31} Lit. \fbib{He}} also provided doors, lintels, and five-sided doorposts for the entrance to the inner sanctuary. \v{32}He installed two doors made of olive wood, inlaying them with carvings of cherubim, palm trees, and blooming flowers, and overlaying them with gold. Then he added more gold to cover the cherubim and palm trees.

\v{33}Solomon\fnote{\fbackref{6:33} Lit. \fbib{He}} also provided four-sided doorposts made of cypress wood for the entrance to the outer sanctuary, \v{34}along with two doors of cypress wood, one door of which had two leaves that turned on hinges, as did the other door, which also had two leaves that turned on hinges.

\v{35}Solomon\fnote{\fbackref{6:35} Lit. \fbib{He}} also inlaid the doors with\fnote{\fbackref{6:35} The Heb. lacks \fbib{the doors with}} cherubim, palm trees, and blooming flowers. He overlaid them with gold that was carefully\fnote{\fbackref{6:35} Or \fbib{evenly}} applied on the engraved work. \v{36}He constructed the inner court with three rows of precut stone and a row of cedar beams.
\passage{Temple Construction is Completed}

\v{37}The foundation for the \divine{Lord}'s Temple was laid in the month of Ziv during the fourth year of Solomon's reign, \v{38}and the Temple was completely finished according to its plans and specifications in the eighth month of the eleventh year of Solomon's\fnote{\fbackref{6:38} Lit. \fbib{his}} reign, that is, during the month of Bul. It took about seven years to build.
\labelchapt{7}
\passage{Solomon's Palace}

\chapt{7}
\v{1}But Solomon took thirteen years to build his own palace, and finally finished it. \v{2}He built his own palace out of timber supplied from the forest of Lebanon. It was 100 cubits\fnote{\fbackref{7:2} I.e. about 150 feet; a cubit was about eighteen inches} long, 50 cubits\fnote{\fbackref{7:2} I.e. about 75 feet; a cubit was about eighteen inches} wide, 20 cubits\fnote{\fbackref{7:2} I.e. about 30 feet; a cubit was about eighteen inches} tall, and was constructed on four rows of cedar pillars, with cedar beams interlocking the pillars. \v{3}There were 45 pillars paneled with cedar above the side chambers, with rows of fifteen pillars, \v{4}with three rows of framed windows facing each other in three ranks. \v{5}All the doorways and doorposts had rectangular frames, with the doorways facing each other in three tiers. \v{6}There was also a hall of pillars 50 cubits\fnote{\fbackref{7:6} I.e. about 75 feet; a cubit was about eighteen inches} long and 30 cubits\fnote{\fbackref{7:6} I.e. about 45 feet; a cubit was about eighteen inches} wide, and a porch in front with pillars, and a canopy in front of the pillars.\fnote{\fbackref{7:6} Lit. \fbib{of them}} \v{7}He constructed the Judgment Hall for the throne room where he would be ruling, paneling it with cedar from floor to ceiling.\fnote{\fbackref{7:7} Lit. \fbib{floor to floor}} \v{8}Solomon's\fnote{\fbackref{7:8} Lit. \fbib{His}} personal dwelling quarters, a separate court behind the hall, was of similar workmanship. Solomon\fnote{\fbackref{7:8} Lit. \fbib{He}} also built a house similar to this for Pharaoh's daughter, whom Solomon had married.

\v{9}All of these were made with expensive stones, pre-cut according to specifications, hand-sawed inside and out from the foundation to the coping, including from inside to the great court. \v{10}The foundation was made of expensive stone, including large stones ten cubits\fnote{\fbackref{7:10} I.e. about 15 feet; a cubit was about eighteen inches} long and stones eight cubits\fnote{\fbackref{7:10} I.e. about 12 feet; a cubit was about eighteen inches} long. \v{11}Above these were expensive stones cut according to specifications, and cedar. \v{12}So the great court was surrounded by three rows of cut stone, along with a row of cedar beams, just like the inner court of the \divine{Lord}'s Temple and the porch surrounding the Temple.
\passage{Contributions by Hiram the Bronzeworker}
\passageinfo{(2 Chronicles 3:15-17; 4:11-18)}

\v{13}King Solomon sent for Hiram\fnote{\fbackref{7:13} 2Chr 2:13 identifies the man as \fbib{Hiram-abi}} from Tyre, \v{14}the son of a widow from the tribe of Naphtali, whose father was from Tyre. A bronze worker, he was wise, knowledgeable, and was skilled in all sorts of bronze working. He went to King Solomon and did all of his work.

\v{15}He fashioned two bronze pillars, each one eighteen cubits\fnote{\fbackref{7:15} I.e. about 27 feet; a cubit was about eighteen inches} high, with a circumference of twelve cubits.\fnote{\fbackref{7:15} I.e. about 18 feet; a cubit was about eighteen inches} \v{16}He also crafted two capitals of cast bronze and set them on top of the pillars. The height of one capital was five cubits,\fnote{\fbackref{7:16} I.e. about seven and a half feet; a cubit was about eighteen inches} and the height of the other capital was five cubits.\fnote{\fbackref{7:16} I.e. about seven and a half feet; a cubit was about eighteen inches} \v{17}A network of latticework on top of the pillars was inlaid with ornamental wreaths and chains, the top of each pillar containing seven groups of ornamental structures. \v{18}The pillars contained two rows of ornaments shaped like pomegranates around the latticework covering the top of each pillar. \v{19}The capitals on top of each pillar above the rounded latticework contained four cubits\fnote{\fbackref{7:19} I.e. about six feet; a cubit was about eighteen inches} of lily designs, \v{20}with the capitals on the two pillars covered by 200 pomegranates in rows around both the capitals above and adjoining the rounded latticework. \v{21}That's how he designed the pillars at the portico of the sanctuary. When he set up the right pillar, he named it Jachin.\fnote{\fbackref{7:21} The name means \fbib{He Established}} When he set up the left pillar, he named it Boaz.\fnote{\fbackref{7:21} The name means \fbib{In Strength}} \v{22}The work on the pillars was finished with a lily design on top of the pillars.
\passage{The Bronze Sea}

\v{23}Hiram\fnote{\fbackref{7:23} Lit. \fbib{He}} also made a sea of cast metal ten cubits\fnote{\fbackref{7:23} I.e. about fifteen feet; a cubit was about eighteen inches} from brim to brim, circular in shape and five cubits\fnote{\fbackref{7:23} I.e. about seven and a half feet; a cubit was about eighteen inches} and 30 cubits\fnote{\fbackref{7:23} I.e. 45 feet; a cubit was about eighteen inches} in its inner circumference. \v{24}Under the brim, completely encircling it, were two rows of gourds inlaid as part of the original casting, ten to a cubit.\fnote{\fbackref{7:24} I.e. ten in each one and a half feet; a cubit was about eighteen inches} \v{25}The sea stood on top of twelve oxen. Three faced north, three faced west, three faced south, and three faced east. The sea was set on top of them, and their hind parts faced the center.\fnote{\fbackref{7:25} Lit. \fbib{were inward}} \v{26}The reservoir, which held about 2,000 baths,\fnote{\fbackref{7:26} I.e. about 12,000 gallons; Cf. 2Chron 4:52, where the volume is given at 3,000 baths} stood about a handbreadth\fnote{\fbackref{7:26} I.e. about three inches; a handbreadth was about one sixth of a cubit} thick, and its rim looked like the brim of a cup or of a lily blossom.
\passage{The Ten Water Carts}

\v{27}Hiram\fnote{\fbackref{7:27} Lit. \fbib{He}} also made ten bronze water carts.\fnote{\fbackref{7:27} Or \fbib{stands}, and so throughout this paragraph} Each one was four cubits\fnote{\fbackref{7:27} I.e. about six feet; a cubit was about eighteen inches} wide, four cubits long,\fnote{\fbackref{7:27} I.e. about six feet; a cubit was about eighteen inches} and three cubits\fnote{\fbackref{7:27} I.e. about four and a half feet; a cubit was about eighteen inches} high. \v{28}The carts were designed with borders between cross-pieces, \v{29}and on the borders between the cross-pieces were lions, oxen, and cherubim. A pedestal was placed above the cross-pieces, and beneath the lions and oxen there were wreaths hanging down. \v{30}Each cart had four bronze wheels equipped with bronze axles with four support feet. Beneath the basin were cast support structures made like wreaths on each side. \v{31}The opening to each water cart inside the crown on top was one cubit\fnote{\fbackref{7:31} I.e. about one and a half feet; a cubit was about eighteen inches} wide, with engravings on the opening. The borders to the frames surrounding the opening were square, not round. \v{32}The four wheels were placed underneath the borders, and the axles for the wheels were on the stand. Each wheel stood one and a half cubits\fnote{\fbackref{7:32} I.e. about 27 inches; a cubit was about eighteen inches} high. \v{33}The wheels resembled those of a chariot, with their axles, rims, spokes, and hubs made of cast bronze. \v{34}Four supports stood at the four corners of each cart, built into the carts themselves. \v{35}On top of each stand was a circular structure one half of one cubit\fnote{\fbackref{7:35} I.e. about 9 inches; a cubit was about eighteen inches} high, with its braces and support frames integral with it, forming a single piece. \v{36}Hiram\fnote{\fbackref{7:36} Lit. \fbib{He}} engraved ornamental cherubim, lions, and palm trees on the surfaces of the supports and frames wherever there was space to do so, and encircled the artwork with wreaths. \v{37}He made ten identical water carts by using the same plans, castings, and shapes for all of them.
\passage{The Other Bronze Implements}

\v{38}Hiram\fnote{\fbackref{7:38} Lit. \fbib{He}} also fashioned ten bronze basins, each holding about 40 baths,\fnote{\fbackref{7:38} I.e. about 240 gallons; a bath held about six gallons} each basin measuring four cubits\fnote{\fbackref{7:38} I.e. about six feet; a cubit was about eighteen inches} in diameter,\fnote{\fbackref{7:38} The Heb. lacks \fbib{in diameter}} with one basin for each stand. \v{39}He set five of the stands on the right side of the Temple and five on the left side of the Temple. He set the bronze sea on the right side of the Temple eastward facing the south. \v{40}Hiram also made the basins, shovels, and bowls to complete the work that he performed for King Solomon in the \divine{Lord}'s Temple, \v{41}including the two pillars and the bowls for the capitals that stood on top of the two pillars, along with the two lattices that covered the two bowls of the capitals that stood on top of the pillars, \v{42}plus the 400 pomegranates for the two lattices (that is, the two rows of pomegranates for each lattice to cover the two bowls of the capitals that stood on top of the pillars), \v{43}the ten stands with the ten basins on the stands, \v{44}the single bronze\fnote{\fbackref{7:44} The Heb. lacks \fbib{bronze}} sea and the twelve oxen that stood under the sea, \v{45}and the pots, shovels, and bowls---all of these utensils that Hiram made for King Solomon for the \divine{Lord}'s Temple were made from polished bronze.

\v{46}The king had them cast in the clay ground between Succoth and Zarethan in the Jordan plain. \v{47}Solomon never inventoried the weight of the bronze used, because there were too many utensils, so the weight of the bronze used was never ascertained. \v{48}Solomon made all the furnishings that were placed in the \divine{Lord}'s Temple, including the golden altar and the golden table on which the bread of the Presence was placed, \v{49}along with the lamp stands (five on the right side and five on the left in front of the inner sanctuary), all made of pure gold, as well as the flower blossoms, lamps, and tongs of gold, \v{50}and the cups, snuffers, bowls, spoons, and the fire pans, all made of pure gold, and hinges for the doors of the inner sanctuary, the Most Holy Place, and for the gates of the Temple that led to the nave, also of gold.

\v{51}Thus all the work that King Solomon performed in the \divine{Lord}'s Temple was finished. Then Solomon brought in the articles that had been dedicated by his father David, including silver, gold, and other utensils, and he placed them into storage in the treasuries of the \divine{Lord}'s Temple.
\labelchapt{8}
\passage{The Temple is Dedicated}
\passageinfo{(2 Chronicles 5:2-6:2)}

\chapt{8}
\v{1}Then Solomon gathered together the elders of Israel, including all the heads of the tribes and the leaders of the ancestral households of the Israelis, to meet with him in Jerusalem so they could bring up the Ark of the Covenant of the \divine{Lord} from Zion, the City of David. \v{2}So all the men gathered together to meet with King Solomon at the Festival of Tents\fnote{\fbackref{8:2} The Heb. lacks \fbib{of Tents}; cf. Lev 23:34} in the month Ethanim, the seventh month. \v{3}All the Elders of Israel showed up, and the priests picked up the ark \v{4}and brought it, the Tent of Meeting, and all the holy implements that were in the tent. The priests and descendants of Levi carried them up to Jerusalem.\fnote{\fbackref{8:4} The Heb. lacks \fbib{to Jerusalem}}

\v{5}King Solomon and the entire congregation of Israel that had assembled to be with him stood in front of the ark, sacrificing so many sheep and oxen that they were neither counted nor inventoried. \v{6}After this, the priests brought the Ark of the Covenant of the \divine{Lord} to the place prepared for it, into the inner sanctuary of the Temple, under the wings of the cherubim in the Most Holy Place. \v{7}The wings of the cherubim spread over the resting place for the ark, so that the cherubim made a covering over the ark and its poles when viewed\fnote{\fbackref{8:7} The Heb. lacks \fbib{when viewed}} from above. \v{8}The poles extended so far that their ends could be seen from the Holy Place in front of the inner sanctuary, but they could not be seen from outside. They remain there to this day. \v{9}The ark was empty except for the two stone tablets that Moses had placed there at Horeb when the \divine{Lord} had made a covenant with the Israelis after they had come out of the land of Egypt. \v{10}When the priests left the Holy Place after setting the ark in place,\fnote{\fbackref{8:10} The Heb. lacks \fbib{after setting the ark in place}} the cloud filled the \divine{Lord}'s Temple \v{11}so that the priests could not stand to minister because of the cloud, since the glory of the \divine{Lord} filled the \divine{Lord}'s Temple.
\passage{Solomon's Speech of Dedication}
\passageinfo{(2 Chronicles 6:3-11)}

\v{12}Then Solomon said, ``The \divine{Lord} has said that he lives shrouded in darkness. \v{13}Now I have been constructing a magnificent Temple dedicated to you that will serve as a place for you to inhabit forever.''

\v{14}Then the king turned to face the entire congregation of Israel while the congregation of Israel remained standing. \v{15}Then Solomon\fnote{\fbackref{8:15} Lit. \fbib{He}} prayed:

\begin{poetry}
\poeml ``Blessed is the \divine{Lord} God of Israel, who made a commitment\fnote{\fbackref{8:15} Lit. \fbib{who spoke by his mouth}} to my father David and then personally\fnote{\fbackref{8:15} Lit. \fbib{and by his hand}} fulfilled what he had promised when he said:\fnote{\fbackref{8:15} Cf. 1Chr 17:5ff}
\end{poetry}

\begin{poetry}
\poeml \v{16}`From the day I brought out my people Israel from Egypt I never chose a city from all the tribes of Israel to build a temple where my name might reside. I have chosen David to be over my people Israel.'
\end{poetry}

\begin{poetry}
\poeml \v{17}``My father David wanted to build a temple for the name of the \divine{Lord} God of Israel. \v{18}The \divine{Lord} told my father David: \\
\poeml `Therefore, since you determined\fnote{\fbackref{8:18} Lit. \fbib{since it was in your heart}} to build a temple for my name, you acted well, because it was your choice\fnote{\fbackref{8:18} Lit. \fbib{because it was in your heart}} to do so. \v{19}Nevertheless, you are not to build the Temple, but your son who will be born\fnote{\fbackref{8:19} Lit. \fbib{will come from your loins}} to you is to build a temple for my name.' \\
\poeml \v{20}``The \divine{Lord} has brought to fulfillment\fnote{\fbackref{8:20} Lit. \fbib{has caused to stand up}} what he promised, and now here I stand,\fnote{\fbackref{8:20} MT verb is a pun on the verb \fbib{brought to fulfillment}} having succeeded my father David to sit on the throne of Israel, as the \divine{Lord} promised. I have built the Temple for the name of the \divine{Lord} God of Israel. \v{21}I have placed there the ark in which the covenant is stored that the \divine{Lord} made with our ancestors when he brought them out of the land of Egypt.''
\end{poetry}
\passage{Solomon's Prayer of Dedication}
\passageinfo{(2 Chronicles 6:12-43)}

\v{22}Then Solomon took his place in front of the \divine{Lord}'s altar in the presence of the entire congregation of Israel, spread out his hands toward heaven, \v{23}and said:

\begin{poetry}
\poeml ``\divine{Lord} God of Israel, there is no one like you, God in heaven above or on the earth below, who watches over\fnote{\fbackref{8:23} Or \fbib{who keeps}} his covenant, showing gracious love to your servants who live their lives in your presence\fnote{\fbackref{8:23} Lit. \fbib{who walk before you}} with all their hearts. \v{24}It is you, \divine{Lord} God,\fnote{\fbackref{8:24} The Heb. lacks \fbib{It is you, \divine{Lord} God}} who have kept your promise to my father, your servant David, that you made to him. Indeed, you made a commitment\fnote{\fbackref{8:24} Lit. \fbib{you spoke by your mouth}} to my father David and then personally fulfilled\fnote{\fbackref{8:24} Lit. \fbib{and by your hand full}} what you had promised today. \\
\poeml \v{25}``Now therefore, \divine{Lord} God of Israel, keep your promise that you made\fnote{\fbackref{8:25} Lit. \fbib{spoke}} to my father, your servant David, when you said, `You will not lack a man to sit on the throne of Israel,\fnote{\fbackref{8:25} Cf. 1King 2:4; 2Chr 7:18} if only your descendants will watch their lives,\fnote{\fbackref{8:25} Lit. \fbib{ways}} to live\fnote{\fbackref{8:25} Lit. \fbib{walk}} in my presence just as you have lived\fnote{\fbackref{8:25} Lit. \fbib{walked}} in my presence.'\fnote{\fbackref{8:25} Or \fbib{have walked before me}} \\
\poeml \v{26}``Now therefore, God of Israel, may your promise that you made\fnote{\fbackref{8:26} Lit. \fbib{spoke}} to your servant David my father be fulfilled{\ldots} \v{27}and yet, will God truly reside on earth? Look! Neither the sky nor the highest heaven can contain you! How much less this Temple that I have built! \v{28}Pay attention to the prayer of your servant and to his request, \divine{Lord} my God, and listen to the cry and prayer that your servant is praying in your presence today. \v{29}Let your eyes always look toward this Temple night and day, toward the location where you have said `My name will reside there.' Listen to the prayer that your servant prays in this direction.\fnote{\fbackref{8:29} Lit. \fbib{prays toward this place}} \v{30}Listen to the requests from your servant and from your people Israel as they pray in this direction,\fnote{\fbackref{8:30} Lit. \fbib{pray toward this place}} listen from the place where you reside in heaven, then hear and forgive. \\
\poeml \v{31}``If a man should sin against his neighbor and he is required to take an oath, and he then comes to take an oath in front of your altar in this Temple, \v{32}then listen in heaven, act, and judge your servants, condemning the wicked by bringing back to him the consequences of his choices\fnote{\fbackref{8:32} Lit. \fbib{by bringing his way upon his head}} and by justifying the righteous by recompensing him according to his righteousness. \\
\poeml \v{33}``If your people Israel are defeated in a battle with\fnote{\fbackref{8:33} Lit. \fbib{defeated before}} their enemy because they have sinned against you, when they return to you and confess to you,\fnote{\fbackref{8:33} Lit. \fbib{confess your name}} pray, and in this Temple they ask you to show grace to them, \v{34}then hear in heaven, forgive the sin of your people Israel, and return them to the soil\fnote{\fbackref{8:34} Or \fbib{land}} that you gave to their ancestors. \\
\poeml \v{35}``When heaven remains closed, and there is no rain because they have sinned against you, and they pray in the direction of this place, confessing your name and turning from their sin when you afflict them,\fnote{\fbackref{8:35} So MT; LXX reads \fbib{you bring them low}} \v{36}then hear in heaven and forgive the sin of your servants and of your people Israel. Indeed, teach them the best way to live and send rain on your land that you have given to your people as an inheritance. \\
\poeml \v{37}``If a famine comes to the land, or if plant diseases, mildew, locust, or grasshoppers\fnote{\fbackref{8:37} Or \fbib{caterpillars}} appear, or if their enemies attack them in their settlements of the land, no matter what the epidemic or illness is, \v{38}whatever prayer or request is made, no matter whether it's made by a single man or by all of your people Israel, each praying out of his own hurting heart and anguish and stretching out his hands toward this Temple, \v{39}then hear from heaven, the place where you reside, and forgive, repaying each person according to all of his ways, since you know their hearts---for you alone know the hearts of all human beings--- \v{40}so they will fear you every day and live on the surface of the land that you have given to our ancestors. \\
\poeml \v{41}``Now concerning the foreigner who is not from your people Israel, when he comes from a land far away for the sake of your name \v{42}(for people will hear of your great name, your mighty acts,\fnote{\fbackref{8:42} Lit. \fbib{hand}} and your obvious power\fnote{\fbackref{8:42} Lit. \fbib{your outstretched arm}}), when he comes and prays facing this Temple, \v{43}then hear in heaven where you reside, and do whatever the foreigner asks of you, so that all the people of the earth may know your name, fear you as do your people Israel, and so they may know that this Temple that I have built is called by your name. \\
\poeml \v{44}``When your people go out to war against their enemies, no matter what way you send them, and they pray to the \divine{Lord} in the direction of the city that you have chosen and in the direction of the Temple that I have built for your name, \v{45}then hear their prayer and their request in heaven, and fight for their cause. \\
\poeml \v{46}``When they sin against you---because there isn't a single human being who doesn't sin---and you become angry with them and deliver them over to their enemy, who takes them away captive to the land that belongs to their enemy, whether near or far away, \v{47}if they turn their hearts back to you\fnote{\fbackref{8:47} The Heb. lacks \fbib{back to you}} in the land where they have been taken captive, repent, and pray to you---even if they do so in the land of their captivity---confessing, `We have sinned, we have committed abominations, and practiced wickedness,' \v{48}if they return to you with all of their heart and with all of their soul in the land of their enemies who have taken them captive, as they pray to you in the direction of their land that you have given to their ancestors and to the city that you have chosen, and to the Temple that I have built for your name, \v{49}then hear their prayer and requests in heaven, where you reside, and fight for their cause, \v{50}forgiving your people who have sinned against you, along with their transgressions by which they have transgressed against you. \\
\poeml ``Show your compassion in the presence of those who have taken them captive, so they may show compassion on them, \v{51}since they are your people and your heritage, which you brought out of Egypt, from an iron fire furnace. \v{52}Do this\fnote{\fbackref{8:52} The Heb. lacks \fbib{Do this}} so your eyes may remain open to the requests of your servant and to the requests of your people's prayers, to listen to them whenever they call out to you, \v{53}because you have separated them to yourself as your heritage from all the people of the earth, as you spoke through your servant Moses when you brought our ancestors out of Egypt, Lord \divine{God}.
\end{poetry}
\passage{Solomon's Blessing to the Assembly}
\passageinfo{(2 Chronicles 6:40-42)}

\v{54}When Solomon had completed saying this entire prayer to the \divine{Lord}, he got up from kneeling with his hands spread out toward heaven in the presence of the \divine{Lord}'s altar, \v{55}stood up, and blessed all of the assembly of Israel in a loud voice. He said:

\begin{poetry}
\poeml \v{56}``Blessed is the \divine{Lord}, who has given security to his people Israel, just as he promised. Not one of his promises has failed to come about that he gave through his servant Moses. \v{57}May the \divine{Lord} our God be with us, just as he was with our ancestors. May he never leave us or abandon us, \v{58}so that he may turn our hearts toward him, so that we may live life\fnote{\fbackref{8:58} Lit. \fbib{may walk in}} his way, keeping his commands, statutes, and ordinances that he gave to our ancestors. \v{59}And may what I've had to say to the \divine{Lord} remain with the \divine{Lord} our God both day and night, so that he may defend the cause of his servant and the cause of his people Israel, as the need of the day may require it, \v{60}so that, in turn,\fnote{\fbackref{8:60} Lit. \fbib{therefore}} all the people of the earth may know that the \divine{Lord} is God---there is no one else. \v{61}Now let your heart be completely devoted to the \divine{Lord} our God, to live according to his statutes and to keep his commands, as we are doing today.''
\end{poetry}
\passage{Solomon's Initial Offerings}
\passageinfo{(2 Chronicles 7:4-11)}

\v{62}Then the king and all of Israel with him offered sacrifices to the \divine{Lord}. \v{63}Solomon offered peace offerings to the \divine{Lord} consisting of 22,000 oxen and 120,000 sheep. So the king and all the Israelis dedicated the \divine{Lord}'s Temple. \v{64}That same day, the king consecrated the middle court that stood in front of the \divine{Lord}'s Temple, because that was where he offered burnt offerings, grain offerings, and fat from the peace offerings and because the bronze altar that was in the \divine{Lord}'s presence was too small to hold the burnt offerings, grain offerings, and fat from the peace offerings. \v{65}So Solomon observed the Festival of Tents\fnote{\fbackref{8:65} The Heb. lacks \fbib{of Tents}; cf. Lev 23:34} at that time, as did all of Israel with him. A large assembly came up from as far away as Lebo-hamath and the Wadi\fnote{\fbackref{8:65} I.e. a seasonal stream or river that channels water during rain seasons but is dry at other times} of Egypt to appear in the presence of the \divine{Lord} our God, not just for seven days, but for seven days after that, a total of fourteen days. \v{66}The following\fnote{\fbackref{8:66} Lit. \fbib{eighth}} day, Solomon\fnote{\fbackref{8:66} Lit. \fbib{he}} sent the people away as they blessed the king. Then they went back to their tents, rejoicing and glad for all the good things that the \divine{Lord} had done for his servant David and to his people Israel.
\labelchapt{9}
\passage{God Appears to Solomon}
\passageinfo{(2 Chronicles 7:11-22)}

\chapt{9}
\v{1}Later, after Solomon had finished building the \divine{Lord}'s Temple, the royal palace, and everything else that Solomon wanted to do, \v{2}the \divine{Lord} appeared to Solomon for a second time, just as he had appeared to him at Gibeon. \v{3}The \divine{Lord} told him:

\begin{poetry}
\poeml ``I've heard your prayer and your request that you made to me. I have consecrated this Temple that you have built by placing my name there forever. My eyes and my heart will be there continuously. \\
\poeml \v{4}``Now as for you, if you commune with me like your father did, with an upright heart of integrity and doing everything that I've commanded you and keeping my statutes and ordinances, \v{5}then I'll make your royal throne secure forever, just as I agreed to do so for your father David when I said, `You are to not lack a man on the throne of Israel.' \v{6}But if you or your descendants abandon me, and do not keep my commandments and statutes that I have given to you, and if you go away, serve other gods, and worship them, \v{7}then I will eliminate Israel from the land that I gave them and from the Temple that I've consecrated for my name. I will throw them out of my sight, and Israel will become the butt of jokes\fnote{\fbackref{9:7} Lit. \fbib{become an object of mockery}} and a means of ridicule among people worldwide! \\
\poeml \v{8}``This Temple will become a pile of ruins. Everyone who passes by it will be so astounded that they will ask, `Why did the \divine{Lord} do this to this land and to this Temple?' \v{9}They will answer, `Because they abandoned the \divine{Lord} their God, who brought their ancestors out of the land of Egypt, and they adopted other gods and served them. That's why the Lord has brought all of this disaster on them.'\,''
\end{poetry}
\passage{Solomon Cedes Cities to Hiram}

\v{10}It took 20 years for Solomon to finish working on the two houses---the \divine{Lord}'s Temple and the royal palace--- \v{11}after which King Solomon gave Hiram 20 cities in the land of Galilee, because King Hiram of Tyre had provided Solomon with as much cedar, cypress timber, and gold that he wanted. \v{12}Hiram came out from Tyre to see the cities that Solomon had given him, but he wasn't happy with them, \v{13}so he asked him, ``What are these cities that you have given to me, my brother?'' That's why these cities were named ``the land of Cabal''\fnote{\fbackref{9:13} The Heb. name \fbib{Cabul} means \fbib{as good as nothing}} to this day. \v{14}Then Hiram paid the king 120 talents\fnote{\fbackref{9:14} I.e. about 9,000 pounds; a talent weighed about 75 pounds} of gold.
\passage{Solomon's Other Accomplishments}
\passageinfo{(2 Chronicles 8:3-16)}

\v{15}Here is a summary of the conscripted labor that King Solomon required to build the \divine{Lord}'s Temple, his royal palace, the terrace ramparts in the City of David,\fnote{\fbackref{9:15} Lit. \fbib{the Millo}, fortified areas of ancient Jerusalem with terraces and retaining walls} the wall of Jerusalem, Hazor, Megiddo, and Gezer. \v{16}Pharaoh, the king of Egypt, had attacked and captured Gezer, burned it down, killed the Canaanites who lived in the city, and then gave it as a dowry for his daughter, Solomon's wife. \v{17}So Solomon rebuilt Gezer, lower Beth-horon, \v{18}Baalath, and Tamar in the wilderness, \v{19}along with the storage cities that Solomon used for his chariots and for his cavalry, everything that Solomon felt like building in Jerusalem, in Lebanon, and in every territory under his control.

\v{20}The people who survived from the Amorites, the Hittites, the Perizzites, the Hivites, and the Jebusites, who were not related to the Israelis, \v{21}and whose descendants had survived them and continued to live in the land because the Israelis were unable to completely eliminate them, Solomon placed under conscripted labor, a situation that remains in effect to this day. \v{22}However, Solomon did not force Israelis into conscripted labor, but they did serve as his soldiers, servants, princes, captains, chariot commanders, and cavalry. \v{23}There were 550 chief officers who supervised Solomon's activities and managed the staff that was doing the work.

\v{24}As soon as Pharaoh's daughter arrived from the City of David to live in her house that Solomon\fnote{\fbackref{9:24} Lit. \fbib{he}} had built for her, then he fortified the terrace ramparts in the City of David.\fnote{\fbackref{9:24} Lit. \fbib{the Millo}, fortified areas of ancient Jerusalem with terraces and retaining walls} \v{25}Three times every year Solomon offered burnt offerings and peace offerings on the altar that he had built to the \divine{Lord}, burning incense with the offerings in the presence of the Lord.

This concludes the record of the Temple construction.
\passage{Solomon's Business Ventures}
\passageinfo{(2 Chronicles 8:17-18)}

\v{26}King Solomon also built a fleet of ships at Ezion-geber, which is near Eloth on the shore of the Reed\fnote{\fbackref{9:26} So MT; LXX reads \fbib{Red}} Sea in the land of Edom. \v{27}Hiram sent his servants to sail with the fleet, since they were expert seamen, and so they accompanied Solomon's servants. \v{28}They sailed as far as Ophir\fnote{\fbackref{9:28} Or \fbib{as a source of fine gold}; cf. 1Chr 29:4} and brought back 420 talents\fnote{\fbackref{9:28} I.e. about 31,500 pounds; a talent weighed about 75 pounds} of gold for Solomon.
\labelchapt{10}
\passage{The Queen of Sheba Visits Solomon}
\passageinfo{(2 Chronicles 9:1-28)}

\chapt{10}
\v{1}When the queen of Sheba heard about Solomon's reputation with the \divine{Lord}, she came to test him\fnote{\fbackref{10:1} Lit. \fbib{Solomon}} with difficult questions. \v{2}She brought along a large retinue, camels laden with spices, and lots of gold and precious stones. Upon her arrival, she spoke with Solomon about everything that was on her mind.\fnote{\fbackref{10:2} Lit. \fbib{was with her heart}} \v{3}Solomon answered all of her questions. Nothing was hidden from Solomon that he did not explain to her. \v{4}When the queen of Sheba had seen all of Solomon's wisdom for herself, the palace that he had built, \v{5}the food set at his table, his servants who sat with him, his ministers in attendance and how they were dressed, his personal staff\fnote{\fbackref{10:5} Lit. \fbib{his cupbearers}} and how they were dressed, and even his personal stairway by which he went up to the \divine{Lord}'s Temple, she was breathless!

\v{6}``Everything I heard about your wisdom and what you have to say is true!'' she gasped, \v{7}``but I didn't believe it at first! But then I came here and I've seen it for myself! It's amazing! I wasn't told half of what's really great about your wisdom. You're far better in person than what the reports have said about you! \v{8}How blessed are your staff! And how blessed are your employees,\fnote{\fbackref{10:8} Lit. \fbib{servants}} who serve you continuously and get to listen to your wisdom! \v{9}And blessed be the \divine{Lord} your God, who is delighted with you! He set you in place on the throne of Israel because the \divine{Lord} loved Israel forever. That's why he made you to be king, so you could carry out justice and implement righteousness.''

\v{10}Then she gave the king 120 talents\fnote{\fbackref{10:10} I.e. about 9,000 pounds; a talent weighed about 75 pounds} of gold, a vast quantity of spices, and precious stones. No spices ever came again that were comparable to those that the queen of Sheba gave to King Solomon. \v{11}Hiram's ships that brought gold from Ophir,\fnote{\fbackref{10:11} Or \fbib{from a source of fine gold}; cf. 1Chr 29:4} also brought from Ophir\fnote{\fbackref{10:11} Or \fbib{from a source of fine gold}; cf. 1Chr 29:4} lots of algum wood\fnote{\fbackref{10:11} Or \fbib{presented Juniper trees}} and precious stones. \v{12}The king used the algum wood\fnote{\fbackref{10:12} Or \fbib{the Juniper trees}} to have supports made for the \divine{Lord}'s Temple and for the royal palace, as well as lyres and harps for the choir,\fnote{\fbackref{10:12} Lit. \fbib{singers}} and nothing like that wood\fnote{\fbackref{10:12} The Heb. lacks \fbib{wood}} has ever come again or even been seen since right to this day. \v{13}In return, King Solomon gave the queen of Sheba everything she wanted and had requested in addition to what he had given her consistent with his generosity. Afterward, she returned to her own land with her servants.
\passage{Solomon's Wealth}
\passageinfo{(2 Chronicles 1:14-17; 10:13-28)}

\v{14}Solomon's annual revenue was 666 talents\fnote{\fbackref{10:14} I.e. about 49,950 pounds; a talent weighed about 75 pounds} of gold, \v{15}not including revenue from traders, merchants, and from all the kings of Arabia and the governors of the land. \v{16}King Solomon made 200 large shields of beaten gold, overlaying each large shield with the gold from 600 gold pieces,\fnote{\fbackref{10:16} MT does not identify the individual unit of measure} \v{17}and 300 shields from beaten gold, overlaying each shield with the gold from 300 gold pieces.\fnote{\fbackref{10:17} MT does not identify the individual unit of measure} The king put them in his palace in the Lebanon forest. \v{18}The king also made a great ivory throne and overlaid it with pure gold. \v{19}Six steps led up to the throne, which had a round canopy fastened to the rear of the throne and armrests on each side of the seat and two lions standing on either side of each armrest. \v{20}Twelve lions were placed on both sides of the six steps leading to the throne,\fnote{\fbackref{10:20} The Heb. lacks \fbib{leading to the throne}} and nothing comparable was made for any other\fnote{\fbackref{10:20} The Heb. lacks \fbib{other}} kingdoms. \v{21}All of King Solomon's drinking vessels were made of\fnote{\fbackref{10:21} The Heb. lacks \fbib{made of}} gold, and all the vessels in his palace in the Lebanon forest were made of\fnote{\fbackref{10:21} The Heb. lacks \fbib{made of}} pure gold. None were of silver, because silver was never considered to be valuable during Solomon's lifetime, \v{22}because the king had ships that sailed to Tarshish accompanied by Hiram's ships. Once every three years ships from Tarshish returned, bringing gold, silver, ivory, apes, and peacocks. \v{23}As a result, King Solomon became greater than all the kings of the earth in regards to wealth and wisdom. \v{24}All the earth continued to seek audiences with Solomon so they could hear the wise things that God had put in his heart. \v{25}Everyone kept on bringing gifts on an annual basis, including items made of silver and gold, garments, myrrh, spices, horses, and mules. \v{26}Solomon accumulated chariots and cavalry. He had 1,400 chariots and 12,000 cavalry soldiers. He stationed them in various chariot cities and with the king in Jerusalem. \v{27}The king made silver as common as\fnote{\fbackref{10:27} The Heb. lacks \fbib{as common as}} stones in Jerusalem, and made cedar trees as abundant as sycamore\fnote{\fbackref{10:27} The sycamore fruit tree native to Israel bears figs} trees in the Shephelah.\fnote{\fbackref{10:27} I.e. the verdant central lowlands of Israel; cf. Josh 10:40} \v{28}Solomon imported horses from Egypt and Kue, and the king's buyers procured them at market price from Kue. \v{29}A chariot from Egypt cost 600 pieces\fnote{\fbackref{10:29} The denomination of silver coin is not specified.} of silver, and a horse 150 pieces of silver,\fnote{\fbackref{10:29} The Heb. lacks \fbib{pieces of silver}} but then they were exported to all the Hittite kings and to the Aramean kings.
\labelchapt{11}
\passage{Solomon's Forbidden Marriages and Idolatry}
\passageinfo{(2 Chronicles 9:1-28)}

\chapt{11}
\v{1}But King Solomon married\fnote{\fbackref{11:1} Lit. \fbib{loved}} many foreign women besides the daughter of Pharaoh: women from Moab, Ammon, Edom, and Sidonia, along with Hittite women, too, \v{2}all of them from nations that the \divine{Lord} had ordered the Israelis, ``You are not to associate with\fnote{\fbackref{11:2} Lit. \fbib{to go in to}} them and they are not to associate with you, because they will most certainly turn your affections\fnote{\fbackref{11:2} Lit. \fbib{hearts}} away to follow their gods.'' Solomon became deeply attached to them by falling in love. \v{3}He had 700 princess wives and 300 mistresses\fnote{\fbackref{11:3} Or \fbib{concubines}; i.e. secondary wives} who\fnote{\fbackref{11:3} Lit. \fbib{mistresses, and his wives}} turned his heart away from the \divine{Lord},\fnote{\fbackref{11:3} The Heb. lacks \fbib{from the \divine{Lord}}} \v{4}because as Solomon grew older, his wives turned his affections away after other gods, and his heart was not fully as devoted to the \divine{Lord} his God as his father David's heart had been. \v{5}Solomon pursued Astarte, the Sidonian goddess, and Milcom, that detestable Ammonite idol. \v{6}Solomon practiced what the \divine{Lord} considered to be evil by not fully following the \divine{Lord}, as had his father David. \v{7}Later, Solomon even constructed a high place on the mountain east of Jerusalem that was dedicated to Chemosh, that detestable Moabite idol, and to Molech, the detestable Ammonite idol. \v{8}Solomon\fnote{\fbackref{11:8} Lit. \fbib{He}} did this for all of his foreign wives, who burned incense and sacrificed to their own gods.

\v{9}The \divine{Lord} became angry at Solomon because his heart wandered away from the \divine{Lord} God of Israel, who had appeared to him twice\fnote{\fbackref{11:9} Cf. 1King 3:5, 9:2} \v{10}and warned him about this so he would not pursue other gods. But he did not obey what the \divine{Lord} had commanded, \v{11}so the \divine{Lord} told Solomon, ``Because you have done this and haven't kept my covenant and statutes that I commanded you, I'm going to tear the kingdom from you and give it to your servant. \v{12}I'm not going to do this during your lifetime, for the sake of your father David, but I will tear it out of your son's control.\fnote{\fbackref{11:12} Lit. \fbib{hand}} \v{13}For the sake of my servant David and for the sake of Jerusalem, I won't tear away the entire kingdom. I'll leave one tribe for your son to govern.''\fnote{\fbackref{11:13} The Heb. lacks \fbib{to govern}}
\passage{Solomon's Enemies}

\v{14}After this, the \divine{Lord} allowed\fnote{\fbackref{11:14} Lit. \fbib{raised up}} Hadad the Edomite to oppose Solomon. He was part of the royal line of Edom. \v{15}During David's military campaign against Edom, when his army commander Joab had gone out to bury the dead, he killed every male in Edom. \v{16}Joab had his entire army of Israel stay there for six months until he had eliminated every male in Edom.

\v{17}But Hadad escaped to Egypt in the company of some of his father's Edomite servants, while Hadad was still a little child. \v{18}They left Midian, arrived in Paran, and left from Paran with some men and traveled on to Egypt, where Pharaoh, king of Egypt, gave him a house to live in, assigned a food allotment to him, and gave him some land. \v{19}Hadad won the affection of the Pharaoh, who gave permission for Hadad to marry the sister of his own wife, Queen Tahpenes. \v{20}Queen Tahpenes' sister bore him his son Genubath, whom Tahpenes weaned in Pharaoh's palace while Genubath lived in Pharaoh's palace with the Pharaoh's own sons.

\v{21}Later on, Hadad learned in Egypt that David had been buried\fnote{\fbackref{11:21} Lit. \fbib{had slept}} with his ancestors and that Joab the army commander was dead. So Hadad asked Pharaoh, ``Please send me out so I can go back to my own land.''

\v{22}Pharaoh asked him, ``But have you lacked anything from me that would make you want to go back to your own country?''

``No,'' he answered, ``but I still really must leave.''

\v{23}God also raised up Eliada's son Rezon, who had escaped from his master King Hadadezer of Zobah. \v{24}He raised an army and commanded a gang of raiders after David had eliminated those who lived in Zobah. Rezon and his army\fnote{\fbackref{11:24} Lit. \fbib{They}} moved to Damascus, remained there, and Rezon ruled from Damascus. \v{25}He opposed Israel during Solomon's entire reign, in addition to all of the evil things that Hadad did. Rezon\fnote{\fbackref{11:25} Lit. \fbib{He}} also hated Israel while he reigned over Aram.
\passage{Jeroboam Rebels against Solomon}

\v{26}Solomon had a servant, Nebat's son Jeroboam, who was an Ephraimite from Zeredah. His widowed mother was named Zeruah. Jeroboam rebelled against Solomon, \v{27}and this is why he rose in rebellion against the king: Solomon had built up the terrace ramparts\fnote{\fbackref{11:27} Lit. \fbib{the Millo}, fortified areas of ancient Jerusalem with terraces and retaining walls} in the city of his father David in order to repair a weakness. \v{28}Jeroboam was a valiant soldier, and because Solomon observed that the young man was able to get things done, he set him in charge over all of the conscripted labor from the household of Joseph. \v{29}During that time, Jeroboam left Jerusalem and the prophet Ahijah the Shilonite met him on the road. Ahijah had wrapped himself up in a new cloak, and both of them were alone on the open road. \v{30}Ahijah grabbed the new cloak that he was wearing and tore it into twelve pieces! \v{31}Then he told Jeroboam, ``Take ten pieces for yourself, because this is what the \divine{Lord} God of Israel says:

\begin{poetry}
\poeml `Pay attention! I'm going to tear the kingdom out of Solomon's control\fnote{\fbackref{11:31} Lit. \fbib{hand}} and give you ten tribes. \v{32}I'll leave him one tribe for the sake of my servant David and one tribe\fnote{\fbackref{11:32} The Heb. lacks \fbib{one tribe}} for the sake of Jerusalem, the city that I chose from all of the tribes of Israel. \v{33}I'm doing this\fnote{\fbackref{11:33} The Heb. lacks \fbib{I'm doing this}} because they have abandoned me and worshipped that Sidonian goddess Astarte, the Moabite god Chemosh, and the Ammonite god Milcom. They haven't lived my way by doing what I consider to be right and observing my statutes and my ordinances, like his father David did. \\
\poeml \v{34}`Nevertheless, I won't take the entire kingdom away from him, but I'll let him reign for the rest of his life, because of my servant David, whom I chose, who obeyed my commandments and statutes, \v{35}but I will take the kingdom away from his son's control\fnote{\fbackref{11:35} Lit. \fbib{hand}} and give ten tribes to you. \v{36}I'll give one tribe to his son, so my servant David will always have a light shining in my presence in Jerusalem, the city that I chose for myself and where I have placed my name. \v{37}I'm going to take you and have you reign over whatever you desire. You will be king over Israel. \v{38}If you listen to everything that I command you to do, and if you live your life my way,\fnote{\fbackref{11:38} Lit. \fbib{you walk in my ways}} and if you do what I consider to be right by observing my statutes and my commandments, just like my servant David did, then I will be with you, I will build an enduring dynasty for you,\fnote{\fbackref{11:38} Lit. \fbib{enduring house}} just like I did for David, and I'll give Israel to you. \v{39}This is how I'm going to afflict David's descendants because of what they have done, though I won't do it continuously.'\,''
\end{poetry}

\v{40}That's why Solomon tried to execute Jeroboam, but Jeroboam got up and fled to Egypt, where he lived as a guest of King Shishak and remained until Solomon had died.
\passage{The Death of Solomon}
\passageinfo{(2 Chronicles 9:29-31)}

\v{41}Now the rest of Solomon's accomplishments, including everything else he did, as well as records of\fnote{\fbackref{11:41} The Heb. lacks \fbib{records of}} his wisdom, are recorded in the Book of the Acts of Solomon, are they not? \v{42}Solomon reigned over all of Israel from Jerusalem for a total of 40 years. \v{43}Then Solomon died, as had\fnote{\fbackref{11:43} Lit. \fbib{Solomon slept with}} his ancestors, and he was buried in the city of his father David. His son Rehoboam reigned in his place.
\labelchapt{12}
\passage{Secession of the Northern Tribes}
\passageinfo{(2 Chronicles 10:1-19)}

\chapt{12}
\v{1}Rehoboam traveled to Shechem because all of Israel went there to install him as king. \v{2}Nebat's son Jeroboam heard about it while he was still in Egypt, where he had fled to get away from King Solomon. Jeroboam returned from Egypt \v{3}after being summoned. When Jeroboam and the entire assembly of Israel arrived, they spoke to Rehoboam, \v{4}``Your father made our burdens unbearable.\fnote{\fbackref{12:4} Lit. \fbib{our yoke heavy}} Therefore lighten your father's requirements and his heavy burdens that he placed on us, and we'll serve you.''

\v{5}``Come again in three days,'' Rehoboam\fnote{\fbackref{12:5} Lit. \fbib{He}} told them. So the people left \v{6}while King Rehoboam conferred with his advisors who had worked for his father Solomon during his administration. He asked them, ``What is your advice as to how I should respond to these people?''

\v{7}They advised him, ``If today you are a servant, you will serve this people by answering them and speaking kindly to them. Then they will serve you forever.''

\v{8}But Rehoboam\fnote{\fbackref{12:8} Lit. \fbib{he}} ignored the counsel that his elder advisors had given him. Instead, he consulted the younger men who had grown up with him and who worked for\fnote{\fbackref{12:8} Lit. \fbib{who stood before}} him. \v{9}As a result, he asked them, ``What's your advice so that we can give an answer to these people who have asked me, `Please lighten the burden that your father put on us.'?''

\v{10}``This is what you should tell these people who asked you `Your father made our burden heavy, but you must make it lighter for us!'\,'' the young men who grew up with Rehoboam\fnote{\fbackref{12:10} Lit. \fbib{him}} replied. ``Tell them, `My little finger will be thicker than my father's whole body!\fnote{\fbackref{12:10} Lit. \fbib{father's loin}} \v{11}Not only that, but since my father loaded you down heavily, I'm going to add to that burden. My father disciplined you with whips, but I'm going to discipline you with scorpions!'\,''

\v{12}So Jeroboam and all the people went back to Rehoboam on the third day, just as they had been directed when the king said, ``Come back again in three days.'' \v{13}But the king gave the people a harsh response, because he was ignoring the counsel that his elders had given him. \v{14}Instead, Rehoboam\fnote{\fbackref{12:14} Lit. \fbib{he}} spoke to them along the lines of what the younger men suggested. He told them ``My father burdened you heavily, but I will add to that burden. If my father disciplined you with whips, I'm going to discipline you with scorpions!''

\v{15}The king would not listen to the people, because the turn of events was from the \divine{Lord}, to fulfill his prediction that the \divine{Lord} spoke by means of Ahijah the Shilonite to Nebat's son Jeroboam \v{16}When all of Israel saw that the king wasn't listening to them, the people responded to the king's message, ``What's the point in following David? We have no inheritance in the descendants of Jesse. Let's go home,\fnote{\fbackref{12:16} Lit. \fbib{Each man to his tent}} Israel! David, take care of your own household!' So Israel left for home.\fnote{\fbackref{12:16} Lit. \fbib{left for their tents}} \v{17}And so Rehoboam ruled over the Israelis who lived in the cities of Judah.

\v{18}King Rehoboam sent Hadoram, who was in charge of conscripted labor, but all of Israel stoned him to death, and King Rehoboam had to jump in his chariot and flee back in a hurry to Jerusalem. \v{19}That's how Israel came to be in rebellion against David's dynasty to this day.
\passage{Jeroboam Reigns over Israel}
\passageinfo{(2 Chronicles 11:1-4)}

\v{20}Now when all of Israel heard that Jeroboam had returned, they sent for him and invited him to visit their assembly, where they installed him as king over all of Israel. Nobody (with the sole exception of the tribe of Judah) would align with David's dynasty. \v{21}As soon as Rehoboam returned to Jerusalem, he assembled 180,000 elite soldiers from the tribes of Judah and Benjamin, intending to attack the dynasty of Israel and restore the kingdom to Solomon's son Rehoboam. \v{22}But a message from God came to Shemaiah, a man of God: \v{23}``Tell Solomon's son Rehoboam, king of Judah, all the dynasty of Judah, Benjamin, and the rest of the people, \v{24}`This is what the \divine{Lord} says: ``You are not to fight or even approach your fellow Israelis in battle. Every soldier is to return to his own home, because this development comes from me.''\,'\,'' So they listened to what the \divine{Lord} had to say and returned home,\fnote{\fbackref{12:24} The Heb. lacks \fbib{home}} just as the \divine{Lord} had directed.
\passage{Jeroboam's Idolatry}

\v{25}Later on, Jeroboam fortified Shechem in the hill country of Ephraim and lived there. He also expanded from there and built Penuel. \v{26}Jeroboam was thinking to himself, ``The kingdom is about to return to David's control.\fnote{\fbackref{12:26} Lit. \fbib{house}} \v{27}If these people keep going up to Jerusalem to offer sacrifices to the \divine{Lord} there, the hearts of these people will return to their lord, King Rehoboam of Judah. Then they'll kill me and return to Rehoboam, king of Judah!'' \v{28}So the king sought some advice and then built two golden calves and announced, ``It's too difficult for you to travel to Jerusalem. So here are your gods, Israel, who brought you up from the land of Egypt!'' \v{29}He set one of them in Bethel and placed the other one in Dan. \v{30}Doing this was sinful, because the people traveled as far as Dan to appear before one of their idols.\fnote{\fbackref{12:30} The Heb. lacks \fbib{of their idols}} \v{31}Jeroboam\fnote{\fbackref{12:31} Lit. \fbib{He}} built temples on the high places, and appointed his own priests from the fringe elements of the people who were not descendants of Levi.

\v{32}Jeroboam invented a festival for the fifteenth day of the eighth month similar to the festival that takes place in Judah. He approached the altar that he had set up in Bethel and sacrificed to the calves that he had made, having stationed in Bethel the priests that he had appointed. \v{33}Then, on the fifteenth day of the eighth month, he went up to burn incense on the altar that he had set up in Bethel, thus beginning the festival that he had made up out of his own heart for the Israelis.
\labelchapt{13}
\passage{Josiah's Desecration Predicted by a Man of God}

\chapt{13}
\v{1}Right when Jeroboam was standing by the altar to burn some incense, a man of God arrived in Bethel from Judah in obedience to a command from the \divine{Lord}. \v{2}He cursed\fnote{\fbackref{13:2} Or \fbib{rebuked}} the altar in this\fnote{\fbackref{13:2} Lit. \fbib{a}} message from the \divine{Lord}: ``Hey altar! Hey altar! This is what the \divine{Lord} says: `Pay attention to this! A son is going to be born in David's dynasty. His name will be Josiah. He will sacrifice the priests who burn incense on you in these high places. Human bones will be burned on you!'\,''\fnote{\fbackref{13:2} Cf. 2King 23:15-16}

\v{3}Later that same day, he gave them a special display of power\fnote{\fbackref{13:3} Or \fbib{a sign}} of what was to come when he said, ``Here's proof\fnote{\fbackref{13:3} Or \fbib{Here's a sign}} that the \divine{Lord} has decreed this:\fnote{\fbackref{13:3} Lit. \fbib{spoken}} Look! This altar will be split apart and the ashes that are on it will spill out.''

\v{4}When he heard the man of God curse\fnote{\fbackref{13:4} Or \fbib{rebuke}} the altar in Bethel, the king pointed at the man of God from where the king was standing at the altar. ``Seize him!'' he ordered. But all of a sudden his hand that he had stretched out dried up, and he could not bring it back to his side! \v{5}Also, the altar broke apart and the ashes that were on it spilled out from the altar, providing just the proof that the man of God had predicted in his message from the \divine{Lord}!

\v{6}``Please!'' the king begged the man of God, ``Ask the \divine{Lord} your God and pray for me that my hand may be restored for me!'' So the man of God asked the \divine{Lord}, and the king's hand was immediately and fully restored, just like it had been before. \v{7}So the king told the man of God, ``Come back to my palace and rest a while. I'd like to give you a reward.''

\v{8}But the man of God replied to the king, ``Even if you were to offer me half of your house, I wouldn't go with you, and I'm sure not going to eat even a piece of bread or drink water in this place, \v{9}because the \divine{Lord} commanded me specifically, `You are not to eat bread, drink water, or return by the way that you came to arrive here!'\,'' \v{10}Then he left, returning a different way than the one by which he had traveled to Bethel.
\passage{An Old Prophet Rebukes the Man of God}

\v{11}Now there was an old prophet who lived in Bethel, and his sons went to him and told him everything that the man of God had accomplished that day in Bethel, including the message that he had delivered to the king. \v{12}``Which way did he go?'' their father asked him, since his sons had observed the way that the man of God had taken to return to Judah from Bethel. \v{13}``Saddle my donkey for me!'' he ordered.\fnote{\fbackref{13:13} The Heb. lacks \fbib{he ordered}} So they saddled the donkey for him \v{14}and he rode off after the man of God and found him sitting under an oak tree.\fnote{\fbackref{13:14} or \fbib{under a terebinth tree}; i.e. an oak tree used in idol worship} ``You're the man of God who came from Judah, aren't you?'' the old prophet\fnote{\fbackref{13:14} Lit. \fbib{He}} asked him.

``I am,'' he replied.

\v{15}``Come home with me and have a meal,'' he told him.

\v{16}But he replied, ``I can't go back with you to your home, be in your company, or even eat food or drink water with you in this place, \v{17}because I've been given a command in the form of this message from the \divine{Lord}: `You are to eat no food, drink no water, and do not return to Judah\fnote{\fbackref{13:17} The Heb. lacks \fbib{to Judah}} by traveling the way by which you go there.'\,''

\v{18}``I'm a prophet like you,'' the old man replied, ``and an angel spoke to me and delivered this message from the \divine{Lord}: `Bring him back with you to your house and give him food and water.'\,'' But he was lying, \v{19}and the man of God\fnote{\fbackref{13:19} Lit. \fbib{So he}} accompanied the old prophet\fnote{\fbackref{13:19} Lit. \fbib{accompanied him}} back to his house, ate some food, and drank some water.

\v{20}Later, while they were sitting down at the table, a message from the \divine{Lord} was delivered to the prophet who had brought him back, \v{21}so he cried out to the man of God from Judah: ``This is what the \divine{Lord} says: `Because you disobeyed a command from the \divine{Lord} and haven't done what the \divine{Lord} your God commanded you to do, \v{22}but instead you returned to eat and drink in the very place that he told you ``Eat no food and drink no water,'' your body will not be buried in the same grave as your ancestors.'\,''
\passage{A Lion Kills the Man of God}

\v{23}After the meal was over, and the man had eaten food and had drunk water, the old prophet saddled the donkey for him---that is, for the man of God whom he had brought back. \v{24}Not long after the man of God\fnote{\fbackref{13:24} Lit. \fbib{after he}} had left, a lion met him along the road and killed him. His body was left lying in the middle of the road with the donkey standing beside it and with the lion also standing next to the body. \v{25}When some men passed by and noticed the body lying in the middle of the road and the lion standing beside the body, they went straight to the city and told what had happened in the city where the old prophet lived.

\v{26}The prophet who had brought the man of God\fnote{\fbackref{13:26} Lit. \fbib{brought him}} back from the road learned about it. ``It's the man of God who disobeyed the message from the \divine{Lord},'' he said. ``That's why the \divine{Lord} gave him to that lion, which mauled him and killed him, just as the message from the \divine{Lord} told rebuke him.'' \v{27}Then he ordered his sons, ``Saddle the donkey for me.'' So they did. \v{28}The old prophet\fnote{\fbackref{13:28} Lit. \fbib{He}} went out, located the body on the road where the donkey and the lion were standing beside the body. The lion had not eaten the body nor mauled the donkey. \v{29}The prophet picked up the body of the man of God, laid it on the donkey, and brought it back to the city where the old man lived so he could mourn and bury him.

\v{30}He buried the corpse in his own grave and his family mourned for him, crying out, ``Oh, no! My brother!''

\v{31}After he had buried the man of God,\fnote{\fbackref{13:31} Lit. \fbib{buried him}} he gave these instructions to his children: ``When I die, bury me in the same grave in which the man of God is buried. Place my bones beside his, \v{32}because what he predicted by a message from the \divine{Lord} against the altar in Bethel and the temples built in the high places of the cities of Samaria will certainly come about.''

\v{33}Despite everything that happened, Jeroboam never did repent of his evil practices. Instead, he appointed even more people to act as priests for the high places. Anyone who wanted to be a priest was ordained to be a priest in the high places. \v{34}This practice became so sinful that the \divine{Lord} decided\fnote{\fbackref{13:34} The Heb. lacks \fbib{that the \divine{Lord} decided}} to erase Jeroboam's dynasty, thus eliminating it from the face of the earth.
\labelchapt{14}
\passage{God Disciplines Jeroboam's Family}

\chapt{14}
\v{1}Right at that time, Jeroboam's son Abijah became ill, \v{2}so Jeroboam suggested to his wife, ``Get up, disguise yourself so that no one will know that you're Jeroboam's wife, and go to Shiloh where the prophet Ahijah lives. He's the one who told me that I would be king over this people. \v{3}Take ten loaves with you, some\fnote{\fbackref{14:3} Lit. \fbib{loaves in your hand}} cakes, and a jar of honey and go visit him. He will tell you what will happen to the boy.''

\v{4}So that's what Jeroboam's wife did. She got up, went to Shiloh, and found Ahijah's home. Ahijah was blind, because his eyes could not focus\fnote{\fbackref{14:4} Lit. \fbib{eyes were set}} due to his age. \v{5}Meanwhile, the \divine{Lord} had spoken to Ahijah, ``Be on your guard! Jeroboam's wife is coming to ask you about her son, because he is ill. You're to say such and such to her. When she arrives, she will pretend to be someone else!''

\v{6}When she arrived, Ahijah heard the sound of her feet as she came through the doorway. He said this to her:

\begin{poetry}
\poeml ``Come in, wife of Jeroboam. What is this pretension at being someone else? I have some harsh news.\fnote{\fbackref{14:6} The Heb. lacks \fbib{news}} \v{7}Go tell Jeroboam: \\
\poeml `I raised you up from among the people. \\
\poeml `I made you Commander-in-Chief\fnote{\fbackref{14:7} Lit. \fbib{Nagid}; i.e. a senior officer entrusted with dual roles of operational oversight and administrative authority} over my people Israel. \\
\poeml \v{8}`I tore the kingdom away from David's dynasty. \\
\poeml `Then I gave it to you. \\
\poeml But you have not lived like my servant David, who kept my commands with all his heart, and did only what I considered to be right. \\
\poeml \v{9}`Instead, you have done more evil than everyone who lived before you. \\
\poeml `You have gone out and crafted other gods for yourself. \\
\poeml `You made cast images. \\
\poeml `You have provoked me to anger. \\
\poeml `You have thrown me behind your back. \\
\poeml \v{10}`Therefore, watch while I bring calamity on Jeroboam's dynasty! \\
\poeml `I will eliminate every male,\fnote{\fbackref{14:10} Lit. \fbib{everyone who urinates against a wall}} both slave and free in Israel, from Jeroboam. \\
\poeml `I will burn up Jeroboam's dynasty, as a man burns up manure until it is gone. \v{11}Dogs will eat anyone who dies in the city that belongs to Jeroboam's household. The birds of the sky will eat anyone who dies in the open field, because the \divine{Lord} has determined it.' \\
\poeml \v{12}``Now get up and go home. When your feet cross the city line, your child will die. \v{13}Everyone in Israel will mourn for him and will bury him, because he alone from Jeroboam's family will receive a decent burial, because something good was observed in him with respect to the \divine{Lord} God of Israel out of all the household of Jeroboam! \\
\poeml \v{14}``In addition to this, the \divine{Lord} will raise up for himself a king over Israel who will eliminate Jeroboam's dynasty, starting today and from now on. \v{15}The \divine{Lord} will attack Israel, and Israel will shake like a reed shakes in a river current! He will uproot Israel from this good land that he gave to their ancestors and he will scatter them beyond the Euphrates\fnote{\fbackref{14:15} The Heb. lacks \fbib{Euphrates}} River, because they erected their Asherim\fnote{\fbackref{14:15} I.e. cultic pillars erected in worship to Canaanite deities} and provoked the \divine{Lord} to become angry! \v{16}He will give up Israel because of Jeroboam's sins that he committed and by which Jeroboam\fnote{\fbackref{14:16} Lit. \fbib{he}} caused Israel to sin.''
\end{poetry}

\v{17}Then Jeroboam's wife got up and left for Tirzah. As soon as she set foot over the threshold of the house, the child died. \v{18}All of Israel mourned him at his burial, just as the \divine{Lord} had said when he spoke through Ahijah the prophet.
\passage{The Death of Jeroboam}

\v{19}Now as for the rest of Jeroboam's accomplishments, including how he waged war and how he reigned, you may read about them in the Book of the Chronicles of the Kings of Israel. \v{20}Jeroboam reigned for 22 years and then died, as had his ancestors, and his son Nadab reigned in his place.
\passage{Rehoboam Reigns over Judah}
\passageinfo{(2 Chronicles 11:5-12:6)}

\v{21}Meanwhile, Solomon's son Rehoboam reigned in Judah. Rehoboam was 41 years old when he became king, and he reigned for seventeen years in Jerusalem, the city where the Lord had chosen from all the tribes of Israel to place his Name. His mother was an Ammonite named Naamah. \v{22}Judah practiced what the \divine{Lord} considered to be evil. They did more to provoke him to jealousy than their ancestors had ever done by committing the sins that they committed. \v{23}They erected high places, sacred pillars, and Asherim\fnote{\fbackref{14:23} I.e. cultic pillars erected in worship to Canaanite deities} for themselves on every high hill and under every green tree. \v{24}They even maintained male shrine prostitutes throughout the land, and imitated every detestable practice that the nations practiced whom the \divine{Lord} had expelled in front of the Israelis.

\v{25}As a result, during the fifth year of the reign of\fnote{\fbackref{14:25} The Heb. lacks \fbib{the reign of}} King Rehoboam, King Shishak of Egypt invaded and attacked Jerusalem. \v{26}He stripped the \divine{Lord}'s Temple and the royal palace of their treasures. He took everything, even the gold shields that Solomon had made. \v{27}King Rehoboam made shields out of bronze to take their place, and then committed them to the care and custody of the commanders of those who guarded the entrance to the royal palace. \v{28}Whenever the king entered the \divine{Lord}'s Temple, the guards would carry them to and from the guard's quarters.

\v{29}As to the rest of Rehoboam's accomplishments, and everything else that he undertook, they are recorded in the Book of the Chronicles of the Kings of Judah, aren't they? \v{30}There was continual warfare between Rehoboam and Jeroboam, \v{31}but eventually Rehoboam died, as had his ancestors, and he was buried with his ancestors in the City of David. His mother's name had been Naamah the Ammonite, and his son Abijah became king to replace him.
\labelchapt{15}
\passage{Abijah's Reign over Judah}
\passageinfo{(2 Chronicles 13:1-14:1)}

\chapt{15}
\v{1}Abijah reigned over Judah starting in the eighteenth year of Nebat's son Jeroboam's reign. \v{2}He reigned for three years in Jerusalem. His mother's name was Maacah, the daughter of Abishalom. \v{3}He practiced the same sins that his father committed before he was born. Unlike his ancestor David, his heart never became devoted to the \divine{Lord} his God. \v{4}Nevertheless, for the sake of David, the \divine{Lord} his God maintained a lamp for David\fnote{\fbackref{15:4} Lit. \fbib{him}} in Jerusalem by raising up his son after him so that Jerusalem would be established, \v{5}because David had practiced what the \divine{Lord} considered to be right. He never avoided anything that the \divine{Lord} had commanded him during his entire lifetime, except for the case of Uriah the Hittite.

\v{6}There was continual military conflict between Rehoboam and Jeroboam throughout his entire lifetime. \v{7}The rest of Abijah's accomplishments, including everything he undertook, are written in the Chronicles of the Kings of Judah, are they not? And a state of war continued to exist between Abijah and Jeroboam. \v{8}Eventually, Abijah died, as did his ancestors, and he was buried in the City of David. His son Asa succeeded him as king.
\passage{Asa Reigns over Judah}
\passageinfo{(2 Chronicles 14:1-15:19)}

\v{9}Asa began to reign as Judah's king during the twentieth year of the reign of\fnote{\fbackref{15:9} The Heb. lacks \fbib{the reign of}} Jeroboam as king over Israel. \v{10}He reigned 41 years in Jerusalem. His mother's name was Maacah, the daughter of Abishalom. \v{11}Asa practiced what the \divine{Lord} considered to be right, just like his ancestor David. \v{12}He also removed the male cult prostitutes\fnote{\fbackref{15:12} Or \fbib{sodomites}} from the land and destroyed all the idols that his ancestors had made. \v{13}He removed his mother Maacah from her position as Queen Mother because she had made a detestable image dedicated to Asherah.\fnote{\fbackref{15:13} I.e. cultic pillars erected in worship to Canaanite deities} Asa cut down his mother's idol, crushed it, and burned it at the Kidron Brook. \v{14}Nevertheless, the high places were not removed, even though Asa's heart was blameless toward the \divine{Lord} all of his life. \v{15}Asa brought into the \divine{Lord}'s Temple the things that his father had dedicated, as well as his own dedicated gifts such as silver, gold, and temple service\fnote{\fbackref{15:15} The Heb. lacks \fbib{temple service}} implements.
\passage{Alliances with Aram against Israel}
\passageinfo{(2 Chronicles 16:1-17:1)}

\v{16}A state of continual military unrest existed between Asa and King Baasha of Israel throughout their lifetimes. \v{17}King Baasha of Israel invaded Judah and interdicted Ramah by building fortifications around it so no one could enter or leave to join King Asa of Judah. \v{18}But Asa removed all the silver and gold from the treasuries of the Lord's Temple and from his royal palace, placed them into the care of some servants, and then sent them to Tabrimmon's son King Ben-hadad of Aram, the grandson of Hezion, who lived in Damascus.

\v{19}``Let's make a treaty between you and me,'' he said, ``just like the one between my father and your father. Notice that I've sent you silver and gold to break your treaty with King Baasha of Israel, so he'll retreat from his attack\fnote{\fbackref{15:19} The Heb. lacks \fbib{his attack}} on me.''

\v{20}So King Ben-hadad did just what King Asa had asked: he sent his commanding officers to attack the cities of Israel, conquering Ijon, Dan, Abel-beth-maacah, all of Chinneroth,\fnote{\fbackref{15:20} I.e. the region encompassing the Sea of Galilee} and the territory of Naphtali. \v{21}When Baasha learned of this, he stopped fortifying Ramah and remained in Tirzah, \v{22}so King Asa published a proclamation throughout Judah (no one was left out) and they carried away the stones and timber with which Baasha had been fortifying Judah. King Asa used them to fortify Geba in Benjamin and Mizpah.

\v{23}The rest of Asa's accomplishments, his strength, everything that he undertook, and the cities that he fortified are written in the Book of the Chronicles of the Kings of Judah, are they not? However, as he approached old age, he became diseased in his feet. \v{24}Then Asa died, as had his ancestors, and he was buried with his ancestors in the City of David, his ancestor. His son Jehoshaphat reigned in his place.
\passage{Nadab Reigns over Israel}

\v{25}Jeroboam's son Nadab became king over Israel during the second year of the reign of\fnote{\fbackref{15:25} The Heb. lacks \fbib{the reign of}} King Asa over Judah. He reigned over Israel for two years, \v{26}practicing what the \divine{Lord} considered to be evil, living the way his father did, committing sins, and leading Israel to sin. \v{27}So Ahijah's son Baasha from the household of Issachar conspired against him and killed Nadab at Gibbethon in Philistia while Nadab and all of Israel were attacking Gibbethon. \v{28}Baasha killed him during the third year of the reign of\fnote{\fbackref{15:28} The Heb. lacks \fbib{the reign of}} King Asa of Judah and took Nadab's\fnote{\fbackref{15:28} Lit. \fbib{his}} place as king.

\v{29}As soon as he was established as king, he killed everyone in the household of Jeroboam. He left not even one single person alive. He destroyed them completely, just as the \divine{Lord} had spoken through his servant Ahijah the Shilonite,\fnote{\fbackref{15:29} Cf. 1King 14:7-16} \v{30}because of the sins that Jeroboam had committed, and because he led Israel into sin, provoking the \divine{Lord} God of Israel to become angry.

\v{31}Now the rest of Nadab's accomplishments, including everything he undertook, are written in the Book of the Chronicles of the Kings of Israel, are they not? \v{32}Meanwhile, a state of war continued to exist between Asa and Baasha king of Israel, throughout their reigns.
\passage{Baasha Reigns over Israel}

\v{33}During the third year of the reign of\fnote{\fbackref{15:33} The Heb. lacks \fbib{the reign of}} King Asa of Judah, Ahijah's son Baasha became king over all of Israel. He reigned for 24 years at Tirzah. \v{34}He practiced what the \divine{Lord} considered to be evil, living like Jeroboam did and leading Israel into sin.
\labelchapt{16}
\passage{Jehu Rebukes Baasha}

\chapt{16}
\v{1}Later, a message came from the \divine{Lord} to Hanani's son Jehu. It was directed to rebuke Baasha, and this is what it said:

\begin{poetry}
\poeml \v{2}I raised you from the dirt to become Commander-in-Chief\fnote{\fbackref{16:2} Lit. \fbib{Nagid}; i.e. a senior officer entrusted with dual roles of operational oversight and administrative authority} over my people Israel, but you've been living like Jeroboam, you've been leading my people Israel into sin, and you've been provoking me to anger with their sins. \v{3}So watch out! I'm going to devour Baasha and his household. I'm going to make your household just like the household of Jeroboam, Nebat's son. \v{4}Anyone from Baasha's household\fnote{\fbackref{16:4} The Heb. lacks \fbib{household}} who dies in the city will be eaten by dogs, and anyone of his who dies in the field the birds of the sky will eat.''
\end{poetry}

\v{5}Now the rest of Baasha's accomplishments, including everything that he undertook, as well as his strengths, are recorded in the Book of the Chronicles of the Kings of Israel, are they not? \v{6}Eventually, Baasha died, as had his ancestors, and he was buried in Tirzah. His son Elah was installed as king in his place.

\v{7}In addition, a message from the \divine{Lord} came through Hanani's son Jehu the prophet against Baasha and his household, not only because of all of the things that Baasha\fnote{\fbackref{16:7} Lit. \fbib{he}} did that the \divine{Lord} considered to be evil, including provoking the \divine{Lord}\fnote{\fbackref{16:7} Lit. \fbib{him}} to anger by what he did and by being like the household of Jeroboam, but also because Baasha\fnote{\fbackref{16:7} Lit. \fbib{he}} had destroyed Jeroboam's household.\fnote{\fbackref{16:7} Lit. \fbib{destroyed it}}
\passage{Elah Reigns over Israel}

\v{8}During the twenty-sixth year of the reign of\fnote{\fbackref{16:8} The Heb. lacks \fbib{the reign of}} King Asa of Judah, Baasha's son Elah became king over Israel and reigned at Tirzah for two years. \v{9}But his servant Zimri, who commanded half of his chariot forces, conspired against Elah while he was drinking himself drunk in the home of Arza, who managed the household at Tirzah. \v{10}Zimri went inside, attacked him, and killed him in the twenty-seventh year of the reign of King Asa of Judah, and then became king in Elah's place. \v{11}As soon as he had consolidated his reign, he executed the entire household of Baasha. He did not leave a single male alive, including any of Baasha's relatives or friends. \v{12}In doing so, Zimri destroyed the entire household of Baasha, in keeping with the message from the \divine{Lord} that he had spoken against Baasha through Jehu the prophet \v{13}because of all the sins that Baasha and his son Elah had committed and because of what they did to lead Israel into sin, thus provoking the \divine{Lord} God of Israel to anger with their idolatry. \v{14}Now the rests of Elah's accomplishments, including everything he undertook, are written in the Book of the Chronicles of the Kings of Israel, are they not?
\passage{Zimri Reigns over Israel}

\v{15}Zimri reigned for seven days at Tirzah during the twenty-seventh year of the reign of\fnote{\fbackref{16:15} The Heb. lacks \fbib{the reign of}} King Asa of Judah. At that time, the army was encamped in a siege against Gibbethon of Philistia. \v{16}The army at the encampment heard this report: ``Zimri has conspired against the king and killed him.'' So the entire army of\fnote{\fbackref{16:16} The Heb. lacks \fbib{army of}} Israel made Omri, their commander, king over Israel. \v{17}Then Omri and the entire army of\fnote{\fbackref{16:17} The Heb. lacks \fbib{army of}} Israel left from Gibbethon and attacked Tirzah. \v{18}When Zimri observed that the city had been captured, he retreated into the king's palace, set fire to the citadel, and died when the palace burned down around him \v{19}because of the sins that he committed by doing what the \divine{Lord} considered to be evil, living like Jeroboam did, and sinning so as to lead Israel into sin. \v{20}The rest of Zimri's accomplishments, including his conspiracy that he carried out, are written in the Book of the Chronicles of the Kings of Israel, are they not?
\passage{Omri Reigns over Israel and Builds Samaria}

\v{21}The army\fnote{\fbackref{16:21} Or \fbib{people}} of Israel was divided into two parties: half of the army\fnote{\fbackref{16:21} Or \fbib{people}} were loyal to Ginath's son Tibni and wanted to make him king, and half were loyal to Omri. \v{22}But the army\fnote{\fbackref{16:22} Or \fbib{people}} that was loyal to Omri was victorious over Ginath's son Tibni. Tibni later died and Omri became king. \v{23}During the thirty-first year of the reign of\fnote{\fbackref{16:23} The Heb. lacks \fbib{the reign of}} King Asa of Judah, Omri became king over Israel. He reigned for twelve years, six of them at Tirzah. \v{24}He bought the hill of Samaria from Shemer for two talents\fnote{\fbackref{16:24} I.e. about 150 pounds; a talent weighed about 75 pounds} of silver, fortified the hill, and named the city Samaria after Shemer, the former owner of the hill. \v{25}Omri practiced what the \divine{Lord} considered to be evil, doing far more evil than anyone who had reigned before him. \v{26}He lived just like Nebat's son Jeroboam, and by his sin he led Israel into sin, provoking the \divine{Lord} God of Israel with their idolatry. \v{27}Now the rest of Omri's accomplishments, including the power that he demonstrated, are recorded in the Book of the Chronicles of the Kings of Israel, are they not? \v{28}So Omri died, as had his ancestors, and he was buried in Samaria. His son Ahab became king in his place.
\passage{Ahab Reigns over Israel and Marries Jezebel}

\v{29}Omri's son Ahab became king over Israel in the thirty-eighth year of King Asa of Judah. He\fnote{\fbackref{16:29} Lit. \fbib{Omri's son Ahab}} reigned over Israel in Samaria for 22 years. \v{30}Omri's son Ahab practiced more of what the \divine{Lord} considered to be evil than anyone who had lived before him. \v{31}In fact, as if it were nothing for him to live like Nebat's son Jeroboam, Ahab married Jezebel, the daughter of King Ethbaal of Sidon. Then he went out to serve Baal and worship him. \v{32}He built an altar for Baal in a temple for Baal that he constructed in Samaria. \v{33}Ahab also erected an Asherah, doing more to provoke the \divine{Lord} God of Israel than all of the kings of Israel who had reigned before him. \v{34}It was during Ahab's reign that Hiel the Bethelite rebuilt Jericho. He laid its foundations just as his firstborn son Abiram was dying, and he erected its gates while his youngest son Segub was dying, thus fulfilling the message that the \divine{Lord} delivered through Nun's son Joshua.\fnote{\fbackref{16:34} Cf. Josh 6:26}
\labelchapt{17}
\passage{Elijah Calls for a Drought}

\chapt{17}
\v{1}Elijah the foreigner,\fnote{\fbackref{17:1} Lit. \fbib{Tishbite}; or \fbib{sojourner}} who was an alien resident from Gilead, told Ahab, ``As the \divine{Lord} God of Israel lives, in whose presence I'm standing, there will be neither dew nor rain these next several years, except when I say so.''

\v{2}Later, this message came to him from the \divine{Lord}: \v{3}``Leave here and go into hiding at the Wadi\fnote{\fbackref{17:3} I.e. a seasonal stream or river that channels water during rain seasons but is dry at other times} Cherith, where it enters the Jordan River.\fnote{\fbackref{17:3} The Heb. lacks \fbib{River}; and so throughout the chapter} \v{4}You will be able to drink from that brook, and I've commanded some crows to sustain you there.''

\v{5}So Elijah\fnote{\fbackref{17:5} Lit. \fbib{he}} left and did exactly what the \divine{Lord} had told him to do---he went to live near the Wadi\fnote{\fbackref{17:5} I.e. a seasonal stream or river that channels water during rain seasons but is dry at other times} Cherith, where it enters the Jordan River. \v{6}Crows would bring him bread and meat both in the morning and in the evening, and he would drink from the brook. \v{7}But after a while,\fnote{\fbackref{17:7} Lit. \fbib{But at the end of days}} the brook dried up because there had been no rain in the land.
\passage{Elijah Visits the Widowed Mother of Zarephath}

\v{8}Then this message came to him from the \divine{Lord}: \v{9}``Get up, move to Zarephath in Sidon, and stay there. Look! I've commanded a widow to sustain you there.''

\v{10}So he got up and went to Zarephath. As he arrived at the entrance to the city, a widow was there gathering sticks. So he asked her, ``Please, may I have some water in a cup so I can have a drink.'' \v{11}While she was on her way to get the water, he called out to her, ``Would you please also bring me a piece of bread while you're at it?''\fnote{\fbackref{17:11} Lit. \fbib{bread in your hand}}

\v{12}``As the \divine{Lord} your God lives,'' she replied, ``I don't have so much as a muffin, just a handful of flour in a bowl and some oil left in a bottle. Now I'm going to find some sticks so I can cook a last meal for my son and for me. Then we're going to eat it and die.''

\v{13}But Elijah told her, ``You can stop being afraid. Go and do what you said, but first make me a muffin and bring it to me. Then make a meal for yourself and for your son, \v{14}because this is what the \divine{Lord} God of Israel says: `That jar of flour will not run out, nor will that bottle of oil become empty until the very day that the \divine{Lord} sends rain on the surface of the ground.'\,''

\v{15}So she went out and did precisely what Elijah told her to do. As a result, Elijah,\fnote{\fbackref{17:15} Lit. \fbib{he}} the widow,\fnote{\fbackref{17:15} Lit. \fbib{she}} and her son\fnote{\fbackref{17:15} Lit. \fbib{household}} were fed for days. \v{16}The jar of flour never ran out and the bottle of oil never became empty, just as the \divine{Lord} had promised\fnote{\fbackref{17:16} Lit. \fbib{spoken}} through\fnote{\fbackref{17:16} Lit. \fbib{through the hand of}} Elijah.
\passage{Elijah Restores the Widow's Son}

\v{17}Sometime later, the son of the woman who owned the house became ill. In fact, his illness became so severe that he died.\fnote{\fbackref{17:17} Lit. \fbib{that no breath remained in him}} \v{18}``What do we have in common, you man of God?'' she accused Elijah. ``You came to me so you could uncover my guilt! And you're responsible for the death of my son!''

\v{19}``Give me your son,'' he replied. Then he took him from her lap, carried him upstairs to the room where he lived, and laid him on his bed. \v{20}Then he called out to the \divine{Lord} and asked him, ``\divine{Lord} my God, have you also brought evil to this dear widow with whom I am living as her guest? Have you caused the death of her son?'' \v{21}Then he stretched himself three times and cried out to the \divine{Lord}, ``\divine{Lord} my God, please cause the soul of this little boy to return to him.''

\v{22}The \divine{Lord} listened to Elijah, and the soul of the little boy returned to him, and he revived. \v{23}Then Elijah took the little boy downstairs from the upper chamber back into the main house and delivered him to his mother. ``Look,'' Elijah told her, ``your son is alive.''

\v{24}The woman responded to Elijah, ``Now at last I've really learned that you are a man of God and that what you have to say about the \divine{Lord}\fnote{\fbackref{17:24} Lit. \fbib{that the word of the \divine{Lord} in your mouth}} is the truth.''
\labelchapt{18}
\passage{Elijah Rebukes Ahab}

\chapt{18}
\v{1}Quite some time later---three years later!---this message from the \divine{Lord} came to Elijah: ``Go visit Ahab, and I'll send some rain to the surface of the ground.'' \v{2}So Elijah went to show himself to Ahab, right when the famine in Samaria was most severe.

\v{3}Ahab called for Obadiah, his household supervisor. This man, who feared the \divine{Lord} very much, \v{4}had taken 100 prophets and had hidden them by fifties in a cave, providing them with food and water when Jezebel was trying to destroy the \divine{Lord}'s prophets.

\v{5}Ahab had instructed Obadiah, ``Go throughout the land to all of the water springs and to all of the valleys. Maybe we'll find some grass to keep the horses and mules alive. Also, maybe we won't have to kill some of our cattle.'' \v{6}So they divided the land between them so they could conduct their survey. Ahab went off by himself in one direction and Obadiah went off by himself in the other.

\v{7}While Obadiah was on the road, Elijah met him. Obadiah recognized him and bowed down with his face to the ground. ``It's you, isn't it, my master Elijah?''

\v{8}``I am,'' he replied. ``Go tell your master, `Look! Elijah!'\,''

\v{9}But Obadiah replied, ``What did I do wrong, that you would put me in a position where Ahab would execute me? \v{10}As surely as the \divine{Lord} your God lives, there isn't a nation or kingdom where my master hasn't tried to find you. Whenever they would say `He isn't here,' he forced that kingdom or nation to swear that they hadn't seen you. \v{11}But now you're saying `Go tell your master, ``Elijah is here!''\,' \v{12}As soon as I've left you, the Spirit of the \divine{Lord} will carry you off to I don't know where! Then when I go tell Ahab and he can't find you, he'll kill me, even though I have been your servant and have feared the \divine{Lord} since I was young! \v{13}Hasn't anyone told you, my master, what I did when Jezebel was killing the \divine{Lord}'s prophets? I hid 100 of the \divine{Lord}'s prophets by fifties in a cave and provided food and water for them. \v{14}Now you're saying, `Go tell your master, ``Elijah's here!''\,' He's sure to kill me!''

\v{15}But Elijah promised him, ``As the \divine{Lord} of the Heavenly Armies lives, in whose presence I stand, I will appear to Ahab today.''

\v{16}So Obadiah went out to meet Ahab and reported to him. Then Ahab went to meet Elijah. \v{17}When Ahab saw Elijah, Ahab asked him, ``Is it really you, you destroyer of Israel?''

\v{18}But Elijah\fnote{\fbackref{18:18} Lit. \fbib{he}} replied, ``I'm no destroyer of Israel. But you and your ancestor's household have been doing that, because you have abandoned the \divine{Lord}'s commandments and have followed the Baals. \v{19}So go gather all of Israel to meet me on Mount Carmel. Bring along 450 prophets of Baal and 400 prophets of the Asherah who are funded at Jezebel's expense.''\fnote{\fbackref{18:19} Lit. \fbib{who eat at Jezebel's table}}
\passage{Elijah Defeats the Prophets of Baal}

\v{20}Ahab sent for the Israelis and brought the prophets together at Mount Carmel, \v{21}where Elijah approached all the people and asked them, ``How long will you keep hesitating\fnote{\fbackref{18:21} Lit. \fbib{dancing}; or \fbib{limping}} between both sides? If the \divine{Lord} is God, go after him. If Baal, go after him.''

But the people didn't say a word.

\v{22}So Elijah told the people, ``I'm the only one left over as a prophet of the \divine{Lord}, am I? But Baal's prophets number 450 men? \v{23}So let them provide two oxen. They can choose one ox for themselves. Cut it up, lay it on top of some wood, but don't set fire to it. I will prepare the other ox and lay it on top of some wood, and I won't set fire to it. \v{24}Then you can call on the name of your god, and I'll call on the name of the \divine{Lord}. Let the God who answers by fire be our God!''

``That's a good idea!'' all the people shouted.

\v{25}So Elijah told the prophets of Baal, ``Choose an ox for yourselves and you prepare it first, since there are so many of you. Call on the name of your god, but don't set fire to the offering.''

\v{26}So they took the ox that was given to them, prepared it, and called on the name of Baal from early morning until noon. ``Baal! Answer us!'' they cried. But there was no response. Nobody answered. So they kept on dancing\fnote{\fbackref{18:26} Or \fbib{limping}} around the altar that they had made.

\v{27}Starting about noon, Elijah began to tease them:

``Shout louder!

``He's a god, so maybe he's busy.

``Maybe he's relieving himself.

``Maybe he's busy someplace.

``Maybe he's taking a nap and somebody needs to wake him up.''

\v{28}So the prophets of Baal\fnote{\fbackref{18:28} Lit. \fbib{So they}} cried even louder and slashed themselves with swords and lances until their blood gushed out all over them, as was their custom. \v{29}They kept on raving right through midday and until it was time to offer the evening sacrifice, but there was still no response. Nobody answered, and nobody paid attention.

\v{30}Eventually, Elijah told everybody, ``Come here!'' So everybody approached him, and he repaired the \divine{Lord}'s altar that had been torn down. \v{31}Elijah took twelve stones, one for each of the tribes of Jacob's descendants, to whom the message from the \divine{Lord} had come that ``Israel is to be your name.'' \v{32}So Elijah used the stones to build an altar to the name of the \divine{Lord}. But then he dug a trench around the altar large enough to hold two measures\fnote{\fbackref{18:32} Lit. \fbib{seahs}; or \fbib{hold four gallons}; i.e. a trench encircling the altar and wide enough that a container holding about four gallons could be laid inside it} of seed. \v{33}Then he laid the wood in order, cut the bull into pieces, and laid them on top of the wood.

``Fill four pitchers with water,'' he ordered. ``Then pour them out on the burnt offering and the wood.''

\v{34}``Do it a second time,'' he ordered. So they did it a second time.

``Do it a third time,'' he said. So they did it a third time. \v{35}The water ran down around the altar and completely filled the trench.\fnote{\fbackref{18:35} Lit. \fbib{trench with water}}
\passage{Elijah's Prayer and God's Answer by Fire}

\v{36}As the time for the evening offering arrived, Elijah the prophet approached and said, ``\divine{Lord} God of Abraham, Isaac, and Israel, let it be known today that you are God in Israel and that I, your servant, have done all of this in obedience to your word. \v{37}Answer me, \divine{Lord}! Answer me so that this people may know that you, \divine{Lord}, are God, and that you are turning back their hearts again.''

\v{38}Right then the \divine{Lord}'s fire fell and consumed the burnt offering, the wood, the stones, the dust, and even the water that was in the trench! \v{39}When all the people saw what had happened, they fell flat on their faces and cried out ``The \divine{Lord} is God! The \divine{Lord} is God!''

\v{40}But Elijah said, ``Arrest the prophets of Baal. Don't let even one of them get away.'' So the people\fnote{\fbackref{18:40} Lit. \fbib{So they}} seized them, and Elijah brought them down to the Wadi\fnote{\fbackref{18:40} I.e. a seasonal stream or river that channels water during rain seasons but is dry at other times} Kishon and executed them there.
\passage{The Rain Storm Ends the Drought}

\v{41}After this, Elijah told Ahab, ``Get up and have something to eat and drink, because there's the sound of a coming rainstorm.'' \v{42}So Ahab got up to get something to eat and drink while Elijah went back up to the top of Mount\fnote{\fbackref{18:42} The Heb. lacks \fbib{Mount}} Carmel, where he bowed low to the ground and placed his face between his knees.

\v{43}Then he told his young servant, ``Go and look toward the sea.''

So he went and looked out to sea. ``Nothing there,'' he said.

But Elijah told him to go back seven times. \v{44}On the seventh look, he said, ``Look! There's a cloud, a small one, about the size of a man's hand. It's coming up out of the sea!''

``Get up and find Ahab!'' Elijah\fnote{\fbackref{18:44} Lit. \fbib{he}} said. ``Tell him, `Mount your chariot and ride down the mountain\fnote{\fbackref{18:44} The Heb. lacks \fbib{the mountain}} so the storm doesn't stop you.'\,''

\v{45}A little while later, the sky turned black with storm clouds and winds, and there was a heavy shower. So Ahab rode off to Jezreel. \v{46}After Ahab had left,\fnote{\fbackref{18:46} Lit. \fbib{Then}} the hand of the \divine{Lord} came upon Elijah, and he tucked his mantle into his belt and outran Ahab in a race to the city gate of Jezreel.
\labelchapt{19}
\passage{Elijah Runs from Jezebel}

\chapt{19}
\v{1}Ahab complained to Jezebel about everything that Elijah had done, especially the part about him killing all the prophets of Baal with a sword. \v{2}Jezebel sent a messenger to tell Elijah, ``May the gods do the same to me and even more if tomorrow about this time I haven't made you like one of those prophets you had killed.''\fnote{\fbackref{19:2} The Heb. lacks \fbib{prophets you had killed}}

\v{3}Elijah was terrified, so he got up and ran for his life to Beer-sheba, which is part of Judah, and left his servant there \v{4}and ran for a day's journey deep into the wilderness. He found a juniper tree, sat down under it, and prayed that he could die. He asked God, ``Enough! \divine{Lord}! Take my life, because I'm not better than my ancestors!'' \v{5}Then he lay down and went to sleep under the juniper tree. All of a sudden, there was an angel, who kept grabbing him and telling him, ``Get up! Eat!''

\v{6}So he looked around, and there near his head was a muffin sitting on top of some heated stones, along with a jar of water. Elijah ate and drank and then lay down again. \v{7}Later, the angel of the \divine{Lord} came a second time, grabbed him, and said ``Get up! Eat! The journey ahead\fnote{\fbackref{19:7} The Heb. lacks \fbib{ahead}} is too difficult for you!'' \v{8}So Elijah\fnote{\fbackref{19:8} Lit. \fbib{he}} got up, ate and drank, and survived on that one meal for 40 days and nights as he set out on his journey to Horeb, God's mountain.
\passage{Elijah Talks to God at Horeb}

\v{9}Elijah\fnote{\fbackref{19:9} Lit. \fbib{He}} arrived at a cave and stayed there. All of a sudden this message came from the \divine{Lord}: ``What are you doing here, Elijah?''

\v{10}``I've been very zealous for the \divine{Lord} God of the Heavenly Armies,'' he replied. ``The Israelis have abandoned your covenant, demolished your altars, executed your prophets with swords, and I---that's right, just me!---am the only one left. Now they're seeking my life, to get rid of me!''

\v{11}``Go out,'' he responded, ``and stand on the mountain in the presence of the \divine{Lord}.'' And there was the \divine{Lord}, passing by! A tremendous, mighty windstorm was tearing at the mountains and breaking the rocks in pieces in the presence of the \divine{Lord}, but the \divine{Lord} was not in the windstorm. After the wind there came an earthquake, but the \divine{Lord} was not in the earthquake. \v{12}After the earthquake there came fire, but the \divine{Lord} was not in the fire. And after the fire, there was the sound of a gentle whisper. \v{13}As soon as Elijah heard it, he covered his face in his mantle, went outside, and stood at the entrance to the cave. And there a voice spoke to him and said, ``What are you doing here, Elijah?''

\v{14}``I've been very zealous for the \divine{Lord} God of the Heavenly Armies,'' he replied. ``The Israelis have abandoned your covenant, demolished your altars, executed your prophets with swords, and I---that's right, just me!---am the only one left. Now they're seeking my life, to get rid of me!''

\v{15}The \divine{Lord} replied to him, ``Go! Return to Damascus, and when you get there, anoint Hazael as king over Aram, \v{16}anoint Nimshi's son Jehu as king over Israel, and anoint Shaphat's son Elisha from Abel-meholah as a prophet to replace you. \v{17}Whoever escapes from Hazael's sword Jehu will execute, and whoever escapes from Jehu's sword Elisha will put to death. \v{18}Nevertheless, I've reserved 7,000 in Israel who have neither bowed their knees to Baal nor kissed him.''
\passage{Elisha Chosen to Replace Elijah}

\v{19}Elijah left there and located Shaphat's son Elisha, who was plowing, along with a total of\fnote{\fbackref{19:19} The Heb. lacks \fbib{a total of}} twelve pairs of oxen.\fnote{\fbackref{19:19} The Heb. lacks \fbib{of oxen}} (He was plowing with the twelfth pair.) As Elijah passed by, he tossed his cloak at Elisha.\fnote{\fbackref{19:19} Lit. \fbib{him}} \v{20}He abandoned the oxen, ran off to follow Elijah, and asked him, ``Please, let me kiss my mother and father good-bye, and then I'll come after you.''

``Go back again,'' Elijah replied. ``What have I done to you?''

\v{21}So Elisha\fnote{\fbackref{19:21} Lit. \fbib{he}} turned back, took the pair of oxen, sacrificed them, boiled their flesh using the farm implements for fuel, and gave the food to the people with him.\fnote{\fbackref{19:21} The Heb. lacks \fbib{with him}} Then he got up, followed Elijah, and became his servant.
\labelchapt{20}
\passage{Ahab Attacks the Arameans}

\chapt{20}
\v{1}A little while later, King Ben-hadad of Aram mustered an army of cavalry and chariots in a military confederacy with 32 kings, invaded Samaria, and set up siege encampments there. \v{2}Then he sent envoys to visit King Ahab of Israel and told him, ``This is what Ben-hadad says: \v{3}`Your silver and gold belong to me. So do the most beautiful of your wives and children.'\,''

\v{4}``Whatever you want, your majesty,'' the king of Israel answered. ``I belong to you, as does everything I own.''

\v{5}After delivering Ahab's answer,\fnote{\fbackref{20:5} The Heb. lacks \fbib{After delivering Ahab's answer} } the envoys returned with this message: ``This is what Ben-hadad says: `I've sent my envoys to you to tell you that your silver, gold, wives, and children are to be given to me. \v{6}About this time tomorrow, I'll send my servants to you, and they'll search through your palace and your servants' houses. Whatever is important to you will be seized\fnote{\fbackref{20:6} Lit. \fbib{seized in their hand}} and taken away.'\,''

\v{7}Then the king of Israel called together all of the elders of the land and told them, ``Please note that this man is here looking for trouble. He sent a message to me, demanding my wives, my children, and my silver and gold, and I haven't refused him.''

\v{8}``Don't listen to him,'' all the elders and the people replied. ``And don't agree to his terms.''\fnote{\fbackref{20:8} The Heb. lacks \fbib{to his terms}}

\v{9}So he told Ben-hadad's envoys, ``Tell his majesty the king, `Everything that you asked for the first time I will do, but this thing I cannot do.'\,'' So the envoys left to deliver Ahab's response. They\fnote{\fbackref{20:9} Lit. \fbib{deliver and}} returned a little while later.

\v{10}Beh-hadad sent this message back: ``May the gods do so to me, and more than that also, if the dust that remains of Samaria is enough to fill up a few handfuls for all of the armies at my disposal.''

\v{11}But the king of Israel replied, ``Tell him, `The one who is starting to strap on his battle armor should never brag like the one who is taking it off.'\,''

\v{12}Ben-hadad received Ahab's response\fnote{\fbackref{20:12} Lit. \fbib{message}} while he was celebrating with his kings in the battle pavilions. ``Sound `Battle Stations!'\,'' he ordered, and the army began to prepare their attack.
\passage{God's Prophets Rebuke Ahab}

\v{13}Right about then, a prophet approached King Ahab of Israel and told him, ``This is what the \divine{Lord} says: `You see all of this great big army, do you? Well now, I'm going to deliver them all right into your hand, and you will learn that I am the \divine{Lord}!'\,''

\v{14}``By whom?'' Ahab asked.

``This is what the \divine{Lord} says,'' the prophet replied. ```By the young men who serve as officials within the provinces.'\,''

``Who is to begin the battle?'' Ahab asked.

``You,'' the prophet answered.

\v{15}So Ahab\fnote{\fbackref{20:15} Lit. \fbib{he}} gathered together 232 young men who served as officials within the provinces and then mustered 7,000 soldiers from among the Israelis. \v{16}They attacked at noon, just as Ben-hadad was drinking himself drunk in the battle pavilions, along with the 32 kings who had joined him. \v{17}The young men who served as officials within the provinces led the charge, and somebody informed Ben-hadad, ``Some men have come out from Samaria.''

\v{18}``Take them alive, whether they've come in peace or not,'' he ordered.

\v{19}Meanwhile, as the young men who served as officials within the provinces left the city, their army followed after them. \v{20}Each man struck down his opponent, and the Arameans ran away with Israel in pursuit. King Ben-hadad of Aram escaped on horseback with the help of\fnote{\fbackref{20:20} The Heb. lacks \fbib{the help of}} his cavalry. \v{21}The king of Israel went out and attacked the cavalry and chariots and killed the Arameans in a massive victory.\fnote{\fbackref{20};21 Or \fbib{slaughter}}

\v{22}The prophet approached the king of Israel and told him, ``Go replenish your forces and prepare for the future, because early this next year the king of Aram will attack you again.''
\passage{The Arameans are Defeated}

\v{23}Sure enough, the advisors to the king of Aram told him, ``Their gods are mountain gods. That's why they were stronger than we were. But when we fight them on the plains, we're certain to be the stronger army! \v{24}So do this: remove the kings from command\fnote{\fbackref{20:24} The Heb. lacks \fbib{from command}} and replace them with captains. \v{25}Then replace the army that you lost, horse-for-horse and chariot-for-chariot. We'll fight them on the plains, and we're certain to be the stronger army.'' Ben-hadad\fnote{\fbackref{20:25} Lit. \fbib{He}} listened to what they had to say and carried out their advice.

\v{26}Early the next year, Ben-hadad mustered the Arameans and invaded Aphek in a battle against Israel. \v{27}The Israelis were mustered, equipped with provisions, and sent out to fight. The Israeli encampment looked like two little flocks of goats compared to how the Aramean encampments\fnote{\fbackref{20:27} The Heb. lacks \fbib{encampments}} filled the countryside!

\v{28}Right about then, a man of God approached and told the king of Israel, ``This is what the \divine{Lord} says: `Because the Arameans keep saying ``The \divine{Lord} is a mountain god, but isn't a valley god,'' I'm going to deliver this entire vast army right into your control, so you'll learn that I really am the \divine{Lord}.'\,'' \v{29}So they remained in opposing camps for seven days. Then on the seventh day the battle commenced, and the Israelis killed 100,000 Aramean infantry troops in a single day. \v{30}The rest of the Aramean army retreated into Aphek, but the city wall collapsed on 27,000 soldiers who had taken shelter there. Ben-hadad himself ran away and hid inside a closet\fnote{\fbackref{20:30} Lit. \fbib{inside an inner room}} somewhere in the city.

\v{31}``Look, now,'' his advisors suggested, ``we've heard that the Israeli kings are merciful. So let's clothe ourselves with sackcloth, tie our hair back with ropes, and go out to the king of Israel. Maybe he'll spare your life.'' \v{32}So they put on some sackcloth, tied their hair back with ropes, and approached the king of Israel. ``Your servant Ben-hadad says this,'' they said. ``Please let me live.''

``Is he still alive?'' Ahab asked. ``He's my brother.''

\v{33}Ben-hadad's advisors,\fnote{\fbackref{20:33} Lit. \fbib{The men}} quickly analyzing the signs in what Ahab was saying, responded, ``Yes, your brother Ben-hadad.''

``Go get him,'' Ahab responded. So Ben-hadad came out to him, and Ahab took him up into his personal chariot.

\v{34}Ben-hadad made this promise to Ahab: ``I will restore the cities that my ancestors took from your ancestors. You'll be able to build streets named after yourself in Damascus, as my father did in Samaria.''

``With this promise I will release you,'' Ahab\fnote{\fbackref{20:34} Lit. \fbib{he}} replied. So Ahab\fnote{\fbackref{20:34} Lit. \fbib{he}} made a treaty with Ben-hadad\fnote{\fbackref{20:34} Lit. \fbib{him}} and let him go.
\passage{Ahab is Condemned}

\v{35}Right about then, one of the members of the guild\fnote{\fbackref{20:35} Lit. \fbib{sons}} of prophets told another through a message from the \divine{Lord}: ``Please strike me!'' But the man refused to do so, \v{36}so he told him, ``Because you haven't obeyed the \divine{Lord}'s voice, as soon as you leave here, a lion will kill you.'' As soon as the man left, a lion found him and killed him.

\v{37}Later, he found another man and told him, ``Please strike me!'' So the man struck him and wounded him. \v{38}Then the prophet left and waited for the king to pass by, disguising himself with a bandage over his eyes.

\v{39}As the king was passing by, he cried out to the king and told him, ``Your servant went out into the middle of the battle, and a soldier turned aside, brought a prisoner to me, and told me, `Guard this man. If he turns up missing for any reason at all, you'll pay for it with your life or be fined one talent\fnote{\fbackref{20:39} I.e. about 75 pounds} of silver.' \v{40}While your servant was busy here and there, the prisoner escaped.''

The king told him, ``By your actions you've earned the proper judgment!''

\v{41}Then the prophet quickly tore off his bandage, and the king of Israel recognized him as being one of the prophets. \v{42}He told the king,\fnote{\fbackref{20:42} Lit. \fbib{told him}} ``This is what the \divine{Lord} says: `Because you let the man whom I had dedicated to destruction go free, therefore your life is to be forfeited for his life, and your people for his people.'\,''

\v{43}After hearing this, the king of Israel rode back to his palace in Samaria, frustrated and in a foul mood.
\labelchapt{21}
\passage{The Naboth Vineyard Incident}

\chapt{21}
\v{1}Meanwhile, there was a man named Naboth from Jezreel who owned a vineyard that was located contiguous to King Ahab's palace in Samaria. \v{2}Ahab addressed Naboth and asked him, ``I would like to plant a vegetable garden near my house. Please exchange your vineyard with a better one from me, or if you'd rather have cash, I'll buy it for its full value.''

\v{3}But Naboth replied to Ahab, ``No way! The \divine{Lord} prohibits the sale to you of the inheritance of my ancestors!''

\v{4}Ahab went back to his palace, sullen and in a foul mood, because Naboth the Jezreelite had turned down Ahab's offer by saying ``I will not transfer my ancestors' inheritance to you!'' He laid down on his bed, curled up with his face to the wall, and refused to eat.

\v{5}But his wife Jezebel went to him and asked him, ``How is it that you're so sullen and refusing to eat?''

\v{6}``I asked Naboth the Jezreelite, `Sell me your vineyard for cash, or if you want, I'll give you a better one in its place.' But he refused. He told me, `I won't give you my vineyard!'\,''

\v{7}``Aren't you the reigning king of Israel,'' his wife Jezebel replied. ``Get up, have a meal, and get ready to be happy. I'll go get you the vineyard that Naboth the Jezreelite owns.'' \v{8}So she wrote some memos in Ahab's name, set his personal seal to them, and sent them to the elders and nobles who lived with Naboth in his city. \v{9}In the memos, she wrote the following directives: ``Proclaim a public fast and seat Naboth in the front row. \v{10}Seat two wicked men in front of him, and make them testify against him. Tell them to claim `You cursed God and the king.' Then take him out and stone him to death.''

\v{11}So the leading men of the city, along with the elders and nobles who lived there, did precisely what Jezebel had directed them to do. They followed the instructions that she had set forth in the memos: \v{12}They proclaimed a public fast and seated Naboth in the front row. \v{13}Two wicked men came in, sat down in front of them, and testified against Naboth in public, ``Naboth cursed God and the king!'' So they took him outside the city and stoned him to death.\fnote{\fbackref{21:13} Lit. \fbib{death with stones}}

\v{14}Afterwards, they sent a message\fnote{\fbackref{21:14} The Heb. lacks \fbib{a message}} to Jezebel that said, ``Naboth has been stoned. He's dead.''

\v{15}When Jezebel heard that Naboth had been stoned to death, she told Ahab, ``Get up and confiscate Naboth's vineyard that he refused to sell you for cash. Naboth the Jezreelite isn't alive anymore. He's dead!'' \v{16}So once he heard that Naboth was dead, Ahab got up, went down to the vineyard of Naboth the Jezreelite, and confiscated it.
\passage{Elijah Rebukes Ahab}

\v{17}That's when this message from the \divine{Lord} came to Elijah the foreigner:\fnote{\fbackref{21:17} Lit. \fbib{Tishbite}; or \fbib{sojourner}} \v{18}``Get up and go down to meet King Ahab of Israel. He's in Samaria. Look! He's in Naboth's vineyard, where he's gone to confiscate it. \v{19}Ask the king, `Did you commit murder? And now you're going to steal as well?' Also tell him, `This is what the \divine{Lord} says: ``Where the dogs were licking up Naboth's blood, dogs will also lick up your blood---that's right---yours!''\,'\,''

\v{20}Later on, Ahab asked Elijah, ``Have you found me, my enemy?''

But Elijah answered, ``I've found you because you sold yourself to do what the \divine{Lord} considers to be evil! \v{21}Now pay attention! I'm going to send evil in your direction! I will completely sweep you away and eliminate from Ahab every male, whether indentured servant or free, throughout Israel. \v{22}I will make your household resemble that of Nebat's son Jeroboam, or like the household of Ahijah's son Baasha, because of how you've provoked me to anger and made Israel to sin. \v{23}The \divine{Lord} also has this to say about Jezebel: `Dogs will eat Jezebel within the outer ramparts of Jezreel. \v{24}Dogs will eat whoever belongs to Ahab and who dies in the city. The birds of the sky will eat whoever dies in the fields.'\,''

\v{25}It can be truly said that no one else sold himself to practice what the \divine{Lord} considered to be evil quite like the way Ahab did, because his wife Jezebel incited him. \v{26}His behavior in pursuing idolatry was detestable, just like the Amorites had done whom the \divine{Lord} had expelled in front of the army of Israel. \v{27}Nevertheless, as soon as Ahab heard this message, he tore his clothes, put on sackcloth, and fasted. He even slept in sackcloth and wandered around meekly.

\v{28}Later, this message from the \divine{Lord} came to Elijah the foreigner:\fnote{\fbackref{21:28} Lit. \fbib{Tishbite}; or \fbib{sojourner}} \v{29}``Have you noticed that Ahab has humbled himself in my presence? Because he has humbled himself in my presence, I will not bring his evil to harvest\fnote{\fbackref{21:29} The Heb. lacks \fbib{to harvest}} during his lifetime, but I will bring evil to his household during his son's lifetime.''
\labelchapt{22}
\passage{King Ahab Invites Jehoshaphat to Invade Aram}
\passageinfo{(2 Chronicles 18:1-11)}

\chapt{22}
\v{1}Three years passed without war between Aram and Israel. \v{2}During that third year, King Jehoshaphat of Judah went to visit the king of Israel. \v{3}The king of Israel asked his servants, ``Were you aware that Ramoth-gilead belongs to us, but we aren't doing anything to remove it from the control of the king of Aram?''

\v{4}Then he asked Jehoshaphat, ``Will you join me in battle against Ramoth-gilead?''

``I'm with you,'' Jehoshaphat answered the king of Israel. ``My army will join yours, and my cavalry will be your cavalry.'' \v{5}But Jehoshaphat also asked the king of Israel, ``Please ask for a message from the \divine{Lord}, first.''

\v{6}So the king of Israel called in about 400 prophets and asked them, ``Should we go attack Ramoth-gilead, or should I call off the attack?''\fnote{\fbackref{22:6} The Heb. lacks \fbib{the attack}}

``Go attack them,'' they all said, ``because the Lord will drop them right into the king's hand!''

\v{7}But Jehoshaphat asked, ``Isn't there a prophet of the \divine{Lord} left here that we could talk to?''

\v{8}``There is still one man left by whom we could ask the \divine{Lord} what to do,''\fnote{\fbackref{22:8} The Heb. lacks \fbib{what to do}} the king of Israel replied to Jehoshaphat, ``but I hate him because he never prophesies anything good about me. Instead, he prophesies evil. He is Imla's son Micaiah.''

But Jehoshaphat rebuked Ahab, ``Kings\fnote{\fbackref{22:8} Lit. \fbib{The king}} should never talk like that.''

\v{9}Nevertheless, the king of Israel called one of his officers and ordered him, ``Bring me Imla's son Micaiah quickly.''

\v{10}Now the king of Israel and King Jehoshaphat of Judah were each sitting on their respective thrones, arrayed in their robes, on the threshing floor at the entrance to the city gate of Samaria, and all of the prophets were prophesying in front of them. \v{11}Chenaanah's son Zedekiah made iron horns for himself and told them, ``This is what the \divine{Lord} says, `With these horns you are to gore the Arameans until they are eliminated!'\,''

\v{12}All the other prophets were saying similar things, like ``Go up to Ramoth-gilead and you will be successful, because the \divine{Lord} will hand it over to the king!''
\passage{Micaiah Predicts Failure}
\passageinfo{(2 Chronicles 18:12-27)}

\v{13}Meanwhile, the messenger who had gone off to summon Micaiah advised him, ``Look, everything that the other prophets were saying was unanimously favorable to the king. So please, cooperate with them and speak favorably.''

\v{14}``As the \divine{Lord} lives,'' Micaiah replied, ``I'll say what my God tells me to say.''

\v{15}When Micaiah\fnote{\fbackref{22:15} Lit. \fbib{he}} approached the king, the king asked him, ``Micaiah, should we go to war against Ramoth-gilead, or should I not?''

``Go to war,'' Micaiah\fnote{\fbackref{22:15} Lit. \fbib{he}} replied, ``and you will be successful, because the \divine{Lord} will hand it over to the king!''

\v{16}When he heard this, the king asked him, ``How many times do I have to make you swear to tell me nothing but the truth? Now do it in the name of the \divine{Lord}!''

\v{17}So Micaiah replied:

\begin{poetry}
\poeml ``I saw all of Israel \\
\poemll    scattered on the mountains \\
\poemlll       like sheep without a shepherd. \\
\poeml And the \divine{Lord} told me, \\
\poemll    `These have no master, \\
\poemlll       so let them each return to his own home in peace.'\,''
\end{poetry}

\v{18}Then the king of Israel told Jehoshaphat, ``Didn't I tell you that he wouldn't prophesy anything good about me, but only evil?''

\v{19}But Micaiah responded, ``Therefore, listen to what the \divine{Lord} has to say. I saw the \divine{Lord}, sitting on his throne, and the entire Heavenly Army was standing around him on his right hand and on his left hand.

\v{20}``The \divine{Lord} asked, `Who will tempt King Ahab of Israel to attack Ramoth-gilead, so that he will die there?' And one was saying one thing and one was saying another.

\v{21}``But then a spirit approached, stood in front of the \divine{Lord}, and said, `I will entice him.'

\v{22}``And the \divine{Lord} asked him, `How?'

```I will go,' he announced, `and I will be a deceiving spirit in the mouth of all of his prophets!'

``So the \divine{Lord} said, `You're just the one to deceive him. You will be successful. Go and do it.'

\v{23}``Now therefore, listen! The \divine{Lord} has placed a lying spirit in the mouth of all of these prophets of yours, because the \divine{Lord} has determined to bring disaster upon you.''

\v{24}Right then, Chenaanah's son Zedekiah approached Micaiah and struck him on the cheek. Then he asked him, ``How did the Spirit of the \divine{Lord} move from me to speak to you?''

\v{25}Micaiah replied, ``You'll see how when the day comes that you run away to hide yourself in a closet!''

\v{26}Then the king of Israel ordered, ``Take Micaiah and place him in the custody of Amon, the city governor. Hand him over to Joash, the king's son. \v{27}Give him this order: `Place him in prison on survival rations of bread and water only until I come back safely.'\,''

\v{28}``If you return alive,'' Micaiah responded, ``then the \divine{Lord} has not spoken by me.'' Then he added, ``Listen, all you people!''
\passage{Ahab Dies at Ramoth-gilead}
\passageinfo{(2 Chronicles 18:28-34)}

\v{29}So the king of Israel and King Jehoshaphat of Judah both attacked Ramoth-gilead. \v{30}The king of Israel suggested to Jehoshaphat, ``I'll go into battle in disguise, but you keep your royal uniform on.'' So the king of Israel disguised himself and they both went into the battle.

\v{31}Meanwhile, the king of Aram had issued these orders to 32 of his chariot commanders: ``Don't attack unimportant soldiers or ranking officers. Go after only the king of Israel.''

\v{32}So when the chariot commanders observed Jehoshaphat, they said by mistake,\fnote{\fbackref{22:32} The Heb. lacks \fbib{by mistake}} ``It's the king of Israel!'' and they turned aside to attack him. But Jehoshaphat cried out. \v{33}When the chariot commanders saw that their target\fnote{\fbackref{22:33} Lit. \fbib{that he}} was not the king of Israel, they stopped pursuing him.

\v{34}Meanwhile, somebody drew his bow aimlessly and struck the king of Israel between the scales where his armor breastplates joined, so he instructed his chariot driver, ``Turn around and take me out of the battle, because I've been severely wounded.'' \v{35}The battle continued on for the rest of the day while the king of Israel was propped up in front of the Arameans until the sun set, at which time he died. The blood from Ahab's wound ran down into the bottom of the chariot.

\v{36}As the day drew to a close, this order was circulated throughout the army telling the soldiers, ``Everybody go back to his city and to his own land.'' \v{37}So the king died and was brought back to Samaria, and they buried the king in Samaria. \v{38}They washed the chariot by the reservoir of Samaria, and the dogs licked up his blood near where the prostitutes went to bathe, in keeping with the message that the \divine{Lord} had spoken.

\v{39}Now as to the rest of Ahab's accomplishments, everything that he undertook, the ivory palace he built, and the cities that he built, they are written in the Book of the Chronicles of the Kings of Israel, are they not? \v{40}That's how Ahab died, just as his ancestors had, and his son Ahaziah became king in his place.
\passage{Jehoshaphat Reigns over Judah}

\v{41}Asa's son Jehoshaphat became king over Judah during the fourth year of the reign of\fnote{\fbackref{22:41} The Heb. lacks \fbib{the reign of}} King Ahab of Israel. \v{42}Jehoshaphat was 35 years old when he became king. He reigned 25 years in Jerusalem. His mother's name was Azubah. She was the daughter of Shilhi. \v{43}He lived like his father Asa and never abandoned that life. He did what the \divine{Lord} considered to be right. Nevertheless, the high places were not demolished, and the people continued to sacrifice and burn incense on the high places.\fnote{\fbackref{22:43} This last sentence of v 43 is v44 in MT, v44 is v45 in MT, and so through the rest of the chapter.} \v{44}Jehoshaphat also made a peace treaty with the king of Israel.

\v{45}Now the rest of Jehoshaphat's accomplishments, the power that he demonstrated, and how he waged war are written in the book of the Chronicles of the Kings of Judah, are they not? \v{46}He also eliminated the male cult prostitutes who still remained from the time of his father Asa.

\v{47}There was no king reigning in Edom; there was only a stand-in\fnote{\fbackref{22:47} Or \fbib{deputy}} king. \v{48}Jehoshaphat had ocean-going vessels from Tarshish sail to Ophir\fnote{\fbackref{22:48} Or \fbib{a source of fine gold}; cf. 1Chr 29:4} for gold, but they never made it because they were shipwrecked at Ezion-geber. \v{49}Ahab's son Ahaziah had offered to go. ``Let my servants go with your servants in the ships!'' he said. But Jehoshaphat was not willing. \v{50}Later, Jehoshaphat died, as did his ancestors, and he was buried alongside his ancestors in the City of David. Jehoram his son became king in his place.
\passage{Ahaziah Reigns over Israel}

\v{51}Ahab's son Ahaziah became king over Israel in Samaria in the seventeenth year of King Jehoshaphat of Judah. He reigned for two years over Israel. \v{52}He practiced what the Lord considered to be evil by living life like his father and mother did. He lived like Nebat's son Jeroboam, who led Israel into sin. \v{53}He served Baal, worshipped him, and provoked the \divine{Lord} God of Israel to anger, in accordance with everything his father had done.
