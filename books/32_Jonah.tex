\bookheader{Jonah}
\labelbook{Jonah}

\bookpretitle{The Book of the Prophet}
\booktitle{Jonah}

\labelchapt{1}
\passage{Jonah is Called to Go to Nineveh}

\chapt{1}
\v{1}Now this message from the \divine{Lord} came to Amittai's son Jonah:\fnote{\fbackref{1:1} The Heb. name \fbib{Jonah} means \fbib{dove}} \v{2}``Get up and go to Nineveh, that great city! Then cry out in protest\fnote{\fbackref{1:2} The Heb. lacks \fbib{in protest}} against it, because their evil has come to my attention.''\fnote{\fbackref{1:2} Lit. \fbib{has come up before me}}
\passage{Jonah Runs from God's Call}

\v{3}But Jonah got up and fled from the \divine{Lord} to Tarshish.\fnote{\fbackref{1:3} I.e. a city far to the West} He went down to Joppa, secured passage on a ship bound for Tarshish, paid the fare, and boarded, intending to go with the mariners\fnote{\fbackref{1:3} Lit. \fbib{with them}} to Tarshish to escape from the \divine{Lord}. \v{4}Then the \divine{Lord} sent\fnote{\fbackref{1:4} Lit. \fbib{threw}} a great wind over the sea, and a severe storm broke out. It seemed as if the ship were\fnote{\fbackref{1:4} Or \fbib{out so that the ship seemed that it was}} about to break up. \v{5}At this point the mariners became terrified, and each man cried out to his gods. They began to throw the cargo into the sea in order to lighten the vessel. But Jonah had gone down into the vessel's hold, had lain down, and was fast asleep. \v{6}So the captain approached him, and told him, ``What are you doing asleep? Get up! Call on your gods! Maybe your\fnote{\fbackref{1:6} The Heb. lacks \fbib{your}} god will think about us so we won't die!''

\v{7}Meanwhile, each crewman told another, ``Come on! Let's cast lots to find out whose fault it is that we're in this trouble.'' So they cast lots, and the lot indicated Jonah! \v{8}So they interrogated him: ``Tell us, why has this trouble come upon us? What's your occupation? Where'd you come from? What's your home country? What's your nationality?''

\v{9}``I'm a Hebrew,'' he replied, ``and I'm afraid of the \divine{Lord} God of heaven, who made the sea---along with the dry land!''

\v{10}In mounting terror, the men asked him, ``What have you done?'' The men were aware that he was fleeing from the \divine{Lord}, because he had admitted this to them.
\passage{Jonah is Thrown Overboard}

\v{11}Because the sea was growing more and more stormy, they asked him, ``What do we have to do to you so the sea will calm down for us?''

\v{12}Jonah\fnote{\fbackref{1:12} Lit. \fbib{He}} told them, ``Pick me up and toss me into the sea. Then the sea will calm down for you, because I know that it's my fault that this mighty storm has come\fnote{\fbackref{1:12} The Heb. lacks \fbib{has come}} upon you.'' \v{13}Even so, the crewmen rowed hard to bring the ship toward dry land, but they were unsuccessful, because the sea was growing more and more stormy.

\v{14}At last they cried out to the \divine{Lord}, ``Please, \divine{Lord}, do not let us perish because of this man's life, and do not hold us responsible for innocent blood, because you, \divine{Lord}, have done what pleased you.'' \v{15}So they picked up Jonah and tossed him into the sea, and the sea stopped raging. \v{16}Then the men feared the \divine{Lord} greatly, offered a sacrifice to the \divine{Lord}, and made vows.

\v{17}\fnote{\fbackref{1:17} This vs. is 2:1 in MT}Now the \divine{Lord} had prepared a large sea creature\fnote{\fbackref{1:17} Lit. \fbib{fish}, and so throughout the book} to swallow Jonah, and Jonah was inside the sea creature for three days and three nights.
\labelchapt{2}
\passage{Jonah's Prayer for Deliverance}

\chapt{2}
\v{1}\fnote{\fbackref{2:1} 2:1 is 2:2 in MT, 2:2 is 2:3 in MT, and so through vs. 10}Then Jonah prayed to the \divine{Lord} his God from inside the sea creature. \v{2}He said:

\begin{poetry}
\poeml ``I called out to the \divine{Lord} from the midst of affliction directed at me,\fnote{\fbackref{2:2} Lit. \fbib{affliction to me}} \\
\poemll    and he answered me. \\
\poeml From the depths\fnote{\fbackref{2:2} Lit. \fbib{belly}} of death\fnote{\fbackref{2:2} Heb. \fbib{Sheol}; i.e. the realm of the dead} I cried out for help; \\
\poemll    and you heard my cry.\fnote{\fbackref{2:2} Or \fbib{voice}} \\
\poeml \v{3}You cast me into the deep--- \\
\poemll    into the heart of the sea. \\
\poeml Flood waters engulfed me. \\
\poemll    All your breakers and your waves swirled over me. \\
\poeml \v{4}So I told myself,\fnote{\fbackref{2:4} Or \fbib{I thought}} `I have been driven away from you.\fnote{\fbackref{2:4} Lit. \fbib{from your attention}} \\
\poemll    How\fnote{\fbackref{2:4} Lit. \fbib{Indeed, surely}} will I again gaze on your holy Temple?' \\
\poeml \v{5}Flood waters encompassed me, \\
\poemll    the deep surrounded me \\
\poemlll       while seaweed wrapped around my head. \\
\poeml \v{6}I sank to the roots of the mountains; \\
\poemll    the earth's prison\fnote{\fbackref{2:6} The Heb. lacks \fbib{prison}} bars closed\fnote{\fbackref{2:6} The Heb. lacks \fbib{closed}} around me forever. \\
\poemlll       Yet you resurrect the dead\fnote{\fbackref{2:6} Lit. \fbib{you bring life up}} from the Pit,\fnote{\fbackref{2:6} I.e. the place of punishment in the afterlife} \divine{Lord} my God! \\
\poeml \v{7}``As my life was fading away, \\
\poemll    I remembered the \divine{Lord}; \\
\poemlll       and my prayer came to you in your holy Temple. \\
\poeml \v{8}Those who cling to vain idols \\
\poemll    leave behind the gracious love that could have been theirs.\fnote{\fbackref{2:8} Or \fbib{leave behind their gracious love}} \\
\poeml \v{9}But as for me, with a voice of thanksgiving I will sacrifice to you; \\
\poemll    what I have vowed I will pay. \\
\poeml Deliverance\fnote{\fbackref{2:9} Or \fbib{Salvation}} is the \divine{Lord}'s!''
\end{poetry}

\v{10}Then the \divine{Lord} spoke to the sea creature, and it spewed Jonah onto the dry land.
\labelchapt{3}
\passage{The \divine{Lord} Again Calls Jonah to Go to Nineveh}

\chapt{3}
\v{1}This message from the \divine{Lord} came to Jonah a second time: \v{2}``Get up and go to Nineveh, that great city, and proclaim to it the message that I tell you.'' \v{3}So Jonah got up and went to Nineveh to do what the \divine{Lord} had ordered.

Now Nineveh was a very large city,\fnote{\fbackref{3:3} Lit. \fbib{a great city of God}; i.e. a city of enormous size} requiring\fnote{\fbackref{3:3} The Heb. lacks \fbib{requiring}} a three-day journey to cross through it.\fnote{\fbackref{3:3} The Heb. lacks \fbib{to cross through it}} \v{4}As Jonah started into the city on the first day's journey, he proclaimed the message, ``40 days more and Nineveh will be overthrown!''
\passage{The City of Nineveh Repents}

\v{5}The people of Nineveh believed God. They called for a fast and put on sackcloth, from the greatest of them to the least important. \v{6}When the message reached the king of Nineveh, he got up from his throne, removed his royal garments, covered himself with sackcloth, and sat down in ashes. \v{7}Then he had this proclamation published throughout Nineveh:

\begin{poetry}
\poeml ``By decree of the king and his nobles: \\
\poeml No man or animal, herd or flock, is to taste anything, graze, or drink water. \v{8}Instead, let both man and animal clothe themselves with sackcloth and cry out to God forcefully. Let every person turn from his evil ways and from his tendency to do violence.\fnote{\fbackref{3:8} Lit. \fbib{from the violence that is in their palms}} \v{9}Who knows but that God may relent, have compassion, and turn from his fierce anger, so that we are not exterminated?''
\end{poetry}

\v{10}God took note of what they did---that they turned from their evil ways. Because God relented concerning the trouble about which he had warned them, he did not carry it out.
\labelchapt{4}
\passage{Jonah's Anger at God's Kindness}

\chapt{4}
\v{1}Greatly displeased, Jonah flew into a rage. \v{2}So he prayed to the \divine{Lord}, ``\divine{Lord}, isn't this what I said while I was still in my home country? That's why I fled previously to Tarshish, because I knew you're a compassionate God, slow to anger, overflowing with gracious love, and reluctant\fnote{\fbackref{4:2} Or \fbib{sorrowful}} to send trouble. \v{3}Therefore, \divine{Lord}, please kill me, because it's better for me to die than to live!''

\v{4}The \divine{Lord} replied, ``Does being angry make you right?''
\passage{Jonah's Discouragement}

\v{5}Then Jonah left the city and sat down on the eastern side.\fnote{\fbackref{4:5} Lit. \fbib{down east of the city}} There he made a shelter for himself and sat down under its shade to see what would happen to the city. \v{6}The \divine{Lord} God prepared a vine plant,\fnote{\fbackref{4:6} Or \fbib{castor bean plant}; or \fbib{gourd}; and so throughout the chapter} and it grew over Jonah to shade his head and provide relief from his misery. Jonah was happy---indeed, he was ecstatic---about the vine plant. \v{7}But at dawn the next day, God provided a worm that attacked the vine plant so that it withered away. \v{8}When the sun rose, God prepared a harsh east wind. The sun beat down on Jonah's head, he became faint, and he begged to die. ``It is better for me to die than to live!'' he said.

\v{9}Then God asked Jonah, ``Is your anger about the vine plant justified?''

And he answered, ``Absolutely! I'm so angry I could die!''

\v{10}But the \divine{Lord} asked, ``You cared about a vine plant that you neither worked on nor cultivated? A vine plant that grew up overnight and died overnight? \v{11}So why shouldn't I be concerned about Nineveh, that great city, in which there are more than 120,000 human beings who do not know their right hand from their left,\fnote{\fbackref{4:11} I.e. young children or infants} as well as a lot of livestock?
