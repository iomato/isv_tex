\bookheader{Isaiah}
\labelbook{Isa}

\bookpretitle{The Book of the Prophet}
\booktitle{Isaiah}

\labelchapt{1}
\passage{The Vision of Isaiah}

\chapt{1}
\v{1}This\fnote{This book has been translated from the Great Isaiah Scroll (1QIsa\textsuperscript{a}) of the DSS. The MT, LXX, Syr, Targ, and other ancient texts are compared in footnotes where they may vary from 1QIsa\textsuperscript{a} and other DSS mss. Some of these ancient readings were incorporated instead of the DSS Isaiah text.} is the vision that Amoz's son Isaiah\fnote{\fbackref{1:1} The Heb. name \fbib{Isaiah} means \fbib{The \divine{Lord} has saved}} had about Judah and Jerusalem during the reigns\fnote{\fbackref{1:1} Lit. \fbib{days}} of Uzziah, Jotham, Ahaz, and Hezekiah, kings of Judah.
\passage{Rebellious Judah}

\begin{poetry}
\poeml \v{2}Listen, you heavens, \\
\poemll    and let the\fnote{\fbackref{1:2} So 1QIsa\textsuperscript{a}; the Heb. lacks \fbib{the}} earth pay attention, \\
\poemlll       because the \divine{Lord} has spoken: \\
\poeml ``I reared children \\
\poemll    and brought them to adulthood, \\
\poemlll       but then they rebelled against me. \\
\poeml \v{3}The ox knows its owner, \\
\poemll    and the donkey its master's feeding trough, \\
\poeml but\fnote{\fbackref{1:3} So 4QIsa\textsuperscript{j}; 1QIsa\textsuperscript{a} MT lack \fbib{but}} Israel doesn't know, \\
\poemll    and\fnote{\fbackref{1:3} So 4QIsa\textsuperscript{j}; 1QIsa\textsuperscript{a} MT lack \fbib{and}} my people don't understand. \\
\poeml \v{4}``Oh, you sinful nation! \\
\poemll    You people burdened down by iniquity! \\
\poeml You offspring of those who keep practicing what is evil! \\
\poemll    You corrupt children! \\
\poeml ``They've abandoned the \divine{Lord}; \\
\poemll    they've despised the Holy One of Israel; \\
\poemlll       in their estrangement, they've walked away from me.\fnote{\fbackref{1:4} Lit. \fbib{they've gone back}} \\
\poeml \v{5}``Why will you still be struck down? \\
\poemll    Why will you continue to rebel? \\
\poeml Your whole head is sick, \\
\poemll    and your whole heart is faint. \\
\poeml \v{6}From the sole of your foot to the top of your head, \\
\poemll    there's no soundness evident\fnote{\fbackref{1:6} Lit. \fbib{soundness in it}}--- \\
\poeml only bruises, sores, \\
\poemll    and festering wounds \\
\poeml that haven't been cleaned out, \\
\poemll    bandaged, or treated\fnote{\fbackref{1:6} Or \fbib{softened}} with oil.''
\passage{God's Diagnosis}
\poeml \v{7}``Your country lies desolate; \\
\poemll    your cities have been incinerated. \\
\poeml Before your very eyes, \\
\poemll    foreigners are devouring your land--- \\
\poeml they've brought devastation on it,\fnote{\fbackref{1:7} So 1QIsa\textsuperscript{a}; MurIsa MT LXX read \fbib{Devastated}} \\
\poemll    while the land is\fnote{\fbackref{1:7} DSS MT lack \fbib{the land is}} overthrown by foreigners. \\
\poeml \v{8}``The daughter of Zion is left abandoned, \\
\poemll    like a booth in a vineyard, \\
\poeml like a hut in a cucumber field, \\
\poemll    or like a city under siege. \\
\poeml \v{9}If the Lord of the Heavenly Armies \\
\poemll    hadn't left us a few survivors, \\
\poeml we would be like Sodom; \\
\poemll    we would be like Gomorrah. \\
\poeml \v{10}``Listen to what the \divine{Lord} says, \\
\poemll    you rulers of Sodom, \\
\poeml and\fnote{\fbackref{1:10} So 1:10 1QIsa\textsuperscript{a} Syr; MT LXX Targ Vulgate lack \fbib{and}} pay attention to the teaching of our God, \\
\poemll    you people of Gomorrah! \\
\poeml \v{11}``How do your voluminous sacrifices benefit me?'' \\
\poemll    the \divine{Lord} is asking. \\
\poeml ``I've had enough of burnt offerings of rams \\
\poemll    and the fat of well-fed beasts. \\
\poeml I don't enjoy the blood of bulls, \\
\poemll    lambs, or goats. \\
\poeml \v{12}``When you come to present yourselves in my presence,\fnote{\fbackref{1:12} Lit. \fbib{come for my face to appear}} \\
\poemll    who has required you \\
\poemlll       to trample on my courts? \\
\poeml \v{13}Stop bringing useless offerings! \\
\poemll    Incense is detestable to me, \\
\poeml as are your New Moons, Sabbaths, and calling of convocations. \\
\poemll    I cannot stand iniquity within\fnote{\fbackref{1:13} Lit. \fbib{and}} a solemn assembly. \\
\poeml \v{14}As for your New Moons and your appointed festivals, \\
\poemll    I abhor\fnote{\fbackref{1:14} Lit. \fbib{festivals, my soul abhors}} them. \\
\poeml They've become a burden to me; \\
\poemll    I've grown weary of carrying that burden.\fnote{\fbackref{1:14} Lit. \fbib{carrying them}} \\
\poeml \v{15}When you spread out your hands in prayer,\fnote{\fbackref{1:15} DSS MT lack \fbib{in prayer}} \\
\poemll    I'll hide my eyes from you. \\
\poeml Even though you pray repeatedly, \\
\poemll    I won't listen. \\
\poeml Your hands are full of blood, \\
\poemll    your fingers drenched\fnote{\fbackref{1:15} DSS lack \fbib{drenched}} with iniquity.''\fnote{\fbackref{1:15} MT 4QIsa\textsuperscript{f} lack this line}
\passage{An Invitation to Reconciliation}
\poeml \v{16}``Wash yourselves, \\
\poemll    and\fnote{\fbackref{1:16} So 1QIsa\textsuperscript{a}; 4QIsa\textsuperscript{f} MT lack \fbib{and}} make yourselves clean; \\
\poeml remove your evil behavior \\
\poemll    from my presence; \\
\poemlll       stop practicing what is evil. \\
\poeml \v{17}Learn to practice what is good; \\
\poemll    seek justice, \\
\poeml alleviate oppression,\fnote{\fbackref{1:17} Or \fbib{rescue the oppressed}} \\
\poemll    defend orphans\fnote{\fbackref{1:17} Or \fbib{defend the fatherless}} in court, \\
\poemlll       and\fnote{\fbackref{1:17} DSS MT lack \fbib{and}} plead the widow's case. \\
\poeml \v{18}``Please come, \\
\poemll    and let's reason together,'' implores the \divine{Lord}. \\
\poeml ``Even though your\fnote{\fbackref{1:18} Lit. \fbib{your} (pl.)} sins are like scarlet, \\
\poemll    they'll be white like snow. \\
\poeml Though they're like crimson,\fnote{\fbackref{1:18} So 1QIsa\textsuperscript{a} LXX; 4QIsa\textsuperscript{f} MT read \fbib{they're red like crimson}} \\
\poemll    they'll become like wool. \\
\poeml \v{19}If you're willing and obedient, \\
\poemll    you'll eat the best that the land produces; \\
\poeml \v{20}but\fnote{\fbackref{1:20} So 1QIsa\textsuperscript{a} LXX Targ Syr; 4QIsa\textsuperscript{f} lacks \fbib{but}} if you refuse and rebel, \\
\poemll    you'll be devoured by the sword,\fnote{\fbackref{1:20} So 1QIsa\textsuperscript{a} Targ Syr; LXX reads \fbib{the sword will devour you}} \\
\poemlll       because the \divine{Lord}\fnote{\fbackref{1:20} Lit. \fbib{\divine{Lord}'s mouth}} has spoken.''
\passage{Diagnosis and Judgment}
\poeml \v{21}``How the faithful city \\
\poemll    has become a whore, \\
\poemlll       she who used to be filled with justice! \\
\poeml Righteousness used to reside within her, \\
\poemll    but now only murderers live there. \\
\poeml \v{22}Your silver has\fnote{\fbackref{1:22} So MT; 1QIsa\textsuperscript{a} reads \fbib{have}} become dross, \\
\poemll    your best wine is diluted with water. \\
\poeml \v{23}Your princes are rebels \\
\poemll    and companions of thieves. \\
\poeml All of them are lovers of\fnote{\fbackref{1:23} So 1QIsa\textsuperscript{a} LXX; MT reads \fbib{Everyone loves}} bribes \\
\poemll    and are runners\fnote{\fbackref{1:23} So 1QIsa\textsuperscript{a} LXX; MT reads \fbib{and run}} after gifts. \\
\poeml They don't bring justice to orphans,\fnote{\fbackref{1:23} Or \fbib{to the fatherless}} \\
\poemll    and the widow's case never comes up for review in their court.''\fnote{\fbackref{1:23} Lit. \fbib{comes before them}}
\end{poetry}
\passage{Restoration and Redemption}

\begin{poetry}
\poeml \v{24}Therefore this is what the Lord \divine{God} of the Heavenly Armies, \\
\poemll    the one who is Israel's Mighty One, declares: \\
\poeml ``Now I'll get relief from his\fnote{\fbackref{1:24} So 1QIsa\textsuperscript{a}; 4QIsa\textsuperscript{f} MT LXX Vulgate read \fbib{my}} enemies \\
\poemll    and avenge myself on his\fnote{\fbackref{1:24} So 1QIsa\textsuperscript{a}; LXX MT read \fbib{my}} foes. \\
\poeml \v{25}When I turn\fnote{\fbackref{1:25} So 1QIsa\textsuperscript{a}; LXX MT read \fbib{And let me turn}} my attention to you,\fnote{\fbackref{1:25} Lit. \fbib{my hand against you}} \\
\poemll    I'll refine your dross as in a furnace.\fnote{\fbackref{1:25} 1QIsa\textsuperscript{a} lacks \fbib{as in a furnace}; MT reads \fbib{as lye}} \\
\poemlll       I'll remove\fnote{\fbackref{1:25} So 1QIsa\textsuperscript{a} 4QIsa\textsuperscript{f}; MT reads \fbib{Let me remove}} all your alloy. \\
\poeml \v{26}Let me restore\fnote{\fbackref{1:26} 1QIsa\textsuperscript{a} MT; 4QIsa\textsuperscript{f} reads \fbib{I will restore}} your judges as at the first, \\
\poemll    and your counselors as at the beginning. \\
\poeml Afterward you'll be called `The Righteous City' \\
\poemll    and `The Faithful City of Zion'.\fnote{\fbackref{1:26} So LXX; the Heb. lacks \fbib{Zion}} \\
\poeml \v{27}``Zion\fnote{\fbackref{1:27} So MT; 1QIsa\textsuperscript{a} reading unclear} will be redeemed by justice, \\
\poemll    and her repentant ones\fnote{\fbackref{1:27} So 1QIsa\textsuperscript{a} MT 4QIsa\textsuperscript{f}; LXX reads \fbib{her captivity}} by righteousness. \\
\poeml \v{28}Rebels and sinners will be broken together, \\
\poemll    and those who forsake the \divine{Lord} will be consumed. \\
\poeml \v{29}They'll be ashamed of the oak trees \\
\poemll    that you desired; \\
\poeml and you'll blush because of the gardens \\
\poemll    that you have chosen. \\
\poeml \v{30}You'll be like an oak \\
\poemll    whose leaf is withering, \\
\poemlll       like an unwatered garden. \\
\poeml \v{31}Your\fnote{\fbackref{1:31} Lit. \fbib{Your} (pl.); so 1QIsa\textsuperscript{a} Vulgate. \fbib{The}} strong one\fnote{\fbackref{1:31} LXX MT read \fbib{Their strength}} will be like tinder, \\
\poemll    and your work\fnote{\fbackref{1:31} So 1QIsa\textsuperscript{a}; MT reads \fbib{their work}; LXX reads \fbib{their deeds}} a spark; \\
\poeml both of them will burn together, \\
\poemll    with no one to quench the flames that burn\fnote{\fbackref{1:31} DSS MT lack \fbib{the flames that burn}} them.''
\end{poetry}
\labelchapt{2}
\passage{A Message for Judah and Jerusalem}

\chapt{2}
\v{1}The message that Amoz's son Isaiah received\fnote{\fbackref{2:1} Lit. \fbib{observed}} concerning Judah and Jerusalem:

\begin{poetry}
\poeml \v{2}``It will come about in the last days \\
\poemll    that the mountain that is the \divine{Lord}'s Temple will be established \\
\poemlll       as the highest of mountains,\fnote{\fbackref{2:2} So 4QIsa\textsuperscript{e} MT LXX; 1QIsa\textsuperscript{a} lacks \fbib{of mountains}} \\
\poeml and will be raised above the hills; \\
\poemll    all the nations will stream to\fnote{\fbackref{2:2} Or \fbib{will flow over}; so 1QIsa\textsuperscript{a} LXX; 4QIsa\textsuperscript{e} 4QIsa\textsuperscript{f} MT LXX read \fbib{will travel to}; cf. Mic 4:1} it. \\
\poeml \v{3}Many groups of people\fnote{\fbackref{2:3} Lit. \fbib{Many peoples}} will come, commenting, \\
\poemll    ``Come! Let's go up to\fnote{\fbackref{2:3} So 1QIsa\textsuperscript{a} 4QIsa\textsuperscript{f}; 4QIsa\textsuperscript{e} MT read \fbib{to the \divine{Lord}'s mountain, to}} the Temple of the God of Jacob, \\
\poeml that they\fnote{\fbackref{2:3} So 1QIsa\textsuperscript{a}; cf. LXX, Mic 4:2; 4QIsa\textsuperscript{e} MT LXX read \fbib{he}} may teach us his ways. \\
\poemll    Then let's walk in his paths.''
\passage{The Rule of God from Jerusalem}
\poeml ``Instruction\fnote{\fbackref{2:3} Or \fbib{For the law}} will proceed from Zion, \\
\poemll    and the word of the \divine{Lord} from Jerusalem. \\
\poeml \v{4}He will judge between the nations, \\
\poemll    and will render verdicts\fnote{\fbackref{2:4} Or \fbib{judgment}} for the benefit of many.\fnote{\fbackref{2:4} Lit. \fbib{many peoples}} \\
\poeml ``They will beat their swords into plowshares, \\
\poemll    and their spears into pruning hooks; \\
\poeml nations will not raise swords against nations, \\
\poemll    and they will not learn warfare anymore. \\
\poeml \v{5}``You house of Jacob!
\end{poetry}

Come! Let's live\fnote{\fbackref{2:5} Lit. \fbib{walk}} in the \divine{Lord}'s light.

\begin{poetry}
\poeml \v{6}For you have rejected your people, \\
\poeml the house of Jacob, \\
\poeml because they are filled with practices learned\fnote{\fbackref{2:6} 1QIsa\textsuperscript{a} MT lack \fbib{with practices learned}} from the East \\
\poeml and they are fortune-tellers like the Philistines. \\
\poeml They cut deals\fnote{\fbackref{2:6} Lit. \fbib{They shake hands}} with foreigners.\fnote{\fbackref{2:6} Lit. \fbib{with descendants of foreigners}} \\
\poeml \v{7}Their land is filled with silver and gold, \\
\poeml and there is no end to their treasures; \\
\poeml their land is filled with horses, \\
\poeml and there is no end to their chariots. \\
\poeml \v{8}Their land is filled with idols; \\
\poeml they bow down to the work of their hands, \\
\poemlll       to what their own fingers have made. \\
\poeml \v{9}``So mankind is humbled, \\
\poemll    each human being is brought low, \\
\poemlll       and\fnote{\fbackref{2:9} So 1QIsa\textsuperscript{a} 4QIsa\textsuperscript{b}; the Heb. lacks \fbib{and}} you won't forgive.''\fnote{\fbackref{2:9} So 1QIsa\textsuperscript{a} 4QIsa\textsuperscript{b}; MT reads \fbib{forgive them}}
\passage{The Coming Day of the \divine{Lord}}
\poeml \v{10}\fnote{\fbackref{2:10} This v. is missing from 1QIsa\textsuperscript{a}} ``Go into the rocks! \\
\poemll    Hide in the dust \\
\poeml to escape\fnote{\fbackref{2:10} Lit. \fbib{from}} the terror of the \divine{Lord} \\
\poemll    and to escape\fnote{\fbackref{2:10} Lit. \fbib{and from}} the glory of his majesty!\fnote{\fbackref{2:10} So MT} \\
\poeml \v{11}The\fnote{\fbackref{2:11} Lit. \fbib{And the}} haughty looks of mankind will be brought low,\fnote{\fbackref{:11} 1QIsa\textsuperscript{a} MT LXX read \fbib{mankind are low}} \\
\poemll    the lofty pride of human beings will be humbled, \\
\poemlll       and the \divine{Lord} alone will be exalted at that time.\fnote{\fbackref{2:11} Lit. \fbib{in that day}} \\
\poeml \v{12}``For the \divine{Lord} of the Heavenly Armies has reserved\fnote{\fbackref{2:12} 1QIsa\textsuperscript{a} MT LXX lack \fbib{reserved}} a time\fnote{\fbackref{2:12} Lit. \fbib{day}} \\
\poemll    to oppose\fnote{\fbackref{2:12} Lit. \fbib{against}} all who are proud and haughty, \\
\poeml and the\fnote{\fbackref{2:12} So 1QIsa\textsuperscript{a}; MT LXX read \fbib{oppose all of the}} self-exalting--- \\
\poemll    they will be humbled. \\
\poeml \v{13}He will take his stand\fnote{\fbackref{2:13} DSS MT LXX lack \fbib{He will take his stand}} against all the cedars\fnote{\fbackref{2:13} I.e. a genus of coniferous evergreen in the family \fbib{Pinaceae}; and so throughout the book} of Lebanon, \\
\poemll    against the proud and self-exalting; \\
\poemlll       and against all the oaks of Bashan; \\
\poeml \v{14}against all the high mountains, \\
\poemll    and against all the lofty hills; \\
\poeml \v{15}against every high tower, \\
\poemll    and against every fortified wall; \\
\poeml \v{16}against all the ships from Tarshish, \\
\poemll    and against all their impressive watercraft. \\
\poeml \v{17}``Humanity's haughtiness will be humbled, \\
\poemll    male arrogance will be brought low, \\
\poeml and the \divine{Lord} alone will be exalted in that day. \\
\poeml \v{18}Their\fnote{\fbackref{2:18} DSS MT LXX lack \fbib{Their}} idols will utterly vanish.\fnote{\fbackref{2:18} So 1QIsa\textsuperscript{a} LXX; MT reads \fbib{He will abolish the idols}} \\
\poeml \v{19}``They will enter caverns in the rocks \\
\poemll    and holes in the ground \\
\poeml to escape\fnote{\fbackref{2:19} Lit. \fbib{from}} the presence of the terror of the \divine{Lord}, \\
\poemll    to escape\fnote{\fbackref{2:19} Lit. \fbib{from}} the splendor of his majesty \\
\poemlll       when he arises to terrify the earth. \\
\poeml \v{20}At that time,\fnote{\fbackref{2:20} Lit. \fbib{day}} mankind will throw \\
\poemll    their silver and gold idols\fnote{\fbackref{2:20} Lit. \fbib{silver idols and gold idols}} \\
\poeml that their fingers have made\fnote{\fbackref{2:20} So 1QIsa\textsuperscript{a}; MT reads \fbib{that they made for themselves}; LXX reads \fbib{that they made}} as objects of worship \\
\poemll    to the moles and to the bats. \\
\poeml \v{21}They will enter caverns in the rocks \\
\poemll    and clefts in the cliffs, \\
\poeml to escape\fnote{\fbackref{2:21} Lit. \fbib{from}} the terror of the \divine{Lord} \\
\poemll    and to escape\fnote{\fbackref{2:21} Lit. \fbib{from}} the splendor of his majesty, \\
\poemlll       when he arises to terrorize the earth. \\
\poeml \v{22}``Stop trusting in human beings, \\
\poemll    whose life breath is in their nostrils, \\
\poemlll       for what are they\fnote{\fbackref{2:22} Lit. \fbib{what is he}} really worth?''\fnote{\fbackref{2:22} LXX lacks this verse}
\end{poetry}
\labelchapt{3}
\passage{Judgment Comes to Judah's Leaders}

\begin{poetry}
\poeml \chapt{3}
\v{1}``Note this! The Lord \divine{God} of the Heavenly Armies \\
\poemll    is taking away from Jerusalem and Judah \\
\poemlll       everything that your society needs---\fnote{\fbackref{3:1} Lit. \fbib{Judah both supply and support}} \\
\poeml all food supplies \\
\poemll    and all water supplies, \\
\poeml \v{2}the mighty man \\
\poemll    and the warrior, \\
\poeml the judge \\
\poemll    and the prophet, \\
\poeml the fortune-teller \\
\poemll    and the elder, \\
\poeml \v{3}the commander of fifty \\
\poemll    and the man of rank, \\
\poeml and the counselor, the expert magician, \\
\poemll    and the medium. \\
\poeml \v{4}``I will make youths their princes, \\
\poemll    and infants will rule over them. \\
\poeml \v{5}People will oppress one another--- \\
\poemll    It will be man against man \\
\poemlll       and neighbor against neighbor. \\
\poeml The young will be disrespectful to the old, \\
\poemll    and the worthless to the honorable. \\
\poeml \v{6}``For a man will grab his brother \\
\poemll    in his own father's house, \\
\poeml and say, `You have a cloak, \\
\poemll    so you be our leader, \\
\poeml and this heap of ruins \\
\poemll    will be under your rule!' \\
\poeml \v{7}``But\fnote{\fbackref{3:7} So 1QIsa\textsuperscript{a} LXX; the Heb. lacks \fbib{But}} at that time,\fnote{\fbackref{3:7} Lit. \fbib{day}} he'll protest!\fnote{\fbackref{3:7} Lit. \fbib{he'll cry out}} \\
\poemll    He'll say, `I won't be your healer. \\
\poeml I have neither food nor clothing in my house! \\
\poemll    You're not going to make me a leader of the people!' \\
\poeml \v{8}``For Jerusalem has stumbled, \\
\poemll    and Judah has fallen, \\
\poeml because what they say and do opposes\fnote{\fbackref{3:8} So 1QIsa\textsuperscript{a}; MT LXX read \fbib{do is towards}} the \divine{Lord}; \\
\poemll    they keep defying him.\fnote{\fbackref{3:8} Lit. \fbib{defying his glorious presence}} \\
\poeml \v{9}``The expressions on their faces give them away.\fnote{\fbackref{3:9} Lit. \fbib{faces bears witness against them}} \\
\poemll    They parade their sin around like Sodom; \\
\poemlll       they don't even try to\fnote{\fbackref{3:9} 1QIsa\textsuperscript{a} MT lack \fbib{try to}} hide it. \\
\poeml How horrible it will be for them, \\
\poemll    because they have brought disaster on themselves!''
\passage{Encouragement to the Righteous}
\poeml \v{10}``Tell\fnote{\fbackref{3:10} So 1QIsa\textsuperscript{a}; the Heb. lacks \fbib{Tell}} the righteous that things will go well, \\
\poemll    because they will enjoy\fnote{\fbackref{3:10} Lit. \fbib{eat}} the fruit of their actions.''
\passage{Warning to the Wicked}
\poeml \v{11}``How terrible it will be for the wicked! \\
\poemll    Disaster is headed their way, \\
\poemlll       because what they did with their hand\fnote{\fbackref{3:11} So 1QIsa\textsuperscript{a}; MT LXX read \fbib{hands}} will be repaid\fnote{\fbackref{3:11} So 1QIsa\textsuperscript{a}; MT reads \fbib{done}} to them. \\
\poeml \v{12}``As for my people, children\fnote{\fbackref{3:12} Or \fbib{youths}} are their oppressors, \\
\poemll    and women rule over them. \\
\poeml My people, your leaders are misleading you--- \\
\poemll    they're giving you confusing directions.''\fnote{\fbackref{3:12} So MT; 1QIsa\textsuperscript{a} reads \fbib{they're devouring your paths}}
\passage{When God Goes to Court}
\poeml \v{13}The \divine{Lord} is taking his place to argue his case; \\
\poemll    he's standing up to judge the people. \\
\poeml \v{14}The \divine{Lord} will go to court\fnote{\fbackref{:14} Lit. \fbib{go into judgment}} \\
\poemll    to oppose\fnote{\fbackref{3:14} Lit. \fbib{with}} the elders and princes of his people: \\
\poeml ``You're the ones who have been devouring the vineyard, \\
\poemll    the plunder of the poor is in your own houses! \\
\poeml \v{15}How dare you crush\fnote{\fbackref{3:15} Lit. \fbib{What do you mean by crushing}} my people \\
\poemll    as you grind down the face of the poor?'' \\
\poemlll       declares the Lord \divine{God} of the Heavenly Armies.\fnote{\fbackref{3:15} So 1QIsa\textsuperscript{a} MT; LXX lacks this line}
\passage{Judgment of Jerusalem's Women}
\poeml \v{16}The \divine{Lord} also says: \\
\poeml ``Because Zion's women are so haughty, \\
\poemll    and walk with outstretched necks, \\
\poeml flirting with their eyes, \\
\poemll    prancing\fnote{\fbackref{3:16} Or \fbib{mincing}} along as they walk, \\
\poemlll       and making tinkling noises with their ankle bracelets,\fnote{\fbackref{3:16} Lit. \fbib{their feet}} \\
\poeml \v{17}therefore the \divine{Lord}\fnote{\fbackref{3:17} So 1QIsa\textsuperscript{a} corrector; 1QIsa\textsuperscript{a} 4QIsa\textsuperscript{b} MT read \fbib{my Lord}; LXX reads \fbib{God}} will afflict sores \\
\poemll    on the heads of Zion's women, \\
\poemlll       and the \divine{Lord}\fnote{\fbackref{3:17} So 1QIsa\textsuperscript{a} corrector MT; 1QIsa\textsuperscript{a} reads \fbib{my Lord}} will expose their private parts.
\end{poetry}

\v{18}``At that time,\fnote{\fbackref{3:18} Lit. \fbib{In that day}} the \divine{Lord}\fnote{\fbackref{3:18} So 1QIsa\textsuperscript{a} LXX; MT 1QIsa\textsuperscript{a} corrector read \fbib{my Lord}} will take away the finery of the ankle bracelets, headbands, crescents, \v{19}pendants, bracelets, veils, \v{20}headdresses, armlets, sashes, perfume boxes, charms, \v{21}signet rings, nose rings, \v{22}fine robes, capes,\fnote{\fbackref{3:22} So 1QIsa\textsuperscript{a}; 4QIsa\textsuperscript{b} MT read \fbib{capes and cloaks}} purses, \v{23}mirrors, linen garments, tiaras, and veils.

\begin{poetry}
\poeml \v{24}``And it will come about that instead of fragrance \\
\poemll    there will be\fnote{\fbackref{3:24} The 1QIsa\textsuperscript{a} lacks \fbib{will be a stench}} a stench; \\
\poeml instead of a belt, a rope; \\
\poemll    instead of well-set hair, baldness; \\
\poeml instead of a fine robe, sackcloth; \\
\poemll    and instead of beauty, shame.\fnote{\fbackref{3:24} So 1QIsa\textsuperscript{a}; MT reads \fbib{burning instead of beauty}; LXX lacks this line} \\
\poeml \v{25}Your men will die violently,\fnote{\fbackref{3:25} Lit. \fbib{will fall by the sword}} \\
\poemll    while your forces\fnote{\fbackref{3:25} So 1QIsa\textsuperscript{a}; MT reads \fbib{force}} fall\fnote{\fbackref{3:25} 1QIsa\textsuperscript{a} MT lack \fbib{fall}} in battle \\
\poeml \v{26}and her gates lament and mourn. \\
\poemll    Ravaged, she will sit on the ground.''
\end{poetry}
\labelchapt{4}

\chapt{4}
\v{1}``At that time,\fnote{\fbackref{4:1} Lit. \fbib{day}} seven women will cling tightly to one man and will make him this offer:\fnote{\fbackref{4:1} Lit. \fbib{will say}} `We'll provide\fnote{\fbackref{4:1} Lit. \fbib{eat}} our own bread. We'll provide our own clothes. Just let us marry you\fnote{\fbackref{4:1} Lit. \fbib{let your name be upon us}} so we won't be stigmatized anymore.'\,''\fnote{\fbackref{4:1} I.e. by appearing to be part of a family}
\passage{The Future Glory of Jerusalem}

\v{2}``At that time,\fnote{\fbackref{4:2} Lit. \fbib{In that day}} the \divine{Lord}'s branch will be beautiful and glorious, and the fruit of the land will be the pride and glory of the survivors of Israel and Judah.\fnote{\fbackref{4:2} So 1QIsa\textsuperscript{a}; MT LXX lack \fbib{and Judah}} \v{3}Whoever\fnote{\fbackref{4:3} Lit. \fbib{It will come about that whoever}} survives in Zion and whoever remains in Jerusalem will be called holy---everyone who has been appointed to survive in Jerusalem--- \v{4}when the \divine{Lord} will have washed away the filth of the women\fnote{\fbackref{4:4} Lit. \fbib{daughters}} of Zion, cleaning up Jerusalem's guilt\fnote{\fbackref{4:4} Lit. \fbib{blood}; i.e. guilt incurred by shedding innocent blood} by a spirit of judgment and a spirit of tempest.\fnote{\fbackref{4:4} So 1QIsa\textsuperscript{a}; MT reads \fbib{of burning}} \v{5}Then the \divine{Lord} will create over the entire site of Mount Zion---including over those who assemble there---a cloud by day\fnote{\fbackref{4:5} So 1QIsa\textsuperscript{a}; MT LXX reads \fbib{day, accompanied by smoke, as well as the brilliance of a flaming fire by night, because over the entire glorious area there will be a canopy and} \fbib{\v{6}a shelter to protect from the heat of the day, and}} \v{6}and also to serve as a refuge and shelter from storms and rain.''
\labelchapt{5}
\passage{The \divine{Lord}'s Vineyard}

\begin{poetry}
\poeml \chapt{5}
\v{1}I will sing\fnote{\fbackref{5:1} So 1QIsa\textsuperscript{a}; MT reads \fbib{Please, let me sing}} for my beloved \\
\poemll    my love-song concerning his vineyard: \\
\poeml ``The one I love had a vineyard \\
\poemll    on a very fertile hill. \\
\poeml \v{2}He plowed its land\fnote{\fbackref{5:2} 1QIsa\textsuperscript{a} MT lack \fbib{its land}} and cleared it of stones. \\
\poemll    Then he planted it with the choicest vines, \\
\poeml built a watchtower in the middle of it, \\
\poemll    and dug a wine vat in it; \\
\poeml He expected\fnote{\fbackref{5:2} Or \fbib{waited for}} it to produce good\fnote{\fbackref{5:2} 1QIsa\textsuperscript{a} MT lack \fbib{good}} grapes, \\
\poemll    but it produced only wild ones.''\fnote{\fbackref{5:2} I.e. grapes unsuitable for wine making} \\
\poeml \v{3}``So now, you inhabitants\fnote{\fbackref{5:3} So 1QIsa\textsuperscript{a}; MT reads \fbib{inhabitant}} of Jerusalem \\
\poemll    and men of Judah, \\
\poeml judge, won't you please, \\
\poemll    between me and my vineyard. \\
\poeml \v{4}What more could I do in\fnote{\fbackref{5:4} So 1QIsa\textsuperscript{a}; MT reads \fbib{for}} my vineyard, \\
\poemll    that I haven't already done? \\
\poeml When I expected it to produce good\fnote{\fbackref{5:4} 1QIsa\textsuperscript{a} MT lack \fbib{good}} grapes, \\
\poemll    why did it yield\fnote{\fbackref{5:4} So 1QIsa\textsuperscript{a}; MT LXX read \fbib{produce}} wild ones?\fnote{\fbackref{5:4} I.e. grapes unsuitable for wine making} \\
\poeml \v{5}``Now, let me tell you, won't you please, \\
\poemll    what I'm going to do to my vineyard. \\
\poeml ``I'm going to take away its protective hedge, \\
\poemll    and it will be devoured;\fnote{\fbackref{5:5} So 1QIsa\textsuperscript{a}; MT \fbib{will be for devouring}; LXX \fbib{will be for plundering}} \\
\poeml I'll break down its wall, \\
\poemll    and it will be trampled. \\
\poeml \v{6}I'll make it a wasteland, \\
\poemll    and it won't be pruned or cultivated. \\
\poeml Instead, briers and thorns will grow up. \\
\poemll    I'll also issue commands to the clouds, \\
\poemlll       that they drop no rain upon it.'' \\
\poeml \v{7}For the vineyard of the \divine{Lord} of the Heavenly Armies \\
\poemll    is the house of Israel, \\
\poeml and the men of Judah \\
\poemll    are the garden in which he delights.\fnote{\fbackref{5:7} So 1QIsa\textsuperscript{a}; MT \fbib{his delightful garden}} \\
\poeml He looked for justice, \\
\poemll    but saw only bloodshed; \\
\poeml he searched\fnote{\fbackref{5:7} 1QIsa\textsuperscript{a} MT lack \fbib{he searched}} for righteousness, \\
\poemll    but heard only an outcry!
\passage{Judgment on Land Barons}
\poeml \v{8}``How terrible it will be for you who join house to house, \\
\poemll    who add field to field, \\
\poeml until there is no more room, \\
\poemll    and you have settled yourselves alone\fnote{\fbackref{5:8} So 1QIsa\textsuperscript{a}; MT reads \fbib{you are made to live alone}} \\
\poemlll       in the middle of the land!'' \\
\poeml \v{9}The \divine{Lord} of the Heavenly Armies has declared this so I could hear it: \\
\poeml ``Surely many houses will become desolate, \\
\poemll    great and beautiful houses, \\
\poemlll       without occupants. \\
\poeml \v{10}For ten acres of vineyard will produce only one bath,\fnote{\fbackref{5:10} I.e. about six gallons} \\
\poemll    and one omer\fnote{\fbackref{5:10} I.e. about ten bushels} of seed\fnote{\fbackref{5:10} 1QIsa\textsuperscript{a} MT lack \fbib{of seed}} will produce only one ephah.''\fnote{\fbackref{5:10} I.e. about one tenth of what was sown}
\passage{Judgment on Alcoholics}
\poeml \v{11}``How terrible it will be for those who rise at dawn \\
\poemll    in order to grab\fnote{\fbackref{5:11} So 1QIsa\textsuperscript{a}; MT LXX \fbib{may run after}} a stiff drink, \\
\poeml for those who stay up late at night \\
\poemll    as wine inflames them! \\
\poeml \v{12}They have the lyre and harp, \\
\poemll    the tambourine and flute, \\
\poemlll       as well as wine at their festivals, \\
\poeml but they don't respect what the \divine{Lord} is doing, \\
\poemll    nor do they consider his actions.\fnote{\fbackref{5:12} Lit. \fbib{consider the work of his hands}} \\
\poeml \v{13}Therefore my people go into exile \\
\poemll    because they lack understanding; \\
\poeml my\fnote{\fbackref{5:13} So 1QIsa\textsuperscript{a}; MT reads \fbib{their}} honored men go hungry, \\
\poemll    and the crowd is parched with thirst. \\
\poeml \v{14}Therefore Sheol's\fnote{\fbackref{5:14} I.e. the realm of the dead} appetite has grown; \\
\poemll    it has opened its mouth beyond limit. \\
\poeml Jerusalem's nobility and her multitudes will go there, \\
\poemll    along with her brawlers and whoever is reveling within her. \\
\poeml \v{15}Humanity is brought low, \\
\poemll    and each one is humbled, \\
\poemlll       while the eyes of the self-exalting are brought low. \\
\poeml \v{16}But the \divine{Lord} of the Heavenly Armies is exalted in justice, \\
\poemll    and the Holy God proves himself to be righteously holy. \\
\poeml \v{17}Then the lambs will graze in their pasture; \\
\poemll    fatlings and foreigners will eat \\
\poemlll       among the waste places of the rich.''
\passage{Judgment on Mockers}
\poeml \v{18}``How terrible it will be for those who parade iniquity with cords of falsehood, \\
\poemll    who draw sin along as\fnote{\fbackref{5:18} 1QIsa\textsuperscript{a} MT lack \fbib{as}} with a cart rope; \\
\poeml \v{19}who say: `Let God\fnote{\fbackref{5:19} Lit. \fbib{him}} be quick, \\
\poemll    let him speed up\fnote{\fbackref{5:19} So 1QIsa\textsuperscript{a}; MT reads \fbib{hurry}} his work \\
\poemlll       so we may see it! \\
\poeml Let it happen! \\
\poemll    let the plan of the Holy One of Israel draw near, \\
\poemlll       so we may recognize it!'\,''
\passage{Judgment on Moral Relativists}
\poeml \v{20}``How terrible it will be for those who call evil good \\
\poemll    and good evil, \\
\poeml who substitute darkness for light \\
\poemll    and light for darkness, \\
\poeml who substitute what is bitter for what is sweet \\
\poemll    and what is sweet for what is bitter!''
\passage{Judgment on the Arrogant}
\poeml \v{21}``How terrible it will be for those who are wise in their own opinion, \\
\poemll    and clever in their own reckoning! \\
\poeml \v{22}``How terrible it will be for those who are heroes at drinking wine, \\
\poemll    and champions in mixing strong drink, \\
\poeml \v{23}who acquit the guilty for a bribe, \\
\poemll    and deprive the innocent of justice!''
\passage{The Effects of Divine Judgment}
\poeml \v{24}Therefore, as flames of fire devour straw, \\
\poemll    as dry grass\fnote{\fbackref{5:24} So MT; 1QIsa\textsuperscript{a} reads \fbib{fire}} collapses in flames, \\
\poeml so their root will be rotten, \\
\poemll    and their blossom will blow away like dust, \\
\poeml because they have rejected the instruction\fnote{\fbackref{5:24} Or \fbib{law}} of the \divine{Lord} of the Heavenly Armies, \\
\poemll    and have despised the word of the Holy One of Israel. \\
\poeml \v{25}Therefore\fnote{\fbackref{5:25} So 1QIsa\textsuperscript{a} MT; LXX reads \fbib{And}} the anger of the \divine{Lord}\fnote{\fbackref{5:25} So 1QIsa\textsuperscript{a} MT; 4QIsa\textsuperscript{b} LXX read \fbib{\divine{Lord} of the Heavenly Armies}} burned against his people, \\
\poemll    so he stretched out his hands\fnote{\fbackref{5:25} So 1QIsa\textsuperscript{a}; MT LXX read \fbib{hand}} against them \\
\poemlll       and afflicted them. \\
\poeml The mountains quaked, \\
\poemll    and their corpses were like refuse \\
\poemlll       in the middle of the streets. \\
\poeml Throughout all of this, his anger has not turned away, \\
\poemll    and his hands are\fnote{\fbackref{5:25} So 1QIsa\textsuperscript{a}; MT LXX read \fbib{his hand is}} still stretched out to attack.\fnote{\fbackref{5:25} DSS MT lack \fbib{to attack}} \\
\poeml \v{26}The \divine{Lord}\fnote{\fbackref{5:26} Lit. \fbib{He}} will signal\fnote{\fbackref{5:26} Lit. \fbib{will send up a signal}} for nations far away, \\
\poemll    whistling for them to come\fnote{\fbackref{5:26} 1QIsa\textsuperscript{a} MT lack \fbib{to come}} \\
\poemlll       from the ends of the earth. \\
\poeml Look how quickly \\
\poemll    and how swiftly they come! \\
\poeml \v{27}No one is weary, no one stumbles,\fnote{\fbackref{5:27} So 1QIsa\textsuperscript{a}; MT reads \fbib{stumbles among them}} \\
\poemll    and no one slumbers or sleeps. \\
\poeml No belt around their waists will come undone, \\
\poemll    nor will their sandal straps be broken. \\
\poeml \v{28}Their arrows are sharp, \\
\poemll    all their bows ready for action.\fnote{\fbackref{5:28} Lit. \fbib{bows bent}} \\
\poeml Their horses' hooves seem like flint, \\
\poemll    and their chariot wheels spin\fnote{\fbackref{5:28} DSS MT lack \fbib{spin}} like a whirlwind. \\
\poeml \v{29}With a roar like a lion, they snarl, \\
\poemll    and like young lions, they growl;\fnote{\fbackref{5:29} So 1QIsa\textsuperscript{a}; MT reads \fbib{Their roaring is like a lion; like young lions they roar. They growl}} \\
\poeml they seize their prey \\
\poemll    and then carry it off, \\
\poemlll       with no one to rescue. \\
\poeml \v{30}They will roar over it\fnote{\fbackref{5:30} I.e. over conquered Judah} at that time,\fnote{\fbackref{5:30} Lit. \fbib{it on that day}} \\
\poemll    like the sea waves roar. \\
\poeml If one surveys the land, watch out! \\
\poemll    There's darkness and distress; \\
\poemlll       even the daylight is darkened by its clouds.
\end{poetry}
\labelchapt{6}
\passage{Holy is the \divine{Lord}}

\chapt{6}
\v{1}In the year that King Uzziah died, I saw the Lord sitting upon his\fnote{\fbackref{6:1} So 1QIsa\textsuperscript{a}; MT LXX read \fbib{a}} throne, high and exalted. The train of his robe filled the Temple. \v{2}The seraphim stood above him. Each had six wings:\fnote{\fbackref{6:2} So 1QIsa\textsuperscript{a}; MT reads \fbib{six wings, six wings}; LXX reads \fbib{six wings and six wings}} with two he covered his face, and with two he covered his feet, and with two he was flying. \v{3}They kept on calling to each other:\fnote{\fbackref{6:3} So 1QIsa\textsuperscript{a}; MT LXX read \fbib{calling and saying}}

\begin{poetry}
\poeml ``Holy, holy, holy\fnote{\fbackref{6:3} So MT LXX; 1QIsa\textsuperscript{a} reads \fbib{Holy, holy}} is the \divine{Lord} of the Heavenly Armies! \\
\poemll    The whole earth is full of his glory!''
\end{poetry}

\v{4}The foundations of the thresholds quaked at the sound of those who kept calling out,\fnote{\fbackref{6:4} So 1QIsa\textsuperscript{a} LXX; MT reads \fbib{of him who called out}} and the Temple was filled with smoke.

\v{5}``How terrible it will be for me!'' I cried, ``because I am ruined! I'm a man with unclean lips, and I live among a people with unclean lips! And my eyes have seen the King, the \divine{Lord} of the Heavenly Armies!''
\passage{The Calling of Isaiah}

\v{6}Then one of the seraphim flew to me, carrying a burning coal in his hand that he had taken from the altar with tongs. \v{7}He touched my mouth and said, ``Look! Now that this has touched your\fnote{\fbackref{6:7} So 1QIsa\textsuperscript{a} MT LXX; 4QIsa\textsuperscript{f} reads \fbib{the}} lips, your guilt is taken away, and your sins\fnote{\fbackref{6:7} So 1QIsa\textsuperscript{a} LXX; MT reads \fbib{sin}} atoned for.''

\v{8}Then I heard the voice of the \divine{Lord} as he was asking, ``Whom will I send? Who will go for us?''

``Here I am!'' I replied. ``Send me.''

\v{9}``Go!'' he responded. ``Tell this people:

\begin{poetry}
\poeml ```Keep on hearing, but do not understand; \\
\poemll    keep\fnote{\fbackref{6:9} So 1QIsa\textsuperscript{a}; MT reads \fbib{and keep}} on seeing, but do not perceive.' \\
\poeml \v{10}Dull the mind\fnote{\fbackref{6:10} Lit. \fbib{Fatten the heart}} of this people, \\
\poemll    deafen their ears, \\
\poemlll       and blind their eyes. \\
\poeml By doing so, they won't see with their eyes, \\
\poemll    hear with their ears, \\
\poeml understand with their minds, \\
\poemll    turn back, \\
\poemlll       and be healed.''
\end{poetry}

\v{11}Then I asked, ``For how long, \divine{Lord}?''\fnote{\fbackref{6:11} So 1QIsa\textsuperscript{a}; MT reads \fbib{Lord}}

He replied:

\begin{poetry}
\poeml ``Until cities lie waste, \\
\poemll    without inhabitants, \\
\poeml and houses without people; \\
\poemll    and the land becomes utterly desolate. \\
\poeml \v{12}Until\fnote{\fbackref{6:12} Lit. \fbib{And}} the \divine{Lord} removes people far away, \\
\poemll    and there are many empty places \\
\poemlll       in the middle of the land. \\
\poeml \v{13}Even though a tenth of its people remain\fnote{\fbackref{6:13} Lit. \fbib{tenth remains}} in it, \\
\poemll    it will once again be burned,\fnote{\fbackref{6:13} Or \fbib{devastated}} \\
\poeml like a terebinth\fnote{\fbackref{6:13} I.e. a sacred tree used for idolatry; cf. Hos 4:13} or an oak tree,\fnote{\fbackref{6:13} Or \fbib{Asherah pole}; i.e. felled oaks used for making idols; cf. Hos 4:13, Isa 44:14} \\
\poemll    the stump of which, though the tree has been\fnote{\fbackref{6:13} 1QIsa\textsuperscript{a} lacks \fbib{though the tree has been}; MT LXX read \fbib{which, when}} felled, \\
\poemlll       still contains holy seed.''\fnote{\fbackref{6:13} So 1QIsa\textsuperscript{a} MT; LXX lacks this line}
\end{poetry}
\labelchapt{7}
\passage{The Message to Ahaz}

\chapt{7}
\v{1}During the reign of Jotham's son Ahaz, Uzziah's grandson, king of Judah, King Rezin of Aram and Remaliah's son Pekah, king of Israel, approached Jerusalem and waged war against it, but they\fnote{\fbackref{7:1} So 1QIsa\textsuperscript{a} LXX; MT reads \fbib{he}} could not mount an attack against it. \v{2}When it was reported to the house of David, ``Aram has joined forces with Ephraim!'' the\fnote{\fbackref{7:2} So 1QIsa\textsuperscript{a}; MT LXX read \fbib{his}} heart of the people of Ahaz\fnote{\fbackref{7:2} So 1QIsa\textsuperscript{a}; MT LXX read \fbib{his heart and the heart of his people}} trembled like forest\fnote{\fbackref{7:2} So 1QIsa\textsuperscript{a}; the Heb. lacks \fbib{forest}} trees in a windstorm.

\v{3}So the \divine{Lord} told Isaiah, ``Go out to meet Ahaz, you and your son Shear-jashub, at the end of the aqueduct of the Upper Pool that proceeds along the highway to Launderer's Field. \v{4}Tell him, `Be careful, be calm, don't be afraid, and don't lose heart because of these two smoldering stumps of torches, that is, because of\fnote{\fbackref{7:4} So 1QIsa\textsuperscript{a}; the Heb. lacks \fbib{because of}} the fierce anger of Rezin, from Aram, and Remaliah's son. \v{5}Aram, Ephraim, and Remaliah's son have plotted this evil against you: \v{6}``Let's go attack Judah, let's terrorize it, and let's conquer it for ourselves. Then we'll install Tabeel's son as king!''\,'

\v{7}`But this is what the Lord \divine{God} has to say:

\begin{poetry}
\poeml ```It won't take place. \\
\poemll    It won't ever happen. \\
\poeml \v{8}Because Aram's head is Damascus, \\
\poemll    and Rezin is its king,\fnote{\fbackref{7:8} Lit. \fbib{is head of Damascus}} \\
\poeml within sixty-five years \\
\poemll    Ephraim will be shattered as a people. \\
\poeml \v{9}Furthermore, Ephraim's head is Samaria, \\
\poemll    and Remaliah's son is its king.\fnote{\fbackref{7:9} Lit. \fbib{is head of Samaria}} \\
\poeml If all of you don't keep on believing,
\end{poetry}

you'll never remain loyal.'\,''\fnote{\fbackref{7:9} Or \fbib{never keep on enduring}}
\passage{God with Us}

\v{10}Later on, the \divine{Lord} spoke to Ahaz again: \v{11}``Ask a sign from the \divine{Lord} your God. Make it as deep as Sheol\fnote{\fbackref{7:11} I.e. the realm of the dead} or as high as heaven above.''

\v{12}But Ahaz replied, ``I won't ask! I won't put the \divine{Lord} to the test.''

\v{13}In reply, the \divine{Lord}\fnote{\fbackref{7:13} Lit. \fbib{reply, he}} announced, ``Please listen, you household of David. Is it such a minor thing for you to try the patience of\fnote{\fbackref{7:13} Lit. \fbib{to wear out}} men? Must you also try the patience of\fnote{\fbackref{7:13} Lit. \fbib{also wear out}} my God?

\v{14}``Therefore the \divine{Lord}\fnote{\fbackref{7:14} So 1QIsa\textsuperscript{a}; MT reads \fbib{Lord}} himself will give you a sign. Watch! The virgin\fnote{\fbackref{7:14} So LXX; 1QIsa\textsuperscript{a} MT read \fbib{The young woman}} is conceiving a child, and will give birth to a son, and his name will be called\fnote{\fbackref{7:14} So 1QIsa\textsuperscript{a}; MT LXX read \fbib{she will name him}; MT alt. reading \fbib{and you will name him}} Immanuel.\fnote{\fbackref{7:14} The Heb. name \fbib{Immanuel} means \fbib{God with us}} \v{15}He'll eat cheese\fnote{\fbackref{7:15} Or \fbib{curds}} and honey, when he knows enough to reject what's wrong and choose what's right. \v{16}However, before the youth knows enough to reject what's wrong and choose what's right, the land whose two kings you dread will be devastated.''
\passage{Conquest by Assyria}

\v{17}``The \divine{Lord} will bring to you, to your people, and to your ancestor's house such a time\fnote{\fbackref{7:17} Lit. \fbib{such days}} as has never been since Ephraim broke away from Judah---the king of Assyria will come.\fnote{\fbackref{7:17} 1QIsa\textsuperscript{a} MT lack \fbib{will come}}

\v{18}``At that time,\fnote{\fbackref{7:18} Lit. \fbib{On that day}} the \divine{Lord} will call\fnote{\fbackref{7:18} Lit. \fbib{whistle}} for flies that will come from far away---from the headwaters of Egypt's rivers---and for bees that are in the land of Assyria. \v{19}They will all come and settle in the steep ravines, in the rocky crevices, in all the thorn bushes, and in all the pastures.\fnote{\fbackref{7:19} Or \fbib{the watering places}} \v{20}At that time,\fnote{\fbackref{7:20} Lit. \fbib{On that day}} the \divine{Lord} will hire a barber\fnote{\fbackref{7:20} Lit. \fbib{razor}} to come\fnote{\fbackref{7:20} 1QIsa\textsuperscript{a} MT lack \fbib{to come}} from beyond the Euphrates\fnote{\fbackref{7:20} 1QIsa\textsuperscript{a} MT lack \fbib{Euphrates}} River---that is, the king of Assyria---and he will shave your heads, your leg\fnote{\fbackref{7:20} Or \fbib{feet}} hair, and your beards, too.

\v{21}``At that time,\fnote{\fbackref{7:21} Lit. \fbib{On that day}} a man will keep alive a heifer and two sheep, \v{22}and because of the abundance of milk that they give, he will have cheese\fnote{\fbackref{7:22} Or \fbib{curds}} to eat, since whoever remains in the land will be eating cheese\fnote{\fbackref{7:22} Or \fbib{curds}} and honey.

\v{23}``At that time,\fnote{\fbackref{7:23} Lit. \fbib{On that day}} every place where once there were a thousand vines worth a thousand shekels\fnote{\fbackref{7:23} I.e. about 400 ounces; a shekel weighed about 0.4 ounces} of silver, only briars and thorns will grow.

\v{24}``People will come there armed with bows\fnote{\fbackref{7:24} So 1QIsa\textsuperscript{a}; MT LXX read \fbib{bow}} and arrows, because the entire land will be nothing but briers and thorns. \v{25}As for all the hills that used to be cultivated with a hoe, you won't go there, because you'll fear iron\fnote{\fbackref{7:25} So 1QIsa\textsuperscript{a}; MT LXX lack \fbib{iron}} briars and thorns. Nevertheless, those hills\fnote{\fbackref{7:25} Lit. \fbib{but they}} will be reserved as a pasture where cattle will feed and where sheep will graze.''\fnote{\fbackref{7:25} Lit. \fbib{tread}}
\labelchapt{8}
\passage{Isaiah's Son is Born}

\chapt{8}
\v{1}The \divine{Lord} also told me, ``Take a large tablet and write on it with a stylus\fnote{\fbackref{8:1} Or \fbib{with an ordinary}} pen, `For Maher-shalal-hash-baz'.\fnote{\fbackref{8:1} The Heb. name \fbib{Maher-shalal-hash-baz} means \fbib{Hurry to the plunder, quick to the loot}} \v{2}Then I will call\fnote{\fbackref{8:2} So 1QIsa\textsuperscript{a} LXX; MT reads \fbib{call in}} Uriah the priest and Jeberechiah's son Zechariah as reliable witnesses to testify on my behalf.''

\v{3}After this, I was intimate with the prophetess and she conceived. Later, she bore a son, and then the \divine{Lord} told me,\fnote{\fbackref{8:3} So 1QIsa\textsuperscript{a} MT LXX; 4QIsa\textsuperscript{e} lacks \fbib{me}} ``Call him\fnote{\fbackref{8:3} Lit. \fbib{Call his name}} `Maher-shalal-hash-baz,' \v{4}for before the young lad knows how to call out to his father or mother,\fnote{\fbackref{8:4} So 1QIsa\textsuperscript{a}; MT reads \fbib{out `My father!' or `My mother!';} LXX reads `\fbib{Father!'} or `\fbib{Mother!'}} the wealth of Damascus and the plunder of Samaria will be carried off by the king of Assyria.''
\passage{Invasion by Assyria}

\v{5}The \divine{Lord} spoke to me again: \v{6}``Because this people have rejected the gently-flowing waters of Shiloah, and because\fnote{\fbackref{8:6} 1QIsa\textsuperscript{a} MT lack \fbib{because}} they keep rejoicing in Rezin and Remaliah's son, \v{7}watch out! The \divine{Lord} God\fnote{\fbackref{8:7} So 1QIsa\textsuperscript{a}; MT reads \fbib{the Lord}} is about to bring the flood waters of the Euphrates\fnote{\fbackref{8:7} 1QIsa\textsuperscript{a} MT lack \fbib{Euphrates}} River against them, mighty and strong.\fnote{\fbackref{8:7} So 4QIsa\textsuperscript{f} MT LXX; 1QIsa\textsuperscript{a} lacks \fbib{mighty and strong}}

``It's the king of Assyria and all of his arrogance! He will rise over all of the river's channels and run over all of its banks. \v{8}He will sweep on into Judah, overflowing as he passes through, like flood waters\fnote{\fbackref{8:8} DSS MT lack \fbib{like flood waters}} reaching up to a person's neck. His outstretched wings will flow as wide as your land, O Immanuel!''

\begin{poetry}
\poeml \v{9}``Band together,\fnote{\fbackref{8:9} So 1QIsa\textsuperscript{a} MT; 4QIsa\textsuperscript{e} 4QIsa\textsuperscript{f} LXX read \fbib{Learn this}; or \fbib{Know this}} you peoples, \\
\poemll    but be shattered! \\
\poemlll       Listen, all you distant countries! \\
\poeml Strap on your armor, \\
\poemlll       but be shattered.\fnote{\fbackref{8:9} So 1QIsa\textsuperscript{a}; MT adds a second \fbib{strap on your armor but be shattered}; cf. LXX} \\
\poeml \v{10}Take counsel together, \\
\poemll    but it will all be for nothing; \\
\poeml go ahead and talk, \\
\poemll    but\fnote{\fbackref{8:10} So 1QIsa\textsuperscript{a} MT; 4QIsa\textsuperscript{e} LXX lack \fbib{but}} it will all be for nothing,\fnote{\fbackref{8:10} Lit. \fbib{it won't stand}} \\
\poemlll       for God is with us.''\fnote{\fbackref{8:10} I.e. a word play on the name \fbib{Immanuel}; cf. 7:14, 8:8}
\end{poetry}
\passage{Waiting on God}

\v{11}For\fnote{\fbackref{8:11} So 1QIsa\textsuperscript{a} 4QIsa\textsuperscript{e} MT; 4QIsa\textsuperscript{f} LXX Syr lack \fbib{For}} this is what the \divine{Lord} spoke to me, as his forceful hand was resting on me, and as he was warning me not to live the way this people were living:\fnote{\fbackref{8:11} Lit. \fbib{not to walk in the way of this people}}

\begin{poetry}
\poeml \v{12}``Don't call conspiracy everything \\
\poemll    that this people calls conspiracy, \\
\poeml and don't fear what they fear, \\
\poemll    or live in terror. \\
\poeml \v{13}The \divine{Lord} of the Heavenly Armies--- \\
\poemll    he's the one you are to regard as holy. \\
\poeml Let him be the one whom you fear, \\
\poemll    and let him be the one before whom you stand in terror! \\
\poeml \v{14}Then he will be a sanctuary, \\
\poemll    but for both houses of Israel \\
\poeml he'll also be a stone with which someone strikes himself, \\
\poemll    a rock one stumbles over, \\
\poemlll       a trap and a snare to those who live in Jerusalem. \\
\poeml \v{15}Many will stumble on them; \\
\poemll    They'll fall and be broken; \\
\poemlll       They'll be snared and captured. \\
\poeml \v{16}``Bind up the testimony, \\
\poemll    and seal up the teaching among my disciples. \\
\poeml \v{17}I'll wait for the \divine{Lord}, \\
\poemll    who is hiding his face from the house of Jacob, \\
\poemlll       and I'll put my trust in him. \\
\poeml \v{18}Watch out! I and the children \\
\poemll    whom the \divine{Lord} has given me \\
\poeml are a sign and a wonder\fnote{\fbackref{8:18} So 1QIsa\textsuperscript{a}; MT LXX read \fbib{are signs and wonders}} in Israel \\
\poemll    from the \divine{Lord} of the Heavenly Armies, \\
\poemlll       who resides on Mount Zion.''
\passage{Rejecting Occultic Wisdom}
\poeml \v{19}``So when they advise you, \\
\poeml `Ask the mediums your questions, \\
\poemll    and quiz the spiritists who chirp and mutter,' \\
\poeml shouldn't a people instead be consulting their God---\fnote{\fbackref{8:19} So 1QIsa\textsuperscript{a} LXX; MT reads \fbib{gods}} \\
\poemll    and not the dead--- \\
\poeml on behalf of those who are living \\
\poeml \v{20}for instruction and for testimony? \\
\poemll    Surely they are speaking like this \\
\poemlll       because the truth\fnote{\fbackref{8:20} 1QIsa\textsuperscript{a} MT lack \fbib{the truth}} hasn't dawned on them. \\
\poeml \v{21}``They'll pass through the land,\fnote{\fbackref{8:21} Lit. \fbib{through it}} \\
\poemll    while\fnote{\fbackref{8:21} So 1QIsa\textsuperscript{a}; the Heb. lacks \fbib{while}} greatly distressed and hungry. \\
\poeml When they are hungry, \\
\poemll    they'll become enraged, \\
\poeml and they'll curse their king and their god.\fnote{\fbackref{8:21} So 1QIsa\textsuperscript{a}; MT reads \fbib{gods}; LXX reads \fbib{idols}} \\
\poemll    They'll turn their faces upwards, \\
\poeml \v{22}or they'll look toward the\fnote{\fbackref{8:22} So 1QIsa\textsuperscript{a} LXX; the Heb. lacks \fbib{the}} earth, \\
\poemll    but they'll see only distress and darkness, \\
\poeml the gloom that comes from anguish, \\
\poemll    and then they'll be thrown into total darkness.''
\end{poetry}
\labelchapt{9}
\passage{The Prince of Peace}

\chapt{9}
\v{1}\fnote{\fbackref{9:1} This v. is 8:23 in the MT}But there will be no gloom for her who was\fnote{\fbackref{9:1} So DSS; MT reads \fbib{those who were}} in distress. Formerly, he brought contempt to the region of Zebulun and the region of Naphtali, but in the future\fnote{\fbackref{9:1} So DSS; MT reads \fbib{the latter time}} he will have made glorious the way of the sea, the territory beyond the Jordan---Galilee of the nations.\fnote{\fbackref{9:1} Or \fbib{gentiles}}

\begin{poetry}
\poeml \v{2}\fnote{\fbackref{9:2} This v. is 9:1 in the MT}The people who walked in darkness \\
\poemll    have seen a great light; \\
\poeml for those living in a land of deep darkness, \\
\poemll    a light has shined upon them. \\
\poeml \v{3}You have increased the nation; \\
\poemll    you have increased its joy; \\
\poeml they rejoice in your presence \\
\poemll    as they rejoice at the harvest, \\
\poeml as they are glad \\
\poemll    when they're dividing the spoils of war.\fnote{\fbackref{9:3} 1QIsa\textsuperscript{a} MT lack \fbib{of war}} \\
\poeml \v{4}Now as to the yoke that has been\fnote{\fbackref{9:4} Lit. \fbib{yoke of}} his burden, \\
\poemll    and the bar laid\fnote{\fbackref{9:4} DSS The Heb. lacks \fbib{laid}} on his shoulder--- \\
\poeml the rod of his oppressor--- \\
\poemll    you have broken it\fnote{\fbackref{9:4} 1QIsa\textsuperscript{a} MT lack \fbib{it}} as on the day of Midiam.\fnote{\fbackref{9:4} So 1QIsa\textsuperscript{a} LXX; MT reads \fbib{Midian}; cf. Judg 7:8-25 2King 15:19; 16:8} \\
\poeml \v{5}For every boot of the tramping soldier in battle tumult \\
\poemll    and every garment rolled in blood \\
\poemlll       will be used for burning as fuel for a fire. \\
\poeml \v{6}For to us a child is born, \\
\poemll    to us a son is given; \\
\poeml and the government will be upon his shoulder, \\
\poemll    and his name is\fnote{\fbackref{9:6} So 1QIsa\textsuperscript{a}; MT 4QIsa\textsuperscript{c} read \fbib{name will be}} called \\
\poeml Wonderful Counselor, Mighty God, \\
\poeml Everlasting Father, Prince of Peace. \\
\poeml \v{7}Of the growth of his government and peace \\
\poemll    there will be no end. \\
\poeml He will rule\fnote{\fbackref{9:7} DSS MT lack \fbib{He will rule}} over his kingdom, \\
\poemll    sitting on the throne of David, \\
\poeml to establish it and to uphold it\fnote{\fbackref{9:7} So 1QIsa\textsuperscript{a}, referring to the throne; MT reads \fbib{it}, referring to the kingdom} \\
\poemll    with justice and righteousness \\
\poemlll       from this time onward and forevermore. \\
\poeml The zeal of the \divine{Lord} of the Heavenly Armies will accomplish this.
\passage{A Rebuke to Jacob and Israel}
\poeml \v{8}``The \divine{Lord}\fnote{\fbackref{9:8} So 1QIsa\textsuperscript{a}; MT reads \fbib{Lord}} has sent a plague\fnote{\fbackref{9:8} So LXX; MT reads \fbib{word}; 1QIsa\textsuperscript{a} can mean \fbib{plague} or \fbib{word}.} against Jacob, \\
\poemll    and it will fall on Israel; \\
\poeml \v{9}and all of the people were evil\fnote{\fbackref{9:9} So 1QIsa\textsuperscript{a}; MT LXX read \fbib{knew}}--- \\
\poemll    Ephraim and the inhabitants of Samaria--- \\
\poemlll       saying proudly with arrogant hearts: \\
\poeml \v{10}`The bricks have fallen, \\
\poemll    but we will build with dressed\fnote{\fbackref{9:10} Or \fbib{quarried}} stones; \\
\poeml the sycamore\fnote{\fbackref{9:10} The sycamore fruit tree native to Israel bears figs} trees have been cut down, \\
\poemll    but we will replace them with cedars.'\fnote{\fbackref{9:10} I.e. a genus of coniferous evergreen in the family \fbib{Pinaceae}} \\
\poeml \v{11}But the \divine{Lord} has raised adversaries\fnote{\fbackref{9:11} So 1QIsa\textsuperscript{a} MT LXX; other MT mss. read \fbib{princes}} from Rezin\fnote{\fbackref{9:11} So 1QIsa\textsuperscript{a} MT; LXX lacks \fbib{from Rezin}} against him, \\
\poemll    and he stirs up his enemies--- \\
\poeml \v{12}Arameans from the east \\
\poemll    and Philistines from the west--- \\
\poeml and they devour Israel with open mouths! \\
\poeml ``Yet\fnote{\fbackref{9:12} So 1QIsa\textsuperscript{a} 4QIsa\textsuperscript{c}; the Heb. lacks \fbib{Yet}} for all this, his anger has not turned away, \\
\poemll    and his hand is still stretched out, ready to strike.''\fnote{\fbackref{9:12} DSS MT lack \fbib{ready to strike}}
\passage{Judgment for Not Repenting}
\poeml \v{13}``But the people have not returned to rely\fnote{\fbackref{9:13} 1QIsa\textsuperscript{a} MT LXX lack \fbib{to rely}} on\fnote{\fbackref{9:13} So 1QIsa\textsuperscript{a}; MT reads \fbib{toward}; LXX reads \fbib{until}} him who struck them, \\
\poemll    nor have they sought the \divine{Lord} of the Heavenly Armies. \\
\poeml \v{14}So the \divine{Lord} has cut off from Israel head and tail, \\
\poemll    palm branch and reed \\
\poemlll       in\fnote{\fbackref{9:14} So 1QIsa\textsuperscript{a} LXX; the The Heb. lacks \fbib{in}} a single day--- \\
\poeml \v{15}the elder and the dignitary is the head, \\
\poemll    and the prophet who teaches lies is the tail. \\
\poeml \v{16}For those who guide this people have been leading them astray, \\
\poemll    and those who are guided by them are swallowed up. \\
\poeml \v{17}Therefore the Lord does not have pity on\fnote{\fbackref{9:17} So 1QIsa\textsuperscript{a};MT LXX read \fbib{rejoice over}} their young men, \\
\poemll    and has no compassion on their orphans\fnote{\fbackref{9:17} Or \fbib{fatherless}} and widows, \\
\poeml because each of them was godless and an evildoer, \\
\poemll    and every mouth spoke folly. \\
\poeml ``Yet\fnote{\fbackref{9:17} So 1QIsa\textsuperscript{a} 4QIsa\textsuperscript{c}; the Heb. lacks \fbib{Yet}} for all this, his anger has not turned away, \\
\poemll    and his hand is still stretched out, ready to strike.\fnote{\fbackref{9:17} DSS MT lack \fbib{ready to strike}} \\
\poeml \v{18}``For wickedness has burned like a blaze \\
\poemll    that consumes briers and thorns; \\
\poeml it sets thickets of the forest on fire, \\
\poemll    and skyward\fnote{\fbackref{9:18} Lit. \fbib{upward}} they swirl \\
\poemlll       in a column of smoke. \\
\poeml \v{19}From\fnote{\fbackref{9:19} So 1QIsa\textsuperscript{a}; MT reads \fbib{By}} the wrath of the \divine{Lord} of the Heavenly Armies \\
\poemll    the land has been scorched, \\
\poeml and the people have become like fuel for the fire; \\
\poemlll       no one will spare his neighbor. \\
\poeml \v{20}They cut meat on the right, \\
\poemll    but they're still hungry, \\
\poeml and they devour also\fnote{\fbackref{9:20} So 1QIsa\textsuperscript{a}; MT LXX lacks \fbib{also}} on the left, \\
\poemll    but they're not satisfied; \\
\poemlll       each devours the flesh of his own children.\fnote{\fbackref{9:20} So 4QIsa\textsuperscript{e}; or \fbib{arms}; 1QIsa\textsuperscript{a} MT read \fbib{offsprin} g or \fbib{arm}; LXX reads \fbib{arm}} \\
\poeml \v{21}Manasseh devours Ephraim, \\
\poemll    and Ephraim devours Manasseh; \\
\poeml together they are against Judah. \\
\poeml ``Yet\fnote{\fbackref{9:21} So 1QIsa\textsuperscript{a}; MT LXX lack \fbib{Yet}} for all this, his anger has not turned away, \\
\poemll    and his hand is still stretched out, ready to strike.''\fnote{\fbackref{9:21} DSS MT lack \fbib{ready to strike}}
\end{poetry}
\labelchapt{10}
\passage{Judgment on Unjust Lawmakers}

\begin{poetry}
\poeml \chapt{10}
\v{1}``How terrible it will be for the one\fnote{\fbackref{10:1} So 1QIsa\textsuperscript{a}; MT reads \fbib{the ones}} who enacts unjust decrees, \\
\poemll    for those who write oppressive laws \\
\poemll    that they have prescribed \\
\poeml \v{2}to deprive the needy of justice \\
\poemll    and to rob the poor of my people of their rights,\fnote{\fbackref{10:2} Lit. \fbib{right}} \\
\poeml so that widows may become their spoil \\
\poemll    and so that they may plunder orphans!\fnote{\fbackref{10:2} Or \fbib{plunder the fatherless}} \\
\poeml \v{3}What will you do on the day of Judgment,\fnote{\fbackref{10:3} Lit. \fbib{reckoning}} \\
\poemll    in the calamity that will come from far away? \\
\poeml To whom will you run for help, \\
\poemll    and where will you leave your wealth, \\
\poeml \v{4}so you won't have to crouch among those in chains\fnote{\fbackref{10:4} So 1QIsa\textsuperscript{a}; MT LXX read \fbib{beneath prisoners}} \\
\poemll    or fall among the slain? \\
\poeml ``Yet\fnote{\fbackref{10:4} So 1QIsa\textsuperscript{a}; MT LXX lack \fbib{Yet}} for all this, his anger has not turned away, \\
\poemll    and his hand is still stretched out, ready to strike.''\fnote{\fbackref{10:4} DSS MT lack \fbib{ready to strike}}
\passage{Assyria is an Instrument of Judgment}
\poeml \v{5}``How terrible it will be \\
\poemll    for Assyria, the rod of my anger! \\
\poemlll       The club is in their hands!\fnote{\fbackref{10:5} So 1QIsa\textsuperscript{a} LXX; MT reads \fbib{is their fury!}} \\
\poeml \v{6}I'm sending my fury\fnote{\fbackref{10:6} So 1QIsa\textsuperscript{a} LXX; MT reads \fbib{sending him}} against a godless nation, \\
\poemll    and I'll command him against the people with whom I'm angry \\
\poeml to seize loot and snatch plunder, \\
\poemll    and to trample them down \\
\poemlll       like mud in the streets. \\
\poeml \v{7}But this is not what he intends, \\
\poemll    and this is not what he thinks in his mind; \\
\poeml but it is in his mind to destroy, \\
\poemll    and to cut down\fnote{\fbackref{10:7} Lit. \fbib{off}} many nations. \\
\poeml \v{8}``Because this is what he is saying: \\
\poemll    `My commanders are all kings, are they not? \\
\poeml \v{9}Isn't Calno like Carchemish? \\
\poemll    Isn't Hamath like Arpad? \\
\poemlll       Isn't Samaria like Damascus? \\
\poeml \v{10}As my hand has reached to the idolatrous kingdoms\fnote{\fbackref{10:10} So 1QIsa\textsuperscript{a}; MT reads \fbib{the idol}} \\
\poemll    whose carved images were greater than those of Jerusalem and Samaria, \\
\poeml \v{11}will I not deal with Jerusalem and her idols \\
\poemll    as I have dealt with Samaria and her images?'\,''
\end{poetry}
\passage{Assyria will be Judged}

\v{12}``For\fnote{\fbackref{10:12} So 1QIsa\textsuperscript{a}; MT LXX read \fbib{And when}} the Lord has finished all his work against Mount Zion and against Jerusalem; he will punish the speech that comes from that willful\fnote{\fbackref{10:12} Lit. \fbib{the fruit of the arrogant}} heart of Assyria's king and the haughty look in his eyes. \v{13}He keeps bragging:\fnote{\fbackref{10:13} Lit. \fbib{saying}; so 1QIsa\textsuperscript{a}; MT reads \fbib{He said}}

\begin{poetry}
\poeml `I've done it by the strength of my hand, \\
\poemll    and by my wisdom, \\
\poemlll       because I'm so clever.\fnote{\fbackref{10:13} Lit. \fbib{I have understanding}} \\
\poeml I removed the boundaries of peoples, \\
\poemll    and plundered their treasures; \\
\poeml like a bull I brought down \\
\poemll    those who sat on thrones. \\
\poeml \v{14}My hand has found, as if in a nest, \\
\poemll    the wealth of the people; \\
\poeml and as one gathers eggs that have been abandoned, \\
\poemll    so I have gathered all the inhabitants of the\fnote{\fbackref{10:14} 1QIsa\textsuperscript{a} MT lack \fbib{inhabitants of the}} earth. \\
\poeml Nothing moved a wing, \\
\poemll    opened its mouth, \\
\poemlll       or chirped.' \\
\poeml \v{15}``Does the ax exalt itself \\
\poemll    over the one who swings it? \\
\poeml Or does the saw magnify itself \\
\poemll    in opposition to the one who wields it? \\
\poeml As if a rod were to wield those who lift\fnote{\fbackref{10:15} So 1QIsa\textsuperscript{a} MT; LXX reads \fbib{the one who lifts}} it, \\
\poemll    or as if a club were to brandish the one who is not wood! \\
\poeml \v{16}Therefore, the Lord \divine{God}\fnote{\fbackref{10:16} So 1QIsa\textsuperscript{a} MT; other LXX MT mss read \fbib{}\divine{Lord}} of the Heavenly Armies will send a wasting disease \\
\poemll    among Assyria's\fnote{\fbackref{10:16} Lit. \fbib{his}} sturdy warriors, \\
\poeml and under its glory a conflagration will be kindled, \\
\poemll    like a blazing bonfire. \\
\poeml \v{17}``The light of Israel will become a fire, \\
\poemll    and its Holy One a flame, \\
\poeml and it will burn \\
\poemll    and consume Assyria's\fnote{\fbackref{10:17} Lit. \fbib{its}} thorns and briers \\
\poemlll       in a single day. \\
\poeml \v{18}The splendor of its forest and its fruitful land \\
\poemll    the \divine{Lord} will destroy--- \\
\poemlll       both soul and body--- \\
\poeml and Assyria\fnote{\fbackref{10:18} Lit. \fbib{it}} will be \\
\poemll    as when a dying man wastes away. \\
\poeml \v{19}What survives of the trees in his forest will be so few \\
\poemll    that a child can count them.''\fnote{\fbackref{10:19} Lit. \fbib{can write them down}}
\end{poetry}
\passage{The Remnant Returns}

\v{20}At that time, the remnant of Israel and the survivors of the house of Jacob will no longer rely on the one who struck them down, but will truly rely on the \divine{Lord}, the Holy One of Israel. \v{21}A remnant will return---a remnant of Jacob---to the Mighty God. \v{22}For even if your people of Israel number as many as the sand of the sea, only a remnant of them will return. Overwhelming, righteous destruction is decreed, \v{23}because the Lord \divine{God} of the Heavenly Armies\fnote{\fbackref{10:23} So 1QIsa\textsuperscript{a}; LXX, MT mss lack \fbib{Lord of the Heavenly Armies}} will bring about destruction, as has been decreed, throughout\fnote{\fbackref{10:23} Lit. \fbib{in the midst of}} the entire region.\fnote{\fbackref{10:23} Lit. \fbib{land}}

\v{24}Therefore this is what the Lord \divine{God} of the Heavenly Armies says: ``My people, you who live in Zion, don't be afraid of the Assyrians, of the rod that beats you,\fnote{\fbackref{10:24} So 1QIsa\textsuperscript{a}; MT reads \fbib{Assyrians, when they strike you with a rod}} who lift up their club against you as the Egyptians did. \v{25}In just a little while, my fury will come to an end, and my anger then will be directed to their destruction.\fnote{\fbackref{10:25} So 1QIsa\textsuperscript{a} MT; MT mss. read \fbib{end}; LXX reads \fbib{counsel}} \v{26}The \divine{Lord} of the Heavenly Armies will brandish a whip against them, as when he struck Midian at the rock of Oreb;\fnote{\fbackref{10:26} Cf. Judg 7:25} and as his staff was stretched out\fnote{\fbackref{10:26} 1QIsa\textsuperscript{a} MT lack \fbib{stretched out}} over the sea,\fnote{\fbackref{10:26} Cf. Exod 14:16,26} so he will lift it up as he did in Egypt. \v{27}At that time,\fnote{\fbackref{10:27} Lit. \fbib{On that day}; so 1QIsa\textsuperscript{a} MT LXX; 4QIsa\textsuperscript{c} reads \fbib{On a day}} his burden will depart from your shoulder and his yoke from your neck. Indeed, the yoke will be broken, because you've become obese.''\fnote{\fbackref{10:27} So 1QIsa\textsuperscript{a} MT; LXX lacks \fbib{because you've become obese}.}
\passage{The Coming Judgment of God}

\begin{poetry}
\poeml \v{28}``The Assyrian commander\fnote{\fbackref{10:28} Lit. \fbib{He}} has come upon\fnote{\fbackref{10:28} So 1QIsa\textsuperscript{a} MT; 4QIsa\textsuperscript{c} LXX read \fbib{to}} Aiath \\
\poemll    and has passed through Migron; \\
\poemlll       he stores his supplies at Michmash. \\
\poeml \v{29}He has\fnote{\fbackref{10:29} So 1QIsa\textsuperscript{a}; MT reads \fbib{They have}} crossed over by\fnote{\fbackref{10:29} So 1QIsa\textsuperscript{a}; the Heb. lacks \fbib{by}} the pass; \\
\poemll    his overnight lodging is at Geba. \\
\poeml Ramah trembles; \\
\poemll    Gibeah of Saul has fled. \\
\poeml \v{30}Cry aloud, you daughter of Gallim! \\
\poemll    Pay attention, Laish!\fnote{\fbackref{10:30} So 1QIsa\textsuperscript{a}; MT LXX read \fbib{Laishah}} \\
\poemlll       Poor Anathoth! \\
\poeml \v{31}Marmenah\fnote{\fbackref{10:31} So 1QIsa\textsuperscript{a} Syr.; MT LXX read \fbib{Madmenah}} is in flight; \\
\poemll    the inhabitants of Gebim take cover. \\
\poeml \v{32}This very day he will halt at Nob;\fnote{\fbackref{10:32} I.e. city where the ephod was stored during the reign of Saul; cf. 1Sam 22:13-20} \\
\poemll    he will shake\fnote{\fbackref{10:32} So 1QIsa\textsuperscript{a}; MT reads \fbib{brandish}} his fists\fnote{\fbackref{10:32} So 1QIsa\textsuperscript{a}; MT reads \fbib{fist}} \\
\poeml at the mountain that is the Daughter of Zion, \\
\poemll    at Jerusalem's hill. \\
\poeml \v{33}Behold, the Lord \divine{God} of the Heavenly Armies \\
\poemll    will lop off its\fnote{\fbackref{10:33} Lit. \fbib{the}} boughs with terrifying power; \\
\poeml the tallest in height will be cut down, \\
\poemll    and the lofty will be brought low. \\
\poeml \v{34}He will cut down the thickets of the forest \\
\poemll    with an ax, \\
\poeml and Lebanon will fall \\
\poemll    by the Majestic One.''\fnote{\fbackref{10:34} Or \fbib{fall, along with its majestic trees}}
\end{poetry}
\labelchapt{11}
\passage{The Reign of the Davidic King}

\begin{poetry}
\poeml \chapt{11}
\v{1}``A shoot will come out \\
\poemll    from the stump of Jesse, \\
\poeml and a branch will bear fruit \\
\poemll    from his roots. \\
\poeml \v{2}The Spirit of the \divine{Lord} will rest upon him, \\
\poemll    the Spirit of wisdom and understanding, \\
\poeml the Spirit of counsel and power, \\
\poemll    the Spirit of knowledge and fear of the \divine{Lord}. \\
\poeml \v{3}His delight will be in the fear of the \divine{Lord}. \\
\poemll    He won't judge by what his eyes see, \\
\poemlll       nor decide disputes by what his ears hear, \\
\poeml \v{4}but with righteousness he will judge the needy, \\
\poemll    and decide with equity for\fnote{\fbackref{11:4} So 1QIsa\textsuperscript{a}; MT LXX read \fbib{for the}} earth's poor.\fnote{\fbackref{11:4} So 1QIsa\textsuperscript{a}; MT LXX read \fbib{humble}} \\
\poeml He will strike the earth with the rod of his mouth,\fnote{\fbackref{11:4} I.e. by pronouncing judgment} \\
\poemll    and the wicked will be killed\fnote{\fbackref{11:4} So 1QIsa\textsuperscript{a}; 1QIsa\textsuperscript{a} corrector MT LXX read \fbib{he will kill the wicked}} with the breath of his lips. \\
\poeml \v{5}Righteousness will be the sash around his loins, \\
\poemll    and faithfulness the belt around his waist.''
\passage{A Transformed Ecology}
\poeml \v{6}``The wolf will live with the lamb; \\
\poemll    the leopard will lie down with the young goat. \\
\poeml The calf and the lion will graze\fnote{\fbackref{11:6} So 1QIsa\textsuperscript{a} LXX; MT reads \fbib{lion and the fattened calf}} together, \\
\poemll    and a little child will lead them. \\
\poeml \v{7}The cow and the bear will graze, \\
\poemll    and\fnote{\fbackref{11:7} So 1QIsa\textsuperscript{a} LXX; the Heb. lacks \fbib{and}} their young will lie down together, \\
\poemlll       and the lion will eat straw like the ox. \\
\poeml \v{8}The nursing child will play \\
\poemll    over the hole of the cobra, \\
\poemlll       and the weaned child will put his hand on vipers' dens.\fnote{\fbackref{11:8} So 1QIsa\textsuperscript{a}; MT reads \fbib{a viper's den}; LXX reads \fbib{a den of vipers}} \\
\poeml \v{9}They will neither harm nor destroy \\
\poemll    on\fnote{\fbackref{11:9} So 1QIsa\textsuperscript{a} LXX; 4QIsa\textsuperscript{c} MT read \fbib{on all}} my holy mountain; \\
\poeml for the earth will be full \\
\poemll    of the knowledge\fnote{\fbackref{11:9} So 1QIsa\textsuperscript{a} MT; 4QIsa\textsuperscript{c} LXX read \fbib{to know}} of the \divine{Lord},\fnote{\fbackref{11:9} So 1QIsa\textsuperscript{a} MT LXX; 4QIsa\textsuperscript{c} reads \fbib{of glory}; cf. Hab 2:14} \\
\poemlll       as the waters cover the sea.''
\end{poetry}
\passage{Israel Regathered}

\v{10}At that time,\fnote{\fbackref{11:10} Lit. \fbib{day}} as to\fnote{\fbackref{11:10} 1QIsa\textsuperscript{a} MT lack \fbib{as to}} the root of Jesse, who will be standing as a banner for the peoples, the nations will rally to him, and his resting place is\fnote{\fbackref{11:10} So 1QIsa\textsuperscript{a}; 4QIsa\textsuperscript{c} MT LXX read \fbib{place will be}} glorious.

\v{11}At that time,\fnote{\fbackref{11:11} Lit. \fbib{day}} the \divine{Lord} will reach out his hand yet a second time to recover the remnant that is left of his people, from Assyria, from Lower Egypt, from Upper Egypt,\fnote{\fbackref{11:11} Lit. \fbib{from Egypt, from Pathros}} from Cush, from Elam, from Shinar, from Hamath, and from the islands\fnote{\fbackref{11:11} Or \fbib{coastlands}} of the sea.

\begin{poetry}
\poeml \v{12}He will raise a banner for the nations \\
\poemll    and will assemble the dispersed of Israel; \\
\poeml he will gather the scattered people of Judah \\
\poemll    from the corners\fnote{\fbackref{11:12} So 1QIsa\textsuperscript{a}; 4QIsa\textsuperscript{a} MT LXX read \fbib{four corners}} of the earth.
\passage{Israel's Victory over Its Enemies}
\poeml \v{13}Ephraim's jealousy will vanish,\fnote{\fbackref{11:13} Lit. \fbib{depart}} \\
\poemll    and those who are hostile to Judah will be eliminated;\fnote{\fbackref{11:13} Lit. \fbib{be cut off}} \\
\poeml Ephraim will no longer be jealous of Judah, \\
\poemll    and Judah will not be hostile to Ephraim. \\
\poeml \v{14}But they\fnote{\fbackref{11:14} So 1QIsa\textsuperscript{a} MT LXX; 4QIsa\textsuperscript{a} Targ read \fbib{he}} will swoop down \\
\poemll    on the slopes\fnote{\fbackref{11:14} Lit. \fbib{backs}} of the Philistines to the west, \\
\poemlll       and they will plunder\fnote{\fbackref{11:14} So pap4QIsa\textsuperscript{e} 1QIsa\textsuperscript{a}; MT LXX read \fbib{the west; together they will plunder}} the people to\fnote{\fbackref{11:14} Lit. \fbib{of}} the east. \\
\poeml They'll lay their hands on Edom and Moab, \\
\poemll    and the Ammonites will be subject to them. \\
\poeml \v{15}The \divine{Lord} will totally destroy \\
\poemll    the gulf\fnote{\fbackref{11:15} Lit. \fbib{tongue}} of the Sea of Egypt. \\
\poeml He will sweep his hand \\
\poemll    over the Euphrates River \\
\poemlll       with a violent wind,\fnote{\fbackref{11:15} So 1QIsa\textsuperscript{a}; MT LXX read \fbib{his violent wind}} \\
\poeml breaking it up into seven streams, \\
\poemll    and making a way for people to cross on foot. \\
\poeml \v{16}And there will be a highway \\
\poemll    for the remnant that is left of his people out of Assyria, \\
\poeml as there was for Israel \\
\poemll    when they came up \\
\poemlll       from the land of Egypt.
\end{poetry}
\labelchapt{12}
\passage{Israel's Praise to the \divine{Lord}}

\begin{poetry}
\poeml \chapt{12}
\v{1}At that time,\fnote{\fbackref{12:1} Lit. \fbib{day}} you will say:
\end{poetry}

\begin{poetry}
\poeml ``I will praise you, \divine{Lord}, \\
\poemll    for although you were angry with me, \\
\poeml your anger has turned away, \\
\poemll    and you have comforted me. \\
\poeml \v{2}``Look! God---yes God---is\fnote{\fbackref{12:2} So 1QIsa\textsuperscript{a}; MT reads \fbib{Look! God is}; LXX reads \fbib{Look! The \divine{Lord} is the God of}} my salvation; \\
\poemll    I will trust, and not be afraid. \\
\poeml For the \divine{Lord}\fnote{\fbackref{12:2} So 1QIsa\textsuperscript{a} MT\textsuperscript{mss} LXX; MT reads \fbib{Lord \divine{God}}} is my strength and my song,\fnote{\fbackref{12:2} So 1QIsa\textsuperscript{a} MT\textsuperscript{mss} LXX; MT reads \fbib{a song}} \\
\poemll    and he has become my salvation.''
\end{poetry}

\v{3}You will draw water joyfully from the wells of salvation. And you will say at that time:\fnote{\fbackref{12:3} Lit. \fbib{say in that day}}

\begin{poetry}
\poeml \v{4}``Give thanks to the \divine{Lord}; \\
\poemll    call on his name. \\
\poeml Make known his actions \\
\poemll    among the nations. \\
\poemlll       Proclaim that his name is exalted. \\
\poeml \v{5}``Sing praises to the \divine{Lord},\fnote{\fbackref{12:5} So 1QIsa\textsuperscript{a}; MT reads \fbib{to the \divine{Lord}}; LXX reads \fbib{to the name of the \divine{Lord}}} \\
\poemll    because he has acted gloriously, \\
\poemlll       being made\fnote{\fbackref{12:5} So Isa\textsuperscript{a} MT\textsuperscript{qere} Syr Targ; the Heb. lacks \fbib{made}} known in all the world. \\
\poeml \v{6}Shout aloud, and sing for joy, \\
\poemll    you who live in Zion, \\
\poeml because great in your midst \\
\poemll    is the Holy One of Israel.''
\end{poetry}
\labelchapt{13}
\passage{The Destruction of Babylon}

\chapt{13}
\v{1}A message\fnote{\fbackref{13:1} Lit. \fbib{An oracle}} that Amoz's son Isaiah received\fnote{\fbackref{13:1} Lit. \fbib{saw}} about Babylon:

\begin{poetry}
\poeml \v{2}``Raise a banner on a bare hilltop! \\
\poemll    Cry out loud to them! \\
\poeml Give a wave of the hand, \\
\poemll    signaling\fnote{\fbackref{13:2} The Heb. lacks \fbib{signaling}} for them to enter\fnote{\fbackref{13:2} So 1QIsa\textsuperscript{a}; MT reads \fbib{for them to enter}; LXX lacks \fbib{to enter}} \\
\poemlll       the gates of the nobles. \\
\poeml \v{3}I myself have commanded my consecrated ones; \\
\poemll    I have also summoned my warriors, \\
\poeml those who rejoice in my triumph, \\
\poemll    to carry out my angry judgments.\fnote{\fbackref{13:3} Lit. \fbib{my anger}} \\
\poeml \v{4}``Listen! There's a noise on the mountains \\
\poemll    like that of a great multitude! \\
\poeml Listen! There's an uproar among the kingdoms, \\
\poemll    like that of nations massing together! \\
\poeml The \divine{Lord} of the Heavenly Armies is mustering \\
\poemll    an army for battle. \\
\poeml \v{5}They're coming from a faraway land, \\
\poemll    from the distant horizon---\fnote{\fbackref{13:5} Lit. \fbib{end of the heavens}} \\
\poeml the \divine{Lord} and the weapons of his anger--- \\
\poemll    to destroy the entire land.''\fnote{\fbackref{13:5} Or \fbib{earth}}
\passage{The Day of the \divine{Lord}}
\poeml \v{6}Wail out loud, because the Day of the \divine{Lord} is near. \\
\poemll    It will come like destruction from the Almighty! \\
\poeml \v{7}Because of this, every hand\fnote{\fbackref{13:7} So 1QIsa\textsuperscript{a}; MT reads \fbib{all hands}} will go limp, \\
\poemll    and every man's courage\fnote{\fbackref{13:7} Lit. \fbib{heart}} will melt. \\
\poeml \v{8}They will be terrified; \\
\poemll    pain and anguish will seize them; \\
\poeml they'll writhe like a woman in labor. \\
\poemll    They'll look aghast at one another; \\
\poemlll       and\fnote{\fbackref{13:8} So 1QIsa\textsuperscript{a}; cf. LXX; 4QIsa\textsuperscript{a} 4QIsa\textsuperscript{b} MT lack \fbib{and}} their faces will be ablaze with fear.\fnote{\fbackref{13:8} DSS MT lack \fbib{with fear}} \\
\poeml \v{9}Watch out! The Day of the \divine{Lord} is coming--- \\
\poemll    cruel, with wrath and fierce anger--- \\
\poeml to turn the entire inhabited\fnote{\fbackref{13:9} So LXX; 1QIsa\textsuperscript{a} lacks \fbib{the entire inhabited}; the Heb. lacks \fbib{entire inhabited}} earth\fnote{\fbackref{13:9} LXX lacks \fbib{earth}} into a desolation \\
\poemll    and to annihilate sinners\fnote{\fbackref{13:9} So 1QIsa\textsuperscript{a} LXX; 4QIsa\textsuperscript{a} 4QIsa\textsuperscript{b} MT read \fbib{its sinners}} from it. \\
\poeml \v{10}For the stars of the heavens and their constellations \\
\poemll    won't shine\fnote{\fbackref{13:10} So 1QIsa\textsuperscript{a}; MT reads \fbib{beam}} their light; \\
\poeml the sun will be dark when it rises, \\
\poemll    and the moon won't shine its light. \\
\poeml \v{11}I'll punish the world for its evil, \\
\poemll    and the wicked for their iniquity; \\
\poeml I'll put an end to the pomposity of the arrogant, \\
\poemll    and overthrow the insolence of tyrants. \\
\poeml \v{12}I'll make people scarcer\fnote{\fbackref{13:12} Lit. \fbib{people more precious}} than pure gold, \\
\poemll    and mankind rarer\fnote{\fbackref{13:12} 1QIsa\textsuperscript{a} MT lack \fbib{rarer}} than gold from Ophir. \\
\poeml \v{13}Therefore I'll make the heavens tremble. \\
\poemll    The earth will shake from its place \\
\poeml at the wrath of the \divine{Lord} of the Heavenly Armies, \\
\poemll    at the time\fnote{\fbackref{13:13} Lit. \fbib{in the day}} of his burning anger.\fnote{\fbackref{13:13} Lit. \fbib{nostrils}} \\
\poeml \v{14}They\fnote{\fbackref{13:14} So 1QIsa\textsuperscript{a} LXX; MT reads \fbib{it}} will be like a hunted gazelle, \\
\poemll    or like sheep with no one to gather them,\fnote{\fbackref{13:14} So 1QIsa\textsuperscript{a} MT LXX; 4QIsa\textsuperscript{a} reads \fbib{banished}} \\
\poeml each will turn to his own people, \\
\poemll    and each will flee to his own land. \\
\poeml \v{15}Whoever is captured will be thrust through, \\
\poemll    and whoever is caught will fall dead, killed\fnote{\fbackref{13:15} So 1QIsa\textsuperscript{a}; MT lacks \fbib{dead, killed}} by the sword. \\
\poeml \v{16}Their infants will be dashed to pieces \\
\poemll    before their eyes, \\
\poeml and\fnote{\fbackref{13:16} So 1QIsa\textsuperscript{a} Syr; cf. LXX; 4QIsa\textsuperscript{a} MT lack \fbib{and}} their houses will be looted, \\
\poemll    and their wives slept with.\fnote{\fbackref{13:16} So 1QIsa\textsuperscript{a} MT\textsuperscript{qere}; 4QIsa\textsuperscript{a} MT read \fbib{raped}; LXX reads \fbib{they will take}; cf. Deut 28:30}
\passage{Babylon Falls}
\poeml \v{17}Watch out! I'm stirring up the Medes against them, \\
\poemll    who care nothing for silver \\
\poemlll       and take no delight in gold. \\
\poeml \v{18}Their bows will dash the young men to pieces; \\
\poemll    they'll show no pity on those not yet born,\fnote{\fbackref{13:18} Lit. \fbib{on the fruit of the womb}} \\
\poemlll       and\fnote{\fbackref{13:18} So 1QIsa\textsuperscript{a} MT\textsuperscript{mss} LXX; the Heb. lacks \fbib{and}} their eyes will not spare children. \\
\poeml \v{19}Babylon, that jewel of kingdoms, \\
\poemll    the splendor and pride of the Chaldeans, \\
\poeml will be like Sodom and Gomorrah, \\
\poemll    when God overthrew them--- \\
\poeml \v{20}It will never be inhabited \\
\poemll    or lived in through all generations; \\
\poeml no Bedouin\fnote{\fbackref{13:20} I.e. Middle Eastern nomadic herders; or \fbib{Arab}} will pitch his tent there; \\
\poemll    no shepherds will make their flocks lie down there. \\
\poeml \v{21}But desert beasts will lie down there, \\
\poemll    and their houses will be full of howling creatures; \\
\poeml there owls\fnote{\fbackref{13:21} Or \fbib{ostriches}} will dwell, \\
\poemll    and goat-demons\fnote{\fbackref{13:21} Or \fbib{satyrs}; or \fbib{wild goats}} will dance there. \\
\poeml \v{22}Hyenas will howl in its strongholds,\fnote{\fbackref{13:22} Lit. \fbib{desolate places}} \\
\poemll    and jackals will make their dens in its citadels.\fnote{\fbackref{13:22} So 1QIsa\textsuperscript{a} LXX; MT reads \fbib{in the citadels of luxury}} \\
\poeml Its\fnote{\fbackref{13:22} So 1QIsa\textsuperscript{a} LXX; MT reads \fbib{And its}} time is close at hand, \\
\poemll    and its days will not be extended any further.\fnote{\fbackref{13:22} So 1QIsa\textsuperscript{a}; MT LXX lack \fbib{any further}}
\end{poetry}
\labelchapt{14}
\passage{Israel Mocks Babylon's King}

\chapt{14}
\v{1}However, the \divine{Lord} will have compassion on Jacob and will once again choose Israel. He will settle them in their own land, and foreigners will join them, affiliating themselves with the house of Jacob. \v{2}Many\fnote{\fbackref{14:2} So 1QIsa\textsuperscript{a}; MT LXX lack \fbib{Many}} nations will take them and bring them to their land and\fnote{\fbackref{14:2} So 1QIsa\textsuperscript{a}; MT LXX read \fbib{and to}} their own place. The house of Israel will put those nations\fnote{\fbackref{14:2} Lit. \fbib{put them}} to conscripted labor\fnote{\fbackref{14:2} Lit. \fbib{to male and female slavery}} in the \divine{Lord}'s land. They will take captive those who were their captors, and will rule continually\fnote{\fbackref{14:2} So 1QIsa\textsuperscript{a}; 4QIsa\textsuperscript{c} LXX MT read \fbib{rule}} over those who oppressed them.

\v{3}At the time,\fnote{\fbackref{14:3} Lit. \fbib{day}} when the \divine{Lord} gives you rest from your suffering, turmoil, and the cruel bondage which they forced you to serve,\fnote{\fbackref{14:3} So 1QIsa\textsuperscript{a} 4QIsa\textsuperscript{e}; MT LXX read \fbib{which you were forced to serve}} \v{4}you will lift up this song of mockery against\fnote{\fbackref{14:4} So 1QIsa\textsuperscript{a} MT LXX; 4QIsa\textsuperscript{e} reads \fbib{to}} the king of Babylon:

\begin{poetry}
\poeml ``How the oppressor has come to an end! \\
\poemll    How the attacker\fnote{\fbackref{14:4} 1QIsa\textsuperscript{a} LXX; MT reads. \fbib{the golden city}} has ceased! \\
\poeml \v{5}The \divine{Lord} has broken the staff of the wicked, \\
\poemll    the scepter of rulers, \\
\poeml \v{6}that struck down peoples in anger \\
\poemll    with unceasing blows, \\
\poeml that oppressed nations in fury \\
\poemll    with relentless persecution. \\
\poeml \v{7}The entire earth is at rest and peace; \\
\poemll    its inhabitants\fnote{\fbackref{14:7} Lit. \fbib{they}} break into song. \\
\poeml \v{8}Even the cypresses rejoice over you, \\
\poemll    as do\fnote{\fbackref{14:8} So 1QIsa\textsuperscript{a}; MT LXX lack \fbib{as do}} the cedars of Lebanon, saying, \\
\poeml `Now that you've been laid low, \\
\poemll    no woodcutter comes up against us.'\fnote{\fbackref{14:8} So 1QIsa\textsuperscript{a} MT LXX; 4QIsa\textsuperscript{e} reads \fbib{against them}} \\
\poeml \v{9}``The afterlife\fnote{\fbackref{14:9} Lit. \fbib{Sheol}, i.e. the realm of the dead} below is all astir \\
\poemll    to meet you when you arrive;\fnote{\fbackref{14:9} Lit. \fbib{at your coming}} \\
\poeml it rouses up the spirits of the dead to greet you--- \\
\poemll    everyone who used to be world leaders. \\
\poeml It has raised up from their thrones \\
\poemll    all who used to be kings of the nations. \\
\poeml \v{10}In answer, all of them\fnote{\fbackref{14:10} So 1QIsa\textsuperscript{a} MT LXX; 4QIsa\textsuperscript{e} lacks \fbib{of them}} will tell you, \\
\poemll    `You've also become as weak as we are! \\
\poeml You have become just like us!' \\
\poeml \v{11}Your\fnote{\fbackref{14:11} 1QIsa\textsuperscript{a} reads \fbib{The;} MT LXX lack \fbib{Your}} pomp has been brought down to Sheol,\fnote{\fbackref{14:11} I.e. the realm of the dead} \\
\poemll    along with the noise of your harps. \\
\poeml Maggots are spread out beneath you, \\
\poemll    and worms are your covering.''\fnote{\fbackref{14:11} So LXX; MT reads \fbib{and a worm covers you}}
\passage{The Fall of the Day Star}
\poeml \v{12}``How you have fallen from heaven, \\
\poemll    Day Star, son of the Dawn!\fnote{\fbackref{14:12} I.e. Lucifer} \\
\poeml How you have been thrown down to earth, \\
\poemll    you who laid low the nation!\fnote{\fbackref{14:12} So 1QIsa\textsuperscript{a}; 4QIsa\textsuperscript{e} MT read \fbib{the nations}; LXX reads \fbib{all the nations}} \\
\poeml \v{13}You said in your heart, \\
\poemll    `I'll ascend to heaven, \\
\poemlll       above the stars of God. \\
\poeml I'll erect my throne; \\
\poemll    I'll sit\fnote{\fbackref{14:13} So 1QIsa\textsuperscript{a} LXX; MT reads \fbib{and I will sit}} on the Mount of Assembly \\
\poemlll       in the far reaches of the north;\fnote{\fbackref{14:13} Lit. \fbib{Zaphon}; or \fbib{the Sacred Mountain}} \\
\poeml \v{14}I'll ascend above the tops of the clouds; \\
\poemll    I'll make myself like the Most High.' \\
\poeml \v{15}But you are brought down to join the dead,\fnote{\fbackref{14:15} Lit. \fbib{to Sheol}, i.e. the realm of the dead} \\
\poemll    to the far reaches of the Pit.\fnote{\fbackref{14:15} I.e. the realm of punishment in the afterlife} \\
\poeml \v{16}``Those who see you will stare at you. \\
\poemll    They will wonder about you: \\
\poeml `Is this the man who\fnote{\fbackref{14:16} So 1QIsa\textsuperscript{a} LXX; the Heb. lacks \fbib{who}} made the earth tremble, \\
\poemll    who made kingdoms quake, \\
\poeml \v{17}who made the world like a desert, \\
\poemll    who\fnote{\fbackref{14:17} So 1QIsa\textsuperscript{a}; MT LXX read \fbib{and who}} destroyed its cities, \\
\poemlll       who would not open the jails for his prisoners?' \\
\poeml \v{18}All the kings of the nations lie\fnote{\fbackref{14:18} So 1QIsa\textsuperscript{a} LXX; MT reads \fbib{nations, every one of them lies}} in state, \\
\poemll    each in his own tomb. \\
\poeml \v{19}But you are cast away from your grave, \\
\poemll    like a repulsive branch, \\
\poeml your clothing is the slain, \\
\poemll    those pierced by the sword; \\
\poemlll       those who go down to the Pit.\fnote{\fbackref{14:19} So 1QIsa\textsuperscript{a}; i.e. to the realm of punishment in the afterlife; LXX reads \fbib{to Hades}; 1QIsa\textsuperscript{a} corrector MT read \fbib{to the stones of the Pit}} \\
\poeml Like a dead body trampled underfoot, \\
\poeml \v{20}you will not be united\fnote{\fbackref{14:20} So 1QIsa\textsuperscript{a}; Lit. \fbib{under}; MT reads \fbib{joined}} with them\fnote{\fbackref{14:20} I.e. with the dead} in burial, \\
\poeml for you have destroyed your land, \\
\poemll    you have slain your people. \\
\poeml People will never mention \\
\poemll    the descendants of those who practice evil again!\fnote{\fbackref{14:20} So 1QIsa\textsuperscript{a}; MT reads \fbib{May the descendants of those who practice of evil never be mentioned again!}; LXX reads \fbib{May you not remain forever, you evil seed!}} \\
\poeml \v{21}Prepare a massacre for his sons \\
\poemll    because of the guilt of their forefathers! \\
\poeml They are not to rise and inherit the earth, \\
\poemll    and cover\fnote{\fbackref{14:21} Lit. \fbib{fill}} the surface of the world with cities.''
\end{poetry}
\passage{Babylon's Desolation}

\v{22}``I will rise up against them,'' declares the \divine{Lord} of the Heavenly Armies, ``and I will eliminate from Babylon her name and survivors, her offspring and descendants,''\fnote{\fbackref{14:22} So 1QIsa\textsuperscript{a}; MT reads \fbib{and her offspring and descendants}; LXX lacks \fbib{and descendants}} declares the \divine{Lord}. \v{23}``And\fnote{\fbackref{14:23} So 1QIsa\textsuperscript{a}; MT LXX lack \fbib{And}} I'm going to make it a possession of the hedgehog---pools\fnote{\fbackref{14:23} So 1QIsa\textsuperscript{a}; MT reads \fbib{and pools}} of water---and I'll sweep\fnote{\fbackref{14:23} So 1QIsa\textsuperscript{a}; MT reads \fbib{sweep it}} with the broom of destruction,'' declares the \divine{Lord} of the Heavenly Armies.

\begin{poetry}
\poeml \v{24}The \divine{Lord} of the Heavenly Armies has sworn: \\
\poeml ``Surely as I have planned, \\
\poemll    that's what she\fnote{\fbackref{14:24} I.e. Babylon} will become;\fnote{\fbackref{14:24} So 1QIsa\textsuperscript{a} LXX; MT reads \fbib{so has she been}} \\
\poeml and just as I have determined, \\
\poemll    so will it remain--- \\
\poeml \v{25}to crush the Assyrian in my land, \\
\poemll    and on my mountains I will trample him down. \\
\poeml His yoke\fnote{\fbackref{14:25} I.e. Assyria's oppressive domination} will turn away from you,\fnote{\fbackref{14:25} So 1QIsa\textsuperscript{a}; MT LXX read \fbib{from them}} \\
\poemll    and his burden from your\fnote{\fbackref{14:25} So 1QIsa\textsuperscript{a}; MT reads \fbib{their}} shoulders.'' \\
\poeml \v{26}``This is what I've planned \\
\poemll    for the whole earth, \\
\poeml and this is the hand that is stretched out \\
\poemll    over all the nations. \\
\poeml \v{27}For the \divine{Lord} of the Heavenly Armies has planned, \\
\poemll    and who can thwart him? \\
\poeml His hand is stretched out, \\
\poemll    and who can turn it back?''
\end{poetry}
\passage{A Pronouncement against Philistia}

\v{28}In the year that King Ahaz died this message\fnote{\fbackref{14:28} Lit. \fbib{oracle}} came:

\begin{poetry}
\poeml \v{29}``Don't rejoice, all of you Philistines, \\
\poemll    that the rod that struck you is broken, \\
\poeml because from the snake's root a viper will spring up, \\
\poemll    and its offspring\fnote{\fbackref{14:29} Lit. \fbib{fruit}} will be a darting, poisonous serpent. \\
\poeml \v{30}The firstborn of the poor will find pasture, \\
\poemll    and the needy will lie down in safety; \\
\poeml but I'll kill your root\fnote{\fbackref{14:30} I.e. the source of their strengths} by famine, \\
\poemll    and I'll\fnote{\fbackref{14:30} So 1QIsa\textsuperscript{a}; MT LXX read \fbib{he}} execute your survivors. \\
\poeml \v{31}Wail, you gate! \\
\poemll    Cry out, you city! \\
\poemlll       Melt away,\fnote{\fbackref{14:31} Or \fbib{melt in fear}} all you Philistines! \\
\poeml For smoke comes from the north, \\
\poemll    and there's no one to take measure\fnote{\fbackref{14:31} So 1QIsa\textsuperscript{a}; 4QIsa\textsuperscript{o} MT read \fbib{no straggler}} in its festivals.\fnote{\fbackref{14:31} So 1QIsa\textsuperscript{a}; 4QIsa\textsuperscript{o} reads \fbib{ranks}} \\
\poeml \v{32}How will they\fnote{\fbackref{14:32} So 1QIsa\textsuperscript{a} LXX; MT reads \fbib{one}} answer the messengers of the nation? \\
\poemll    ``The \divine{Lord} has founded Zion, \\
\poemlll       and in it\fnote{\fbackref{14:32} So 1QIsa\textsuperscript{a} LXX; MT reads \fbib{her}} the afflicted among his people will find refuge.''
\end{poetry}
\labelchapt{15}
\passage{Moab's Pending Judgment}

\chapt{15}
\v{1}A message\fnote{\fbackref{15:1} Lit. \fbib{An oracle}} concerning Moab:

\begin{poetry}
\poeml ``For Ir\fnote{\fbackref{15:1} Or \fbib{For the city}; so 1QIsa\textsuperscript{a}; MT reads \fbib{Ar}; LXX lacks \fbib{For Ir}} in Moab is destroyed in a night, \\
\poemll    and Moab\fnote{\fbackref{15:1} So 1QIsa\textsuperscript{a}; MT reads \fbib{it;} LXX lacks \fbib{it}} is ruined! \\
\poeml Because Ir\fnote{\fbackref{15:1} Or \fbib{Because the city}; so 1QIsa\textsuperscript{a}; 4QIsa\textsuperscript{o} MT LXX read \fbib{Because the wall} or \fbib{Because Kir}} in Moab is destroyed in a single night, \\
\poemll    Moab is ruined! \\
\poeml \v{2}He has gone up to the temple, and to Dibon, \\
\poemll    to the high places to weep; \\
\poeml over Nebo and over Medeba \\
\poemll    Moab wails. \\
\poeml His head is completely\fnote{\fbackref{15:2} So 1QIsa\textsuperscript{a}; MT reads \fbib{all its heads are}; MT\textsuperscript{mss} LXX read \fbib{Over every head}; cf. Jer 48:37} bald, \\
\poemll    and\fnote{\fbackref{15:2} So 1QIsa\textsuperscript{a} Mt\textsuperscript{mss} LXX; the Heb. lacks \fbib{and}} every beard is shaved off. \\
\poeml \v{3}In its streets they wear sackcloth; \\
\poemll    on its rooftops and in its squares \\
\poemlll       everyone wails and falls down\fnote{\fbackref{15:3} So 1QIsa\textsuperscript{a}; MT reads \fbib{falling down}; LXX lacks \fbib{and falls down}} weeping. \\
\poeml \v{4}Heshbon and Elealeh cry out, \\
\poemll    their voices are heard as far as Jahaz; \\
\poeml therefore the loins\fnote{\fbackref{15:4} So 1QIsa\textsuperscript{a}; cf. LXX; MT reads \fbib{armed men}} of Moab cry aloud; \\
\poemll    its heart quakes for itself. \\
\poeml \v{5}My heart cries out over Moab; \\
\poemll    her fugitives flee as far as Zoar, \\
\poemlll       as far as Eglath-shelishiyah. \\
\poeml For at the ascent to Luhith \\
\poemll    they go up weeping; \\
\poeml on the road to Horonaim \\
\poemll    they raise a cry of destruction. \\
\poeml \v{6}The Nimrim waters are desolate; \\
\poemll    the grass is withered, \\
\poeml its vegetation gone; \\
\poemlll       there is\fnote{\fbackref{15:6} So 1QIsa\textsuperscript{a}; MT reads \fbib{was} or \fbib{there has been}; LXX reads \fbib{there will be}} no foliage left. \\
\poeml \v{7}Therefore the wealth they have acquired \\
\poemll    and what they have stored up--- \\
\poeml they carry them away \\
\poemll    over the Arab\fnote{\fbackref{15:7} So 1QIsa\textsuperscript{a}; cf. LXX; MT reads \fbib{Willow}} Wadi.\fnote{\fbackref{15:7} I.e. a seasonal stream or river that channels water during rain seasons but is dry at other times} \\
\poeml \v{8}For the cry has gone out \\
\poemll    along the border of Moab; \\
\poeml her wailing reaches as far as Eglaim, \\
\poemll    her wailing reaches as far as Beer-elim. \\
\poeml \v{9}The Dibon\fnote{\fbackref{15:9} So 1QIsa\textsuperscript{a}; MT reads \fbib{Dimon}; LXX reads \fbib{Remmon}} streams are full of blood; \\
\poemll    but I will bring upon Dibon\fnote{\fbackref{15:9} So 1QIsa\textsuperscript{a}; MT reads \fbib{Dimon}; LXX reads \fbib{Remmon}} even more--- \\
\poeml a lion will pounce\fnote{\fbackref{15:9} 1QIsa\textsuperscript{a} LXX MT lack \fbib{will pounce}} upon those of Moab who escape, \\
\poeml upon\fnote{\fbackref{15:9} So 1QIsa\textsuperscript{a} LXX; the Heb. lacks \fbib{upon}} those who remain in the land.''
\end{poetry}
\labelchapt{16}
\passage{Moab's Destruction}

\begin{poetry}
\poeml \chapt{16}
\v{1}``Send a lamb to the ruler of the land, \\
\poemll    from Selah,\fnote{\fbackref{16:1} So 1QIsa\textsuperscript{a}; MT reads \fbib{Sela}; LXX reads \fbib{not the rock}} by way of the desert, \\
\poemlll       to the mountain of the Daughter of Zion.\fnote{\fbackref{16:1} I.e. Mt. Zion} \\
\poeml \v{2}Like fluttering birds, \\
\poemll    like an abandoned nest, \\
\poeml so are the daughters of Moab \\
\poemll    at the fords of the Arnon River.\fnote{\fbackref{16:2} So 1QIsa\textsuperscript{a}; the Heb. lacks \fbib{River}} \\
\poeml \v{3}``Give us advice; \\
\poemll    reach a decision! \\
\poeml Cast your shadow as if night had come \\
\poemll    at high noon. \\
\poeml Shelter the fugitives, \\
\poemll    And don't betray a single refugee. \\
\poeml \v{4}Let the fugitives\fnote{\fbackref{16:4} So 1QIsa\textsuperscript{a}; MT reads \fbib{my fugitives}} from Moab \\
\poemll    settle among you; \\
\poeml be a shelter to them \\
\poemll    from the destroyer. \\
\poeml When the oppressor comes to an end, \\
\poemll    and destruction has\fnote{\fbackref{16:4} So 1QIsa\textsuperscript{a}; MT reads \fbib{have}} ceased, \\
\poemlll       and the marauder\fnote{\fbackref{16:4} Lit. \fbib{the one who tramples underfoot}} has\fnote{\fbackref{16:4} So 1QIsa\textsuperscript{a}; MT reads \fbib{have}} vanished from the land, \\
\poeml \v{5}then a throne will be established in gracious love, \\
\poemll    and there will sit in faithfulness--- \\
\poemlll       in the Tent of David--- \\
\poeml one who judges, seeks justice, \\
\poemll    and is swift to do what is right.'' \\
\poeml \v{6}``We've heard about Moab's pride--- \\
\poemll    so very proud he became!---\fnote{\fbackref{16:6} So 1QIsa\textsuperscript{a} MT\textsuperscript{mss}; cf. Jer 48:29; MT reads \fbib{how very proud he was}} \\
\poeml his arrogance, his pride, and his insolence; \\
\poemll    therefore he is alone.\fnote{\fbackref{16:6} So 1QIsa\textsuperscript{a}; MT LXX read \fbib{but his boasts mean nothing}} \\
\poeml \v{7}Therefore, let Moab not\fnote{\fbackref{16:7} So 1QIsa\textsuperscript{a}; the Heb. lacks \fbib{not}} wail, \\
\poemll    let everyone wail for Moab. \\
\poeml Lament and grieve deeply \\
\poemll    for the ruined remains\fnote{\fbackref{16:7} Or \fbib{for the raisin cakes}} of Kir-hareseth. \\
\poeml \v{8}For the fields of Heshbon wither, \\
\poemll    as well as the vines of Sibmah.\fnote{\fbackref{16:8} So 1QIsa\textsuperscript{a}; MT LXX include the rest of v. 8: \fbib{The rulers {\ldots} vine of Sibmah} in v. 9} \\
\poeml The rulers of the nations \\
\poeml have struck down its choicest vines, \\
\poeml which once reached Jazer \\
\poeml and pushed to the desert. \\
\poeml Its shoots spread out \\
\poeml and passed over the sea.''
\passage{Isaiah Weeps for Moab}
\poeml \v{9}``Therefore I weep with the tears of Jazer \\
\poemll    for the vines of Sibmah. \\
\poeml I drench you with my tears, \\
\poemll    O Heshbon and Elealeh--- \\
\poeml for the shouts of joy over your summer fruit \\
\poemll    and your grain harvest have ended. \\
\poeml \v{10}Joy and gladness are taken away from the orchards, \\
\poemll    in the vineyards people will sing no songs,\fnote{\fbackref{16:10} So 1QIsa\textsuperscript{a} LXX; MT reads \fbib{no songs are sung}} \\
\poeml and\fnote{\fbackref{16:10} So 1QIsa\textsuperscript{a} MT\textsuperscript{mss}; the Heb. lacks \fbib{and}} no cheers are raised. \\
\poemll    No vintner\fnote{\fbackref{16:10} Lit. \fbib{treader}} treads out wine in the presses, \\
\poemlll       because I've put an end to the shouting. \\
\poeml \v{11}Therefore my insides\fnote{\fbackref{16:11}Or \fbib{heart}; lit. \fbib{bowels}} moan like a lyre for Moab, \\
\poemll    and my innermost being\fnote{\fbackref{16:11} Or \fbib{my very soul}} for Kir-hareseth. \\
\poeml \v{12}When Moab appears, \\
\poemll    when he arrives\fnote{\fbackref{16:12} So 1QIsa\textsuperscript{a}; MT LXX read \fbib{tires himself}} upon the high place \\
\poeml and comes to his sanctuary to pray, \\
\poemll    he will not prevail.''
\end{poetry}

\v{13}This was the message that the \divine{Lord} spoke concerning Moab in the past. \v{14}But now the \divine{Lord} has spoken again: ``Within three years, like the years of a contract worker,\fnote{\fbackref{16:14} I.e. as if carefully counted pursuant to an employment contract; cf. Isa 21:16} Moab's glory will be brought into contempt, in spite of all its great multitude, and its survivors will be very few and\fnote{\fbackref{16:14} So 1QIsa\textsuperscript{a} LXX; the Heb. lacks \fbib{and}} of no importance.''
\labelchapt{17}
\passage{A Rebuke to Damascus}

\chapt{17}
\v{1}A message\fnote{\fbackref{17:1} Lit. \fbib{An oracle}} about Damascus:\fnote{\fbackref{17:1} So MT LXX; 1QIsa\textsuperscript{a} reads \fbib{Dramascus}}

\begin{poetry}
\poeml ``Look! Damascus\fnote{\fbackref{17:1} So MT LXX; 1QIsa\textsuperscript{a} reads \fbib{Dramascus}} will cease to be a city. \\
\poemll    Instead, it will become a pile of ruins. \\
\poeml \v{2}The cities of Oraru\fnote{\fbackref{17:2} So 1QIsa\textsuperscript{a}; MT reads \fbib{Aroer}, a pun on the Heb. word for \fbib{ruins}; LXX reads \fbib{forever}} will be deserted--- \\
\poemll    they will be devoted to herds that will lay at rest, \\
\poemlll       and terrorism will be no more.\fnote{\fbackref{17:2} Lit. \fbib{and no one will make them afraid}} \\
\poeml \v{3}The fortress will disappear from Ephraim, \\
\poemll    and royal authority from Damascus;\fnote{\fbackref{17:3} So MT LXX; 1QIsa\textsuperscript{a} reads \fbib{Dramascus}} \\
\poeml the survivors\fnote{\fbackref{17:3} I.e. believing Jews who return} from Aram\fnote{\fbackref{17:3} I.e. \fbib{Syria}} will be like the glory of the Israelis,'' \\
\poemll    declares the \divine{Lord} of the Heavenly Armies.
\passage{A Time of Weakness for Israel}
\poeml \v{4}``At that time,\fnote{\fbackref{17:4} Lit. \fbib{On that day}} Jacob's glory will have become weakened, \\
\poemll    and his strong\fnote{\fbackref{17:4} Lit. \fbib{fat}} flesh will turn gaunt; \\
\poeml \v{5}it will be as if harvesters gather standing grain, \\
\poemll    reaping the ears by hand,\fnote{\fbackref{17:5} Lit. \fbib{ears with his arm}} \\
\poeml or it will be as if grain is harvested \\
\poemll    in the valley of Rephaim.\fnote{\fbackref{17:5} Lit. \fbib{Giants}} \\
\poeml \v{6}Nevertheless, gleanings will remain in Israel,\fnote{\fbackref{17:6} Lit. \fbib{it}} \\
\poemll    as when an olive tree is beaten---\fnote{\fbackref{17:6} Or \fbib{harvested}} \\
\poeml two or three ripe olives left in the topmost branches, \\
\poemll    four or five left among the branches of a fruit-filled tree,''\fnote{\fbackref{17:6} So 1QIsa\textsuperscript{a}; MT reads \fbib{among its branches}} \\
\poemlll       declares the \divine{Lord} God of Israel.
\end{poetry}
\passage{Revival to Come to Israel}

\v{7}At that time, men will look upon\fnote{\fbackref{17:7} So 1QIsa\textsuperscript{a}; cf. LXX; MT reads \fbib{to}} their Maker, and their eyes will honor the Holy One of Israel. \v{8}They will not look upon\fnote{\fbackref{17:8} So 1QIsa\textsuperscript{a}; cf. LXX; MT reads \fbib{to}} the altars, the products\fnote{\fbackref{17:8} So 1QIsa\textsuperscript{a}; MT LXX read \fbib{the product}} that their own fingers\fnote{\fbackref{17:8} So 1QIsa\textsuperscript{a}; MT LXX read \fbib{hands}} have made, and they will have no regard for Asherah poles\fnote{\fbackref{17:8} I.e. images of the Babylonian-Canaanite goddess of fortune} or incense altars.\fnote{\fbackref{17:8} So 1QIsa\textsuperscript{a} MT; LXX reads \fbib{for trees or for their abominations}}
\passage{Desolation to the Nations}

\begin{poetry}
\poeml \v{9}``At that time,\fnote{\fbackref{17:9} Lit. \fbib{On that day,}} their fortified cities \\
\poemll    that they abandoned because of the Israelis \\
\poeml will be like desolate places\fnote{\fbackref{17:9} So 1QIsa\textsuperscript{a}; cf. LXX; MT reads \fbib{place}} of the forests and hilltops---\fnote{\fbackref{17:9} Or \fbib{the Hivites and Amorites}} \\
\poemll    there will be desolation. \\
\poeml \v{10}For you have forgotten the God of your salvation \\
\poemll    and have not remembered the Rock \\
\poemlll       that is your strength. \\
\poeml Therefore even though you plant delightful plants, \\
\poemll    sowing them with imported vine-seedlings, \\
\poeml \v{11}at the time that you plant them, \\
\poemll    carefully making them grow, \\
\poeml the very morning you make your seed to sprout, \\
\poemll    your harvest will be ruined\fnote{\fbackref{17:11} Lit. \fbib{become a pile}} \\
\poemlll       in a time of grief and unbearable pain.''\fnote{\fbackref{17:11} Lit. \fbib{and sorrow}} \\
\poeml \v{12}``How terrible it will be for many peoples, \\
\poemll    who rage like the roaring sea! \\
\poeml Oh, how the uproar of nations \\
\poemll    is like the sound of rushing, mighty water--- \\
\poemll    How they roar! \\
\poeml \v{13}The nations roar like the rushing of many waters,\fnote{\fbackref{17:13} So 1QIsa\textsuperscript{a} MT; cf. LXX; MT\textsuperscript{ms} Syr lack this line} \\
\poemll    but the \divine{Lord}\fnote{\fbackref{17:13} Lit. \fbib{but he}} will rebuke them, \\
\poeml and they will run far away, \\
\poemll    chased like chaff blown down from the mountains \\
\poeml or like thick dust\fnote{\fbackref{17:13} Lit. \fbib{like something}} that rolls along, \\
\poemll    blown along by a wind storm. \\
\poeml \v{14}When the evening arrives, watch out---sudden terror! \\
\poemll    By morning they will be there no longer! \\
\poeml So it will be for those who plunder us \\
\poemll    and what will happen to those who rob us.''
\end{poetry}
\labelchapt{18}
\passage{A Rebuke to Cush}

\begin{poetry}
\poeml \chapt{18}
\v{1}Woe to the land of whirring wings \\
\poemll    that is beyond the rivers of Cush,\fnote{\fbackref{18:1} I.e. Nubia, south of Egypt (modern northern Sudan)} \\
\poeml \v{2}which sends envoys by the sea,\fnote{\fbackref{18:2} Or \fbib{Nile}} \\
\poemll    in papyrus boats over the water! \\
\poeml Go, swift messengers, \\
\poemll    to a tall, smooth-skinned nation, \\
\poeml to a people feared far and wide, \\
\poemll    a nation that metes out\fnote{\fbackref{18:2} Or \fbib{nation of strange speech}; so 1QIsa\textsuperscript{a} MT; LXX reads \fbib{nation without hope}} punishment\fnote{\fbackref{18:2} 1QIsa\textsuperscript{a} MT LXX lack \fbib{punishment}} and oppresses, \\
\poemlll       whose land the rivers divide. \\
\poeml \v{3}All you inhabitants of the world, \\
\poemll    you who live on the earth, \\
\poeml when a banner is raised on the mountains, \\
\poemll    you'll see it. \\
\poeml When a trumpet sounds, \\
\poemll    you'll hear it! \\
\poeml \v{4}For this is what the \divine{Lord} told me: \\
\poeml ``I will remain quiet and watch in my dwelling place \\
\poemll    like dazzling heat in sunshine, \\
\poemlll       like a cloud of dew in the heat\fnote{\fbackref{18:4} So 1QIsa\textsuperscript{a} MT; MT\textsuperscript{mss} LXX read \fbib{on the day}} of harvest.'' \\
\poeml \v{5}For before the harvest, when the season of\fnote{\fbackref{18:5} 1QIsa\textsuperscript{a} MT lack \fbib{season of}} budding is over, \\
\poemll    and sour grapes ripen into mature grapes,\fnote{\fbackref{18:5} Lit. \fbib{flowers}} \\
\poeml he cuts off the shoots with pruning knives, \\
\poemll    clearing away the spreading branches \\
\poemlll       as he lops them off. \\
\poeml \v{6}And\fnote{\fbackref{18:6} So 1QIsa\textsuperscript{a} MT LXX; 4QIsa\textsuperscript{b} lacks \fbib{And}} they will all be left \\
\poemll    for birds of prey that live on the mountains\fnote{\fbackref{18:6} So 4QIsa\textsuperscript{b}; 1QIsa\textsuperscript{a} MT read \fbib{of mountains}; LXX reads \fbib{mountains of heaven}} \\
\poemlll       and for wild animals.\fnote{\fbackref{18:6} Lit. \fbib{for beasts of the earth}; i.e. non-domesticated animals, as opposed to domesticated \fbib{livestock;}} \\
\poeml Birds of prey will pass the summer feeding on them, \\
\poemll    and all the wild animals\fnote{\fbackref{18:6} Lit. \fbib{the beasts of the earth}; so 1QIsa\textsuperscript{a}; i.e. non-domesticated animals, as opposed to domesticated \fbib{livestock;} MT LXX read \fbib{every beast of the field}} will pass the winter feeding\fnote{\fbackref{18:6} 1QIsa\textsuperscript{a} MT lack \fbib{feeding}} on them. \\
\poeml \v{7}At that time tribute will be brought to the \divine{Lord} of the Heavenly Armies \\
\poemll    from\fnote{\fbackref{18:7} So 1QIsa\textsuperscript{a} LXX; 4QIsa\textsuperscript{b} MT lack \fbib{from}} a tall and smooth-skinned people, \\
\poeml from a people feared far and wide, \\
\poemll    a nation that metes out\fnote{\fbackref{18:7} Or \fbib{nation of strange speech}; so 1QIsa\textsuperscript{a} MT; LXX reads \fbib{nation with hope}} punishment\fnote{\fbackref{18:7} 1QIsa\textsuperscript{a} MT LXX lack \fbib{punishment}} and oppresses, \\
\poeml whose land the rivers divide, \\
\poemll    to Mount Zion, \\
\poemlll       the place that bears\fnote{\fbackref{18:7} Lit. \fbib{place of}} the name of the \divine{Lord}.\fnote{\fbackref{18:7} So 1QIsa\textsuperscript{a}; 4QIsa\textsuperscript{b} MT LXX read \fbib{the \divine{Lord} of the Heavenly Armies}}
\end{poetry}
\labelchapt{19}
\passage{A Rebuke to Egypt}

\chapt{19}
\v{1}A message\fnote{\fbackref{19:1} Lit. \fbib{An oracle}} about Egypt:

\begin{poetry}
\poeml ``Watch out! The \divine{Lord} rides on a swift cloud, \\
\poemll    and is coming to Egypt. \\
\poeml The idols of Egypt tremble before him, \\
\poemll    and the hearts of the Egyptians melt within them. \\
\poeml \v{2}I will stir up Egyptians against Egyptians, \\
\poemll    and everyone will fight against his brother, \\
\poeml everyone against his neighbor, \\
\poemll    city\fnote{\fbackref{19:2} So 1QIsa\textsuperscript{a}; MT LXX lack \fbib{and}} against city, \\
\poemlll       kingdom against kingdom. \\
\poeml \v{3}The spirits of the Egyptians within them will be drained of courage,\fnote{\fbackref{19:3} 1QIsa\textsuperscript{a} MT lack \fbib{of courage}} \\
\poemll    and I will bring their plans to nothing. \\
\poeml They will consult idols\fnote{\fbackref{19:3} So 1QIsa\textsuperscript{a}; MT reads \fbib{consult the idols}; LXX reads \fbib{consult their idols}} and spirits of the dead, \\
\poemll    and mediums and spiritists. \\
\poeml \v{4}I will hand the Egyptians over \\
\poemll    to the power of a cruel master, \\
\poemlll       and a fierce king will rule over them,'' \\
\poeml declares the Lord \divine{God} of the Heavenly Armies.
\passage{A Rebuke to Egypt's Ecology and Industry}
\poeml \v{5}``The water sources\fnote{\fbackref{19:5} 1QIsa\textsuperscript{a} MT lack \fbib{sources}} of the Nile\fnote{\fbackref{19:5} Or \fbib{the sea}} will be dried up, \\
\poemll    and the river\fnote{\fbackref{19:5} I.e. the Nile} will become dry and parched. \\
\poeml \v{6}The canals will stink, \\
\poemll    and\fnote{\fbackref{19:6} So 1QIsa\textsuperscript{a} LXX; the Heb. lacks \fbib{and}} the tributaries of Egypt will dwindle and dry up. \\
\poemlll       Reeds and rushes will wither away.\fnote{\fbackref{19:6} So 4QIsa\textsuperscript{b} MT; 1QIsa\textsuperscript{a} lacks this line} \\
\poeml \v{7}And the bulrushes along the Nile,\fnote{\fbackref{19:7} So 1QIsa\textsuperscript{a} MT; LXX lacks this line} \\
\poemll    along the mouth of the Nile,\fnote{\fbackref{19:7} Lit. \fbib{the River}} will wither away. \\
\poeml All the sown fields of the Nile will become parched, \\
\poemll    and\fnote{\fbackref{19:7} So 1QIsa\textsuperscript{a}; the Heb. lacks \fbib{and}} they will be blown away; \\
\poemlll       there will be nothing left.\fnote{\fbackref{19:7} So 1QIsa\textsuperscript{a}; 4QIsa\textsuperscript{b} MT read \fbib{and will be no more}; LXX lacks this line} \\
\poeml \v{8}The fishermen will groan, \\
\poemll    and all who cast hooks into the Nile will lament; \\
\poeml those who spread nets upon the water \\
\poemll    will become weaker and weaker. \\
\poeml \v{9}The workers\fnote{\fbackref{19:9} So 1QIsa\textsuperscript{a} 4QIsa\textsuperscript{b}; MT LXX read \fbib{And the workers}} in combed flax \\
\poemll    and the weavers of white linen \\
\poemlll       will be in despair. \\
\poeml \v{10}Egypt's\fnote{\fbackref{19:10} Lit. \fbib{Its}} workers in cloth\fnote{\fbackref{19:10} Lit. \fbib{Its weavers}} will be crushed, \\
\poemll    and all who work for wages will be sick at heart.''
\passage{A Rebuke to Egypt's Leaders}
\poeml \v{11}Zoan's princes are nothing but fools; \\
\poemll    the wisest advisors of Pharaoh give stupid advice. \\
\poeml How can you say to Pharaoh, \\
\poemll    ``I'm a descendant of wise men, \\
\poemlll       a descendant of ancient kings''? \\
\poeml \v{12}Where are your wise men now? \\
\poemll    Let them tell you, \\
\poeml let them make known \\
\poemll    what the \divine{Lord}\fnote{\fbackref{19:12} So 1QIsa\textsuperscript{a}; 1QIsa\textsuperscript{a} corrector MT LXX read \fbib{\divine{Lord} of the Heavenly Armies}} has planned against Egypt. \\
\poeml \v{13}The princes of Zoan have become fools, \\
\poemll    and the princes of Memphis deluded; \\
\poeml the leaders\fnote{\fbackref{19:13} Or \fbib{cornerstones}} of its tribes \\
\poemll    have led Egypt astray. \\
\poeml \v{14}The \divine{Lord} has mixed\fnote{\fbackref{19:14} So 1QIsa\textsuperscript{a} MT LXX; 4QIsa\textsuperscript{b} reads \fbib{has poured}} within them\fnote{\fbackref{19:14} Lit. \fbib{it}} a spirit of confusion; \\
\poemll    so they make Egypt stagger in all that it does, \\
\poemlll       like a drunkard staggers around in his vomit. \\
\poeml \v{15}As a result, there will be nothing for Egypt \\
\poemll    that head or tail, palm branch or reed, can do.\fnote{\fbackref{19:15} So 1QIsa\textsuperscript{a} MT LXX; 4QIsa\textsuperscript{b} reads \fbib{do at that time}}
\end{poetry}
\passage{Egypt and Syria Will Worship God}

\v{16}At that time,\fnote{\fbackref{19:16} Lit. \fbib{On that day}; so 1QIsa\textsuperscript{a} MT LXX; 4QIsa\textsuperscript{b} lacks \fbib{At that time}} the Egyptians will be like women---they\fnote{\fbackref{19:16} So 1QIsa\textsuperscript{a}; 4QIsa\textsuperscript{b} MT read \fbib{he}} will shudder and be afraid before the uplifted hand of the \divine{Lord} of the Heavenly Armies, when he brandishes his hand against her.\fnote{\fbackref{19:16} I.e. Egypt; so 1QIsa\textsuperscript{a}; MT reads \fbib{Armies, which he brandishes against it}; LXX reads \fbib{Armies, which he brandishes against them}} \v{17}And the land of Judah will become a terror to the Egyptians. Everyone to whom it is mentioned will be afraid, because of the uplifted hand\fnote{\fbackref{19:17} Lit. \fbib{plotting}; so 4QIsa\textsuperscript{b}; 1QIsa\textsuperscript{a} MT LXX read \fbib{plan}} of the \divine{Lord} of the Heavenly Armies that\fnote{\fbackref{19:17} So 4QIsa\textsuperscript{b}; 1QIsa\textsuperscript{a} MT LXX read \fbib{hand that the \divine{Lord} of the Heavenly Armies}} is turning in their direction.

\v{18}At that time,\fnote{\fbackref{19:18} Lit. \fbib{On that day}} there will be five cities in the land of Egypt that speak the language of Canaan and swear allegiance to the \divine{Lord} of the Heavenly Armies. One of them will be called the City of the Sun.\fnote{\fbackref{19:18} So 1QIsa\textsuperscript{a} 4QIsa\textsuperscript{b} MT\textsuperscript{mss}; MT reads \fbib{Destruction}; LXX reads \fbib{Asedek City}; i.e. \fbib{City of Righteousness}}

\v{19}At that time,\fnote{\fbackref{19:19} Lit. \fbib{On that day}} there will be an altar to the \divine{Lord} of the Heavenly Armies\fnote{\fbackref{19:19} So 4QIsa\textsuperscript{b}; 1QIsa\textsuperscript{a} MT LXX lack \fbib{of the Heavenly Armies}} in the heart of the land of Egypt, and a monument to the \divine{Lord}\fnote{\fbackref{19:19} So 1QIsa\textsuperscript{a} MT LXX; 4QIsa\textsuperscript{b} reads \fbib{\divine{Lord} of hosts}} at its border. \v{20}It will be a sign and a witness to the \divine{Lord} of the Heavenly Armies in the land of Egypt; when they cry out to the \divine{Lord} because of their oppressors, he will send them a savior, and he will come down\fnote{\fbackref{19:20} So 1QIsa\textsuperscript{a}; MT LXX read \fbib{will defend}} and rescue them. \v{21}So the \divine{Lord} will make himself known to the Egyptians, and the Egyptians will acknowledge the \divine{Lord}.

At that time,\fnote{\fbackref{19:21} Lit. \fbib{On that day}} they will worship\fnote{\fbackref{19:21} So 1QIsa\textsuperscript{a}; MT LXX read \fbib{\divine{Lord} on that day. And they will worship}} with sacrifices and offerings, and they will make vows to the \divine{Lord} and carry them out. \v{22}The \divine{Lord} will strike Egypt with a plague, striking but then healing. Then they will turn to the \divine{Lord}, and he will respond to their pleas and heal them.

\v{23}At that time,\fnote{\fbackref{19:23} Lit. \fbib{On that day}} there will be a highway from Egypt to Assyria. The Assyrians will come into Egypt, and the Egyptians into Assyria, and they\fnote{\fbackref{19:23} So 1QIsa\textsuperscript{a}; 4QIsa\textsuperscript{b} MT LXX read \fbib{and the Egyptians}} will worship with the Assyrians.

\v{24}At that time,\fnote{\fbackref{19:24} Lit. \fbib{On that day}} Israel will be in a triple alliance\fnote{\fbackref{19:24} Lit. \fbib{will be the third}} with Egypt and Assyria; they will be\fnote{\fbackref{19:24} DSS MT lack \fbib{they will be}} a blessing in the midst of the earth. \v{25}The \divine{Lord} of the Heavenly Armies has blessed them, saying, ``Blessed be Egypt my people, Assyria the work of my hands, and Israel my inheritance.''
\labelchapt{20}
\passage{The Conquest of Egypt and Cush}

\chapt{20}
\v{1}In the year that the supreme commander, sent by Sargon the king of Assyria, came to Ashdod, attacked it, and captured it--- \v{2}at that time the \divine{Lord} spoke through Amoz's son Isaiah: ``Go loosen the sackcloth that's around your waist,\fnote{\fbackref{20:2} Lit. \fbib{your hips and lower back}} and take your sandals\fnote{\fbackref{20:2} So 1QIsa\textsuperscript{a} LXX; MT reads \fbib{sandal}} off your feet.'' So that's what he did: he went around naked and barefoot.

\v{3}Then the \divine{Lord} said, ``Just as my servant Isaiah has walked around naked and barefoot for three years as a sign and a warning for Egypt and Ethiopia,\fnote{\fbackref{20:3} I.e. Nubia, south of Egypt (modern northern Sudan)} \v{4}so the king of Assyria will lead away the Egyptian captives and exiles from Cush,\fnote{\fbackref{20:4} I.e. Nubia, south of Egypt (modern northern Sudan)} both the young and the old, naked and barefoot---with even their buttocks uncovered---to the shame\fnote{\fbackref{20:4} Or \fbib{nakedness}} of Egypt. \v{5}Then they will be dismayed and put to shame because of Cush,\fnote{\fbackref{20:5} I.e. Nubia, south of Egypt (modern northern Sudan)} their hope, and Egypt, their jewel.\fnote{\fbackref{20:5} Or \fbib{pride}} \v{6}At that time, the inhabitants of this coastland will say, `See, this is what has happened to those on whom we counted and relied\fnote{\fbackref{20:6} So 1QIsa\textsuperscript{a}; MT LXX read \fbib{and to whom we fled}} for help and deliverance from the king of Assyria! How, then, can we escape?'\,''
\labelchapt{21}
\passage{Elam and Media are Rebuked}

\chapt{21}
\v{1}A message\fnote{\fbackref{21:1} Lit. \fbib{An oracle}} concerning the pasture\fnote{\fbackref{21:1} Or \fbib{plague}; cf. Isa 5:17; 1King 8:37; Jer 14:12; MT LXX read \fbib{wilderness}} by the Sea.

\begin{poetry}
\poeml ``Like whirlwinds in the Negev\fnote{\fbackref{21:1} I.e. southern regions of the Sinai peninsula; cf. Josh 10:40} sweep on, \\
\poemll    it comes from the desert, \\
\poemlll       from a distant\fnote{\fbackref{21:1} So 1QIsa\textsuperscript{a}; 1QIsa\textsuperscript{a} corrector MT LXX read \fbib{terrible}} land. \\
\poeml \v{2}A dire vision has been announced to me: \\
\poemll    the traitor betrays, \\
\poemlll       and the plunderer takes loot. \\
\poeml Get up, Elam! \\
\poemll    Attack, Media! \\
\poeml I am putting a stop \\
\poemll    to all the groaning she has caused. \\
\poeml \v{3}Therefore my body is\fnote{\fbackref{21:3} Or \fbib{waist}; lit. \fbib{my hips and lower back are}} racked with pain; \\
\poemll    pangs have seized me, \\
\poemlll       like the pangs of a woman in labor; \\
\poeml I am so upset that I cannot hear; \\
\poemll    I am so frightened that I cannot see \\
\poemll    while I'm reeling around.\fnote{\fbackref{21:3} So 1QIsa\textsuperscript{a} 4QIsa\textsuperscript{a}; MT LXX begin v. 4 with this line} \\
\poeml \v{4}And as for my heart,\fnote{\fbackref{21:4} So 1QIsa\textsuperscript{a} 4QIsa\textsuperscript{a}; MT LXX read \fbib{\v{4}My mind reels}} horror has terrified me; \\
\poemll    the twilight I longed for \\
\poemlll       has started to make me tremble. \\
\poeml \v{5}They set the tables; \\
\poemll    they spread the carpets;\fnote{\fbackref{21:5} So 1QIsa\textsuperscript{a} MT; LXX lacks this line} \\
\poemlll       they eat, they drink! \\
\poeml Get up, you officers! \\
\poemll    Oil the shields!''
\passage{The Fall of Babylon}
\poeml \v{6}For this is what the \divine{Lord} told me: \\
\poeml ``Go post a lookout. \\
\poemlll       Have him report what he sees. \\
\poeml \v{7}When he sees chariots, each man\fnote{\fbackref{21:7} So 1QIsa\textsuperscript{a} 4QIsa\textsuperscript{a}; cf. v. 9; MT LXX lack \fbib{each man}} with a pair of horses, \\
\poemll    riders on donkeys or riders on\fnote{\fbackref{21:7} So 1QIsa\textsuperscript{a} LXX; MT reads \fbib{train of donkeys or train of}} camels, \\
\poeml let him pay attention, \\
\poemll    full attention.'' \\
\poeml \v{8}Then the lookout\fnote{\fbackref{21:8} So 1QIsa\textsuperscript{a} Syr; MT reads \fbib{Then a lion}} shouted: \\
\poemll    ``Upon a watchtower I stand, O Lord, \\
\poemlll       continually by day, \\
\poeml and I am stationed at my post \\
\poemll    throughout the night. \\
\poeml \v{9}Look! Here come riders,\fnote{\fbackref{21:9} So 1QIsa\textsuperscript{a} LXX; MT reads \fbib{chariot}; cf. v. 7} \\
\poemll    each man with a pair of horses!'' \\
\poemlll       They're shouting out the answer: \\
\poeml ``Babylon has fallen, has fallen, \\
\poemll    and they have shattered \\
\poemlll       all the images of her gods\fnote{\fbackref{21:9} So 1QIsa\textsuperscript{a}; LXX reads \fbib{all the images of her gods are shattered}; MT reads \fbib{He has shattered all the images of her gods}} on the ground! \\
\poeml \v{10}O my downtrodden people,\fnote{\fbackref{21:10} 1QIsa\textsuperscript{a} lacks \fbib{people}} my wall!\fnote{\fbackref{21:10} So 1QIsa\textsuperscript{a}; MT reads \fbib{my threshing floor}} \\
\poemll    I'll tell you what I have heard \\
\poemlll       from the \divine{Lord} of the Heavenly Armies, the God of Israel.''
\passage{A Message about Dumah}
\poeml \v{11}A message\fnote{\fbackref{21:11} Lit. \fbib{An oracle}} concerning Dumah. \\
\poeml ``Someone is calling to me from Seir: \\
\poemll    `Watchman, what is left of the night?\fnote{\fbackref{21:11} Or \fbib{What time of night?}} \\
\poemlll       Watchman, what is left of the night?'\fnote{\fbackref{21:11} Or \fbib{What time of night?}} \\
\poeml \v{12}The watchman replies: \\
\poemll    `Morning is coming, but also the night. \\
\poeml If you want to ask, then ask; \\
\poemll    come back again.'\,''
\passage{A Message about Arabia}
\poeml \v{13}A message\fnote{\fbackref{21:13} Lit. \fbib{An oracle}} concerning Arabia. \\
\poeml ``You will camp in the thickets in Arabia, \\
\poemll    you caravans of the Dedanites. \\
\poeml \v{14}Bring water for the thirsty, \\
\poemll    you who live in the land of Tema. \\
\poemlll       Meet the fugitive with bread,\fnote{\fbackref{21:14} So 1QIsa\textsuperscript{a} LXX; MT reads \fbib{with his bread}; 4QIsa\textsuperscript{a} reads \fbib{and with his bread}} \\
\poeml \v{15}For he has fled\fnote{\fbackref{21:15} So 1QIsa\textsuperscript{a}; MT reads \fbib{they have fled}} from swords, \\
\poemll    from the drawn sword, \\
\poeml from the bent bow, \\
\poemll    and from the heat of battle.''
\end{poetry}

\v{16}For this is what the \divine{Lord}\fnote{\fbackref{21:16} So 1QIsa\textsuperscript{a} 4QIsa\textsuperscript{a}; MT reads \fbib{Lord}} is saying to me: ``Within three years,\fnote{\fbackref{21:16} So 1QIsa\textsuperscript{a}; MT LXX read \fbib{Within a year}} according to the years of a contract worker,\fnote{\fbackref{21:16} I.e. as if carefully counted pursuant to an employment contract; cf. Isa 16:14} the pomp\fnote{\fbackref{21:16} So 1QIsa\textsuperscript{a} LXX; MT reads \fbib{years, all the pomp}} of Kedar will come to an end. \v{17}And there will be few archers, those who are descendants of Kedar, who survive, because the \divine{Lord}, the God of Israel, has spoken.''
\labelchapt{22}
\passage{Jerusalem is Rebuked}

\chapt{22}
\v{1}A message\fnote{\fbackref{22:1} Lit. \fbib{An oracle}} concerning the Valley of Vision.\fnote{\fbackref{22:1} I.e. a poetic allusion to the Hinnom Valley in Jerusalem}

\begin{poetry}
\poeml ``What troubles you, \\
\poemll    now that you've all gone up \\
\poemlll       to the rooftops, \\
\poeml \v{2}you who are full of commotion, \\
\poemll    you passionate city, \\
\poemlll       you rollicking town? \\
\poeml Your slain weren't killed by the sword, \\
\poemll    nor are they dead in battle. \\
\poeml \v{3}All your leaders have fled together; \\
\poemll    she is captured\fnote{\fbackref{22:3} So 1QIsa\textsuperscript{a}; MT reads \fbib{they were captured}} without using bows. \\
\poeml All of you who were caught were captured together, \\
\poeml although they had fled \\
\poemlll       while the enemy was still\fnote{\fbackref{22:3} The Heb. lacks \fbib{while the enemy was still}} far away. \\
\poeml \v{4}Therefore I said: \\
\poemll    ``Look away from me; \\
\poemlll       and let me weep bitter tears; \\
\poeml don't try to console\fnote{\fbackref{22:4} Lit. \fbib{don't insist on consoling}} me \\
\poemll    over the destruction of the daughter of my people.''\fnote{\fbackref{22:4} I.e. \fbib{the \divine{Lord}'s beloved people}} \\
\poeml \v{5}For to the Lord \divine{God} of the Heavenly Armies \\
\poemll    belongs the day of tumult, trampling, and confusion \\
\poemlll       in the Valley of Vision,\fnote{\fbackref{22:5} I.e. a poetic allusion to the Hinnom Valley in Jerusalem} \\
\poeml and the pulling down of his Temple on\fnote{\fbackref{22:5} Or \fbib{his Holy Place on}; so 1QIsa\textsuperscript{a}; MT reads \fbib{and a crying out for help to}} its mountain. \\
\poeml \v{6}Elam takes up the quiver \\
\poemll    with chariots and cavalry, \\
\poemlll       while Kir unsheathes the shield. \\
\poeml \v{7}And it will come about\fnote{\fbackref{22:7} So 1QIsa\textsuperscript{a}; MT reads \fbib{it came about}} \\
\poemll    that your choicest valleys will be filled with chariots, \\
\poemlll       and horsemen will take their positions at the gates. \\
\poeml \v{8}He has uncovered the defenses of Judah.''
\end{poetry}

At that time,\fnote{\fbackref{22:8} Lit. \fbib{On that day}} you looked at the arsenal of the Palace of the Forest,\fnote{\fbackref{22:8} Cf. 1King 10:16-17} \v{9}and saw that there were many breaches in the City of David. So you stored up water from the Lower Pool, \v{10}counted the houses of Jerusalem, tore down certain houses to strengthen the wall, \v{11}and built a reservoir between the walls to store water from the Old Pool. But you did not look at\fnote{\fbackref{22:11} So 1QIsa\textsuperscript{a}; MT LXX 4QIsa\textsuperscript{c} read \fbib{to}} the One who did it, nor did you see the One who planned it long ago.

\begin{poetry}
\poeml \v{12}On that day the Lord \divine{God} of the Heavenly Armies \\
\poemll    called for weeping and mourning, \\
\poemlll       for shaving heads\fnote{\fbackref{22:12} Lit. \fbib{for baldness}} and wearing sackcloth. \\
\poeml \v{13}But look! \\
\poemll    There is joy and festivity, \\
\poeml slaughtering of cattle \\
\poemll    and killing of sheep, \\
\poeml eating meat \\
\poemll    and drinking\fnote{\fbackref{22:13} So 1QIsa\textsuperscript{a} MT; 1QIsa\textsuperscript{c} reads \fbib{and they drink}} wine. \\
\poeml ``Let us eat and drink, you say, \\
\poemll    because we die tomorrow.'' \\
\poeml \v{14}``Nevertheless, the \divine{Lord} of the Heavenly Armies has revealed himself to my hearing: \\
\poeml ```Surely because of you\fnote{\fbackref{22:14} So 1QIsa\textsuperscript{a}; 4QIsa\textsuperscript{c} MT LXX lack \fbib{because of you}} \\
\poemll    this iniquity will not be forgiven \\
\poemlll       you until you die,' \\
\poeml says the Lord \divine{God} of the Heavenly Armies.''
\end{poetry}
\passage{The \divine{Lord} Rebukes Shebna}

\v{15}This is what the Lord \divine{God} of the Heavenly Armies\fnote{\fbackref{22:15} So 1QIsa\textsuperscript{a} MT; MT\textsuperscript{mss} LXX read \fbib{\divine{Lord} of the Heavenly Armies}} says:

``Come, go to this steward, to Shebna who is in charge of the household, and ask him: \v{16}`What are you doing here, and who are your relatives here\fnote{\fbackref{22:16} Lit. \fbib{whom do you have here}} that you could carve out a grave for yourself here---cutting out a tomb at the choicest location,\fnote{\fbackref{22:16} Lit. \fbib{at the height}} chiseling out a resting place for yourself out of solid rock? \v{17}Look Out! The \divine{Lord} is about to hurl you away violently, my strong fellow! He\fnote{\fbackref{22:17} So 1QIsa\textsuperscript{a}; 4QIsa\textsuperscript{a} 4QIsa\textsuperscript{b} MT LXX read \fbib{And he}} will fold you up completely, \v{18}rolling you up tightly like a ball and throwing you into a large country. There\fnote{\fbackref{22:18} So 4QIsa\textsuperscript{f}; 1QIsa\textsuperscript{a} 1QIsa\textsuperscript{b} 4QIsa\textsuperscript{a} MT read \fbib{To there}; cf. LXX} you will die, and there\fnote{\fbackref{22:18} So 4QIsa\textsuperscript{f}; 1QIsa\textsuperscript{a} 1QIsa\textsuperscript{b} 4QIsa\textsuperscript{a} MT read \fbib{and to there}; cf. LXX} your splendid chariots will lie. You're a disgrace to your master's house! \v{19}I will depose you from your office, ousting you\fnote{\fbackref{22:19} Lit. \fbib{he has ousted you}; so 1QIsa\textsuperscript{a}; 4QIsa\textsuperscript{f} MT LXX\textsuperscript{ms} read \fbib{he will oust you}; LXX lacks \fbib{he has ousted you}} from your position.

\v{20}``At that time,\fnote{\fbackref{22:20} Lit. \fbib{On that day}} I'll call for my servant, Hilkiah's\fnote{\fbackref{22:20} Lit. \fbib{Hilkyah}; so 1QIsa\textsuperscript{a} 4QIsa\textsuperscript{f}} son Eliakim, \v{21}and I'll clothe him with your robe and fasten your sash around him. I'll transfer your authority to him,\fnote{\fbackref{22:21} Lit. \fbib{to his hand}} and he'll be a father to those who live in Jerusalem and to the house of Judah.

\v{22}``I'll place on his shoulder the key to the house of David---what he opens, no one will shut, and what he shuts, no one will open. \v{23}I'll set him like a peg into a secure place; he will become a throne of honor to his father's house. \v{24}The entire reputation of his father's house will hang on him: its offspring and offshoots---all its smaller vessels, from the cups to all the jars. \v{25}At that time,''\fnote{\fbackref{22:25} Lit. \fbib{On that day}} declares the \divine{Lord}\fnote{\fbackref{22:25} So 1QIsa\textsuperscript{a} 4QIsa\textsuperscript{a} MT LXX; 4QIsa\textsuperscript{f} reads \fbib{the Lord \divine{God}}} of the Heavenly Armies, ``the peg that was driven into a secure place will give way; it will be sheared and will fall, and the load hanging on it will be cut down.''

The \divine{Lord} has spoken.
\labelchapt{23}
\passage{Tyre is Rebuked}

\chapt{23}
\v{1}A message\fnote{\fbackref{23:1} Lit. \fbib{An oracle}} concerning Tyre.

\begin{poetry}
\poeml ``Wail, you ships of Tarshish, \\
\poeml for Tyre is destroyed \\
\poemlll       and is without house or harbor! \\
\poeml From the land of Cyprus\fnote{\fbackref{23:1} Lit. \fbib{of the Kittim}} \\
\poeml it was revealed to them. \\
\poeml \v{2}``Be silent,\fnote{\fbackref{23:2} So 1QIsa\textsuperscript{a} 4QIsa\textsuperscript{a} MT; LXX reads \fbib{To whom are they like?}} you inhabitants of the coast, \\
\poemll    you merchants of Sidon, \\
\poemlll       whose messengers crossed over the sea,\fnote{\fbackref{23:2} So 1QIsa\textsuperscript{a} 4QIsa\textsuperscript{a} LXX; MT reads \fbib{you whom the merchants of Sidon, passing over the sea}, \fbib{have replenished}} \\
\poeml \v{3}and were on mighty waters. \\
\poemll    Her revenue was the grain of Shihor, \\
\poeml the harvest of the Nile; \\
\poemll    and she became the marketplace of nations. \\
\poeml \v{4}Be ashamed, Sidon, because the sea\fnote{\fbackref{23:4} So 1QIsa\textsuperscript{a}; 4QIsa\textsuperscript{a} MT read \fbib{for he}} has spoken, \\
\poemll    the fortress of the sea: \\
\poeml I have neither been in labor nor given birth, \\
\poemll    I have neither reared young men \\
\poemlll       nor brought up young women.'' \\
\poeml \v{5}When the news reaches Egypt, \\
\poemll    they will be in anguish \\
\poemlll       at the report about Tyre. \\
\poeml \v{6}``You who are crossing over\fnote{\fbackref{23:6} So 1QIsa\textsuperscript{a}; MT LXX read \fbib{Cross over}} to Tarshish--- \\
\poemll    Wail, you inhabitants of the coast! \\
\poeml \v{7}Is this your exciting\fnote{\fbackref{23:7} Or \fbib{happy}} city, \\
\poemll    that was founded long ago, \\
\poeml whose feet carried her \\
\poemll    to settle in far-off lands? \\
\poeml \v{8}Who has planned this \\
\poemll    against Tyre, \\
\poemlll       that bestower of crowns, \\
\poeml whose merchants were princes,\fnote{\fbackref{23:8} So 1QIsa\textsuperscript{a} corrector} \\
\poemll    whose traders were the most renowned on earth? \\
\poeml \v{9}The \divine{Lord} of the Heavenly Armies has planned it--- \\
\poemll    to neutralize all the hubris of grandeur,\fnote{\fbackref{23:9} So 1QIsa\textsuperscript{a} LXX; MT reads \fbib{the hubris of all grandeur}} \\
\poemlll       to discredit all the renowned men of earth. \\
\poeml \v{10}``Cultivate\fnote{\fbackref{23:10} Or \fbib{Worship}; so 1QIsa\textsuperscript{a} LXX; 4QIsa\textsuperscript{c} MT read \fbib{Pass through}} your land like the Nile, \\
\poemll    you daughter of Tarshish; \\
\poemlll       for there is no longer a harbor. \\
\poeml \v{11}He has stretched out his hand over the sea; \\
\poemll    he has made\fnote{\fbackref{23:11} So 1QIsa\textsuperscript{a} 4QIsa\textsuperscript{a} MT LXX; 4QIsa\textsuperscript{c} reads \fbib{to make}} kingdoms tremble. \\
\poeml The \divine{Lord} has issued orders concerning Canaan \\
\poemll    to destroy its strongholds. \\
\poeml \v{12}And he said: \\
\poeml `You will revel no longer,\fnote{\fbackref{23:12} So 1QIsa\textsuperscript{a} MT LXX; 4QIsa\textsuperscript{c} reads \fbib{won't take refuge to revel}; or \fbib{won't revel with gusto}} \\
\poemll    you virgin daughter of Sidon, \\
\poemlll       now crushed. \\
\poeml Get up, cross over to Cyprus--- \\
\poemll    but even there you will find no rest.'\,''
\end{poetry}
\passage{Tyre's Desolation and Restoration}

\begin{poetry}
\poeml \v{13}``Look at the land of the Chaldeans! \\
\poemll    This is a people that no longer exist; \\
\poeml Assyria destined her\fnote{\fbackref{23:13} I.e. Tyre} for desert creatures.\fnote{\fbackref{23:13} Or \fbib{demons}} \\
\poemll    They raised up her\fnote{\fbackref{23:13} So 1QIsa\textsuperscript{a}; MT reads \fbib{his}} siege towers, \\
\poeml they stripped her fortresses bare \\
\poemll    and turned her into a ruin. \\
\poeml \v{14}Wail, you ships of Tarshish, \\
\poemll    because your\fnote{\fbackref{23:14} So 1QIsa\textsuperscript{a} (sing.); MT LXX (pl.)} stronghold is destroyed!''
\end{poetry}

\v{15}It will happen at that time that Tyre will be forgotten for 70 years, the span of a king's life. Then, at the end of those 70 years, it will turn out for Tyre as in the prostitute's song:\fnote{\fbackref{23:15} So 1QIsa\textsuperscript{c} MT LXX; 1QIsa\textsuperscript{a} lacks \fbib{that Tyre will be forgotten for 70 years, the span of a king's life. Then, at the end of those 70 years}}

\begin{poetry}
\poeml \v{16}``Take a harp; \\
\poemll    walk around the city, \\
\poemlll       you forgotten whore! \\
\poeml Make sweet melody; \\
\poemll    sing many songs, \\
\poemlll       and perhaps you'll be remembered.''
\end{poetry}

\v{17}At the end of 70 years, the \divine{Lord} will deal with Tyre, at which time she'll return to her courtesan's trade, and prostitute herself with the kingdoms\fnote{\fbackref{23:17} So 1QIsa\textsuperscript{a}; MT LXX read \fbib{all the kingdoms}} of the world on the surface of the earth. \v{18}Nevertheless, her profits and her earnings will be dedicated\fnote{\fbackref{23:18} Or \fbib{given}} to the \divine{Lord}; they will not be stored up or hoarded---but her profits will go to those who live in the \divine{Lord}'s presence,\fnote{\fbackref{23:18} So 1QIsa\textsuperscript{a} MT; 4QIsa\textsuperscript{c} reads \fbib{but will be for those who live in the \divine{Lord}'s presence. And her profits will be}} for abundant food and choice clothing.
\labelchapt{24}
\passage{The Earth is Judged}

\begin{poetry}
\poeml \chapt{24}
\v{1}``Watch out! The \divine{Lord}\fnote{\fbackref{24:1} So 1QIsa\textsuperscript{a} MT; 4QIsa\textsuperscript{c} reads \fbib{Lord}} is about to depopulate the land \\
\poemll    and devastate it; \\
\poeml he will turn it upside down\fnote{\fbackref{24:1} Or \fbib{distort its surface}} \\
\poemll    and scatter its inhabitants. \\
\poeml \v{2}It will be the same for the lay people as for priests, \\
\poemll    the same for servants as for their masters, \\
\poeml for female servants as for their mistresses, \\
\poemll    for buyers as for sellers, \\
\poeml for lenders as for borrowers, \\
\poemll    and for creditors as for debtors. \\
\poeml \v{3}The earth will be utterly depopulated \\
\poemll    and completely laid waste --- \\
\poemlll       for the \divine{Lord} has spoken this message.\fnote{\fbackref{24:3} Lit. \fbib{word}} \\
\poeml \v{4}``The earth dries up and withers; \\
\poemll    the world languishes and fades away; \\
\poeml heaven fades away, \\
\poemll    along with the earth.\fnote{\fbackref{24:4} So 1QIsa\textsuperscript{a} 4QIsa\textsuperscript{c}; MT reads \fbib{the heavens fade away}; cf. LXX reads \fbib{the exalted ones of the earth mourn}} \\
\poeml \v{5}The earth lies defiled \\
\poemll    beneath its inhabitants; \\
\poeml because they have transgressed the laws,\fnote{\fbackref{24:5} So 1QIsa\textsuperscript{a} MT; 4QIsa\textsuperscript{c} LXX read \fbib{the Law}} \\
\poeml violated the statutes, \\
\poeml and broken the everlasting covenant. \\
\poeml \v{6}Therefore the curse keeps on consuming,\fnote{\fbackref{24:6} So 1QIsa\textsuperscript{a}; 4QIsa\textsuperscript{c} MT LXX read \fbib{consuming the earth}} \\
\poemll    and its inhabitants are declared guilty. \\
\poeml Furthermore, the inhabitants of earth are ablaze, \\
\poemll    and few people are left. \\
\poeml \v{7}The new wine evaporates; \\
\poemll    the vine and the oil\fnote{\fbackref{24:7} So 4QIsa\textsuperscript{c}; 1QIsa\textsuperscript{a} MT LXX lack \fbib{and the oil}} dry up; \\
\poemlll       all the merrymakers groan. \\
\poeml \v{8}``The celebrations of the tambourine have ended, \\
\poemll    the noise of the jubilant has stopped, \\
\poemlll       and the mirth that the harp produces has ended. \\
\poeml \v{9}No longer do they drink wine \\
\poemll    accompanied by singing; \\
\poemlll       even beer\fnote{\fbackref{24:9} Or \fbib{and strong drink}} tastes bitter to those who drink it. \\
\poeml \v{10}The chaotic city lies broken down; \\
\poemll    every house is closed up \\
\poemlll       so that no one can enter them.\fnote{\fbackref{24:10} 1QIsa\textsuperscript{a} MT lack \fbib{them}} \\
\poeml \v{11}There is an outcry in the streets over wine; \\
\poemll    all cheer turns to gloom; \\
\poemlll       the fun times of the earth are banished. \\
\poeml \v{12}Desolation remains in the city \\
\poemll    whose gates lie battered into ruins. \\
\poeml \v{13}So it will be on the earth \\
\poemll    and among the nations--- \\
\poeml as when an olive tree is beaten, \\
\poemll    or as gleanings when the grape harvest has ended.''
\passage{Glorifying God}
\poeml \v{14}``They raise their voices; \\
\poemll    they shout for joy;\fnote{\fbackref{24:14} So 1QIsa\textsuperscript{a} MT; 4QIsa\textsuperscript{c} reads \fbib{and they shout}} \\
\poeml from the west\fnote{\fbackref{24:14} Lit. \fbib{sea}; so 1QIsa\textsuperscript{a} MT; cf. LXX; 4QIsa\textsuperscript{c} reads \fbib{day}} they shout aloud\fnote{\fbackref{24:14} So 1QIsa\textsuperscript{a} MT; 4QIsa\textsuperscript{c} reads \fbib{And they cry out}} \\
\poemll    over the \divine{Lord}'s majesty. \\
\poeml \v{15}Therefore, you in the east,\fnote{\fbackref{24:15} So 1QIsa\textsuperscript{a} MT; 4Q1sa\textsuperscript{c} reads \fbib{in the east, in Aram}; LXX lacks \fbib{in the east}} \\
\poemll    give glory to the \divine{Lord}! \\
\poeml You in the coastlands of the sea, \\
\poemll    give glory to the name of the \divine{Lord} God of Israel! \\
\poeml \v{16}From the ends of the earth \\
\poemll    we hear songs of praise: \\
\poemlll       `Glory to the Righteous One!' \\
\poeml ``But I say, `I am pining away, \\
\poemll    I'm pining away. \\
\poemlll       How terrible things are for me! \\
\poeml For treacherous people betray--- \\
\poemll    treacherous people are betraying with treachery!'\,''
\passage{The Universal Impact of Judgment}
\poeml \v{17}``Terror and pit and snare are coming in your direction, \\
\poemll    you inhabitants of the earth! \\
\poeml \v{18}Whoever flees at the sound of terror \\
\poemll    will fall into a pit, \\
\poeml and whoever climbs out of the pit \\
\poemll    will be caught in a snare. \\
\poeml For the windows of judgment\fnote{\fbackref{24:18} 1QIsa\textsuperscript{a} MT lack \fbib{of judgment}} from above are opened, \\
\poemll    and the foundations of the earth are shaken. \\
\poeml \v{19}The earth is utterly shattered, \\
\poemll    the earth is split apart, \\
\poemlll       the earth is violently shaken. \\
\poeml \v{20}The earth\fnote{\fbackref{24:20} So 1QIsa\textsuperscript{a}; MT LXX read \fbib{Earth}} reels to and fro like a drunkard; \\
\poemll    it sways like a hut;\fnote{\fbackref{24:20} So MT; 1QIsa\textsuperscript{a} lacks \fbib{like a hut}} \\
\poeml its transgression lies so heavy upon it, \\
\poemll    that it falls, never to rise again. \\
\poeml \v{21}``And it will come about at that time,\fnote{\fbackref{24:21} Lit. \fbib{about on that day}} \\
\poemll    the \divine{Lord} will punish \\
\poeml the armies of the exalted ones in the heavens,\fnote{\fbackref{24:21} Or \fbib{ones on high}} \\
\poemll    and the rulers\fnote{\fbackref{24:21} Lit. \fbib{kings}} of the earth on earth. \\
\poeml \v{22}They\fnote{\fbackref{24:22} So 1QIsa\textsuperscript{a}; 4QIsa\textsuperscript{c} MT LXX read \fbib{And they}} will be herded together\fnote{\fbackref{24:22} So 1QIsa\textsuperscript{a} LXX; 4QIsa\textsuperscript{c} MT read \fbib{together like prisoners}} \\
\poemll    into the Pit;\fnote{\fbackref{24:22} I.e., the place of punishment in the afterlife; or \fbib{into a dungeon};} \\
\poeml they will be shut up in prison, \\
\poemll    and after many days they will be punished. \\
\poeml \v{23}Then the moon will be embarrassed \\
\poemll    and the sun ashamed, \\
\poeml for the \divine{Lord} of the Heavenly Armies will reign \\
\poemll    on Mount Zion and in Jerusalem; \\
\poeml and in the presence of its elders \\
\poemll    there will be glory.''
\end{poetry}
\labelchapt{25}
\passage{Praise to the Victorious God}

\begin{poetry}
\poeml \chapt{25}
\v{1}\divine{Lord}, you are my God; \\
\poemll    I will exalt you and praise your name, \\
\poeml for you have done marvelous things, \\
\poemlll       plans made long ago in faithfulness and truth. \\
\poeml \v{2}For you have made the city a heap of rubble, \\
\poemll    the fortified city into a ruin; \\
\poeml the foreigners' citadel\fnote{\fbackref{25:2} So 1QIsa\textsuperscript{a} MT; MT\textsuperscript{mss} LXX read \fbib{citadel of arrogant people}} is no longer a city--- \\
\poemll    it will never be rebuilt! \\
\poeml \v{3}Therefore strong peoples will glorify you; \\
\poemll    cities of ruthless nations will revere you. \\
\poeml \v{4}For you have been a stronghold for the poor, \\
\poemll    a stronghold for the needy in distress, \\
\poeml a shelter from the storm \\
\poemll    and a shade from the heat--- \\
\poeml for the blistering attack\fnote{\fbackref{25:4} 1QIsa\textsuperscript{a} MT lack \fbib{attack}} from the ruthless \\
\poemll    is like a rainstorm beating against a wall, \\
\poeml \v{5}and the noise of foreigners is like the heat of the desert. \\
\poemll    Just as you subdue heat by the shade of clouds, \\
\poemlll       so the victory songs of violent men will be stilled.
\passage{Celebration of the Righteous}
\poeml \v{6}``On this mountain,\fnote{\fbackref{25:6} I.e. Mount Zion; cf. 24:23} the \divine{Lord} of the Heavenly Armies will prepare for all peoples \\
\poemll    a banquet of rich food, \\
\poeml a banquet of well-aged wines--- \\
\poemll    rich food full of marrow, \\
\poemlll       and refined wines of the finest\fnote{\fbackref{25:6} 1QIsa\textsuperscript{a} MT lack \fbib{the finest}} vintage \\
\poeml \v{7}And on this mountain,\fnote{\fbackref{25:7} I.e. Mount Zion; cf. 24:23} he will swallow up \\
\poemll    the burial\fnote{\fbackref{25:7} So 1QIsa\textsuperscript{a}; the Heb. lacks \fbib{burial}} shroud that enfolds all peoples, \\
\poeml the veil that is spread over all nations--- \\
\poeml \v{8}he has swallowed up\fnote{\fbackref{25:8} So 1QIsa\textsuperscript{a} MT\textsuperscript{ms}; MT reads \fbib{And he will swallow up}; cf. LXX Syr Theodotian 1Cor 15:54} death forever! \\
\poeml Then the Lord \divine{God} will wipe away the tears from all faces, \\
\poemll    and he will take away the disgrace of his people \\
\poemlll       from the entire earth.'' \\
\poeml for the \divine{Lord} has spoken. \\
\poeml \v{9}``And you\fnote{\fbackref{25:9} So 1QIsa\textsuperscript{a} Syriac; 4QIsa\textsuperscript{c} MT read \fbib{he}; LXX reads \fbib{they}} will say at that time,\fnote{\fbackref{25:9} Lit. \fbib{say on that day}} \\
\poemll    `Look! It's the \divine{Lord}!\fnote{\fbackref{25:9} So 1QIsa\textsuperscript{a}; MT LXX lack \fbib{It's the \divine{Lord}}} This is our God! \\
\poeml We waited for him, \\
\poemll    and he saved us. \\
\poeml This is the \divine{Lord}! \\
\poemll    We waited for him, \\
\poeml so let us rejoice, \\
\poemll    and we will\fnote{\fbackref{25:9} So 1QIsa\textsuperscript{a}; MT reads \fbib{and let us}} be glad that he has saved us.''
\passage{The Misery of Moab}
\poeml \v{10}For the \divine{Lord}'s power\fnote{\fbackref{25:10} Lit. \fbib{hand}} will rest on this mountain,\fnote{\fbackref{25:10} I.e. Mount Zion; cf. 24:23} \\
\poemll    but the Moabites will be trodden down beneath him, \\
\poeml just as straw is trodden down \\
\poemll    in the slime of\fnote{\fbackref{25:10} Lit. \fbib{in the water of}; so 1QIsa\textsuperscript{a} MT; MT\textsuperscript{qere} reads \fbib{in}; LXX reads \fbib{in wagons}} a manure pit. \\
\poeml \v{11}They will spread out their hands in the thick of it, \\
\poemll    just as swimmers spread out their hands to swim, \\
\poeml but the \divine{Lord} will bring down their pride, \\
\poemll    together with the cleverness of their hands. \\
\poeml \v{12}He brings down the high fortifications of your walls \\
\poemll    and lays them low; \\
\poeml he will raze them\fnote{\fbackref{25:12} Or \fbib{reach}; so 1QIsa\textsuperscript{a}; 4QIsa\textsuperscript{c} MT read \fbib{he casts them}} to the ground, \\
\poemll    right down to the dust.
\end{poetry}
\labelchapt{26}
\passage{The Song of Redeemed Judah}

\begin{poetry}
\poeml \chapt{26}
\v{1}At that time,\fnote{\fbackref{26:1} Lit. \fbib{On that day}} people will sing this song\fnote{\fbackref{26:1} So 1QIsa\textsuperscript{a}; 1QIsa\textsuperscript{b} 4QIsa\textsuperscript{c} MT read \fbib{time, this song will be sung}; LXX reads \fbib{time, they will sing that song}} in the land of Judah: \\
\poeml ``We have a strong city; \\
\poemll    God crafts victory, \\
\poemlll       its walls and ramparts.\fnote{\fbackref{26:1} So 4QIsa\textsuperscript{c}; 1QIsa\textsuperscript{a} MT read \fbib{walls and ramparts}; LXX reads \fbib{wall and rampart}} \\
\poeml \v{2}Open your\fnote{\fbackref{26:2} So 1QIsa\textsuperscript{a}; MT LXX read \fbib{the}} gates, \\
\poemll    so the righteous nation that safeguards its faith may enter. \\
\poeml \v{3}You will keep perfectly peaceful\fnote{\fbackref{26:3} Lit. \fbib{peace, peace}; so 1QIsa\textsuperscript{a} MT; LXX Syr read \fbib{peace}} \\
\poemll    the one whose mind remains focused on you, \\
\poemlll       because he remains\fnote{\fbackref{26:3} So 1QIsa\textsuperscript{a} 1QIsa\textsuperscript{b} LXX; 4QIsa\textsuperscript{c} MT read \fbib{trusts}} in you. \\
\poeml \v{4}``Trust in the \divine{Lord} forever, \\
\poemll    for in the \divine{Lord God}\fnote{\fbackref{26:4} So 1QIsa\textsuperscript{a} MT; 4QIsa\textsuperscript{b} reads \fbib{the \divine{Lord} God}; LXX reads \fbib{the Lord, the Lord}} you have an everlasting rock. \\
\poeml \v{5}For he has made drunk\fnote{\fbackref{26:5} So MT 1QIsa\textsuperscript{a}; 1QIsa\textsuperscript{b} 4QIsa\textsuperscript{b} 4QIsa\textsuperscript{c} MT read \fbib{has brought low}; LXX reads \fbib{has humbled and brought down}} \\
\poemll    the inhabitants of the height, \\
\poemlll       the lofty city. \\
\poeml He lays it low\fnote{\fbackref{26:5} So 1QIsa\textsuperscript{a} LXX; MT reads \fbib{He levels it, he levels it}} to the ground \\
\poeml casting it down to the dust, \\
\poeml \v{6}by the feet of the oppressed who trample it,\fnote{\fbackref{26:6} So 1QIsa\textsuperscript{a} LXX; MT reads \fbib{The foot tramples it}} \\
\poeml by the footsteps of the needy.\fnote{\fbackref{26:6} So 1QIsa\textsuperscript{a} LXX; MT reads \fbib{oppressed}} \\
\poeml \v{7}``The path of the righteous is level; \\
\poemll    O Upright One,\fnote{\fbackref{26:7} So 1QIsa\textsuperscript{a} MT; 4QIsa\textsuperscript{c} reads \fbib{they go straight ahead}; LXX lacks this line} \\
\poemlll       you make safe\fnote{\fbackref{26:7} So 1QIsa\textsuperscript{a}; MT LXX read \fbib{smooth} or \fbib{you prepare}} the way of justice.\fnote{\fbackref{26:7} So 1QIsa\textsuperscript{a} 4QIsa\textsuperscript{c}; MT LXX read \fbib{of the righteous ones}} \\
\poeml \v{8}Yes, \divine{Lord}, in the path of your judgments we wait;\fnote{\fbackref{26:8} So 1QIsa\textsuperscript{a} LXX; MT reads \fbib{we wait for you}} \\
\poemll    your name and your Law\fnote{\fbackref{26:8} So 1QIsa\textsuperscript{a}; 4QIsa\textsuperscript{c} MT read \fbib{your renown}; cf. LXX} are the\fnote{\fbackref{26:8} So 1QIsa\textsuperscript{a} MT; 4QIsa\textsuperscript{b} reads \fbib{my}} soul's desire. \\
\poeml \v{9}My soul yearns for you in the night; \\
\poemll    my spirit within me searches for you. \\
\poeml For when your judgments come upon the earth, \\
\poemll    the world's inhabitants learn righteousness. \\
\poeml \v{10}If favor is shown to the wicked, \\
\poemll    they don't learn righteousness; \\
\poeml even in a land of uprightness they act perversely \\
\poemll    and do not perceive the majesty of the \divine{Lord}. \\
\poeml \v{11}``\divine{Lord}, your hand is lifted up, \\
\poemll    but they do not see it. \\
\poeml And\fnote{\fbackref{26:11} So 1QIsa\textsuperscript{a}; cf. LXX; the Heb. lacks \fbib{And}} let them see your zeal for your\fnote{\fbackref{26:11} Lit. \fbib{the}; so 1QIsa\textsuperscript{a}; the Heb. lacks \fbib{your}} people \\
\poemll    and be put to shame--- \\
\poemlll       yes, let the fire reserved for your enemies consume them! \\
\poeml \v{12}\divine{Lord}, you will decide\fnote{\fbackref{26:12} So 1QIsa\textsuperscript{a}; MT reads \fbib{prepare} or \fbib{give}; LXX reads \fbib{\divine{Lord}, give}} peace for us, \\
\poemll    for you have indeed accomplished \\
\poemlll       all our achievements for us. \\
\poeml \v{13}O \divine{Lord} our God, \\
\poemll    other lords besides you have ruled over us, \\
\poeml but through you alone \\
\poemll    we acknowledge your name. \\
\poeml \v{14}The dead won't live, \\
\poemll    and\fnote{\fbackref{26:14} So 1QIsa\textsuperscript{a} LXX; 1QIsa\textsuperscript{a} corrector MT lack \fbib{and}} the departed spirits won't rise--- \\
\poeml to that end, you punished and destroyed them, \\
\poemll    then locked away\fnote{\fbackref{26:14} So 1QIsa\textsuperscript{a}; MT LXX read \fbib{wiped out}} all memory of them. \\
\poeml \v{15}``But you have enlarged the nation,\fnote{\fbackref{26:15} So 1QIsa\textsuperscript{a} 4QIsa\textsuperscript{b}} \divine{Lord}; \\
\poemll    you have enlarged the nation.\fnote{\fbackref{26:15} So 1QIsa\textsuperscript{a}; MT\textsuperscript{ms} lacks this line} \\
\poeml You have gained honor; \\
\poemll    you have extended all the borders of the land. \\
\poeml \v{16}\divine{Lord}, they\fnote{\fbackref{26:16} So 1QIsa\textsuperscript{a} MT; MT\textsuperscript{mss} LXX\textsuperscript{mss} read \fbib{we;} LXX reads \fbib{I remembered you}} came to you in distress; \\
\poemll    they poured out their secret\fnote{\fbackref{26:16} So 1QIsa\textsuperscript{a}; 4QIsa\textsuperscript{b} MT read \fbib{out a magical}} prayer \\
\poemlll       when your chastenings were\fnote{\fbackref{26:16} So 1QIsa\textsuperscript{a}; MT LXX read \fbib{your chastening was}} afflicting\fnote{\fbackref{26:16} Lit. \fbib{upon}} them. \\
\poeml \v{17}Just as a pregnant woman writhes \\
\poemll    and cries out during her labor \\
\poeml when she is about to give birth, \\
\poemll    so were we because of you, \divine{Lord}. \\
\poeml \v{18}We were pregnant, writhing in pain, \\
\poemll    but we gave birth only to wind. \\
\poeml We have not won your\fnote{\fbackref{26:18} So 1QIsa\textsuperscript{a} LXX; MT lacks \fbib{your}} victory on earth, \\
\poeml nor have the inhabitants of the world been born.''
\passage{The Resurrection of the Dead}
\poeml \v{19}``But your dead will live; their bodies will rise. \\
\poeml Those who live in the dust will wake up and shout for joy!\fnote{\fbackref{26:19} So 1QIsa\textsuperscript{a}; MT reads \fbib{Wake up and shout for joy}, \fbib{you}; LXX reads \fbib{Those in the dust will rejoice, for}} \\
\poeml For your dew is like the dew of dawn, \\
\poeml and the earth will give birth to the dead. \\
\poeml \v{20}Come, my people, enter your rooms \\
\poemll    and shut your doors\fnote{\fbackref{26:20} So 1QIsa\textsuperscript{a} MT; MT\textsuperscript{qere} reads \fbib{door}} behind you. \\
\poeml Hide yourselves\fnote{\fbackref{26:20} So 1QIsa\textsuperscript{a}; MT reads \fbib{yourself}} for a little while \\
\poemll    until the fury has passed by. \\
\poeml \v{21}For see, the \divine{Lord} is coming from his place \\
\poemll    to punish the inhabitants of the earth for their sins; \\
\poeml the earth will reveal the blood that has been shed on it, \\
\poemll    and will no longer conceal its slain.''
\end{poetry}
\labelchapt{27}
\passage{Israel's Deliverance}

\chapt{27}
\v{1}At that time,\fnote{\fbackref{27:1} Lit. \fbib{On that day}} with his fierce, mighty, and powerful sword, the \divine{Lord} will punish the gliding serpent Leviathan---the coiling serpent Leviathan---and he will kill the dragon that's in the sea.

\begin{poetry}
\poeml \v{2}At that time,\fnote{\fbackref{27:2} Lit. \fbib{On that day}} \\
\poemll    ``A fermenting\fnote{\fbackref{27:2} So 1QIsa\textsuperscript{a}; MT LXX read \fbib{pleasant}} vineyard--- \\
\poemlll       sing about it! \\
\poeml \v{3}I, the \divine{Lord}, watch over it \\
\poemll    And I water it continuously. \\
\poeml I guard it night and day \\
\poemll    so no one can harm it. \\
\poeml \v{4}I am not angry. \\
\poemll    If only the vineyard\fnote{\fbackref{27:4} Lit. \fbib{only it}} could give me briers and thorns\fnote{\fbackref{27:4} So 1QIsa\textsuperscript{a}; MT reads \fbib{thorns}} to battle, \\
\poeml I would march against it, \\
\poemll    and\fnote{\fbackref{27:4} So 1QIsa\textsuperscript{a}; the Heb. lacks \fbib{and}} I would burn it all up. \\
\poeml \v{5}Or else let it lay claim to my protection; \\
\poemll    let it make peace with me, \\
\poemlll       yes, let it make peace with me.'' \\
\poeml \v{6}In times to come, Jacob will take root, \\
\poemll    and\fnote{\fbackref{27:6} So 1QIsa\textsuperscript{a} LXX; the Heb. lacks \fbib{and}} Israel will blossom, sprout shoots, \\
\poemlll       and fill the whole world with fruit. \\
\poeml \v{7}Has the \divine{Lord}\fnote{\fbackref{27:7} Lit. \fbib{Has he}} struck them down, \\
\poemll    just as he struck down those who struck them? \\
\poeml Or have they been killed, \\
\poemll    just as their killers were killed? \\
\poeml \v{8}Measure by measure,\fnote{\fbackref{27:8} Or \fbib{With war cries}} \\
\poemll    in their exile you contended with them; \\
\poeml with his fierce blast he removed them, \\
\poemll    as on a day when the east wind blows. \\
\poeml \v{9}By this, then, Jacob's guilt will be atoned for, \\
\poemll    and this will be the full harvest \\
\poemlll       that comes from the removal of his sin: \\
\poeml when he makes all the altar stones \\
\poemll    like pulverized chalkstones, \\
\poeml no Asherah\fnote{\fbackref{27:9} Or \fbib{sacred}} poles or incense altars will be left standing. \\
\poeml \v{10}For the fortified city stands desolate, \\
\poemll    a settlement abandoned and forsaken like the desert; \\
\poeml calves graze there, \\
\poemll    and there they lie down \\
\poemlll       and strip bare its branches. \\
\poeml \v{11}When its branches are dry, \\
\poemll    they are broken off, \\
\poeml and women come and kindle fires with them, \\
\poemll    since this is a people who show no consideration. \\
\poeml That is why the One who made them shows them no compassion; \\
\poemll    the One who created them shows them no mercy.
\end{poetry}
\passage{Assyria and Egypt Exiles Redeemed}

\v{12}At that time,\fnote{\fbackref{27:12} Lit. \fbib{On that day}} the \divine{Lord} will winnow grain from the Euphrates River\fnote{\fbackref{27:12} DSS MT lack \fbib{River}} channel to the Wadi\fnote{\fbackref{27:12} I.e. a seasonal stream or river that channels water during rain seasons but is dry at other times} of Egypt,\fnote{\fbackref{27:12} I.e. the southwestern-most border of ancient Philistia} and you will be gathered in one by one, O people of Israel. \v{13}Furthermore, at that time,\fnote{\fbackref{27:13} Lit. \fbib{Furthermore, on that day}} a great trumpet will be sounded, and those who were perishing in the land of Assyria and those who had been expelled\fnote{\fbackref{27:13} Or \fbib{exiled}} to the land of Egypt will come and worship the \divine{Lord} on his holy mountain at Jerusalem.
\labelchapt{28}
\passage{The Captivity of Ephraim}

\begin{poetry}
\poeml \chapt{28}
\v{1}How terrible it will be for that arrogant garland--- \\
\poeml the drunks of Ephraim! \\
\poeml How terrible it will be for that fading flower of his glorious beauty, \\
\poemll    which sits on the heads of people bloated with food,\fnote{\fbackref{28:1} Lit. \fbib{the valley of those grown fat}} \\
\poemlll       of people overcome with wine! \\
\poeml \v{2}Look! The \divine{Lord}\fnote{\fbackref{28:2} So 1QIsa\textsuperscript{a}; MT reads \fbib{Lord}} has one who is mighty and strong, \\
\poemll    like a hailstorm and destructive tempest, \\
\poeml like a storm of mighty, overflowing water--- \\
\poemll    and\fnote{\fbackref{28:2} So 1QIsa\textsuperscript{a}; the Heb. lacks \fbib{and}} he will give rest to the land. \\
\poeml \v{3}With hands\fnote{\fbackref{28:3} So 1QIsa\textsuperscript{a}; cf. LXX; or \fbib{He will throw them forcefully down to the ground}; cf. MT} and feet, that proud garland\fnote{\fbackref{28:3} So 1QIsa\textsuperscript{a} LXX; or \fbib{\v{3}Underfoot that proud garland}; cf. MT}--- \\
\poemll    those drunks of Ephraim---will be trampled. \\
\poeml \v{4}And that fading flower, his glorious beauty, \\
\poemll    which sits on the heads of people bloated with food, \\
\poeml will be like an early fig before summer--- \\
\poemll    whenever someone sees it, he swallows it \\
\poemlll       as soon as it's in his hand. \\
\poeml \v{5}At that time,\fnote{\fbackref{28:5} Lit. \fbib{On that day}} the \divine{Lord} of the Heavenly Armies will become a glorious crown, \\
\poemll    a beautiful diadem for the remnant of his people, \\
\poeml \v{6}and a spirit of justice to the one who sits in judgment, \\
\poemll    a source of strength to those who turn back the battle at the gate. \\
\poeml \v{7}These people also\fnote{\fbackref{28:7} 1QIsa\textsuperscript{a} lacks \fbib{people also}; MT lacks \fbib{people}} stagger from wine \\
\poemll    and reel from strong drink. \\
\poeml Priests and prophets stagger from strong drink; \\
\poemll    they're drunk from\fnote{\fbackref{28:7} Lit. \fbib{are devoured by}} wine; \\
\poeml they reel from strong drink, \\
\poemll    waver when seeing visions, \\
\poemlll       and stumble when rendering decisions. \\
\poeml \v{8}For all the tables are covered in vomit and filth, \\
\poemll    with no clean\fnote{\fbackref{28:8} 1QIsa\textsuperscript{a} MT lack \fbib{clean}} space left.
\passage{Misuse of God's Word}
\poeml \v{9}To whom will he teach knowledge, \\
\poemll    and to whom will he explain the message? \\
\poeml To children just weaned from milk? \\
\poemll    To those just taken from the breast? \\
\poeml \v{10}For it is: ``Do this and do that, \\
\poemll    do this and do that, \\
\poeml Line upon line, line upon line, \\
\poemll    a little here, a little there.'' \\
\poeml \v{11}Very well, then, through the mouths of foreigners\fnote{\fbackref{28:11} Or \fbib{through foreign lips}} \\
\poemll    and foreign languages \\
\poemlll       the \divine{Lord} will speak to this people \\
\poeml \v{12}to whom he said, \\
\poemll    ``This is the resting place, \\
\poemlll       so give rest to the weary''\,' \\
\poeml and, \\
\poemll    ``This is the place of repose''--- \\
\poemlll       but they would not listen. \\
\poeml \v{13}So, then, the message from the \divine{Lord} to them will become: \\
\poemll    ``Do this and do that, \\
\poemlll       do this and do that, \\
\poemll    line upon line, \\
\poemlll       line upon line, \\
\poemll    a little here, \\
\poemlll       a little there,'' \\
\poeml so that they will go, \\
\poemll    but fall backward, \\
\poemlll       and be injured, snared, and captured.
\passage{God's Precious Cornerstone}
\poeml \v{14}``Therefore hear\fnote{\fbackref{28:14} So 1QIsa\textsuperscript{a}; 4QIsa\textsuperscript{c} MT LXX read \fbib{hear} (pl.)} the message from the \divine{Lord}, you scoffers \\
\poemll    who rule this people that are in Jerusalem. \\
\poeml \v{15}Because you said: \\
\poemll    `We have entered into a covenant with death, \\
\poemlll       and we have an agreement with Sheol,\fnote{\fbackref{28:15} I.e. the place where the dead dwell in the afterlife} \\
\poemll    so when the overwhelming scourge makes its choice,\fnote{\fbackref{28:15} So 1QIsa\textsuperscript{a}; MT LXX read \fbib{sweeps by}} \\
\poemlll       it cannot reach us, \\
\poemll    since we have made lies our refuge \\
\poemlll       and have concealed ourselves inside falsehood' \\
\poeml \v{16}therefore this is what the \divine{Lord} God\fnote{\fbackref{28:16} So 1QIsa\textsuperscript{a} corrector} says: \\
\poemll    ``Look! I am laying\fnote{\fbackref{28:16} So 1QIsa\textsuperscript{a} 1QIsa\textsuperscript{b}; MT reads \fbib{I have laid}} a foundation stone in Zion, a tested stone, \\
\poemlll       a precious cornerstone for a sure\fnote{\fbackref{28:16} So 1QIsa\textsuperscript{a} MT; MT\textsuperscript{mss} LXX lack \fbib{sure}} foundation: \\
\poemll    Whoever believes firmly will not act hastily. \\
\poeml \v{17}And I will make justice the measuring line, \\
\poemll    and righteousness the plumb line; \\
\poeml hail will sweep away your refuge of lies, \\
\poemll    and floods\fnote{\fbackref{28:17} Lit. \fbib{waters}} will overflow your hiding place. \\
\poeml \v{18}``Then your covenant with death will be annulled, \\
\poemll    and your agreement with Sheol\fnote{\fbackref{28:18} I.e. the place where the dead dwell in the afterlife} will not stand; \\
\poeml when the overwhelming scourge sweeps by, \\
\poemll    you will be trampled by it. \\
\poeml \v{19}As often as it sweeps through, \\
\poemll    it will carry you away, \\
\poeml for it will sweep by morning after morning in the day; \\
\poemll    but understanding this message will bring sheer terror at night,\fnote{\fbackref{28:19} So 1QIsa\textsuperscript{a}; MT reads \fbib{by day and by night; and there will be sheer terror.}} \\
\poeml \v{20}because the bed is too short to stretch out on, \\
\poemll    and its blankets too narrow to wrap around oneself!
\passage{God is on Mount Perazim}
\poeml \v{21}For the \divine{Lord} will stand upon\fnote{\fbackref{28:21} So 1QIsa\textsuperscript{a} LXX; MT reads \fbib{as}} Mount Perazim,\fnote{\fbackref{28:21} I.e. a mountain near Jerusalem, perhaps the Mount of Olives; cf. 2Sam 6:8} \\
\poemll    he will rouse himself in\fnote{\fbackref{28:21} So 1QIsa\textsuperscript{a}; MT reads \fbib{as}} the Valley of Gibeon; \\
\poeml to carry out his work--- \\
\poemll    his strange deed, \\
\poeml and to perform his task--- \\
\poemll    his alien task! \\
\poeml \v{22}But as for you,\fnote{\fbackref{28:22} So 1QIsa\textsuperscript{a} LXX; MT reads \fbib{So now}} don't start mocking, \\
\poemll    or your chains will become tighter; \\
\poeml for I have heard from the \divine{Lord}\fnote{\fbackref{28:22} So 1QIsa\textsuperscript{a} MT\textsuperscript{mss} LXX Syr; MT reads \fbib{the \divine{Lord God}}} of the Heavenly Armies about destruction, \\
\poemll    and it is decreed against the whole land.
\passage{The God who Plows and Harvests}
\poeml \v{23}``Pay attention! \\
\poemll    Listen to what I have to say; \\
\poeml Pay attention, \\
\poemll    and hear my speech. \\
\poeml \v{24}Does he who plows for sowing plow all the time? \\
\poemll    Does he keep on breaking up and harrowing his field? \\
\poeml \v{25}When he has leveled its surface, \\
\poemll    he scatters caraway\fnote{\fbackref{28:25} Or \fbib{scatters black cumin}} and sows cumin, doesn't he? \\
\poeml He plants wheat in rows, \\
\poemll    barley in its designated place, \\
\poeml and feed for livestock\fnote{\fbackref{28:25} Lit. \fbib{and spelt}; i.e. a grass grown and used as fodder} around its borders,\fnote{\fbackref{28:25} So 1QIsa\textsuperscript{a}; MT reads \fbib{its border}; LXX reads \fbib{your borders}} doesn't he? \\
\poeml \v{26}His God instructs him regarding the correct way, \\
\poemll    directing him how to plant.\fnote{\fbackref{28:26} DSS MT lack \fbib{how to plant}} \\
\poeml \v{27}For caraway is not threshed with a sharp sledge, \\
\poemll    nor is a cart wheel rolled over cumin. \\
\poeml Instead, caraway is winnowed with a stick, \\
\poemll    and cumin with a rod. \\
\poeml \v{28}It\fnote{\fbackref{28:28} I.e. \fbib{grain}; so 1QIsa\textsuperscript{a} MT; 4QIsa\textsuperscript{k} reads \fbib{And it}} must be ground;\fnote{\fbackref{28:28} So 1QIsa\textsuperscript{a}; 4QIsa\textsuperscript{k} MT read \fbib{must be ground for bread}} \\
\poemll    one cannot keep threshing it forever. \\
\poeml Even if he drives his cart\fnote{\fbackref{8:28} So 1QIsa\textsuperscript{a}; 1QIsa\textsuperscript{a} corrector MT read \fbib{the wheel of his cart}} and horses over it, \\
\poemll    he cannot crush it. \\
\poeml \v{29}This insight also comes from the \divine{Lord} of the Heavenly Armies, \\
\poemll    who is distinguished\fnote{\fbackref{28:29} So 1QIsa\textsuperscript{a}; MT reads \fbib{wonderful}} in practical advice \\
\poemlll       and\fnote{\fbackref{28:29} So 1QIsa\textsuperscript{a}; the Heb. lacks \fbib{and}} magnificent in sound wisdom.''
\end{poetry}
\labelchapt{29}
\passage{Judgment is Coming to Jerusalem}

\begin{poetry}
\poeml \chapt{29}
\v{1}``How terrible it will be for you, Aruel, Aruel,\fnote{\fbackref{29:1} So 1QIsa\textsuperscript{a}; MT LXX read \fbib{Ariel, Ariel}; i.e. a nickname assigned by the prophet for Jerusalem} \\
\poemll    the city where David encamped! \\
\poeml Year after year, \\
\poemll    let your festivals run their cycle. \\
\poeml \v{2}Then I'll besiege Aruel,\fnote{\fbackref{29:2} So 1QIsa\textsuperscript{a}; MT LXX read \fbib{Ariel, Ariel}; i.e. an allusion to Jerusalem} \\
\poemll    and there will be sorrow and mourning; \\
\poemlll       she will become to me like an altar fireplace.\fnote{\fbackref{29:2} Lit. \fbib{an Ariel}; i.e. perhaps a pun on the name \fbib{Aruel}} \\
\poeml \v{3}Then I'll encamp against you like David,\fnote{\fbackref{29:3} So 4QIsa\textsuperscript{k} MT\textsuperscript{mss} LXX; 1QIsa\textsuperscript{a} MT read \fbib{you on all sides}} \\
\poemll    and I'll lay siege to you with towers, \\
\poeml raise siege works against you, \\
\poeml \v{4}and you will be brought down. \\
\poeml You will speak from the ground, \\
\poemll    and your speech will mumble from the dust. \\
\poeml Your voice will come ghostlike from the ground, \\
\poemll    and your speech will whisper from the dust. \\
\poeml \v{5}``But the hordes of your enemies\fnote{\fbackref{29:5} So 1QIsa\textsuperscript{a}; MT reads \fbib{foreigners}; LXX reads \fbib{the ungodly}} \\
\poemll    will become like fine dust, \\
\poemlll       and the hordes of tyrants like flying chaff. \\
\poeml Then suddenly, in an instant, \\
\poeml \v{6}you will be visited by the \divine{Lord} of the Heavenly Armies--- \\
\poeml with thunder, an earthquake, and great noise, \\
\poemll    with a windstorm, a tempest, \\
\poemlll       and flames from a devouring fire. \\
\poeml \v{7}Then the hordes of all the nations that fight against Aruel,\fnote{\fbackref{29:7} So 1QIsa\textsuperscript{a}; MT LXX read \fbib{Ariel, Ariel}; i.e. an allusion to Jerusalem} \\
\poemll    all that attack her and her fortification\fnote{\fbackref{29:7} So 1QIsa\textsuperscript{a}; MT reads \fbib{her mountain stronghold}; LXX reads \fbib{Jerusalem}} and besiege her, \\
\poemlll       will become like a dream, with its visions in the night--- \\
\poeml \v{8}as when a hungry man dreams--- \\
\poemll    he eats, but wakes up still hungry; \\
\poeml or when a thirsty man dreams--- \\
\poemll    he drinks, but wakes up faint, \\
\poemlll       with his thirst unquenched. \\
\poeml So will it be with the hordes of all the nations \\
\poemll    that fight against Mount Zion.
\passage{Blind to God's Words}
\poeml \v{9}``Act stupid! \\
\poemll    Be astonished! \\
\poeml Act blind, \\
\poemll    and be blind! \\
\poeml Be drunk,\fnote{\fbackref{29:9} So 1QIsa\textsuperscript{a}; MT reads \fbib{They have become drunk}} but not from\fnote{\fbackref{29:9} So 1QIsa\textsuperscript{a} LXX; MT lacks \fbib{from}} wine; \\
\poemll    stagger around,\fnote{\fbackref{29:9} So 1QIsa\textsuperscript{a}; MT reads \fbib{They stagger around}} but not from strong drink. \\
\poeml \v{10}For the \divine{Lord} has poured out upon you \\
\poemll    a spirit of deep sleep--- \\
\poeml he has closed your eyes, you prophets, \\
\poemll    he has covered your heads, you seers!''
\end{poetry}

\v{11}``And this entire vision has become for you like the words of a sealed book. When people give it to someone who can read, and say, `Read this, please,' he answers,\fnote{\fbackref{29:11} So 1QIsa\textsuperscript{a}; MT LXX read \fbib{he will answer}} `I cannot, because it is sealed.' \v{12}Or when they give the book\fnote{\fbackref{29:12} So 1QIsa\textsuperscript{a}; MT LXX read \fbib{the book will be given}} to someone who cannot read, and say, `Read this, please,' he answers,\fnote{\fbackref{29:12} So 1QIsa\textsuperscript{a}; MT LXX read \fbib{he will answer}} `I don't know how to read.'\,''
\passage{A Rebuke of Hypocritical Worship}

\begin{poetry}
\poeml \v{13}Then the Lord said: \\
\poeml ``Because these people draw near with their mouths \\
\poemll    and honor me with their lips, \\
\poemlll       but their hearts are far from me, \\
\poeml worship of me\fnote{\fbackref{29:13} So 1QIsa\textsuperscript{a}; MT reads \fbib{Their worship of me}} has become \\
\poemll    merely like\fnote{\fbackref{29:13} So 1QIsa\textsuperscript{a}; MT LXX lack \fbib{like}} rules taught by human beings. \\
\poeml \v{14}Therefore, watch out! \\
\poeml ``As for me,\fnote{\fbackref{29:14} So 1QIsa\textsuperscript{a} LXX; the Heb. lacks \fbib{as for me}} I will once again \\
\poemll    do amazing things with this people, \\
\poemlll       wonder upon wonder. \\
\poeml The wisdom of their wise men will perish, \\
\poemll    and the insights\fnote{\fbackref{29:14} So 1QIsa\textsuperscript{a}; MT LXX read \fbib{insight}} of their discerning men will stay hidden.''
\passage{A Rebuke to the Deceptive}
\poeml \v{15}``How terrible it will be for you who go to great depths \\
\poemll    to hide your plans from the \divine{Lord}, \\
\poeml you whose deeds have been\fnote{\fbackref{29:15} So 1QIsa\textsuperscript{a}; MT LXX read \fbib{deeds are} (or \fbib{will be})} done\fnote{\fbackref{29:15} DSS MT lack \fbib{done}} in the dark, \\
\poemll    and who say, `Who can see us? \\
\poemlll       Who has recognized\fnote{\fbackref{29:15} So 1QIsa\textsuperscript{a}; MT LXX read \fbib{recognizes}} us?' \\
\poeml \v{16}He has turned the tables on you\fnote{\fbackref{29:16} So 1QIsa\textsuperscript{a}; MT reads \fbib{You turn things upside down!}}--- \\
\poemll    as if the potter were thought to be like heat.\fnote{\fbackref{29:16} I.e. the fire in a kiln; so 1QIsa\textsuperscript{a}; MT LXX read \fbib{clay}} \\
\poeml Can what is made say of the one who made it, \\
\poemll    `He did not make me?' \\
\poeml Or can what is formed say of the ones\fnote{\fbackref{29:16} So 1QIsa\textsuperscript{a}; MT LXX read \fbib{one}} who formed it, \\
\poemll    `He has no skill?' \\
\poeml \v{17}``In a very little while, \\
\poemll    will not Lebanon be turned into a garden of fruit,\fnote{\fbackref{29:17} Lit. \fbib{into Carmel}} \\
\poemlll       and the garden of fruit\fnote{\fbackref{29:17} Lit. \fbib{and Carmel}} seem like a forest? \\
\poeml \v{18}On that day the deaf will hear \\
\poemll    the words of a scroll, \\
\poeml and out of gloom and darkness \\
\poemll    the eyes of the blind will see. \\
\poeml \v{19}The humble will again experience joy in the \divine{Lord}, \\
\poemll    and the poorest people will rejoice in the Holy One of Israel. \\
\poeml \v{20}For the ruthless will vanish, \\
\poemll    and mockers will disappear, \\
\poemlll       and all who have an eye for evil will be cut down--- \\
\poeml \v{21}those who make a person appear to be the offender in a lawsuit, \\
\poemll    who set a trap for someone \\
\poemlll       who is making his defense in court,\fnote{\fbackref{29:21} Lit. \fbib{in the gate}} \\
\poeml and push aside the innocent \\
\poemll    with specious arguments.
\end{poetry}

\v{22}``Therefore, this is what the \divine{Lord}, who redeemed Abraham, says concerning the house of Jacob:

\begin{poetry}
\poeml `No longer will Jacob be ashamed; \\
\poemll    no longer will his face grow pale. \\
\poeml \v{23}For when he sees in his midst his children, \\
\poemll    the work of my hands, \\
\poeml they will keep my name holy; \\
\poemll    they will sanctify the Holy One of Jacob \\
\poemlll       and stand in awe of the God of Israel. \\
\poeml \v{24}Moreover, those who go astray in spirit will gain\fnote{\fbackref{29:24} Lit. \fbib{discover}} understanding, \\
\poemll    and those who complain will accept instruction.'\,''
\end{poetry}
\labelchapt{30}
\passage{Foolish Trust in Egypt}

\begin{poetry}
\poeml \chapt{30}
\v{1}``Oh, you stubborn children,'' declares the \divine{Lord}, \\
\poeml ``who carry out plans--- \\
\poemll    but they are not mine, \\
\poeml and who make alliances--- \\
\poemll    but not by my Spirit, \\
\poemlll       piling sin upon sin. \\
\poeml \v{2}They set out to go down to Egypt, \\
\poemll    without asking my advice; \\
\poeml taking refuge in Pharaoh's protection, \\
\poemll    and seeking shelter in Egypt's shadow. \\
\poeml \v{3}But Pharaoh's protection will become your shame, \\
\poemll    and sheltering in Egypt's shadow your longing.\fnote{\fbackref{30:3} So 1QIsa\textsuperscript{a}; MT LXX read \fbib{disgrace}} \\
\poeml \v{4}And it will turn out that\fnote{\fbackref{30:4} So 1QIsa\textsuperscript{a}; MT LXX read \fbib{For even though}} his officials are at Zoan, \\
\poemll    and his envoys will reach Hanes. \\
\poeml \v{5}There is only loathsome destruction\fnote{\fbackref{30:5} So 1QIsa\textsuperscript{a}; MT reads \fbib{Everyone comes to shame}} \\
\poemll    through a people that cannot benefit them, \\
\poeml who bring neither help nor profit, \\
\poemll    but only shame and disgrace.''
\passage{The Animals of the Negev}
\poeml \v{6}An oracle about the animals of the Negev:\fnote{\fbackref{30:6} I.e. southern regions of the Sinai peninsula; cf. Josh 10:40} \\
\poeml ``Through a land of trouble, dryness,\fnote{\fbackref{30:6} So 1QIsa\textsuperscript{a}; MT LXX lack \fbib{dryness}; cf. Isa 41:18} and distress, \\
\poemll    of lionesses and roaring lions, \\
\poemlll       where there is no water,\fnote{\fbackref{30:6} So 1QIsa\textsuperscript{a}; MT LXX read \fbib{from whence come}} \\
\poeml a land of vipers and darting snakes, \\
\poemll    he carries\fnote{\fbackref{30:6} So 1QIsa\textsuperscript{a}; MT LXX read \fbib{they carry}} their riches on donkeys' backs, \\
\poeml and their treasures on the humps of camels, \\
\poemll    to a nation that cannot benefit them, \\
\poeml \v{7}to Egypt, which gives help that is worthless and useless. \\
\poemll    Therefore I call her, \\
\poemlll       `Rahab,\fnote{\fbackref{30:7} The Heb. word \fbib{Rahab} means \fbib{The One who Storms}; i.e. Egypt; cf. Isa 51:9; Ps 87:4} who just sits still.'\,''
\passage{The Illusions of False Prophecy}
\poeml \v{8}``Go now, and write it down\fnote{\fbackref{30:8} So 1QIsa\textsuperscript{a} MT; 4QIsa\textsuperscript{c} LXX read \fbib{write down}} on a tablet in their presence, \\
\poemll    inscribing it in a book, \\
\poeml so that for times to come \\
\poemll    it may be an everlasting witness. \\
\poeml \v{9}For they are a rebellious people, \\
\poemll    deceitful children, \\
\poeml children unwilling to hear \\
\poemll    the \divine{Lord}'s instruction. \\
\poeml \v{10}They say to the seers, \\
\poemll    `Don't see visions,' \\
\poeml and to the prophets, \\
\poemll    `Don't give us visions of what is right! \\
\poemlll       Instead, tell us welcome things, prophesy illusions, \\
\poeml \v{11}get out of the way, \\
\poemll    turn aside from the path, \\
\poemlll       and stop confronting us with the Holy One of Israel.''\fnote{\fbackref{30:11} Lit. \fbib{bring to an end the Holy One from before us}.}
\passage{Rejecting God's Message}
\poeml \v{12}Therefore, this is what the Holy One of Israel says: \\
\poeml ``Because you reject this message, \\
\poemll    and put your trust in oppression and enjoy it,\fnote{\fbackref{30:12} Apparent meaning 1QIsa\textsuperscript{a}; MT reads \fbib{and are perverse}} \\
\poemlll       and since you rely on it, \\
\poeml \v{13}therefore, for you this sin will become \\
\poemll    like a breach in a high wall that is about to collapse, \\
\poemlll       bulging out, \\
\poemll    and whose crash comes suddenly---in an instant. \\
\poeml \v{14}Its breaking will be like when potters' vessels are broken, \\
\poemll    shattered so ruthlessly\fnote{\fbackref{30:14} Lit. \fbib{broken---they do not take pity}; so 1QIsa\textsuperscript{a}; MT reads \fbib{broken---he does not take pity}} \\
\poeml that among its fragments not even a broken sliver will be found \\
\poemll    for taking fire from a hearth \\
\poemlll       or scooping water out of a cistern.'' \\
\poeml \v{15}For this is what the \divine{Lord}\fnote{\fbackref{30:15} So 1QIsa\textsuperscript{a}} \divine{God},\fnote{\fbackref{30:15} So 1QIsa\textsuperscript{a} corrector} the Holy One of Israel, says: \\
\poeml ``In repentance and rest you will be saved; \\
\poemll    in staying calm and trusting will be your strength. \\
\poemlll       But you refused. \\
\poeml \v{16}Instead, you said, \\
\poemll    `No! We'll escape on horses!' \\
\poemlll       Therefore, you'll flee away. \\
\poeml And you said, \\
\poemll    `We'll ride off on swift steeds!' \\
\poemlll       Therefore your pursuers will be swift. \\
\poeml \v{17}A thousand will flee at the threat of one; \\
\poemll    and run away, pursued by\fnote{\fbackref{30:17} So 1QIsa\textsuperscript{a}; MT LXX read \fbib{away at the threat of}} five, \\
\poeml until you are left \\
\poemll    like a flagpole on a mountaintop,\fnote{\fbackref{30:17} So 1QIsa\textsuperscript{a} LXX; MT reads \fbib{the mountaintop}} \\
\poemlll       like a banner on a hill.''
\passage{Restoration is Promised to Israel}
\poeml \v{18}``Nevertheless, the \divine{Lord} will wait \\
\poemll    so he can be gracious to you; \\
\poemlll       and thus he will rise up to show you mercy. \\
\poeml For the \divine{Lord} is a God of justice. \\
\poemll    How blessed are all those who wait for him.''
\end{poetry}

\v{19}Indeed, you people who live in Zion and in Jerusalem,\fnote{\fbackref{30:19} So 1QIsa\textsuperscript{a}; cf. LXX; MT reads \fbib{at Jerusalem}} you\fnote{\fbackref{30:19} So 1QIsa\textsuperscript{a} (pl.); MT (sing.)} will weep no more. How gracious the \divine{Lord}\fnote{\fbackref{30:19} So 1QIsa\textsuperscript{a}; MT LXX read \fbib{he}} will be to you at the sound of your cry! As soon as he hears it, he will answer you. \v{20}And although the \divine{Lord} gives you the bread of adversity and the water\fnote{\fbackref{30:20} So 1QIsa\textsuperscript{a}; MT lacks the correct Heb. construct} of affliction, your teachers won't hide themselves\fnote{\fbackref{30:20} So 1QIsa\textsuperscript{a}; MT reads \fbib{himself}} anymore, but your own eyes will see your teachers. \v{21}And whether you turn to the right or turn to the left, your ears will hear a message behind you: ``This is the way, walk in it.'' \v{22}Then you will defile your carved idols that are overlaid with silver and your images plated with gold. You'll throw them away like disgusting objects\fnote{\fbackref{30:22} Lit. \fbib{like menstrual rags}} and say to them, ``Away with you!''

\v{23}He will also provide rain for your seed that you sow in the ground, and the food that comes from the ground will be\fnote{\fbackref{30:23} So 1QIsa\textsuperscript{a} LXX; MT reads \fbib{and it will be}} rich and abundant. At that time,\fnote{\fbackref{30:23} Lit. \fbib{On that day}} your cattle will graze in broad meadows, \v{24}and oxen and donkeys that work the ground will eat seasoned\fnote{\fbackref{30:24} Lit. \fbib{salted}} fodder that workers will winnow with shovels and forks. \v{25}And on every lofty mountain and every high hill there will be brooks and canals\fnote{\fbackref{30:25} So 1QIsa\textsuperscript{a}; MT reads \fbib{streams}} running with water on the day of the great slaughter, when the towers fall.

\v{26}Moreover, the light of the moon will be like the light of the sun, and the sun's light will be seven times brighter, like the light of seven full days,\fnote{\fbackref{30:26} So 1QIsa\textsuperscript{a} MT; LXX lacks \fbib{like the light of seven full days}} when the \divine{Lord} binds up the bruises of his people and heals the wounds inflicted by his blow.

\begin{poetry}
\poeml \v{27}See, the name of the \divine{Lord} comes from far away, \\
\poemll    burning with his anger, and in thick rising smoke; \\
\poeml his lips are full of fury, \\
\poemll    and his tongue is like a devouring fire. \\
\poeml \v{28}His breath is like an overflowing torrent, \\
\poemll    and it rises right up to the neck, \\
\poeml to shake\fnote{\fbackref{30:28} So 1QIsa\textsuperscript{a}; MT reads \fbib{to sift}; LXX reads \fbib{to confuse}} the nations in the sieve of destruction, \\
\poemll    and to place in the jaws of the peoples a bit that leads them astray.
\end{poetry}

\v{29}You will have songs as on nights when people celebrate a holy festival,\fnote{\fbackref{30:29} So 1QIsa\textsuperscript{a}; MT reads \fbib{one celebrates a holy festival}} and gladness of heart, as when they set out with flutes to go to the \divine{Lord}'s mountain, to the Rock of Israel.
\passage{God's Judgment on Assyria}

\v{30}And the \divine{Lord} will make heard---yes, he will make heard\fnote{\fbackref{30:30} So 1QIsa\textsuperscript{a}; MT LXX read \fbib{heard only once}}---his majestic voice, and make his arm\fnote{\fbackref{30:30} I.e. \fbib{the Messiah}} seen descending in raging anger and in a flame of consuming fire, with a cloudburst, thunderstorm and hailstones. \v{31}Indeed, the Assyrians will be shattered at the \divine{Lord}'s voice, when he strikes them with his scepter. \v{32}And every stroke of his punishing rod\fnote{\fbackref{30:32} So MT\textsuperscript{mss}; 1QIsa\textsuperscript{a} reads \fbib{the rod of his foundation}; MT reads \fbib{the rod of foundation}} that the \divine{Lord} brings down on them will be to the sound of tambourines and harps, as he fights against her\fnote{\fbackref{30:32} So 1QIsa\textsuperscript{a} MT; MT\textsuperscript{qere, mss} read \fbib{against them}} in battle with a brandished arm.

\v{33}For the Fire Pit\fnote{\fbackref{30:33} Lit. \fbib{the Topheth}; i.e. a fire pit near Jerusalem where the Canaanite deity Molech was worshipped} has long been prepared; truly it is for the king; it will indeed be made ready.\fnote{\fbackref{30:33} So 1QIsa\textsuperscript{a}; MT reads \fbib{it is made ready for the king}; cf. LXX} And\fnote{\fbackref{30:33} So 1QIsa\textsuperscript{a}; MT lacks \fbib{And}} its pyre will be deep and wide, with abundant fire and wood. Like a stream of burning sulfur, the breath of the \divine{Lord} will set it ablaze.
\labelchapt{31}
\passage{Only the \divine{Lord} can Help}

\begin{poetry}
\poeml \chapt{31}
\v{1}``How terrible it will be for those who go down to\fnote{\fbackref{31:1} So 1QIsa\textsuperscript{a} LXX; the Heb. lacks \fbib{to}} Egypt for help, \\
\poemll    who rely on horses, \\
\poeml who trust in the chariot, \\
\poemll    because there are so many, \\
\poeml and in charioteers,\fnote{\fbackref{31:1} Or \fbib{horsemen}} \\
\poemll    because they are so strong--- \\
\poeml but do not look to\fnote{\fbackref{31:1} So 1QIsa\textsuperscript{a}; MT LXX read \fbib{upon}} the Holy One of Israel \\
\poemll    or seek the \divine{Lord}! \\
\poeml \v{2}Yet he is also wise and can bring disaster; \\
\poemll    he does not take back his words, \\
\poeml but will rise up against the house of those who practice evil \\
\poemll    and against anyone who assists people who work iniquity. \\
\poeml \v{3}The Egyptians are men, not God, \\
\poemll    and their horses are physical,\fnote{\fbackref{31:3} Or \fbib{flesh}} not spirit. \\
\poeml When the \divine{Lord} stretches out his hand, \\
\poemll    anyone who assists will stumble, \\
\poeml and the one who is helped will fall; \\
\poemll    and they will all perish together.''
\passage{The \divine{Lord} will Defend Jerusalem}
\poeml \v{4}For this is what the \divine{Lord} told me: \\
\poeml ``Just as a lion or a young lion growls over his objects of prey,\fnote{\fbackref{31:4} So 1QIsa\textsuperscript{a}; MT LXX read \fbib{his prey} (sing.)}--- \\
\poemll    even when a whole band of shepherds is called out against it, \\
\poeml it is not alarmed at their shouting \\
\poemll    or disturbed by their clamor--- \\
\poeml so the \divine{Lord} of the Heavenly Armies will come down \\
\poemll    to do battle on Mount Zion and on its hill. \\
\poeml \v{5}Like birds hovering overhead, \\
\poemll    so the \divine{Lord} of the Heavenly Armies will protect Jerusalem; \\
\poeml he will shield and deliver it; \\
\poemll    and\fnote{\fbackref{31:5} So 1QIsa\textsuperscript{a}; cf. LXX; the Heb. lacks \fbib{and}} he will pass over\fnote{\fbackref{31:5} I.e. as the Angel of Death passed over the Israelis; cf. Exod 12:13, 23, 27} and bring it to safety.\fnote{\fbackref{31:5} So 1QIsa\textsuperscript{a}; MT reads \fbib{rescue it}}
\end{poetry}

\v{6}Turn back to him, yes to him whom\fnote{\fbackref{31:6} So 1QIsa\textsuperscript{a}; MT reads \fbib{back to him whom}} your people\fnote{\fbackref{31:6} Lit. \fbib{whom they}} have so greatly betrayed, you people of Israel. \v{7}For at that time,\fnote{\fbackref{31:7} Lit. \fbib{on that day}} everyone will throw away their\fnote{\fbackref{31:7} Lit. \fbib{his}} idols of silver and their\fnote{\fbackref{31:7} Lit. \fbib{his}} idols of gold that your hands have sinfully made for yourselves.

\begin{poetry}
\poeml \v{8}``Then Assyria will fall by a sword \\
\poemll    that is not from human beings only\fnote{\fbackref{31:8} Lit. \fbib{not of man}}--- \\
\poemlll       a sword not wielded by mortal beings will devour them. \\
\poeml They will flee from the sword, \\
\poemll    and their young men will be put to forced labor. \\
\poeml \v{9}Their stronghold will vanish by reason of terror, \\
\poemll    and their commanders will be filled with alarm \\
\poemlll       because of the battle standard,'' \\
\poeml declares the \divine{Lord}, whose fire is in Zion \\
\poemll    and whose furnace is in Jerusalem.
\end{poetry}
\labelchapt{32}
\passage{The Government of Justice}

\begin{poetry}
\poeml \chapt{32}
\v{1}``Look, a king will reign in righteousness, \\
\poemll    and rulers will rule with justice. \\
\poeml \v{2}Each one will be like a shelter from the wind \\
\poemll    and a hiding place from\fnote{\fbackref{32:2} So 1QIsa\textsuperscript{a}; MT reads \fbib{of}} storms, \\
\poeml like streams of water in the desert, \\
\poemll    in\fnote{\fbackref{32:2} So 1QIsa\textsuperscript{a}; MT reads \fbib{like}} the shadow of a great rock in an exhausted\fnote{\fbackref{32:2} Or \fbib{thirsty}} land. \\
\poeml \v{3}Then the eyes of those who can see won't turn away, \\
\poemll    and the ears of those who can hear will listen. \\
\poeml \v{4}The hearts of reckless people will understand sound judgment, \\
\poemll    and the tongues of those who stammer will be ready to speak clearly. \\
\poeml \v{5}People will no longer call a fool\fnote{\fbackref{32:5} So 1QIsa\textsuperscript{a} LXX; MT reads \fbib{No longer will a fool will be called}} noble, \\
\poemll    nor will a bad person be declared honorable. \\
\poeml \v{6}For fools utter contempt, \\
\poemll    and their minds plot\fnote{\fbackref{32:6} So 1QIsa\textsuperscript{a} LXX; MT reads \fbib{work}} wrong things: \\
\poeml practicing ungodliness, \\
\poemll    spreading lies about the \divine{Lord}, \\
\poeml leaving the pangs of hungry people unsatisfied, \\
\poemll    and depriving thirsty people of drink. \\
\poeml \v{7}Furthermore, the crimes of bad people are evil; \\
\poemll    and\fnote{\fbackref{32:7} So 1QIsa\textsuperscript{a} LXX; the Heb. lacks \fbib{and}} they devise wicked schemes, \\
\poeml destroying the poor\fnote{\fbackref{32:7} 1QIsa\textsuperscript{a} and MT use two different synonyms} with lying words, \\
\poemll    even when needy people plead\fnote{\fbackref{32:7} So 1QIsa\textsuperscript{a}; cf. LXX; MT reads \fbib{a needy one pleads}} a just cause. \\
\poeml \v{8}But those who are decent plan noble things, \\
\poemll    and by noble deeds they stand.''
\passage{A Rebuke for Complacent Women}
\poeml \v{9}``As for you ladies of leisure--- \\
\poemll    Get up and listen to my voice! \\
\poeml You daughters who feel so complacent--- \\
\poemll    hear what I have to say! \\
\poeml \v{10}In little more than a year, \\
\poemll    you complacent women will shudder; \\
\poeml for the grape harvest will fail, \\
\poemll    and the fruit harvest will not\fnote{\fbackref{32:10} So 1QIsa\textsuperscript{a}; MT reads \fbib{without}} come. \\
\poeml \v{11}So tremble, you ladies of leisure! \\
\poemll    Shudder, you daughters who feel so complacent! \\
\poeml Strip down and make yourselves naked down to the waist!\fnote{\fbackref{32:11} Lit. \fbib{the loins}; so 1QIsa\textsuperscript{a} LXX; MT reads \fbib{to loins}} \\
\poemll    Then wrap yourself in\fnote{\fbackref{32:11} 1QIsa\textsuperscript{a} MT LXX lack \fbib{yourself in}} sackcloth and beat your breasts.\fnote{\fbackref{32:11} So 1QIsa\textsuperscript{a}; MT LXX lack \fbib{and beat your breasts}} \\
\poeml \v{12}For people will be beating their breasts \\
\poemll    in mourning\fnote{\fbackref{32:12} 1QIsa\textsuperscript{a} MT lack \fbib{mourning}} over the pleasant fields, \\
\poemlll       over the fruitful vines, \\
\poeml \v{13}and over the land of my people \\
\poemll    overgrown with thorns and\fnote{\fbackref{32:13} So 1QIsa\textsuperscript{a} LXX; the Heb. lacks \fbib{and}} briers--- \\
\poeml yes, over all the houses of merriment \\
\poemll    and over this city of revelry. \\
\poeml \v{14}``For the palace will be abandoned, \\
\poemll    the noisy city deserted; \\
\poeml the citadel and watchtower \\
\poemll    will become barren wastes forever, \\
\poeml the delight of wild donkeys, \\
\poemll    and a pasture for\fnote{\fbackref{32:14} So 1QIsa\textsuperscript{a}; MT reads \fbib{of}; cf. LXX} flocks, \\
\poeml \v{15}until the Spirit from on high is poured upon us, \\
\poemll    and the desert becomes a fertile field, \\
\poemlll       and the fertile field seems like a forest.''
\passage{Restoration of God's Reign}
\poeml \v{16}``Then justice will live in the wilderness, \\
\poemll    and righteousness will dwell in the fertile field. \\
\poeml \v{17}The effect of righteousness will be peace, \\
\poemll    and the result of righteousness will be quietness and confidence forever. \\
\poeml \v{18}My people will live in peaceful dwellings, \\
\poemll    in secure homes and in undisturbed resting places. \\
\poeml \v{19}But it will hail when the forest comes down, \\
\poemll    and the wood\fnote{\fbackref{32:19} So 1QIsa\textsuperscript{a}; MT reads \fbib{the city}; LXX lacks \fbib{the wood}} will be leveled completely. \\
\poeml \v{20}How happy you will be, sowing your seed beside every stream, \\
\poemll    and\fnote{\fbackref{32:20} So 1QIsa\textsuperscript{a}; the Heb. lacks \fbib{and}} letting your\fnote{\fbackref{32:20} Lit. \fbib{letting the feet of your}} cattle and donkeys range freely!''
\end{poetry}
\labelchapt{33}
\passage{God's Judgment}

\begin{poetry}
\poeml \chapt{33}
\v{1}``How terrible it will be for you, destroyer, \\
\poemll    you who have not been destroyed yourself! \\
\poeml And how terrible it will be for you, traitor, \\
\poemll    one whom\fnote{\fbackref{33:1} So 1QIsa\textsuperscript{a} MT; MT\textsuperscript{mss} read \fbib{you whom}} people have not betrayed! \\
\poeml When you have sunk so low in\fnote{\fbackref{33:1} So 1QIsa\textsuperscript{a}; MT reads \fbib{stopped}} destroying others, \\
\poemll    you will be destroyed; \\
\poeml and when you have finished betraying, \\
\poemll    they will betray you.''
\passage{A Prayer for Grace}
\poeml \v{2}``\divine{Lord}, be gracious to us; we long for you; \\
\poemll    and\fnote{\fbackref{33:2} So 1QIsa\textsuperscript{a}; the Heb. lacks \fbib{and}} be our strength\fnote{\fbackref{33:2} Lit. \fbib{arm}} every morning, \\
\poemlll       our salvation in times of trouble. \\
\poeml \v{3}At the thunder of your voice, the peoples flee; \\
\poemll    at your silence,\fnote{\fbackref{33:3} So 1QIsa\textsuperscript{a}; MT reads \fbib{when you rise up}; LXX reads \fbib{from fear of you}} the nations scatter. \\
\poeml \v{4}Your plunder is gathered as when grasshoppers gather; \\
\poemll    just like\fnote{\fbackref{33:4} So 1QIsa\textsuperscript{a}; the Heb. lacks \fbib{just like}} locusts pounce, people have pounced\fnote{\fbackref{33:4} So 1QIsa\textsuperscript{a}; MT reads \fbib{people pounce}} on it. \\
\poeml \v{5}``The \divine{Lord} is exalted, for he lives on high; \\
\poemll    he has filled Zion with justice and righteousness. \\
\poeml \v{6}He will be a sure foundation for your times, \\
\poemll    abundance and salvation,\fnote{\fbackref{33:6} So 1QIsa\textsuperscript{a}; MT reads \fbib{of salvation}} wisdom and knowledge --- \\
\poemlll       the fear of the \divine{Lord} is Zion's treasure.''
\passage{Israel's Unenviable Plight}
\poeml \v{7}``Listen! Their brave men cry out in the streets; \\
\poemll    the envoys of peace weep bitterly. \\
\poeml \v{8}The highways are deserted; \\
\poemll    travelers have quit the road. \\
\poeml The enemy\fnote{\fbackref{33:8} Lit. \fbib{He}} has broken treaties; \\
\poemll    he despises their witnesses,\fnote{\fbackref{33:8} So 1QIsa\textsuperscript{a}; MT reads \fbib{cities}} \\
\poemlll       and respects no one. \\
\poeml \v{9}The land mourns and wastes away; \\
\poemll    Lebanon feels ashamed and withers. \\
\poeml Sharon is like a desert; \\
\poemll    Bashan\fnote{\fbackref{33:9} So 1QIsa\textsuperscript{a}; MT reads \fbib{and Bashan}} and Carmel shake off their leaves.''
\passage{God is Exalted}
\poeml \v{10}``Now I'll rise up,'' the \divine{Lord} has said,\fnote{\fbackref{33:10} So 1QIsa\textsuperscript{a}; MT LXX read \fbib{says the \divine{Lord}}} \\
\poemll    ``now I'll exalt myself; \\
\poemlll       now I'll be lifted up. \\
\poeml \v{11}You conceive dried grass, you give birth to stubble; \\
\poemll    your breath is a fire that will consume you. \\
\poeml \v{12}And the peoples will be burned as if to ashes; \\
\poemll    like cut thorn bushes, they will be set ablaze. \\
\poeml \v{13}``Those who are far away have heard\fnote{\fbackref{33:13} So 1QIsa\textsuperscript{a}; cf. LXX; MT reads \fbib{You who are far away, hear}} what I've done; \\
\poemll    and those that are near have acknowledged\fnote{\fbackref{33:13} So 1QIsa\textsuperscript{a} LXX; MT reads \fbib{you that are near, acknowledge}} my power. \\
\poeml \v{14}The sinners in Zion are terrified; \\
\poemll    trembling grips the godless: \\
\poeml ``Who among us can live with the consuming fire? \\
\poemll    Who among us can live with everlasting flames?'' \\
\poeml \v{15}The one who walks righteously and has spoken\fnote{\fbackref{33:15} So 1QIsa\textsuperscript{a}; MT LXX read \fbib{and who speaks}} sincere words, \\
\poemll    who rejects gain from extortion \\
\poeml and waves his hand, \\
\poemll    rejecting bribes, \\
\poemll    who blocks his ears from hearing plots of murder \\
\poemlll       and shuts his eyes against seeing evil--- \\
\poeml \v{16}this is the one who will live on the heights; \\
\poemll    his refuge will be a mountain fortress. \\
\poeml His food will be supplied, \\
\poemll    and his water will be guaranteed. \\
\poeml \v{17}``Your eyes will see the king in his elegance, \\
\poemll    and will view a land that stretches afar. \\
\poeml \v{18}Your mind will ponder at that time of terror: \\
\poemll    `Where is the king's accountant? \\
\poeml Where is the one who weighed the revenue? \\
\poemll    Where is the officer who supervises\fnote{\fbackref{33:18} Lit. \fbib{counts}} the towers?' \\
\poeml \v{19}No longer will you\fnote{\fbackref{33:19} So 1QIsa\textsuperscript{a} (pl.); MT (sing.)} see those arrogant people, \\
\poemll    those people with their obscure speech you cannot comprehend, \\
\poemll    stammering in a language you cannot understand. \\
\poeml \v{20}``Look at Zion, city of our festivals!\fnote{\fbackref{33:20} So 1QIsa\textsuperscript{a} MT\textsuperscript{mss}; MT reads \fbib{our festival}} \\
\poemll    Your eyes will see Jerusalem, \\
\poemlll       an undisturbed abode, an immovable tent; \\
\poeml its stakes will never be pulled up, \\
\poemll    nor will any of its ropes be broken. \\
\poeml \v{21}But there the \divine{Lord} in majesty will be for us \\
\poemll    our source\fnote{\fbackref{33:21} Lit. \fbib{us a place}} of broad rivers and streams, \\
\poeml where no galley with oars can go, \\
\poemll    where no stately ship can sail. \\
\poeml \v{22}For the \divine{Lord} is our judge, \\
\poemll    and the \divine{Lord} is our lawgiver; \\
\poeml and the \divine{Lord} is our king, \\
\poemll    and it is he who will save us. \\
\poeml \v{23}``Your rigging hangs loose; \\
\poemll    it cannot reliably\fnote{\fbackref{33:23} So 1QIsa\textsuperscript{a}; MT reads \fbib{firmly}} hold the mast in its place, \\
\poemlll       and the sail cannot spread out.\fnote{\fbackref{33:23} So 1QIsa\textsuperscript{a}; MT reads \fbib{they cannot spread the sail}} \\
\poeml Then an abundance of spoils will be divided --- \\
\poemll    even the lame will carry off plunder. \\
\poeml \v{24}And no one living there will say, `I am ill.' \\
\poemll    The people living there will have their sins forgiven.''
\end{poetry}
\labelchapt{34}
\passage{Judgment of the Nations}

\begin{poetry}
\poeml \chapt{34}
\v{1}``Come near, you nations, to listen, \\
\poemll    and pay attention, you peoples! \\
\poeml Let the earth hear, and all that fills it; \\
\poemll    the world, and all that comes out of it. \\
\poeml \v{2}For the \divine{Lord} is angry against all the nations, \\
\poemll    and furious against all their armies. \\
\poeml He has doomed them to destruction, \\
\poemll    and\fnote{\fbackref{34:2} So 1QIsa\textsuperscript{a}; the Heb. lacks \fbib{and}} given them up to be slaughtered.\fnote{\fbackref{34:2} So 1QIsa\textsuperscript{a}; MT LXX read \fbib{for slaughter}} \\
\poeml \v{3}Their slain\fnote{\fbackref{34:3} Or \fbib{mortally wounded}} will be thrown out; \\
\poemll    and as for their dead bodies--- \\
\poeml their stench will ascend; \\
\poemll    the\fnote{\fbackref{34:3} So 1QIsa\textsuperscript{a}; the Heb. lacks \fbib{the}} mountains will be soaked with their blood. \\
\poeml \v{4}The valleys will be split, \\
\poemll    all the stars\fnote{\fbackref{34:4} Lit. \fbib{host}} in the heavens will fall down,\fnote{\fbackref{34:4} So 1QIsa\textsuperscript{a}; MT reads \fbib{All the stars of the heavens will rot away} ; LXX lacks this line} \\
\poemlll       and the skies will be rolled up like a scroll. \\
\poeml All their starry host will fade away \\
\poemll    like leaves withering on a vine, \\
\poemlll       or fruit withering on a fig tree. \\
\poeml \v{5}For my sword will be seen\fnote{\fbackref{34:5} So 1QIsa\textsuperscript{a}; MT reads \fbib{has drunk its fill}; cf. LXX} in the heavens. \\
\poemll    Look! It descends in judgment on Edom, \\
\poemlll       on the people I have doomed to destruction. \\
\poeml \v{6}The \divine{Lord} has a sword bathed in blood; \\
\poemll    it's covered\fnote{\fbackref{34:6} Or \fbib{satiated}} with fat, \\
\poeml with the blood of lambs and goats, \\
\poemll    and with fat from the kidneys of rams.''
\passage{Judgment on Bozrah and Edom}
\poeml ``For the \divine{Lord} holds a sacrifice in Bozrah, \\
\poemll    and a great slaughter in the land of Edom. \\
\poeml \v{7}Wild oxen will fall together with them--- \\
\poemll    young steers and mighty bulls. \\
\poeml Their land will be drenched\fnote{\fbackref{34:7} So 1QIsa\textsuperscript{a} LXX; MT reads \fbib{drench}} with blood, \\
\poemll    and their soil will be swollen with fat. \\
\poeml \v{8}For the \divine{Lord} has a day of vengeance, \\
\poemll    a year of recompense for Zion's cause. \\
\poeml \v{9}Edom's\fnote{\fbackref{34:9} Lit. \fbib{Its}} streams will be turned into burning sulfur, \\
\poemll    and its dust into sulfur; \\
\poemlll       its land will become pitch. \\
\poeml \v{10}It will burn night and day, \\
\poemll    and will never be extinguished. \\
\poeml Its smoke will rise from generation to generation, \\
\poemll    and it will lie desolate forever and ever. \\
\poemlll       And no one will pass through it.\fnote{\fbackref{34:10} So 1QIsa\textsuperscript{a}; MT reads \fbib{blazing pitch}. \fbib{\v{10}Night and day it will not be extinguished. Its smoke will go up forever; from generation to generation it will lie desolate. No one will ever pass through it again}. LXX lacks the last line} \\
\poeml \v{11}``But hawks and hedgehogs will possess it; \\
\poemll    owls and ravens will nest in it. \\
\poeml God\fnote{\fbackref{34:11} Lit. \fbib{He}} will stretch out over it a measuring line, and chaos,\fnote{\fbackref{34:11} So 1QIsa\textsuperscript{a}; MT reads \fbib{a measuring line of chaos}} \\
\poemll    and plumb lines of emptiness, and its nobles.\fnote{\fbackref{34:11} So 1QIsa\textsuperscript{a}; MT reads \fbib{And plumb lines of emptiness are} \fbib{\v{12}its} nobles; LXX reads \fbib{and satyrs will live in it}. \fbib{\v{12}Its nobles}} \\
\poeml \v{12}They will name it ``No Kingdom There,'' \\
\poemll    and all its princes will come to nothing. \\
\poeml \v{13}Thorns will grow over its palaces, \\
\poemll    nettles and brambles its fortresses. \\
\poeml It will become a haunt for jackals, \\
\poemll    a home for ostriches. \\
\poeml \v{14}And desert creatures will meet with hyenas, \\
\poemll    and goat-demons will call out to each other. \\
\poeml There also Liliths\fnote{\fbackref{34:14} I.e. desert demons of the night} will settle, \\
\poemll    and find for themselves\fnote{\fbackref{34:14} So 1QIsa\textsuperscript{a}; MT reads \fbib{The Lilith will settle, and find for itself}} a resting place. \\
\poeml \v{15}Owls\fnote{\fbackref{34:15} Or \fbib{tree-snakes}; LXX reads \fbib{Hedgehogs}} will nest there, lay eggs, \\
\poemll    hatch them, and care for their young \\
\poemlll       under the shadow of their wings;\fnote{\fbackref{34:15} Lit. \fbib{in her shadow}} \\
\poeml yes\fnote{\fbackref{34:15} So 1QIsa\textsuperscript{a}; MT lacks \fbib{yes}} indeed, vultures will gather there, \\
\poemll    each one with its mate.''
\passage{The Certainty of God's Deliverance}
\poeml \v{16}``Study and read from the book of the \divine{Lord}: \\
\poemll    And not one\fnote{\fbackref{34:16} So 1QIsa\textsuperscript{a}; MT reads \fbib{Not one of them}} will be missing, \\
\poemlll       each will not long for its mate.\fnote{\fbackref{34:16} So MT; 1QIsa\textsuperscript{a} reads \fbib{each its mate}} \\
\poeml For it is the mouth of the \divine{Lord} that has issued the order, \\
\poemll    and it is his Spirit that has gathered them. \\
\poeml \v{17}It is he who has allotted their portions,\fnote{\fbackref{34:17} Lit. \fbib{has cast the lot for them}} \\
\poemll    and his hand has divided it for them with a measuring line forever.\fnote{\fbackref{34:17} So 1QIsa\textsuperscript{a}; the Heb. lacks \fbib{forever}} \\
\poeml They will possess it forever;\fnote{\fbackref{34:17} So MT; 1QIsa\textsuperscript{a} lacks \fbib{Forever}} \\
\poemll    from generation to generation they will live in it.''\fnote{\fbackref{34:17} So MT; 1QIsa\textsuperscript{a} corrector reads \fbib{will they possess {\ldots} live in it}.}
\end{poetry}
\labelchapt{35}
\passage{The Future of Israel's Land}

\begin{poetry}
\poeml \chapt{35}
\v{1}``The desert and the dry land will rejoice; \\
\poemll    the desert will celebrate and blossom. Like crocuses, \\
\poeml \v{2}it will burst into bloom, \\
\poemll    and rejoice with gladness and shouts of joy. \\
\poeml The glory of Lebanon will be given to it, \\
\poemll    the splendor of Carmel and Sharon. \\
\poeml They will see the glory of the \divine{Lord}, \\
\poemll    the splendor of our God.\fnote{\fbackref{35:2} So MT LXX 1QIsa\textsuperscript{a} corrector; 1QIsa\textsuperscript{a} lacks vss. 1-2} \\
\poeml \v{3}Strengthen the feeble hands, \\
\poemll    and support the stumbling knees. \\
\poeml \v{4}Say to those with anxious hearts, \\
\poemll    `Be strong, do not be afraid! \\
\poeml Here is your God--- \\
\poemll    he will bring\fnote{\fbackref{35:4} So 1QIsa\textsuperscript{a} LXX; MT reads \fbib{he will come}} vengeance, \\
\poeml he will bring\fnote{\fbackref{35:4} So 1QIsa\textsuperscript{a}; MT LXX read \fbib{he will come}} divine retribution, \\
\poemll    and he will save you.' \\
\poeml \v{5}``Then the eyes of the blind will be opened, \\
\poemll    and the ears of the deaf unblocked; \\
\poeml \v{6}then the lame will leap like deer, \\
\poemll    and the tongues of speechless people will sing for joy. \\
\poeml Yes, waters will gush forth in the wilderness, \\
\poemll    and streams will run\fnote{\fbackref{35:6} So 1QIsa\textsuperscript{a}; MT LXX lack \fbib{will run}} through the desert; \\
\poeml \v{7}the burning sands will become a pool, \\
\poemll    and the thirsty ground fountains of water. \\
\poeml In the haunts of jackals there will be \\
\poemll    a verdant resting place with\fnote{\fbackref{35:7} So 1QIsa\textsuperscript{a}; MT reads \fbib{is her resting place; the grass will become}} reeds and rushes.''
\passage{God's Holy Highway}
\poeml \v{8}``A highway will be there---yes, there---\fnote{\fbackref{35:8} So 1QIsa\textsuperscript{a}; MT LXX lack \fbib{yes, there}} \\
\poemll    and people will call it\fnote{\fbackref{35:8} So 1QIsa\textsuperscript{a}; MT LXX read \fbib{it will be called}} `The Holy Way'.\fnote{\fbackref{35:8} So 1QIsa\textsuperscript{a} LXX; MT reads \fbib{Way, yes, Way}} \\
\poeml As for unclean people, \\
\poemll    they will not journey on it, \\
\poeml but it will be for whomever\fnote{\fbackref{35:8} So 1QIsa\textsuperscript{a}; MT reads \fbib{but it will be for the one}; cf. LXX} is traveling on that Way--- \\
\poemlll       not even fools will get lost. \\
\poeml \v{9}No lions will be there--- \\
\poemll    no---\fnote{\fbackref{35:9} So 1QIsa\textsuperscript{a}; MT LXX lack \fbib{no}} nor will any ferocious beasts get up on it, \\
\poemlll       and\fnote{\fbackref{35:9} So 1QIsa\textsuperscript{a} LXX; the Heb. lacks \fbib{and}} they will not be found there. \\
\poeml ``But the redeemed will walk there, \\
\poeml \v{10}and the \divine{Lord}'s ransomed ones will return \\
\poemlll       and enter Zion with singing. \\
\poeml Everlasting joy will rest upon their heads, \\
\poemll    gladness and joy will overtake them,\fnote{\fbackref{35:10} So 1QIsa\textsuperscript{a}; 1QIsa\textsuperscript{a} corrector lacks \fbib{them}; MT reads \fbib{they will attain gladness and joy}} \\
\poemlll       and sorrow and mourning will flee away.''
\end{poetry}
\labelchapt{36}
\passage{Sennacherib Attacks}

\chapt{36}
\v{1}In the fourteenth year of King Hezekiah,\fnote{\fbackref{36:1} The Heb. name \fbib{Hezekiah} is usu. spelled \fbib{Hizqiyah} in 1QIsa\textsuperscript{a}; 4QIsa\textsuperscript{b} MT spell the name \fbib{Hizqiyahu}.} King Sennacherib of Assyria attacked all the fortified cities of Judah and captured them. \v{2}Then the king of Assyria sent his field commander,\fnote{\fbackref{36:2} Or \fbib{sent Rab-shakeh}} along with a very\fnote{\fbackref{36:2} So 1QIsa\textsuperscript{a}; MT LXX lack \fbib{very}} large army, from Lachish to King Hezekiah at Jerusalem. When the field commander stopped at the aqueduct at the Upper Pool on the road to Laundryman's Field, \v{3}Hilkiah's son Eliakim, who was in charge of the palace, Shebna the secretary, and Asaph's son Joah, the recorder, went out to him.

\v{4}The field commander told them:

\begin{poetry}
\poeml ``Tell Hezekiah, king of Judah,\fnote{\fbackref{36:4} So 1QIsa\textsuperscript{a}; 1QIsa\textsuperscript{a} corrector deleted \fbib{king of Judah}; MT LXX lack \fbib{king of Judah}} `This is what the mighty king, the king of Assyria, has to say: What is this ``guarantee'' that makes you yourself\fnote{\fbackref{36:4} So 1QIsa\textsuperscript{a}; MT LXX lack \fbib{yourself}} rely on it?\fnote{\fbackref{36:4} So 1QIsa\textsuperscript{a}; MT LXX lack \fbib{on it}} \v{5}Do you really think that guarantees alone can withstand\fnote{\fbackref{36:5} Lit. \fbib{that words alone equal}} strategy and military strength? On whom are you now depending, that you're rebelling against me? \v{6}Take note: you're relying on Egypt, that splintered reed of a staff, which pierces the palm of anyone who leans on it. This is what Pharaoh king of Egypt is like to everybody who depends on him! \\
\poeml \v{7}But if you all\fnote{\fbackref{36:7} So 1QIsa\textsuperscript{a} LXX; MT reads \fbib{you} (sing.)} say to me, ``We are depending on the \divine{Lord} our God''---isn't he the one whose high places and altars Hezekiah removed, while he kept on telling Judah and Jerusalem, `You are to worship in front of this altar in\fnote{\fbackref{36:7} So 1QIsa\textsuperscript{a} MT; LXX lacks \fbib{while he kept on telling Judah and Jerusalem, `You are to worship in front of this altar in Jerusalem'}} Jerusalem'?\fnote{\fbackref{36:7} So 1QIsa\textsuperscript{a}; 1QIsa\textsuperscript{a} corrector deleted \fbib{in Jerusalem}; the Heb. lacks \fbib{in Jerusalem}} \v{8}Come now, all of you,\fnote{\fbackref{36:8} So 1QIsa\textsuperscript{a} LXX; MT reads \fbib{you} (sing.)} make a bet with my master, the king of Assyria: I will give you two thousand horses, if you can furnish riders for them! \v{9}How, then, can you repulse even one officer from\fnote{\fbackref{36:9} So 1QIsa\textsuperscript{a}; MT reads \fbib{one of}} the least of my master's officials, when you are depending for yourselves\fnote{\fbackref{36:9} So 1QIsa\textsuperscript{a}; MT reads \fbib{yourself}} on Egypt for chariots and horsemen? \v{10}One other thing: have I really marched against this country to destroy it apart from the \divine{Lord}'s direction?\fnote{\fbackref{36:10} 1QIsa\textsuperscript{a} MT lack `\fbib{s direction}} The \divine{Lord} himself ordered me, `March against this country to\fnote{\fbackref{36:10} So 1QIsa\textsuperscript{a}; MT reads \fbib{and}} destroy it.'\,''\fnote{\fbackref{36:10} So 1QIsa\textsuperscript{a} MT; LXX lacks \fbib{The \divine{Lord} himself ordered me, `March against this country to destroy it.'}}
\end{poetry}

\v{11}Then Eliakim, Shebna, and Joah replied to him,\fnote{\fbackref{36:11} So 1QIsa\textsuperscript{a} LXX; MT reads \fbib{to the field commander}} ``Please speak with\fnote{\fbackref{36:11} So 1QIsa\textsuperscript{a}; MT reads \fbib{to}} your servants---with us\fnote{\fbackref{6:11} So 1QIsa\textsuperscript{a}; MT LXX lack ---\fbib{with us---}}---in Aramaic, since we understand it. Don't speak to us in Hebrew\fnote{\fbackref{36:11} Lit. \fbib{in these words}; \fbib{s} o 1QIsa\textsuperscript{a}; MT LXX read \fbib{in the Judean language}} where the people sitting on\fnote{\fbackref{36:11} So 1QIsa\textsuperscript{a}; the Heb. lacks \fbib{sitting}; cf. LXX} the wall can hear.''

\v{12}But the field commander asked, ``Was it only to all of you and to your\fnote{\fbackref{36:12} So 1QIsa\textsuperscript{a} (pl.); MT reads \fbib{your} (sing.) \fbib{master and to you} (sing.)} master that my master sent me to speak these things? Wasn't it also to the men sitting on the wall---who, like you, will have to eat their own excrement and drink their own urine?''

\v{13}Then the\fnote{\fbackref{36:13} So 1QIsa\textsuperscript{a}; the Heb. lacks \fbib{the}} commander stood up and shouted out loud in Hebrew:\fnote{\fbackref{36:13} Or \fbib{the Judean language}}

\begin{poetry}
\poeml ``Hear the words of the great king, the king of Assyria! \v{14}This is what the king of Assyria\fnote{\fbackref{36:14} So 1QIsa\textsuperscript{a}; MT LXX lack \fbib{of Assyria}} says: `Don't let Hezekiah deceive you---for he cannot save you! \v{15}Don't let Hezekiah persuade you to rely on the \divine{Lord} when he says, ``The \divine{Lord} will really deliver\fnote{\fbackref{36:15} Or \fbib{save}} us!'' and\fnote{\fbackref{36:15} So 1QIsa\textsuperscript{a} LXX; MT lacks \fbib{and}} ``This city will never be handed over to the king of Assyria!'' \v{16}Don't listen to Hezekiah, because this is what the king of Assyria says: `Make your peace with me and come out to me. Then everyone will eat from his own vine and from his own fig tree, and everyone will drink water from his own cistern, \v{17}until I come and take you away to a land like your own land---to\fnote{\fbackref{36:17} So 1QIsa\textsuperscript{a}; the Heb. lacks \fbib{to}} a land of grain and new wine, a land of bread and vineyards.' \v{18}Be careful not to let Hezekiah mislead you when he says, ``The \divine{Lord} will save us.'' Has any god of any nation ever delivered\fnote{\fbackref{36:18} Or \fbib{saved}} his country from the\fnote{\fbackref{36:18} Lit. \fbib{the hand of the}} king of Assyria? \v{19}Where are the gods of Hamath and Arpad? Where are the gods of Sephar-vaim? Have they saved Samaria from me?\fnote{\fbackref{36:19} Lit. \fbib{from my hand}} \v{20}Who among all the gods of these countries has delivered\fnote{\fbackref{36:20} Or \fbib{saved}} their land from me?\fnote{\fbackref{36:20} Lit. \fbib{from my hand}} How then can the \divine{Lord} deliver\fnote{\fbackref{36:20} Or \fbib{saved}} Jerusalem from me?'\,''\fnote{\fbackref{36:20} Lit. \fbib{from my hand}}
\end{poetry}

\v{21}But the people remained silent and didn't respond to him with so much as a single word, because the king had commanded, ``Don't answer him.''

\v{22}Then Hilkiah's son Eliakim, who was in charge of the palace, Shebna the secretary, and Asaph's son Joah, the recorder, approached Hezekiah with their clothes torn,\fnote{\fbackref{36:22} I.e. as a symbol of pending disaster} and let him know what the field commander had said.
\labelchapt{37}
\passage{Hezekiah Seeks Isaiah's Counsel}

\chapt{37}
\v{1}As soon as Hezekiah the king\fnote{\fbackref{37:1} So 1QIsa\textsuperscript{a}; MT LXX read \fbib{the king Hezekiah}} heard this, he tore his clothes, dressed himself in sackcloth, and went into the \divine{Lord}'s Temple. \v{2}Then he sent Eliakim, who was in charge of the palace, Shebna the secretary, and the senior priests, all wearing sackcloth, to Amoz's son, the prophet Isaiah. \v{3}``Here is what Hezekiah says,'' they told him. ``This day is a day of trouble, rebuke, and disgrace, as when children come to the point of birth and there is no energy to deliver them. \v{4}Perhaps the \divine{Lord} your God will hear the words of the field commander, whom his master, the king of Assyria, sent to mock the living God, and perhaps he will rebuke the words that the \divine{Lord} your God has heard. So lift up a prayer for the remnant that still survives in this city.''\fnote{\fbackref{37:4} So 1QIsa\textsuperscript{a}; MT LXX lack \fbib{in this city}} \v{5}That's why King Hezekiah's officials came to Isaiah.
\passage{Isaiah Responds to Hezekiah}

\v{6}``Here is what to tell your master,'' Isaiah told them. ``This is what the \divine{Lord} says: `Don't be afraid of the words you've heard---those words with which the underlings of the king of Assyria have insulted me. \v{7}Watch this! I'm going to place an attitude\fnote{\fbackref{37:7} Or \fbib{to put a spirit}} within him,\fnote{\fbackref{37:7} So 1QIsa\textsuperscript{a}; MT LXX read \fbib{put a spirit in him}} so that when he hears a certain report, he'll return to his own country. Then I'll have him cut down by the sword in his own land.''\fnote{\fbackref{37:5-7} So MT LXX 1QIsa\textsuperscript{a} corrector; 1QIsa\textsuperscript{a} lacks vss. 5-7}
\passage{Sennacherib Retreats}

\v{8}So the field commander returned and found the king of Assyria fighting against Libnah, since he had heard that the king of Assyria\fnote{\fbackref{37:8} Lit. \fbib{that he}} had left Lachish. \v{9}Now King Sennacherib\fnote{\fbackref{37:9} Lit. \fbib{Now he}} had received this report concerning King Tirhakah of Cush: ``He has marched out to fight against you.''

When he heard it, he returned and\fnote{\fbackref{37:9} So 1QIsa\textsuperscript{a} LXX; cf. 2Kgs 19:9 MT; the Heb. lacks \fbib{returned and}} sent messengers to Hezekiah: \v{10}``Say this to Hezekiah king of Judah: `Don't let your God on whom you depend deceive you when he says, ``Jerusalem will not be handed over to the king of Assyria.'' \v{11}Surely you have heard what the kings of Assyria have done to all countries, dooming them to destruction. So do you think you will be saved? \v{12}Did the gods of the nations that were destroyed by my ancestors save them---the nations of Gozan, Haran, Rezeph, and the people of Eden, who were in Tel-assar? \v{13}Where is the king of Hamath, the king of Arpad, the king of the city of Sephar-vaim, or of Hena, or of Ivvah, or of Samaria?'\,''\fnote{\fbackref{37:13} So 1QIsa\textsuperscript{a}; MT LXX lack \fbib{or of Samaria}}
\passage{Hezekiah Prays}

\v{14}Hezekiah received the letters from the messengers, and read them.\fnote{\fbackref{37:14} So 1QIsa\textsuperscript{a}; MT LXX read \fbib{it}} Then he\fnote{\fbackref{37:14} Lit. \fbib{Hezekiah}} went up to the \divine{Lord}'s Temple and spread the letters\fnote{\fbackref{37:14} Lit. \fbib{it}} in front of the \divine{Lord}. \v{15}Hezekiah prayed to the \divine{Lord}:

\begin{poetry}
\poeml \v{16}``O \divine{Lord} of the Heavenly Armies, God of Israel, enthroned above the cherubim, you alone are the God of all the kingdoms of the earth. You made heaven and earth. \v{17}Extend your ear, \divine{Lord}, and listen! Open your eyes, \divine{Lord}, and look! Listen to all the words Sennacherib has sent to mock the living God. \v{18}It is true, \divine{Lord}, that Assyrian kings have devastated all these countries,\fnote{\fbackref{37:18} So 1QIsa\textsuperscript{a}; MT reads \fbib{countries and their land}; some MT\textsuperscript{mss} read \fbib{nations and their land}} \v{19}and have thrown their gods into the fire---but they are not gods, but rather the products\fnote{\fbackref{37:19} So 1QIsa\textsuperscript{a} LXX; MT reads \fbib{work}} of human hands, mere wood and stone. So the Assyrians\fnote{\fbackref{37:19} Lit. \fbib{So they}} destroyed them. \v{20}So now, \divine{Lord} our God, save us from his oppressive\fnote{\fbackref{37:20} 1QIsa\textsuperscript{a} LXX MT lack \fbib{oppressive}} hand, so that all kingdoms on earth may know that you alone, \divine{O Lord}, are God.''\fnote{\fbackref{37:20} So 1QIsa\textsuperscript{a}; MT reads \fbib{alone are \divine{Lord}}; LXX reads \fbib{alone are God}}
\end{poetry}
\passage{God's Answer}

\v{21}Then Amoz's son Isaiah sent this message to Hezekiah: ``This is what the \divine{Lord}, the God of Israel, says, to whom you prayed\fnote{\fbackref{37:21} So 1QIsa\textsuperscript{a}; MT reads \fbib{because you prayed to me}; cf. LXX} concerning Sennacherib king of Assyria. \v{22}This is the message that the \divine{Lord} has spoken in opposition to him:

\begin{poetry}
\poeml ```The Virgin Daughter of Zion \\
\poemll    despises and mocks you; \\
\poeml the Daughter of Jerusalem--- \\
\poemll    she tosses her head behind you as you flee. \\
\poeml \v{23}Whom have you insulted and reviled? \\
\poemll    Against whom have you raised your voice \\
\poeml and lifted your eyes in pride? \\
\poemll    Against the Holy One of Israel! \\
\poeml \v{24}By your messengers\fnote{\fbackref{37:24} Lit. \fbib{servants}} you have insulted the \divine{Lord}, \\
\poemll    and you have said, \\
\poeml ``With my many chariots \\
\poemll    I have climbed the heights of mountains, \\
\poemlll       the utmost heights of Lebanon. \\
\poeml I cut down its tallest cedars, \\
\poemll    the choicest of its pines; \\
\poeml I reached its remotest heights, \\
\poemll    the most verdant of its forests. \\
\poeml \v{25}I myself dug wells\fnote{\fbackref{37:25} So 1QIsa\textsuperscript{a}; MT reads \fbib{dug}; LXX reads \fbib{appointed}} \\
\poemll    and drank foreign\fnote{\fbackref{37:25} So 1QIsa\textsuperscript{a}; MT LXX lack \fbib{foreign}} waters; \\
\poeml with the soles of my feet \\
\poemll    I dried up all the streams of Egypt.'' \\
\poeml \v{26}```Didn't you hear \\
\poemll    how in the distant past I decided to do it, \\
\poemlll       how\fnote{\fbackref{37:26} So 1QIsa\textsuperscript{a}; MT reads \fbib{and how}} I planned from days of old? \\
\poeml Now I've made it happen--- \\
\poemll    that fortified cities become devastated, besieged heaps.\fnote{\fbackref{37:26} So 1QIsa\textsuperscript{a}; MT reads \fbib{you should make fortified cities crash into ruined heaps}} \\
\poeml \v{27}Their inhabitants are devoid of power, \\
\poemll    and are terrified and put to shame. \\
\poeml They've become like plants in the field, \\
\poemll    like\fnote{\fbackref{37:27} So 1QIsa\textsuperscript{a}; MT reads \fbib{and like}} green shoots, \\
\poeml like grass on rooftops, \\
\poemll    scorched by the east wind.\fnote{\fbackref{37:27} So 1QIsa\textsuperscript{a}; MT reads \fbib{and a field before the standing grain}} \\
\poeml \v{28}```I know when you rise up \\
\poemll    and\fnote{\fbackref{37:28} So 1QIsa\textsuperscript{a}; MT LXX lack \fbib{when you rise up and}} when you sit down, \\
\poeml your comings and goings--- \\
\poemll    and how you've become enraged at me. \\
\poeml \v{29}Your insolence\fnote{\fbackref{37:29} So 1QIsa\textsuperscript{a}; MT reads \fbib{because your raging against me and your insolence}; cf. LXX} has reached my ears, \\
\poemll    so I'll put my hook in your nose \\
\poemlll       and my bit in your mouth,\fnote{\fbackref{37:29} Lit. \fbib{lips}; so 1QIsa\textsuperscript{a} LXX; MT reads \fbib{lip}} \\
\poeml and I'll make you turn back on the road \\
\poemll    by which you came.
\end{poetry}

\v{30}``And this will be your sign, Hezekiah:\fnote{\fbackref{37:30} So 1QIsa\textsuperscript{a}; the Heb. lacks \fbib{Hezekiah}} Eat this year what grows on its own, and in the second year what springs from that. But in the third year sow, reap, plant vineyards, and eat their fruit. \v{31}Then the ones belonging to the house of Judah who have escaped will gather,\fnote{\fbackref{37:31} So 1QIsa\textsuperscript{a}; MT reads \fbib{be increased}} and those who are found\fnote{\fbackref{37:31} So 1QIsa\textsuperscript{a}; MT reads \fbib{and the remainder}; cf. LXX} will take root downward and bear fruit upward. \v{32}For a remnant will come out of Zion,\fnote{\fbackref{37:32} So 1QIsa\textsuperscript{a}; 4QIsa\textsuperscript{b} MT LXX read \fbib{Jerusalem}} and a band of survivors from Jerusalem.\fnote{\fbackref{37:32} So 1QIsa\textsuperscript{a}; 4QIsa\textsuperscript{b} MT LXX read \fbib{Mount Zion}} The zeal of the \divine{Lord} of the Heavenly Armies will accomplish this.

\v{33}``Therefore this what the \divine{Lord} says concerning the king of Assyria: `He won't enter this city, build up a siege ramp against it, shoot an arrow here, or threaten it with a shield.\fnote{\fbackref{37:33} So 1QIsa\textsuperscript{a}; MT reads \fbib{or shoot an arrow here, or threaten it with a shield, or build up a siege ramp against it}} \v{34}By the same way that he came, he will return; he won't enter this city,' declares the \divine{Lord}, \v{35}`because I will defend this city and deliver\fnote{\fbackref{37:35} Or \fbib{save}} it, for my own sake and for the sake of my servant David!'\,''
\passage{Sennacherib is Defeated}

\v{36}After this, the angel of the \divine{Lord} went out and put to death 185,000 men in the Assyrian camp. When Hezekiah's army\fnote{\fbackref{37:36} Lit. \fbib{When the people}} awakened in the morning---there were all the dead bodies!

\v{37}King Sennacherib broke camp, retreated, returned home to Nineveh, and remained there. \v{38}Later, while he was worshiping in\fnote{\fbackref{37:38} So 1QIsa\textsuperscript{a} LXX; the Heb. lacks \fbib{in}} the house of his god Nisroch, his sons Adrammelech and Sharezer cut him down with swords and escaped to the land of Ararat. Then Sennacherib's\fnote{\fbackref{37:38} Lit. \fbib{his}} son Esar-haddon reigned in his place.
\labelchapt{38}
\passage{Hezekiah's Illness and Recovery}

\chapt{38}
\v{1}During that time,\fnote{\fbackref{38:1} Lit. \fbib{During those days}} Hezekiah became ill and was at the point of death. Then Amoz's son Isaiah the prophet came to him and told him, ``This is what the \divine{Lord} says: `Put your house in order, because you are going to die. You won't recover.'\,''

\v{2}Then Hezekiah turned his face to the wall and prayed to the \divine{Lord}. \v{3}``Please, \divine{Lord},'' he said, ``Remember how I have walked before you faithfully and with a true heart, and I have done what pleases you.''\fnote{\fbackref{38:3} Lit. \fbib{done what is good in your eyes}} And Hezekiah wept bitterly.

\v{4}Then this message\fnote{\fbackref{38:4} Lit. \fbib{Then the word}} from the \divine{Lord} came to Isaiah: \v{5}``Go tell Hezekiah, `This is what the \divine{Lord} God of your ancestor David has to say: ``I've heard your prayer and\fnote{\fbackref{38:5} So 1QIsa\textsuperscript{a} LXX; the Heb. lacks \fbib{and}} I've seen your tears; so I will add fifteen years to your life. \v{6}I'll save you and this city from the\fnote{\fbackref{38:6} Lit. \fbib{the hand of the}} king of Assyria, and I'll defend this city, for my own sake and for my servant David's sake.\fnote{\fbackref{38:6} So 1QIsa\textsuperscript{a}; MT LXX lack \fbib{for my own sake and for my servant David's sake}} \v{7}This is the \divine{Lord}'s sign to you that the \divine{Lord} will carry out this thing he has promised: \v{8}Watch! I will make the shadow on the steps of the upper\fnote{\fbackref{38:8} So 1QIsa\textsuperscript{a}; the Heb. lacks \fbib{upper}} dial of Ahaz that marks the sun go ten steps backwards.''\,'\,''

Then the sunlight turned back on the dial the ten steps by which it had gone down.
\passage{Hezekiah's Prayer}

\v{9}A composition by King Hezekiah of Judah, following his illness and recovery:

\begin{poetry}
\poeml \v{10}I said, ``Must I leave in the prime of my life? \\
\poemll    Must I be consigned to the control\fnote{\fbackref{38:10} Lit. \fbib{gates}; i.e. the place where legal cases were adjudicated} of Sheol?\fnote{\fbackref{38:10} I.e. the realm of the afterlife} \\
\poemlll       Bitter are\fnote{\fbackref{38:10} So 1QIsa\textsuperscript{a}; MT LXX read \fbib{the rest of}} my years!'' \\
\poeml \v{11}I said, ``I won't see the \divine{Lord}\fnote{\fbackref{38:11} Lit. \fbib{Yah}; So 1QIsa\textsuperscript{a}; MT reads \fbib{Yah Yah}; MT\textsuperscript{mss} read \fbib{Lord}} in the land of the living; \\
\poemll    and\fnote{\fbackref{38:11} So 1QIsa\textsuperscript{a}; the Heb. lacks \fbib{and}} I'll no longer observe human beings \\
\poemlll       among the denizens of the grave.\fnote{\fbackref{38:11} Lit. \fbib{cessation}; or \fbib{the end}; So 1QIsa\textsuperscript{a} MT; MT\textsuperscript{mss} read \fbib{the world}} \\
\poeml \v{12}My house has been plucked up and vanishes\fnote{\fbackref{38:12} So 1QIsa\textsuperscript{a}; 1QIsa\textsuperscript{b} MT read \fbib{and has been taken away}} from me \\
\poemll    like a shepherd's tent; \\
\poeml like a weaver, I've taken account of\fnote{\fbackref{38:12} So 1QIsa\textsuperscript{a}; MT reads \fbib{have rolled up}} my life, \\
\poemll    and he cuts me off from the loom--- \\
\poemlll       day and night you make an end of me. \\
\poeml \v{13}I've been swept bare\fnote{\fbackref{8:13} So 1QIsa\textsuperscript{a}; or \fbib{I cried for help}; MT reads \fbib{I was composed}; cf. Targ} until morning; \\
\poemll    just like a lion, he breaks all my bones--- \\
\poemlll       day and night you make an end of me. \\
\poeml \v{14}Like a swallow or a crane I chirp, \\
\poemll    I moan like a dove. \\
\poeml My eyes look weakly upward. \\
\poemll    O Lord,\fnote{\fbackref{38:14} So 1QIsa\textsuperscript{a} MT; 1QIsa\textsuperscript{b} reads \fbib{\divine{Lord}}} I am oppressed, so\fnote{\fbackref{38:14} So 1QIsa\textsuperscript{a}; the Heb. lacks \fbib{so}} stand up for me! \\
\poeml \v{15}What can I say, so I tell myself,\fnote{\fbackref{38:15} So 1QIsa\textsuperscript{a}; MT reads \fbib{for he has spoken to me}} \\
\poemll    since he has done this to me?\fnote{\fbackref{38:15} So 1QIsa\textsuperscript{a}; MT reads \fbib{and it is he who has done it}} \\
\poeml I will walk slowly all my years \\
\poemll    because of my soul's anguish. \\
\poeml \v{16}``My Lord is against them, yet they live, \\
\poemll    and among all of them who live is his spirit.\fnote{\fbackref{38:16} So 1QIsa\textsuperscript{a}; MT reads \fbib{is the life of my spirit}} \\
\poeml Now you have restored me to health, \\
\poemll    so let me live! \\
\poeml \v{17}Yes, it was for my own good \\
\poemll    that I suffered extreme anguish.\fnote{\fbackref{38:17} So 1QIsa\textsuperscript{a}; 1QIsa\textsuperscript{b} MT read \fbib{bitter, bitter}} \\
\poeml But in love you have held back\fnote{\fbackref{38:17} So 1QIsa\textsuperscript{a}; cf. LXX; MT reads \fbib{you have loved}} my life \\
\poemll    from the Pit\fnote{\fbackref{38:17} I.e. the realm of punishment in the afterlife} in which it has been confined;\fnote{\fbackref{38:17} So 1QIsa\textsuperscript{a}; MT reads \fbib{pit of destruction}} \\
\poeml you have tossed all my sins \\
\poemll    behind your back. \\
\poeml \v{18}For Sheol\fnote{\fbackref{38:18} I.e. the realm of the afterlife} cannot thank you, \\
\poemll    death cannot\fnote{\fbackref{38:18} So 1QIsa\textsuperscript{a} LXX; implied in 1QIsa\textsuperscript{b} MT} sing your praise; \\
\poeml and\fnote{\fbackref{38:18} So 1QIsa\textsuperscript{a}; the Heb. lacks \fbib{and}} those who go down to the Pit\fnote{\fbackref{38:18} I.e. the realm of punishment in the afterlife} cannot hope \\
\poemll    for your faithfulness. \\
\poeml \v{19}The living---yes the living---they thank you, \\
\poemll    just as I am doing today; \\
\poeml fathers will tell their children \\
\poemll    about your faithfulness. \\
\poeml \v{20}The \divine{Lord} will save me,\fnote{\fbackref{38:20} At this point a later scribe inserted into 1QIsa\textsuperscript{a} a repetition of v. 19 and the beginning of v. 20, but with some different spellings and a word missing.} \\
\poemll    and we will play my music on strings \\
\poeml all the days of our lives \\
\poemll    in the \divine{Lord}'s Temple.\fnote{\fbackref{38:20} The same second scribe continued with the rest of this verse; not originally in 1QIsa\textsuperscript{a}.}
\end{poetry}

\v{21}Now Isaiah had said, ``Let them prepare\fnote{\fbackref{38:21} So MT; LXX reads \fbib{Take}; 1QIsa\textsuperscript{a} lacks \fbib{Let them prepare}} a poultice of figs and apply it to the boil, so that he may recover.''

\v{22}Hezekiah also had asked, ``What will be the sign for me to go up to the \divine{Lord}'s Temple?''\fnote{\fbackref{38:21-22} So 1QIsa\textsuperscript{b} MT LXX; 1QIsa\textsuperscript{a} lacks vs. 21-22; a later, third scribe, includes vs. 21-22}
\labelchapt{39}
\passage{The Visit by Merodach-baladan}

\chapt{39}
\v{1}At that time Merodach-baladan, the son of Baladan, king of Babylon, sent letters and a gift to Hezekiah, when\fnote{\fbackref{39:1} So 1QIsa\textsuperscript{a} 1QIsa\textsuperscript{b} MT; 4QIsa\textsuperscript{b} LXX read \fbib{because}} he heard he had been sick and had survived.\fnote{\fbackref{39:1} So 1QIsa\textsuperscript{a}; 1QIsa\textsuperscript{b} MT read \fbib{had recovered}} \v{2}Hezekiah was delighted with them, and showed them everything in\fnote{\fbackref{39:2} So 1QIsa\textsuperscript{a} MT\textsuperscript{mss}; the Heb. lacks \fbib{in}} his treasure-houses\fnote{\fbackref{39:2} So 1QIsa\textsuperscript{a}; MT LXX read \fbib{treasure-house}; MT\textsuperscript{qere} reads \fbib{his treasure-house}}---the silver, the gold, the spices, the precious oils, his entire armory, and everything found in his treasuries. There was nothing in his palace or in all his kingdom\fnote{\fbackref{39:2} So 1QIsa\textsuperscript{a}; MT reads \fbib{realm}; LXX lacks \fbib{kingdom}} that Hezekiah did not show them.
\passage{Isaiah Rebukes Hezekiah}

\v{3}Then the prophet Isaiah came to King Hezekiah and asked him, ``What did these men have to say? And from where did they come to you?''

Hezekiah replied, ``From a distant land---they came to me from Babylon.''

\v{4}``What did they see in your palace?'' he asked.

``They saw everything in my palace,'' Hezekiah replied. ``There is nothing in my treasuries that I did not show them.''

\v{5}Then Isaiah told Hezekiah, ``Listen to this message\fnote{\fbackref{39:5} Lit. \fbib{word}} from the \divine{Lord} of the Heavenly Armies: \v{6}`The days are surely coming when everything in your palace and all that your ancestors have stored up to this day will be carried off\fnote{\fbackref{39:6} So 1QIsa\textsuperscript{a} (pl.); cf. LXX; 1QIsa\textsuperscript{b} MT (sing.)} to Babylon. They will come in, and\fnote{\fbackref{39:6} So 1QIsa\textsuperscript{a} LXX; the Heb. lacks \fbib{and}} nothing will be left,' says the \divine{Lord}. \v{7}`Then some of your own sons, who will come from your loins,\fnote{\fbackref{39:7} So 1QIsa\textsuperscript{a}; 4QIsa\textsuperscript{b} MT read \fbib{from you}} whom you will father, will be taken away to become eunuchs in the palace of the king of Babylon.'\,''

\v{8}``The message from the \divine{Lord} that you have spoken is good,'' Hezekiah replied to Isaiah, since he was thinking, ``{\ldots}at least there will be peace and security in my lifetime.''
\labelchapt{40}
\passage{God Comforts His People}

\begin{poetry}
\poeml \chapt{40}
\v{1}``Comfort! Yes, comfort my people,'' \\
\poemll    says your God. \\
\poeml \v{2}``Speak tenderly to Jerusalem, \\
\poemll    and proclaim to her \\
\poeml that her heavy service has been completed, \\
\poemll    that her penalty has been paid, \\
\poeml that she has received from the \divine{Lord}'s hand \\
\poemll    double for all her sins.'' \\
\poeml \v{3}A voice cries out: \\
\poemll    `In the wilderness prepare the way for the \divine{Lord}; \\
\poemlll       and\fnote{\fbackref{40:3} So1QIsa\textsuperscript{a}; MT LXX lack \fbib{and}} in the desert a straight highway for our God.' \\
\poeml \v{4}Every valley will be lifted up, \\
\poemll    and every mountain and hill will be lowered; \\
\poeml the rough ground will become level, \\
\poemll    and the mountain ridges made a plain. \\
\poeml \v{5}Then the glory of the \divine{Lord} will be revealed, \\
\poemll    and all humanity will see it at once; \\
\poemlll       for the mouth of the \divine{Lord} has spoken.''
\passage{The Word of God Endures Forever}
\poeml \v{6}A voice says, ``Cry out!'' \\
\poemll    So I\fnote{\fbackref{40:6} So 1QIsa\textsuperscript{a} LXX; MT reads \fbib{he}} asked, ``What am I to cry out?'' \\
\poeml ``All humanity is grass, \\
\poemll    and all its loyalty\fnote{\fbackref{40:6} Or \fbib{glory}} is like the flowers of the field. \\
\poeml \v{7}Grass withers and flowers fade away \\
\poemll    when the \divine{Lord}'s breath blows on them; \\
\poemlll       surely the people are like grass.\fnote{\fbackref{40:7} So MT; 1QIsa\textsuperscript{a} LXX lack this v.} \\
\poeml \v{8}Grass withers and flowers fade away, \\
\poemll    when the \divine{Lord}'s breath blows on them, \\
\poemlll       but the word of\fnote{\fbackref{40:8} So MT LXX; 1QIsa\textsuperscript{a} lacks \fbib{the word of}} our God will stand forever. ``
\passage{Here is Your God}
\poeml \v{9}``Climb up a high mountain, \\
\poemll    you messenger of good news to Zion! \\
\poeml Lift up your voice with strength, \\
\poemll    you messenger to Jerusalem! \\
\poeml Lift it up! \\
\poemll    Don't be afraid! \\
\poeml Say to the towns of Judah, \\
\poemll    `Here is your God!' \\
\poeml \v{10}Look! The Lord \divine{God} comes with strength, \\
\poemll    and his arm\fnote{\fbackref{40:10} I.e. the Messiah} rules for him. \\
\poeml Look! His reward is with him, \\
\poemll    and his payment accompanies him. \\
\poeml \v{11}Like a shepherd, he tends his flock. \\
\poemll    He gathers the lambs in his arms, \\
\poeml carries them close to his heart, \\
\poemll    and gently leads the mother sheep.''
\passage{Who is Like the \divine{Lord}?}
\poeml \v{12}``Who has measured the waters of the sea\fnote{\fbackref{40:12} Lit. \fbib{from}} \\
\poemll    in the hollow of his hand \\
\poeml and marked off the heavens \\
\poemll    by the width of his hand?\fnote{\fbackref{40:12} So 1QIsa\textsuperscript{a} Syr; MT LXX read \fbib{by a hand's width}} \\
\poeml Who has enclosed the dust of the earth \\
\poemll    in a measuring bowl, \\
\poeml or weighed the mountains \\
\poemll    in scales \\
\poemlll       and the hills in a balance? \\
\poeml \v{13}Who has fathomed the Spirit of the \divine{Lord}, \\
\poemll    or as his counselor has taught him?\fnote{\fbackref{40:13} I.e. the Spirit; so 1QIsa\textsuperscript{a}; MT reads \fbib{the \divine{Lord}}} \\
\poeml \v{14}With whom did he consult \\
\poemll    to enlighten and instruct him on the path of justice? \\
\poeml Or who taught him knowledge \\
\poemll    and showed him the way of wisdom? \\
\poeml \v{15}``Look! The nations are like a drop in a bucket, \\
\poemll    and are reckoned as dust on the scales. \\
\poeml Look! He even lifts up the islands like powder! \\
\poeml \v{16}Lebanon would not provide enough fuel, \\
\poemll    nor are its animals enough for a burnt offering.\fnote{\fbackref{40:14}b-16 So MT LXX; 1QIsa\textsuperscript{a} includes these lines by a later scribe} \\
\poeml \v{17}All the nations are as nothing before him--- \\
\poemll    they are reckoned by him as\fnote{\fbackref{40:17} So 1QIsa\textsuperscript{a} LXX; MT reads \fbib{as less than}} nothing and chaos. \\
\poeml \v{18}``To whom, then, will you compare me,\fnote{\fbackref{40:18} So 1QIsa\textsuperscript{a}; MT LXX lack \fbib{me}} the One who is\fnote{\fbackref{40:18} 1QIsa\textsuperscript{a} LXX MT lack \fbib{the One who is}} God? \\
\poemll    Or to what image will you liken me?\fnote{\fbackref{40:18} So 1QIsa\textsuperscript{a}; MT LXX read \fbib{him}} \\
\poeml \v{19}To an idol? A craftsman makes\fnote{\fbackref{40:19} So 1QIsa\textsuperscript{a} LXX; MT reads \fbib{casts}} the image, \\
\poemll    and a goldsmith overlays it with gold \\
\poemlll       and casts silver chains. \\
\poeml \v{20}To the impoverished person? \\
\poemll    He prepares\fnote{\fbackref{40:20} 1QIsa\textsuperscript{a} MT LXX lack \fbib{prepares}} an offering---\fnote{\fbackref{40:20} So MT; later 1QIsa\textsuperscript{a} scribe includes this line} \\
\poemlll       wood that won't rot--- \\
\poeml Or to the one who chooses a skilled craftsman\fnote{\fbackref{40:20} So 1QIsa\textsuperscript{a}; MT LXX read \fbib{chooses wood}} \\
\poemll    and\fnote{\fbackref{40:20} So 1QIsa\textsuperscript{a}; MT lacks \fbib{and}} seeks\fnote{\fbackref{40:20} So 1QIsa\textsuperscript{a}; MT reads \fbib{seeks a skilled craftsman}} to erect an idol that won't topple?''
\passage{The Majesty of the \divine{Lord}}
\poeml \v{21}``You know, don't you? \\
\poemll    You have heard, haven't you? \\
\poeml Hasn't it been told you from the beginning? \\
\poemll    Haven't you understood from the foundations of the\fnote{\fbackref{40:21} So MT; implied in 1QIsa\textsuperscript{a}} earth? \\
\poeml \v{22}He's the one who sits above the disk of the earth, \\
\poemll    and its inhabitants are like grasshoppers. \\
\poeml He's the one who stretches out the heavens like a curtain, \\
\poemll    and spreads them like a tent to live in, \\
\poeml \v{23}who brings princes to nothing, \\
\poemll    and makes void the rulers of the earth. \\
\poeml \v{24}No sooner are they planted, \\
\poemll    no sooner are they sown, \\
\poemlll       no sooner have\fnote{\fbackref{40:24} So 1QIsa\textsuperscript{a}; 4QIsa\textsuperscript{b} MT LXX read \fbib{has}} their stems taken root in the earth, \\
\poeml than\fnote{\fbackref{40:24} So 1QIsa\textsuperscript{a}; 4QIsa\textsuperscript{b} MT read \fbib{and then}; LXX lacks \fbib{than}} he blows on them, and they wither, \\
\poemll    and the tempest sweeps them away like stubble. \\
\poeml \v{25}``To\fnote{\fbackref{40:25} So 1QIsa\textsuperscript{a}; 4QIsa\textsuperscript{b} MT read \fbib{And to}} whom, then, will you compare me, \\
\poemll    and to whom should I be equal?'' \\
\poemlll       asks the Holy One. \\
\poeml \v{26}``Lift your eyes up to heaven and see \\
\poemll    who created all these--- \\
\poeml the one who leads out their vast array of stars by number, \\
\poemll    calling them all by name--- \\
\poeml because of his great might \\
\poemll    and his\fnote{\fbackref{40:26} So 1QIsa\textsuperscript{a}; MT LXX lack \fbib{his}} powerful strength\fnote{\fbackref{40:26} So 1QIsa\textsuperscript{a} LXX; MT reads \fbib{strong}}--- \\
\poemlll       and\fnote{\fbackref{40:26} So 1QIsa\textsuperscript{a}; MT LXX lack \fbib{and}} not one is missing.''
\passage{The \divine{Lord} Watches Israel}
\poeml \v{27}``Jacob, why do you say--- \\
\poemll    and Israel, why do you complain--- \\
\poeml `My predicament is hidden from the \divine{Lord}, \\
\poemll    and my cause is ignored by my God.'? \\
\poeml \v{28}Don't you know? \\
\poemll    Haven't you heard? \\
\poeml The \divine{Lord} is the eternal God, \\
\poemlll       the Creator of the ends of the earth. \\
\poeml He does not grow tired or weary; \\
\poemll    and\fnote{\fbackref{40:28} So 1QIsa\textsuperscript{a} LXX; the Heb. lacks \fbib{and}} his understanding cannot be fathomed. \\
\poeml \v{29}He's the\fnote{\fbackref{40:29} So 1QIsa\textsuperscript{a}; the Heb. lacks \fbib{The}} one who gives might to the faint, \\
\poemll    renewing strength for the powerless. \\
\poeml \v{30}Even boys grow tired and weary, \\
\poemll    and young men collapse and fall, \\
\poeml \v{31}but those who keep waiting for the \divine{Lord} will renew their strength. \\
\poemll    Then\fnote{\fbackref{40:31} So 1QIsa\textsuperscript{a}; MT LXX lack \fbib{Then}} they'll soar on wings like eagles; \\
\poeml they'll run and not grow weary; \\
\poemll    they'll walk and not grow tired.''
\end{poetry}
\labelchapt{41}
\passage{The \divine{Lord} Comes as Judge}

\begin{poetry}
\poeml \chapt{41}
\v{1}``Be silent before me, you coastlands, \\
\poemll    and let the people renew their strength! \\
\poeml Let them come forward, \\
\poemll    then let them speak together--- \\
\poemlll       let's draw near for a ruling. \\
\poeml \v{2}Who has aroused victory from the east, \\
\poemll    and\fnote{\fbackref{41:2} So 1QIsa\textsuperscript{a}; MT LXX lack \fbib{and}} has summoned it to his service, \\
\poemlll       and\fnote{\fbackref{41:2} So 1QIsa\textsuperscript{a}; the Heb. lacks \fbib{and}} has handed over nations to him? \\
\poeml Who brings down kings, \\
\poemll    and\fnote{\fbackref{41:2} So 1QIsa\textsuperscript{a}; the Heb. lacks \fbib{and}} turns them into dust with his sword, \\
\poemlll       into windblown stubble with his bow? \\
\poeml \v{3}And\fnote{\fbackref{41:3} So 1QIsa\textsuperscript{a} LXX; the Heb. lacks \fbib{and}} who pursues them \\
\poemll    and\fnote{\fbackref{41:3} So 1QIsa\textsuperscript{a}; MT LXX lack \fbib{and}} moves on unscathed \\
\poemlll       by a path that his feet don't know?\fnote{\fbackref{41:3} So 1QIsa\textsuperscript{a}; MT reads \fbib{travel}} \\
\poeml \v{4}Who has performed and carried this out, \\
\poemll    calling the generations from the beginning? \\
\poeml I, the \divine{Lord}---the first \\
\poemll    and will be with the last \\
\poemlll       ---I am the One!''
\passage{Idolaters Encourage Each Other}
\poeml \v{5}``The coastlands have looked and are afraid; \\
\poemll    the ends of the earth have drawn near together\fnote{\fbackref{41:5} So 1QIsa\textsuperscript{a} LXX; MT reads \fbib{earth tremble}} \\
\poemlll       and come forward. \\
\poeml \v{6}Each helps his neighbor, \\
\poemll    saying\fnote{\fbackref{41:6} So 1QIsa\textsuperscript{a}; 1QIsa\textsuperscript{b} MT read \fbib{and saying}; cf. LXX} to each other, `Be strong!' \\
\poeml \v{7}The craftsman encourages the goldsmith, \\
\poemll    and the hammersmith\fnote{\fbackref{41:7} Lit. \fbib{he who smoothes with the hammer}} encourages the one who strikes the anvil. \\
\poeml He says\fnote{\fbackref{41:7} So 1QIsa\textsuperscript{a}; MT reads \fbib{saying}} about the welding, `It's good!' \\
\poemll    and he reinforces it with nails so that it won't topple.''
\passage{The \divine{Lord} Encourages Israel}
\poeml \v{8}``But as for you, Israel, my servant, \\
\poemll    Jacob, whom I've chosen, \\
\poemlll       the offspring of my friend Abraham--- \\
\poeml \v{9}you whom I encouraged from the ends of the earth \\
\poemll    and called from its farthest corners, \\
\poeml and told you, `You're my servant, \\
\poemll    I've chosen you \\
\poemlll       and haven't cast you aside.' \\
\poeml \v{10}Don't be afraid, \\
\poemll    because I'm with you; \\
\poeml don't be anxious, \\
\poemll    because I am your God. \\
\poeml I keep on strengthening you; \\
\poemll    I'm truly helping you. \\
\poeml I'm surely upholding you \\
\poemll    with my victorious right hand.''
\passage{The Coming Defeat of God's Enemies}
\poeml \v{11}``Look! All who are enraged at you \\
\poemll    will be put to shame and disgraced; \\
\poeml those who contend with you \\
\poemll    will all die.\fnote{\fbackref{41:11} So 1QIsa\textsuperscript{a}; MT reads \fbib{will become nothing and die}; cf. LXX} \\
\poeml \v{12}Those who quarrel with you\fnote{\fbackref{41:12} So 1QIsa\textsuperscript{a}; 1QIsa\textsuperscript{b} MT LXX read \fbib{You will seek them and not find those who quarrel with you}} \\
\poemll    will be as nothing; \\
\poeml those who fight you \\
\poemll    like nothing at all!''
\passage{A Call to Courage}
\poeml \v{13}``For I am the \divine{Lord} your God, \\
\poemll    who takes hold of your right hand, \\
\poeml who says to you, `Don't be afraid. \\
\poemll    I'll help you. \\
\poeml \v{14}Don't be afraid, you little worm Jacob, \\
\poemll    and\fnote{\fbackref{41:14} So 1QIsa\textsuperscript{a}; the Heb. lacks \fbib{and}} you insects of Israel! \\
\poeml I myself will help you,'\fnote{\fbackref{41:14} 1QIsa\textsuperscript{a} is masculine; MT is feminine} declares the \divine{Lord}, \\
\poemll    your\fnote{\fbackref{41:14} 1QIsa\textsuperscript{a} is masculine; MT is feminine} Redeemer, the Holy One of Israel.''
\passage{A Promise of Victory}
\poeml \v{15}``See, I'm making you\fnote{\fbackref{41:15} 1QIsa\textsuperscript{a} is masculine; MT is feminine} into \\
\poemll    a new, sharp, and multi-tooth threshing sledge. \\
\poeml You'll thresh and crush the mountains, \\
\poemll    and make the hills like chaff. \\
\poeml \v{16}You'll winnow them, and the wind will lift them up, \\
\poemll    and a tempest will blow them away. \\
\poeml Then you'll rejoice in the \divine{Lord}, \\
\poemll    and\fnote{\fbackref{41:16} So 1QIsa\textsuperscript{a}; the Heb. lacks \fbib{and}} you'll make your boast in the Holy One of Israel.'' \\
\poeml \v{17}``As for the poor, the needy, those seeking\fnote{\fbackref{41:17} So 1QIsa\textsuperscript{a}; MT LXX read \fbib{poor and the needy seeking}} water--- \\
\poemll    when there is none \\
\poemlll       and their tongues are parched from thirst--- \\
\poeml I, the \divine{Lord}, will answer them. \\
\poemll    I, the God of Israel, won't abandon them. \\
\poeml \v{18}I'll open up rivers on the barren heights, \\
\poemll    and fountains in the midst of the valleys. \\
\poeml I'll turn the\fnote{\fbackref{41:18} So 1QIsa\textsuperscript{a}; MT implies \fbib{the}} desert into a pool of water, \\
\poemll    and the parched land into springs of water. \\
\poeml \v{19}I'll put cedar trees in the wilderness, \\
\poemll    along with acacia, myrtle, and olive trees. \\
\poeml I'll plant cypresses in the desert--- \\
\poemll    box\fnote{\fbackref{41:19} Or \fbib{elm}} trees, and pine trees together--- \\
\poeml \v{20}all so that people may see and recognize, \\
\poemll    perceive,\fnote{\fbackref{41:20} So 1QIsa\textsuperscript{a}; 1QIsa\textsuperscript{a} corrector MT LXX read \fbib{to consider}} consider, and comprehend at the same time, \\
\poeml that the hand of the \divine{Lord} has done this, \\
\poemll    and that the Holy One of Israel has created it.''
\passage{Can God's Enemies Predict the Future?}
\poeml \v{21}``Put forward your case!'' says the \divine{Lord}. \\
\poemll    ``Submit your arguments!'' says Jacob's King. \\
\poeml \v{22}Let them approach and ask us, \\
\poemll    `What will happen? \\
\poeml As to the former things, what were they? \\
\poemll    Tell us, so that we may consider them and know. \\
\poeml Or\fnote{\fbackref{41:22} So 1QIsa\textsuperscript{a}; MT reads \fbib{them. And we may know}} the latter things or the things to come--- \\
\poemlll       let us hear. \\
\poeml \v{23}Tell us what the future holds, \\
\poemll    so we may know that you are gods! \\
\poeml Yes, do something good or something bad, \\
\poemll    so we may hear\fnote{\fbackref{41:23} So 1QIsa\textsuperscript{a}; MT reads \fbib{stare at one another}; cf. LXX} and gaze at it together.'\,'' \\
\poeml \v{24}``Look! You and your work are less than nothing;\fnote{\fbackref{41:24} So 1QIsa\textsuperscript{a}; MT reads \fbib{you are nothing, and your work means nothing}} \\
\poemll    whoever finds you pleasing is disgusting.'' \\
\poeml \v{25}``You are stirring up\fnote{\fbackref{41:25} So 1QIsa\textsuperscript{a}; MT LXX read \fbib{I am raising}} one from the north, \\
\poemll    and they are coming\fnote{\fbackref{41:25} So 1QIsa\textsuperscript{a}; MT reads \fbib{I am raising}} from the rising of the sun; \\
\poemlll       and\fnote{\fbackref{41:25} So 1QIsa\textsuperscript{a}; MT LXX lack \fbib{and}} he will be called by his\fnote{\fbackref{41:25} So 1QIsa\textsuperscript{a}; MT LXX read \fbib{my}} name. \\
\poeml Rulers will arrive like mud;\fnote{\fbackref{41:25} I.e. like an overflowing river; so 1QIsa\textsuperscript{a} LXX; MT reads \fbib{He will come, rulers like mud}} \\
\poemll    just\fnote{\fbackref{41:25} So 1QIsa\textsuperscript{a}; MT reads \fbib{and}} like a potter, he will trample the clay. \\
\poeml \v{26}Who told of this from the beginning, \\
\poemll    so we could know, \\
\poeml or beforehand, so we could ask, \\
\poemll    `Is it right?'\fnote{\fbackref{41:26} So 1QIsa\textsuperscript{a} LXX; MT reads \fbib{He was right}} \\
\poeml Indeed, no one told of this, \\
\poemll    no one made an announcement, \\
\poemlll       and no one heard your words: \\
\poeml \v{27}First, to Zion: ``There is slumber.''\fnote{\fbackref{41:27} So 1QIsa\textsuperscript{a}; MT reads \fbib{look, there they are}} \\
\poemll    And to Jerusalem: ``I'll send a messenger with good news.'' \\
\poeml \v{28}But when I look, there is no one--- \\
\poemll    among them there's no one to give counsel, \\
\poemlll       no one to give an answer when I ask them. \\
\poeml \v{29}See, none of them exist, and their deeds are nothing.\fnote{\fbackref{41:29} So 1QIsa\textsuperscript{a}; MT reads \fbib{all of their works are utterly nothing}} \\
\poemll    Their metal images are only wind and confusion.'\,''
\end{poetry}
\labelchapt{42}
\passage{The Servant of the \divine{Lord}}

\begin{poetry}
\poeml \chapt{42}
\v{1}``Here is my servant, whom I support, \\
\poemll    my chosen one, in whom I delight.\fnote{\fbackref{42:1} Lit. \fbib{whom my soul delights}} \\
\poeml I've placed my Spirit upon him; \\
\poemll    and\fnote{\fbackref{42:1} So 1QIsa\textsuperscript{a}; MT LXX lack \fbib{and}} he'll deliver his\fnote{\fbackref{42:1} So 1QIsa\textsuperscript{a}; MT LXX lack \fbib{his}} justice throughout the world.\fnote{\fbackref{42:1} Lit. \fbib{justice to the nations}} \\
\poeml \v{2}He won't shout, \\
\poemll    or raise his voice, \\
\poemlll       or make it heard in the street. \\
\poeml \v{3}A crushed reed he will not break, \\
\poemll    and a fading candle\fnote{\fbackref{42:3} Lit. \fbib{fading linen wick}} he won't snuff out.\fnote{\fbackref{42:3} So 1QIsa\textsuperscript{a} LXX; MT reads \fbib{quench it}} \\
\poemlll       He'll bring forth\fnote{\fbackref{42:3} I.e. \fbib{will demonstrate}} justice for the truth. \\
\poeml \v{4}And\fnote{\fbackref{42:4} So 1QIsa\textsuperscript{a}; 4QIsa\textsuperscript{h} MT lack \fbib{And}} he won't grow faint or be crushed \\
\poemll    until he establishes justice on the mainland, \\
\poemlll       and the coastlands take ownership of\fnote{\fbackref{42:4} So 1QIsa\textsuperscript{a}; MT LXX read \fbib{the islands wait for} (cf. 4QIsa\textsuperscript{h})} his Law.''
\passage{God Speaks about the Servant}
\poeml \v{5}This is what God says--- \\
\poemll    the God\fnote{\fbackref{42:5} So 1QIsa\textsuperscript{a}; 1QIsa\textsuperscript{a} corrector reads \fbib{and}; MT reads \fbib{the \divine{Lord}}} who created the heavens \\
\poemlll       and stretched them out, \\
\poeml who spread out the earth and its produce, \\
\poemll    who gives breath to the people on it \\
\poemlll       and life\fnote{\fbackref{42:5} Lit. \fbib{spirit}} to those who walk in it: \\
\poeml \v{6}``I've\fnote{\fbackref{42:6} So 1QIsa\textsuperscript{a}; 4QIsa\textsuperscript{h} MT read \fbib{I, the \divine{Lord}, have}; LXX reads \fbib{I, the \divine{Lord} God, have}} called you in righteousness. \\
\poemll    I'll take hold of your hand. \\
\poeml I'll preserve you and appoint you \\
\poemll    as a covenant to the people,\fnote{\fbackref{42:6} So 1QIsa\textsuperscript{a} MT LXX; 4QIsa\textsuperscript{h} reads \fbib{as an everlasting covenant}} \\
\poemlll       as a light for the nations, \\
\poeml \v{7}to open blind eyes \\
\poemll    and to bring out those who are bound\fnote{\fbackref{42:7} So 1QIsa\textsuperscript{a} LXX; 4QIsa\textsuperscript{h} MT read \fbib{out prisoners}} from their cells, \\
\poemlll       and\fnote{\fbackref{42:7} So 1QIsa\textsuperscript{a}; 4QIsa\textsuperscript{h} MT lack \fbib{and}} those sitting in darkness from prison. \\
\poeml \v{8}I, the \divine{Lord}, am the one, \\
\poemll    and I won't give my name and\fnote{\fbackref{42:8} So 1QIsa\textsuperscript{a}; 4QIsa\textsuperscript{h} MT LXX read \fbib{I am the \divine{Lord}; that is my name. I won't give my}} glory to another, \\
\poemlll       nor my praise to idols. \\
\poeml \v{9}See, the former things have taken place, \\
\poemll    and I'm announcing the\fnote{\fbackref{42:9} So 1QIsa\textsuperscript{a}; 4QIsa\textsuperscript{b} 4QIsa\textsuperscript{h} MT LXX lack \fbib{the}} new things--- \\
\poeml before they spring into being
\end{poetry}

I'm telling you about them.''
\passage{Praise in Song to God}

\begin{poetry}
\poeml \v{10}Sing to the \divine{Lord} a new song, \\
\poemll    and\fnote{\fbackref{42:10} So 1QIsa\textsuperscript{a}; 4QIsa\textsuperscript{h} MT lack \fbib{and}} his praise from the ends of the earth, \\
\poeml you who sail down the sea and by everything in it, \\
\poemll    you coastlands and their inhabitants. \\
\poeml \v{11}Let the desert cry out,\fnote{\fbackref{42:11} So 1QIsa\textsuperscript{a} 4QIsa\textsuperscript{h} LXX (sing.); MT (pl.)} \\
\poemll    its towns and the\fnote{\fbackref{42:11} So 1QIsa\textsuperscript{a}; MT LXX read \fbib{and its towns}} villages where Kedar lives; \\
\poeml and\fnote{\fbackref{42:11} So 1QIsa\textsuperscript{a}; the Heb. lacks \fbib{and}} let those who live in Sela sing for joy. \\
\poemll    Let them shout aloud\fnote{\fbackref{42:11} So 1QIsa\textsuperscript{a} LXX; MT reads \fbib{them cry joyfully}} from the mountaintops. \\
\poeml \v{12}Let them give glory to the \divine{Lord}, \\
\poemll    and declare his praise in the islands. \\
\poeml \v{13}The \divine{Lord} marches out like a warrior; \\
\poemll    he stirs up his rage like a man of war; \\
\poeml he makes his anger heard; \\
\poemll    he shouts aloud;\fnote{\fbackref{42:13} So 1QIsa\textsuperscript{a}; MT reads \fbib{He makes a war cry and shouts out his anger}} \\
\poemlll       he declares his mastery over his enemies: \\
\poeml \v{14}``I have certainly\fnote{\fbackref{42:14} So 1QIsa\textsuperscript{a}; the Heb. lacks \fbib{certainly}} stayed silent for a long time; \\
\poemll    I've kept still and held myself back. \\
\poeml Now, like a woman giving birth, I'll cry out. \\
\poemll    All of a sudden I'll gasp and pant. \\
\poeml \v{15}I'll devastate the mountains and hills, \\
\poemll    and dry up all their vegetation; \\
\poeml I'll turn rivers into islands, \\
\poemll    and dry up the ponds. \\
\poeml \v{16}I'll help the blind walk, \\
\poemll    even\fnote{\fbackref{42:16} Or \fbib{and}; so 1QIsa\textsuperscript{a}; the Heb. lacks \fbib{even}} on a road they do not know; \\
\poeml I'll guide them \\
\poemll    in directions\fnote{\fbackref{42:16} Lit. \fbib{paths}} they do not know. \\
\poeml I'll turn the dark places\fnote{\fbackref{42:16} So 1QIsa\textsuperscript{a} (misspelling \fbib{places}); MT LXX read \fbib{darkness}} into light in front of them, \\
\poemll    and the rough places into level ground. \\
\poeml These are the things I will do, \\
\poemll    and I won't abandon them. \\
\poeml \v{17}Those who trust in carved idols \\
\poemll    will turn back and\fnote{\fbackref{42:17} So 1QIsa\textsuperscript{a}; the Heb. lacks \fbib{and}} be completely disappointed,\fnote{\fbackref{42:17} So 1QIsa\textsuperscript{a} MT LXX; MT\textsuperscript{ms} lacks \fbib{disappointed}} \\
\poeml along with those\fnote{\fbackref{42:17} 1QIsa\textsuperscript{a} MT LXX lack \fbib{along with those}} who say to metal images, \\
\poemll    `You are our gods.'\,''
\passage{God Rebukes Israel}
\poeml \v{18}``Listen, you deaf people, \\
\poemll    and look up, you blind people, so you may see! \\
\poeml \v{19}Who is blind except my servant, \\
\poemll    or deaf like my messenger I am sending? \\
\poeml Who is blind like the one committed to me, \\
\poemll    or blind like the \divine{Lord}'s servant? \\
\poeml \v{20}You've seen\fnote{\fbackref{42:20} So 1QIsa\textsuperscript{a} MT; MT\textsuperscript{qere} reads \fbib{To see} (or \fbib{He sees} )} many things, but you pay no\fnote{\fbackref{42:20} So 1QIsa\textsuperscript{a} MT LXX; MT\textsuperscript{mss} read \fbib{he pays no}} attention. \\
\poemll    His ears are open,\fnote{\fbackref{42:20} So 1QIsa\textsuperscript{a}; MT reads \fbib{to open}; or \fbib{are open}} but he doesn't listen. \\
\poeml \v{21}The \divine{Lord} was pleased, for the sake of his vindication, \\
\poemll    that he should magnify his Law and make it glorious. \\
\poeml \v{22}But this is a people who have been robbed and plundered, \\
\poemll    all of them trapped in pits \\
\poemlll       or hidden away in prisons.\fnote{\fbackref{42:22} So 1QIsa\textsuperscript{a} MT LXX; 4QIsa\textsuperscript{g} reads \fbib{prison}} \\
\poeml They have become prey, with no one to rescue them; \\
\poemll    they have been made loot, with no one to say, `Send them back!' \\
\poeml \v{23}``Who among you will listen, \\
\poemll    and\fnote{\fbackref{42:23} So 1QIsa\textsuperscript{a}; MT LXX lack \fbib{and}} pay attention, \\
\poemlll       and listen for the time to come?''
\passage{God Punishes Israel}
\poeml \v{24}``Who handed Jacob over to looters, \\
\poemll    and Israel to robbers? \\
\poeml Was it not the \divine{Lord}, against whom we have sinned? \\
\poemll    After all, they weren't willing to walk in his ways, \\
\poemlll       and they wouldn't obey\fnote{\fbackref{42:24} Or \fbib{wouldn't listen to}} his instruction, \\
\poeml \v{25}so he drenched him with\fnote{\fbackref{42:25} Lit. \fbib{he poured out on him}} the heat that is his anger,\fnote{\fbackref{42:25} So 1QIsa\textsuperscript{a} LXX; 4QIsa\textsuperscript{g} MT read \fbib{the heat, his anger}} \\
\poemll    the violence of war. \\
\poeml It enveloped him in flames, \\
\poemll    but still he had no insight. \\
\poeml It burned him up, \\
\poemll    but he didn't take it to heart.''
\end{poetry}
\labelchapt{43}
\passage{The Future Redemption of Israel}

\chapt{43}
\v{1}But now this is what the \divine{Lord} says,

\begin{poetry}
\poemll    the one who created you, Jacob, \\
\poemlll       the one who formed you, Israel: \\
\poeml ``Do not be afraid, because I've redeemed you. \\
\poemll    I've called you by name; \\
\poemlll       you are mine. \\
\poeml \v{2}When you pass through the waters, I'll be with you; \\
\poemll    and through the rivers, they won't sweep over you. \\
\poeml when you walk through fire you won't be scorched, \\
\poemll    and the flame won't set you ablaze. \\
\poeml \v{3}``I\fnote{\fbackref{43:3} So 1QIsa\textsuperscript{a}; 1QIsa\textsuperscript{b} 4QIsa\textsuperscript{g} MT LXX read \fbib{For I}} am the \divine{Lord} your God, \\
\poemll    the Holy One of Israel, your Redeemer.\fnote{\fbackref{43:3} So 1QIsa\textsuperscript{a}; MT LXX read \fbib{Savior}} \\
\poeml And\fnote{\fbackref{43:3} So 1QIsa\textsuperscript{a}; the Heb. lacks \fbib{And}} I've given Egypt as your ransom,\fnote{\fbackref{43:3} So 1QIsa\textsuperscript{a}; 1QIsa\textsuperscript{b} MT LXX read \fbib{as your ransom Egypt}} \\
\poemll    Cush and the people of Seba\fnote{\fbackref{43:3} So 1QIsa\textsuperscript{a}; 1QIsa\textsuperscript{b} 4QIsa\textsuperscript{g} MT read \fbib{and Seba}} in exchange for you. \\
\poeml \v{4}Since you're precious in my sight \\
\poemll    and honored, \\
\poeml and because I love you, \\
\poemll    I'm giving up people in your place, \\
\poemll    and nations in exchange for your life.''
\passage{The Regathering of Israel}
\poeml \v{5}``Don't be afraid, for I am with you; \\
\poemll    I'll bring your children from the east, \\
\poemlll       and gather you from the west. \\
\poeml \v{6}I'll say to the north, \\
\poemll    `Give them up'! \\
\poeml and to the south, \\
\poemll    `Don't keep them back!' \\
\poeml Bring\fnote{\fbackref{43:6} 1QIsa\textsuperscript{a} employs masculine pl.; MT employs feminine pl.} my sons from far away \\
\poemll    and my daughters from the ends of the earth--- \\
\poeml \v{7}everyone who is called by my name, \\
\poemll    whom I created for my glory, \\
\poemlll       whom\fnote{\fbackref{43:7} So 1QIsa\textsuperscript{a} MT; 1QIsa\textsuperscript{b} reads \fbib{and whom}} I formed and made. \\
\poeml \v{8}``Bring out the people who are blind, yet still have eyes, \\
\poemll    who are deaf, yet still have ears! \\
\poeml \v{9}Let all the nations be gathered together, \\
\poemll    and let the peoples be assembled. \\
\poeml ``Who is there among them who\fnote{\fbackref{43:9} So 1QIsa\textsuperscript{a}; MT LXX lack \fbib{who}} can declare this, \\
\poemll    or announce\fnote{\fbackref{43:9} So 1QIsa\textsuperscript{a}; MT reads \fbib{announce to us}; LXX reads \fbib{announce to you}} the former things? \\
\poeml Let them produce their witnesses to prove them right, \\
\poemll    and let them proclaim\fnote{\fbackref{43:9} So 1QIsa\textsuperscript{a}; MT reads \fbib{hear}} so people will say, `It's true.' \\
\poeml \v{10}``You are my witnesses,'' declares the \divine{Lord}, \\
\poemll    ``and my servant whom I have chosen, \\
\poeml so that you may know and trust me \\
\poemll    and understand that I am the One.\fnote{\fbackref{43:10} Or \fbib{am he}} \\
\poeml Before me no God was formed, \\
\poemll    nor will there be one after me. \\
\poeml \v{11}I, yes I, am the \divine{Lord}, \\
\poemll    and apart from me there is no savior. \\
\poeml \v{12}I've revealed and saved and proclaimed, \\
\poemll    when there was no foreign god among you --- \\
\poemlll       and you are my witnesses,'' declares the \divine{Lord}. \\
\poeml \v{13}``I am God; also\fnote{\fbackref{43:13} So 1QIsa\textsuperscript{a}; MT reads \fbib{God---Yes,}; cf. LXX} from ancient days\fnote{\fbackref{43:13} Or \fbib{from this day on,}} I am the one. \\
\poemll    And there is no one who can deliver out of my hand; \\
\poemlll       when I act, who can reverse it?'' \\
\poeml \v{14}This is what the \divine{Lord} says, \\
\poemll    your Redeemer, the Holy One of Israel: \\
\poeml ``For your sake I will send to Babylon,\fnote{\fbackref{43:14} So 1QIsa\textsuperscript{a}; 4QIsa\textsuperscript{b} MT each spell this line differently} \\
\poemll    and bring them all down as fugitives. \\
\poeml Now as for the Babylonians, \\
\poemll    their ringing cry will become lamentation. \\
\poeml \v{15}I am the \divine{Lord}, your Holy One, \\
\poemll    Creator of Israel, and your King.''
\passage{Something New for Israel}
\poeml \v{16}This is what the \divine{Lord} says --- \\
\poemll    who makes a way through the sea, \\
\poemlll       a path through the mighty waters, \\
\poeml \v{17}who brings out chariots and horsemen, \\
\poemll    and\fnote{\fbackref{43:17} So 1QIsa\textsuperscript{a} LXX; the Heb. lacks \fbib{and}} armies and warriors at the same time. \\
\poeml They lay there, never to rise again, \\
\poemll    extinguished, snuffed out like a candle:\fnote{\fbackref{43:17} Lit. \fbib{linen wick}} \\
\poeml \v{18}``Don't remember\fnote{\fbackref{43:18} 1QIsa\textsuperscript{a} employs second person sing.; MT LXX employ second person pl.} the former things; \\
\poemll    don't dwell on things past. \\
\poeml \v{19}Watch! I'm about to carry out something new! \\
\poemll    And\fnote{\fbackref{43:19} So 1QIsa\textsuperscript{a}; the Heb. lacks \fbib{And}} now it's springing up--- \\
\poemlll       don't you recognize it? \\
\poeml I'm making a way in the wilderness \\
\poemll    and paths\fnote{\fbackref{43:19} So 1QIsa\textsuperscript{a}; MT LXX read \fbib{streams}} in the desert. \\
\poeml \v{20}Wild animals, jackals, and owls\fnote{\fbackref{43:20} Or \fbib{ostriches}} will honor me \\
\poemll    because I provide\fnote{\fbackref{43:20} So 1QIsa\textsuperscript{a}; MT reads \fbib{have provided}} water in the desert \\
\poeml and streams in the wilderness \\
\poemll    to give drink to my people, my chosen ones,\fnote{\fbackref{43:20} So 1QIsa\textsuperscript{a}; MT reads \fbib{my chosen people}} \\
\poeml \v{21}the people whom I formed for myself \\
\poemll    and\fnote{\fbackref{43:21} So 1QIsa\textsuperscript{a}; 4QIsa\textsuperscript{g} lacks \fbib{my chosen ones}} so that they may speak\fnote{\fbackref{43:21} So 1QIsa\textsuperscript{a}; 4QIsa\textsuperscript{g} MT LXX read \fbib{recount}} my praise.''
\passage{God is Weary of Israel}
\poeml \v{22}``And\fnote{\fbackref{43:22} So 1QIsa\textsuperscript{a} MT; MT\textsuperscript{mss} lack \fbib{And}} yet you didn't call upon me, Jacob; \\
\poemll    indeed, you are tired of me, Israel! \\
\poeml \v{23}You haven't brought me your sheep for a burnt offering,\fnote{\fbackref{43:23} So 1QIsa\textsuperscript{a}; MT reads \fbib{for your burnt-offerings}; or \fbib{your burnt-offering}} \\
\poemll    nor have you honored me with\fnote{\fbackref{43:23} So 1QIsa\textsuperscript{a} LXX; implicit in MT} your sacrifices, \\
\poeml nor have you made meal offerings for me\fnote{\fbackref{43:23} So 1QIsa\textsuperscript{a}; 4QIsa\textsuperscript{g} MT read \fbib{I have not burdened you with grain offerings}; LXX lacks this line}--- \\
\poemll    yet I have not tired you about incense! \\
\poeml \v{24}You\fnote{\fbackref{43:24} So 1QIsa\textsuperscript{a} MT; 4QIsa\textsuperscript{g} reads \fbib{And you}} haven't bought me sweet cane with money, \\
\poemll    nor have you satisfied me with the fat of your sacrifices. \\
\poeml You have only burdened me with your sins \\
\poemll    and made me tired with your iniquities. \\
\poeml \v{25}``I, I am the one \\
\poemll    who blots out your transgression\fnote{\fbackref{43:25} So 1QIsa\textsuperscript{a}; MT reads \fbib{transgressions}; cf. LXX} for my own sake, \\
\poemlll       and I'll remember your sins no more.\fnote{\fbackref{43:25} So 1QIsa\textsuperscript{a}; MT reads \fbib{not remember your sins}} \\
\poeml \v{26}Recount the brief! \\
\poemll    Let's argue the matter together; \\
\poeml Present your case, \\
\poemll    so that you may be proved right. \\
\poeml \v{27}Your first ancestor sinned, \\
\poemll    and your mediators rebelled against me. \\
\poeml \v{28}So I'll disgrace the leaders of the Temple, \\
\poemll    and I'll consign Jacob to total destruction\fnote{\fbackref{43:28} The Heb. term refers to involuntary dedication to God of the thing destroyed} \\
\poemlll       and Israel to contempt.
\end{poetry}
\labelchapt{44}
\passage{God's Blessing on Israel}

\begin{poetry}
\poeml \chapt{44}
\v{1}``But now listen, Jacob my servant \\
\poemll    and Israel whom I have chosen: \\
\poeml \v{2}This what the \divine{Lord} says, the one who made you, \\
\poemll    formed you from the womb, \\
\poemlll       and who will help\fnote{\fbackref{44:2} So 1QIsa\textsuperscript{a}; MT reads \fbib{he will help}} you: \\
\poeml ``Don't be afraid, Jacob my servant, \\
\poemll    and Jeshurun,\fnote{\fbackref{44:2} I.e. a poetic term for national Israel; the Heb. name means \fbib{Upright One}; cf. Deut 32:15; 33:5, 26} whom I have chosen. \\
\poeml \v{3}For I'll pour water upon thirsty ground \\
\poemll    and streams on parched land. \\
\poeml So\fnote{\fbackref{44:3} So 1QIsa\textsuperscript{a}; MT lacks \fbib{So}} will I pour my Spirit upon your offspring, \\
\poemll    and my blessing upon your descendants. \\
\poeml \v{4}They'll\fnote{\fbackref{44:4} So 1QIsa\textsuperscript{a}; MT LXX read \fbib{And they}} spring up as among\fnote{\fbackref{44:4} So 1QIsa\textsuperscript{a} MT\textsuperscript{mss} LXX Targ; MT reads \fbib{up among}} the green grass, \\
\poemll    like willows by flowing streams. \\
\poeml \v{5}One will say, `I belong to the \divine{Lord},' \\
\poemll    and another will call himself by the name of Jacob; \\
\poeml still another will have written on his hand, `the \divine{Lord}'s,' \\
\poemll    and will adopt the name of Israel.''
\passage{I am the First and the Last}
\poeml \v{6}This is what the \divine{Lord} says, the King of Israel \\
\poemll    and its Redeemer--- \\
\poemlll       the \divine{Lord} of the Heavenly Armies is his name---\fnote{\fbackref{44:6} So 1QIsa\textsuperscript{a}; MT LXX lack \fbib{is his name}} \\
\poeml ``I am the first and I am the last, \\
\poemll    and apart from me there is no God. \\
\poeml \v{7}Who is like me? Let him proclaim \\
\poemll    and declare it, and lay it out for himself---\fnote{\fbackref{44:7} So 1QIsa\textsuperscript{a}; MT LXX read \fbib{me}} \\
\poeml since he made\fnote{\fbackref{44:7} Or \fbib{himself, making him}; so 1QIsa\textsuperscript{a}; MT LXX read \fbib{since I made}} an ancient people. \\
\poemll    And let him speak\fnote{\fbackref{44:7} So 1QIsa\textsuperscript{a}; 4QIsa\textsuperscript{c} MT LXX lack \fbib{let him speak}} future events; \\
\poemlll       let them tell him what\fnote{\fbackref{44:7} So 1QIsa\textsuperscript{a}; 4QIsa\textsuperscript{c} MT read \fbib{and what}} will happen. \\
\poeml \v{8}Don't tremble, and don't be afraid.\fnote{\fbackref{44:8} So 1QIsa\textsuperscript{a} MT; LXX lacks \fbib{and do not be afraid}} \\
\poemll    Didn't I tell you and announce it long ago? \\
\poeml You are my witnesses. \\
\poemll    Is there any God besides me? \\
\poeml There is no other Rock--- \\
\poemll    I don't know of any.''
\end{poetry}
\passage{A Rebuke to Idol Worship}

\v{9}Now,\fnote{\fbackref{44:9} So 1QIsa\textsuperscript{a}; MT LXX lack \fbib{Now}} all the forming of\fnote{\fbackref{44:9} So 1QIsa\textsuperscript{a}; MT LXX read \fbib{those who form}} images means nothing, and the things they treasure are worthless. Their own witnesses cannot see, and they\fnote{\fbackref{44:9} So MT LXX; 1QIsa\textsuperscript{a} reads \fbib{They}} know nothing. So they will be put to shame.

\v{10}Who would shape a god or cast an image that profits nothing? \v{11}To be sure, all who associate with it will be put to shame; and as for the craftsmen, they are only human. Let them all gather together and\fnote{\fbackref{44:11} So 1QIsa\textsuperscript{a} LXX; MT lacks \fbib{and}} take their stand. Then\fnote{\fbackref{44:11} So 1QIsa\textsuperscript{a}; MT LXX lack \fbib{Then}} let them be terrified---they will be humiliated together.

\v{12}The blacksmith prepares a tool and works in the coals, then\fnote{\fbackref{44:12} So 1QIsa\textsuperscript{a}; MT LXX lack \fbib{then}} fashions an idol with hammers, working by the strength of his arm. He even becomes hungry and loses his strength; he drinks no water and grows faint.

\v{13}The carpenter measures it\fnote{\fbackref{44:13} I.e. the idol; so 1QIsa\textsuperscript{a} LXX; MT lacks \fbib{it}} with a line; he traces its shape with a stylus, then fashions it with planes and shapes it with a compass. He makes the idol like a human figure, with human beauty, to be at home\fnote{\fbackref{44:13} Lit. \fbib{to rest}} in a shrine. \v{14}He cuts down cedars, or chooses a cypress tree or an oak, and lets it grow strong among the trees of the forest. Or he plants a cedar, and the rain makes it grow. \v{15}He divides it up\fnote{\fbackref{44:15} So 1QIsa\textsuperscript{a}; MT reads \fbib{It is}; LXX reads \fbib{so that it is}} for people to burn. Taking part of it, he warms himself, makes a fire, and bakes bread. Or perhaps\fnote{\fbackref{44:15} So 1QIsa\textsuperscript{a}; MT reads \fbib{Also}} he constructs a god and worships it. He makes it an idol and bows down to it. \v{16}Half the wood he burns in the fire, and\fnote{\fbackref{44:16} So 1QIsa\textsuperscript{a} LXX; MT lacks \fbib{and}} over\fnote{\fbackref{44:16} 1QIsa\textsuperscript{a} lacks \fbib{over}, but inserts it above the line} that half he places\fnote{\fbackref{44:16} Lit. \fbib{half is}} meat so he can eat. He sits by its coals, warms himself,\fnote{\fbackref{44:16} So 1QIsa\textsuperscript{a}; MT LXX read \fbib{eats meat he roasted as a roast and is satisfied. He also warms himself}} and says, ``Ah! I am warm in front of\fnote{\fbackref{44:16} So 1QIsa\textsuperscript{a}; MT reads \fbib{I see}; LXX reads \fbib{and I have seen}} the fire.'' \v{17}And the rest of it he makes into a god. To blocks\fnote{\fbackref{44:17} Or \fbib{to his Baals} (i.e. to Canaanite deities); so 1QIsa\textsuperscript{a} copyist error; MT LXX read \fbib{god, for his idol}} of wood he bows down, worships, prays, and says, ``Save me, since you are my god.''

\v{18}They don't realize; they don't understand, because their eyes are plastered over so they cannot see, and their minds, too, so they cannot understand. \v{19}No one stops to think. No one has the knowledge or understanding to think---yes to think!---\fnote{\fbackref{44:19} So 1QIsa\textsuperscript{a}; MT lacks ---\fbib{yes to think!---}}``Half of it I burned in the fire. I even baked bread on its coals, and\fnote{\fbackref{44:19} So 1QIsa\textsuperscript{a}; MT lacks \fbib{and}} I roasted meat and ate it. And\fnote{\fbackref{44:19} So 1QIsa\textsuperscript{a} MT; 4QIsa\textsuperscript{b} lacks \fbib{And}} am I about to make detestable things\fnote{\fbackref{44:19} So 1QIsa\textsuperscript{a}; 4QIsa\textsuperscript{b} MT LXX read \fbib{make a detestable thing}} from what is left? Am I about to bow down to blocks\fnote{\fbackref{44:19} So 1QIsa\textsuperscript{a}; MT reads \fbib{to a block}} of wood?'' \v{20}He tends ashes. A deceived mind has lead him astray. It cannot be his life,\fnote{\fbackref{44:20} So 1QIsa\textsuperscript{a}; 4QIsa\textsuperscript{b} MT LXX read \fbib{He does not save his life}} nor can he say, ``There's a lie in my right hand.''\fnote{\fbackref{44:20} So 1QIsa\textsuperscript{a}; 4QIsa\textsuperscript{b} MT LXX read ``\fbib{There's no lie in my right hand, is there?''}}
\passage{A Call to Remembrance and Joy}

\begin{poetry}
\poeml \v{21}``Remember these things, Jacob, \\
\poemll    Israel,\fnote{\fbackref{44:21} So 1QIsa\textsuperscript{a}; 4QIsa\textsuperscript{b} MT LXX read \fbib{and Israel}} for you are my servant. \\
\poeml I have formed you; \\
\poemll    you are a servant to me. \\
\poemlll       Israel,\fnote{\fbackref{44:21} So 1QIsa\textsuperscript{a} MT; 4QIsa\textsuperscript{b} LXX read \fbib{and Israel}} you must not mislead me.\fnote{\fbackref{44:21} So 1QIsa\textsuperscript{a}; 4QIsa\textsuperscript{b} MT read \fbib{you won't be forgotten by me}; LXX reads \fbib{don't forget me}} \\
\poeml \v{22}I've wiped away your transgressions like a cloud \\
\poemll    and your sins like mist. \\
\poeml Return to me; \\
\poemll    because I've redeemed you. \\
\poeml \v{23}``Shout for joy, you heavens, \\
\poemll    for the \divine{Lord} has done it! \\
\poeml Shout aloud, you depths of the earth! \\
\poemll    Burst out with singing, you mountains, \\
\poemlll       you forest, and all your trees! \\
\poeml For the \divine{Lord} has redeemed Jacob \\
\poemll    and will display his glory in Israel. \\
\poeml \v{24}``This is what the \divine{Lord} says, your Redeemer \\
\poemll    and the one who formed you in the womb: \\
\poeml `I am the \divine{Lord}, who has made everything, \\
\poemll    who alone stretched out the heavens, \\
\poeml who spread out the earth--- \\
\poemll    Who was with me at that time?---\fnote{\fbackref{44:24} 1QIsa\textsuperscript{a} 4QIsa\textsuperscript{b} MT LXX lack \fbib{at that time}} \\
\poeml \v{25}who frustrates the omens of idle talkers, \\
\poemll    and drives diviners mad, \\
\poeml who turns back the wise, \\
\poemll    and makes their knowledge foolish;\fnote{\fbackref{44:25} So 1QIsa\textsuperscript{a} 1QIsa\textsuperscript{b} 4QIsa\textsuperscript{b} LXX; MT reads \fbib{wise}} \\
\poeml \v{26}who carries out the words of his servants, \\
\poemll    and fulfills the predictions of his messengers, \\
\poeml who says of Jerusalem, ``It will be inhabited,'' \\
\poemll    and of Judah's cities, ``They will be rebuilt,'' \\
\poeml and of her ruins, ``I'll raise them up''; \\
\poeml \v{27}who says to the watery deep, ``Be dry--- \\
\poemll    I will dry up your rivers;'' \\
\poeml \v{28}who says about Cyrus, ``He's my shepherd, \\
\poemll    and he'll carry out everything that I please: \\
\poeml He'll say of Jerusalem, `Let it be rebuilt,' \\
\poemll    and of my\fnote{\fbackref{44:28} So 1QIsa\textsuperscript{a} LXX; 4QIsa\textsuperscript{b} MT read \fbib{the}} Temple, `Let its foundations be laid again.'\,''\,'\,''
\end{poetry}
\labelchapt{45}
\passage{Cyrus: God's Deliverer}

\begin{poetry}
\poeml \chapt{45}
\v{1}This is what the \divine{Lord} says to his anointed, Cyrus, \\
\poemll    whose right hand I have grasped \\
\poeml to subdue nations before him, \\
\poemll    as I strip kings of their armor,\fnote{\fbackref{45:1} Lit. \fbib{I expose the loins of kings}} \\
\poeml to open doors\fnote{\fbackref{45:1} So 1QIsa\textsuperscript{a} (pl.); MT (dual)} before him \\
\poemll    and gates that cannot keep closed: \\
\poeml \v{2}``I myself will go before you, \\
\poemll    and he\fnote{\fbackref{45:2} So 1QIsa\textsuperscript{a}; MT LXX read \fbib{I}} will make the mountains\fnote{\fbackref{45:2} So 1QIsa\textsuperscript{a} 1QIsa\textsuperscript{b} LXX; MT reads \fbib{hills}} level; \\
\poeml I'll shatter bronze doors \\
\poemll    and cut through iron bars. \\
\poeml \v{3}I'll give you concealed treasures\fnote{\fbackref{45:3} Lit. \fbib{treasures of darkness}} \\
\poemll    and riches hidden in secret places, \\
\poeml so that you'll know that it is I, the \divine{Lord}, \\
\poemll    the God of Israel, who calls you by name. \\
\poeml \v{4}For the sake of Jacob my servant, \\
\poemll    Israel\fnote{\fbackref{45:4} So 1QIsa\textsuperscript{a}; MT LXX read \fbib{and Israel}} my chosen, \\
\poeml I've called you, \\
\poemll    and he has established you with a name,\fnote{\fbackref{45:4} So 1QIsa\textsuperscript{a}; MT reads \fbib{I have called you by your name, given you a title}; LXX reads \fbib{I will call you by your name, and receive you}} \\
\poemlll       although you have not acknowledged me. \\
\poeml \v{5}I am the \divine{Lord}, and there is no other besides me: \\
\poemll    and there are no gods.\fnote{\fbackref{45:5} So 1QIsa\textsuperscript{a}; MT 1QIsa\textsuperscript{b} read \fbib{there is no other: besides me there are no gods}} \\
\poeml I'm strengthening you, \\
\poemll    although you have not acknowledged me, \\
\poeml \v{6}so that from the sun's rising\fnote{\fbackref{45:6} I.e. \fbib{the east}} to the west \\
\poemll    people may know that there is none besides me. \\
\poeml ``I am the \divine{Lord}, and there is no other.''
\passage{God is Sovereign}
\poeml \v{7}``I form light and create darkness, \\
\poeml I make goodness\fnote{\fbackref{45:7} So 1QIsa\textsuperscript{a}; MT reads \fbib{well-being}; LXX reads \fbib{peace}} and create disaster. \\
\poemll    I am the \divine{Lord}, who does all these things. \\
\poeml \v{8}``Shout,\fnote{\fbackref{45:8} So 1QIsa\textsuperscript{a} LXX; MT reads \fbib{Shower}} you skies above, and you clouds, \\
\poemll    and let righteousness stream down.\fnote{\fbackref{45:8} So 1QIsa\textsuperscript{a}; 1QIsa\textsuperscript{b} MT LXX read \fbib{Shower, you skies above, and let the clouds stream down righteousness}} \\
\poeml I am\fnote{\fbackref{45:8} 1QIsa\textsuperscript{a} MT LXX lack \fbib{I am}} the one who says to the earth, `Let salvation blossom, \\
\poemll    and let righteousness sprout forth.'\fnote{\fbackref{45:8} So 1QIsa\textsuperscript{a}; 1QIsa\textsuperscript{b} 1QIsa\textsuperscript{c} MT read \fbib{Let the earth open up, let them bear the fruit of salvation, and let righteousness sprout forth also. I the \divine{Lord} have created it}} \\
\poeml \v{9}``Woe to the one who quarrels with his makers,\fnote{\fbackref{45:9} So 1QIsa\textsuperscript{a}; MT reads \fbib{maker}} \\
\poemll    a mere potsherd with the potsherds of the\fnote{\fbackref{45:9} So 1QIsa\textsuperscript{a}; MT lacks \fbib{the}} earth! \\
\poeml Woe to the one who says\fnote{\fbackref{45:9} So 1QIsa\textsuperscript{a}; 1QIsa\textsuperscript{b} MT LXX read \fbib{Will clay say}} to the one forming him, \\
\poemll    `What are you making?' \\
\poemlll       or `Your work has no human\fnote{\fbackref{45:9} So 1QIsa\textsuperscript{a}; MT LXX lack \fbib{human}} hands?'! \\
\poeml \v{10}Woe to the\fnote{\fbackref{45:10} So 1QIsa\textsuperscript{a}; MT lacks \fbib{the}} one who says to his father, \\
\poemll    `What are you begetting?' \\
\poemlll       or to a woman, `To what are you giving birth?'!'' \\
\poeml \v{11}This is what the Lord\fnote{\fbackref{45:11} So 1QIsa\textsuperscript{a}; MT LXX 1QIsa\textsuperscript{a} corrector reads \fbib{the \divine{Lord}, the Holy One of Israel}} says, \\
\poemll    the Creator of the signs: \\
\poeml ``Question me about my children?\fnote{\fbackref{45:11} So 1QIsa\textsuperscript{a} LXX; Lit. \fbib{my sons and daughters}; MT reads \fbib{says, and its creator: Question me of things to come about my children}} \\
\poemll    Or give me orders about the work of my hands? \\
\poeml \v{12}I myself made the earth \\
\poemll    and personally created humankind upon it. \\
\poeml My own hands stretched out the skies; \\
\poemll    I marshaled all their starry hosts.''
\passage{God will Bless Cyrus}
\poeml \v{13}``I have aroused him\fnote{\fbackref{45:13} I.e. Cyrus, king of Persia} in righteousness, \\
\poemll    and I'll make all his pathways smooth. \\
\poeml It is he who will rebuild my city \\
\poemll    and set my exiles free, \\
\poeml but not for a price nor reward, '' \\
\poemll    says the \divine{Lord} of the Heavenly Armies. \\
\poeml \v{14}This is what the \divine{Lord} says: \\
\poeml ``The wealth of Egypt, and the merchandise of Ethiopia, \\
\poemll    those\fnote{\fbackref{45:14} Lit. \fbib{and those}; so 1QIsa\textsuperscript{a}; MT LXX read \fbib{and the}} Sabeans, men of great heights.\fnote{\fbackref{45:14} So 1QIsa\textsuperscript{a}; MT reads \fbib{height}} \\
\poeml They'll come over to you and will be yours; \\
\poemll    They'll trudge behind you--- \\
\poemlll       coming over in chains, they'll bow down to you. \\
\poeml They'll plead with you, \\
\poemll    `Surely God is in you; \\
\poemlll       and there is no other God at all.'\,''
\passage{God as Savior of Israel}
\poeml \v{15}``Truly you are a God who hides himself, \\
\poemll    O God of Israel, the Savior. \\
\poeml \v{16}All of them will be put to shame---indeed, disgraced! \\
\poemll    The makers of idols will go off in disgrace together. \\
\poeml \v{17}But Israel will be saved by the \divine{Lord} \\
\poemll    with everlasting salvation; \\
\poeml you won't be put to shame or disgraced ever again.'' \\
\poeml \v{18}For this is what the \divine{Lord} says, \\
\poemll    who created the heavens--- \\
\poeml he is God, \\
\poemll    and\fnote{\fbackref{45:18} So 1QIsa\textsuperscript{a}; MT LXX lack \fbib{and}} the one who formed the earth and made it, \\
\poeml and\fnote{\fbackref{45:18} So 1QIsa\textsuperscript{a}; MT lacks \fbib{and}} he is the one who established it; \\
\poemll    he didn't create it for\fnote{\fbackref{45:18} Or \fbib{it to remain in a state of}; so 1QIsa\textsuperscript{a} LXX; MT lacks \fbib{for}} chaos, \\
\poemlll       but formed it to be inhabited--- \\
\poeml ``I am the \divine{Lord} and there is no other. \\
\poeml \v{19}I didn't speak in secret, \\
\poemlll       from somewhere in a land of darkness; \\
\poeml I didn't say to Jacob's descendants, \\
\poemll    `Seek me in chaos.' \\
\poeml I, the \divine{Lord}, speak truth, \\
\poemll    declaring what is right. \\
\poeml \v{20}``Gather together and come; \\
\poemll    draw near and enter,\fnote{\fbackref{45:20} So 1QIsa\textsuperscript{a}; MT LXX read \fbib{together}} \\
\poemlll       your fugitives from the nations. \\
\poeml Those who carry around their wooden idols \\
\poemll    know nothing, \\
\poeml nor do those who keep praying to a god \\
\poemll    that cannot save. \\
\poeml \v{21}Explain and present a case! \\
\poemll    Yes, let them take counsel together. \\
\poeml Who announced this long ago, \\
\poemll    who declared it from the distant past? \\
\poemlll       Was it not I, the \divine{Lord}? \\
\poeml And there is no other God besides me, \\
\poemll    a righteous God and Savior; \\
\poemlll       and\fnote{\fbackref{45:21} So 1QIsa\textsuperscript{a}; MT LXX lack \fbib{and}} there is none besides me. \\
\poeml \v{22}Turn to me and be saved, \\
\poemll    all you ends of the earth. \\
\poemlll       For I am God, and there is no other.
\passage{Every Knee will Bow}
\poeml \v{23}By myself I have sworn--- \\
\poemll    from my mouth has gone out integrity, \\
\poemlll       a promise\fnote{\fbackref{45:23} Lit. \fbib{word}} that won't be revoked: \\
\poeml `To me every knee will bow, \\
\poemll    and\fnote{\fbackref{45:23} So 1QIsa\textsuperscript{a}; MT lacks \fbib{and}} every tongue will swear. ' \\
\poeml \v{24}One will say of me,\fnote{\fbackref{45:24} So 1QIsa\textsuperscript{a}; MT reads \fbib{one said of me}; LXX reads \fbib{saying}} \\
\poemll    `Only in the \divine{Lord} are victories and might.' \\
\poeml All who raged against him will come\fnote{\fbackref{45:24} So 1QIsa\textsuperscript{a} MT\textsuperscript{mss} LXX (pl.); MT (sing.)} to him \\
\poemll    and will be put to shame. \\
\poeml \v{25}In the \divine{Lord} all the descendants of Israel \\
\poemll    will triumph and make their boast.''
\end{poetry}
\labelchapt{46}
\passage{God is Unique and Eternal}

\begin{poetry}
\poeml \chapt{46}
\v{1}``Bel\fnote{\fbackref{46:1} I.e. the Babylonian sun god Marduk} bows down, Nebo\fnote{\fbackref{46:1} I.e. Nabu, the Babylonian god of astronomy and learning, son of Marduk} stoops low. \\
\poemll    Their idols are on beasts, on\fnote{\fbackref{46:1} So 1QIsa\textsuperscript{a}; MT LXX read \fbib{and on}} livestock. \\
\poemlll       Your loads are more burdensome than their reports.\fnote{\fbackref{46:1} So 1QIsa\textsuperscript{a}; 4QIsa\textsuperscript{b} MT LXX read \fbib{burdensome, a load for the weary}} \\
\poeml \v{2}They stoop, they bow down together, \\
\poemll    and\fnote{\fbackref{46:2} So 1QIsa\textsuperscript{a}; MT lacks \fbib{and}} they are not able to rescue the burden, \\
\poemlll       but they themselves go off\fnote{\fbackref{46:2} So 1QIsa\textsuperscript{a} LXX (pl.); 4QIsa\textsuperscript{b} MT (sing.)} into captivity. \\
\poeml \v{3}``Listen\fnote{\fbackref{46:3} So 1QIsa\textsuperscript{a} (sing.); MT LXX (pl.)} to me, house of Jacob, \\
\poemll    and all you remnant of the house of Israel, \\
\poeml who have been upheld from before your birth, \\
\poemll    and who have been carried from the womb. \\
\poeml \v{4}Even\fnote{\fbackref{46:4} So 1QIsa\textsuperscript{a}; MT reads \fbib{and even}} until your\fnote{\fbackref{46:4} 1QIsa\textsuperscript{a} MT LXX lack \fbib{your}} old age, I am the one, \\
\poemll    and I'll carry you even until your gray hairs come.\fnote{\fbackref{46:4} 1QIsa\textsuperscript{a} MT LXX lack \fbib{come}} \\
\poeml It is I who have created,\fnote{\fbackref{46:4} Or \fbib{made}} and I who will carry, \\
\poemll    and it is I who will bear and save. \\
\poeml \v{5}``To whom will you compare me, \\
\poemll    count me equal,\fnote{\fbackref{46:5} So 1QIsa\textsuperscript{a} (sing.); 1QIsa\textsuperscript{b} MT read \fbib{consider equal} (pl.); LXX reads \fbib{see}} or liken me, \\
\poemlll       so that I\fnote{\fbackref{46:5} So 1QIsa\textsuperscript{a}; MT reads \fbib{we}} may be compared? \\
\poeml \v{6}Those who pour out gold in\fnote{\fbackref{46:6} So 1QIsa\textsuperscript{a}; MT LXX read \fbib{from}} a purse, \\
\poemll    weigh silver in a balance, \\
\poeml hire a goldsmith in order to make\fnote{\fbackref{46:6} So 1QIsa\textsuperscript{a} LXX; 4QIsa\textsuperscript{b} MT read \fbib{make it}} a god, \\
\poemll    and then\fnote{\fbackref{46:6} Lit. \fbib{and}; so 1QIsa\textsuperscript{a} 1QIsa\textsuperscript{b} LXX; MT lacks \fbib{then}} they bow down and even worship it. \\
\poeml \v{7}And\fnote{\fbackref{46:7} So 1QIsa\textsuperscript{a}; 1QIsa\textsuperscript{b} MT LXX lack \fbib{And}} they lift it on their shoulders, carry it, \\
\poemll    set it up in its place, and there it stands. \\
\poemlll       It cannot move\fnote{\fbackref{46:7} So 1QIsa\textsuperscript{a} LXX; MT reads \fbib{one cannot remove it}} from that spot. \\
\poeml One may even call to\fnote{\fbackref{46:7} So 1QIsa\textsuperscript{a}; MT reads \fbib{may cry out}} it, but it cannot answer \\
\poemll    nor save him from his distress. \\
\poeml \v{8}``Remember this, and stand firm; \\
\poemll    take it again to heart, you rebels. \\
\poeml \v{9}Remember the former things from long ago, \\
\poemll    Because I am God, and there is no one else; \\
\poemlll       I am God, and there is none like me. \\
\poeml \v{10}I declare from the beginning things to follow,\fnote{\fbackref{46:10} So 1QIsa\textsuperscript{a} LXX 4QIsa\textsuperscript{c}; MT reads \fbib{the future}} \\
\poemll    and from ancient times things that have not yet been done; \\
\poeml saying, `My purpose will stand, \\
\poemll    and he\fnote{\fbackref{46:10} So 1QIsa\textsuperscript{a}; 1QIsa\textsuperscript{b} 4QIsa\textsuperscript{c} MT LXX read \fbib{I}} will accomplish everything that I please.' \\
\poeml \v{11}I am calling a bird of prey from the east, \\
\poemll    and from a far country a man with his\fnote{\fbackref{46:11} So 1QIsa\textsuperscript{a} 4QIsa\textsuperscript{d} MT; 1QIsa\textsuperscript{b} MT (vocalization) LXX read \fbib{of my}} purpose. \\
\poeml Indeed, I've spoken; \\
\poemll    I will certainly make it happen; \\
\poeml I've planned it;\fnote{\fbackref{46:11} So 1QIsa\textsuperscript{a}; 1QIsa\textsuperscript{b} 4QIsa\textsuperscript{c} MT LXX lack \fbib{it}} \\
\poemll    and I will certainly carry it out. \\
\poeml \v{12}``Listen to me, you hard-hearted, \\
\poemll    you who are far removed from righteousness: \\
\poeml \v{13}My righteousness is brought near\fnote{\fbackref{46:13} So 1QIsa\textsuperscript{a}; MT LXX 4QIsa\textsuperscript{c} read \fbib{I have brought near;}} and\fnote{\fbackref{46:13} So 1QIsa\textsuperscript{a}; MT lacks \fbib{and}} it's not far off, \\
\poemll    and my salvation won't delay. \\
\poeml I'll\fnote{\fbackref{46:13} So 1QIsa\textsuperscript{a} LXX; 1QIsa\textsuperscript{b} 4QIsa\textsuperscript{c} MT read \fbib{And I}} grant salvation in Zion, \\
\poemll    and\fnote{\fbackref{46:13} So 1QIsa\textsuperscript{a} 4QIsa\textsuperscript{c}; 1QIsa\textsuperscript{b} 4QIsa\textsuperscript{d} MT LXX lack \fbib{and}} to Israel, my glory.''
\end{poetry}
\labelchapt{47}
\passage{The Fall of Babylon}

\begin{poetry}
\poeml \chapt{47}
\v{1}``Come down and sit in the dust, \\
\poemll    Virgin Daughter of Babylon. \\
\poeml Sit on\fnote{\fbackref{47:1} So 1QIsa\textsuperscript{a}; 1QIsa\textsuperscript{b} MT use another preposition} the ground without a chair, \\
\poemll    Daughter of the Chaldeans! \\
\poeml For no longer will they call you \\
\poemll    tender and attractive. \\
\poeml \v{2}Take millstones and grind flour. \\
\poemll    Remove your veil, \\
\poeml strip off your robes,\fnote{\fbackref{47:2} So 1QIsa\textsuperscript{a}; 1QIsa\textsuperscript{b} 4QIsa\textsuperscript{d} MT read \fbib{skirt}} \\
\poemll    bare your legs, \\
\poemlll       and wade through the rivers. \\
\poeml \v{3}Your nakedness will be\fnote{\fbackref{47:3} So 1QIsa\textsuperscript{a}; 1QIsa\textsuperscript{b} MT read \fbib{let it be}} exposed, \\
\poemll    and your disgrace will also be seen. \\
\poeml I'll take vengeance, \\
\poemll    and I will spare no mortal. \\
\poeml \v{4}``Our Redeemer--- \\
\poemll    the \divine{Lord} of the Heavenly Armies is his name--- \\
\poemlll       is the Holy One of Israel. \\
\poeml \v{5}``Sit silent,\fnote{\fbackref{47:5} 1QIsa\textsuperscript{a} and MT use alternate forms of the same term} and enter into the darkness, \\
\poemll    you daughter of the Chaldeans; \\
\poeml for no more will they call you \\
\poemll    Queen of Kingdoms. \\
\poeml \v{6}I was angry with my people, \\
\poemll    and\fnote{\fbackref{47:6} So 1QIsa\textsuperscript{a}; 1QIsa\textsuperscript{b} MT lack \fbib{and}} I desecrated my heritage, \\
\poeml and gave them into your control. \\
\poemll    You showed them no mercy; \\
\poemlll       even on the aged you laid your yoke most heavily. \\
\poeml \v{7}You said, `I will always continue---Queen forever!' \\
\poemll    You didn't take these things into your thinking, \\
\poemlll       nor did you think about their consequences.\fnote{\fbackref{47:7} 1QIsa\textsuperscript{a} and MT use alternate forms; LXX reads \fbib{the last things}} \\
\poeml \v{8}``Now hear this, you wanton creature, \\
\poemll    lounging with no cares, \\
\poeml and saying to herself: \\
\poemll    `I am the one, and there will be none besides me; \\
\poeml I won't live as a widow, \\
\poemll    nor will I see\fnote{\fbackref{47:8} So 1QIsa\textsuperscript{a}; 4QIsa\textsuperscript{d} MT LXX read \fbib{know}} the loss of children.' \\
\poeml \v{9}Both of these things will overtake you \\
\poemll    suddenly on a single day: \\
\poeml loss of children and widowhood. \\
\poemll    They will come upon you in full measure, \\
\poeml despite the multitude of your incantations \\
\poemll    and the great power of your spells.''
\passage{Self-Deception of the Babylonians}
\poeml \v{10}``You trusted in your own knowledge.\fnote{\fbackref{47:10} So 1QIsa\textsuperscript{a}; MT LXX read \fbib{evil}} \\
\poemll    You said, `No one sees me.' \\
\poeml Your wisdom and knowledge have misled you. \\
\poemll    You said in your heart, \\
\poemlll       `I am the one, and there will be none besides me.' \\
\poeml \v{11}``But disaster will come\fnote{\fbackref{47:11} So 1QIsa\textsuperscript{a} (feminine); MT (masculine [incorrectly])} upon you, \\
\poemll    and\fnote{\fbackref{47:11} So 1QIsa\textsuperscript{a} LXX; MT lacks \fbib{and}} you will not know how to charm it away. \\
\poeml A calamity will befall you \\
\poemll    that you will not be able to\fnote{\fbackref{47:11} So 1QIsa\textsuperscript{a}; MT lacks \fbib{to}} ward off; \\
\poeml and devastation will come upon you suddenly, \\
\poemll    and\fnote{\fbackref{47:11} So 1QIsa\textsuperscript{a} LXX; MT lacks \fbib{and}} you won't anticipate it. \\
\poeml \v{12}``But\fnote{\fbackref{47:12} So 1QIsa\textsuperscript{a}; 1QIsa\textsuperscript{b} MT LXX lack \fbib{But}} stand up now with your spells \\
\poemll    and your many incantations, \\
\poeml at which you have labored from your childhood until today,\fnote{\fbackref{47:12} So 1QIsa\textsuperscript{a}; \fbib{childhood. Perhaps you can gain some profit; perhaps you may inspire fear}. LXX reads \fbib{if you can gain some profit}.} \\
\poeml \v{13}according to\fnote{\fbackref{47:12} So 1QIsa\textsuperscript{a}; MT LXX read \fbib{You are wearied by}} your multiple schemes. \\
\poeml Let them stand up now--- \\
\poemll    those who conjure\fnote{\fbackref{47:12} So 1QIsa\textsuperscript{a} LXX; MT reads \fbib{divide}} the heavens \\
\poeml and\fnote{\fbackref{47:12} So 1QIsa\textsuperscript{a}; MT LXX lack \fbib{and}} gaze at the stars, \\
\poemll    predicting at the new moons--- \\
\poemlll       and save you from what is about to happen to them.\fnote{\fbackref{47:12} So 1QIsa\textsuperscript{a}; MT reads \fbib{what things are about to happen to you}; LXX reads \fbib{what is about to happen to you}} \\
\poeml \v{14}``See, they are just like stubble; \\
\poemll    fire burns them up. \\
\poeml They could not\fnote{\fbackref{47:14} So 1QIsa\textsuperscript{a}; MT reads \fbib{cannot}} even save themselves \\
\poemll    from the power of the flame. \\
\poeml There will be no coals for warming oneself, \\
\poemll    no fire to sit by. \\
\poeml \v{15}So will they be to you---those with whom you toiled \\
\poemll    and did business since your childhood--- \\
\poeml they wander about, each in his own direction; \\
\poemll    there is not one who can save you.
\end{poetry}
\labelchapt{48}
\passage{God the Creator and Redeemer}

\begin{poetry}
\poeml \chapt{48}
\v{1}``Listen to this, house of Jacob, \\
\poemll    you who are called by the name of Israel, \\
\poemlll       and who have come forth from Judah's loins;\fnote{\fbackref{48:1} Lit. \fbib{waters}; this word is misspelled in both 1QIsa\textsuperscript{a} and MT} \\
\poeml you who swear oaths in the name of the \divine{Lord} \\
\poemll    and invoke the God of Israel--- \\
\poemlll       but not in truth, nor in good faith. \\
\poeml \v{2}For they name themselves after the holy city, \\
\poemll    and rely on the God of Israel--- \\
\poemlll       the \divine{Lord} of the Heavenly Armies is his name. \\
\poeml \v{3}I foretold the former things long ago; \\
\poemll    it\fnote{\fbackref{48:3} So 1QIsa\textsuperscript{a}; MT LXX read \fbib{they}} went forth from my mouth, \\
\poemlll       and I disclosed them; \\
\poeml Suddenly, I acted, \\
\poemll    and they came to pass. \\
\poeml \v{4}Because I knew\fnote{\fbackref{48:4} Lit. \fbib{Because of my knowledge}} that you are obstinate, \\
\poemll    and because your neck is an iron sinew, \\
\poemlll       and your forehead is bronze, \\
\poeml \v{5}I told you these things long ago; \\
\poemll    I announced them to you before they happened \\
\poeml so that you couldn't say, `My idol did them; \\
\poemll    my\fnote{\fbackref{48:5} So 1QIsa\textsuperscript{a}; MT reads \fbib{and my}} carved image or metal idol ordained them.' \\
\poeml \v{6}``You have heard---now look at them all! \\
\poemll    How\fnote{\fbackref{48:6} Lit. \fbib{And how}} can you not admit them? \\
\poeml From now on, I'll make you hear new things, \\
\poemll    hidden things that you have not known. \\
\poeml \v{7}They are created now, and not long ago; \\
\poemll    you didn't hear them before today, \\
\poemlll       so you cannot say, `Yes, I knew them.' \\
\poeml \v{8}And\fnote{\fbackref{48:8} So 1QIsa\textsuperscript{a}; 4QIsa\textsuperscript{b} MT LXX lack \fbib{And}} neither had you heard, nor did you understand, \\
\poemll    nor did you open\fnote{\fbackref{48:8} So 1QIsa\textsuperscript{a} Targ; MT reads \fbib{did your ear open itself}; CaiGen Syr Vulg read \fbib{was your ear uncovered}} your ear long ago.\fnote{\fbackref{48:8} Lit. \fbib{ear from of old}} \\
\poeml Indeed, I knew that\fnote{\fbackref{48:8} So 1QIsa\textsuperscript{a} LXX; MT lacks \fbib{that}} you would act very deceitfully, \\
\poemll    and they would call\fnote{\fbackref{48:8} So 1QIsa\textsuperscript{a}; LXX Targ read \fbib{and you would be called}; 4QIsa\textsuperscript{d} MT read \fbib{deceitfully, calling}} you a rebel from birth. \\
\poeml \v{9}I defer my anger for my name's sake, \\
\poemll    and as my first act\fnote{\fbackref{48:9} Lit. \fbib{and for my commencement}; or \fbib{and for my profanation}; so 1QIsa\textsuperscript{a}; 4QIsa\textsuperscript{d} MT read \fbib{for my praise}; LXX reads \fbib{for my glorious deeds}} I'm restraining it for you, \\
\poemlll       so as not to cut you off. \\
\poeml \v{10}Look, I have refined you, but not like silver; \\
\poemll    I have purified\fnote{\fbackref{48:10} So 1QIsa\textsuperscript{a}; MT LXX read \fbib{chosen}} you in the furnace of affliction. \\
\poeml \v{11}For my own sake---Yes, for my own sake!---I'm doing it; \\
\poemll    indeed, how can I be profaned?\fnote{\fbackref{48:11} Or \fbib{can I wait}; so 1QIsa\textsuperscript{a}; 4QIsa\textsuperscript{d} MT read \fbib{can it be profaned}; 4QIsa\textsuperscript{d} MT read \fbib{can it wait}; LXX reads \fbib{can my name be profaned}} \\
\poemlll       Furthermore, I won't give my glory to another.''
\passage{The \divine{Lord} Calls Israel}
\poeml \v{12}``Listen to these things,\fnote{\fbackref{48:12} So 1QIsa\textsuperscript{a}; MT LXX read \fbib{me}} Jacob, \\
\poemll    and Israel, whom I have called. \\
\poeml I am the One: I am the first, \\
\poemll    I am even\fnote{\fbackref{48:12} So 1QIsa\textsuperscript{a} MT; 4QIsa\textsuperscript{d} reads \fbib{also}} the last. \\
\poeml \v{13}Moreover, my hands laid\fnote{\fbackref{48:13} So 1QIsa\textsuperscript{a}; MT LXX read \fbib{hand laid}} the earth's foundation, \\
\poemll    and my right hand spread out the heavens. \\
\poeml I call out to them, \\
\poemll    and\fnote{\fbackref{48:13} So 1QIsa\textsuperscript{a} 4QIsa\textsuperscript{c} 4QIsa\textsuperscript{d} LXX; MT lacks \fbib{and}} they stand up together. \\
\poeml \v{14}Let all of them come together and listen:\fnote{\fbackref{48:14} So 1QIsa\textsuperscript{a} LXX; 4QIsa\textsuperscript{d} MT read \fbib{Come together, all of you, and listen!}} \\
\poemll    Who is there among them that could declare\fnote{\fbackref{48:14} So 1QIsa\textsuperscript{a}; 4QIsa\textsuperscript{d} MT LXX read \fbib{has declared}} these things? \\
\poeml ``The \divine{Lord} loves me,\fnote{\fbackref{48:14} So 1QIsa\textsuperscript{a}; 4QIsa\textsuperscript{d} MT read \fbib{him}; LXX reads \fbib{you}} \\
\poemll    and he will accomplish\fnote{\fbackref{48:14} So 1QIsa\textsuperscript{a} (misspelling \fbib{carry out}); 4QIsa\textsuperscript{d} MT read \fbib{He will carry out}; LXX reads \fbib{I have carried out}} my purpose\fnote{\fbackref{48:14} So 1QIsa\textsuperscript{a}; 4QIsa\textsuperscript{d} MT LXX\textsuperscript{ms} read \fbib{his purpose}; LXX reads \fbib{your purpose}} against Babylon; \\
\poemlll       his arm\fnote{\fbackref{48:14} I.e. the Messiah; so 1QIsa\textsuperscript{a}; MT reads \fbib{and his arm}; LXX reads \fbib{to do away with the offspring}} will be against the Chaldeans. \\
\poeml \v{15}I---Yes, I!---have spoken; \\
\poemll    indeed, I've called and\fnote{\fbackref{48:15} So 1QIsa\textsuperscript{a}; 4QIsa\textsuperscript{d} MT read \fbib{called him}; 4QIsa\textsuperscript{c} LXX read \fbib{brought him}} I've brought him, \\
\poemlll       and he will make his path successful.\fnote{\fbackref{48:15} Or \fbib{his path will be successful}; so 1QIsa\textsuperscript{a} (feminine) and 4QIsa\textsuperscript{d} MT (masculine); 4QIsa\textsuperscript{c} LXX read \fbib{I will make his path successful}} \\
\poeml \v{16}Draw near to me, and\fnote{\fbackref{48:16} So 1QIsa\textsuperscript{a} LXX; MT lacks \fbib{and}} listen to this: \\
\poemll    `From the beginning I haven't spoken in secret; \\
\poeml at\fnote{\fbackref{48:16} So 1QIsa\textsuperscript{a} LXX; MT reads \fbib{from}} the time it happened, I was there.' \\
\poemll    And now the \divine{Lord} God, and his Spirit, has sent me.\fnote{\fbackref{48:16} Or \fbib{has sent me and his Spirit}.} \\
\poeml \v{17}``This is what the \divine{Lord} says, \\
\poemll    your Redeemer, the Holy One of Israel: \\
\poeml ``I am the \divine{Lord} your God, \\
\poemll    who teaches you how to succeed, \\
\poemlll       who directs you\fnote{\fbackref{48:17} 1QIsa\textsuperscript{a} misspells this word} in the path by which\fnote{\fbackref{48:17} So 1QIsa\textsuperscript{a} LXX; 1QIsa\textsuperscript{b} 4QIsa\textsuperscript{c} 4QIsa\textsuperscript{d} MT lack \fbib{which}} you should go. \\
\poeml \v{18}Now\fnote{\fbackref{48:18} So 1QIsa\textsuperscript{a} 1QIsa\textsuperscript{b} 4QIsa\textsuperscript{c} LXX; MT lacks \fbib{Now}} if only you had paid attention to my commandments! \\
\poemll    Then your peace would have been like a river, \\
\poemlll       and your success like the waves of the sea. \\
\poeml \v{19}Your descendants would've been like the sand, \\
\poemll    and your offspring\fnote{\fbackref{48:19} So 1QIsa\textsuperscript{a}; 1QIsa\textsuperscript{b} MT LXX read \fbib{the offspring of your loins}} like its numberless grains. \\
\poeml Their name wouldn't have been cut off \\
\poemll    or annihilated out of my reach. \\
\poeml \v{20}``Go out from Babylon, flee from the Chaldeans! \\
\poemll    With happy shouts, announce \\
\poeml and\fnote{\fbackref{48:20} So 1QIsa\textsuperscript{a} LXX; 1QIsa\textsuperscript{b} MT lack \fbib{and}} proclaim this\fnote{\fbackref{48:20} So 1QIsa\textsuperscript{a}; 1QIsa\textsuperscript{b} 4QIsa\textsuperscript{d} MT LXX read \fbib{this. Send it forth}} to the ends\fnote{\fbackref{48:20} So 1QIsa\textsuperscript{a}; 4QIsa\textsuperscript{d} MT LXX read \fbib{end}} of the earth: \\
\poemll    Say, `The \divine{Lord} has redeemed his servant Jacob!' \\
\poeml \v{21}They didn't thirst when he led him\fnote{\fbackref{48:21} So 1QIsa\textsuperscript{a} LXX\textsuperscript{mss}; 4QIsa\textsuperscript{d} MT LXX read \fbib{them}} through the desolate places. \\
\poemll    He made water gush\fnote{\fbackref{48:21} So 1QIsa\textsuperscript{a} Syr; cf. Ps 78:20 and 105:41; 1QIsa\textsuperscript{b} 4QIsa\textsuperscript{d} MT read \fbib{flow;} LXX reads \fbib{he will bring forth}} from a rock for them; \\
\poemlll       he split open the rock, and water gushed out. \\
\poeml \v{22}``But\fnote{\fbackref{48:22} So 1QIsa\textsuperscript{a}; 1QIsa\textsuperscript{b} MT LXX lack \fbib{But}} there is no peace,'' says the \divine{Lord}, ``for the wicked.''
\end{poetry}
\labelchapt{49}
\passage{The Servant of the \divine{Lord}}

\begin{poetry}
\poeml \chapt{49}
\v{1}``Listen to me, you coastlands! \\
\poemll    Pay\fnote{\fbackref{49:1} So 1QIsa\textsuperscript{a} LXX; 1QIsa\textsuperscript{b} MT read \fbib{and pay}} attention, you people\fnote{\fbackref{49:1} Lit. \fbib{peoples}; i.e. non-Israelis then in the land} from far away! \\
\poeml The \divine{Lord} called me from the womb; \\
\poemll    while I was still in my mother's body, \\
\poemlll       he pronounced my name. \\
\poeml \v{2}He made my mouth like a sharp sword; \\
\poemll    he hid me in the shadow of his hands.\fnote{\fbackref{49:2} So 1QIsa\textsuperscript{a}; 1QIsa\textsuperscript{b} MT LXX read \fbib{hand}} \\
\poeml He made me like a polished arrow \\
\poemll    and hid me away in his quivers.\fnote{\fbackref{49:2} So 1QIsa\textsuperscript{a}; 1QIsa\textsuperscript{b} 4QIsa\textsuperscript{d} MT LXX read \fbib{quiver}} \\
\poeml \v{3}He said to me: `You are my servant, \\
\poemll    Israel, in whom I will glorify myself.' \\
\poeml \v{4}``I\fnote{\fbackref{49:4} So 1QIsa\textsuperscript{a}; 4QIsa\textsuperscript{d} MT LXX read \fbib{But I}} said: `I've labored for nothing. \\
\poemll    I've exhausted my strength on futility and on\fnote{\fbackref{49:4} So 1QIsa\textsuperscript{a} LXX; 1QIsa\textsuperscript{b} MT lack \fbib{on}} emptiness.' \\
\poeml Yet surely my recompense is with the \divine{Lord}, \\
\poemll    and my reward is with my God. \\
\poeml \v{5}``And now, says the \divine{Lord}, \\
\poemll    who formed you\fnote{\fbackref{49:5} So 1QIsa\textsuperscript{a}; 1QIsa\textsuperscript{b} MT LXX \fbib{me}} from the womb as his servant \\
\poeml to bring Jacob back to him \\
\poemll    so that Israel might be gathered\fnote{\fbackref{49:5} So 1QIsa\textsuperscript{a} MT\textsuperscript{qere} LXX; 4QIsa\textsuperscript{d} MT \fbib{might not be gathered} (misspelling)} to him--- \\
\poeml and I am honored in the \divine{Lord}'s sight \\
\poemll    and my God has been my help\fnote{\fbackref{49:5} So 1QIsa\textsuperscript{a}; 1QIsa\textsuperscript{b} MT LXX \fbib{strength}}--- \\
\poeml \v{6}he says: ``It is too small a thing for you to be my servant, \\
\poemll    to raise up the tribes of Israel\fnote{\fbackref{49:6} So 1QIsa\textsuperscript{a}; 1QIsa\textsuperscript{b} MT LXX \fbib{Jacob}} \\
\poemlll       and bring back those of Jacob\fnote{\fbackref{49:6} So 1QIsa\textsuperscript{a}; 1QIsa\textsuperscript{b} MT LXX \fbib{Israel}} I have preserved. \\
\poeml I'll also make you as a light to the nations, \\
\poemll    to be my salvation to the ends\fnote{\fbackref{49:6} So 1QIsa\textsuperscript{a}; 1QIsa\textsuperscript{b} MT LXX \fbib{end}} of the earth. \\
\poeml \v{7}``This is what my \divine{Lord}\fnote{\fbackref{49:7} So 1QIsa\textsuperscript{a} 1QIsa\textsuperscript{b}; MT LXX lacks \fbib{my Lord}} says--- \\
\poemll    the \divine{Lord} your Redeemer, O Israel,\fnote{\fbackref{49:7} So 1QIsa\textsuperscript{a} LXX; 1QIsa\textsuperscript{b} MT read \fbib{the Redeemer of Israel}} \\
\poemlll       and his Holy One--- \\
\poeml to one despised by people,\fnote{\fbackref{49:7} So 1QIsa\textsuperscript{a} 4QIsa\textsuperscript{d}; MT CaiGen LXX reads \fbib{to one people despise}} \\
\poemll    to those abhorred\fnote{\fbackref{49:7} So 1QIsa\textsuperscript{a}; MT LXX reads \fbib{one abhorred}} as a nation, \\
\poemlll       to the servant of rulers: \\
\poeml ``Kings see\fnote{\fbackref{49:7} So 1QIsa\textsuperscript{a}; 1QIsa\textsuperscript{b} MT read \fbib{Kings will see}; LXX reads \fbib{Kings will see him}} and arise, \\
\poemll    and princes\fnote{\fbackref{49:7} So 1QIsa\textsuperscript{a}; 1QIsa\textsuperscript{b} reads \fbib{They will rise}; MT LXX read \fbib{princes will rise, and they}} will bow down, \\
\poeml because of the \divine{Lord} who is faithful, \\
\poemll    the Holy One of Israel, \\
\poemlll       the one who has\fnote{\fbackref{49:7} So 1QIsa\textsuperscript{a}; MT reads \fbib{and the one who has}; LXX reads \fbib{and I have}} chosen you.''
\passage{The Restoration of Israel}
\poeml \v{8}``This what the \divine{Lord} says: \\
\poeml ``I'll answer\fnote{\fbackref{49:8} So 1QIsa\textsuperscript{a}; MT LXX reads \fbib{I have answered}} you in a time of favor, \\
\poemll    and on a day of salvation I'll help\fnote{\fbackref{49:8} So 1QIsa\textsuperscript{a}; 1QIsa\textsuperscript{b} MT LXX read \fbib{I have helped}} you. \\
\poeml I have watched over you, \\
\poemll    and given you as a covenant for the people, \\
\poeml to restore the land, \\
\poemll    to reassign the inheritances that have been devastated; \\
\poeml \v{9}saying to captives, `Come out!' \\
\poemll    and\fnote{\fbackref{49:9} So 1QIsa\textsuperscript{a} LXX; MT lack \fbib{and}} to those who are in darkness, `Be free!'\fnote{\fbackref{49:9} Or \fbib{darkness,} `\fbib{Show yourselves!'}} \\
\poeml ``They will feed on all the mountains,\fnote{\fbackref{49:9} So 1QIsa\textsuperscript{a}; MT reads \fbib{by the roads}; LXX reads \fbib{in all their roads}} \\
\poemll    and their pasture will be on all the barren hills. \\
\poeml \v{10}They won't hunger or thirst, \\
\poemll    nor will the desert heat or sun beat upon them; \\
\poeml for the one who has compassion on them will drive them \\
\poemll    and guide them alongside springs of water. \\
\poeml \v{11}I'll turn all my mountains into a road, \\
\poemll    and my highways will be raised up. \\
\poeml \v{12}``Watch! They'll come from far away--- \\
\poemll    some from the north and from the west, \\
\poemlll       and others from the Aswan region.''\fnote{\fbackref{49:12} Lit. \fbib{Syenes}; so 1QIsa\textsuperscript{a}; MT reads \fbib{Syene}; LXX reads \fbib{Persians}} \\
\poeml \v{13}Shout with joy, you heavens, \\
\poemll    and rock with glee, you earth! \\
\poemlll       Break out in song, you mountains!\fnote{\fbackref{49:13} So 1QIsa\textsuperscript{a}; MT LXX read \fbib{Let the mountains break out}; MT\textsuperscript{qere, mss} read \fbib{And break out, you mountains}} \\
\poeml The \divine{Lord} is comforting\fnote{\fbackref{49:13} So 1QIsa\textsuperscript{a}; MT LXX read \fbib{has comforted}} his people \\
\poemll    and will have compassion on his afflicted ones.
\passage{Zion Not Forgotten}
\poeml \v{14}``But Zion said, `The \divine{Lord} has abandoned me, \\
\poemll    and my God\fnote{\fbackref{49:14} So 1QIsa\textsuperscript{a} corrector; Lit. \fbib{my God} written above \fbib{my Lord}; MT LXX read \fbib{my Lord}} has forgotten me.' \\
\poeml \v{15}``Can a woman forget her nursing child, \\
\poemll    or have no compassion for the child of her womb? \\
\poeml Even these mothers may forget; \\
\poemll    But as for me, I'll never forget you! \\
\poeml \v{16}Look! I've inscribed you on the palms of my hands, \\
\poemll    and\fnote{\fbackref{49:16} So 1QIsa\textsuperscript{a}; MT LXX read \fbib{and}} your walls are forever before me. \\
\poeml \v{17}Your builders\fnote{\fbackref{49:17} So 1QIsa\textsuperscript{a} Aquila Vulg LXX; MT reads \fbib{sons}} are working faster than your destroyers, \\
\poemll    and those who devastated you go away from you. \\
\poeml \v{18}Lift up your eyes and look around--- \\
\poemll    they have all gathered together \\
\poemlll       and are coming to you. \\
\poeml ``As surely as I live,'' says the \divine{Lord}, \\
\poemll    ``you will clothe yourself with all of them like ornaments, \\
\poemlll       and tie them on like a bride. \\
\poeml \v{19}Indeed, your ruins, your desolate places, \\
\poemll    and your devastated land \\
\poeml will now be too crowded for your inhabitants, \\
\poemll    while those who swallowed you up will be far away. \\
\poeml \v{20}``The children who are grieving at present\fnote{\fbackref{49:20} Lit. \fbib{children of your bereavement}} \\
\poemll    will yet say in your hearing, \\
\poeml `This place is too crowded for me; \\
\poemll    make room for me, \\
\poemlll       so I may have a place to live.' \\
\poeml \v{21}Then you'll ask\fnote{\fbackref{49:21} Lit. \fbib{say}} in your heart, \\
\poemll    `Who bore these children for me, \\
\poeml although I was childless and barren, \\
\poemll    and\fnote{\fbackref{49:21} So 1QIsa\textsuperscript{a}; MT LXX lack \fbib{and}} an exile\fnote{\fbackref{49:21} LXX lacks \fbib{and an exile}} and cast aside? \\
\poeml Who\fnote{\fbackref{49:21} So 1QIsa\textsuperscript{a}. MT LXX read \fbib{And who}} brought these up? \\
\poemll    Look!\fnote{\fbackref{49:21} So 1QIsa\textsuperscript{a} MT; LXX lacks \fbib{Look!}} For my part I was left all alone; \\
\poemlll       but as for these, where have they come from?' \\
\poeml \v{22}``For\fnote{\fbackref{49:22} So 1QIsa\textsuperscript{a}; MT LXX lack \fbib{For}} this what the \divine{Lord}\fnote{\fbackref{9:22} So 1QIsa\textsuperscript{a} LXX; MT reads \fbib{my Lord the \divine{Lord}}} says, \\
\poemll    `Watch! I'll lift up my hand to the nations \\
\poeml and raise my banner to the\fnote{\fbackref{49:22} So 1QIsa\textsuperscript{a}; MT lacks \fbib{the}} peoples.\fnote{\fbackref{49:22} So 1QIsa\textsuperscript{a} MT; LXX reads \fbib{islands}} \\
\poemll    They will bring your sons in their arms, \\
\poemlll       and your daughters will be carried on their shoulders.' \\
\poeml \v{23}``Oh, yes!\fnote{\fbackref{49:23} Or \fbib{And aha!}; so 1QIsa\textsuperscript{a} cf. Isa 55:1; MT LXX read \fbib{And it will happen that}} Kings will be your foster fathers, \\
\poemll    and their queens will be your nursing mothers. \\
\poeml They will bow to you with their faces to the ground, \\
\poemll    and lick the dust from your feet. \\
\poeml Then you will know that I am the \divine{Lord}; \\
\poemll    those who hope in me will not be disappointed. \\
\poeml \v{24}``Can they seize plunder\fnote{\fbackref{49:24} So 1QIsa\textsuperscript{a} LXX; MT reads \fbib{plunder be seized}} from warriors, \\
\poemll    or\fnote{\fbackref{49:24} So 1QIsa\textsuperscript{a} LXX; MT lacks \fbib{or}} can the captives of tyrants\fnote{\fbackref{49:24} So 1QIsa\textsuperscript{a} LXX Targ Vulg; MT reads \fbib{righteous ones}} be rescued? \\
\poeml \v{25}But this is what the \divine{Lord} says: \\
\poeml ``He will seize\fnote{\fbackref{49:25} So 1QIsa\textsuperscript{a} LXX; MT reads \fbib{will be seized};} even the warriors' plunder,\fnote{\fbackref{49:25} So 1QIsa\textsuperscript{a}; MT LXX reads \fbib{captives}} \\
\poemll    and the captives\fnote{\fbackref{49:25} So 1QIsa\textsuperscript{a}; MT LXX reads \fbib{plunder}} of tyrants will be rescued. \\
\poeml I myself will quarrel with those who have a quarrel with you,\fnote{\fbackref{49:25} So 1QIsa\textsuperscript{a}; MT Vulg read \fbib{with your contender}; LXX reads \fbib{your cause}} \\
\poemll    and I myself will save your children. \\
\poeml \v{26}``I'll make those who mistreat you\fnote{\fbackref{49:26} So 1QIsa\textsuperscript{a} probably misspells this word as \fbib{I will eat}.} eat their own flesh, \\
\poemll    and they will get drunk on their own blood, as with new wine. \\
\poeml ``Then all mankind will know that I am the \divine{Lord} \\
\poemll    your Savior and your Redeemer, \\
\poemlll       the Mighty One of Jacob.''
\end{poetry}
\labelchapt{50}
\passage{A Call to Return to God}

\begin{poetry}
\poeml \chapt{50}
\v{1}This is what the \divine{Lord} says: \\
\poeml ``Where is your mother's certificate of divorce \\
\poemll    with which I sent her away? \\
\poeml Or to which of my creditors did I sell you? \\
\poemll    Look! It's because of your sins that you were sold, \\
\poemlll       and because of your transgressions that your mother was sent away. \\
\poeml \v{2}Why is it that when I came, no one was there? \\
\poemll    Why was there no answer when I called? \\
\poeml Was my arm\fnote{\fbackref{50:2} Lit. \fbib{my hand}} too short to redeem you? \\
\poemll    Do\fnote{\fbackref{50:2} So 1QIsa\textsuperscript{a}; MT LXX read \fbib{Or do}} I lack the strength to rescue you? \\
\poeml Look! By my mere rebuke I dry up the sea, \\
\poemll    I turn rivers into a desert. \\
\poeml Their fish stink for lack of water \\
\poemll    and die of thirst. \\
\poeml \v{3}I clothe the skies with darkness \\
\poemll    and make sackcloth their covering.''
\passage{The Servant's Obedience}
\poeml \v{4}``The Lord \divine{God} has given me \\
\poemll    a learned tongue, so that I may know \\
\poemlll       how to sustain the weary with words. \\
\poeml And\fnote{\fbackref{50:4} So 1QIsa\textsuperscript{a}; MT LXX lack \fbib{And}} morning after morning he wakens, \\
\poemll    and\fnote{\fbackref{50:4} So 1QIsa\textsuperscript{a}; MT LXX lack \fbib{and}} he wakens my ear to \\
\poemlll       listen like those who are being taught. \\
\poeml \v{5}My Lord \divine{God}\fnote{\fbackref{50:5} So 1QIsa\textsuperscript{a}; MT reads \fbib{the Lord}} has opened my ears, \\
\poemll    and I did not rebel; \\
\poemlll       I did not shrink back. \\
\poeml \v{6}I gave my back to those who beat me \\
\poemll    and my cheeks to those who pulled out my beard.\fnote{\fbackref{50:6} So MT; 1QIsa\textsuperscript{a} employs an incorrect reading; LXX reads \fbib{to blows}} \\
\poeml I did not turn away\fnote{\fbackref{50:6} So 1QIsa\textsuperscript{a} LXX; MT reads \fbib{hide}} my face \\
\poemll    from insults and spitting. \\
\poeml \v{7}For the Lord \divine{God} helps me, \\
\poemll    so I won't be disgraced. \\
\poeml Therefore I've made my face like flint, \\
\poemll    and I know that I won't be put to shame.''
\passage{The Servant's Vindication}
\poeml \v{8}The one who vindicates me is near. \\
\poemll    Who, then, will bring a charge against me? \\
\poemlll       Let's face each other! \\
\poeml Who has a case against me? \\
\poemll    Let him confront me! \\
\poeml \v{9}See! It is the Lord \divine{God} who will help me. \\
\poemll    Who is it that will declare me guilty? \\
\poeml See! They will all wear out like a garment; \\
\poemll    moths will eat them up. \\
\poeml \v{10}Who among you fears\fnote{\fbackref{50:10} So 1QIsa\textsuperscript{a} (pl.); MT LXX read \fbib{fear} (sing.)} the \divine{Lord}, \\
\poemll    obeying the voice of his servant, \\
\poeml who among you\fnote{\fbackref{50:10} 1QIsa\textsuperscript{a} MT LXX lack \fbib{among you}} walks\fnote{\fbackref{50:10} So 1QIsa\textsuperscript{a} LXX (pl.); MT reads \fbib{walks} (sing.)} in darkness\fnote{\fbackref{50:10} So 1QIsa\textsuperscript{a}; MT employs a different form} \\
\poemll    and has no light? \\
\poeml Let him trust in the name of the \divine{Lord}, \\
\poemll    and rely upon his God. \\
\poeml \v{11}Look! All those\fnote{\fbackref{50:11} So 1QIsa\textsuperscript{a}; MT LXX read \fbib{you}} who light a fire, \\
\poemll    who surround yourselves with flaming torches--- \\
\poeml walk by the light of your fire, \\
\poemll    and by the torches that you have set ablaze! \\
\poeml This is what you will receive from my hand: \\
\poemll    you will lie down in torment.
\end{poetry}
\labelchapt{51}
\passage{Deliverance for Zion}

\begin{poetry}
\poeml \chapt{51}
\v{1}``Listen to me, you who pursue righteousness, \\
\poemll    you who seek the \divine{Lord}! \\
\poeml Look to the rock from which you were cut, \\
\poemll    to the quarry from which you were hewn. \\
\poeml \v{2}Look to Abraham your father, \\
\poemll    and to Sarah who gave you birth. \\
\poeml For when he was only one person I called him, \\
\poemll    but I made him fruitful\fnote{\fbackref{51:2} So 1QIsa\textsuperscript{a}; MT reads \fbib{blessed him}; LXX reads \fbib{blessed him and loved him}} and made him many. \\
\poeml \v{3}For the \divine{Lord} will have compassion on Zion, \\
\poemll    have compassion on all her ruins. \\
\poeml He will make her wilderness like Eden, \\
\poemll    and her deserts like the garden of the \divine{Lord}. \\
\poeml Joy and gladness will be found in her, \\
\poemll    thanksgiving, and the sound of singing.\fnote{\fbackref{51:3} Or \fbib{music}} \\
\poemlll       Sorrow and sighing will flee away.\fnote{\fbackref{51:3} So 1QIsa\textsuperscript{a}; MT LXX lack this line; cf Isa 51:11} \\
\poeml \v{4}``Pay attention to me, my people! \\
\poemll    Listen to me, my nation! \\
\poeml For instruction\fnote{\fbackref{51:4} Or \fbib{For the Law}} will go out from me, \\
\poemll    and my justice will become a light for the nations.\fnote{\fbackref{51:4} Lit. \fbib{peoples}} \\
\poeml I will quickly bring \v{5}my deliverance near; \\
\poemll    my salvation is on the way. \\
\poeml His arm\fnote{\fbackref{51:5} So 1QIsa\textsuperscript{a}; 1QIsa\textsuperscript{b} reads \fbib{My arms}; or \fbib{My arm}; MT LXX read \fbib{My arms}} will bring justice to\fnote{\fbackref{51:5} The verb is pl. in 1QIsa\textsuperscript{a} MT} the nations;\fnote{\fbackref{51:5} Lit. \fbib{peoples}} \\
\poemll    the coastlands will hope for him,\fnote{\fbackref{51:5} So 1QIsa\textsuperscript{a}; 1QIsa\textsuperscript{b} MT LXX read \fbib{me}} \\
\poemlll       and they will wait for his arm.\fnote{\fbackref{51:5} So 1QIsa\textsuperscript{a}; 1QIsa\textsuperscript{b} MT LXX read \fbib{my arm}} \\
\poeml \v{6}``Lift up your eyes, you\fnote{\fbackref{51:6} So 1QIsa\textsuperscript{a}; MT LXX read \fbib{to the}} heavens \\
\poemll    and look to the earth beneath; \\
\poemlll       and see who created these.\fnote{\fbackref{51:6} So 1QIsa\textsuperscript{a}; 1QIsa\textsuperscript{b} MT LXX read \fbib{for the heavens will vanish like smoke, and the earth will wear out like a garment}} \\
\poeml Its inhabitants will die just like this;\fnote{\fbackref{51:6} Or \fbib{like gnats}} \\
\poemll    but my salvation will be forever, \\
\poemlll       and my deliverance will never fail. \\
\poeml \v{7}Listen to me, you who know righteousness, \\
\poemll    you people who have my instruction\fnote{\fbackref{51:7} Or \fbib{Law}} in their hearts. \\
\poeml Don't fear the insults of mortals, \\
\poemll    and don't be dismayed at their hateful words.\fnote{\fbackref{51:7} So 1QIsa\textsuperscript{a} 1QIsa\textsuperscript{b}; spelling differs from MT; LXX reads \fbib{contempt}} \\
\poeml \v{8}For moths will eat them up just like a garment, \\
\poemll    and worms will devour them like wool; \\
\poeml but my deliverance will last\fnote{\fbackref{51:8} Lit. \fbib{be}} forever, \\
\poemll    and my salvation to all generations. \\
\poeml \v{9}``Awake! Awake! Clothe yourself with strength, \\
\poemll    you arm\fnote{\fbackref{51:9} I.e. \fbib{the Messiah}} of the \divine{Lord}! \\
\poeml Awake, as in days gone by, \\
\poemll    as in generations of long ago. \\
\poeml Was it not you who split apart\fnote{\fbackref{51:9} So 1QIsa\textsuperscript{a} 4QIsa\textsuperscript{c} Vulg (cf. Job 26:12). MT LXX\textsuperscript{mss} read \fbib{cut in pieces}} Rehob,\fnote{\fbackref{51:9} So 1QIsa\textsuperscript{a} MT qere reads \fbib{Rahab}} \\
\poemll    who pierced that sea monster through?\fnote{\fbackref{51:9} So 1QIsa\textsuperscript{a} 4QIsa\textsuperscript{c} MT Vulg; LXX lacks \fbib{Was it{\ldots}through?}} \\
\poeml \v{10}Was it not you who dried up the sea, \\
\poemll    the waters of the great deep, \\
\poeml who made a road in\fnote{\fbackref{51:10} So 1QIsa\textsuperscript{a}; MT LXX lack \fbib{in}} the depths of the sea \\
\poemll    so that the redeemed could cross over?''
\passage{A Promise of Return to the Land}
\poeml \v{11}``The scattered ones\fnote{\fbackref{51:11} So 1QIsa\textsuperscript{a}; MT LXX read \fbib{ransomed ones}; 1QIsa\textsuperscript{a} corrector wrote \fbib{redeemed} then erased and wrote \fbib{scattered ones}} of the \divine{Lord} will return, \\
\poemll    and they will enter Zion with singing. \\
\poeml Everlasting joy will be upon their heads; \\
\poemll    they will attain joy and gladness, \\
\poemlll       and\fnote{\fbackref{51:11} So 1QIsa\textsuperscript{a} 4QIsa\textsuperscript{c} LXX; MT LXX lack \fbib{and}} sorrow and sighing will flee away.\fnote{\fbackref{51:11} So 1QIsa\textsuperscript{a} MT\textsuperscript{ms} (sing.); cf Isa 51:3; MT (pl.)} \\
\poeml \v{12}``I---yes, I---am the one who comforts you. \\
\poemll    Who are you, that you are so afraid of humans who will die, \\
\poemlll       descendants of mere\fnote{\fbackref{51:12} 1QIsa\textsuperscript{a} MT LXX lack \fbib{mere}} men, who have been made\fnote{\fbackref{51:12} So 1QIsa\textsuperscript{a}; MT reads \fbib{are made}; LXX reads \fbib{will be dried up}} like grass? \\
\poeml \v{13}As a result, you have forgotten the \divine{Lord} who made you, \\
\poemll    who stretched out the heavens \\
\poeml and laid the earth's foundations, \\
\poemll    and you live in constant fear every day \\
\poeml because of the oppressor's fury, \\
\poemll    since he's ready to destroy. \\
\poemlll       Now where is the\fnote{\fbackref{51:13} So 1QIsa\textsuperscript{a}; 1QIsa\textsuperscript{a} omitted \fbib{oppressor's{\ldots}the} then inserted the missing lines at the top of column 43} oppressor's fury? \\
\poeml \v{14}Distress\fnote{\fbackref{51:14} So 1QIsa\textsuperscript{a}; MT reads \fbib{The cowering one}; LXX lacks \fbib{Distress}} will quickly be set free. \\
\poemll    He won't die in the Pit,\fnote{\fbackref{51:14} I.e. the realm of punishment in the afterlife} \\
\poemlll       nor will he lack food.''
\passage{A Promise of Restoration}
\poeml \v{15}``For I am the \divine{Lord} your God, \\
\poemll    who churns up the sea, so that its waves roar, \\
\poemlll       `The \divine{Lord} of the Heavenly Armies is his name.' \\
\poeml \v{16}I have put my words in your mouth \\
\poemll    and have covered you with the shadow of my hand, \\
\poeml so that I could plant the heavens \\
\poemll    and lay the earth's foundations, \\
\poemlll       to say to Zion, `You are my people.' \\
\poeml \v{17}``Awake, Awake! \\
\poemll    Stand up, Jerusalem, \\
\poeml you who have drunk from the \divine{Lord}'s hand \\
\poemll    from the cup that is\fnote{\fbackref{51:17} Lit. \fbib{hand, the cup of}} his anger. \\
\poeml You have drunk to the dregs \\
\poemll    the cup that makes you stagger,\fnote{\fbackref{51:17} Lit. \fbib{the cup of staggering}} \\
\poemlll       and have drained it. \\
\poeml \v{18}There is no one to guide you\fnote{\fbackref{51:18} So 1QIsa\textsuperscript{a}; MT reads \fbib{no one to guide her}; LXX reads \fbib{no one who comforted you}} \\
\poemll    out of all the children she bore, \\
\poeml no one to take her by the hand \\
\poemll    out of all the children she brought up. \\
\poeml \v{19}``These twin things have come upon you \\
\poemll    (who can feel sorry for you?): \\
\poeml ruin and destruction, \\
\poemll    famine and the sword--- \\
\poemlll       who can console you? \\
\poeml \v{20}Your children have fainted. \\
\poemll    They lie at the head of every street, \\
\poemlll       like antelope caught in a trap, \\
\poeml filled with the anger of the \divine{Lord} \\
\poemll    and the rebuke of your God. \\
\poeml \v{21}Now listen to this, you afflicted one, \\
\poemll    made drunk, but not with wine: \\
\poeml \v{22}This is what your \divine{Lord}, the \divine{Lord},\fnote{\fbackref{51:21} 1QIsa\textsuperscript{a} MT; MT\textsuperscript{ms} lacks \fbib{}\divine{Lord}} says, \\
\poemll    your\fnote{\fbackref{51:22} So 1QIsa\textsuperscript{a}; MT reads \fbib{and your}} God, who defends his people's cause: \\
\poeml ``See, I have taken from your hand the cup that made you stagger.\fnote{\fbackref{51:22} Lit. \fbib{the cup of staggering}} \\
\poemll    And you will never again drink to the dregs the cup that is my anger. \\
\poeml \v{23}But I will put it into the hands of those who tormented and oppressed you,\fnote{\fbackref{51:23} So 1QIsa\textsuperscript{a}; MT reads \fbib{tormented you}; LXX reads \fbib{harmed you and humiliated you}} \\
\poemll    those who said to you, \\
\poeml `Lie down,\fnote{\fbackref{51:23} \fbib{Lit.bow down}} so we can step over you,' \\
\poemll    so that you had to make your back like the ground, \\
\poemlll       and like a street for them to walk over.''
\end{poetry}
\labelchapt{52}
\passage{The Redemption of Zion}

\chapt{52}
\v{1}Awake, awake!

\begin{poetry}
\poemll    Clothe yourself with strength,\fnote{\fbackref{52:1} So 1QIsa\textsuperscript{a}; MT LXX read \fbib{with your strength}} O Zion! \\
\poeml Put on your beautiful garments, \\
\poemll    O Jerusalem, the holy city, \\
\poemlll       for the uncircumcised and the unclean won't enter you.\fnote{\fbackref{52:1} So 1QIsa\textsuperscript{a}; MT LXX read \fbib{you any more}} \\
\poeml \v{2}Shake yourself from the dust and\fnote{\fbackref{52:2} So 1QIsa\textsuperscript{a} LXX; 4QIsa\textsuperscript{b} MT lack \fbib{and}} arise, \\
\poemll    and\fnote{\fbackref{52:2} So 1QIsa\textsuperscript{a}; MT LXX lack \fbib{and}} sit on your throne, O Jerusalem! \\
\poeml Loosen the bonds from your neck, \\
\poemll    O captive daughter of Zion.
\end{poetry}

\v{3}For this is what the \divine{Lord} says: ``You were sold for nothing, and you'll be redeemed without money.''

\v{4}For this is what the \divine{Lord}\fnote{\fbackref{52:4} So 1QIsa\textsuperscript{a} LXX; MT reads \fbib{the \divine{Lord} \divine{God}}} says: ``My people went down long ago into Egypt to live\fnote{\fbackref{52:4} Or \fbib{sojourn}} there; the Assyrian, too, has oppressed them without cause.

\v{5}``Now therefore, what\fnote{\fbackref{52:5} So 1QIsa\textsuperscript{a} LXX; MT reads \fbib{who}} am I doing here,'' asks\fnote{\fbackref{52:5} Lit. \fbib{declares}} the \divine{Lord}, ``seeing that my people are taken away without cause? Those who rule over them are deluded,''\fnote{\fbackref{52:5} So 1QIsa\textsuperscript{a}; MT LXX read \fbib{them wail}; or \fbib{them taunt}} says the \divine{Lord}, ``and continuously, all the day long, my name is blasphemed. \v{6}Therefore my people will know my name; in that day\fnote{\fbackref{52:6} So 1QIsa\textsuperscript{a} LXX. MT reads \fbib{therefore in that day}} they'll know that it is I who speaks, `Here I am!'

\begin{poetry}
\poeml \v{7}``How beautiful\fnote{\fbackref{52:7} Lit. \fbib{they are beautiful}; so 1QIsa\textsuperscript{a} MTT LXX\textsuperscript{mss}; 4QIsa\textsuperscript{b} reads \fbib{It is beautiful}} on the mountains \\
\poemll    are the feet of the one who brings news of peace,\fnote{\fbackref{52:7} So 1QIsa\textsuperscript{a}; MT LXX read \fbib{who brings good news}} \\
\poeml who announces good things, \\
\poemll    who announces salvation,\fnote{\fbackref{52:7} So 1QIsa\textsuperscript{a}; MT reads \fbib{who announces peace, who brings news of good things, who announces salvation}; cf. LXX} \\
\poemlll       who says to Zion, `Your God reigns!' \\
\poeml \v{8}Listen! Your watchmen lift up their voices,\fnote{\fbackref{52:8} Lit. \fbib{their voice}; so 1QIsa\textsuperscript{a}; MT reads \fbib{the voice}} \\
\poemll    together they sing for joy; \\
\poeml for they will see in plain sight \\
\poemll    the return of the \divine{Lord} to Zion with compassion.\fnote{\fbackref{52:8} So 1QIsa\textsuperscript{a} LXX; MT lacks \fbib{with compassion}} \\
\poeml \v{9}``Break forth together into singing,\fnote{\fbackref{52:9} Lit. \fbib{sing for joy} (sing.); so 1QIsa\textsuperscript{a}; MT reads \fbib{sing for joy} (pl.)} \\
\poemll    you ruins of Jerusalem; \\
\poeml for the \divine{Lord} has comforted his people, \\
\poemll    and\fnote{\fbackref{52:9} So 1QIsa\textsuperscript{a} LXX; MT lacks \fbib{and}} he has redeemed Jerusalem. \\
\poeml \v{10}The \divine{Lord} has bared his holy arm\fnote{\fbackref{52:10} I.e. \fbib{the Messiah}} \\
\poemll    in the eyes of all the nations; \\
\poeml and all the ends of the earth will see \\
\poemll    the salvation of our God. \\
\poeml \v{11}``Depart! Depart! Go out from there; \\
\poemll    touch no unclean thing; \\
\poeml go out from the midst of her; \\
\poemll    purify yourselves, \\
\poemlll       you who carry the vessels of the \divine{Lord}. \\
\poeml \v{12}For you won't go out in haste, \\
\poemll    nor will you go in flight; \\
\poeml for the \divine{Lord} will go before you; \\
\poemll    and the God of Israel will be your rear guard. \\
\poemlll       He is called the God of all the earth.''\fnote{\fbackref{52:12} So 1QIsa\textsuperscript{a}; MT LXX lack \fbib{this line}}
\passage{The Suffering Servant}
\poeml \v{13}``Look! My servant will prosper, \\
\poemll    and\fnote{\fbackref{52:13} So 1QIsa\textsuperscript{a}; 1QIsa\textsuperscript{b} 4QIsa\textsuperscript{c} MT lack \fbib{and}; LXX lacks \fbib{and he will be exalted}} he will be exalted and lifted up, \\
\poemlll       and will be very high. \\
\poeml \v{14}Just as many were astonished at you\fnote{\fbackref{52:14} So 1QIsa\textsuperscript{a} MT LXX; MT\textsuperscript{mss} Syriac read \fbib{at him}}--- \\
\poemll    so was he marred in\fnote{\fbackref{52:14} Or \fbib{was my marring}; so 1QIsa\textsuperscript{a}; MT reads \fbib{was the marring of}} his appearance, more than any human, \\
\poeml and his form beyond that of human semblance\fnote{\fbackref{52:14} Lit. \fbib{of the descendants of humans}; so 1QIsa\textsuperscript{a}; LXX reads \fbib{of the humans}; MT reads \fbib{of the human}}--- \\
\poeml \v{15}so will he startle\fnote{\fbackref{52:15} Or \fbib{sprinkle}} many nations. \\
\poeml Kings will shut their mouths at him; \\
\poemll    for what had not been told them they will see, \\
\poemlll       and what they had not heard they will understand.
\end{poetry}
\labelchapt{53}

\chapt{53}
\v{1}``Who\fnote{\fbackref{53:1} So 1QIsa\textsuperscript{a} 1QIsa\textsuperscript{b} MT; LXX reads \fbib{Lord, who}} has believed our message,

\begin{poetry}
\poemll    and\fnote{\fbackref{53:1} So 1QIsa\textsuperscript{a} 1QIsa\textsuperscript{b}; MT lacks \fbib{and}} to\fnote{\fbackref{53:1} So 1QIsa\textsuperscript{a} 1QIsa\textsuperscript{b}; MT reads \fbib{upon}} whom has the arm\fnote{\fbackref{53:1} I.e. \fbib{the Messiah}} of the \divine{Lord} been revealed? \\
\poeml \v{2}For he grew up before him like a tender plant, \\
\poemll    and like a root out of a dry ground; \\
\poeml he had no form and he had\fnote{\fbackref{53:2} So 1QIsa\textsuperscript{a}; 1QIsa\textsuperscript{b} MT LXX lack \fbib{he had}} no majesty that we should look at him,\fnote{\fbackref{53:2} So 1QIsa\textsuperscript{a} 1QIsa\textsuperscript{b} MT LXX; 1QIsa\textsuperscript{a} may read \fbib{at ourselves}} \\
\poemll    and there is no attractiveness that we should desire him.\fnote{\fbackref{53:2} So 1QIsa\textsuperscript{a} MT LXX; 1QIsa\textsuperscript{a} may read \fbib{desire ourselves}} \\
\poeml \v{3}``He was despised and rejected by others, \\
\poemll    and\fnote{\fbackref{53:3} So 1QIsa\textsuperscript{a}; 1QIsa\textsuperscript{b} MT LXX lack \fbib{and}} a man of sorrows, \\
\poemlll       intimately familiar with\fnote{\fbackref{53:3} So 1QIsa\textsuperscript{a} LXX; MT reads \fbib{and acquainted with}; 1QIsa\textsuperscript{b} reads \fbib{and knowing}} suffering; \\
\poeml and like one from whom people hide their faces; \\
\poemll    and\fnote{\fbackref{53:3} So 1QIsa\textsuperscript{a} 1QIsa\textsuperscript{b}; MT LXX lacks \fbib{and}} we despised him\fnote{\fbackref{53:3} So 1QIsa\textsuperscript{a}; MT LXX read \fbib{he was despised}} \\
\poemlll       and did not value him. \\
\poeml \v{4}``Surely he has borne our sufferings \\
\poemll    and carried our sorrows; \\
\poeml yet we considered him stricken, \\
\poemll    and\fnote{\fbackref{53:4} So 1QIsa\textsuperscript{a} LXX; MT lacks \fbib{and}} struck down by God, \\
\poemlll       and afflicted. \\
\poeml \v{5}But he was wounded for our transgressions, \\
\poemll    and\fnote{\fbackref{53:5} So 1QIsa\textsuperscript{a} LXX; MT lacks \fbib{and}} he was crushed for our iniquities, \\
\poeml and\fnote{\fbackref{53:5} So 1QIsa\textsuperscript{a} 1QIsa\textsuperscript{b}; MT LXX lack \fbib{and}} the punishment that made us whole was upon him, \\
\poemll    and by his bruises we are healed. \\
\poeml \v{6}All we like sheep have gone astray, \\
\poemll    we have turned, each of us, to his own way; \\
\poeml and the \divine{Lord} has laid on him \\
\poemll    the iniquity of us all. \\
\poeml \v{7}He was oppressed and he was afflicted, \\
\poemll    yet he didn't open his mouth; \\
\poeml like a lamb that is led to the slaughter, \\
\poemll    as\fnote{\fbackref{53:7} So 1QIsa\textsuperscript{a}; MT LXX read \fbib{and as}} a sheep that before its shearers is silent, \\
\poemlll       so he did not open\fnote{\fbackref{53:7} So 1QIsa\textsuperscript{a}; MT reads \fbib{does not open}} his mouth. \\
\poeml \v{8}``From detention and\fnote{\fbackref{53:8} So 1QIsa\textsuperscript{a} MT; 1QIsa\textsuperscript{b} lacks \fbib{and}} judgment he was taken away\fnote{\fbackref{53:8} So 1QIsa\textsuperscript{a} MT LXX; 1QIsa\textsuperscript{b} reads \fbib{judgment they took him away}}--- \\
\poemll    and who can even think about his descendants?\fnote{\fbackref{53:8} Or \fbib{future}} \\
\poeml For he was cut off from the land of the living, \\
\poemll    he was stricken\fnote{\fbackref{53:8} So 1QIsa\textsuperscript{a}; MT reads \fbib{living, an affliction}} for the transgression of my people. \\
\poeml \v{9}Then they made\fnote{\fbackref{53:9} So 1QIsa\textsuperscript{a}; 4QIsa\textsuperscript{d} MT read \fbib{he made}; LXX reads \fbib{I will give}} his grave with the wicked, \\
\poemll    and with rich people\fnote{\fbackref{53:9} So 1QIsa\textsuperscript{a}; 1QIsa\textsuperscript{a} corrector MT read \fbib{with a rich man}} in his death,\fnote{\fbackref{53:9} So 1QIsa\textsuperscript{a} LXX; MT reads \fbib{deaths}} \\
\poeml although he had committed no violence, \\
\poemll    nor was there any deceit in his mouth.''
\passage{The Exaltation of the Servant}
\poeml \v{10}``Yet the \divine{Lord} was willing to crush him, \\
\poemll    and he made him suffer.\fnote{\fbackref{53:10} So 1QIsa\textsuperscript{a}; 4QIsa\textsuperscript{d} MT read \fbib{he made him suffer}; LXX reads \fbib{with a blow}} \\
\poeml Although you make his soul an offering for sin, \\
\poemll    he\fnote{\fbackref{53:10} So MT LXX; 1QIsa\textsuperscript{a} reads \fbib{And he}} will see his offspring, \\
\poeml and\fnote{\fbackref{53:10} So 1QIsa\textsuperscript{a} 4QIsa\textsuperscript{d}; 1QIsa\textsuperscript{b} MT lack \fbib{and}} he will prolong his days, \\
\poemll    and the will of the \divine{Lord} will triumph in his hand. \\
\poeml \v{11}Out of the suffering of his soul he will see light\fnote{\fbackref{53:11} So 1QIsa\textsuperscript{a} 1QIsa\textsuperscript{b} 4QIsa\textsuperscript{d} LXX; MT reads \fbib{He will see of the suffering of his soul}} \\
\poemll    and\fnote{\fbackref{53:11} So 1QIsa\textsuperscript{a} 4QIsa\textsuperscript{d}; MT lacks \fbib{and}} find satisfaction. \\
\poeml And\fnote{\fbackref{53:11} So 1QIsa\textsuperscript{a}; 4QIsa\textsuperscript{d} MT lacks \fbib{And}} through his knowledge his servant,\fnote{\fbackref{53:11} So 1QIsa\textsuperscript{a}; 4QIsa\textsuperscript{d} MT read \fbib{my servant}} the righteous one, \\
\poemll    will make many righteous, \\
\poemlll       and he will bear their iniquities. \\
\poeml \v{12}Therefore I will allot him a portion with the great,\fnote{\fbackref{53:12} I.e. an allusion to the resurrection} \\
\poemll    and he will divide the spoils with the strong; \\
\poeml because he poured out his life to death, \\
\poemll    and was numbered with the transgressors; \\
\poeml yet he carried the sins\fnote{\fbackref{53:12} So 1QIsa\textsuperscript{a} 1QIsa\textsuperscript{b} 4QIsa\textsuperscript{d} LXX; MT reads \fbib{the sin}} of many, \\
\poemll    and made intercession for their transgressions.''\fnote{\fbackref{53:12} So 1QIsa\textsuperscript{a} 1QIsa\textsuperscript{b} 4QIsa\textsuperscript{d} LXX; MT reads \fbib{for the transgressors}}
\end{poetry}
\labelchapt{54}
\passage{The Coming Glory of Israel}

\begin{poetry}
\poeml \chapt{54}
\v{1}``Sing, you barren woman, \\
\poemll    even the\fnote{\fbackref{54:1} Or \fbib{and}; so 1QIsa\textsuperscript{a} 4QIsa\textsuperscript{d}; 1QIsa\textsuperscript{b} MT LXX lack \fbib{even}} one who never bore a child! \\
\poeml Burst into song and shout for joy, \\
\poemll    even\fnote{\fbackref{54:1} Or \fbib{and}; so 1QIsa\textsuperscript{a}; 4QIsa\textsuperscript{d} MT LXX lack \fbib{even}} you who were never in labor! \\
\poeml For the children of the desolate woman will be more \\
\poemll    than the children of her that is married,'' \\
\poemlll       says the \divine{Lord}. \\
\poeml \v{2}``Enlarge the location\fnote{\fbackref{54:2} Or \fbib{place}} of your tent, \\
\poemll    let the curtains of your dwellings be stretched wide, \\
\poemlll       and\fnote{\fbackref{54:2} So 1QIsa\textsuperscript{a}; MT LXX lack \fbib{and}} don't hold back. \\
\poeml Lengthen your cords; \\
\poemlll       strengthen your stakes. \\
\poeml \v{3}For you will spread out to the right hand and to the left, \\
\poemll    and your descendants\fnote{\fbackref{54:3} Lit. \fbib{seed}} will possess\fnote{\fbackref{54:3} 1QIsa\textsuperscript{a} is pl.; MT is sing.} the nations \\
\poemlll       and will populate the deserted towns. \\
\poeml \v{4}``Don't be afraid, \\
\poemll    because you won't be ashamed; \\
\poeml don't fear shame, \\
\poemll    for you won't be humiliated--- \\
\poeml because you will forget the disgrace of your youth, \\
\poemll    and the reproach of your widowhood you will remember no more. \\
\poeml \v{5}For your Maker is your husband; \\
\poemll    the \divine{Lord} of the Heavenly Armies is his name, \\
\poeml and the Holy One of Israel is your Redeemer; \\
\poemll    he is called the God of the whole earth. \\
\poeml \v{6}For the \divine{Lord} has called you back \\
\poemll    like a wife deserted and grieved in spirit, \\
\poeml like the wife of a man's youth when she is cast off,'' \\
\poemll    says the \divine{Lord} your God.\fnote{\fbackref{54:6} So 1QIsa\textsuperscript{a}; MT LXX read \fbib{says your God}} \\
\poeml \v{7}``For a brief moment I abandoned you; \\
\poemll    but I'll gather you with great compassion. \\
\poeml \v{8}I hid my face from you for a moment in a surge of anger, \\
\poemll    but I will have compassion on you with my\fnote{\fbackref{54:8} So 1QIsa\textsuperscript{a} 4QIsa\textsuperscript{c}; MT LXX lack \fbib{my}} everlasting gracious love,'' \\
\poemlll       says the \divine{Lord} your Redeemer.
\passage{God's Reconciliation with Israel}
\poeml \v{9}``For this is like the waters of Noah to me, \\
\poemll    when I swore that the waters of Noah \\
\poemlll       would never again spread over the earth; \\
\poeml so have I sworn that I won't be angry with you again\fnote{\fbackref{54:9} So 1QIsa\textsuperscript{a} LXX; MT lacks \fbib{again}} \\
\poemll    and that I\fnote{\fbackref{54:9} 1QIsa\textsuperscript{a} LXX MT lack \fbib{that I}} won't rebuke you. \\
\poeml \v{10}For the mountains may collapse \\
\poemll    and the hills may reel, \\
\poeml but my gracious love will not depart from you, \\
\poemll    neither will my covenant of peace totter,'' \\
\poemlll       says the \divine{Lord}, who has compassion on you. \\
\poeml \v{11}``O afflicted one,\fnote{\fbackref{54:11} I.e. the city of Jerusalem} passed back and forth,\fnote{\fbackref{54:11} So 1QIsa\textsuperscript{a}; 4QIsa\textsuperscript{d} MT read \fbib{storm-tossed}; LXX reads \fbib{unsteady}} and not comforted, \\
\poemll    Look! I am about to set your stones in antimony, \\
\poemlll       and lay your foundations with sapphires. \\
\poeml \v{12}And I'll make your battlements of rubies, \\
\poemll    and your gates of jewels, \\
\poemlll       and all your walls of precious stones. \\
\poeml \v{13}Then all your children will be taught by the \divine{Lord}, \\
\poemll    and great will be your children's prosperity. \\
\poeml \v{14}``In righteousness you'll be established; \\
\poemll    you will be far from tyranny, \\
\poemlll       for you won't be afraid, \\
\poemll    and from terror, \\
\poemlll       for it won't come near you. \\
\poeml \v{15}Watch! If anyone does attack you, \\
\poemll    it will not be from me; \\
\poeml whoever may attack\fnote{\fbackref{54:15} So 1QIsa\textsuperscript{a}; MT reads \fbib{whoever attacks}} you will fall\fnote{\fbackref{54:15} So 1QIsa\textsuperscript{a} 4QIsa\textsuperscript{c}; MT reads \fbib{he will fall}; LXX reads \fbib{they will flee}} because of you. \\
\poeml \v{16}Look! It is I who have created the blacksmith \\
\poemll    who fans coals in the fire, \\
\poemlll       and produces a weapon for his purpose. \\
\poeml It\fnote{\fbackref{54:16} So 1QIsa\textsuperscript{a}; MT reads \fbib{And it}; cf. LXX} is I who have created the ravager to wreak havoc; \\
\poeml \v{17}no weapon that is forged against you will be effective.\fnote{\fbackref{54:17} So 1QIsa\textsuperscript{a}; 4QIsa\textsuperscript{c} MT LXX read \fbib{effective, and you will refute every tongue that rises against you in judgment}} \\
\poeml This is the heritage of the \divine{Lord}'s servants, \\
\poemll    and their righteousness from me,'' \\
\poemlll       says the \divine{Lord}.
\end{poetry}
\labelchapt{55}
\passage{An Invitation to Life}

\chapt{55}
\v{1}``Come, everyone who is thirsty,

\begin{poetry}
\poemll    come to the waters! \\
\poeml Also, you that have no money, come, \\
\poemll    buy, and eat! \\
\poeml Come! Buy\fnote{\fbackref{55:1} So MT LXX; 1QIsa\textsuperscript{a} skips from the first \fbib{come, buy} to the second \fbib{come, buy,} omitting the words in between} wine and milk \\
\poemll    without money and without price. \\
\poeml \v{2}Why spend your money on what is not bread, \\
\poemll    and your labor on what does not satisfy?\fnote{\fbackref{55:2} Lit. \fbib{what is not satisfaction}; so 1QIsa\textsuperscript{a}; MT reads \fbib{what is not for satisfaction}} \\
\poeml Listen carefully to me, \\
\poemll    and eat what is good, \\
\poemlll       and let your soul delight itself in rich food. \\
\poeml \v{3}Pay attention\fnote{\fbackref{55:3} Lit. \fbib{Turn your ear}} to me, \\
\poemll    come to me; \\
\poemlll       and\fnote{\fbackref{55:3} So 1QIsa\textsuperscript{a}; MT LXX lacks \fbib{and}} listen, so that you may live; \\
\poeml then I'll make\fnote{\fbackref{55:3} So 1QIsa\textsuperscript{a}; 4QIsa\textsuperscript{c} MT read \fbib{then let me make}} an everlasting covenant with you, \\
\poemll    as promised by\fnote{\fbackref{55:3} 1QIsa\textsuperscript{a} 4QIsa\textsuperscript{c} MT LXX lack \fbib{as promised by}} my faithful, sure love for David. \\
\poeml \v{4}``Look! I have made him a witness to the peoples, \\
\poemll    a leader and commander of the peoples. \\
\poeml \v{5}``Look! You will call a nation that you do not know, \\
\poemll    and a nation that does\fnote{\fbackref{55:5} So 1QIsa\textsuperscript{a}; MT reads \fbib{a nation} (pl.) \fbib{that do}; LXX reads \fbib{nations that do}} not know you will run\fnote{\fbackref{55:5} So 1QIsa\textsuperscript{a} (sing.); MT LXX (pl.)} to you, \\
\poeml because of the \divine{Lord} your God, even\fnote{\fbackref{55:5} So 1QIsa\textsuperscript{a}; MT 1QIsa\textsuperscript{a} corrector read \fbib{and because of}} the Holy One of Israel, \\
\poemll    for he has glorified you.''
\passage{Steps to Reconciliation}
\poeml \v{6}``Seek the \divine{Lord} while he\fnote{\fbackref{55:6} So 1QIsa\textsuperscript{a} 1QIsa\textsuperscript{b} MT LXX; implied in 4QIsa\textsuperscript{c}} may be found, \\
\poemll    call upon him while he is near. \\
\poeml \v{7}Let the wicked forsake his way, \\
\poemll    and the unrighteous person his thoughts. \\
\poeml Let him return to the \divine{Lord}, \\
\poemll    So he'll have mercy upon him, \\
\poeml and to our God, \\
\poemll    for he'll pardon abundantly. \\
\poeml \v{8}For my thoughts are not your thoughts, \\
\poemll    nor are your ways my ways,'' declares the \divine{Lord}. \\
\poeml \v{9}``For just as\fnote{\fbackref{55:9} So 1QIsa\textsuperscript{a} LXX; MT lacks \fbib{just as}} the heavens are higher than the earth, \\
\poemll    so are my ways higher than your ways, \\
\poemlll       and my thoughts than your thoughts. \\
\poeml \v{10}``For just as the rain and snow come down from heaven, \\
\poemll    and do not return there without watering the earth, \\
\poeml making it bring forth and sprout, \\
\poemll    yielding seed for the sower and bread for eating,\fnote{\fbackref{55:10} So 1QIsa\textsuperscript{a}; MT reads \fbib{for the eater}} \\
\poeml \v{11}so will my message be that goes out of my mouth--- \\
\poemll    it won't return to me empty. \\
\poeml Instead, it will accomplish what I desire, \\
\poemll    and achieve the purpose for which I sent it. \\
\poeml \v{12}``For you will go out in joy, \\
\poemll    and come back\fnote{\fbackref{55:12} So 1QIsa\textsuperscript{a}. MT reads \fbib{and be led back} cf. LXX} with peace; \\
\poeml the mountains and the hills \\
\poemll    will burst into song before you, \\
\poemlll       and all the trees in\fnote{\fbackref{55:12} Or \fbib{of}} the fields\fnote{\fbackref{55:12} Or \fbib{orchards}} will clap their hands. \\
\poeml \v{13}Instead of thornbushes, pine trees will grow, \\
\poemll    and\fnote{\fbackref{55:13} So 1QIsa\textsuperscript{a} MT\textsuperscript{mss}; MT LXX lacks \fbib{and}} instead of briers, myrtles will grow; \\
\poeml and they\fnote{\fbackref{55:13} So 1QIsa\textsuperscript{a}; MT LXX read \fbib{it}} will be a sign for the \divine{Lord}, \\
\poemll    and an everlasting name\fnote{\fbackref{55:13} So 1QIsa\textsuperscript{a}; MT LXX read \fbib{a name, an everlasting sign}} that will not be cut off.''
\end{poetry}
\labelchapt{56}
\passage{The Covenant Extended to the Righteous}

\chapt{56}
\v{1}For\fnote{\fbackref{56:1} So 1QIsa\textsuperscript{a}; MT LXX lack \fbib{For}} this is what the \divine{Lord} says:

\begin{poetry}
\poemll    ``Maintain justice, and do what is right, \\
\poeml for soon my salvation will come, \\
\poemll    and soon my deliverance will be revealed. \\
\poeml \v{2}Blessed is the one who does this, \\
\poemll    and the person that holds it fast, \\
\poeml who observes the Sabbath without profaning it, \\
\poemll    and restrains his hands\fnote{\fbackref{56:2} So 1QIsa\textsuperscript{a} LXX 1QIsa\textsuperscript{b} MT read \fbib{hand}} from practicing any evil. \\
\poeml \v{3}``Let\fnote{\fbackref{56:3} So 1QIsa\textsuperscript{a} LXX; 1QIsa\textsuperscript{b} MT read \fbib{And let}} no foreigner who has joined himself to the \divine{Lord} say: \\
\poemll    `The \divine{Lord} will surely exclude me from his people.' \\
\poeml Furthermore, let no eunuch say, \\
\poemll    `Look!\fnote{\fbackref{56:3} So 1QIsa\textsuperscript{a} 1QIsa\textsuperscript{b} MT; LXX lacks \fbib{Look}!} I am just a dry tree.'\,'' \\
\poeml \v{4}For this is what the \divine{Lord} says: \\
\poeml ``To the eunuchs who observe my Sabbaths, \\
\poemll    who choose the things that please me, \\
\poemlll       and who hold fast my covenant--- \\
\poeml \v{5}to them I will give in my house\fnote{\fbackref{56:5} I.e. God's Temple} and within my walls \\
\poemll    a monument and a name \\
\poemlll       better than sons and daughters. \\
\poeml I will give them\fnote{\fbackref{56:5} So 1QIsa\textsuperscript{a} LXX; 1QIsa\textsuperscript{b} MT read \fbib{him}} an everlasting name \\
\poemll    that will not be cut off.\fnote{\fbackref{56:5} The Heb. verb is a word play on the Heb. word \fbib{eunuch}} \\
\poeml \v{6}``Also, the foreigners who join themselves to\fnote{\fbackref{56:6} So 1QIsa\textsuperscript{a} (cf. v 3); 1QIsa\textsuperscript{b} MT read \fbib{upon}} the \divine{Lord}, \\
\poemll    to minister to him, \\
\poemlll       to love the name of the \divine{Lord},\fnote{\fbackref{56:6} So 1QIsa\textsuperscript{b} MT LXX; 1QIsa\textsuperscript{a} lacks this line} \\
\poeml to be his servants, \\
\poemll    and to bless the \divine{Lord}'s name, \\
\poeml observing\fnote{\fbackref{56:6} So 1QIsa\textsuperscript{a}; 1QIsa\textsuperscript{b} MT LXX read \fbib{all who observe}} the Sabbath without profaning it, \\
\poemll    and who hold fast my covenant--- \\
\poeml \v{7}these I will bring to my holy mountain, \\
\poemll    and make them joyful in my house of prayer. \\
\poeml Their burnt offerings and their sacrifices \\
\poemll    will rise up to be accepted\fnote{\fbackref{56:7} 1QIsa\textsuperscript{a}; 1QIsa\textsuperscript{b} 4QIsa\textsuperscript{i} MT LXX read \fbib{will be accepted}} on my altar; \\
\poeml for my house will be called a house of prayer \\
\poemll    for everyone.''\fnote{\fbackref{56:7} Lit. \fbib{for all peoples}}
\passage{A Rebuke to Israel's Guardians}
\poeml \v{8}This is what the Lord \divine{God} says, \\
\poemll    the one who gathers the outcasts of Israel: \\
\poeml ``I'll gather still others to them \\
\poemll    besides those already gathered.\fnote{\fbackref{56:8} Lit. \fbib{besides their gathered ones}} \\
\poeml \v{9}``All you wild animals,\fnote{\fbackref{56:9} So 1QIsa\textsuperscript{a} LXX; 1QIsa\textsuperscript{b} MT read \fbib{Every wild animal}} come and devour--- \\
\poemll    even\fnote{\fbackref{56:9} So 1QIsa\textsuperscript{a}; 1QIsa\textsuperscript{b} MT LXX lack \fbib{even}} all of you wild animals.\fnote{\fbackref{56:9} So 1QIsa\textsuperscript{a} LXX; 1QIsa\textsuperscript{b} MT read \fbib{every wild animal}} \\
\poeml \v{10}His\fnote{\fbackref{56:10} I.e. Israel's; so 1QIsa\textsuperscript{a} MT\textsuperscript{q}} watchmen\fnote{\fbackref{56:10} MT reads \fbib{His watchman}; LXX reads \fbib{Look! They all}} are blind; \\
\poemll    they are all without knowledge. \\
\poeml They are all dumb dogs--- \\
\poemll    they cannot bark. \\
\poeml They keep on dreaming and lying around, \\
\poemll    and they're lovers of sleep!\fnote{\fbackref{56:10} So 1QIsa\textsuperscript{a} LXX; MT reads \fbib{sleeping}} \\
\poeml \v{11}Meanwhile,\fnote{\fbackref{56:11} Lit. \fbib{And}} the dogs have a mighty appetite--- \\
\poemll    they can never get enough. \\
\poeml And as for them, they are the shepherds\fnote{\fbackref{56:11} So 1QIsa\textsuperscript{a}; MT reads \fbib{shepherds}; LXX reads \fbib{evil}} who lack understanding; \\
\poemll    they have all turned to their own way, \\
\poeml each one to his gain, \\
\poemll    each and every one. \\
\poeml \v{12}```Come!' they say, `let's\fnote{\fbackref{56:12} So 1QIsa\textsuperscript{a} MT\textsuperscript{ms}; 1QIsa\textsuperscript{b} reads \fbib{I will}; MT reads \fbib{let me}} have some wine, \\
\poemll    and let's fill ourselves with strong drink! \\
\poeml Then,\fnote{\fbackref{56:12} Or \fbib{And}} tomorrow will be like today,\fnote{\fbackref{56:12} 1QIsa\textsuperscript{a} reads \fbib{this the day}; 1QIsa\textsuperscript{b} MT read \fbib{this day}} \\
\poemll    or even much better!'\,''
\end{poetry}
\labelchapt{57}
\passage{Israel's Idolatry}

\chapt{57}
\v{1}``Also\fnote{\fbackref{57:1} So 1QIsa\textsuperscript{a}; 1QIsa\textsuperscript{b} MT lacks \fbib{Also}; cf. LXX} the righteous are perishing,\fnote{\fbackref{57:1} So 1QIsa\textsuperscript{a}; 1QIsa\textsuperscript{b} MT LXX read \fbib{righteous person has perished}}

\begin{poetry}
\poemll    but no one takes it to heart; \\
\poeml devout people\fnote{\fbackref{57:1} Lit. \fbib{people of the mercy}; so 1QIsa\textsuperscript{a} MT reads \fbib{people of mercy}; LXX reads \fbib{just men}} are taken away, \\
\poemll    while no one understands \\
\poemlll       that the righteous person is taken away from calamity. \\
\poeml \v{2}Then\fnote{\fbackref{57:2} Or \fbib{and}; so 1QIsa\textsuperscript{a}; MT LXX lack \fbib{Then}} he enters into peace, \\
\poemll    and\fnote{\fbackref{57:2} So 1QIsa\textsuperscript{a}; MT lacks \fbib{and}} they'll rest on his\fnote{\fbackref{57:2} So 1QIsa\textsuperscript{a}; 1QIsa\textsuperscript{b} MT read \fbib{their}} couches, \\
\poemlll       each one living righteously.\fnote{\fbackref{57:2} Lit. \fbib{one walking in his uprightness}; so 1QIsa\textsuperscript{a} 1QIsa\textsuperscript{b}; MT reads \fbib{her uprightness}} \\
\poeml \v{3}``But as for you, come here, \\
\poemll    you children of a sorceress, \\
\poemlll       you offspring of adulterers and prostitutes!\fnote{\fbackref{57:3} So LXX (cf. Syriac); 1QIsa\textsuperscript{a} MT read \fbib{she has practiced prostitution}} \\
\poeml \v{4}Whom are you mocking? \\
\poemll    And\fnote{\fbackref{57:4} So 1QIsa\textsuperscript{a} LXX; 1QIsa\textsuperscript{b} MT lack \fbib{And}} against whom do you make a wide mouth \\
\poemlll       and stick out your tongue? \\
\poeml Are you not children of transgression, \\
\poemll    the offspring of lies, \\
\poeml \v{5}you who burn with lust among the oaks, \\
\poemll    under every spreading tree, \\
\poeml who slaughter your children in the ravines, \\
\poemll    under the clefts of the rocks? \\
\poeml \v{6}``Among the smooth stones\fnote{\fbackref{57:6} I.e. among the idols} of the ravines is your portion--- \\
\poemll    there they are as\fnote{\fbackref{57:6} So 1QIsa\textsuperscript{a}; 4QIsa\textsuperscript{i} MT reads \fbib{they---yes, they!---are}; LXX reads \fbib{there this is}} your lot. \\
\poeml To them you have poured out drink offerings; \\
\poemll    you have brought grain offerings. \\
\poemlll       Should I be lenient over such things? \\
\poeml \v{7}``You have made your bed \\
\poemll    on a high and lofty mountain, \\
\poemlll       and you went up to offer sacrifice there. \\
\poeml \v{8}Behind the doors and the doorposts \\
\poemll    you have set up your pagan sign.'' \\
\poeml For in deserting me you have uncovered your bed--- \\
\poemll    you have climbed up into it \\
\poemlll       and have opened it wide. \\
\poeml And you\fnote{\fbackref{57:8} 1QIsa\textsuperscript{a} reads sing.; MT reads pl.} have made a pact for yourself with them; \\
\poemll    you have loved their bed, \\
\poemlll       you have looked on their private parts.\fnote{\fbackref{57:8} Lit. \fbib{their hand}; i.e. a euphemism for the male sex organ} \\
\poeml \v{9}You went to Molech\fnote{\fbackref{57:9} I.e. to the Canaanite deity; or \fbib{to the king}} with olive oil \\
\poemll    and increased your perfumes; \\
\poeml you sent your ambassadors far away, \\
\poemll    you sent them down even to Sheol\fnote{\fbackref{57:9} I.e. the afterlife} itself! \\
\poeml \v{10}You grew tired with your many wanderings,\fnote{\fbackref{57:10} So 1QIsa\textsuperscript{a} LXX; MT reads \fbib{your wandering}} \\
\poemll    but you wouldn't say: `It is hopeless.' \\
\poeml You found new strength for your desire, \\
\poemll    and so you did not falter. \\
\poeml \v{11}``Whom did you so dread--- \\
\poemll    and while you feared me\fnote{\fbackref{57:11} So 1QIsa\textsuperscript{a}; 4QIsa\textsuperscript{d} MT LXX read \fbib{and fear}}--- \\
\poeml that you lied, \\
\poemll    and you did not remember me, \\
\poemlll       and\fnote{\fbackref{57:11} So 1QIsa\textsuperscript{a} 4QIsa\textsuperscript{d} LXX; MT lacks \fbib{and}} did not lay to heart these things?\fnote{\fbackref{57:11} So 1QIsa\textsuperscript{a}; 4QIsa\textsuperscript{d} MT lack \fbib{thing}; LXX reads \fbib{lay me to heart}} \\
\poeml Haven't I remained silent for a long time, \\
\poemll    and still you don't fear me?'' \\
\poeml \v{12}``I will denounce your righteousness\fnote{\fbackref{57:12} 1QIsa\textsuperscript{a} MT; 4QIsa\textsuperscript{d} reads \fbib{justice}} and your works, \\
\poemll    for your collections of idols\fnote{\fbackref{57:12} 1QIsa\textsuperscript{a} lacks \fbib{of idols}; 4QIsa\textsuperscript{d} MT LXX read \fbib{for they}} will not benefit you. \\
\poeml \v{13}When you cry out, let your collection deliver you! \\
\poemll    The wind will carry them all off, \\
\poemlll       and\fnote{\fbackref{57:13} So 1QIsa\textsuperscript{a} LXX; MT lacks \fbib{and}} a mere breath will sweep them all away.''
\passage{God's Reward for the Faithful}
\poeml ``But whoever\fnote{\fbackref{57:13} Lit. \fbib{But the one}; so 4QIsa\textsuperscript{d} MT; 1QIsa\textsuperscript{a} reads \fbib{But one} (i.e. without article)} takes refuge in me will possess the land, \\
\poemll    and will inherit my holy mountain. \\
\poeml \v{14}And one has said:\fnote{\fbackref{57:14} So 1QIsa\textsuperscript{a}; MT reads \fbib{one will say}; or \fbib{I will say}; LXX reads \fbib{they will say}} \\
\poemll    `Build up! Build up the road!\fnote{\fbackref{57:14} So 1QIsa\textsuperscript{a} LXX; MT lacks \fbib{the road}} \\
\poemlll       Prepare the highway! \\
\poeml Remove every obstacle from my people's way.' \\
\poeml \v{15}``For this is what the high and lofty One says, \\
\poemll    who inhabits eternity, whose name is Holy: \\
\poeml ``He lives\fnote{\fbackref{57:15} So 1QIsa\textsuperscript{a} 4QIsa\textsuperscript{d}; MT reads \fbib{I live}} in the height and in holiness,\fnote{\fbackref{57:15} So 1QIsa\textsuperscript{a}; 4QIsa\textsuperscript{d} MT read \fbib{in the high and holy place}} \\
\poemll    and also with the one who is of a contrite and humble spirit, \\
\poeml to revive the spirit of the humble, \\
\poemll    and to revive the heart of the contrite. \\
\poeml \v{16}For I won't accuse forever, \\
\poemll    nor will I always be angry; \\
\poeml for then the human spirit would grow faint before me--- \\
\poemll    even the souls that I have created. \\
\poeml \v{17}Because of his wicked greed I was angry, \\
\poemll    so I punished him; \\
\poeml and\fnote{\fbackref{57:17} So 1QIsa\textsuperscript{a} 4QIsa\textsuperscript{d} LXX; 1QIsa\textsuperscript{b} MT lack \fbib{and}} I hid my face, and was angry--- \\
\poemll    but he kept turning back to his stubborn will.\fnote{\fbackref{57:17} Lit. \fbib{into the way of his heart}} \\
\poeml \v{18}I've seen his ways,\fnote{\fbackref{57:18} So 1QIsa\textsuperscript{a} MT LXX; 4QIsa\textsuperscript{d} reads \fbib{way}} yet I will heal him,\fnote{\fbackref{57:18} So 1QIsa\textsuperscript{a}; MT reads \fbib{him, and I will guide him}; LXX reads \fbib{him, and I will exhort him}} \\
\poemll    and restore for him\fnote{\fbackref{57:18} So 1QIsa\textsuperscript{a}; the reading is probably an error: MT LXX lack \fbib{for him}} comfort\fnote{\fbackref{57:18} So 1QIsa\textsuperscript{a}; 1QIsa\textsuperscript{b} MT use different but related words} to him \\
\poemlll       and for those who mourn for him\fnote{\fbackref{57:18} Lit. \fbib{for his mourners}} \\
\poeml \v{19}when\fnote{\fbackref{57:19} So 1QIsa\textsuperscript{a}; 1QIsa\textsuperscript{b} 4QIsa\textsuperscript{d} MT lack \fbib{when}} I create the fruit of the lips: \\
\poemll    Peace\fnote{\fbackref{57:19} So 1QIsa\textsuperscript{a}; 1QIsa\textsuperscript{b} MT read \fbib{Peace, peace} cf. LXX} to the one who is far away or near,'' says the \divine{Lord}, \\
\poemlll       ``and I'll heal him. \\
\poeml \v{20}But the wicked are tossed like the sea;\fnote{\fbackref{57:20} So 1QIsa\textsuperscript{a} LXX; 4QIsa\textsuperscript{d} MT read \fbib{are like the tossing sea}} \\
\poemll    for it is not able to\fnote{\fbackref{57:20} So 1QIsa\textsuperscript{a}; 4QIsa\textsuperscript{d} MT lack \fbib{able to}} keep still, \\
\poemlll       and its waters toss up mire and mud. \\
\poeml \v{21}``Yet\fnote{\fbackref{57:21} So 1QIsa\textsuperscript{a}; MT LXX lack \fbib{Yet}} there is no peace,'' says my God, ``for the wicked.''
\end{poetry}
\labelchapt{58}
\passage{False and True Worship}

\chapt{58}
\v{1}``Shout aloud!

\begin{poetry}
\poemll    Don't hold back! \\
\poemlll       Lift up your voice like a trumpet! \\
\poeml Declare to my people their rebellions,\fnote{\fbackref{58:1} So 1QIsa\textsuperscript{a} LXX; 1QIsa\textsuperscript{b} MT read \fbib{rebellion}} \\
\poemll    and to the house of Jacob their sins. \\
\poeml \v{2}They\fnote{\fbackref{58:2} So 1QIsa\textsuperscript{a} 1QIsa\textsuperscript{b} 4QIsa\textsuperscript{d} LXX; MT reads \fbib{And they}} seek me day after day,\fnote{\fbackref{58:2} Lit. \fbib{me day and day}; so 1QIsa\textsuperscript{a}; 1QIsa\textsuperscript{b} 4QIsa\textsuperscript{d} MT read \fbib{me day, day}} \\
\poemll    and are eager to know my ways, \\
\poeml as if they were a nation that practices righteousness \\
\poemll    and has not forsaken the justice of their God. \\
\poeml ``They ask me to reveal just decisions; \\
\poemll    they are eager to draw near to God. \\
\poeml \v{3}`Why have we fasted,' they ask,\fnote{\fbackref{58:3} 1QIsa\textsuperscript{a} 1QIsa\textsuperscript{b} 4QIsa\textsuperscript{d} LXX MT lack \fbib{they ask}} \\
\poemll    `but you do not see? \\
\poeml `Why have we humbled ourselves,'\fnote{\fbackref{58:3} So 1QIsa\textsuperscript{a} 1QIsa\textsuperscript{b} LXX; MT reads \fbib{ourself}} they ask,\fnote{\fbackref{58:3} 1QIsa\textsuperscript{a} 1QIsa\textsuperscript{b} 4QIsa\textsuperscript{d} LXX MT lack \fbib{they ask}} \\
\poemll    `but you take no notice?'\,''
\passage{Fasting that God Approves}
\poeml ``Look! On your fast day you serve your own interest \\
\poemll    and oppress all your workers. \\
\poeml \v{4}``Look! You fast only for quarreling, and for\fnote{\fbackref{58:4} So 1QIsa\textsuperscript{b} 1QIsa\textsuperscript{a}; MT LXX lack \fbib{for}} fighting, \\
\poemll    and for hitting with wicked fists. \\
\poeml You cannot fast as you do today \\
\poemll    and have your voice heard on high. \\
\poeml \v{5}``Is this the kind of fast that I have chosen, \\
\poemll    merely a day for a person to humble himself? \\
\poeml Is it merely for bowing down one's head like a bulrush, \\
\poemll    for lying\fnote{\fbackref{58:5} So 1QIsa\textsuperscript{a} 1QIsa\textsuperscript{b}; MT LXX read \fbib{and for lying}} on sackcloth and ashes? \\
\poeml Is this what you\fnote{\fbackref{58:5} So 1QIsa\textsuperscript{a} 4QIsa\textsuperscript{d} LXX (pl.); 1QIsa\textsuperscript{b} MT (sing.)} call a fast, \\
\poemll    an\fnote{\fbackref{58:5} So 1QIsa\textsuperscript{a} 1QIsa\textsuperscript{b}; MT reads \fbib{and an}} acceptable day to the \divine{Lord}? \\
\poeml \v{6}Isn't this the\fnote{\fbackref{58:6} So 1QIsa\textsuperscript{a}; 1QIsa\textsuperscript{b} MT LXX lack \fbib{the}} fast that\fnote{\fbackref{58:6} So 1QIsa\textsuperscript{a}; 1QIsa\textsuperscript{b} MT LXX lack \fbib{that}} I have been choosing: \\
\poemll    to loose the bonds of injustice, \\
\poeml and\fnote{\fbackref{58:6} So 1QIsa\textsuperscript{a}; 1QIsa\textsuperscript{b} MT LXX lack \fbib{and}} to untie the cords of the yoke, \\
\poemll    and\fnote{\fbackref{58:6} So 1QIsa\textsuperscript{a} MT; 1QIsa\textsuperscript{b} 4QIsa\textsuperscript{d} LXX lack \fbib{and}} to let the oppressed go free, \\
\poemlll       and to break every yoke? \\
\poeml \v{7}Isn't it to share your bread with the hungry, \\
\poemll    and to bring the homeless poor into your house; \\
\poeml when you see the naked, \\
\poemll    to cover him with clothing,\fnote{\fbackref{58:7} So 1QIsa\textsuperscript{a}; 1QIsa\textsuperscript{b} MT LXX lack \fbib{with clothing}} \\
\poemlll       and not to raise yourself up\fnote{\fbackref{58:7} So 1QIsa\textsuperscript{a}; 1QIsa\textsuperscript{b} MT read \fbib{to hide yourself}; LXX reads \fbib{to disregard}} from your own flesh and blood?''
\passage{God's Reward}
\poeml \v{8}``Then your light will break forth like the dawn, \\
\poemll    and your healing will spring up quickly; \\
\poeml and your vindication will go before you, \\
\poemll    and\fnote{\fbackref{58:8} So 1QIsa\textsuperscript{a} 1QIsa\textsuperscript{b} LXX; MT lacks \fbib{and}} the glory of the \divine{Lord} will guard your back. \\
\poeml \v{9}Then you'll call, \\
\poemll    and the \divine{Lord} will answer; \\
\poeml you'll cry for help, \\
\poemll    and he'll respond, `Here I am.' \\
\poeml ``If you do away with the yoke among you, \\
\poemll    and\fnote{\fbackref{58:9} So 1QIsa\textsuperscript{a} LXX; 1QIsa\textsuperscript{b} MT lack \fbib{and}} pointing fingers and malicious talk; \\
\poeml \v{10}if you pour yourself out for the hungry \\
\poemll    and satisfy the needs of afflicted souls, \\
\poeml then your light will rise in darkness, \\
\poemll    and your night will be like noonday. \\
\poeml \v{11}And the \divine{Lord} will guide you continually, \\
\poemll    and satisfy your soul in parched places,\fnote{\fbackref{58:11} 1QIsa\textsuperscript{a} spells the word \fbib{places} incorrectly} \\
\poemlll       and they\fnote{\fbackref{58:11} So 1QIsa\textsuperscript{a} 1QIsa\textsuperscript{b}; MT reads \fbib{he}; LXX reads \fbib{and your bones will be strengthened}} will strengthen your bones; \\
\poeml and you'll be like a watered garden, \\
\poemll    like a spring of water, \\
\poemlll       whose waters never fail. \\
\poeml \v{12}And your people will rebuild the ancient ruins; \\
\poemll    You'll raise up the age-old foundations,\fnote{\fbackref{58:12} Lit. \fbib{the foundations of many generations}} \\
\poeml and people will call you\fnote{\fbackref{58:12} So 1QIsa\textsuperscript{a}; 1QIsa\textsuperscript{b} MT LXX read \fbib{you will be called}} `Repairer of Broken Walls,' \\
\poemll    `Restorer of Streets to Live In.' \\
\poeml \v{13}``If you keep your feet from trampling the Sabbath, \\
\poemll    from\fnote{\fbackref{58:13} So 1QIsa\textsuperscript{a} 4QIsa\textsuperscript{n}; 1QIsa\textsuperscript{b} MT lack \fbib{from}} pursuing your own interests on my holy day, \\
\poeml if you call the Sabbath a delight \\
\poemll    and\fnote{\fbackref{58:13} So 1QIsa\textsuperscript{a} 1QIsa\textsuperscript{b} 4QIsa\textsuperscript{n}; MT LXX lack \fbib{and}} the \divine{Lord}'s holy day honorable; \\
\poeml and if you honor it by not going your own ways\fnote{\fbackref{58:13} So 1QIsa\textsuperscript{a} 4QIsa\textsuperscript{n} MT; 1QIsa\textsuperscript{b} reads \fbib{way}} \\
\poemll    and\fnote{\fbackref{58:13} So 1QIsa\textsuperscript{a}; 1QIsa\textsuperscript{b} MT lack \fbib{and}} seeking your own pleasure or speaking merely idle\fnote{\fbackref{58:13} 1QIsa\textsuperscript{a} 1QIsa\textsuperscript{b} 4QIsa\textsuperscript{n} MT LXX lack \fbib{merely idle}} words, \\
\poeml \v{14}then you will take delight in the \divine{Lord}, \\
\poemll    and he\fnote{\fbackref{58:14} So 1QIsa\textsuperscript{a} 1QIsa\textsuperscript{b} 4QIsa\textsuperscript{n} LXX; MT reads \fbib{and I}} will make you ride upon the heights of the earth; \\
\poeml and he\fnote{\fbackref{58:14} So 1QIsa\textsuperscript{a} LXX; 1QIsa\textsuperscript{b} 4QIsa\textsuperscript{n} MT read \fbib{and I}} will make you feast on the inheritance of your ancestor Jacob, your father. \\
\poeml ``Yes! The mouth of the \divine{Lord} has spoken.''
\end{poetry}
\labelchapt{59}
\passage{Sins that Separate from God}

\chapt{59}
\v{1}``See, the \divine{Lord}'s hand is not too short to save,

\begin{poetry}
\poemll    nor are his ears\fnote{\fbackref{59:1} So 1QIsa\textsuperscript{a}; MT LXX read \fbib{ear}} too dull to hear. \\
\poeml \v{2}Instead, your iniquities have been barriers \\
\poemll    between you and your God, \\
\poeml and your sins have concealed his face from you \\
\poemll    so that he won't listen. \\
\poeml \v{3}For your hands are defiled with blood, \\
\poemll    and your fingers with iniquity; \\
\poemlll       your tongue\fnote{\fbackref{59:3} So 1QIsa\textsuperscript{a}; MT LXX read \fbib{your lips have spoken lies, your tongue}} mutters wickedness. \\
\poeml \v{4}No one brings a lawsuit fairly, \\
\poemll    and no one goes to law honestly; \\
\poeml they have relied\fnote{\fbackref{59:4} So 1QIsa\textsuperscript{a} 1QIsa\textsuperscript{b}; MT reads \fbib{they rely}} on empty arguments \\
\poemll    and they tell lies; \\
\poeml they conceive\fnote{\fbackref{59:4} So 1QIsa\textsuperscript{a}; 1QIsa\textsuperscript{b} MT reads \fbib{to conceive}} trouble \\
\poemll    and give birth\fnote{\fbackref{59:4} So 1QIsa\textsuperscript{a} 1QIsa\textsuperscript{b}; MT reads \fbib{and to give birth}} to iniquity. \\
\poeml \v{5}They hatch\fnote{\fbackref{59:5} So 1QIsa\textsuperscript{a}; 1QIsa\textsuperscript{b} MT reads \fbib{They have hatched}; LXX reads \fbib{hatched}} adders' eggs\fnote{\fbackref{59:5} So 1QIsa\textsuperscript{a} LXX; 1QIsa\textsuperscript{b} MT read \fbib{an adder's eggs}} \\
\poemll    and weave\fnote{\fbackref{59:5} So 1QIsa\textsuperscript{a}; MT reads \fbib{weave}, but with a different Heb. word} a spider's web; \\
\poeml whoever eats their eggs dies, \\
\poemll    and any crushed egg hatches out futility.\fnote{\fbackref{59:5} Lit. \fbib{a viper}; so 1QIsa\textsuperscript{a}; 1QIsa\textsuperscript{b} MT LXX utilize feminine form} \\
\poeml \v{6}Their cobwebs cannot become clothing, \\
\poemll    they cannot cover themselves with what they make. \\
\poeml Their deeds are deeds of iniquity, \\
\poemll    and acts of violence fill their hands. \\
\poeml \v{7}Their feet rush to evil, \\
\poemll    and they are quick to shed innocent blood. \\
\poeml Their thoughts are thoughts of iniquity; \\
\poemll    ruin, destruction, and violence\fnote{\fbackref{59:7} So 1QIsa\textsuperscript{a}; MT LXX lack \fbib{and violence}} are in their paths. \\
\poeml \v{8}The pathway of peace they do not know, \\
\poemll    and there is no justice in their courses. \\
\poeml They have made their roads crooked; \\
\poemll    no one who walks in them will know peace.''
\passage{A Commitment to Wait on God}
\poeml \v{9}``So justice is far from us, \\
\poemll    and righteousness does not reach us. \\
\poeml We wait for light, but look---there is darkness; \\
\poemll    we wait for brightness, but we walk in deep darkness.\fnote{\fbackref{59:9} So 1QIsa\textsuperscript{a} LXX; MT reads \fbib{darknesses}} \\
\poeml \v{10}Let's grope\fnote{\fbackref{59:10} So 1QIsa\textsuperscript{a}; MT reads \fbib{We grope} LXX reads \fbib{They grope}} along the wall like the blind; \\
\poemll    let us grope like those who have no eyes. \\
\poeml We stumble at midday as if it were twilight, \\
\poemll    in desolate places\fnote{\fbackref{59:10} Or \fbib{among vigorous people}} like dead people. \\
\poeml \v{11}We all growl like bears; \\
\poemll    we\fnote{\fbackref{59:11} So 1QIsa\textsuperscript{a}; MT reads \fbib{and we}} sigh mournfully like doves. \\
\poeml We look for justice, but there is none, \\
\poemll    and\fnote{\fbackref{59:11} So 1QIsa\textsuperscript{a}; Not in MT LXX} for deliverance, but it's far from us. \\
\poeml \v{12}``For our transgressions before you are many, \\
\poemll    and our sins testify\fnote{\fbackref{59:12} 1QIsa\textsuperscript{a} cf. LXX; MT reads \fbib{sin testifies}} against us; \\
\poeml for our transgressions are with us, \\
\poemll    and as for our iniquities, \\
\poemlll       we acknowledge them: \\
\poeml \v{13}they've rebelled\fnote{\fbackref{59:13} So 1QIsa\textsuperscript{a}; MT reads \fbib{rebellion} LXX reads \fbib{we have sinned}} in\fnote{\fbackref{59:13} Lit. \fbib{and}} treachery against the \divine{Lord}, \\
\poemll    and are turning away from following our God; \\
\poeml and they've spoken\fnote{\fbackref{59:13} So 1QIsa\textsuperscript{a}; MT reads \fbib{speaking} LXX reads \fbib{we have spoken}} oppression and revolt, \\
\poemll    and are conceiving\fnote{\fbackref{59:13} So 1QIsa\textsuperscript{a}; MT reads \fbib{conceiving and uttering}; LXX reads \fbib{we have conceived and thought about}} lying words from the heart. \\
\poeml \v{14}I'll drive back justice,\fnote{\fbackref{59:14} So 1QIsa\textsuperscript{a}; MT reads \fbib{Justice is driven back} LXX reads \fbib{We withdrew justice}} \\
\poemll    and righteousness stands at a distance; \\
\poeml for truth has fallen in the public square, \\
\poemll    and honesty cannot enter. \\
\poeml \v{15}Truth is missing, \\
\poemll    and whoever turns away from evil becomes a prey.''
\passage{God Brings His Own Salvation}
\poeml ``Then the \divine{Lord} looked, and it displeased him \\
\poemll    that there was no justice. \\
\poeml \v{16}He saw that there was no one, \\
\poemll    and was appalled that there was no one to intervene; \\
\poeml so his own arm\fnote{\fbackref{59:16} I.e. \fbib{the Messiah}} brought him victory, \\
\poemll    and his righteous acts upheld him. \\
\poeml \v{17}He put on righteousness like a breastplate, \\
\poemll    and a helmet of salvation on his head; \\
\poeml he put on garments of vengeance for clothing, \\
\poemll    and wrapped himself in fury like a cloak. \\
\poeml \v{18}So he will repay according to their action: \\
\poemll    Anger to his enemies, retribution to his foes; \\
\poemlll       to the coastlands he will render their due. \\
\poeml \v{19}So people will fear the name of the \divine{Lord} from the west, \\
\poemll    and his glories\fnote{\fbackref{59:19} So 1QIsa\textsuperscript{a}; MT reads \fbib{glory} LXX reads \fbib{his glorious name}} from the rising of the sun; \\
\poeml for he will come as a pent-up stream \\
\poemll    that the breath of the \divine{Lord} drives along. \\
\poeml \v{20}``And a Redeemer will come to Zion, \\
\poemll    to those in Jacob who turn from transgression,'' says the \divine{Lord}.
\end{poetry}

\v{21}``As for me, this is my covenant with them,''\fnote{\fbackref{59:21} So 1QIsa\textsuperscript{a} MT\textsuperscript{mss} LXX; 1QIsa\textsuperscript{b} MT read \fbib{them}, but the reading contains a grammatical object error} says the \divine{Lord}. ``And\fnote{\fbackref{59:21} So 1QIsa\textsuperscript{a}; 1QIsa\textsuperscript{b} MT LXX lack \fbib{And}} my Spirit that is upon you, and my words that I have put in your mouth, won't depart from your mouth, or from the mouths of your children, or from the mouths of your children's children,\fnote{\fbackref{59:21} So 1QIsa\textsuperscript{a}; 1QIsa\textsuperscript{b} MT LXX read \fbib{children, says the \divine{Lord},}} from now on and forever.''
\labelchapt{60}
\passage{The Light of God's Deliverance}

\begin{poetry}
\poeml \chapt{60}
\v{1}``Arise, shine! \\
\poemll    For your light has come; \\
\poemlll       the\fnote{\fbackref{60:1} So 1QIsa\textsuperscript{a}; 1QIsa\textsuperscript{b} MT LXX read \fbib{and the glory}} glory of the \divine{Lord} has risen upon you. \\
\poeml \v{2}For look! Darkness will cover the earth \\
\poemll    and thick darkness is over the people,\fnote{\fbackref{60:2} Lit. \fbib{peoples}} \\
\poeml but the \divine{Lord} will arise upon you, \\
\poemll    and his glory will appear over you. \\
\poeml \v{3}Nations will come to your light, \\
\poemll    and kings before\fnote{\fbackref{60:3} So 1QIsa\textsuperscript{a}; MT reads \fbib{kings to the brightness of}; cf. LXX} your dawn. \\
\poeml \v{4}``Lift up your eyes and look around: \\
\poemll    They all gather together, they come to you; \\
\poeml your sons will come from far away, \\
\poemll    and your daughters will be carried on the hip.''\fnote{\fbackref{60:4} Or \fbib{arm}} \\
\poeml \v{5}Then you will look and be radiant; \\
\poemll    your heart will swell with joy,\fnote{\fbackref{60:5} So 1QIsa\textsuperscript{a}; 1QIsa\textsuperscript{b} MT read \fbib{will throb and swell with joy} LXX reads \fbib{you will be amazed in your heart}} \\
\poeml because the abundance of the seas will be diverted to you, \\
\poemll    and the riches of the nations will come to you. \\
\poeml \v{6}Throngs of camels will blanket you: \\
\poemll    the young camels of Midian and Ephu;\fnote{\fbackref{60:6} So 1QIsa\textsuperscript{a}; 1QIsa\textsuperscript{b} MT read \fbib{Ephah}} \\
\poemlll       all those from Shebu\fnote{\fbackref{60:6} So 1QIsa\textsuperscript{a}; 1QIsa\textsuperscript{b} MT read \fbib{Sheba}} will come. \\
\poeml They'll carry gold and frankincense, \\
\poemll    and proclaim the praise of the \divine{Lord}. \\
\poeml \v{7}All Kedar's flocks will be gathered to you, \\
\poemll    the rams of Nebaioth will serve you. \\
\poeml and\fnote{\fbackref{60:7} So 1QIsa\textsuperscript{a} LXX; 1QIsa\textsuperscript{b} MT lack \fbib{and}} they'll come up with acceptance upon\fnote{\fbackref{60:7} So 1QIsa\textsuperscript{a} cf. LXX; 1QIsa\textsuperscript{b} MT read \fbib{upon the acceptance of}} my altar, \\
\poemll    and I'll glorify my glorious house.''
\passage{The Future Restoration of Zion}
\poeml \v{8}``Who are these that fly like clouds, \\
\poemll    and like doves to their windows?\fnote{\fbackref{60:8} I.e. \fbib{dovecotes}, cages in which pet doves are housed} \\
\poeml \v{9}For the coastlands will look to me, \\
\poemll    with the ships of Tarshish in the lead, \\
\poeml to bring my\fnote{\fbackref{60:9} So 1QIsa\textsuperscript{a}; 1QIsa\textsuperscript{b} MT LXX read \fbib{your}} children from far away, \\
\poemll    their silver and gold with them, \\
\poeml to the name of the \divine{Lord} your God, \\
\poemll    and to the Holy One of Israel, \\
\poemlll       because he has glorified you. \\
\poeml \v{10}``Foreigners will rebuild your walls, \\
\poemll    and their kings will serve you. \\
\poeml Though in my anger I struck you down, \\
\poemlll       in my favor I have shown you mercy. \\
\poeml \v{11}Your gates will always stand open \\
\poemll    by day or night, and\fnote{\fbackref{60:11} So 1QIsa\textsuperscript{a}; 1QIsa\textsuperscript{b} MT LXX lack \fbib{and}} they will not be shut, \\
\poeml so that nations will bring you their wealth, \\
\poemll    with their kings led in procession. \\
\poeml \v{12}For the nation or kingdom \\
\poemll    that will not serve you will perish; \\
\poemlll       those nations will be utterly ruined. \\
\poeml \v{13}``He has given you\fnote{\fbackref{60:13} So 1QIsa\textsuperscript{a}; 1QIsa\textsuperscript{b} MT LXX lack \fbib{He has given you}} the glory of Lebanon, \\
\poemll    and it will come\fnote{\fbackref{60:13} So 1QIsa\textsuperscript{a}; 1QIsa\textsuperscript{b} MT LXX read \fbib{Lebanon will come}} to you, \\
\poemlll       the cypress, and\fnote{\fbackref{60:13} So 1QIsa\textsuperscript{a}; 1QIsa\textsuperscript{b} MT lack \fbib{and}} the plane tree,\fnote{\fbackref{60:13} i.e. a species of trees that could readily be stripped of their bark; cf. Gen 30:37} and the pine, \\
\poeml to adorn the place of my sanctuary; \\
\poemll    and I will make the place of my feet glorious. \\
\poeml \v{14}``All\fnote{\fbackref{60:14} So 1QIsa\textsuperscript{a}; 1QIsa\textsuperscript{b} MT LXX lack \fbib{All}} the descendants of those who oppressed you \\
\poemll    will come bending low before you, \\
\poeml and all those who despised you \\
\poemll    will bow down at your feet. \\
\poeml They'll call you `The City of the \divine{Lord},' \\
\poemll    `Zion of the Holy One of Israel.'\,''
\passage{Israel: the Joy of Generations}
\poeml \v{15}``Although you have been forsaken and despised, \\
\poemll    with no one traveling through, \\
\poeml I will make you the everlasting pride, \\
\poemll    the joy of all generations. \\
\poeml \v{16}You'll suck the milk of nations, \\
\poemll    You'll suck the breasts of kings. \\
\poeml Then you will realize that I, the \divine{Lord}, am your Savior \\
\poemll    and your Redeemer, the Mighty One of Jacob. \\
\poeml \v{17}``Instead of bronze, I'll bring gold, \\
\poemll    and instead of iron, I'll bring silver; \\
\poeml instead of wood, bronze, \\
\poemll    and instead of stones, iron. \\
\poeml I'll appoint peace as your supervisor \\
\poemll    and righteousness as your taskmaster. \\
\poeml \v{18}Then\fnote{\fbackref{60:18} So 1QIsa\textsuperscript{a} LXX; 1QIsa\textsuperscript{b} MT lack \fbib{Then}} violence will no longer be heard in your land, \\
\poemll    nor devastation or destruction within your borders; \\
\poeml but you'll call your walls `Salvation', \\
\poemll    and your gates `Praise'. \\
\poeml \v{19}``The sun will no longer be your light by day, \\
\poemll    nor for brightness will the moon shine on you by night\fnote{\fbackref{60:19} So 1QIsa\textsuperscript{a} LXX. Not in 1QIsa\textsuperscript{b} MT}--- \\
\poeml for the \divine{Lord} will be your everlasting light,\fnote{\fbackref{60:19} 1QIsa\textsuperscript{b} lacks \fbib{and your God {\ldots} light}; 1QIsa\textsuperscript{a} MT LXX contain the longer reading} \\
\poemll    and your God will be your glory. \\
\poeml \v{20}Your sun won't\fnote{\fbackref{60:20} So 1QIsa\textsuperscript{a} LXX; MT reads \fbib{will no longer}} set, \\
\poemll    nor will your moon withdraw itself--- \\
\poeml for the \divine{Lord} will be your everlasting light, \\
\poemll    and your days of mourning will end. \\
\poeml \v{21}Then your people will all be righteous; \\
\poemll    They'll possess the land forever. \\
\poeml They are the shoot\fnote{\fbackref{60:21} So 1QIsa\textsuperscript{a} 4QIsa\textsuperscript{m} MT; 1QIsa\textsuperscript{a} MT\textsuperscript{ms} lacks \fbib{the shoot}; LXX reads \fbib{Guarding}} that the \divine{Lord} planted,\fnote{\fbackref{60:21} Lit. \fbib{of the plantings of the \divine{Lord}}; so 1QIsa\textsuperscript{a}; 1QIsa\textsuperscript{b} reads \fbib{of his plantings} MT reads \fbib{of his planting} MT\textsuperscript{qere} reads \fbib{of my planting} LXX reads \fbib{the planting}} \\
\poemll    the works\fnote{\fbackref{60:21} So 1QIsa\textsuperscript{a}; 1QIsa\textsuperscript{b} MT read \fbib{work}} of his hands, \\
\poemlll       so that I might be glorified. \\
\poeml \v{22}The least of them will become a thousand, \\
\poemll    and the smallest one a mighty nation. \\
\poeml ``I am the \divine{Lord}; \\
\poemll    When the time is right,\fnote{\fbackref{60:22} Lit. \fbib{In its time}} I will do this swiftly.''
\end{poetry}
\labelchapt{61}
\passage{Good News of Deliverance}

\begin{poetry}
\poeml \chapt{61}
\v{1}``The Spirit of the \divine{Lord}\fnote{\fbackref{61:1} So 1QIsa\textsuperscript{a} 1QIsa\textsuperscript{b} LXX; 4QIsa\textsuperscript{m} MT read \fbib{the Lord \divine{God}}} is upon me, \\
\poemll    because the \divine{Lord} has anointed me; \\
\poeml he has sent me to bring good news to the oppressed \\
\poemll    and\fnote{\fbackref{61:1} So 1QIsa\textsuperscript{a}; MT LXX lacks \fbib{and}} to bind up the brokenhearted, \\
\poeml to proclaim freedom for the captives, \\
\poemll    and release from darkness\fnote{\fbackref{61:1} Or \fbib{prison}; or \fbib{and opening of the eyes}} for the prisoners; \\
\poeml \v{2}to proclaim the year of the \divine{Lord}'s favor, \\
\poemll    the\fnote{\fbackref{61:2} So 1QIsa\textsuperscript{a} LXX\textsuperscript{ms}; 4QIsa\textsuperscript{b} MT LXX read \fbib{and the}} day of vengeance of our God; \\
\poemlll       to comfort all who mourn; \\
\poeml \v{3}to provide for those who grieve in Zion--- \\
\poemll    to bestow on them a crown of beauty instead of ashes, \\
\poeml the oil of gladness instead of mourning, \\
\poemll    a mantle of praise instead of a spirit of despair.'' \\
\poeml ``Then people will call them\fnote{\fbackref{61:3} So 1QIsa\textsuperscript{a}; MT LXX read \fbib{they will be called}} ``Oaks of Righteousness'', \\
\poemll    ``The Planting of the \divine{Lord}'', \\
\poemlll       in order to display his splendor. \\
\poeml \v{4}They will rebuild the ancient ruins; \\
\poemll    they will restore the places long devastated; \\
\poeml they will build again the ruined cities, \\
\poemll    they will build again\fnote{\fbackref{61:4} So 1QIsa\textsuperscript{a}; MT LXX lacks \fbib{again}} the places devastated for many generations. \\
\poeml \v{5}Strangers will stand and feed your flocks, \\
\poemll    and foreigners will work your land \\
\poemlll       and dress your vines. \\
\poeml \v{6}But as for you, you will be called priests of the \divine{Lord}, \\
\poemll    and\fnote{\fbackref{61:6} So 1QIsa\textsuperscript{a}; MT LXX lacks \fbib{and}} you will be named ministers of our God. \\
\poeml You will feed on the wealth of the nations, \\
\poemll    and you will boast about their riches. \\
\poeml \v{7}Instead of your shame you will receive double, \\
\poemll    and instead of disgrace people will shout with joy over your\fnote{\fbackref{61:7} So 1QIsa\textsuperscript{a}; MT reads \fbib{their}} inheritance; \\
\poeml therefore you\fnote{\fbackref{61:7} So 1QIsa\textsuperscript{a}; MT LXX reads \fbib{they}} will inherit a double portion in their land; \\
\poemll    everlasting joy will be yours.''\fnote{\fbackref{61:7} So 1QIsa\textsuperscript{a}; MT reads \fbib{theirs}; LXX reads \fbib{over their head}} \\
\poeml \v{8}``For I, the \divine{Lord}, love justice, \\
\poemll    and\fnote{\fbackref{61:8} So 1QIsa\textsuperscript{a} LXX; MT lacks \fbib{and}} I hate robbery and iniquity; \\
\poeml I will faithfully present your reward\fnote{\fbackref{61:8} So 1QIsa\textsuperscript{a}; MT LX read \fbib{present their reward}} \\
\poemll    and make an everlasting covenant with you.\fnote{\fbackref{61:8} So 1QIsa\textsuperscript{a}; MT LXX read \fbib{with them}} \\
\poeml \v{9}Your\fnote{\fbackref{61:9} So 1QIsa\textsuperscript{a}; MT LXX read \fbib{Their}} offspring will be known among the nations, \\
\poemll    and your\fnote{\fbackref{61:9} So 1QIsa\textsuperscript{a}; MT LXX read \fbib{their}} descendants among the people.\fnote{\fbackref{61:9} Lit. \fbib{peoples}} \\
\poeml All who see them will acknowledge them, \\
\poemll    that they are an offspring the \divine{Lord} has blessed.''
\passage{Rejoicing in God's Deliverance}
\poeml \v{10}``I will heartily rejoice in the \divine{Lord}, \\
\poemll    my soul will delight in my God; \\
\poeml for he has wrapped me in garments of salvation; \\
\poemll    he has arrayed me in a robe of righteousness, \\
\poeml just like a bridegroom, \\
\poemll    like a priest\fnote{\fbackref{61:10} So 1QIsa\textsuperscript{a}; MT reads \fbib{bridegroom decks himself like a priest}; LXX reads \fbib{bridegroom decks me}} with a garland, \\
\poemlll       and like a bride adorns herself with her jewels. \\
\poeml \v{11}For just as the soil brings forth its shoots, \\
\poemll    and as a garden makes what is sown within it spring up, \\
\poeml so the \divine{Lord} God\fnote{\fbackref{61:11} So 1QIsa\textsuperscript{a}; MT reads \fbib{the Lord \divine{God}}; LXX reads \fbib{the \divine{Lord}}} will make righteousness and praise \\
\poemll    spring up before all the nations for Zion's sake.''\fnote{\fbackref{61:11} So MT LXX; 1QIsa\textsuperscript{a} ends v. 11 with \fbib{for Zion's sake}}
\end{poetry}
\labelchapt{62}
\passage{The Vindication of Jerusalem}

\begin{poetry}
\poeml \chapt{62}
\v{1}``And\fnote{\fbackref{62:1} So 1QIsa\textsuperscript{a}; MT LXX lack \fbib{And}} I won't remain silent,\fnote{\fbackref{62:1} 1QIsa\textsuperscript{a} and MT use different Hebrew verbs for \fbib{silent}} \\
\poemll    and for Jerusalem's sake I won't stay quiet, \\
\poeml until her vindication shines out like brightness, \\
\poemll    and her salvation like a burning torch. \\
\poeml \v{2}The nations will see your vindication, \\
\poemll    and all the kings your glory; \\
\poeml and people will call you\fnote{\fbackref{62:2} So 1QIsa\textsuperscript{a}; MT reads \fbib{and you will be called}; LXX reads \fbib{he will call you}} by a new name \\
\poemll    that the mouth of the \divine{Lord} will bestow. \\
\poeml \v{3}You will be a crown of splendor in the \divine{Lord}'s hand, \\
\poemll    and a royal diadem in the hand of your God. \\
\poeml \v{4}And\fnote{\fbackref{62:4} So 1QIsa\textsuperscript{a} LXX; MT lacks \fbib{And}} you'll no longer be called `Deserted,' \\
\poemll    and your land will no longer be called `Desolate'; \\
\poeml but people will call you\fnote{\fbackref{62:4} So 1QIsa\textsuperscript{a}; 1QIsa\textsuperscript{b} MT lacks \fbib{will call you}} `Hephzibah,'\fnote{\fbackref{62:4} The Heb. word \fbib{Hephzibah} means \fbib{My Delight is in Her}} \\
\poemll    and your land `Beulah'\fnote{\fbackref{62:4} The Heb. word \fbib{Beulah} means \fbib{Married}}--- \\
\poeml for the \divine{Lord} will take delight in you, \\
\poemll    and your land will be married.'' \\
\poeml \v{5}``For just as\fnote{\fbackref{62:5} So 1QIsa\textsuperscript{a} LXX; 1QIsa\textsuperscript{b} MT lack \fbib{For just as}} a young man marries a maiden, \\
\poemll    so your sons will marry you; \\
\poeml and just as a bridegroom rejoices over his bride, \\
\poemll    so your God will rejoice over you. \\
\poeml \v{6}``Upon your walls, Jerusalem, \\
\poemll    I have posted watchmen; \\
\poeml all day and all night \\
\poemll    they won't\fnote{\fbackref{62:6} So 1QIsa\textsuperscript{a} 1QIsa\textsuperscript{b}; MT LXX read \fbib{never}} remain silent. \\
\poeml You who make mention of the \divine{Lord}, \\
\poemll    take no rest, \\
\poeml \v{7}and give him no rest \\
\poemll    until he prepares, establishes\fnote{\fbackref{62:7} So 1QIsa\textsuperscript{a}; 1QIsa\textsuperscript{b} MT LXX read \fbib{he establishes}} and makes Jerusalem \\
\poemlll       a song of praise throughout the earth. \\
\poeml \v{8}``The \divine{Lord} has sworn by his right hand \\
\poemll    and by his mighty arm:\fnote{\fbackref{62:8} I.e. \fbib{the Messiah}} \\
\poeml `I will never again give your grain\fnote{\fbackref{62:8} So 1QIsa\textsuperscript{a} 1QIsa\textsuperscript{b}; MT reads \fbib{I will never give your grain again}} \\
\poemll    as food for your enemies; \\
\poeml never\fnote{\fbackref{62:8} So 1QIsa\textsuperscript{a}; 1QIsa\textsuperscript{b} MT LXX read \fbib{and never}} again will foreigners drink your new wine \\
\poemll    for which you have toiled; \\
\poeml \v{9}but surely\fnote{\fbackref{62:9} So 1QIsa\textsuperscript{a} LXX; 1QIsa\textsuperscript{b} MT lack \fbib{but surely}} those who harvest it will eat it \\
\poemll    and praise the name of\fnote{\fbackref{62:9} So 1QIsa\textsuperscript{a}; 1QIsa\textsuperscript{b} MT LXX lack \fbib{the name of}} the \divine{Lord}, \\
\poeml and those who gather it will drink it \\
\poemll    in the courts of my sanctuary,' says your God.''\fnote{\fbackref{62:9} So 1QIsa\textsuperscript{a}; 1QIsa\textsuperscript{b} MT LXX lack \fbib{says your God}}
\passage{The Coming of God to Reign}
\poeml \v{10}``Pass through\fnote{\fbackref{62:10} So 1QIsa\textsuperscript{a} LXX; 1QIsa\textsuperscript{b} MT read \fbib{Pass through! Pass through}} the gates! \\
\poemll    prepare the way for the people! \\
\poeml Build up! Build up the highway! \\
\poemll    Clear it of stumbling stones,\fnote{\fbackref{62:10} So 1QIsa\textsuperscript{a}; cf. Isa 8:14; 1QIsa\textsuperscript{b} MT LXX read \fbib{of stones}} \\
\poemlll       speak among the peoples.\fnote{\fbackref{62:10} So 1QIsa\textsuperscript{a}; 1QIsa\textsuperscript{b} MT LXX read \fbib{raise a banner over the peoples}} \\
\poeml \v{11}Here is the \divine{Lord}! \\
\poemll    Proclaim\fnote{\fbackref{62:11} So 1QIsa\textsuperscript{a}; 1QIsa\textsuperscript{b} MT LXX read \fbib{See, the \divine{Lord} has proclaimed}} to the ends\fnote{\fbackref{62:11} So 1QIsa\textsuperscript{a}; 1QIsa\textsuperscript{b} MT LXX read \fbib{end}} of the earth, \\
\poeml say to the inhabitants\fnote{\fbackref{62:11} Lit. \fbib{daughter}} of Zion: \\
\poemll    `See, your salvation is coming! \\
\poeml See, his reward is with him, \\
\poemll    and his recompenses are\fnote{\fbackref{62:11} So 1QIsa\textsuperscript{a}; 1QIsa\textsuperscript{b} MT read \fbib{recompense is}; 1QIsa\textsuperscript{b} MT LXX read \fbib{work is}} before him.' \\
\poeml \v{12}People will call them, `The Holy People,' \\
\poemll    `The Redeemed of the \divine{Lord}'; \\
\poeml and they will call you,\fnote{\fbackref{62:12} So 1QIsa\textsuperscript{a}; 1QIsa\textsuperscript{b} MT LXX read \fbib{you will be called}} `Sought After,' \\
\poemll    `The City Not Deserted.'\,''
\end{poetry}
\labelchapt{63}
\passage{God's Day of Vengeance}

\chapt{63}
\v{1}``Who is this coming from Edom,

\begin{poetry}
\poemll    from Bozrah, in garments stained crimson? \\
\poeml Who is this, robed in such splendor, \\
\poemll    marching in his great might? \\
\poeml It is I, speaking in vindication, \\
\poemll    mighty to save. \\
\poeml \v{2}``Why is your clothing red, \\
\poemll    and your garments like those worn by the ones who tread in the winepress?\fnote{\fbackref{63:2} So 1QIsa\textsuperscript{a}; or \fbib{coriander} or \fbib{clothing}} \\
\poeml \v{3}``I have trodden the winepress alone, \\
\poemll    and from my people\fnote{\fbackref{63:3} So 1QIsa\textsuperscript{a}; 1QIsa\textsuperscript{b} MT LXX read \fbib{from the peoples}} no one was with me, \\
\poeml I trampled them in my anger \\
\poemll    and trod them down in my wrath; \\
\poeml their lifeblood spattered on my garments,\fnote{\fbackref{63:3} So 1QIsa\textsuperscript{b} MT LXX; 1QIsa\textsuperscript{a} lacks \fbib{I trampled{\ldots}my garment}} \\
\poemll    and I stained\fnote{\fbackref{63:3} So 1QIsa\textsuperscript{a} 1QIsa\textsuperscript{b}; MT verb \fbib{I stained} is problematic.} all my clothing. \\
\poeml \v{4}``For the day of vengeance was in my heart, \\
\poemll    and the year for my redeeming work had come. \\
\poeml \v{5}I looked, but there was no helper, \\
\poemll    I was appalled that there was no one to give support;\fnote{\fbackref{63:5} So 1QIsa\textsuperscript{a}; 1QIsa\textsuperscript{b} MT read \fbib{to support me}} \\
\poeml so my own arm\fnote{\fbackref{63:5} I.e. \fbib{the Messiah}} brought me victory, \\
\poemll    and as for my wrath, it supported me. \\
\poeml \v{6}I trampled people\fnote{\fbackref{63:6} Lit. \fbib{peoples}} in my anger; \\
\poemll    in my wrath I made them drunk \\
\poemlll       and I poured out their lifeblood on the ground.''
\passage{God's Grace to Israel}
\poeml \v{7}I will recount the gracious deeds of the \divine{Lord}, \\
\poemll    the praiseworthy acts of the \divine{Lord}, \\
\poeml according to all the \divine{Lord} has done for us--- \\
\poemll    yes, the great goodness to the house of Israel \\
\poeml that he has granted them according to his mercy, \\
\poemll    according to the abundance of his gracious love. \\
\poeml \v{8}For he said, ``Surely they are my people, \\
\poemll    children who won't act falsely.'' \\
\poemlll       And so he became their savior. \\
\poeml \v{9}In all their distress he wasn't distressed,\fnote{\fbackref{63:9} So 1QIsa\textsuperscript{a} MT; some MT\textsuperscript{mss} read \fbib{he was distressed}.} \\
\poemll    but the angel of his presence saved them; \\
\poeml in his acts of love\fnote{\fbackref{63:9} So 1QIsa\textsuperscript{a}; MT reads \fbib{in his love}; LXX reads \fbib{because of his love for them}} and in his acts\fnote{\fbackref{63:9} So 1QIsa\textsuperscript{a}; MT reads \fbib{act of pity}} of pity he redeemed them; \\
\poemll    he carried them and lifted them up\fnote{\fbackref{63:9} So 1QIsa\textsuperscript{a}; 1QIsa\textsuperscript{b} MT read \fbib{lifted them up and carried them}} all the days of old. \\
\poeml \v{10}Yet they rebelled \\
\poemll    and grieved his Holy Spirit; \\
\poeml so he changed and became their enemy, \\
\poemll    and\fnote{\fbackref{63:10} So 1QIsa\textsuperscript{a} LXX; MT lacks \fbib{and}} he himself fought against them. \\
\poeml \v{11}Then they\fnote{\fbackref{63:11} I.e. \fbib{his people}} remembered the days of old, \\
\poemll    of Moses his servant. \\
\poeml Where is the one who brought up\fnote{\fbackref{63:11} So 1QIsa\textsuperscript{a} LXX; MT reads \fbib{brought them up}} out of the sea \\
\poemll    the\fnote{\fbackref{63:11} So 1QIsa\textsuperscript{a}; MT reads \fbib{with the}} shepherds of his flock? \\
\poeml Where is the one who put his Holy Spirit among them, \\
\poeml \v{12}and\fnote{\fbackref{63:12} So 1QIsa\textsuperscript{a}; MT LXX lack \fbib{and}} who made his glorious arm\fnote{\fbackref{63:12} I.e. the Meessiah; lit. \fbib{arm of his glories}; so 1QIsa\textsuperscript{a}; MT reads \fbib{arm of his glory}} march at Moses' right hand, \\
\poeml who divided the waters in front of them \\
\poemll    to win\fnote{\fbackref{63:12} So 1QIsa\textsuperscript{a}; MT LXX read \fbib{to win for himself}} an everlasting name, \\
\poeml \v{13}who led them through the depths? \\
\poemll    Like a horse in the open desert, \\
\poemlll       they did not stumble; \\
\poeml \v{14}like cattle that go down into the valley, \\
\poemll    the Spirit of the \divine{Lord} gave them rest. \\
\poeml For\fnote{\fbackref{63:14} So 1QIsa\textsuperscript{a}; MT LXX read \fbib{This is how}} you led your people, \\
\poemll    to win for yourself a glorious name.
\passage{God the Father}
\poeml \v{15}Look down from heaven, and see \\
\poemll    from your holy and glorious dwelling. \\
\poeml Where are your zeal and your might? \\
\poemll    Where are the yearning of your heart and your compassion? \\
\poemlll       They are held back from me. \\
\poeml \v{16}But you are our Father, \\
\poemll    even\fnote{\fbackref{63:16} Lit. \fbib{and}; so 1QIsa\textsuperscript{a}; MT LXX read \fbib{but} or \fbib{although}} Abraham does not know us \\
\poemlll       and Israel has not acknowledged\fnote{\fbackref{63:16} So 1QIsa\textsuperscript{a} LXX; MT reads \fbib{Israel does not acknowledge}} us; \\
\poeml you are he,\fnote{\fbackref{63:16} So 1QIsa\textsuperscript{a}; 1QIsa\textsuperscript{b} MT LXX read \fbib{you}} O \divine{Lord}, our Father, \\
\poemll    from long ago, `Our Redeemer' is your name. \\
\poeml \v{17}Why, \divine{Lord}, do you make us wander\fnote{\fbackref{63:17} So 1QIsa\textsuperscript{a}; MT LXX read \fbib{do you make us wander, \divine{Lord}}} from your ways \\
\poemll    and harden our hearts, so that we do not fear you? \\
\poeml Turn back for the sake of your servants, \\
\poemll    for the sake of the tribes that are your heritage. \\
\poeml \v{18}Your holy people took possession\fnote{\fbackref{63:18} So 1QIsa\textsuperscript{a} (sing.); MT is pl.; LXX reads \fbib{so that we may take possession}} for a little while, \\
\poemll    but now our enemies have trampled down your sanctuary. \\
\poeml \v{19}For a long time we have been those you do not rule, \\
\poemll    those who are not called by your name.
\end{poetry}
\labelchapt{64}
\passage{A Prayer for God to Intervene}

\begin{poetry}
\poeml \chapt{64}
\v{1}\fnote{\fbackref{64:1} This v. is 63:19 in the MT}If only you would tear open the heavens and\fnote{\fbackref{64:1} So 1QIsa\textsuperscript{a}; MT lacks \fbib{and}} come down, \\
\poemll    so that the mountains would quake at your presence--- \\
\poeml \v{2}\fnote{\fbackref{64:2} This v. is 64:1 in the MT}just as fire sets twigs\fnote{\fbackref{64:2} Or \fbib{brushwood}} ablaze \\
\poemll    and the fire causes water to boil--- \\
\poeml to make known your name to your enemies, \\
\poemll    yes, to your enemies before you,\fnote{\fbackref{64:2} So 1QIsa\textsuperscript{a} LXX; MT reads \fbib{to make your name known to your adversaries}} \\
\poemlll       so that the nations might quake at your presence! \\
\poeml \v{3}When you did awesome deeds that we expected,\fnote{\fbackref{64:3} So 1QIsa\textsuperscript{a}; MT reads \fbib{did not expect}} \\
\poemll    you came down, \\
\poemlll       and the mountains shuddered before you. \\
\poeml \v{4}Since\fnote{\fbackref{64:4} So 1QIsa\textsuperscript{a} LXX; MT reads \fbib{And since}} ancient times no one has heard, \\
\poemll    and\fnote{\fbackref{64:4} So 1QIsa\textsuperscript{a}; MT lacks \fbib{and}} no ear has perceived, \\
\poeml and\fnote{\fbackref{64:4} So 1QIsa\textsuperscript{a} LXX; MT lacks \fbib{and}} no eye has seen any God besides you, \\
\poemll    who acts on behalf of those who wait for him. \\
\poeml \v{5}You come to the aid of those who gladly do what's right, \\
\poemll    To those who remember you in your ways. \\
\poeml See, you were angry, \\
\poemll    and we sinned against them for a long time, \\
\poemlll       but we will be saved. \\
\poeml \v{6}All of us have become like one who is unclean, \\
\poemll    and\fnote{\fbackref{64:6} So 1QIsa\textsuperscript{a} 4QIsa\textsuperscript{b} LXX; MT lacks \fbib{and}} all our righteous acts are like a filthy rag; \\
\poeml we all shrivel up like a leaf, \\
\poemll    and like the wind, our iniquities\fnote{\fbackref{64:6} The 1QIsa\textsuperscript{a} utilizes a masculine noun; 1QIsa\textsuperscript{b} MT utilize a feminine noun} sweep us away. \\
\poeml \v{7}There is no one who calls on your name \\
\poemll    or rouses himself to take hold of you; \\
\poeml for you have hidden your face from us, \\
\poemll    and have given us\fnote{\fbackref{64:7} So 1QIsa\textsuperscript{a}; MT reads \fbib{have melted us}; LXX reads \fbib{have delivered us}} into the control\fnote{\fbackref{64:7} Lit. \fbib{hand}} of our iniquity.
\passage{God, our Father, will Act}
\poeml \v{8}But as for you,\fnote{\fbackref{64:8} So 1QIsa\textsuperscript{a}; MT LXX read \fbib{But now}} O \divine{Lord}, you are our Father; \\
\poemll    and\fnote{\fbackref{64:8} So 1QIsa\textsuperscript{a} LXX; MT lacks \fbib{and}} we are clay,\fnote{\fbackref{64:8} So 1QIsa\textsuperscript{a} LXX; 1QIsa\textsuperscript{b} MT read \fbib{the clay}} \\
\poeml and you are our potter; \\
\poemll    we are all the work of your hands.\fnote{\fbackref{64:8} So 1QIsa\textsuperscript{a} LXX; MT reads \fbib{your hand}} \\
\poeml \v{9}Don't be angry beyond measure, \divine{Lord}, \\
\poemll    and don't remember our iniquity for a season.\fnote{\fbackref{64:9} So 1QIsa\textsuperscript{a} LXX; MT reads \fbib{for ever}} \\
\poemlll       Please look now, we are all your people. \\
\poeml \v{10}Your holy cities have become a desert; \\
\poemll    Zion has become like\fnote{\fbackref{64:10} So 1QIsa\textsuperscript{a} LXX; 1QIsa\textsuperscript{b} MT lack \fbib{like}} a desert, \\
\poemlll       Jerusalem a desolation. \\
\poeml \v{11}Our holy Temple and our splendor, \\
\poemll    where our ancestors praised you, \\
\poeml have become\fnote{\fbackref{64:11} So 1QIsa\textsuperscript{a}; MT LXX read \fbib{Our holy and glorious Temple {\ldots} has become}} a conflagration of fire, \\
\poemll    and all our dearest places have become\fnote{\fbackref{64:11} So 1QIsa\textsuperscript{a} LXX; MT reads \fbib{all our dearest places has become}; MT\textsuperscript{mss} read \fbib{our every dearest place has become}} ruins. \\
\poeml \v{12}\divine{Lord}, after all this, can you hold yourself back? \\
\poemll    Can you keep silent and punish us so severely?
\end{poetry}
\labelchapt{65}
\passage{God's Response}

\begin{poetry}
\poeml \chapt{65}
\v{1}``I let myself be sought by those who didn't ask for me;\fnote{\fbackref{65:1} So 1QIsa\textsuperscript{a} LXX; MT reads \fbib{ask}} \\
\poemll    I let myself be found by those who didn't seek me. \\
\poeml I said, `Here I am! Here I am!' \\
\poemll    to a nation that didn't call on my name. \\
\poeml \v{2}I held out my hands all day long \\
\poemll    to a disobedient\fnote{\fbackref{65:2} So 1QIsa\textsuperscript{a}; MT reads \fbib{an obstinate}} people, \\
\poeml who walk in a way that isn't good, \\
\poemll    following their own inclinations--- \\
\poeml \v{3}a people who continually provoke me to my face; \\
\poemll    they\fnote{\fbackref{65:3} So 1QIsa\textsuperscript{a} LXX; MT reads \fbib{who}} keep sacrificing in gardens \\
\poemlll       and waving their hands\fnote{\fbackref{65:3} So 1QIsa\textsuperscript{a}; MT LXX read \fbib{and offering incense}} over stone\fnote{\fbackref{65:3} So 1QIsa\textsuperscript{a}; MT LXX read \fbib{brick}} altars; \\
\poeml \v{4}who sit among graves, \\
\poemll    and spend the night in secret places; \\
\poeml who eat pigs' meat, \\
\poemll    with the broth\fnote{\fbackref{65:4} So 1QIsa\textsuperscript{a} MT\textsuperscript{qere} LXX Targ Vulg; MT reads \fbib{violence}; or \fbib{crumbs}} of detestable things in\fnote{\fbackref{65:4} So 1QIsa\textsuperscript{a}; MT LXX lack \fbib{in}} their pots; \\
\poeml \v{5}who say, `Keep to yourself!' \\
\poemll    `Don't touch\fnote{\fbackref{65:5} So 1QIsa\textsuperscript{a}; MT LXX read \fbib{come near to}} me!' and `I am\fnote{\fbackref{65:5} So 1QIsa\textsuperscript{a}; MT LXX read \fbib{for I am}} too holy for you!' \\
\poeml ``Such people are smoke in my nostrils, \\
\poemll    a fire that keeps burning all day long. \\
\poeml \v{6}Watch out! It stands written before me: \\
\poemll    `I won't keep silent, but I will pay back in full; \\
\poemlll       I'll indeed repay into\fnote{\fbackref{65:6} So 1QIsa\textsuperscript{a}; MT LXX read \fbib{upon}} their laps \\
\poeml \v{7}both your iniquities and your ancestors'\fnote{\fbackref{65:7} So 1QIsa\textsuperscript{a} MT; LXX Syr read \fbib{both their iniquities and their}} iniquities together,'\,' \\
\poemll    says the \divine{Lord}. \\
\poeml ``Because they offered incense on the mountains \\
\poemll    and insulted me on hills,\fnote{\fbackref{65:7} So 1QIsa\textsuperscript{a}; MT LXX read \fbib{the hills}} \\
\poeml I'll measure into\fnote{\fbackref{65:7} So 1QIsa\textsuperscript{a} MT\textsuperscript{qere}; MT LXX read \fbib{upon}} their laps \\
\poemll    full payment for their earlier actions.''
\passage{A Remnant will be Preserved}
\poeml \v{8}This is what the \divine{Lord} says: \\
\poemll    ``Just as new wine is found in the cluster, \\
\poeml and people have said,\fnote{\fbackref{65:8} So 1QIsa\textsuperscript{a}; MT LXX read \fbib{say}} `Don't destroy it, \\
\poemll    for there is a gift in it,' \\
\poeml so I'll act for my servants' sake, \\
\poemll    by not destroying them all. \\
\poeml \v{9}I'll bring forth descendants from Jacob, \\
\poemll    and from Judah they\fnote{\fbackref{65:9} Lit. \fbib{he}; so 1QIsa\textsuperscript{a} LXX} will inherit\fnote{\fbackref{65:9} So 1QIsa\textsuperscript{a} LXX; MT reads \fbib{Judah, the one about to inherit}} my mountains; \\
\poeml my chosen people will inherit it, \\
\poemll    and my servants will live there. \\
\poeml \v{10}Sharon will become a pasture for flocks, \\
\poemll    and the Valley of Achor a fold for herds,\fnote{\fbackref{65:10} So 1QIsa\textsuperscript{a} LXX; MT reads \fbib{a place for herds to lie down}} \\
\poemlll       for my people who have sought me. \\
\poeml \v{11}But as for you who forsake the \divine{Lord}, \\
\poemll    who forget my holy mountain, \\
\poeml who spread a table for Fortune\fnote{\fbackref{65:11} I.e. Fortune personified as a god} \\
\poemll    and\fnote{\fbackref{65:11} So 1QIsa\textsuperscript{a}; MT LXX reads \fbib{and who}} fill drink offerings\fnote{\fbackref{65:11} So 1QIsa\textsuperscript{a}; MT LXX read \fbib{cups of mixed wine}, or \fbib{mixing vessels}} for Destiny,\fnote{\fbackref{65:11} I.e. Destiny personified as a god} \\
\poeml \v{12}I'll consign\fnote{\fbackref{65:12} Lit. \fbib{destine}} you to the sword, \\
\poemll    and all of you will bend down for the slaughter--- \\
\poeml because when I called, you didn't answer, \\
\poemll    when I spoke, you didn't listen; \\
\poeml but you did what was evil in my sight, \\
\poemll    and chose what I took no pleasure in.''
\passage{The Righteous and Wicked Contrasted}
\poeml \v{13}Therefore, this is what the \divine{Lord}\fnote{\fbackref{65:13} So 1QIsa\textsuperscript{a} LXX; 1QIsa\textsuperscript{a} reads \fbib{Adonai}; MT reads \fbib{Lord \divine{God}}} says: \\
\poeml ``See, my servants will eat, \\
\poemll    but you'll go hungry; \\
\poeml my servants will drink, \\
\poemll    but you'll go thirsty; \\
\poeml my servants will rejoice, \\
\poemll    but you'll be put to shame. \\
\poeml \v{14}My servants will sing in gladness\fnote{\fbackref{65:14} So 1QIsa\textsuperscript{a} LXX; MT reads \fbib{out of gladness}} of heart, \\
\poemll    but you'll cry for help\fnote{\fbackref{65:14} So 1QIsa\textsuperscript{a}; MT LXX read \fbib{cry out}} from anguish of heart, \\
\poemlll       and you'll howl from brokenness of spirit. \\
\poeml \v{15}You'll leave your name to my chosen ones as a curse, \\
\poemll    and the Lord \divine{God} will put you to death permanently.\fnote{\fbackref{65:15} Or \fbib{for good}; so 1QIsa\textsuperscript{a}; MT LXX reads \fbib{but he will call his servants by a different name}} \\
\poeml \v{16}Then whoever takes an oath\fnote{\fbackref{65:16} So 1QIsa\textsuperscript{a}; MT LXX read \fbib{whoever invokes a blessing in the land will bless}} by the God of faithfulness, \\
\poemll    and whoever takes an oath in the land, \\
\poemlll       will swear by the God of faithfulness, \\
\poeml because the former troubles are forgotten \\
\poemll    and are hidden from my eyes.
\passage{A New Universe}
\poeml \v{17}Take notice! I'm about to create new heavens \\
\poemll    and a new earth; \\
\poeml the former things won't be remembered, \\
\poemll    nor will they come to mind. \\
\poeml \v{18}But be glad\fnote{\fbackref{65:18} Sing. 1QIsa\textsuperscript{a}; pl. MT} and rejoice\fnote{\fbackref{65:18} Sing. 1QIsa\textsuperscript{a}; pl. 1QIsa\textsuperscript{b} MT} forever \\
\poemll    in what I am creating, \\
\poeml for I am about to create Jerusalem as a joy, \\
\poemll    and its people as a delight. \\
\poeml \v{19}I'll rejoice over Jerusalem, \\
\poemll    and take delight in my people; \\
\poeml no longer will the sound of weeping be heard in it, \\
\poemll    nor the cry of distress. \\
\poeml \v{20}``And\fnote{\fbackref{65:20} So 1QIsa\textsuperscript{a} LXX; 1QIsa\textsuperscript{b} MT lack \fbib{And}} there will no longer be in it \\
\poemll    a young boy\fnote{\fbackref{65:20} So 1QIsa\textsuperscript{a}; 1QIsa\textsuperscript{b} MT read \fbib{an infant}; cf. 49:15} who lives only a few days, \\
\poemlll       or an old person who does not live out his days; \\
\poeml for one who dies at a hundred years will be thought a mere youth, \\
\poemll    and one who falls short of a hundred years will be considered accursed. \\
\poeml \v{21}People\fnote{\fbackref{65:21} Lit. \fbib{They}} will build houses and live in them; \\
\poemll    They'll plant vineyards and eat their fruit. \\
\poeml \v{22}They won't build for others to inhabit; \\
\poemll    they won't plant for others to eat--- \\
\poeml for like the lifetime\fnote{\fbackref{65:22} Lit. \fbib{days}} of a tree,\fnote{\fbackref{65:22} So 1QIsa\textsuperscript{a}; MT LXX read \fbib{the tree}} so will the lifetime\fnote{\fbackref{65:22} Lit. \fbib{days}} of my people be, \\
\poemll    and my chosen ones will long enjoy\fnote{\fbackref{65:22} Lit. \fbib{consume} or \fbib{wear out}} the work of their hands. \\
\poeml \v{23}They won't toil in vain \\
\poemll    nor bear children doomed to misfortune, \\
\poeml for they will be offspring blessed\fnote{\fbackref{65:23} Sing. 1QIsa\textsuperscript{a} LXX; pl. 1QIsa\textsuperscript{b} MT} by the \divine{Lord}, \\
\poemll    they and their descendants with them. \\
\poeml \v{24}Before they call, I will answer, \\
\poemll    while they are still speaking, I'll hear. \\
\poeml \v{25}``The wolf and the lamb will feed together, \\
\poemll    and the lion will eat straw like the ox; \\
\poeml but as for the serpent--- \\
\poemll    its food will be dust! \\
\poeml They won't harm or destroy \\
\poemll    on my entire holy mountain,''
\end{poetry}

says the \divine{Lord}.
\labelchapt{66}
\passage{The Worship that God Commands}

\chapt{66}
\v{1}This is what the \divine{Lord} says:

\begin{poetry}
\poeml ``Heaven is my throne, \\
\poemll    and the earth is my footstool. \\
\poeml Where is the house\fnote{\fbackref{66:1} I.e. a reconstructed Temple} that you would build for me, \\
\poemll    and where will my resting place be? \\
\poeml \v{2}All these things my hand has made, \\
\poemll    and so all these things came into being,'' \\
\poemlll       declares the \divine{Lord}. \\
\poeml ``But this is the one to whom I will look favorably: \\
\poemll    to the one who is humble and contrite in spirit, \\
\poemlll       and who\fnote{\fbackref{66:2} So 1QIsa\textsuperscript{a}; 1QIsa\textsuperscript{b} MT lack \fbib{who}} trembles at\fnote{\fbackref{66:2} So 1QIsa\textsuperscript{a}; 1QIsa\textsuperscript{b} MT use different Heb. prepositions} my message. \\
\poeml \v{3}``Whoever slaughters an ox \\
\poemll    is just like\fnote{\fbackref{66:3} So 1QIsa\textsuperscript{a} LXX; 1QIsa\textsuperscript{b} MT lack \fbib{just like}} one who kills a human being; \\
\poeml whoever sacrifices a lamb \\
\poemll    is just like one who breaks a dog's neck; \\
\poeml whoever makes a grain offering \\
\poemll    is just like one who offers pig's blood; \\
\poeml and whoever makes a memorial offering of frankincense \\
\poemll    is just like one who blesses an idol. \\
\poeml Yes, these have chosen their own ways, \\
\poemll    and they take delight in their contaminated actions. \\
\poeml \v{4}Therefore\fnote{\fbackref{66:4} Or \fbib{So}} I, too, will choose harsh treatment for them, \\
\poemll    and will bring upon them what they dread. \\
\poeml For when I called, no\fnote{\fbackref{66:4} So 1QIsa\textsuperscript{a}; MT LXX read \fbib{and no}} one answered; \\
\poemll    when I spoke, they didn't listen; \\
\poeml but they did what I consider to be evil, \\
\poemll    and chose what doesn't please me.''
\passage{The \divine{Lord} Vindicates Zion}
\poeml \v{5}``Hear this message from the \divine{Lord}, \\
\poemll    you who tremble at his words:\fnote{\fbackref{66:5} So 1QIsa\textsuperscript{a}; MT LXX read \fbib{his word}} \\
\poeml ``Your own brothers who hate you \\
\poemll    and exclude you because of my name \\
\poeml have said: `Let the \divine{Lord} be glorified; \\
\poemll    he will see\fnote{\fbackref{66:5} So 1QIsa\textsuperscript{a}; MT reads \fbib{so that we may see}; LXX reads \fbib{so that the name of the Lord may be glorified}} your joy,' \\
\poemlll       yet it is they who will be put to shame. \\
\poeml \v{6}``Listen to that uproar in\fnote{\fbackref{66:6} So 1QIsa\textsuperscript{a}; MT LXX read \fbib{from}} the city! \\
\poemll    Listen to that noise from the Temple! \\
\poeml It is the sound of the \divine{Lord} \\
\poemll    paying back retribution to his enemies! \\
\poeml \v{7}``Before she goes into labor she gives birth; \\
\poemll    before her pains come upon her \\
\poemlll       she has delivered\fnote{\fbackref{66:7} So 1QIsa\textsuperscript{a}; 1QIsa\textsuperscript{b} MT LXX read \fbib{and she has delivered}} a son. \\
\poeml \v{8}Who has ever heard of such a thing? \\
\poemll    And\fnote{\fbackref{66:8} So 1QIsa\textsuperscript{a} LXX; 1QIsa\textsuperscript{b} MT lack \fbib{And}} who ever sees\fnote{\fbackref{66:8} So 1QIsa\textsuperscript{a}. 1QIsa\textsuperscript{b} MT LXX read \fbib{has seen}} such things? \\
\poeml Can a country be born in a single day, \\
\poemll    or can a nation be brought forth in a single moment? \\
\poeml Yet no sooner was Zion in labor \\
\poemll    than she delivered her children. \\
\poeml \v{9}Am I to open the womb and not deliver?'' \\
\poemll    asks\fnote{\fbackref{66:9} Lit. \fbib{says}} the \divine{Lord}. \\
\poeml ``And when I bring to delivery, am I to close\fnote{\fbackref{66:9} So 1QIsa\textsuperscript{a} (imperfect); 1QIsa\textsuperscript{b} MT (perfect)} the womb?'' \\
\poemll    asks your God. \\
\poeml \v{10}``Rejoice with Jerusalem, and be happy for her, \\
\poemll    all you who love her; \\
\poeml rejoice with her in gladness, \\
\poemll    all you who mourn over her, \\
\poeml \v{11}so that you may nurse and be satisfied \\
\poemll    at her consoling breasts, \\
\poeml and so that you may drink deeply and take delight \\
\poemll    from her glorious bosom.''
\passage{The Rule of God}
\poeml \v{12}This\fnote{\fbackref{66:12} So 1QIsa\textsuperscript{a}; MT LXX read \fbib{For this}} is what the \divine{Lord} says: \\
\poeml ``See, I will extend prosperity to her like a river, \\
\poemll    and the wealth of nations like a flooding stream; \\
\poeml and you will nurse, \\
\poemll    and you\fnote{\fbackref{66:12} 1QIsa\textsuperscript{a} feminine pl.; MT masculine pl.} will be carried on her hip,\fnote{\fbackref{66:12} Or \fbib{arm}} \\
\poemlll       and bounced upon her knees. \\
\poeml \v{13}Like a child whom his mother comforts, \\
\poemll    so I will comfort you; \\
\poemlll       and you will be comforted in Jerusalem. \\
\poeml \v{14}And when you look, your hearts will rejoice \\
\poemll    and your bodies will flourish like grass; \\
\poeml and it will be made known \\
\poemll    that the \divine{Lord}'s hand is with his servants, \\
\poemlll       but his fury is with his enemies. \\
\poeml \v{15}``Take notice! The \divine{Lord} will come with fire, \\
\poemll    and his chariot\fnote{\fbackref{66:15} So 1QIsa\textsuperscript{a}; MT LXX read \fbib{his chariots}} will be\fnote{\fbackref{66:15} 1QIsa\textsuperscript{a} MT LXX lack \fbib{will be}} like a whirlwind, \\
\poeml to pay back his anger---yes, his anger!---\fnote{\fbackref{66:15} So 1QIsa\textsuperscript{a}; 1QIsa\textsuperscript{b} MT LXX lack \fbib{yes, his anger!}} in fury, \\
\poemll    and his menacing rebukes\fnote{\fbackref{66:15} So 1QIsa\textsuperscript{a}; 1QIsa\textsuperscript{b} MT read \fbib{his rebuke}} in flames of fire. \\
\poeml \v{16}For with fire and with his sword the \divine{Lord} will proceed to judgment\fnote{\fbackref{66:16} So 1QIsa\textsuperscript{a}; 1QIsa\textsuperscript{b} MT read \fbib{settle his claim}} on all humanity,\fnote{\fbackref{66:16} Lit. \fbib{on the humanity}; so 1QIsa\textsuperscript{a}; MT reads \fbib{on humanity}} \\
\poemll    and those slain by the \divine{Lord} will be many.''
\end{poetry}

\v{17}``Those who consecrate and purify themselves to enter the groves,\fnote{\fbackref{66:17} I.e. pagan sacred worship sites located in forested areas} following the one at the center of those who eat the meat of pigs, disgusting things,\fnote{\fbackref{66:17} Or \fbib{vermin}} and rats, are all alike,''\fnote{\fbackref{66:17} So 1QIsa\textsuperscript{a}; 1QIsa\textsuperscript{b} MT LXX read \fbib{alike}---\fbib{they will meet their end together}} says\fnote{\fbackref{66:17} So 1QIsa\textsuperscript{a}; MT reads \fbib{declares}} the \divine{Lord}. \v{18}``But as for me, I know their actions and their thoughts. Come\fnote{\fbackref{66:18} So IQIsa\textsuperscript{a}; MT LXX read \fbib{I am about to come}} and gather all nations and languages, and they will come and see my glory.

\v{19}``I will put up signs\fnote{\fbackref{66:19} So IQIsa\textsuperscript{a} LXX; 1QIsa\textsuperscript{b} MT read \fbib{a sign}} among them, and from them I will send survivors to the nations---to Tarshish, Libya,\fnote{\fbackref{66:19} Lit. \fbib{Put}} and Lydia,\fnote{\fbackref{66:19} Lit. \fbib{Lud}} (who draw the bow),\fnote{\fbackref{66:19} The Lydians were known for their skills at archery} to Tubal and Greece,\fnote{\fbackref{66:19} Lit. \fbib{Javan}} to the far off coastlands that have not heard of my fame or seen my glory. Then they will proclaim my glory among the nations. \v{20}They will bring all---yes, all!---\fnote{\fbackref{66:20} So 1QIsa\textsuperscript{a}; 1QIsa\textsuperscript{b} MT read \fbib{bring all}; LXX reads \fbib{bring}} of your kindred from all the nations to\fnote{\fbackref{66:20} So 1QIsa\textsuperscript{a} LXX; MT reads \fbib{upon}} my holy mountain Jerusalem as an offering to the \divine{Lord}---on horses, in chariots, in wagons, and on mules---yes, even on mules!---\fnote{\fbackref{66:20} So 1QIsa\textsuperscript{a}; 1QIsa\textsuperscript{b} MT LXX lack \fbib{yes, even on mules!}} and on camels,'' says the \divine{Lord}, ``just as the Israelis bring a grain offering in a clean vessel to the \divine{Lord}'s house. \v{21}Then I will also select some of them for myself\fnote{\fbackref{66:21} So 1QIsa\textsuperscript{a} LXX; 1QIsa\textsuperscript{b} MT LXX lack \fbib{for myself}} as priests and as Levites,''\fnote{\fbackref{66:21} I.e. the ministry formerly held by the descendants of Levi} says the \divine{Lord}.

\begin{poetry}
\poeml \v{22}``For as the new heavens and the new earth \\
\poemll    that I am about to make \\
\poeml will endure before me,'' says the \divine{Lord}, \\
\poemll    ``so will your descendants and your name endure. \\
\poeml \v{23}And from New Moon to New Moon, \\
\poemll    and from Sabbath to Sabbath,\fnote{\fbackref{66:23} So 1QIsa\textsuperscript{a} 4QIsa\textsuperscript{c}; lit. \fbib{to her Sabbath}; MT reads \fbib{to his Sabbath}} \\
\poeml all\fnote{\fbackref{66:23} Lit. \fbib{the}; so 1QIsa\textsuperscript{a}; 4QIsa\textsuperscript{b} MT LXX lack \fbib{all}} humanity will come to worship before me,'' \\
\poemll    says the \divine{Lord}.
\end{poetry}

\v{24}``Then they will go out and look upon the dead bodies of the people who rebelled against me. For their worm will not die, nor will their fire be extinguished, and they will remain an object of revulsion to all\fnote{\fbackref{66:24} Lit. \fbib{the}; so 1QIsa\textsuperscript{a}; 4QIsa\textsuperscript{b} MT LXX lack \fbib{all}} humanity.''
