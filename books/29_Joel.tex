\bookheader{Joel}
\labelbook{Joel}

\bookpretitle{The Book of the Prophet}
\booktitle{Joel}

\labelchapt{1}
\passage{The Coming Invasion}

\chapt{1}
\v{1}This message from the \divine{Lord} came to Pethuel's son Joel.\fnote{\fbackref{1:1} The Heb. name \fbib{Joel} means \fbib{The \divine{Lord} is God}}

\begin{poetry}
\poeml \v{2}``Hear this, you elders! \\
\poeml Listen, all of you residents of the land! \\
\poeml Has there ever been anything like this during your lifetime,\fnote{\fbackref{1:2} Lit. \fbib{this in your days}} \\
\poemll    or even when your ancestors were alive?\fnote{\fbackref{1:2} Lit. \fbib{even in the days of your ancestors}} \\
\poeml \v{3}Pass it on to your children, \\
\poemll    and from\fnote{\fbackref{1:3} The Heb. lacks \fbib{from}} your children to their children, \\
\poemlll       and from\fnote{\fbackref{1:3} The Heb. lacks \fbib{from}} their children to the following generation. \\
\poeml \v{4}Whatever the devouring locust left behind \\
\poemll    the locust swarm has consumed! \\
\poeml Whatever the locust swarm has left behind, \\
\poemll    the young locust\fnote{\fbackref{1:4} Or \fbib{caterpillar}} has consumed! \\
\poeml Whatever the young locust\fnote{\fbackref{1:4} Or \fbib{caterpillar}} has left behind, \\
\poemll    the ravaging locust has consumed!''
\passage{A Call to Mourning}
\poeml \v{5}``Wake up, you drunkards! \\
\poemll    Cry aloud and howl, you wine drinkers, \\
\poemlll       because your supply of new wine has been snatched from you.\fnote{\fbackref{1:5} Lit. \fbib{from your lips}} \\
\poeml \v{6}Indeed, a nation has invaded my land--- \\
\poemll    it is strong and its population is too large to count\fnote{\fbackref{1:6} Lit. \fbib{and innumerable}}--- \\
\poeml with teeth like a lion \\
\poemll    and fangs\fnote{\fbackref{1:6} Or \fbib{jaws}} like a lioness. \\
\poeml \v{7}That nation\fnote{\fbackref{1:7} Lit. \fbib{It}} laid waste my vines, \\
\poemll    and stripped bare my fig tree, \\
\poemlll       discarding it. \\
\poeml It stripped off\fnote{\fbackref{1:7} Lit. \fbib{made white}} its bark. \\
\poeml \v{8}``Grieve like a virgin, \\
\poemll    who, dressed in her mourner's clothes,\fnote{\fbackref{1:8} Or \fbib{in sackcloth}} \\
\poemlll       cries out in memory\fnote{\fbackref{1:8} The Heb. lacks \fbib{in memory}} of the man she was going to marry.\fnote{\fbackref{1:8} Lit. \fbib{the husband of her youth}} \\
\poeml \v{9}Both grain offering and wine offering have been removed from the \divine{Lord}'s Temple;\fnote{\fbackref{1:9} Or \fbib{house}; and so throughout the book} \\
\poemll    the priests and ministering servants of the \divine{Lord} are mourning.''
\passage{The Coming Famine}
\poeml \v{10}``The fields lie in ruins \\
\poemll    and the ground is dried up.\fnote{\fbackref{1:10} Or \fbib{ground mourns}} \\
\poeml Indeed, the grain is ruined, \\
\poemll    the new wine has evaporated, \\
\poemlll       and the olive oil has run out. \\
\poeml \v{11}Be dismayed, you farmers! \\
\poemll    Cry aloud, you vintners, \\
\poemlll       for the wheat and barley, \\
\poemll    because the harvest in your fields has been lost. \\
\poeml \v{12}The grapevine is shriveled \\
\poemll    and the fig tree is withered, \\
\poeml along with the pomegranate tree, the palm tree, the apple tree \\
\poemll    and all of the cultivated trees.\fnote{\fbackref{1:12} Lit. \fbib{the trees of the field}} \\
\poeml Truly, joy has evaporated from Adam's children.''\fnote{\fbackref{1:12} Lit. \fbib{from sons of mankind}}
\passage{A Call to Mourn and Repent}
\poeml \v{13}``Put on your mourning clothes, you priests; \\
\poemll    and cry aloud, you ministering servants at the altar! \\
\poeml Come! Stay the night in mourner's clothes,\fnote{\fbackref{1:13} Or \fbib{in sackcloth}} you ministers of my God, \\
\poemll    because the grain offering and the wine offering is held back from the Temple of your God. \\
\poeml \v{14}Set apart time for a fast! \\
\poemll    Call a solemn assembly! \\
\poeml Gather the elders and everyone living in the land to the Temple of the \divine{Lord} your God, \\
\poemll    and cry out to the \divine{Lord}!''
\passage{A Lament about the Day of the \divine{Lord}}
\poeml \v{15}Oh, no! For the Day of the \divine{Lord} approaches, \\
\poemll    and like destruction from the Almighty, it will come! \\
\poeml \v{16}Isn't our food supply cut off right in front of us,\fnote{\fbackref{1:16} Lit. \fbib{cut off before our eyes}} \\
\poemll    along with joy and gladness from the Temple of our God? \\
\poeml \v{17}Seeds shrivel within their furrows, \\
\poemll    the storehouses lie empty, \\
\poeml and granaries stand in ruins \\
\poemll    because the grain has withered. \\
\poeml \v{18}Oh, how the livestock groan! \\
\poemll    The herds of cattle\fnote{\fbackref{1:18} Or \fbib{oxen}} wander about \\
\poemlll       because they have no pasture. \\
\poeml Even flocks of sheep suffer! \\
\poeml \v{19}To you, \divine{Lord}, I cry out, \\
\poemll    because fire has devoured the open pastures, \\
\poemlll       and has set all the cultivated trees\fnote{\fbackref{1:19} Lit. \fbib{the trees of the field}} ablaze. \\
\poeml \v{20}The livestock also cries out to you, \\
\poemll    because their water sources have evaporated \\
\poemlll       and because fire has consumed the open pastures.
\end{poetry}
\labelchapt{2}
\passage{The Warning of God}

\begin{poetry}
\poeml \chapt{2}
\v{1}``Sound the ram's horn in Zion! \\
\poeml Sound an alarm on my holy mountain! \\
\poeml Tremble, all of you\fnote{\fbackref{2:1} The Heb. lacks \fbib{of you}} inhabitants of the land, \\
\poeml because the Day of the \divine{Lord} is coming. \\
\poemlll       Oh, how near it is! \\
\poeml \v{2}A day of doom and gloom, \\
\poemll    a day of clouds and shadows\fnote{\fbackref{2:2} Cf. Zeph 1:15b} \\
\poeml like the dawn spreads out to cover the mountains--- \\
\poemll    a people strong and robust. \\
\poeml Never has there been anything like it, \\
\poemll    neither will anything follow to compare with\fnote{\fbackref{2:2} The Heb. lacks \fbib{to compare with}} it, \\
\poemlll       even through the lifetime of generation upon generation.''\fnote{\fbackref{2:2} Lit. \fbib{the years of generation and generation}}
\passage{Joel's Description of the Approaching Army}
\poeml \v{3}``A fire blazes in their presence, \\
\poemll    and behind them a conflagration rages. \\
\poeml Before they come, the land is like the garden in Eden; \\
\poemll    after they leave, there is only a barren wasteland. \\
\poemlll       Indeed, nothing escapes them. \\
\poeml \v{4}As to their form, they're like horses; \\
\poemll    and like chariot horses, how they can\fnote{\fbackref{2:4} The Heb. lacks \fbib{can}} run! \\
\poeml \v{5}They leap like the rumbling of chariots echoing from mountain tops, \\
\poemll    like the roar of wild fire that devours the chaff, \\
\poemlll       as an army\fnote{\fbackref{2:5} Lit. \fbib{people}} firmly established in battle array. \\
\poeml \v{6}The people are terrified in their presence; \\
\poemll    every face grows pale.\fnote{\fbackref{2:6} Lit. \fbib{gathers blackness}; cf. Nah 2:10b} \\
\poeml \v{7}They run like elite soldiers, \\
\poemll    climbing ramparts like men trained for war. \\
\poeml Each man advances in proper order, \\
\poemll    never breaking rank. \\
\poeml \v{8}Neither does a man crowd his fellow soldier;\fnote{\fbackref{2:8} Lit. \fbib{his brother}} \\
\poemll    each one marches in his own path. \\
\poeml When they fall by the sword \\
\poemll    they are not injured. \\
\poeml \v{9}They swarm through the city, \\
\poemll    running upon its ramparts. \\
\poeml Climbing atop the houses, \\
\poemll    they enter through windows like a thief.''
\passage{Great is the Day of the \divine{Lord}}
\poeml \v{10}``The land quivers in their presence; \\
\poemll    even the heavens shake. \\
\poeml The sun and moon will grow dark, \\
\poemll    and the stars will stop shining. \\
\poeml \v{11}The \divine{Lord} will shout in the presence of his forces, \\
\poemll    because his encampment is very great; \\
\poeml for powerful is he who carries out his message. \\
\poemll    Truly the Day of the \divine{Lord} is great, and very terrifying. \\
\poemlll       Who will be able to survive\fnote{\fbackref{2:11} Or \fbib{comprehend}} it?''
\passage{Repentance and Restoration}
\poeml \v{12}``Yet even now,'' declares the \divine{Lord}, \\
\poemll    ``Turn back to me with your whole heart, \\
\poemlll       with fasting, tears, and mourning. \\
\poeml \v{13}Tear your hearts, not your garments;\fnote{\fbackref{2:13} An allusion to Heb. custom of tearing the outer clothing in response to mourning} \\
\poemll    and turn back to the \divine{Lord} your God. \\
\poeml For he is gracious and compassionate, \\
\poemll    slow to become angry, \\
\poeml overflowing in gracious love, \\
\poemll    and grieves about this evil. \\
\poeml \v{14}Who knows? He will turn back and relent, will he not, \\
\poemll    leaving behind a blessing, \\
\poemlll       even a grain offering and drink offering for the \divine{Lord} your God?''
\passage{A Public Call to a Solemn Assembly}
\poeml \v{15}``Sound the ram's horn in Zion! \\
\poemll    Dedicate a fast and call for a solemn assembly! \\
\poeml \v{16}Gather the people! \\
\poemll    Dedicate the congregation! \\
\poeml Bring in the elders. \\
\poemll    Gather the youngsters \\
\poemlll       and even the nursing infants. \\
\poeml Call the bridegroom from his wedding preparations,\fnote{\fbackref{2:16} Lit. \fbib{Bring out the bridegroom from his wedding chamber}} \\
\poemll    and the bride from her dressing room. \\
\poeml \v{17}As they serve\fnote{\fbackref{2:17} The Heb. lacks \fbib{As they serve}} between the porch and the altar, \\
\poemll    let the priests and ministers of the \divine{Lord} weep and pray: \\
\poeml `Spare your people, \divine{Lord}, \\
\poemll    and do not make your heritage a disgrace \\
\poemlll       so that nations ridicule them. \\
\poeml Why should they say among the people, \\
\poemll    ``Where is their God?''\,'\,''
\passage{Response to the People's Repentance}
\poeml \v{18}Then the \divine{Lord} will show great concern for his land, \\
\poemll    and will have compassion on his people. \\
\poeml \v{19}The \divine{Lord} will say to his people, \\
\poemll    ``Look! I will send you grain, new wine, and oil, \\
\poemlll       and you will be content with them. \\
\poeml I will no longer cause you to be a disgrace among the nations.''
\passage{Destruction of the Invaders}
\poeml \v{20}``I will remove the northerners\fnote{\fbackref{2:20} Lit. \fbib{the North}; i.e. the army that comes from the North} from you, \\
\poemll    driving them\fnote{\fbackref{2:20} Lit. \fbib{him}; i.e. the northern army symbolized as an individual} to a barren and desolate land--- \\
\poeml the front toward the Dead Sea\fnote{\fbackref{2:20} Lit. \fbib{the eastern sea}} \\
\poemll    and the back toward the Mediterranean.\fnote{\fbackref{2:20} Lit. \fbib{the western sea}} \\
\poeml Their stench will rise, \\
\poemll    and their stinking odor will ascend, \\
\poemlll       because they have done great things.''
\passage{The \divine{Lord}'s Restoration of the Land}
\poeml \v{21}``Stop being afraid, land! \\
\poemll    Rejoice and be glad, \\
\poemlll       because the \divine{Lord} will do great things. \\
\poeml \v{22}Stop being afraid, beasts of the field, \\
\poemll    because the desert pastures will bloom, \\
\poeml the trees will bear their fruit, \\
\poemll    and the fig tree and vine will deliver their wealth. \\
\poeml \v{23}And so be glad, children of Zion, \\
\poemll    and rejoice in the \divine{Lord} your God, \\
\poeml because he has given you the right amount of early rain, \\
\poemll    and he will cause the rain to fall for you, \\
\poemlll       both the early rain and the later rain as before. \\
\poeml \v{24}The threshing floors will be smothered in grain, \\
\poemll    and the vats will overflow with wine and oil. \\
\poeml \v{25}``Then I will restore to you the years that the locust swarm devoured, \\
\poemll    as did the young locust, the other locusts, and the ravaging locust, \\
\poemlll       that great army of mine that I sent among you. \\
\poeml \v{26}You will have plenty to eat, and will be fully satisfied. \\
\poemll    You will praise the name of the \divine{Lord} your God, \\
\poeml who has performed wonders specifically for you. \\
\poemll    And my people will never be ashamed. \\
\poeml \v{27}As a result, you will know that I am in the midst of Israel; \\
\poemll    that I myself am the \divine{Lord} your God--- \\
\poemlll       and there is none other! \\
\poeml And my people will never be ashamed.''
\passage{The Day of the \divine{Lord}}
\poeml \v{28}\fnote{\fbackref{2:28} This verse is 3:1 in MT, v. 29 is 3:2, and so through the end of the chapter.}``Then it will come about at a later time \\
\poemll    that I will pour out my Spirit on every person. \\
\poeml Your sons and your daughters will prophesy. \\
\poemll    Your elderly people will dream dreams, \\
\poemlll       and your young people will see visions. \\
\poeml \v{29}Also at that time I will pour out my Spirit \\
\poemll    upon men and women servants. \\
\poeml \v{30}I will display warnings in the heavens, \\
\poemll    and on the earth blood, fire, and columns of smoke. \\
\poeml \v{31}The sun will be given over to darkness, \\
\poemll    and the moon to blood, \\
\poemlll       before the coming of the great and terrifying Day of the \divine{Lord}. \\
\poeml \v{32}And everyone who calls upon the name of the \divine{Lord} will be delivered. \\
\poemll    For as the \divine{Lord} has said, \\
\poemlll       `In Mount Zion and in Jerusalem there will be those who escape, \\
\poemlll       the survivors whom the \divine{Lord} is calling.'\,''
\end{poetry}
\labelchapt{3}
\passage{The Coming Judgment of Nations}

\begin{poetry}
\poeml \chapt{3}
\v{1}\fnote{\fbackref{3:1} This verse is 4:1 in MT, and so through the end of the chapter.}``Look, now! In those very days and at that time, \\
\poeml when I restore prosperity to\fnote{\fbackref{3:1} Or \fbib{bring back the captivity of}} Judah and Jerusalem, \\
\poeml \v{2}I will gather all nations, \\
\poemll    bringing them down to the Valley of Jehoshaphat. \\
\poeml I will set out my case against\fnote{\fbackref{3:2} Or \fbib{will judge}} them there, \\
\poemll    on behalf of my people, my heritage Israel, \\
\poeml whom they scattered among the nations, \\
\poemll    apportioning my land among themselves.\fnote{\fbackref{3:2} The Heb. lacks \fbib{among themselves}} \\
\poeml \v{3}They cast lots for my people--- \\
\poemll    they sold a young boy in exchange for a prostitute, \\
\poeml and a girl for wine, \\
\poemll    so they could drink.''
\passage{The \divine{Lord}'s Judgment upon Philistia}
\poeml \v{4}``Furthermore, what have you to do with me, \\
\poemll    Tyre, Sidon, and all the sea coasts of Philistia? \\
\poemlll       Are you taking revenge on me? \\
\poeml If you are taking revenge on me, \\
\poemll    I'll send it back on you\fnote{\fbackref{3:4} Lit. \fbib{your head}} swiftly and promptly, \\
\poeml \v{5}since you took my silver and gold, \\
\poemll    carried my precious treasures into your temples, \\
\poeml \v{6}and sold Judah's and Jerusalem's descendants to the Greeks,\fnote{\fbackref{3:6} Lit. \fbib{Jevanites}; i.e. descendants of Javan} \\
\poemll    so you can remove them far from their homeland! \\
\poeml \v{7}``Look, I will bring them up from where you sold them, \\
\poemll    I will turn your revenge back upon you,\fnote{\fbackref{3:7} Lit. \fbib{turn back your reward}} \\
\poeml \v{8}and I will sell your sons and daughters into the control of the people of Judah. \\
\poemll    And they will sell them to the people of Sheba, a country far away.'' \\
\poemlll       Indeed, the \divine{Lord} has spoken.''
\passage{The \divine{Lord}'s Call to Judgment}
\poeml \v{9}``Declare this among the nations: \\
\poemll    `Prepare for war! \\
\poeml Wake up your elite forces! \\
\poemll    Let all the soldiers draw near! \\
\poemlll       Call them up! \\
\poeml \v{10}Beat your plow blades into swords, \\
\poemll    and your pruning knives into spears! \\
\poemlll       Let the frail say, ``I am strong!'' \\
\poeml \v{11}Hurry and come, all you gentiles! \\
\poemll    Gather yourselves together!'\,''
\end{poetry}

\begin{poetry}
\poemlll       ``\divine{Lord}, cause your mighty army\fnote{\fbackref{3:11} Lit. \fbib{mighty ones}} to come down. \\
\poeml \v{12}``Let the nations be awakened \\
\poemll    and come to the Valley of Jehoshaphat; \\
\poemlll       because I will sit to judge all the surrounding nations. \\
\poeml \v{13}Put in the sickle, \\
\poemll    because the harvest is ripe. \\
\poeml Come and go down, \\
\poemll    because the winepress is full. \\
\poeml The wine vats are overflowing, \\
\poemll    because their evil is great! \\
\poeml \v{14}``Multitudes, multitudes \\
\poemll    in the Valley of Judgment! \\
\poeml For the Day of the \divine{Lord} is near \\
\poemll    in the Valley of Judgment! \\
\poeml \v{15}The sun and moon will grow dark, \\
\poemll    and the stars will stop shining. \\
\poeml \v{16}``The \divine{Lord} will roar from Zion, \\
\poemll    and shout from Jerusalem. \\
\poeml The heavens and the earth will shake, \\
\poemll    but the \divine{Lord} will be the refuge of his people, \\
\poemlll       and the strength of the people of Israel.''
\passage{God's Blessings on His People}
\poeml \v{17}``And truly you will know that I am the \divine{Lord} your God, \\
\poemll    dwelling in Zion, my holy mountain. \\
\poeml Then Jerusalem will be holy, \\
\poemll    and no foreigners will invade her again. \\
\poeml \v{18}It will come about at that time \\
\poemll    that the mountains will drip with newly pressed wine, \\
\poeml the hills will flow with milk, \\
\poemll    and the streams of Judah will flow abundantly. \\
\poeml A fountain will spring from the Temple of the \divine{Lord}, \\
\poemll    to water the Valley of the Acacias. \\
\poeml \v{19}Egypt will be desolate, \\
\poemll    and Edom will be a desert, \\
\poeml because of violence against the people of Judah \\
\poemll    since they shed innocent blood in their land. \\
\poeml \v{20}But Judah will live forever, \\
\poemll    and Jerusalem from generation to generation. \\
\poeml \v{21}I will acquit their bloodguilt that has not yet been acquitted. \\
\poemll    For the \divine{Lord} lives in Zion!''\end{poetry}
