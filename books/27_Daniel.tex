\bookheader{Daniel}
\labelbook{Dan}

\bookpretitle{The Book of the Prophet}
\booktitle{Daniel}

\labelchapt{1}
\passage{Hostages of the Babylonian Captivity}

\chapt{1}
\v{1}In the third year of the reign of King Jehoiakim of Judah, King Nebuchadnezzar of Babylon came to Jerusalem and laid siege to it. \v{2}Within a week, the Lord handed King Jehoiakim of Judah over to him, along with valuable objects from the house of God. Nebuchadnezzar\fnote{\fbackref{1:2} Lit. \fbib{He}} brought them to the temple of his god in the land of Shinar\fnote{\fbackref{1:2} I.e. Babylon} and stored them\fnote{\fbackref{1:2} Lit. \fbib{the valuable objects}} in its treasure house.\fnote{\fbackref{1:2} Lit. \fbib{in the treasure house of his god}}

\v{3}Later, the king ordered Ashpenaz, his chief officer,\fnote{\fbackref{1:3} Lit. \fbib{eunuch}; i.e. an overseer in the king's court; and so throughout the chapter} to bring in some Israelis of royal and noble descent. \v{4}They were to be young men without physical defect, handsome in appearance, skilled in all wisdom, quick to learn, prudent in how they used knowledge, and capable of serving in the king's palace. They were to learn the literature and language of the Chaldeans.\fnote{\fbackref{1:4} I.e. Aramaic speaking wise men from Mesopotamia; or magi-astrologers; and so throughout the book; cf. Matt 2:1}

\v{5}The king assigned them fine food and choice wine on a daily basis, ordering them to be trained for three years, at the end of which time they would enter the king's service.\fnote{\fbackref{1:5} Lit. \fbib{would stand before the king}} \v{6}Included among the people of Judah were Daniel,\fnote{\fbackref{1:6} The Heb. name \fbib{Daniel} means \fbib{God is my judge}} Hananiah, Mishael, and Azariah. \v{7}The chief officer assigned the name ``Belteshazzar'' to Daniel, the name ``Shadrach'' to Hananiah, the name ``Meshach'' to Mishael, and the name ``Abednego'' to Azariah.
\passage{Daniel Chooses God's Standard}

\v{8}Daniel determined within himself not to become defiled by the king's menu of rich foods or by the king's wine, so he requested permission\fnote{\fbackref{1:8} The Heb. lacks \fbib{permission}} from the chief officer not to defile himself. \v{9}God granted to Daniel grace and compassion on the part of the chief officer. \v{10}The chief officer told Daniel, ``I fear his majesty the king, who has determined what you eat and drink. If he notices that your faces are more pale than the other\fnote{\fbackref{1:10} The Heb. lacks \fbib{other}} young men in your group, I will forfeit my head to the king.''

\v{11}But Daniel told the guard whom the chief officer had appointed over Daniel, Hananiah, Mishael, and Azariah, \v{12}``Please test your servants for ten days and let us be given vegetables to eat and water to drink. \v{13}Then compare how we\fnote{\fbackref{1:13} Lit. \fbib{they}} look with the young men who ate the king's rich food, and treat your servants in accordance with what you observe.''

\v{14}So he listened to what Daniel said\fnote{\fbackref{1:14} Lit. \fbib{listened according to this word}} and tested them for ten days. \v{15}At the end of ten days their appearance was better and their faces were well-nourished\fnote{\fbackref{1:15} Lit. \fbib{were fatter of flesh}} compared to the young men who ate the king's rich food. \v{16}So the guard took away their rich food and wine,\fnote{\fbackref{1:16} Lit. \fbib{wine of their drinks}} giving them vegetables. \v{17}As for these four young men, God gave them knowledge, aptitude for learning, and wisdom. Daniel also could understand all kinds of visions and dreams.

\v{18}Then at the end of the training period\fnote{\fbackref{1:18} Lit. \fbib{the days}} that the king had established, the chief officer brought them in before Nebuchadnezzar. \v{19}When the king spoke to them, none of them compared to Daniel, Hananiah, Mishael, or Azariah as they stood before the king. \v{20}In every matter of wisdom or understanding that the king discussed with\fnote{\fbackref{1:20} Lit. \fbib{king sought from}} them, he found them ten times superior to all the astrologers and enchanters\fnote{\fbackref{1:20} Or \fbib{occult practitioners}} in his entire palace.

\v{21}So Daniel remained there in service\fnote{\fbackref{1:21} The Heb. lacks \fbib{in service}} until the first year of King Cyrus.\fnote{\fbackref{1:21} I.e. until the fall of Babylon as a world empire}
\labelchapt{2}
\passage{King Nebuchadnezzar's Dream}

\chapt{2}
\v{1}During the second year of Nebuchadnezzar's reign, Nebuchadnezzar had dreams that troubled him.\fnote{\fbackref{2:1} Lit. \fbib{troubled his spirit}} As a result, he couldn't sleep.\fnote{\fbackref{2:1} Lit. \fbib{result, sleep departed from him}} \v{2}So the king gave orders to summon diviners, enchanters,\fnote{\fbackref{2:2} Or \fbib{conjurers} and so throughout the book} sorcerers, and Chaldeans to reveal to the king what he had dreamed. When they came and stood before him,\fnote{\fbackref{2:2} Lit. \fbib{before the king}} \v{3}the king told them, ``I have dreamed a dream and I\fnote{\fbackref{2:3} Lit. \fbib{and my spirit}} will remain troubled until I can understand it.''\fnote{\fbackref{2:3} Lit. \fbib{understand the dream}}

\v{4}The Chaldeans responded to the king in Aramaic:\fnote{\fbackref{2:4} At this point the text changes from Heb. to Aram. until the end of ch. 7.} ``May the king live forever. Tell the dream to your servants, and we'll reveal its meaning.''

\v{5}In reply the king told the Chaldeans, ``Here is what I have commanded: If you don't tell me both the dream and its meaning, you'll be destroyed and your houses will be reduced to rubble. \v{6}But if you do relate the dream to me as well as its meaning, you'll receive gifts, rewards, and great honor from me. Therefore reveal the dream to me, along with its meaning.''

\v{7}They replied again, ``Let the king tell his servants the dream, and we'll disclose its meaning.''

\v{8}The king responded,\fnote{\fbackref{2:8} Lit. \fbib{responded and said}} ``I'm convinced that you're stalling for time because you're aware of what I've commanded. \v{9}So if you don't disclose the dream to me, there will be only one sentence for all of you. You have conspired together to present lies and corrupt interpretations until the situation changes. Now tell me the dream and I'll know that you can reveal its true\fnote{\fbackref{2:9} The Aram. lacks \fbib{true}} meaning.''

\v{10}The Chaldeans answered the king directly, ``There's not a single man on earth who can do\fnote{\fbackref{2:10} Lit. \fbib{reveal}} what the king has commanded. No king, lord, or ruler has ever asked such a thing from any diviner, enchanter,\fnote{\fbackref{2:10} Or \fbib{occult practitioner}} or Chaldean. \v{11}Furthermore, what the king is asking is so difficult that no one can reveal it except the gods---and they don't live with human beings.''

\v{12}At this point, the king flew into a rage\fnote{\fbackref{2:12} Lit. \fbib{king was furious and very angry}} and issued an order to destroy all the advisors\fnote{\fbackref{2:12} Lit. \fbib{the wise men}} of Babylon. \v{13}When the order went out to kill the advisors,\fnote{\fbackref{2:13} Lit. \fbib{the wise men}} they searched for Daniel and his friends to kill them, too.\fnote{\fbackref{2:13} The Aram. lacks \fbib{too}}
\passage{Daniel Requests Time to Answer the King}

\v{14}Daniel responded with wisdom and discretion to Arioch, the king's executioner, who had gone out to execute the advisors\fnote{\fbackref{2:14} Lit. the \fbib{wise men}} of Babylon. \v{15}He asked\fnote{\fbackref{2:15} Lit. \fbib{answered and said}} him,\fnote{\fbackref{2:15} Lit. \fbib{asked Arioch, the king's executioner}} ``Why such a harsh decree from the king?'' Then Arioch informed Daniel, \v{16}so Daniel went to ask Nebuchadnezzar\fnote{\fbackref{2:16} Lit. \fbib{the king}} for an appointment to see him\fnote{\fbackref{2:16} Lit. \fbib{for time}}, and it was granted him so that he could reveal the meaning to the king. \v{17}Then Daniel went home and told his friends Hananiah, Mishael, and Azariah about the king's\fnote{\fbackref{2:17} The Aram. lacks \fbib{king's}} command. \v{18}Daniel\fnote{\fbackref{2:18} Lit. \fbib{He}} was seeking mercy, in order to ask about this mystery in the presence of the God of heaven, so that Daniel and his friends might not be executed along with the rest of the advisors\fnote{\fbackref{2:18} Lit. the \fbib{wise men}} of Babylon.
\passage{The King's Dream is Revealed to Daniel}

\v{19}When the mystery was revealed to Daniel in a vision later that night, Daniel blessed the God of heaven \v{20}and said,

\begin{poetry}
\poeml ``May the name of God be blessed forever and ever; \\
\poemll    wisdom and power are his for evermore. \\
\poeml \v{21}It is God\fnote{\fbackref{2:21} Lit. \fbib{he}} who alters the times and seasons, \\
\poemll    and he removes kings and promotes kings. \\
\poeml He gives wisdom to the wise \\
\poemll    and knowledge to the discerning. \\
\poeml \v{22}He reveals what is profoundly mysterious \\
\poemll    and knows what is in the darkness; \\
\poemlll       with him dwells light. \\
\poeml \v{23}To you, God of my ancestors, I give thanks and praise, \\
\poemll    because you have given me wisdom and power; \\
\poeml you have now revealed to me what we asked of you \\
\poemll    by making known to us what the king commanded.''
\end{poetry}

\v{24}After this,\fnote{\fbackref{2:24} Lit. \fbib{All on account of that}} Daniel went to Arioch, whom the king had appointed to execute the advisors\fnote{\fbackref{2:24} Lit. \fbib{the wise men}} of Babylon. He told him, ``Don't destroy the advisors\fnote{\fbackref{2:24} Lit. \fbib{the wise men}} of Babylon. Bring me before the king and I'll explain the meaning to him.''\fnote{\fbackref{2:24} Lit. \fbib{the king}}

\v{25}Then Arioch quickly brought Daniel into the king's presence and informed him: ``I've found a man from the Judean captives who will make known the meaning to the king.''
\passage{Daniel Reveals the Meaning of the Dream}

\v{26}King Nebuchadnezzar\fnote{\fbackref{2:26} Lit. \fbib{The king}} replied by saying to Daniel (whose Babylonian\fnote{\fbackref{2:26} The Aram. lacks \fbib{Babylonian}} name is Belteshazzar), ``Are you able to tell me about the dream\fnote{\fbackref{2:26} Lit. \fbib{dream I saw}} and its meaning?''

\v{27}By way of answer, Daniel addressed the king:\fnote{\fbackref{2:27} Lit. \fbib{answered and said before the king}}

\begin{poetry}
\poeml ``None of the advisors,\fnote{\fbackref{2:27} Lit. \fbib{the wise men}} enchanters,\fnote{\fbackref{2:27} Or \fbib{occult practitioners}} diviners, or astrologers\fnote{\fbackref{2:27} Or \fbib{those who gaze at entrails}} can explain the secret that the king has requested to be made known.\fnote{\fbackref{2:27} The Aram. lacks \fbib{to be made known}} \v{28}But there is a God in heaven who reveals secrets, and he is making known to King Nebuchadnezzar what will happen in the latter days. \\
\poeml ``While you were in bed, the dream and the visions that came to your head were as follows: \v{29}Your majesty,\fnote{\fbackref{2:29} Lit. \fbib{O King}, and so throughout the book} when you were in bed, thoughts came to your mind\fnote{\fbackref{2:29} The Aram. lacks \fbib{to your mind}} about what would happen in the future, and the Revealer of Secrets has made known to you what will take place. \v{30}As for me, this secret was made known to me, not because my own wisdom is greater than anyone else alive, but in order that the meaning may be made known to the king, and that you might understand the thoughts of your heart. \\
\poeml \v{31}``Your majesty, while you were watching, you observed an enormous statue. This magnificent statue stood before you with extraordinary brilliance. Its appearance was terrifying. \v{32}That statue had a head made\fnote{\fbackref{2:32} The Aram. lacks \fbib{made}} of pure gold, with its chest and arms made\fnote{\fbackref{2:32} The Aram. lacks \fbib{made}} of silver, its abdomen and thighs made\fnote{\fbackref{2:32} The Aram. lacks \fbib{made}} of bronze, \v{33}its legs made\fnote{\fbackref{2:33} The Aram. lacks \fbib{made}} of iron, and its feet made\fnote{\fbackref{2:33} The Aram. lacks \fbib{made}} partly of iron and partly of clay. \\
\poeml \v{34}``As you were watching, a rock was quarried---but not with human hands---and it struck the iron and clay feet of the statue, breaking them to pieces. \v{35}Then the iron, the clay, the bronze, the silver, and the gold were broken in pieces together and became like chaff from a summer threshing floor that the breeze carries away without leaving a trace.\fnote{\fbackref{2:35} Lit. \fbib{trace of them}} Then the rock that struck the statue grew into\fnote{\fbackref{2:35} Lit. \fbib{became}} a huge mountain and filled the entire earth. \\
\poeml \v{36}``This was the dream, and we'll now relate its meaning to the king. \v{37}You, your majesty, king of kings---to whom the God of heaven has given the kingdom, the power, the strength, and the glory, \v{38}so that wherever people,\fnote{\fbackref{2:38} Lit. \fbib{sons of mankind}} wild animals, or birds of the sky live, he has placed them under your control, giving you dominion over them all---you're that head of gold. \\
\poeml \v{39}``After you, another kingdom will arise that is inferior to\fnote{\fbackref{2:39} Or \fbib{lower than}} yours, and then a third kingdom of bronze will arise to rule all the earth. \v{40}Then there will be a fourth kingdom, as strong as iron. Just as all things are broken to pieces and shattered by iron, so it will shatter and crush everything. \\
\poeml \v{41}``The feet and toes that you saw, made partly of potter's clay and partly of iron, represent\fnote{\fbackref{2:41} The Aram. lacks \fbib{represent}} a divided kingdom. It will still have the strength of iron, in that you saw iron mixed with clay. \v{42}Just as their toes and feet are part iron and part clay, so will the kingdom be both strong and brittle. \v{43}Just as you saw iron mixed with clay, so they will mix themselves with human offspring.\fnote{\fbackref{2:43} Lit. \fbib{seed}} Furthermore,\fnote{\fbackref{2:43} Or \fbib{will develop alliances through intermarriages, and}} they won't remain together, just as iron doesn't mix with clay. \\
\poeml \v{44}``During the reigns of those kings, the God of heaven will set up a kingdom that will never be destroyed, nor its sovereignty\fnote{\fbackref{2:44} Or \fbib{kingdom}} left in the hands of another people. It will shatter and crush all of these kingdoms, and it will stand forever. \v{45}Now, just as you saw that the stone was cut out of the mountain without human hands---and that it crushed the iron, bronze, clay, silver, and gold to pieces---so also the great God has revealed to the king what will take place after this. Your dream will come true, and its meaning will prove trustworthy.''
\end{poetry}
\passage{Nebuchadnezzar Promotes Daniel and His Friends}

\v{46}Then King Nebuchadnezzar fell on his face before Daniel, paid honor to him, and commanded that an offering and incense be presented on his behalf. \v{47}The king told Daniel, ``Truly your God is the God of gods, the Lord of kings, and the Revealer of Secrets, because you were able to reveal this mystery.'' \v{48}Then the king promoted Daniel to a high position and lavished many great gifts on him, including making him ruler over the entire province of Babylon and chief administrator over the advisors\fnote{\fbackref{2:48} Lit. \fbib{the wise men}} of Babylon. \v{49}Moreover, Daniel requested that the king appoint Shadrach, Meshach, and Abednego administrators over the province of Babylon, while Daniel himself remained in the royal court.
\labelchapt{3}
\passage{Dedicating the Image to Nebuchadnezzar}

\chapt{3}
\v{1}Some time later, king Nebuchadnezzar built a golden statue, making it 60 cubits\fnote{\fbackref{3:1} I.e. about 90 feet; a cubit was about eighteen inches} high and six cubits\fnote{\fbackref{3:1} I.e. about nine feet; a cubit was about eighteen inches} wide. He set it up in the Dura Valley\fnote{\fbackref{3:1} Or \fbib{Plain}} within the province of Babylon. \v{2}Then King Nebuchadnezzar summoned the regional authorities,\fnote{\fbackref{3:2} Or \fbib{satraps}} governors, deputy governors, advisors, treasurers, judges, magistrates, and all of the other\fnote{\fbackref{3:2} The Aram. lacks \fbib{other}} administrators of the provinces, ordering them to come to the dedication of the statue that he\fnote{\fbackref{3:2} Lit. \fbib{Nebuchadnezzar}} had erected.

\v{3}So the regional authorities,\fnote{\fbackref{3:3} Or \fbib{satraps}} governors, deputy governors, advisors, treasurers, judges, magistrates, and all of the other\fnote{\fbackref{3:3} The Aram. lacks \fbib{other}} administrators of the provinces assembled to dedicate the statue that King Nebuchadnezzar had erected. They took their places in front of the statue that he\fnote{\fbackref{3:3} Lit. \fbib{Nebuchadnezzar}} had erected. \v{4}Then a herald proclaimed aloud:

\begin{poetry}
\poeml ``People of all\fnote{\fbackref{3:4} The Aram. lacks \fbib{of all}} nations, and languages are commanded: \v{5}Whenever you hear the sound of the trumpet, the flute, the lyre, the four-stringed lyre, and the harp, playing together along with various instruments, you are to fall down and worship the golden statue that was set up by King Nebuchadnezzar. \v{6}Anyone who does not fall down and worship is immediately to be thrown into the blazing fire furnace.''
\end{poetry}

\v{7}Therefore, when all of the people ``heard the sound of the trumpet, the flute, the lyre, the four-stringed lyre, and the harp, playing together along with various other\fnote{\fbackref{3:7} The Aram. lacks \fbib{other}} instruments,'' all the ``people, nations, and languages'' began to fall down and worship the golden statue that King Nebuchadnezzar had set up.
\passage{Daniel's Friends are Accused}

\v{8}Just then, certain influential Chaldeans took this opportunity to come forward and denounce the Jews. \v{9}They told King Nebuchadnezzar, ``Your majesty, live forever. \v{10}You, your majesty, issued this decree:

\begin{poetry}
\poeml `Every man who hears the sound of the trumpet, the flute, the lyre, the four-stringed lyre, and the harp, playing together along with various other\fnote{\fbackref{3:10} The Aram. lacks \fbib{other}} instruments is to fall down and worship the golden statue. \v{11}Whoever does not fall down and worship is to be thrown into a blazing fire furnace.'
\end{poetry}

\v{12}``Certain influential Jewish men whom you appointed to manage the city of Babylon---Shadrach, Meshach, and Abednego---have neither paid attention to you, your majesty, nor served your gods. And they won't worship the golden statue that you set up.''
\passage{The Threat of the Fire Furnace}

\v{13}Nebuchadnezzar flew into a rage and furiously ordered that Shadrach, Meshach, and Abednego be brought before him.\fnote{\fbackref{3:13} Lit. \fbib{before the king}} \v{14}Nebuchadnezzar asked them, ``Is it true, Shadrach, Meshach, and Abednego, that you don't worship my gods and that you don't worship the golden statue that has been set up? \v{15}Now, if you are ready at this very moment to obey `the sound of the trumpet, the flute, the lyre, the four-stringed lyre, and the harp,' and worship the image that I have made{\ldots} If you do not so worship, you will immediately have cast yourselves into the middle of the blazing fire, and what god is there who can deliver you from my power?''\fnote{\fbackref{3:15} Lit. \fbib{hands}}
\passage{Daniel's Friends Answer King Nebuchadnezzar}

\v{16}Shadrach, Meshach, and Abednego answered King Nebuchadnezzar, ``It's not necessary for us to respond in this matter. \v{17}Your majesty, if it be his will,\fnote{\fbackref{3:17} The Aram. lacks \fbib{his will}} our God whom we serve can deliver us from the blazing fire furnace, and he will deliver us from you.\fnote{\fbackref{3:17} Lit. \fbib{from your hand}} \v{18}But if not, rest assured, your majesty, that we won't serve your gods, and we won't worship the golden statue that you have set up.''
\passage{The King Orders an Execution}

\v{19}Out of control with rage, Nebuchadnezzar's facial expression changed toward Shadrach, Meshach, and Abednego, and he ordered\fnote{\fbackref{3:19} Lit. \fbib{answered and ordered}} that the furnace be heated seven times hotter than usual. \v{20}Then he issued orders to his elite guard to bind Shadrach, Meshach, and Abednego with ropes\fnote{\fbackref{3:20} The Aram. lacks \fbib{with ropes}; and so through vs. 24} and throw them into the blazing fire furnace. \v{21}So the elite guard tied them up fully clothed, still wearing their robes, tunics, and turbans, and threw them into the blazing fire furnace, \v{22}because the king's command was so drastic. Since the furnace was blazing hot, its flames killed those who threw Shadrach, Meshach, and Abednego into the blazing fire. \v{23}Bound firmly with ropes, these three men Shadrach, Meshach, and Abednego fell into the blazing fire furnace.
\passage{The Fourth Man in the Furnace}

\v{24}Astonished, King Nebuchadnezzar stood up in terror, and asked his advisers, ``Didn't we throw three men into the fire, bound firmly with ropes?''

In reply they told the king, ``Yes, your majesty.''

\v{25}``Look!'' he told them,\fnote{\fbackref{3:25} Lit. \fbib{answered and said}} ``I see four men walking untied and unharmed in the middle of the fire, and the appearance of the fourth resembles a divine being.''\fnote{\fbackref{3:25} Lit. \fbib{a son of the gods}}

\v{26}Then Nebuchadnezzar approached the opening of the blazing fire furnace. He shouted out, ``Shadrach, Meshach, and Abednego, servants of the Most High God, come out and come here!'' So Shadrach, Meshach, and Abednego came out of the fire. \v{27}The regional authorities,\fnote{\fbackref{3:27} Or \fbib{satraps}} viceroys, governors, and royal advisors gazed at those men and saw that the fire had no effect on their bodies---not a hair on their head was singed, their clothes were not burned, and they did not smell of fire.

\v{28}Nebuchadnezzar spoke up and announced:

\begin{poetry}
\poeml ``Blessed be the God of Shadrach, Meshach and Abednego! He sent his angel to deliver his servants who trusted in him. They disobeyed the king's command and were willing to risk their lives in order not to serve or worship any god except their own God. \v{29}So I decree that people from any nation or language who say anything against the God of Shadrach, Meshach and Abednego will be destroyed and their house reduced to rubble, because there is no other god who can save like this.''
\end{poetry}

\v{30}Then the king promoted Shadrach, Meshach, and Abednego within the province of Babylon.
\labelchapt{4}
\passage{Nebuchadnezzar's Testimonial}

\chapt{4}
\v{1}\fnote{\fbackref{4:1} This v. is 3:31 in MT, and so through v. 3.}\divine{An Official Statement}\fnote{\fbackref{4:1} The Aram. lacks \fbib{An Official Statement}}

\divine{from Nebuchadnezzar}

\divine{the King}

To the people of all nations and languages who live on earth.

Peace and prosperity to you!

\begin{poetry}
\poeml \v{2}It gives me great pleasure to tell about the signs and wonders that the Most High God has done for me. \\
\poeml \v{3}How great are his signs! \\
\poeml How powerful are his wonders! \\
\poeml His kingdom is an eternal kingdom, and his dominion lasts from generation to generation.
\passage{Nebuchadnezzar's Dream}
\poeml \v{4}\fnote{\fbackref{4:4} This v. is 4:1 in MT, and so throughout the chapter.}I, Nebuchadnezzar, was resting in my home and prospering in my palace. \v{5}I had a dream that made me afraid. The thoughts that went through my mind while in bed and the visions in my head terrified me. \v{6}So I gave an order to bring in all of the advisors\fnote{\fbackref{4:6} Lit. \fbib{the wise men}} of Babylon so they would tell me the interpretation of the dream. \\
\poeml \v{7}Then the diviners, enchanters,\fnote{\fbackref{4:7} Or \fbib{occult practitioners}} Chaldeans, and astrologers\fnote{\fbackref{4:7} Or \fbib{those who gaze at entrails}} came in, and I told them the dream. But they could not reveal its interpretation to me. \v{8}Eventually, Daniel appeared before me. (He is called Belteshazzar, in accordance with the name of my god, and the spirit of the holy gods is within him.) I told him my dream: \\
\poeml \v{9}``Belteshazzar, chief of the diviners, since I know that the spirit of the holy gods is within you, and no mystery too difficult for you, explain to me the vision of my dream that I saw, along with its interpretation. \v{10}This is what I saw in the visions of my head while I was in bed: I was looking and---listen carefully!---I saw a tree in the middle of the earth, the height of which was very great. \v{11}The tree grew large, became strong, and its top reached the sky. It could be seen to the ends of the earth. \v{12}Its foliage was beautiful, its fruit bountiful, and its food sufficient for everyone. The animals of the field found shade under it, the birds of the sky lived in its branches, and every creature was fed from it. \\
\poeml \v{13}``Then I saw in the visions of my head while I was in bed---and take careful notice!---I saw a holy observer descend from heaven. \v{14}He called out aloud: \\
\poeml `Cut down the tree and cut off its branches. Strip off its foliage and scatter its fruit. Let the animals get out from under it, and let the birds leave\fnote{\fbackref{4:14} The Aram. lacks \fbib{leave}} its branches. \v{15}Nevertheless, leave the stump and its roots in the ground, but bind it with iron and bronze in the field grass. Let him be drenched with dew from the sky, and let him graze with the animals in the grass of the earth. \v{16}Let his mind be changed from that of a man, and let him be given the mind of an animal until seven seasons of time pass by for him. \v{17}This order is announced by the observers, and the holy ones declare the verdict, so that the living may know that the Most High is sovereign over human kingdoms and grants them to whomever he desires, and he places the least important of men over them.' \\
\poeml \v{18}``This is the dream that I, King Nebuchadnezzar, saw. Belteshazzar, tell me its meaning, since none of the advisors\fnote{\fbackref{4:18} Lit. \fbib{the wise men}} in my kingdom can tell me its interpretation. But you are able to do so\fnote{\fbackref{4:18} The Aram. lacks \fbib{to do so}} because the spirit of the holy gods is in you.''
\end{poetry}
\passage{Daniel's Interpretation}

\v{19}Then Daniel (also known as Belteshazzar) was greatly troubled for a while and was terrified by his thoughts. The king said, ``Belteshazzar, don't let the dream or its meaning terrify you.''

Belteshazzar responded, ``Your majesty, if only\fnote{\fbackref{4:19} The Aram. lacks \fbib{if only}} the dream were about your enemies and its meaning about those who oppose you! \v{20}The tree that you saw, which grew large and strong until its top reached the sky and became visible to the whole earth \v{21}with beautiful leaves and abundant fruit---enough food for everyone---and under which wild animals of the field found shelter and the birds of the air had nests in its branches--- \v{22}it's you, your majesty! You've become great and strong, your greatness has grown to the heavens, and your dominion reaches to the distant parts of the earth.

\v{23}``Your majesty saw a holy observer descending from heaven and saying, `Cut down the tree and destroy it, but leave the stump in the ground, along with its roots, bound with iron and bronze in the field grass. Let him be soaked with the dew of the sky and live with the wild animals of the field until seven seasons pass over him.'

\v{24}``This is the meaning, your majesty, and this is the decree that the Most High has issued against his majesty, the king: \v{25}You'll be driven from people, and you'll live among wild animals of the field. You'll eat grass like cattle and be soaked with the dew of the sky while seven years pass you by\fnote{\fbackref{4:25} Lit. \fbib{seven seasons pass over you}}---until you realize that the Most High is sovereign over human kingdoms and grants them to whomever he desires. \v{26}Just as it was ordered to leave the stump of the tree in the ground\fnote{\fbackref{4:26} The Aram. lacks \fbib{in the ground}} along with its roots, so your kingdom will be restored to you when you realize that Heaven rules over everything.\fnote{\fbackref{4:26} The Aram. lacks \fbib{everything}} \v{27}Therefore, your majesty, may my advice be acceptable to you: Stop your sinning, do what's right, and put a stop to your wickedness by showing kindness to the oppressed. Perhaps your tranquility will continue.''
\passage{The Dream Comes True}

\v{28}All of this happened to King Nebuchadnezzar. \v{29}About a year later,\fnote{\fbackref{4:29} Lit. \fbib{At the end of 12 months}} as the king was walking on the roof of the royal palace of Babylon, \v{30}he\fnote{\fbackref{4:30} Lit. \fbib{the king}} commented to himself,\fnote{\fbackref{4:30} Lit. \fbib{commented and said}} ``Isn't Babylon great? I've built a royal palace in it by my own might and power, for the sake\fnote{\fbackref{4:30} Lit. \fbib{glory}} of my majesty.''

\v{31}As these words were being spoken by the king, a voice came out of heaven: ``King Nebuchadnezzar, this is declared to you:

\begin{poetry}
\poeml `The kingdom has been taken\fnote{\fbackref{4:31} Or \fbib{has departed}} from you! \v{32}You're to be driven away from people. You're to live with the wild animals of the field. You are to be made to eat grass like cattle, and seven years will pass you by\fnote{\fbackref{4:32} Lit. \fbib{seven seasons will pass over you}} until you realize that the Most High is sovereign over human kingdoms and grants them to whomever he desires.'\,''
\end{poetry}

\v{33}The decree was fulfilled against Nebuchadnezzar immediately. He was driven away from people to eat grass like cattle, and his body was drenched with dew from the sky, until his hair grew like eagles' feathers and his nails like birds' claws.''
\passage{The King's Sanity Returns}

\v{34}``When that period of time was over, I, Nebuchadnezzar, lifted my eyes to heaven and my sanity returned to me. I blessed the Most High, praising and honoring the one who lives forever:

\begin{poetry}
\poeml For his sovereignty is eternal, \\
\poemll    and his kingdom continues from generation to generation. \\
\poeml \v{35}All who live on the earth \\
\poemll    are nothing compared to him. \\
\poeml He does what he wishes \\
\poemll    with the heavenly armies \\
\poemlll       and with those who live on earth. \\
\poeml No one can hold back his power \\
\poemll    or say to him, `What did you do?'
\end{poetry}

\v{36}At that moment I recovered my sanity, and my honor and majesty returned to me, for the sake\fnote{\fbackref{4:36} Lit. \fbib{glory}} of my kingdom. My advisors and officials sought me out, my throne was restored, and even more greatness than I had before was added to me. \v{37}In conclusion, I, Nebuchadnezzar, praise, exalt, and give glory to the King of heaven:

\begin{poetry}
\poeml For everything he does is true, \\
\poemll    his ways are just, \\
\poemlll       and he is able to humble those who walk in pride.''
\end{poetry}
\labelchapt{5}
\passage{Belshazzar's Festival}

\chapt{5}
\v{1}King Belshazzar put on a great festival for a thousand of his officials. He joined all\fnote{\fbackref{5:1} The Aram. lacks \fbib{all}} one thousand of them in getting drunk. \v{2}Under the influence of wine, Belshazzar ordered that the gold and silver vessels his grandfather Nebuchadnezzar had taken from the Temple in Jerusalem be brought in so the king, his officials, his wives, and his mistresses\fnote{\fbackref{5:2} Or \fbib{concubines}; i.e. secondary wives} could drink from them. \v{3}As ordered, they brought in the gold vessels that had been taken from the sanctuary of God's Temple in Jerusalem, and the king, his officials, his wives, and mistresses\fnote{\fbackref{5:3} Or \fbib{concubines}; i.e. secondary wives} drank from them. \v{4}As they drank the wine, they praised gods of gold, silver, bronze, iron, wood, and stone.
\passage{The Handwriting on the Wall}

\v{5}At that moment, humanlike fingers of a hand appeared near the lamp stand of the royal palace and wrote on the plaster of the wall. \v{6}While the king watched the back of the hand as it was writing, his facial expression changed. Utterly frightened, he lost control of his own bowels\fnote{\fbackref{5:6} Lit. \fbib{frightened, the joints of his loins were loosened}; i.e. an involuntary physiological response from terror} and his knees knocked together.

\v{7}The king cried out to bring in enchanters,\fnote{\fbackref{5:7} Or \fbib{occult practitioners}} Chaldeans, and astrologers. He announced to the advisors\fnote{\fbackref{5:7} Or \fbib{the wise men}} of Babylon, ``Whoever can read this writing and tell me its meaning will be clothed in purple, have a gold chain placed around his neck, and will become the third highest ruler in the kingdom.''

\v{8}Then all the king's advisors came in, but they were unable to read the writing or tell the king what it meant. \v{9}So King Belshazzar became even more frightened, and his facial expression showed it. His officials also were thrown into confusion.

\v{10}Hearing\fnote{\fbackref{5:10} The Aram. lacks \fbib{Hearing}} the voices of the king and his officials, the queen entered the banquet hall. ``Your majesty, live forever,'' the queen said. ``Don't be frightened by your thoughts or allow your facial expression to show it. \v{11}There's a man in your kingdom in whom dwells\fnote{\fbackref{5:11} The Aram. lacks \fbib{dwells}} the spirit of the holy gods. During your grandfather's reign, he was found to have insight, intelligence, and wisdom, like that\fnote{\fbackref{5:11} Lit. \fbib{like the wisdom}} of the gods. Your grandfather, King Nebuchadnezzar---your kingly predecessor---appointed him to be chief administrator over the magicians, enchanters,\fnote{\fbackref{5:11} Or \fbib{occult practitioners}} Chaldeans, and astrologers, \v{12}because he was found to have an extraordinary spirit, knowledge, and understanding, along with an ability to interpret dreams, explain riddles, and solve difficult problems. His name is Daniel, whom the king renamed Belteshazzar. Call for Daniel, and he will reveal the meaning of the writing.''\fnote{\fbackref{5:12} The Aram. lacks \fbib{of the writing}}
\passage{Daniel Interprets the Handwriting}

\v{13}Then Daniel was brought before the king. The king spoke up and told Daniel, ``So you are Daniel, one of the Judean exiles whom my grandfather the king brought from Judah! \v{14}I've heard about you, that a spirit of the gods is in you and that you have insight, discernment, and extraordinary wisdom. \v{15}Take note that the advisors\fnote{\fbackref{5:15} Lit. the \fbib{wise men}} and enchanters\fnote{\fbackref{5:15} Or \fbib{occult practitioners}} were brought before me to read the writing and explain its meaning, but they were unable to do so.\fnote{\fbackref{5:15} Lit. \fbib{to tell the matter}} \v{16}However, I've heard that you can provide meaning and interpretation, and that you can solve difficult problems. If you are able to read the writing and report its meaning, you will be clothed in purple, have a gold chain placed around your neck, and you will become the third highest ruler in the kingdom.''

\v{17}At this, Daniel answered, speaking directly to\fnote{\fbackref{5:17} Lit. \fbib{speaking before}} the king, ``Let your gifts and rewards be given to someone else. However, I'll read the writing for the king and tell him its meaning. \v{18}Your majesty, the Most High God gave your grandfather Nebuchadnezzar sovereignty, as well as greatness, glory, and splendor. \v{19}And because of the greatness that he gave him, all peoples, nations, and languages revered and feared him. He executed those whom he desired to execute, he spared those whom he wished to spare, he promoted those whom he desired to promote, and he humbled those whom he wished to humble. \v{20}But when he\fnote{\fbackref{5:20} Lit. \fbib{his heart}} became arrogant and his spirit hardened, he was removed from his royal throne and his glory was taken away from him. \v{21}He was driven away from human society\fnote{\fbackref{5:21} Lit. \fbib{sons of the people}} and given the mind of an animal. He lived with wild donkeys, ate grass like cattle, and his body was soaked with dew from the sky until he realized that the Most High God is sovereign over human kingdoms and places over them whomever he desires.

\v{22}``But you, Belshazzar, his grandson, haven't humbled yourself, even though you knew all of this.

\v{23}``You've exalted yourself against the Lord of heaven.

``You've had the vessels from his Temple brought into your presence.

``And you, your officials, and your wives and mistresses drank wine from them.

``You praised gods of silver, gold, bronze, iron, wood, and stone, which can't see, hear, or demonstrate knowledge.

``But you didn't honor God, who holds in his power your very life and all your ways.

\v{24}``Therefore, the hand\fnote{\fbackref{5:24} Lit. \fbib{the palm of the hand}} that wrote this inscription was sent from his presence. \v{25}This is the written inscription:

\divine{MENE, MENE, TEKEL and PARSIN}

\v{26}These are the meanings of the words:

\begin{poetry}
\poeml MENE: God has audited\fnote{\fbackref{5:26} Lit. \fbib{numbered}} your kingdom---and has ended it. \\
\poeml \v{27}TEKEL: You've been weighed on the scales---and you don't measure up.\fnote{\fbackref{5:27} Lit. \fbib{and found lacking}} \\
\poeml \v{28}PERES: Your kingdom has been divided---and will be given to the Medes and Persians.''
\end{poetry}

\v{29}Then Belshazzar gave orders to clothe Daniel in purple, to place a chain of gold around his neck, and to proclaim him the third highest ruler of the kingdom.

\v{30}That night Belshazzar, king of the Chaldeans, was killed, \v{31}\fnote{\fbackref{5:31} This v. is 6:1 in MT}and Darius the Mede took over the kingdom at the age of 62.
\labelchapt{6}
\passage{Daniel's Service to Darius}

\chapt{6}
\v{1}\fnote{\fbackref{6:1} This v. is 6:2 in MT, and so throughout the chapter.}It pleased Darius to appoint 120 regional authorities\fnote{\fbackref{6:1} Or \fbib{satraps}} over the kingdom throughout the realm, \v{2}along with three chief administrators from them, one of which was Daniel. The regional authorities\fnote{\fbackref{6:2} Or \fbib{satraps}} reported to these three administrators,\fnote{\fbackref{6:2} The Aram. lacks \fbib{three administrators}} so that the king would experience no losses. \v{3}Daniel distinguished himself among all the administrators and regional authorities,\fnote{\fbackref{6:3} Or \fbib{satraps}} because he was of an extraordinary spirit. Therefore the king planned to appoint him over the whole kingdom.
\passage{A Plot to Destroy Daniel}

\v{4}Because of this, the administrators and regional authorities\fnote{\fbackref{6:4} Or \fbib{satraps}} tried to bring allegations of dereliction of duty in government affairs against Daniel, but they were unable to find any charges of corruption. Daniel\fnote{\fbackref{6:4} Lit. \fbib{he}} was trustworthy, and no evidence of\fnote{\fbackref{6:4} The Aram. lacks \fbib{of evidence}} negligence or corruption could be found against him. \v{5}So these men said, ``We'll never find any basis for complaint against Daniel unless we build it on the requirements of his God.''

\v{6}Then these administrators and regional authorities\fnote{\fbackref{6:6} Or \fbib{satraps}} went as a group to the king and said this, ``Your majesty, live forever! \v{7}All of the royal administrators, prefects, regional authorities,\fnote{\fbackref{6:7} Or \fbib{satraps}} scribes, and governors have concluded that the king should establish and enforce an edict that anyone who prays to any god or man for the next 30 days (except to you, your majesty) is to be thrown into the lions' pit. \v{8}Therefore, your majesty, establish the decree and sign the written document so it can't be changed, in accordance with the laws of the Medes and Persians that can't be repealed.'' \v{9}So King Darius signed the edict contained in the written document.
\passage{Daniel is Accused}

\v{10}When Daniel learned that the written document had been signed, he went to an upstairs room in his house that had windows opened facing Jerusalem. Three times a day he would kneel down, pray, and give thanks to his God, just as he had previously done.

\v{11}The conspirators\fnote{\fbackref{6:11} Lit. \fbib{These men}} then went as a group and found Daniel praying and seeking help before his God. \v{12}So they approached the king and asked, ``Didn't you sign an edict that for the next 30 days if anyone prays to any god or man, except to you, your majesty, he would be thrown into the lions' pit?''

The king responded, ``The decree has been established, in accordance with the laws of the Medes and Persians that can't be repealed.''

\v{13}Then they told the king, ``Daniel, who is one of the Judean exiles, pays no attention to you, your majesty, or to the written decree, since he is still praying three times a day.''

\v{14}When the king heard this, he was greatly upset, because he was determined to make every effort to save Daniel before the sun set. \v{15}But the men who had gone as a group to the king told him,\fnote{\fbackref{6:15} Lit. \fbib{the king}} ``Remember, your majesty, that according to the laws of the Medes and Persians, any decree or edict that the king establishes cannot be repealed.''
\passage{Daniel in the Lions' Pit}

\v{16}At this point, the king ordered Daniel brought in and thrown into the lions' pit. The king spoke to Daniel, ``Your God, whom you serve constantly, will deliver you himself.'' \v{17}A stone was brought and placed over the opening to the pit, and the king affixed a seal to it with his personal signet ring and with the signet rings of his officials so that no one would interfere with Daniel's situation. \v{18}Then the king retired to his palace to spend the night fasting. He enjoyed no entertainment, and he couldn't sleep.

\v{19}The king got up at dawn and went quickly to the lions' pit. \v{20}As he approached where Daniel was in the pit, he cried out to him\fnote{\fbackref{6:20} Lit. \fbib{Daniel}} in a voice filled with anguish, ``Daniel, servant of the living God, has your God, whom you serve constantly, been able to deliver you from the lions?''

\v{21}Daniel replied to the king, ``May your majesty live forever! \v{22}My God sent his angel and sealed the mouths of the lions. They have not harmed me, proving that I'm innocent before him. Also against you, your majesty, I've committed no offense.''

\v{23}The king was ecstatic, so he gave orders for Daniel to be released from the pit. Daniel was taken up from the pit, and no injury was found to have been inflicted on him, because he had believed in his God. \v{24}Then the king gave orders to bring those men who had tried to have Daniel devoured, and they threw them, their children, and their wives into the lions' pit. They had not reached the floor of the pit before the lions had overtaken them and crushed all their bones.
\passage{Darius Exonerates Daniel}

\v{25}Afterward, King Darius wrote to all peoples, nations, and languages who lived throughout his realm:

``May great prosperity be yours!

\v{26}``I hereby decree that in every area of my kingdom men\fnote{\fbackref{6:26} Lit. \fbib{they}} are to fear and tremble before the God of Daniel.

\begin{poetry}
\poeml For he is the living God, \\
\poemll    who endures forever. \\
\poeml His kingdom is one that will not be destroyed, \\
\poemll    and his dominion continues forever. \\
\poeml \v{27}He delivers and rescues \\
\poemll    and performs signs and wonders \\
\poemlll       in heaven and on earth. \\
\poeml He has delivered Daniel \\
\poemll    from the power of the lions.''
\end{poetry}

\v{28}Daniel achieved success during the reigns of Darius and Cyrus the Persian.
\labelchapt{7}
\passage{The Vision of the Four Beasts}

\chapt{7}
\v{1}In the first year of the reign of King Belshazzar of Babylon, Daniel dreamed a dream, receiving visions in his mind while in bed, after which he recorded the dream, relating this summary of events.

\v{2}Daniel said, ``I observed the vision during the night. Look! The four winds of the skies were stirring up the Mediterranean\fnote{\fbackref{7:2} Lit. \fbib{Great}} Sea. \v{3}Four magnificent animals were rising from the sea, each different from the other. \v{4}The first resembled a lion, but it had eagles' wings. I continued to watch until its wings were plucked off, it was lifted up off the ground, and it was forced to stand on two feet like a man. A human soul\fnote{\fbackref{7:4} Lit. \fbib{heart}} was imparted to it.

\v{5}``Then look!---a second animal resembling a bear followed it.\fnote{\fbackref{7:5} The Aram. lacks \fbib{it}} It was raised up on one side, with three ribs held between the teeth in its mouth. Therefore people kept telling it, `Get up and devour lots of meat!'

\v{6}``After this, I continued to watch---and look!---there was another one, resembling a leopard with four birds' wings on its back. The animal also had four heads, and authority was imparted to it.

\v{7}``After this, I continued to observe the night visions. And look!---there was a fourth awe-inspiring, terrifying, and viciously strong animal! It had large, iron teeth. It devoured and crushed things,\fnote{\fbackref{7:7} The Aram. lacks \fbib{things}} and trampled under its feet whatever remained. Different from all of the other previous animals, it had ten horns.

\v{8}``While I was thinking about the horns---look---another horn, this time\fnote{\fbackref{7:8} The Aram. lacks \fbib{this time}} a little one, grew up among them. Three of the first horns were yanked up by their roots right in front of it. Look! It had eyes like those of a human being and a mouth that boasted with audacious claims.''
\passage{The Vision of the Ancient of Days}

\v{9}``I kept on watching until the Ancient of Days was seated. His clothes were white, like snow, and the hair on his head was like pure wool. His throne burned with flaming fire, and its wheels burned with fire. \v{10}A river of fire flowed out from before him. Thousands upon thousands were serving him, with millions upon millions waiting before him. The court sat in judgment,\fnote{\fbackref{7:10} The Aram. lacks \fbib{in judgment}} and record books were unsealed.

\v{11}``I continued watching because of the audacious words that the horn was speaking. I kept observing until the animal was killed and its body destroyed and given over to burning fire. \v{12}Now as to the other animals, their authority was removed, but they were granted a reprieve from execution\fnote{\fbackref{7:12} Lit. \fbib{a prolonging of life}} for an appointed period of time.''
\passage{The Vision of the Son of Man}

\v{13}``I continued to observe the night vision---and look!---someone like the Son of Man was coming, accompanied by heavenly clouds. He approached the Ancient of Days and was presented before him. \v{14}To him dominion was bestowed, along with glory and a kingdom, so that all peoples, nations, and languages are to serve him. His dominion is an everlasting dominion---it will never pass away---and his kingdom is one that will never be destroyed.''
\passage{The Vision Interpreted}

\v{15}``Now as for me, Daniel, I was emotionally troubled, and what I had seen in the visions kept alarming me. \v{16}So I approached one of those who was standing nearby and began to ask the meaning of all of this. He spoke to me and caused me to understand the interpretation of these things. \v{17}He said, `These four great animals are four kings who will rise to power from the earth. \v{18}But the saints of the Highest will receive the kingdom forever, inheriting it\fnote{\fbackref{7:18} The Aram. lacks \fbib{inheriting it}} forever and ever.'

\v{19}``I wanted to learn the precise significance of the fourth animal that was different from all the others, extremely awe-inspiring, with iron teeth and bronze claws, and that had devoured and crushed things,\fnote{\fbackref{7:19} The Aram. lacks \fbib{things}} trampling under its feet whatever remained. \v{20}Also, I wanted to learn the significance of\fnote{\fbackref{7:20} The Aram. lacks \fbib{I wanted to learn the significance of}} the ten horns on its head and the other horn that had arisen, before which three of them had fallen---that is, the horn with eyes and a mouth that uttered magnificent things and which was greater in appearance than its fellows.

\v{21}``As I continued to watch, that same horn waged war against the saints, and was prevailing against them \v{22}until the Ancient of Days arrived to pass judgment in favor of the saints of the Highest One and the time came for the saints to take possession of the kingdom. \v{23}So he said:

\begin{poetry}
\poeml `The fourth animal will be a fourth kingdom on the earth, different from all the kingdoms. It will devour the entire earth, trampling it down and crushing it. \v{24}Now as to the ten horns, ten kings will rise to power from this kingdom, and another king\fnote{\fbackref{7:24} The Aram. lacks \fbib{king}} will rise to power after them. He will be different from the previous kings,\fnote{\fbackref{7:24} The Aram. lacks \fbib{kings}} and will defeat three kings. \v{25}He'll speak out against the Most High and wear down the saints of the Highest One. He'll attempt to alter times and laws, and they'll be given into his control for a time, times, and half a time. \v{26}Nevertheless, the court will convene, and his authority will be removed, annulled, and destroyed forever. \v{27}Then the kingdom, authority, and magnificence of all nations of the earth\fnote{\fbackref{7:27} Lit. \fbib{of the kingdoms under the whole heaven}} will be given to the people who are the saints of the Highest One. His kingdom will endure forever, and all authorities will serve him and obey him.'
\end{poetry}

\v{28}``At this point the vision ended. As for me, Daniel, my thoughts continued to alarm me, and I lost my natural color, but I kept quiet about the matter.''\fnote{\fbackref{7:28} Lit. \fbib{kept the matter in my heart}}
\labelchapt{8}
\passage{The Vision of the Ram and Goat}

\chapt{8}
\v{1}\fnote{\fbackref{8:1} At this point the text reverts to Heb. for the rest of the book.}``During the third year of King Belshazzar's reign, I, Daniel, saw a vision after the earlier vision that had appeared to me. \v{2}As I observed the vision, I looked around the citadel of Susa in Elam Province. While I watched, I found myself beside the Ulai Canal. \v{3}``Then I turned my head\fnote{\fbackref{8:3} Lit. \fbib{eyes}} to look, and to my surprise, a two-horned ram was standing beside the canal. The two horns grew long,\fnote{\fbackref{8:3} Or \fbib{higher}; Lit. \fbib{horns were exalted}} the first one growing longer than\fnote{\fbackref{8:3} Lit. \fbib{one exalted from}} the second, with the longer one springing up last. \v{4}I watched the ram charging westward, northward, and southward. No animal could stand before him, nor was there anyone who could deliver from his control.\fnote{\fbackref{8:4} Lit. \fbib{hand}} He did as he pleased and exalted himself.

\v{5}``As I watched and wondered, a male goat was coming from the west over the surface of the entire earth without touching the ground. The goat had a distinctive horn between its eyes. \v{6}It approached the ram with the two horns that I had observed while standing beside the canal, and charged at him, out of control with rage.\fnote{\fbackref{8:6} Lit. \fbib{him in his mighty wrath}} \v{7}I saw it approach the ram, overflowing with fury at him, and run into him with the full force of its strength. The goat\fnote{\fbackref{8:7} Lit. \fbib{It}} shattered the ram's\fnote{\fbackref{8:7} Lit. \fbib{shattered his}} two horns, and the ram could not oppose it. So the goat\fnote{\fbackref{8:7} Lit. \fbib{it}} threw him to the ground and trampled him. No one could rescue the ram from its control.\fnote{\fbackref{8:7} Lit. \fbib{hand}} \v{8}Then the goat grew extremely great, but when it was strong, its great horn was shattered. In its place, four distinctive horns grew out in all directions.''\fnote{\fbackref{8:8} Lit. \fbib{out to the four winds of heaven}}
\passage{The Insignificant Horn}

\v{9}``A somewhat insignificant horn emerged from one of them. It moved\fnote{\fbackref{8:9} Or \fbib{expanded} and so throughout the chapter} rapidly\fnote{\fbackref{8:9} Or \fbib{remarkably}} against the south, against the east, and against the Glory.\fnote{\fbackref{8:9} Or \fbib{Beauty}; i.e. God} \v{10}Then it moved against the Heavenly Army. It persuaded some of the Heavenly Army to fall to the earth, along with some of the stars, and it trampled them. \v{11}Then it set itself in arrogant opposition to the Prince of the Heavenly Army, from whom the regular burnt offering was taken away, in order to overthrow his sanctuary. \v{12}Because of the transgression, the Heavenly Army will be given over, along with the regular burnt offering, and in that rebellion truth will be cast to the ground, while he continues to prosper and to act.''
\passage{The Duration of the Desolation}

\v{13}``Then I heard one holy person speaking, and another holy person addressed the one who was speaking: `In the vision about the regular burnt offering, how much time elapses while the desecration terrifies and both the Holy Place and the Heavenly Army are trampled?'

\v{14}``He told me, `For 2,300 days.\fnote{\fbackref{8:14} Lit. \fbib{2,300 twilights and dawnings}} Then the Holy Place will be restored.'\,''
\passage{Gabriel Interprets the Vision}

\v{15}``After I, Daniel, had seen the vision, I tried to understand it. All of a sudden, there was standing in front of me one who appeared to be valiant. \v{16}I heard the voice of a man calling out from the Ulai Canal,\fnote{\fbackref{8:16} The Heb. lacks \fbib{Canal}} `Gabriel, interpret what that fellow has been seeing.'

\v{17}``As he approached where I was standing, I became terrified and fell on my face. But he told me, `Son of man, understand that the vision pertains to the time of the end.'

\v{18}``While he had been speaking with me, I had fainted\fnote{\fbackref{8:18} Lit. \fbib{had fallen into a deep sleep}} on my face, but he touched me and enabled me to stand upright on my feet. \v{19}Then he said,

\begin{poetry}
\poeml `Pay attention! I'm going to brief you about what will happen at the end of the period of wrath, because its end is appointed. \v{20}The ram that you saw with a pair of horns are the kings of Media and Persia. \v{21}The demonic\fnote{\fbackref{8:21} Lit. \fbib{shaggy}} goat is the king of Greece,\fnote{\fbackref{8:21} Lit. \fbib{Javan}} and the great horn between its eyes is its first king. \v{22}The shattered horn\fnote{\fbackref{8:22} The Heb. lacks \fbib{horn}} and the four that took its place are four kingdoms that will come from his nation, but they will not have his strength. \\
\poeml \v{23}``Toward the end of their rule, as the desecrations proceed, an insolent king will arise, proficient at deception. \v{24}Mighty will be his skills, but not from his own abilities. He'll be remarkably destructive, will succeed, and will do whatever he wants, destroying mighty men and the holy people. \v{25}Through his skill he'll cause deceit to prosper under his leadership. He'll promote himself and will destroy many while they are secure. He'll take a stand against the Prince of Princes, yet he'll be crushed without human help.\fnote{\fbackref{8:25} Lit. \fbib{without a hand}} \v{26}The vision about the twilights and dawnings that has been related is trustworthy, but keep its vision secret, because it pertains to the distant future.'
\end{poetry}

\v{27}Then I, Daniel, was exhausted and ill for days, but afterward I got up and went about the king's business. Nevertheless, I was astonished by the vision, and could not understand it.''
\labelchapt{9}
\passage{Daniel's Prayer}

\chapt{9}
\v{1}``In the first year of the reign of Darius son of Ahasuerus, a descendant of the Medes, who was made king over the kingdom of the Chaldeans\fnote{\fbackref{9:1} Or \fbib{Babylonians}}--- \v{2}in the first year of his reign I, Daniel, noted in the Scripture the total years that were assigned\fnote{\fbackref{9:2} The Heb. lacks \fbib{assigned}} by the message from the \divine{Lord} to Jeremiah the prophet for the completion of the desolations of Jerusalem: 70 years.

\v{3}``So I turned my attention to the Lord God, seeking him in prayer and supplication, accompanied with fasting, sackcloth, and ashes. \v{4}I prayed to the \divine{Lord} my God, confessing and saying:

\begin{poetry}
\poeml `Lord! Great and awesome God, who keeps his\fnote{\fbackref{9:4} The Heb. lacks \fbib{his}} covenant and gracious love for those who love him and obey his commandments, \v{5}we've sinned, we've practiced evil, we've acted wickedly, and we've rebelled, turning away from your commands and from your regulations. \v{6}Furthermore, we haven't listened to your servants, the prophets, who spoke in your name to our kings, to our officials, to our ancestors, and to all of the people of the land. \\
\poeml \v{7}`To you, Lord, belongs righteousness, but to us, open humiliation---even to this day, to the men of Judah, the residents of Jerusalem, and to all Israel, both those who are nearby and those who are far away in all the lands to which you drove them because of their unfaithful acts that they committed against you. \\
\poeml \v{8}`Open humiliation belongs to us, \divine{Lord}, to our kings, our officials, and our ancestors, because we've sinned against you. \v{9}But to the Lord our God belong mercy and forgiveness, though we've rebelled against him \v{10}and have not obeyed the voice of the \divine{Lord} our God by walking in his laws that he gave us through his servants the prophets. \v{11}And all Israel flouted your Law, turning aside from it and not obeying your voice. Because we've sinned against him, the curse has been poured upon us, along with the oath written in the Law of Moses the servant of God. \\
\poeml \v{12}`He has confirmed his accusation\fnote{\fbackref{9:12} Lit. \fbib{word}} that he spoke against us and against our rulers who governed us by bringing upon us great calamity, because nowhere in the universe\fnote{\fbackref{9:12} Lit. \fbib{because under all of the heavens}} has anything been done like what has been done to Jerusalem. \v{13}As it's written in the Law of Moses,\fnote{\fbackref{9:13} Cf. Lev. 26:14-15; Deut 28:15-68} all this calamity has befallen us, but we still haven't sought the \divine{Lord} our God by turning from our lawlessness to pay attention to your truth. \v{14}So the \divine{Lord} watched for the right time to bring the calamity upon us, because the \divine{Lord} our God is righteous regarding everything he does, but we have not obeyed his voice. \\
\poeml \v{15}`And now, Lord our God, who brought your people from the land of Egypt with a mighty hand and who made a name for yourself that remains to this day---we've sinned. We've acted wickedly. \v{16}Lord, in view of all your righteous acts, please turn your anger and wrath away from your city Jerusalem, your holy mountain. Because of our sins and the iniquities of our ancestors, Jerusalem and your people have become an embarrassment to all of those around us. \\
\poeml \v{17}`So now, O\fnote{\fbackref{9:17} Lit. \fbib{our}} God, listen to the prayer of your servant and to his requests, and look with favor on your desolate sanctuary, for the sake of the Lord. \v{18}Turn your ear and listen, O God. Open your eyes and look at our desolation and at the city that is called by your name. We're not presenting our requests before you because of our righteousness, but because of your great compassion. \\
\poeml \v{19}`Lord, listen! \\
\poeml `Lord, forgive! \\
\poeml `Lord, take note and take action! \\
\poeml `For your own sake, don't delay, my God, because your city and your people are called by your name.'\,''
\end{poetry}
\passage{Gabriel's Answer: The Seventy Weeks}

\v{20}``While I was still speaking in prayer, confessing my sin and the sin of my people Israel and placing my request in the presence of the \divine{Lord} my God on behalf of the holy mountain of God--- \v{21}while I was still speaking, Gabriel the man of God whom I had seen in the previous vision, appeared to me about the time of the evening offering. \v{22}He gave instructions, and this is what he spoke to me:

\begin{poetry}
\poeml `Daniel, I've now come to give you insight and understanding. \v{23}Because you're highly regarded, the answer was issued when you began your prayer, and I've come to tell you. Pay attention to my message and you'll understand the vision. \v{24}Seventy weeks\fnote{\fbackref{9:24} Lit. \fbib{sevens}; i.e. seven time periods of unspecified duration, and so through v. 27} have been decreed concerning your people and your holy city: to restrain transgression, to put an end to sin, to make atonement for lawlessness, to establish everlasting righteousness, to conclude vision and prophecy, and to anoint the Most Holy Place. \v{25}So be informed and discern that seven weeks and 62 weeks will elapse\fnote{\fbackref{9:25} The Heb. lacks \fbib{will elapse}} from the issuance of the command to restore and rebuild Jerusalem until the Anointed Commander.\fnote{\fbackref{9:25} Lit. \fbib{until Messiah Nagid}; i.e. a senior officer entrusted with dual roles of operational oversight and management authority} The plaza and moat will be rebuilt, though in troubled times. \v{26}Then after the 62 weeks, the anointed one\fnote{\fbackref{9:26} Or \fbib{the Messiah}} will be cut down (but not for himself).\fnote{\fbackref{9:26} Or \fbib{cut off, and will have no successor}; the Heb. lacks \fbib{successor}} Then the people of the Coming Commander\fnote{\fbackref{9:26} Lit. \fbib{Nagid}; i.e. a senior officer entrusted with dual roles of operational oversight and management authority} will destroy both the city and the Sanctuary. Its ending will come like a flood, and until the end there will be war, with desolations having been decreed. \v{27}He will make a binding covenant with many for one week, and for half of the week he will suspend both the sacrifice and grain offerings. Destructive people will cause desolation on the pinnacle until it is complete and what has been decreed is poured out on the desolator.'\,''
\end{poetry}
\labelchapt{10}
\passage{Daniel's Vision}

\chapt{10}
\v{1}In the third year of Cyrus, king of Persia, a message was revealed to Daniel (also known as Belteshazzar). The message was trustworthy and concerned a great conflict. He understood it and had insight concerning the vision.

\v{2}``At that time I, Daniel, had been mourning for three straight weeks.\fnote{\fbackref{10:2} Lit. \fbib{for three weeks of days}} \v{3}I ate no fancy foods---neither meat nor wine entered my mouth. Furthermore, I didn't use any ointment until the end of the entire three weeks.\fnote{\fbackref{10:3} Lit. \fbib{the three weeks of days}} \v{4}On the twenty-fourth day of the first month, while I was beside the bank of the great Tigris\fnote{\fbackref{10:4} Lit. \fbib{Hiddekel}} River, \v{5}I lifted up my eyes to look, and to my surprise, there was a certain man dressed in linen, whose waist was encircled with gold from Uphaz! \v{6}His body was like beryl,\fnote{\fbackref{10:6} Lit. \fbib{Tarshish}; a yellow semi-precious stone named after its region of origin} his face flashed like lightning, his eyes were like flaming torches, his arms and legs were like polished bronze, and his speech roared\fnote{\fbackref{10:6} The Heb. lacks \fbib{roared}} like that of a crowd.

\v{7}``Now I, Daniel, was the only one to receive the vision---the men who were with me didn't see it.\fnote{\fbackref{10:7} Lit. \fbib{see the vision}} However, an enormous fear overwhelmed them, so they ran away to hide, \v{8}and I was left alone to observe this magnificent vision. Nevertheless, no strength remained in me---my face lost its color, and I became weak. \v{9}As I listened to the sound of his words, I fell down on my face unconscious, with my face to the ground.''
\passage{Daniel is Given Understanding}

\v{10}``All of a sudden, a hand touched me and lifted me upon my hands and knees. \v{11}He told me, `Daniel, man highly regarded, understand the message that I'm about to relate to you. Stand up, because I've been sent to you.' When he spoke this statement to me, I stood there trembling.

\v{12}```Don't be afraid, Daniel,' he told me, `because from the first day that you committed yourself to understand and to humble yourself before your God, your words were heard. I've come in answer to\fnote{\fbackref{10:12} The Heb. lacks \fbib{answer to}} your prayers. \v{13}However, the prince of the kingdom of Persia opposed me for 21 days. Then all of a sudden, Michael, one of the chief angels,\fnote{\fbackref{10:13} Lit. \fbib{princes}} came to assist me! I had been detained there near the kings of Persia. \v{14}Now I've come to help you understand what will happen to your people in the days to come, because the vision pertains to those days.'

\v{15}``After he had spoken to me like this, I bowed my face to the ground, unable to speak. \v{16}But suddenly someone who resembled a human being touched my lips, so addressing the one who was standing in front of me, I opened my mouth and said, `Sir,\fnote{\fbackref{10:16} Lit. \fbib{My lord}} I'm overwhelmed with anguish by this vision. I have no strength left.\fnote{\fbackref{10:16} The Heb. lacks \fbib{left}} \v{17}So how can a servant of my lord talk with someone like you, sir?\fnote{\fbackref{10:17} Lit. \fbib{like my lord}} And as for me, there's no strength left in me, and I can hardly breathe.'

\v{18}``Then this person who looked like a man touched me again and strengthened me \v{19}and said, `Don't be afraid, man highly regarded. Be at peace, and be strong.'

``As soon as he spoke to me, I gained strength and replied, `Sir, please\fnote{\fbackref{10:19} Lit. \fbib{May my lord}} speak, now that you've strengthened me.'

\v{20}``Then he said, `Do you understand why I came to you? Soon I'll return to fight the prince of Persia. I'm going forth to war---and take note---the prince of Greece\fnote{\fbackref{10:20} Lit. \fbib{Javan}} is coming! \v{21}I'll inform you about what has been recorded in the Book of Truth. No one stands firmly with me against these opponents,\fnote{\fbackref{10:21} The Heb. lacks \fbib{opponents}} except Michael your prince.\chapt{11}
\v{1}In year one of King Darius the Mede, I arose to fortify and strengthen him.'\,''
\labelchapt{11}
\passage{International Conflicts to Come}

\v{2}```Now I'll tell you the truth: Pay attention! Three more kings will arise in Persia. Then a fourth will gain more than them all. As soon as he gains power by means of his wealth, he'll stir up everyone against the Grecian kingdoms.

\v{3}```A mighty king will come to power, and he'll rule with awesome energy, doing whatever he pleases. \v{4}However, after he has come to power, his kingdom will be broken and parceled out in all directions.\fnote{\fbackref{11:4} Lit. \fbib{out to the four winds of heaven}} It won't go to his succeeding descendants, nor will its power match how he ruled, because his sovereignty will be uprooted and given to successors besides them.

\v{5}```The southern king will become strong, along with one of his officials, who will become stronger than he and rule over his own realm with great power. \v{6}After a number of years, they'll become allies and the daughter of the southern king will go to the northern king in order to craft alliances. But she won't remain in power, nor will he retain his power. Instead, she'll be surrendered, along with her entourage, the one who fathered her, and the one who supported her at that time.

\v{7}```One of her family line will replace him. He'll come against the army and enter the fortress of the northern king, conquering them and becoming victorious. \v{8}He'll also take their gods, their molten images, and their valuable vessels of silver and gold into Egypt as hostages. He'll avoid the northern king for a number of years. \v{9}Then he'll come against the realm of the southern king and then return to his own territory. \v{10}His sons will prepare for war, assembling an army of considerable force. One of them will come on forcefully, overflowing, passing through, and waging war up to his own fortress.

\v{11}```The southern king will fly into a rage and march out to fight the northern king. He'll gather a large army, but that army will be handed over to him. \v{12}When that army has been defeated, he'll become overconfident and slaughter many thousands, but he won't succeed. \v{13}The northern king will return and raise a greater army than before. After a few years, he'll advance with a great force and with a vast amount of armaments.'\,''
\passage{Rebellion against the Southern King}

\v{14}```During those years, many will rebel against the southern king. The more violent ones among your people will rebel in order to fulfill this vision, but they will fail. \v{15}Then the northern king will come, erect a siege ramp, and capture a fortified city. The southern forces won't prevail---not even with their best troops---and they'll have no strength to take a stand.

\v{16}```However, the one who invades him will do whatever he wants to do. No one will oppose him. He'll establish himself in the Beautiful Land, wielding devastating power. \v{17}He'll decide to come with the full power of his kingdom, bringing with him an alliance that he'll implement. He'll give him a daughter in marriage to overthrow it, but it won't succeed or work out for him. \v{18}Then he'll turn his attention to the coastal lands\fnote{\fbackref{11:18} Or \fbib{islands}} and will capture many. But a commander will put an end to his insolence, repaying him for his scorn. \v{19}He'll turn his attention toward the fortresses in his own territory, but he'll stumble and fall, and won't endure. \v{20}His successor will send out a tax collector for royal splendor, but in a short period of time he'll be shattered, though neither in anger nor in battle.'\,''
\passage{The Despicable King}

\v{21}```In his place there will arise a despicable person, upon whom no royal authority has been conferred, but he'll invade in a time of tranquility, taking over the kingdom through deception. \v{22}Overwhelming forces will be carried away before him, along with the Commander-in-Chief\fnote{\fbackref{11:22} Lit. \fbib{Nagid}; i.e. a senior officer entrusted with dual roles of operational oversight and administrative authority} of the covenant. \v{23}From the time that an alliance is made with him, he'll act deceitfully, and he will go up and take power with only a small group of nations. \v{24}He'll invade the most prosperous areas of the province during a time of tranquility, accomplishing what neither his predecessors nor his ancestors ever could. He'll distribute war spoils, booty, and wealth to them, and he'll plot the overthrow of fortresses, though only for a time. \v{25}He'll encourage himself against the southern king by raising\fnote{\fbackref{11:25} The Heb. lacks \fbib{raising}} a large army. As a result, the southern king will mobilize for war with a large and powerful army, but he won't succeed because they will devise elaborate schemes against him. \v{26}His own security detail\fnote{\fbackref{11:26} Lit. \fbib{Those who eat his delicacies}} will undermine him, his army will be swept away, and many will fall and be killed in battle.\fnote{\fbackref{11:26} The Heb. lacks \fbib{in battle}} \v{27}Now as for the two kings, their intentions will be evil, and they'll promote deception at their dinner table, but none of this will succeed, because the end won't have come yet. \v{28}Then he'll return to his homeland with great wealth, will focus his attention against the holy covenant, and will take action as he returns to his land.'\,''
\passage{Desecration of the Sanctuary}

\v{29}```At the scheduled time he'll return, moving southward, but the end result won't be as before, \v{30}because ships will come against him from the Mediterranean islands.\fnote{\fbackref{11:30} Lit. \fbib{from Kittim}} Disheartened, he'll return, incited to vehemence against the holy covenant, and he'll take action. As he returns, he'll show deference to those who abandon the holy covenant. \v{31}Armed forces will arise from his midst, and they'll desecrate the fortified Sanctuary, abolish the daily sacrifice, and establish the destructive desecration. \v{32}Through flattery he'll corrupt those who act wickedly toward the covenant, but people who know their God will be strong and take action. \v{33}Insightful people\fnote{\fbackref{11:33} I.e. believers; cf. v. 35} will impart understanding to many, though they'll fall by sword, by fire, by captivity, and as war booty for a while.\fnote{\fbackref{11:33} Lit. \fbib{for days}} \v{34}When they fall, they'll be given some relief, but many will join them by pretending to be sympathetic to their cause. \v{35}Some of the insightful will fall so they may be refined, purged, and purified until the time of the end, since it will surely come about.'\,''
\passage{The King Who Calls Himself God}

\v{36}```The king will do as he pleases. He'll exalt and magnify himself above every god, speaking amazing things against the God of Gods. He'll succeed until the indignation is completed, because what has been determined must be carried out. \v{37}He'll recognize neither the gods of his ancestors nor those desired by women---he won't recognize any god, because he'll exalt himself above everything. \v{38}He'll glorify the god of fortresses,\fnote{\fbackref{11:38} Or \fbib{forces}} a god whom his ancestors never knew, honoring him with gold, silver, valuable jewels, and treasures. \v{39}He'll take action against the strongest fortresses. With the help of a foreign god, he'll recognize those who honor him, making them rule over many, and he'll parcel out the land for a profit.

\v{40}```At the time of the end, the southern king will oppose him, and the northern king will overrun him with chariots, cavalry, and many ships. He'll invade countries, moving swiftly and sweeping through. \v{41}He'll enter the Beautiful Land, and many will fall, even though these will escape his control: Edom, Moab, and certain Ammonite officials.

\v{42}He'll extend his power over other countries, and even the land of Egypt won't escape. \v{43}He'll capture treasures of gold, silver, and all the treasures of Egypt, with the Libyans and Cushites\fnote{\fbackref{11:43} Or \fbib{Nubians}} at his feet. \v{44}However, reports from the east and the north will alarm him, and he'll march out in great anger, intending to destroy and to desolate many. \v{45}When he pitches his royal pavilions between the seas\fnote{\fbackref{11:45} I.e. between the Mediterranean Sea and the Dead Sea} facing the mountain of holy Glory, he'll come to his end, and no one will help him.'\,''
\labelchapt{12}
\passage{The End Times}

\chapt{12}
\v{1}```At that time, Michael\fnote{\fbackref{12:1} Or \fbib{time, the One who is like God}; i.e. the Messiah} will arise, the great prince who will stand up on behalf of your people, and a time of trouble will come like there has never been since nations began until that time. Also at that time, your people will be delivered---everyone who will have been written in the book. \v{2}Many of those who are sleeping in the dust of the earth will awaken---some to life everlasting, and some to disgrace and everlasting contempt. \v{3}Those who manifest wisdom will shine like the brightness of the expanse of heaven, and those who turn many to righteousness will shine\fnote{\fbackref{12:3} The Heb. lacks \fbib{will shine}} like the stars for ever and ever. \v{4}Now as for you, Daniel, roll up your scroll and seal your words until the time of the end. Many will rush around, while knowledge increases.'\,''
\passage{The Vision of the Two Speakers}

\v{5}``Then while I, Daniel, continued watching, suddenly two others stood there, one on this side of the river bank and one on the other side. \v{6}One asked the man dressed in linen clothes, who was standing\fnote{\fbackref{12:6} The Heb. lacks \fbib{standing}} above the waters of the river, `How long until the fulfillment of the wonders?'

\v{7}``I heard the man dressed in linen clothes, who was standing\fnote{\fbackref{12:7} The Heb. lacks \fbib{standing}} above the waters of the river as he lifted his right and left hands to heaven and swore by the one who lives forever that it would be for a time, times, and a half. When the shattering of the power of the holy people has occurred, all these things will conclude.''
\passage{Daniel's Unanswered Question}

\v{8}``I heard, but I didn't understand. So I asked, `Sir,\fnote{\fbackref{12:8} Lit. \fbib{My lord}} what happens next?'

\v{9}``He answered, `Go on your way, Daniel, because these matters\fnote{\fbackref{12:9} Or \fbib{words}} are wrapped up and sealed until the time of the end. \v{10}Many will be purified, cleansed, and refined, though the wicked will continue to act wickedly, and none of the wicked will understand. Nevertheless, the insightful\fnote{\fbackref{12:10} Or \fbib{wise}} will understand. \v{11}There will be\fnote{\fbackref{12:11} The Heb. lacks \fbib{There will be}} 1,290 days from the time the regular burnt offering\fnote{\fbackref{12:11} Or \fbib{sacrifice}} is rescinded and the destructive desolation established. \v{12}Blessed is the one who perseveres and attains to the 1,335 days. \v{13}Now as for you, keep on going until the end---you'll rest and then rise to receive your reward at the end of the age.'\,''\fnote{\fbackref{12:13} Lit. \fbib{days}}
