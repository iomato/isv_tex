\bookheader{Hebrews}
\labelbook{Heb}

\bookpretitle{The Letter to the}
\booktitle{Hebrews}

\labelchapt{1}
\passage{God Has Spoken to Us}

\chapt{1}
\v{1}God, having spoken in former times in fragmentary and varied fashion to our forefathers by the prophets, \v{2}has in these last days spoken to us by a Son whom he appointed to be the heir of everything and through whom he also made the universe. \v{3}He is the reflection\fnote{\fbackref{1:3} Or \fbib{radiance}} of God's glory and the exact likeness of his being, and he holds everything together by his powerful word. After he had provided a cleansing from sins, he sat down at the right hand of the Highest Majesty \v{4}and became as much superior to the angels as the name he has inherited is better than theirs.
\passage{God's Son is Superior to the Angels}

\v{5}For to which of the angels did God\fnote{\fbackref{1:5} Lit. \fbib{he}} ever say, ``You are my Son. Today I have become your Father''?\fnote{\fbackref{1:5} Cf. Ps 2:7} Or again, ``I will be his Father, and he will be my Son''?\fnote{\fbackref{1:5} Cf. 2 Sam 7:14} \v{6}And again, when he brings\fnote{\fbackref{1:6} Or \fbib{And when he again brings}} his firstborn into the world, he says, ``Let all God's angels worship him.''\fnote{\fbackref{1:6} Deut 32:43 (LXX); Ps 97:7} \v{7}Now about the angels he says,

\begin{poetry}
\poeml ``He makes his angels winds,
\end{poetry}

and his servants flames of fire.''\fnote{\fbackref{1:7} Cf. Ps 104:4}

\v{8}But about the Son he says,

\begin{poetry}
\poeml ``Your throne, O God, \\
\poemll    is forever and ever, \\
\poeml and the scepter of your kingdom \\
\poemll    is a righteous scepter. \\
\poeml \v{9}You have loved righteousness \\
\poemll    and hated wickedness. \\
\poeml That is why God, your God, \\
\poemll    anointed you rather than your companions \\
\poemlll       with the oil of gladness.''\fnote{\fbackref{1:9} Cf. Ps 45:6-7}
\end{poetry}

\v{10}And,

\begin{poetry}
\poeml ``In the beginning, Lord,\fnote{\fbackref{1:10} MT source citation reads \fbib{}\divine{Lord}} \\
\poemll    you laid the foundation of the earth, \\
\poemlll       and the heavens are the work of your hands. \\
\poeml \v{11}They will come to an end, \\
\poemll    but you will remain forever. \\
\poemlll       They will all wear out like clothes. \\
\poeml \v{12}You will roll them up like a robe, \\
\poemll    and they will be changed like clothes. \\
\poeml But you remain the same, \\
\poemll    and your life\fnote{\fbackref{1:12} Lit. \fbib{years}} will never end.''\fnote{\fbackref{1:12} Cf. Ps 102:25-27}
\end{poetry}

\v{13}But to which of the angels did he ever say,

\begin{poetry}
\poeml ``Sit at my right hand \\
\poemll    until I make your enemies a footstool for your feet''?\fnote{\fbackref{1:13} Cf. Ps 110:1}
\end{poetry}

\v{14}All of them are spirits on a divine mission, sent to serve those who are about to inherit salvation, aren't they?
\labelchapt{2}
\passage{We Must Not Neglect Our Salvation}

\chapt{2}
\v{1}For this reason we must pay closer attention to the things we have heard, or we may drift away, \v{2}because if the message spoken by angels was reliable, and every violation and act of disobedience received its just punishment, \v{3}how will we escape if we neglect a salvation as great as this? It was first proclaimed by the Lord himself, and then it was confirmed to us by those who heard him, \v{4}while God added his testimony through signs, wonders, various miracles, and gifts of the Holy Spirit distributed according to his will.
\passage{Jesus is the Source of Our Salvation}

\v{5}For he did not put the coming world we are talking about under the control of angels. \v{6}Instead, someone has declared somewhere,

\begin{poetry}
\poeml ``What is man that you should remember him, \\
\poemll    or the son of man that you should care for him? \\
\poeml \v{7}You made him a little lower than the angels, \\
\poemll    yet you crowned him with glory and honor \\
\poeml \v{8}and put everything under his feet.''\fnote{\fbackref{2:8} Cf. Ps 8:5-7 (LXX)}
\end{poetry}

Now when God\fnote{\fbackref{2:8} Lit. \fbib{he}} put everything under him, he left nothing outside his control. However, at the present time we do not yet see everything put under him. \v{9}But we do see someone who was made a little lower than the angels. He is Jesus, who is crowned with glory and honor because he suffered death, so that by the grace of\fnote{\fbackref{2:9} Other mss. read \fbib{so that apart from}} God he might experience\fnote{\fbackref{2:9} Lit. \fbib{taste}} death for everyone.

\v{10}It was fitting that God, for whom and through whom everything exists, should make the pioneer of their salvation perfect through suffering as part of his plan to glorify many children, \v{11}because both the one who sanctifies and those who are being sanctified all have the same Father.\fnote{\fbackref{2:11} Lit. \fbib{are all of one}} That is why Jesus\fnote{\fbackref{2:11} Lit. \fbib{he}} is not ashamed to call them brothers \v{12}when he says, ``I will announce your name to my brothers. I will praise you within the congregation.''\fnote{\fbackref{2:12} Cf. Ps 22:22} \v{13}And again, ``I will trust him.''\fnote{\fbackref{2:13} Cf. Isa 8:17 (LXX)} And again, ``I am here with the children God has given me.''\fnote{\fbackref{2:13} Cf. Isa 8:18}

\v{14}Therefore, since the children have flesh and blood, he himself also shared the same things, so that by his death he might destroy the one who has the power of death (that is, the devil) \v{15}and might free those who were slaves all their lives because they were terrified by death. \v{16}For it is clear that he did not come to help angels. No, he came to help Abraham's descendants, \v{17}thereby becoming like his brothers in every way, so that he could be a merciful and faithful high priest in service to God and could atone for the people's sins. \v{18}Because he himself suffered when he was tempted, he is able to help those who are being tempted.
\labelchapt{3}
\passage{The Messiah is Superior to Moses}

\chapt{3}
\v{1}Therefore, holy brothers, partners in a heavenly calling, keep your focus on Jesus, the apostle and high priest of our confession. \v{2}He was faithful to the one who appointed him, just as Moses was in all God's\fnote{\fbackref{3:2} Lit. \fbib{his}} household, \v{3}because he is worthy of greater glory than Moses in the same way that the builder of a house has greater honor than the house itself. \v{4}After all, every house is built by someone, but God is the builder of everything. \v{5}Moses was faithful in all God's\fnote{\fbackref{3:5} Lit. \fbib{his}} household as a servant who was to testify to what would be said later, \v{6}but the Messiah\fnote{\fbackref{3:6} Or \fbib{Christ}} was faithful\fnote{\fbackref{3:6} The Gk. lacks \fbib{was faithful}} as the Son in charge of God's\fnote{\fbackref{3:6} Lit. \fbib{his}} household, and we are his household if we hold on to our courage and the hope in which we rejoice.\fnote{\fbackref{3:6} Lit. \fbib{the boast of our hope}}
\passage{A Rest for the People of God}

\v{7}Therefore, as the Holy Spirit says,

\begin{poetry}
\poeml ``Today, if you hear his voice, \\
\poeml \v{8}do not harden your hearts \\
\poeml as they did when they provoked me \\
\poemll    during the time of testing in the wilderness. \\
\poeml \v{9}There your ancestors tested me, \\
\poemll    even though they had seen my actions \v{10}for 40 years. \\
\poeml That is why I was indignant with that generation and said, \\
\poemll    `They are always going astray in their hearts, \\
\poemlll       and they have not known my ways.' \\
\poeml \v{11}So in my anger I swore a solemn oath \\
\poemll    that they would never enter my rest.''\fnote{\fbackref{3:11} Cf. Ps 95:7-11}
\end{poetry}

\v{12}See to it, my brothers, that no evil, unbelieving heart is found in any of you, as shown by your turning away from the living God. \v{13}Instead, continue to encourage one another every day, as long as it is called ``Today,'' so that none of you may be hardened by the deceitfulness of sin, \v{14}because we are the Messiah's\fnote{\fbackref{3:14} Or \fbib{Christ's}} partners only if we hold on to our original confidence to the end.\fnote{\fbackref{3:14} Other mss. lack \fbib{to the end}} \v{15}As it is said,

\begin{poetry}
\poeml ``Today, if you hear his voice, \\
\poemll    do not harden your hearts \\
\poemlll       as they did when they provoked me.''\fnote{\fbackref{3:15} Cf. Ps 95:7-8}
\end{poetry}

\v{16}Now who heard him and provoked him? Was it not all those who came out of Egypt led\fnote{\fbackref{3:16} The Gk. lacks \fbib{led}} by Moses? \v{17}And with whom was he angry for 40 years? Was it not with those who sinned and whose bodies fell dead in the wilderness? \v{18}And to whom did he swear that they would never enter his rest? It was to those who disobeyed him, was it not? \v{19}So we see that they were unable to enter because of their unbelief.
\labelchapt{4}
\passage{We Must Enter the Rest}

\chapt{4}
\v{1}Therefore, as long as the promise of entering his rest remains valid, let us be afraid! Otherwise, some of you will fail\fnote{\fbackref{4:1} Lit. \fbib{afraid lest someone among you fails}} to reach it, \v{2}because we have had the good news told to us as well as to them. But the message they heard did not help them, because they were not united by faith with those who listened to it. \v{3}We who have believed are entering that rest, just as he has said,

\begin{poetry}
\poeml ``So in my anger I swore a solemn oath \\
\poemll    that they would never enter my rest,''\fnote{\fbackref{4:3} Cf. Ps 95:11}
\end{poetry}

even though his actions had been finished since the creation\fnote{\fbackref{4:3} Lit. \fbib{foundation}; or \fbib{beginning}} of the world. \v{4}Somewhere he has spoken about the seventh day as follows: ``On the seventh day God rested from all his actions,''\fnote{\fbackref{4:4} Cf. Gen 2:2} \v{5}and again in this passage,\fnote{\fbackref{4:5} The Gk. lacks \fbib{passage}} ``They will never enter my rest.''\fnote{\fbackref{4:5} Cf. Ps 95:11} \v{6}Therefore, since it is still true that some will enter it, and since those who once heard the good news failed to enter it because of their disobedience, \v{7}he again fixes a definite day---``Today''---saying long afterward through David, as already quoted,

\begin{poetry}
\poeml ``Today, if you hear his voice, \\
\poemll    do not harden your hearts.''\fnote{\fbackref{4:7} Ps 95:7-8}
\end{poetry}

\v{8}For if Joshua\fnote{\fbackref{4:8} The Gk. name \fbib{Jesus} appears to be a word play on the Heb. name \fbib{Joshua}.} had given them rest, he would not have spoken later about another day.

\v{9}There remains, therefore, a Sabbath rest for the people of God to keep, \v{10}because the one who enters God's\fnote{\fbackref{4:10} Lit. \fbib{his}} rest has himself rested from his own actions, just as God did\fnote{\fbackref{4:10} The Gk. lacks \fbib{did}} from his. \v{11}Let us, therefore, make every effort to enter that rest, so that no one may fail by following their example of disobedience. \v{12}For the word of God is living and active, sharper than any double-edged sword, piercing until it divides soul and spirit, joints and marrow, as it judges the thoughts and purposes of the heart. \v{13}No creature can hide from him, but everyone is exposed and helpless before the eyes of the one to whom we must give a word of explanation.
\passage{Our Compassionate High Priest}

\v{14}Therefore, since we have a great high priest who has gone to heaven, Jesus the Son of God, let us live our lives consistent with\fnote{\fbackref{4:14} Lit. \fbib{us hold tightly to}} our confession of faith.\fnote{\fbackref{4:14} The Heb. lacks \fbib{of faith}} \v{15}For we do not have a high priest who is unable to sympathize with our weaknesses. Instead, we have one who in every respect has been tempted as we are, yet he never sinned. \v{16}So let us keep on coming boldly to the throne of grace, so that we may obtain mercy and find grace to help us in our time of need.
\labelchapt{5}
\passage{Qualifications for the Priesthood}

\chapt{5}
\v{1}For every high priest selected from among men is appointed to officiate on their behalf\fnote{\fbackref{5:1} Lit. \fbib{on behalf of men}} in matters relating to God, that is, to offer gifts and sacrifices for sins. \v{2}He can deal gently with people who are ignorant and easily deceived, since he himself is subject to weakness. \v{3}For that reason he is obligated to offer sacrifices for his own sins as well as for those of the people. \v{4}No one takes this honor upon himself but he is called to it by God, just as Aaron was.
\passage{The Messiah's Qualifications as High Priest}

\v{5}In the same way, the Messiah\fnote{\fbackref{5:5} Or \fbib{Christ}} did not take upon himself the glory of being a high priest. No, it was God who said\fnote{\fbackref{5:5} Lit. \fbib{He said}} to him,

\begin{poetry}
\poeml ``You are my Son. \\
\poemll    Today I have become your Father.''\fnote{\fbackref{5:5} Cf. Ps 2:7}
\end{poetry}

\v{6}As he also says in another place,

\begin{poetry}
\poeml ``You are a priest forever \\
\poemll    according to the order of Melchizedek.''\fnote{\fbackref{5:6} Cf. Ps 110:4}
\end{poetry}

\v{7}As a mortal man,\fnote{\fbackref{5:7} Lit. \fbib{During the days of his flesh}} he offered up prayers and appeals with loud cries and tears to the one who was able to save him from death, and he was heard because of his devotion to God. \v{8}Son though he was, he learned obedience through his sufferings \v{9}and, once made perfect, he became the source of eternal salvation for all who obey him, \v{10}having been designated by God to be a high priest according to the order of Melchizedek.
\passage{You Still Need Someone to Teach You}

\v{11}We have much to say about this,\fnote{\fbackref{5:11} Or \fbib{about him}} but it is difficult to explain because you have become too lazy to understand. \v{12}In fact, though by now you should be teachers, you still need someone to teach you the basic truths of God's word.\fnote{\fbackref{5:12} Or \fbib{oracles}} You have become people who need milk instead of solid food. \v{13}For everyone who lives on milk is still a baby and does not yet know the difference between right and wrong.\fnote{\fbackref{5:13} Lit. \fbib{and is inexperienced in the message of righteousness}} \v{14}But solid food is for mature people, whose minds are trained by practice to distinguish good from evil.
\labelchapt{6}
\passage{The Peril of Immaturity}

\chapt{6}
\v{1}Therefore, leaving behind the elementary teachings about the Messiah,\fnote{\fbackref{6:1} Or \fbib{Christ}} let us continue to be carried along to maturity, not laying again a foundation of repentance from dead actions, faith toward God, \v{2}instruction about baptisms, the laying on of hands, the resurrection of the dead, and eternal judgment. \v{3}And this we will do,\fnote{\fbackref{6:3} Other mss. read \fbib{Let us do this}} if God permits.

\v{4}For it is impossible to keep on restoring to repentance time and again people who have once been enlightened, who have tasted the heavenly gift, who have become partners with the Holy Spirit, \v{5}who have tasted the goodness of God's word and the powers of the coming age, \v{6}and who have fallen away, as long as they continue to crucify the Son of God to their own detriment by exposing him to public ridicule. \v{7}For when the ground soaks up rain that often falls on it and continues producing vegetation useful to those for whom it is cultivated, it receives a blessing from God. \v{8}However, if it continues to produce thorns and thistles, it is worthless and in danger of being cursed, and in the end will be burned.
\passage{Be Diligent}

\v{9}Even though we speak like this, dear friends, we are convinced of better things in your case, things that point to salvation. \v{10}For God is not so unjust as to forget your work and the love you have shown him\fnote{\fbackref{6:10} Lit. \fbib{shown for his name}} as you have ministered to the saints and continue to minister to them. \v{11}But we want each of you to continue to be diligent to the very end, in order to give full assurance to your hope. \v{12}Then, instead of being lazy, you will imitate those who are inheriting the promises through faith and patience.
\passage{God's Promise is Reliable}

\v{13}For when God made his promise to Abraham, he swore an oath by himself, since he had no one greater to swear by. \v{14}He said, ``I will certainly bless you and give you many descendants.''\fnote{\fbackref{6:14} Cf. Gen 22:17} \v{15}And so he obtained what he had been promised, because he patiently waited for it. \v{16}For people swear by someone greater than themselves, and an oath given as confirmation puts an end to all argument. \v{17}In the same way, when God wanted to make the unchangeable character of his purpose perfectly clear to the heirs of his promise, he guaranteed it with an oath, \v{18}so that by these two unchangeable things, in which it is impossible for God to prove false, we who have taken refuge in him might be encouraged to seize the hope set before us. \v{19}That hope,\fnote{\fbackref{6:19} The Gk. lacks \fbib{hope}} firm and secure like an anchor for our souls, reaches behind the curtain \v{20}where Jesus, our forerunner, has gone on our behalf, having become a high priest forever according to the order of Melchizedek.
\labelchapt{7}
\passage{The Messiah is Superior to Melchizedek}

\chapt{7}
\v{1}Now this man Melchizedek, king of Salem and priest of the Most High God, met Abraham and blessed him when he was returning from defeating the kings. \v{2}Abraham gave Melchizedek\fnote{\fbackref{7:2} Lit. \fbib{him}} a tenth of everything.\fnote{\fbackref{7:2} Cf. Gen 14:18-20} In the first place, his name means ``king of righteousness,'' and then he is also king of Salem, that is, ``king of peace.'' \v{3}He has no father, mother, or genealogy, no birth date recorded for him, nor a date of death.\fnote{\fbackref{7:3} Lit. \fbib{had neither beginning of days nor end of life}} Like the Son of God, he continues to be a priest forever.

\v{4}Just look at how great this man was! Even Abraham---the patriarch himself---gave him a tenth of what he had captured! \v{5}The descendants of Levi who accept the priesthood have a commandment in the Law to collect a tenth from the people, that is, from their own brothers, even though they are also descendants of Abraham. \v{6}But this man, whose descent is not traced from them, collected a tenth from Abraham and blessed the man who had received the promises. \v{7}It is beyond dispute that the less important person is blessed by the more important person. \v{8}Mortal men collect tithes, but we are informed by Scripture\fnote{\fbackref{7:8} The Gk. lacks \fbib{by Scripture}} that\fnote{\fbackref{7:8} Or \fbib{it is declared that}} Melchizedek\fnote{\fbackref{7:8} Lit. \fbib{he}} keeps on living. \v{9}One might even say that Levi, who collects the tenth, paid the tenth through Abraham, \v{10}because Levi\fnote{\fbackref{7:10} Lit. \fbib{he}} was still inside his ancestor when Melchizedek met him.

\v{11}Now if perfection could have been attained through the Levitical priesthood---for on this basis the people received the Law---what further need would there be to speak of appointing another kind of priest according to the order of Melchizedek, not one according to the order of Aaron? \v{12}When a change in the priesthood takes place, there must also be a change in the Law. \v{13}For the person we are talking about belonged to a different tribe, and no one from that tribe has ever served\fnote{\fbackref{7:13} Lit. \fbib{from which no one has served}} at the altar. \v{14}Furthermore, it is obvious that our Lord was a descendant of Judah, and Moses said nothing about priests coming from that tribe. \v{15}This point is even more obvious in that another priest who is like Melchizedek has appeared \v{16}who was appointed to be a priest,\fnote{\fbackref{7:16} The Gk. lacks \fbib{to be a priest}} not on the basis of a genealogical registry, but rather on the power of an indestructible life. \v{17}For it is declared about him,

\begin{poetry}
\poeml ``You are a priest forever \\
\poemll    according to the order of Melchizedek.''\fnote{\fbackref{7:17} Cf. Ps 110:4}
\end{poetry}

\v{18}Indeed, because it was weak and ineffective, the former commandment has been annulled, \v{19}since the Law made nothing perfect, and a better hope is presented, by which we approach God.

\v{20}Now none of this happened without an oath. Others became priests without any oath, \v{21}but Jesus\fnote{\fbackref{7:21} Lit. \fbib{he}} became a priest\fnote{\fbackref{7:21} The Gk. lacks \fbib{became a priest}} with an oath when God\fnote{\fbackref{7:21} Lit. \fbib{he}} told him,

\begin{poetry}
\poeml ``The Lord\fnote{\fbackref{7:21} MT source citation reads \fbib{}\divine{Lord}} has taken an oath \\
\poemll    and will not change his mind. \\
\poeml You are a priest forever.''\fnote{\fbackref{7:21} Cf. Ps 110:4}
\end{poetry}

\v{22}In this way, Jesus has become the guarantor of a better covenant.

\v{23}There have been many priests, since each one of them had to stop serving in office when he died. \v{24}But because Jesus\fnote{\fbackref{7:24} Lit. \fbib{he}} lives forever, he has a permanent priesthood. \v{25}Therefore, because he always lives to intercede for them, he is able to save completely\fnote{\fbackref{7:25} Or \fbib{thoroughly}} those who come to God through him.

\v{26}We need such a high priest---one who is holy, innocent, pure, set apart from sinners, exalted above the heavens. \v{27}He has no need to offer sacrifices every day like high priests do, first for his own sins and then for those of the people, since he did this once for all when he sacrificed himself. \v{28}For the Law appoints as high priests men who are weak, but the promised oath, which came after the Law, results in a Son who is eternally perfect.
\labelchapt{8}
\passage{The Messiah Has a Better Ministry}

\chapt{8}
\v{1}Now the main point in what we are saying is this: we do have this kind of high priest, who sat down at the right hand of the throne of the Majesty in heaven \v{2}and who serves in the sanctuary, the true tent set up by the Lord and not by any human. \v{3}For every high priest is appointed to offer both gifts and sacrifices. Therefore, this high priest\fnote{\fbackref{8:3} Lit. \fbib{this one}} had to offer something, too. \v{4}Now if he were on earth, he would not even be a priest, because other men offer the gifts prescribed by the Law. \v{5}They serve in a sanctuary that is a copy, a shadow of the heavenly one. This is why Moses was warned when he was about to build the tent: ``See to it that you make everything according to the pattern that was shown you on the mountain.''\fnote{\fbackref{8:5} Cf. Exod 25:40} \v{6}However, Jesus\fnote{\fbackref{8:6} Lit. \fbib{he}} has now obtained a more superior ministry, since the covenant he mediates is founded on better promises.
\passage{The New Covenant is Better than the Old}

\v{7}If the first covenant had been faultless, there would have been no need to look for a second one, \v{8}but God\fnote{\fbackref{8:8} Lit. \fbib{he}} found something wrong with his people\fnote{\fbackref{8:8} Lit. \fbib{with them}} when he said,

\begin{poetry}
\poeml ``Look! The days are coming, declares the Lord,\fnote{\fbackref{8:8} MT source citation reads \fbib{}\divine{Lord}} \\
\poemll    when I will establish a new covenant \\
\poemlll       with the house of Israel \\
\poemlll       and with the house of Judah. \\
\poeml \v{9}It will not be like the covenant that I made with their ancestors at the time \\
\poemll    when I took them by the hand \\
\poemlll       and brought them out of the land of Egypt. \\
\poeml Because they did not remain loyal to my covenant, \\
\poemll    I ignored them, declares the Lord.\fnote{\fbackref{8:9} MT source citation reads \fbib{}\divine{Lord}} \\
\poeml \v{10}For this is the covenant that I will make with the house of Israel \\
\poemll    after that time, declares the Lord:\fnote{\fbackref{8:10} MT source citation reads \fbib{}\divine{Lord}} \\
\poeml I will put my laws in their minds \\
\poemll    and write them on their hearts. \\
\poeml I will be their God, \\
\poemll    and they will be my people. \\
\poeml \v{11}Never again will everyone teach his neighbor \\
\poemll    or his brother by saying, `Know the Lord,'\fnote{\fbackref{8:11} MT source citation reads \fbib{}\divine{Lord}} \\
\poeml because all of them will know me, \\
\poemll    from the least important to the most important. \\
\poeml \v{12}For I will be merciful regarding their wrong deeds, \\
\poemll    and I will never again remember their sins.''\fnote{\fbackref{8:12} Cf. Jer 31:31-34}
\end{poetry}

\v{13}In speaking of a ``new'' covenant, he has made the first one obsolete, and what is obsolete and aging will soon disappear.
\labelchapt{9}
\passage{The Earthly Sanctuary and Its Ritual}

\chapt{9}
\v{1}Now even the first covenant\fnote{\fbackref{9:1} The Gk. lacks \fbib{covenant}} had regulations for worship and an earthly sanctuary. \v{2}For a tent was set up, and in the first part were the lamp stand, the table, and the bread of the Presence.\fnote{\fbackref{9:2} Lit. \fbib{the presentation of the bread}} This was called the Holy Place. \v{3}Behind the second curtain was the part of the tent called the Most Holy Place, \v{4}which had the gold altar for incense and the Ark of the Covenant completely covered with gold. In it were the gold jar holding the manna, Aaron's staff that had budded, and the Tablets of the Covenant. \v{5}Above it were the cherubim of glory overshadowing the place of atonement. (We cannot discuss these things in detail now.)

\v{6}When everything had been arranged like this, the priests always went into the first part of the tent to perform their duties. \v{7}But only the high priest went\fnote{\fbackref{9:7} The Gk. lacks \fbib{went}} into the second part, and then only once a year, and never without blood, which he offered for himself and for the sins committed by the people in ignorance. \v{8}The Holy Spirit was indicating by this that the way into the Most Holy Place had not yet been disclosed as long as the first part of the tent was still standing. \v{9}This illustration for today indicates that the gifts and sacrifices being offered could not clear the conscience of a worshiper, \v{10}since they deal only with food, drink, and various washings, which are required for the body until the time when things would be set right.
\passage{The Messiah Has Offered a Superior Sacrifice}

\v{11}But when the Messiah\fnote{\fbackref{9:11} Or \fbib{Christ}} came as a high priest of the good things that have come,\fnote{\fbackref{9:11} Other mss. read \fbib{that are to come}} he went\fnote{\fbackref{9:11} The Gk. lacks \fbib{went}} through the greater and more perfect tent that was not made by human\fnote{\fbackref{9:11} The Gk. lacks \fbib{human}} hands and that is not a part of this creation. \v{12}Not with the blood of goats and calves, but with his own blood he went into the Most Holy Place once for all and secured our eternal redemption. \v{13}For if the blood of goats and bulls and the ashes of a heifer sprinkled on those who are unclean purifies them physically, \v{14}how much more will the blood of the Messiah,\fnote{\fbackref{9:14} Or \fbib{Christ}} who through the eternal Spirit\fnote{\fbackref{9:14} Other mss. read \fbib{through the Holy Spirit}} offered himself unblemished to God, cleanse our\fnote{\fbackref{9:14} Other mss. read \fbib{your}} consciences from dead actions so that we may serve the living God!
\passage{The Messiah is the Mediator of a New Covenant}

\v{15}This is why the Messiah\fnote{\fbackref{9:15} Lit. \fbib{why he}} is the mediator of a new covenant; so that those who are called may receive the eternal inheritance promised them, since a death has occurred that redeems them from the offenses committed under the first covenant. \v{16}For where there is a will, the death of the one who made it must be established. \v{17}For a will is in force only when somebody has died, since it never takes effect as long as the one who made it is alive. \v{18}This is why even the first covenant was not put into effect without blood. \v{19}For after every commandment in the Law had been spoken to all the people by Moses, he took the blood of calves and goats,\fnote{\fbackref{9:19} Other mss. lack \fbib{and goats}} together with some water, scarlet wool, and branches of hyssop, and sprinkled the scroll and all the people, \v{20}saying, ``This is the blood of the covenant that God ordained for you.''\fnote{\fbackref{9:20} Cf. Exod 24:8} \v{21}In the same way, he sprinkled with the blood both the tent and everything used in worship. \v{22}In fact, under the Law almost everything is cleansed with blood, and without the shedding of the blood there is no forgiveness.
\passage{The Messiah's Perfect Sacrifice}

\v{23}Thus it was necessary for these earthly\fnote{\fbackref{9:23} The Gk. lacks \fbib{earthly}} copies of the things in heaven to be cleansed by these sacrifices,\fnote{\fbackref{9:23} Lit. \fbib{by these things}} but the heavenly things themselves are made clean\fnote{\fbackref{9:23} The Gk. lacks \fbib{are made clean}} with better sacrifices than these. \v{24}For the Messiah\fnote{\fbackref{9:24} Or \fbib{Christ}} did not go into a sanctuary made by human\fnote{\fbackref{9:24} The Gk. lacks \fbib{human}} hands that is merely a copy of the true one, but into heaven itself, to appear now in God's presence on our behalf. \v{25}Nor did he go into heaven\fnote{\fbackref{9:25} The Gk. lacks \fbib{did he go into heaven}} to sacrifice himself again and again, the way the high priest goes into the Holy Place every year with blood that is not his own. \v{26}Then he would have had to suffer repeatedly since the creation of the world. But now, at the end of the ages, he has appeared once for all to remove sin by his sacrifice. \v{27}Indeed, just as people are destined to die once and after that to be judged,\fnote{\fbackref{9:27} Lit. \fbib{after that the judgment}} \v{28}so the Messiah\fnote{\fbackref{9:28} Or \fbib{Christ}} was sacrificed once to take away the sins of many people. And he will appear a second time, not to deal with sin,\fnote{\fbackref{9:28} Lit. \fbib{a second time without sin}} but to bring salvation to those who eagerly wait for him.
\labelchapt{10}
\passage{The Law is a Reflection}

\chapt{10}
\v{1}For the Law, being only\fnote{\fbackref{10:1} The Gk. lacks \fbib{only}} a reflection\fnote{\fbackref{10:1} Or \fbib{shadow}} of the blessings to come and not their substance, can never make perfect those who come near by the same sacrifices repeatedly offered year after year. \v{2}Otherwise, would they not have stopped offering them, because the worshipers, cleansed once for all, would no longer be aware of any sins? \v{3}Instead, through those sacrifices there is a reminder of sins year after year, \v{4}for it is impossible for the blood of bulls and goats to take away sins.
\passage{The Messiah Offered One Sacrifice}

\v{5}For this reason, the Scriptures\fnote{\fbackref{10:5} The Gk. lacks \fbib{the Scriptures}} say, when the Messiah\fnote{\fbackref{10:5} Lit. \fbib{when he}} was about to come into the world:

\begin{poetry}
\poeml ``You did not want sacrifices and offerings, \\
\poemll    but you prepared a body for me. \\
\poeml \v{6}In burnt offerings and sin offerings \\
\poemll    you never took delight. \\
\poeml \v{7}Then I said, `See, I have come to do your will, O God' \\
\poemll    In the volume of the scroll this is written about me.''\fnote{\fbackref{10:7} Cf. Ps 40:6-8}
\end{poetry}

\v{8}In this passage he says, ``You never wanted or took delight in sacrifices, offerings, burnt offerings, and sin offerings,''\fnote{\fbackref{10:8} Cf. Ps 40:6} which are offered according to the Law. \v{9}Then he says, ``See, I have come to do your will.''\fnote{\fbackref{10:9} Cf. Ps 40:7} He takes away the first in order to establish the second. \v{10}By God's will we have been sanctified once and for all through the sacrifice of the body of Jesus, the Messiah.\fnote{\fbackref{10:10} Or \fbib{Christ}}

\v{11}Day after day every priest stands and repeatedly offers the same sacrifices that can never take away sins. \v{12}But when this priest\fnote{\fbackref{10:12} Lit. \fbib{this one}} had offered for all time one sacrifice for sins, ``he sat down at the right hand of God.''\fnote{\fbackref{10:12} Cf. Ps 110:1} \v{13}Since that time, he has been waiting for his enemies to be made a footstool for his feet. \v{14}For by a single offering he has perfected for all time those who are being sanctified. \v{15}The Holy Spirit also assures us of this, for he said:

\begin{poetry}
\poeml \v{16}``This is the covenant that I will make with them \\
\poemll    after those days, declares the Lord:\fnote{\fbackref{10:16} MT source citation reads \fbib{}\divine{Lord}} \\
\poeml I will put my laws in their hearts \\
\poemll    and will write them on their minds, \\
\poeml \v{17}and I will never again remember their sins \\
\poemll    and their lawless deeds.''\fnote{\fbackref{10:17} Jer 31:34}
\end{poetry}

\v{18}Now where there is forgiveness of these sins,\fnote{\fbackref{10:18} Lit. \fbib{of these things}} there is no longer any offering for sin.
\passage{How We Should Live}

\v{19}Therefore, my brothers, since we have confidence to enter the sanctuary by the blood of Jesus, \v{20}the new and living way that he opened for us through the curtain (that is, through his flesh), \v{21}and since we have a great high priest over the household of God, \v{22}let us continue to come near with sincere hearts in the full assurance that faith provides, because our hearts have been sprinkled clean from a guilty conscience, and our bodies have been washed with pure water. \v{23}Let us continue to hold firmly to the hope that we confess without wavering, for the one who made the promise is faithful. \v{24}And let us continue to consider how to motivate one another to love and good deeds, \v{25}not neglecting to meet together, as is the habit of some, but encouraging one another even more as you see the day of the Lord\fnote{\fbackref{10:25} The Gk. lacks \fbib{of the Lord}} coming nearer.

\v{26}For if we choose to go on sinning after we have learned the full truth, there no longer remains a sacrifice for sins, \v{27}but only a terrifying prospect of judgment and a raging fire that will consume the enemies of God.\fnote{\fbackref{10:27} The Gk. lacks \fbib{of God}} \v{28}Anyone who violates the Law of Moses dies without mercy ``on the testimony of two or three witnesses.''\fnote{\fbackref{10:28} Cf. Deut 17:6} \v{29}How much more severe a punishment do you think that person deserves who tramples on God's Son, treats as common the blood of the covenant by which it\fnote{\fbackref{10:29} Or \fbib{he}} was sanctified, and insults the Spirit of grace? \v{30}For we know the one who said, ``Vengeance belongs to me; I will pay them back,''\fnote{\fbackref{10:30} Cf. Deut 32:35} and again, ``The Lord\fnote{\fbackref{10:30} MT source citation reads \fbib{}\divine{Lord}} will judge his people.''\fnote{\fbackref{10:30} Cf. Deut 32:36; Ps 135:14} \v{31}It is a terrifying thing to fall into the hands of the living God!

\v{32}But you must continue to remember those earlier days, how after you were enlightened you endured a hard and painful struggle. \v{33}At times you were made a public spectacle by means of insults and persecutions, while at other times you associated with people who were treated this way. \v{34}For you sympathized\fnote{\fbackref{10:34} Or \fbib{suffered}} with the prisoners and cheerfully submitted to the violent seizure of your property, because you know that you have a better and more permanent possession.

\v{35}So don't lose your confidence, since it holds a great reward for you. \v{36}For you need endurance, so that after you have done God's will you can receive what he has promised. \v{37}For

\begin{poetry}
\poeml ``in a very little while \\
\poemll    the one who is coming will return--- \\
\poemlll       he will not delay; \\
\poeml \v{38}but my righteous one will live by faith, \\
\poemll    and if he turns back, \\
\poemlll       my soul will take no pleasure in him.''\fnote{\fbackref{10:38} Cf. Isa 26:20 (LXX); Hab 2:3-4 (LXX)}
\end{poetry}

\v{39}Now, we do not belong to those who turn back and are destroyed, but to those who have faith and are saved.
\labelchapt{11}
\passage{The Meaning of Faith}

\chapt{11}
\v{1}Now faith is the assurance that what we hope for will come about\fnote{\fbackref{11:1} The Gk. lacks \fbib{will come about}} and the certainty that what we cannot see exists.\fnote{\fbackref{11:1} The Gk. lacks \fbib{exist}} \v{2}By faith our ancestors won approval.

\v{3}By faith we understand that time was created by the word of God, so that what is seen was made from things that are invisible.

\v{4}By faith Abel offered to God a better sacrifice than Cain did,\fnote{\fbackref{11:4} The Gk. lacks \fbib{did}} and by faith\fnote{\fbackref{11:4} Lit. \fbib{it}} he was declared to be righteous, since God himself accepted his offerings. And by faith\fnote{\fbackref{11:4} Lit. \fbib{it}} he continues to speak, even though he is dead.

\v{5}By faith Enoch was taken away without experiencing death. He could not be found, because God had taken him away. For before he was taken, he won approval as one who pleased God. \v{6}Now without faith it is impossible to please God, for whoever comes to him must believe that he exists and that he rewards those who diligently search for him.

\v{7}By faith Noah, when warned about things not yet seen, reverently prepared an ark to save his family, and by faith\fnote{\fbackref{11:7} Lit. \fbib{it}} he condemned the world and inherited the righteousness that comes by faith.

\v{8}By faith Abraham, when called to go to a place he would later receive as his inheritance, obeyed and went, even though he did not know where he was going.

\v{9}By faith he made his home in the promised land like a stranger, living in tents, as did Isaac and Jacob, who also inherited the same promise, \v{10}because he was waiting for the city with permanent foundations, whose architect and builder is God.

\v{11}By faith Sarah, even though she was old and barren, received the strength to conceive, because she was convinced that the one who had made the promise was faithful. \v{12}Abraham\fnote{\fbackref{11:12} Lit. \fbib{He}} was as good as dead, yet from this one man came descendants as numerous as the stars in the sky and as countless as the sand on the seashore.

\v{13}All these people died having faith. They did not receive the things that were promised, yet they saw them in the distant future and welcomed them, acknowledging that they were strangers and foreigners on earth. \v{14}For people who say such things make it clear that they are looking for a country of their own. \v{15}If they had been thinking about what they had left behind, they would have had an opportunity to go back. \v{16}Instead, they were longing for a better country, that is, a heavenly one. That is why God is not ashamed to be called their God, because he has prepared a city for them.

\v{17}By faith Abraham, when he was tested, offered Isaac---he who had received the promises was about to offer his unique son\fnote{\fbackref{11:17} Lit. \fbib{unique one}} in sacrifice, \v{18}about whom it had been said, ``It is through Isaac that descendants will be named for you.''\fnote{\fbackref{11:18} Cf. Gen 21:12} \v{19}Abraham\fnote{\fbackref{11:19} Lit. \fbib{He}} was certain that God could raise the dead, and figuratively speaking, he did get Isaac\fnote{\fbackref{11:19} Lit. \fbib{him}} back in this way.

\v{20}By faith Isaac blessed Jacob and Esau in regard to their future.

\v{21}By faith Jacob, when he was dying, blessed each of Joseph's sons ``and worshipped while leaning\fnote{\fbackref{11:21} The Gk. lacks \fbib{while leaning}} on the top of his staff.''

\v{22}By faith Joseph, when his end was near, spoke about the exodus of the Israelis and gave them instructions about burying\fnote{\fbackref{11:22} The Gk. lacks \fbib{burying}} his bones.

\v{23}By faith Moses was hidden by his parents for three months after he was born, because they saw that he was a beautiful child and were not afraid of the king's order.

\v{24}By faith Moses, when he had grown up, refused to be called a son of Pharaoh's daughter, \v{25}because he preferred being mistreated with God's people to enjoying the pleasures of sin for a short time. \v{26}He thought that being insulted for the sake of the Messiah\fnote{\fbackref{11:26} Or \fbib{Christ}} was of greater value than the treasures of Egypt, because he was looking ahead to his reward.

\v{27}By faith he left Egypt, without being afraid of the king's anger, and he persevered because he saw the one who is invisible.

\v{28}By faith he established the Passover and the sprinkling of blood to keep the destroyer of the firstborn from touching the people.\fnote{\fbackref{11:28} Lit. \fbib{them}}

\v{29}By faith they went through the Red Sea as if it were dry land. When the Egyptians tried to do this, they were drowned.

\v{30}By faith the walls of Jericho fell down after they had been encircled for seven days.

\v{31}By faith Rahab the prostitute did not die with those who were disobedient, because she had welcomed the spies with a greeting of\fnote{\fbackref{11:31} The Gk. lacks \fbib{a greeting of}} peace.

\v{32}And what more should I say? For time would fail me to tell you about Gideon, Barak, Samson, Jephthah, David, Samuel, and the prophets. \v{33}Through faith they conquered kingdoms, administered justice, received promises, shut the mouths of lions, \v{34}put out raging fires, escaped death by\fnote{\fbackref{11:34} Lit. \fbib{by the edge of}} the sword, found strength in weakness, became powerful in battle, and routed foreign armies. \v{35}Women received their dead raised back to life. Other people were brutally tortured, but refused to be ransomed, so that they might gain a better resurrection. \v{36}Still others endured taunts and floggings, and even chains and imprisonment. \v{37}They were stoned to death, sawed in half, and killed with swords. They went around in sheepskins and goatskins. They were needy, oppressed, and mistreated. \v{38}The world wasn't worthy of them. They wandered in deserts and mountains, and from caves to holes in the ground.

\v{39}All these people won approval for their faith but they did not receive what was promised, \v{40}since God had planned something better for us, so that they would not be perfected without us.
\labelchapt{12}
\passage{We Must Look Off to Jesus}

\chapt{12}
\v{1}Therefore, having so vast a cloud of witnesses surrounding us, and throwing off everything that hinders us and especially the sin that so easily entangles\fnote{\fbackref{12:1} Other mss. read \fbib{distracts}} us, let us keep running with endurance the race set before us, \v{2}fixing our attention on Jesus, the pioneer and perfecter of the faith, who, in view of\fnote{\fbackref{12:2} Or \fbib{instead}} the joy set before him, endured the cross, disregarding its shame, and has sat down at the right hand of the throne of God.
\passage{The Father Disciplines Us}

\v{3}Think about the one who endured such hostility from sinners, so that you may not become tired and give up. \v{4}In your struggle against sin you have not yet resisted to the point of shedding your\fnote{\fbackref{12:4} The Gk. lacks \fbib{the point of shedding your}} blood. \v{5}You have forgotten the encouragement that is addressed to you as sons:

\begin{poetry}
\poeml ``My son, do not think lightly of the Lord's\fnote{\fbackref{12:5} MT source citation reads \fbib{}\divine{Lord}} discipline \\
\poemll    or give up when you are corrected by him. \\
\poeml \v{6}For the Lord\fnote{\fbackref{12:6} MT source citation reads \fbib{}\divine{Lord}} disciplines the one he loves, \\
\poemll    and he punishes\fnote{\fbackref{12:6} Or \fbib{whips}} every son he accepts.''\fnote{\fbackref{12:6} Cf. Prov 3:11-12}
\end{poetry}

\v{7}What you endure disciplines you: God is treating you as sons. Is there a son whom his father does not discipline? \v{8}Now if you are without any discipline, in which all sons share, then you are illegitimate and not God's\fnote{\fbackref{12:8} Lit. \fbib{his}} sons. \v{9}Furthermore, we had earthly fathers who disciplined us, and we respected them for it. We should submit even more to the Father of our spirits and live, shouldn't we? \v{10}For a short time they disciplined us as they thought best, but God\fnote{\fbackref{12:10} Lit. \fbib{he}} does it for our good, so that we may share in his holiness. \v{11}No discipline seems pleasant at the time, but painful. Later on, however, for those who have been trained by it, it produces a harvest of righteousness and peace.
\passage{Live as God's People}

\v{12}Therefore, strengthen your tired arms and your weak knees, \v{13}and straighten the paths of your life,\fnote{\fbackref{12:13} Lit. \fbib{feet}} so that your lameness may not become worse, but instead may be healed.

\v{14}Pursue peace with everyone, as well as holiness, without which no one will see the Lord. \v{15}See to it that no one fails to obtain the grace of God and that no bitter root grows up and causes you trouble, or many of you will become defiled. \v{16}No one should be immoral or godless like Esau, who sold his birthright for a single meal. \v{17}For you know that afterwards, when he wanted to inherit the blessing, he was rejected because he could not find any opportunity to repent, even though he begged to repent\fnote{\fbackref{12:17} Lit. \fbib{begged for it}} with tears.

\v{18}You have not come to something\fnote{\fbackref{12:18} Other mss. read \fbib{to a mountain}} that can be touched, to a blazing fire, to darkness, to gloom, \v{19}to a trumpet's blast, or to a voice that made the hearers beg that not another word be spoken to them. \v{20}For they could not endure the command that was given: ``If even an animal touches the mountain, it must be stoned to death.''\fnote{\fbackref{12:20} Cf. Exod 19:12-13} \v{21}Indeed, the sight was so terrifying that Moses said, ``I am trembling with fear.''\fnote{\fbackref{12:21} Cf. Deut 9:19} \v{22}Instead, you have come to Mount Zion, to the city of the living God, to the heavenly Jerusalem, to tens of thousands of angels joyfully gathered together, \v{23}to the assembly\fnote{\fbackref{12:23} Or \fbib{church}} of the firstborn who are enrolled in heaven, to a judge who is the God of all, to the spirits of righteous people who have been made perfect, \v{24}to Jesus, the mediator of a new covenant, and to the sprinkled blood that speaks a better message than Abel's.

\v{25}See to it that you do not ignore the one who is speaking. For if the hearers\fnote{\fbackref{12:25} Lit. \fbib{if they}} did not escape when they ignored the one who warned them on earth, how much less will we escape\fnote{\fbackref{12:25} The Gk. lacks \fbib{escape}} if we turn away from the one who is from heaven! \v{26}At that time his voice shook the earth, but now he has promised, ``Once more I will shake not only the earth but also heaven.''\fnote{\fbackref{12:26} Cf. Hag 2:6} \v{27}The expression ``once more''\fnote{\fbackref{12:27} Cf. Hag 2:6} signifies the removal of what can be shaken, that is, what he has made, so that what cannot be shaken may remain. \v{28}Therefore, since we are receiving a kingdom that cannot be shaken, let us be thankful and worship God in reverence and fear in a way that pleases him. \v{29}For ``our God is an all-consuming fire.''\fnote{\fbackref{12:29} Cf. Deut 4:24}
\labelchapt{13}
\passage{Concluding Words}

\chapt{13}
\v{1}Let brotherly love continue. \v{2}Stop neglecting to show hospitality to strangers, for by showing hospitality\fnote{\fbackref{13:2} Lit. \fbib{by this}} some have had angels as their guests without being aware of it. \v{3}Continue to remember those in prison as if you were in prison with them, as well as those who are mistreated, since they also are only mortal.\fnote{\fbackref{13:3} Lit. \fbib{are in the body}}

\v{4}Let marriage be kept honorable in every way, and the marriage bed undefiled. For God will judge those who commit sexual sins, especially those who commit adultery.

\v{5}Keep your lives free from the love of money, and be content with what you have, for God\fnote{\fbackref{13:5} Lit. \fbib{he}} has said, ``I will never leave you or abandon you.''\fnote{\fbackref{13:5} Cf. Deut 31:6} \v{6}Hence we can confidently say, ``The Lord\fnote{\fbackref{13:6} MT source citation reads \fbib{}\divine{Lord}} is my helper; I will not be afraid. What can anyone do to me?''\fnote{\fbackref{13:6} Cf. Ps 118:6}

\v{7}Remember your leaders, those who have spoken God's word to you. Think about the impact of their lives, and imitate their faith. \v{8}Jesus, the Messiah,\fnote{\fbackref{13:8} Or \fbib{Christ}} is the same yesterday and today---and forever!

\v{9}Stop being\fnote{\fbackref{13:9} Or \fbib{Do not be}} carried away by all kinds of unusual teachings, for it is good that the heart be strengthened by grace, not by food laws\fnote{\fbackref{13:9} Lit. \fbib{by foods}} that have never helped those who follow them.

\v{10}We have an altar, and those who serve in the tent have no right to eat at it. \v{11}For the bodies of animals, whose blood is taken into the sanctuary by the high priest as an offering for sin, are burned outside the camp. \v{12}That is why Jesus, in order to sanctify the people by his own blood, also suffered outside the city gate. \v{13}Therefore go to him outside the camp and endure the insults he endured. \v{14}For here we have no permanent city but are looking for the one that is coming. \v{15}Therefore, through him let us always bring God a sacrifice of praise, that is, the fruit of our lips that confess his name. \v{16}Do not neglect to do good and to be generous, for God is pleased with such sacrifices.

\v{17}Continue to follow and be submissive to your leaders, since they are watching over your souls as those who will have to give a word of explanation. By doing this, you will be letting them carry out their duties joyfully, and not with grief, for that would be harmful for you.

\v{18}Pray for us, for we are sure that we have a clear conscience and desire to live honorably in every way. \v{19}I especially ask you to do this so that I may be brought back to you sooner.

\v{20}Now may the God of peace, who by the blood of the eternal covenant brought back from the dead our Lord Jesus, the Great Shepherd of the sheep, \v{21}equip you with everything good\fnote{\fbackref{13:21} Other mss. read \fbib{for every good work}} to do his will, accomplishing in us\fnote{\fbackref{13:21} Other mss. read \fbib{you}} what pleases him through Jesus, the Messiah.\fnote{\fbackref{13:21} Or \fbib{Christ}} To him be glory forever and ever!\fnote{\fbackref{13:21} Other mss. lack \fbib{and ever}} Amen.
\passage{Final Greeting}

\v{22}I urge you, brothers, to listen patiently to my encouraging message,\fnote{\fbackref{13:22} Or \fbib{word of exhortation}} for I have written you a short letter.\fnote{\fbackref{13:22} Lit. \fbib{written you briefly}} \v{23}You should know that our brother Timothy has been set free. If he comes soon, he will be with me when I see you.

\v{24}Greet all your leaders and all the saints. Those who are from Italy greet you.

\v{25}May grace be with all of you!\fnote{\fbackref{13:25} Other mss. read \fbib{with all of you! Amen}}
