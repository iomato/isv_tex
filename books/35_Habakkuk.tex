\bookheader{Habakkuk}
\labelbook{Hab}

\bookpretitle{The Book of the Prophet}
\booktitle{Habakkuk}

\labelchapt{1}
\passage{Habakkuk's Oracle}

\chapt{1}
\v{1}The pronouncement\fnote{Or \fbib{revelation}} that the prophet Habakkuk\fnote{The Heb. name \fbib{Habakkuk} means \fbib{embrace}} perceived.
\passage{The Prophet's First Complaint}

\begin{poetry}
\poeml \v{2}``How long, \divine{Lord}, must I cry out for help, \\
\poeml but you won't listen? \\
\poeml I'm crying out to you, `Violence!' \\
\poemll    but you aren't providing deliverance. \\
\poeml \v{3}Why are you forcing me to look at iniquity \\
\poemll    and to stare at wickedness? \\
\poeml Social havoc and oppression are all around me; \\
\poemll    there are legal conflicts, and disputes abound. \\
\poeml \v{4}Therefore, the Law has become paralyzed, \\
\poemll    and justice never comes about. \\
\poeml Because criminals outnumber\fnote{Lit. \fbib{are surrounding}} the righteous, \\
\poemll    whenever judgments are issued, they come out crooked.''
\passage{God's Response: The Coming Chaldean Invasion}
\poeml \v{5}``Look out at the nations and pay attention! \\
\poemll    Be astounded! Be really astounded! \\
\poeml Because something is happening in your lifetime \\
\poemll    that you won't believe, even if it were described down to the smallest detail.\fnote{The Heb. lacks \fbib{down to the smallest detail}} \\
\poeml \v{6}Watch out! For I am bringing in the Chaldeans,\fnote{I.e. Babylonian invaders} \\
\poemll    that cruel and impetuous\fnote{Or \fbib{rash}} people, \\
\poeml who sweep across the earth \\
\poemll    dispossessing people\fnote{The Heb. lacks \fbib{people}} from homes not their own. \\
\poeml \v{7}They are terrible and fearsome; \\
\poemll    their brand of justice and sense of honor derive only from themselves! \\
\poeml \v{8}Their horses are swifter than leopards, \\
\poemll    and more cunning than wolves that attack at night. \\
\poeml Their horsemen are galloping \\
\poemll    as they approach from far away. \\
\poeml They swoop in like ravenous vultures.\fnote{Or \fbib{eagles}} \\
\poeml \v{9}``They all come to oppress--- \\
\poemll    hordes of them, their faces pressing onward--- \\
\poeml they take prisoners as numerous as\fnote{The Heb. lacks \fbib{as numerous as}} the desert sand! \\
\poeml \v{10}They make fun of kings, \\
\poemll    deriding those who rule. \\
\poeml They laugh at all of the fortified places, \\
\poemll    constructing ramps to seize them. \\
\poeml \v{11}Then like\fnote{The Heb. lacks \fbib{like}} the wind sweeping by \\
\poemll    they will pass through--- \\
\poeml they're guilty because they say\fnote{The Heb. lacks \fbib{they say}} their power is their god.''
\passage{The Prophet's Second Complaint}
\poeml \v{12}``Haven't you existed forever, \\
\poemll    \divine{Lord} my God, my Holy One? \\
\poemlll       We won't die! \\
\poeml \divine{Lord}, you've prepared them\fnote{I.e. the Babylonian invaders} for judgment; \\
\poemll    Rock, you've sentenced them\fnote{I.e. the Babylonian invaders} to correction. \\
\poeml \v{13}Your eyes are too pure to gaze upon evil; \\
\poemll    and you cannot tolerate wickedness. \\
\poeml So why do you tolerate the treacherous? \\
\poemll    And why do you stay silent \\
\poemlll       while the wicked devour those who are more righteous than they are? \\
\poeml \v{14}``You have fashioned mankind like fish in the ocean, \\
\poemll    like creeping things that have no ruler. \\
\poeml \v{15}The adversary\fnote{I.e. the Babylonian invaders} captures them with a hook, \\
\poemll    gathering them up in a fishing net. \\
\poeml He collects them with a dragnet, \\
\poemll    rejoicing and gloating over his catch.\fnote{The Heb. lacks \fbib{over his catch}} \\
\poeml \v{16}Therefore he sacrifices to his fishing net, \\
\poemll    and burns incense in the presence of his dragnet, \\
\poeml because by them his assets increase \\
\poemll    and he gets plenty of food. \\
\poeml \v{17}Is he to continue to empty his fishing net? \\
\poemll    Will he ever stop killing entire\fnote{The Heb. lacks \fbib{entire}} nations without mercy?''
\end{poetry}
\labelchapt{2}
\passage{Habakkuk Waits for God's Answer}

\begin{poetry}
\poeml \chapt{2}
\v{1}``I will stand at my guard post \\
\poeml and station myself on a tower. \\
\poeml I will wait and see what the \divine{Lord}\fnote{The Heb. lacks \fbib{the \divine{Lord}}} will say about me \\
\poemll    and what I\fnote{Syr \fbib{he}} will answer when he reprimands me.\fnote{Lit. \fbib{answer at my reprimand}}''
\passage{God's Reply to the Prophet's Complaint}
\poeml \v{2}When he answered, the \divine{Lord} told me: \\
\poeml ``Write out the revelation, \\
\poemll    engraving it clearly on the tablets, \\
\poemlll       so that a courier may run with it.\fnote{Or \fbib{that whoever reads it may run}} \\
\poeml \v{3}For the revelation pertains to an appointed time--- \\
\poemll    it speaks truthfully\fnote{Lit. \fbib{speaks without deception}} about the end. \\
\poeml Though it delays, wait for it, \\
\poemll    because it will surely come about--- \\
\poemlll       it will not be late! \\
\poeml \v{4}``Notice their\fnote{I.e. the Babylonian invaders} arrogance--- \\
\poemll    they have no inward uprightness\fnote{Lit. \fbib{no uprightness of soul}}--- \\
\poemlll       but the righteous will live by their faith. \\
\poeml \v{5}Moreover, just as wine leads astray the proud and powerful man, \\
\poemll    he\fnote{I.e. Babylonian invaders personified in their king} remains restless; \\
\poeml he\fnote{I.e. Babylonian invaders personified in their king} has expanded his appetite--- \\
\poemll    like the afterlife\fnote{Lit. \fbib{Sheol}; i.e. the realm of the dead} or death itself, he\fnote{I.e. Babylonian invaders personified in their king} is never satisfied. \\
\poeml He\fnote{I.e. Babylonian invaders personified in their king} gathers to himself all of the nations, \\
\poemll    taking captive all of the people for himself.''
\passage{Judgment on the Plunderer of Nations}
\poeml \v{6}``Will not all of these ridicule him \\
\poemll    with mocking scorn? They will say, \\
\poeml `Woe to the one who hordes for himself what isn't his. \\
\poemll    How long will you enrich yourself by extortion?'\fnote{Lit. \fbib{by loans}} \\
\poeml \v{7}Won't your creditors revolt unexpectedly? \\
\poemll    Won't those who make you tremble wake up? \\
\poemlll       As a result, you'll become their prey! \\
\poeml \v{8}Because you plundered many nations, \\
\poemll    all of their remnants will plunder you. \\
\poeml Human blood has been shed,\fnote{The Heb. lacks \fbib{has been shed}} \\
\poemll    and violence has been done to\fnote{Lit. \fbib{violence of}} the land, \\
\poemlll       to the city, and to all who live in it.''
\passage{Judgment on Those who Think They are Safe}
\poeml \v{9}``Woe to the one who amasses profit upon unjust profit \\
\poemll    in order to establish his household, \\
\poeml so he can establish a secure place\fnote{Lit. \fbib{establish his nest}} on the heights \\
\poemll    and escape from the power of evil. \\
\poeml \v{10}You have brought shame to yourself\fnote{Lit. \fbib{to your house}} by killing many people--- \\
\poemll    you are forfeiting your own life. \\
\poeml \v{11}Indeed, the stone will cry out from the wall \\
\poemll    and the rafter will respond from the woodwork.''
\passage{Judgment on the Lawless}
\poeml \v{12}``Woe to the one who founds a city upon bloodshed, \\
\poemll    and constructs a city by lawlessness. \\
\poeml \v{13}Is it not because of the \divine{Lord} of the Heavenly Armies \\
\poemll    that people grow tired putting out fires,\fnote{Lit. \fbib{tired for the sufficiency of fire}} \\
\poemlll       and nations weary themselves over nothing? \\
\poeml \v{14}Indeed, the earth will be filled \\
\poemll    with knowledge of the glory of the \divine{Lord}, \\
\poemlll       as water fills\fnote{Lit. \fbib{covers}} the sea.''
\passage{Judgment on the Violent}
\poeml \v{15}``Woe to the one who supplies his neighbor with a drink! \\
\poemll    You are forcing your bottle\fnote{Lit. \fbib{wine skin}} on him,\fnote{The Heb. lacks \fbib{on him}} \\
\poemlll       making him drunk so you can see them naked. \\
\poeml \v{16}You are filled with dishonor instead of glory. \\
\poemll    So go ahead,\fnote{The Heb. lacks \fbib{go ahead}} drink and be naked! \\
\poemll    The \divine{Lord}\fnote{Lit. \fbib{The power of the \divine{Lord}'s right hand}} will turn against you, \\
\poemll    and utter disgrace will debase your reputation.\fnote{Lit. \fbib{will come upon your glory}} \\
\poeml \v{17}Indeed, the violence done to Lebanon will overtake you, \\
\poemll    and the destruction of the beasts will terrorize you---\fnote{The Heb. lacks \fbib{you}} \\
\poeml because you shed human blood \\
\poemll    and did violence to\fnote{Lit. \fbib{violence of}} the land, to the city, and to all who live in it.''
\passage{Judgment on the Idol Maker}
\poeml \v{18}``Where is the benefit in owning\fnote{The Heb. lacks \fbib{owning}} a carved image, \\
\poemll    that motivates its maker to carve\fnote{Lit. \fbib{because its maker carved}} it? \\
\poeml It is only a cast image--- \\
\poemll    a teacher that lies--- \\
\poeml because the engraver entrusts himself to his carving, \\
\poemll    crafting speechless idols. \\
\poeml \v{19}``Woe to the one who says to a tree, `Wake up!' \\
\poemll    or `Arise!' to a speechless stone. \\
\poeml Idols\fnote{Lit. \fbib{Things}} like this can't teach, can they? \\
\poemll    Look, even though it is overlaid with gold and silver, \\
\poemlll       there's no breath in it at all.''
\passage{The \divine{Lord}'s Final Counsel to Habakkuk}
\poeml \v{20}``The \divine{Lord} is in his holy Temple. \\
\poemll    All the earth---be quiet in his presence.''
\end{poetry}
\labelchapt{3}
\passage{Habakkuk's Prayer of Faith}

\chapt{3}
\v{1}A prayer by the prophet Habakkuk, set to music.\fnote{Lit. \fbib{prayer upon Shigionoth}}

\begin{poetry}
\poeml \v{2}\divine{Lord}, as I listen to what has been said about you, \\
\poemll    I am afraid. \\
\poeml \divine{Lord}, revive your work throughout all of our lives--- \\
\poemll    reveal yourself\fnote{The Heb. lacks \fbib{yourself}} throughout all of our lives--- \\
\poeml when you\fnote{The Heb. lacks \fbib{you}} are angry, \\
\poemll    remember compassion. \\
\poeml \v{3}God comes from Teman\fnote{I.e. an Edomite desert town}--- \\
\poemll    the Holy One from Mount Paran.\fnote{I.e. in the Sinai desert}
\end{poetry}
\interlude{Interlude}

\begin{poetry}
\poeml His glory spreads throughout the heavens, \\
\poemll    and praises about him fill the earth. \\
\poeml \v{4}His radiance is like sunlight; \\
\poemll    beams of light shine\fnote{The Heb. lacks \fbib{shine}} from his hand, \\
\poemlll       where his strength lays hidden. \\
\poeml \v{5}Before him pestilence walks, \\
\poemll    and disease follows behind him.\fnote{Lit. \fbib{follows at his feet}} \\
\poeml \v{6}He stood up and shook the land; \\
\poemll    with his stare he startled the nations. \\
\poeml The age-old mountains were shattered, \\
\poemll    and the ancient hilltops bowed down. \\
\poeml His ways are eternal. \\
\poeml \v{7}I saw the tents of Cushan in distress, \\
\poemll    and the tent curtains of the land of Midian in anguish. \\
\poeml \v{8}Was the \divine{Lord} displeased with the rivers? \\
\poemll    Was your anger directed\fnote{The Heb. lacks \fbib{directed}} against the watercourses \\
\poemlll       or your wrath against the sea? \\
\poeml Indeed, you rode upon your horses, \\
\poemll    upon your chariots of deliverance. \\
\poeml \v{9}Your bow was exposed, \\
\poemll    and your\fnote{The Heb. lacks \fbib{your}} arrows targeted by command.
\end{poetry}
\interlude{Interlude}

\begin{poetry}
\poemlll       You split the earth with rivers. \\
\poeml \v{10}When the mountains looked upon you, they trembled; \\
\poemll    the overflowing water passed by, \\
\poeml the ocean shouted, \\
\poemll    and its waves\fnote{Lit. \fbib{hands}} surged upward. \\
\poeml \v{11}The sun and moon stand still in their orbits; \\
\poemll    at the glint of your arrows they speed along, \\
\poemlll       even at the gleam of your flashing spear. \\
\poeml \v{12}You march through the land in righteous\fnote{The Heb. lacks \fbib{righteous}} indignation; \\
\poemll    you tread down the nations in anger. \\
\poeml \v{13}You marched out to deliver your people, \\
\poemll    to deliver with your anointed. \\
\poeml You struck the head of the house of the wicked; \\
\poemll    you stripped him naked from head to foot.
\end{poetry}
\interlude{Interlude}

\begin{poetry}
\poeml \v{14}With his own lances you pierced the heads of his warriors, \\
\poemll    who came out like a windstorm to scatter us\fnote{Lit. \fbib{me}}--- \\
\poemlll       their joy is to devour the afflicted who are in hiding. \\
\poeml \v{15}You rode on the sea with your horses, \\
\poemll    even riding\fnote{The Heb. lacks \fbib{even riding}} the crested waves of mighty waters.
\passage{Habakkuk's Response}
\poeml \v{16}I heard and I trembled within. \\
\poemll    My lips quivered at the noise. \\
\poeml My legs gave way beneath me,\fnote{Lit. \fbib{Rottenness enters my bones}} \\
\poemll    and I trembled. \\
\poeml Nevertheless, I await the day of distress \\
\poemll    that will dawn on our invaders. \\
\poeml \v{17}Even though the fig tree does not blossom, \\
\poemll    and there are no grapes on the vines; \\
\poeml even if the olive harvest fails, \\
\poemll    and the fields produce nothing edible; \\
\poeml even if the flock is snatched from the sheepfold, \\
\poemll    and there is no herd in the stalls--- \\
\poeml \v{18}as for me, I will rejoice in the \divine{Lord}. \\
\poemll    I will find my joy in the God who delivers me. \\
\poeml \v{19}The \divine{Lord} God is my strength--- \\
\poemll    he will make my feet like those of a deer, \\
\poemlll       equipping me to tread on my mountain heights.
\end{poetry}
\psalminfo{For the choir director:}
\psalminfo{On my stringed instruments.}
