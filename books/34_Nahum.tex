\bookheader{Nahum}
\labelbook{Nah}

\bookpretitle{The Book of the Prophet}
\booktitle{Nahum}

\labelchapt{1}
\passage{Nahum's Vision}

\chapt{1}
\v{1}A pronouncement\fnote{\fbackref{1:1} Or \fbib{revelation}} about Nineveh: The record of the vision of Nahum\fnote{\fbackref{1:1} The Heb. name \fbib{Nahum} means \fbib{comfort}} from Elkosh.
\passage{The \divine{Lord}'s Anger against Assyria}

\begin{poetry}
\poeml \v{2}A jealous God, the \divine{Lord} avenges. \\
\poemll    The \divine{Lord} avenges; \\
\poeml The Lord is an angry husband. \\
\poeml The \divine{Lord} takes vengeance on his enemies, \\
\poemll    reserving anger for his adversaries. \\
\poeml \v{3}The \divine{Lord} is slow to anger and powerful, \\
\poemll    and he will never let the guilty\fnote{\fbackref{1:3} The Heb. lacks \fbib{the guilty}} go unpunished. \\
\poeml The \divine{Lord}'s path is in the windstorm and hurricane; \\
\poemll    thunderclouds are dust beneath his feet. \\
\poeml \v{4}He rebukes the sea, and it evaporates; \\
\poemll    he dries up all the rivers. \\
\poeml Bashan and Carmel wither, \\
\poemll    while the flowers of Lebanon languish. \\
\poeml \v{5}Mountains shake because of him, \\
\poemll    and the hills melt. \\
\poeml The earth goes into upheaval at his presence, \\
\poemll    as does the world with all of its inhabitants. \\
\poeml \v{6}Who can stand before his fury? \\
\poemll    And who can endure his fierce anger? \\
\poeml His displeasure pours out like fire, \\
\poemll    and rocks are broken to pieces because of him.
\passage{The \divine{Lord}'s Goodness in the Midst of Judgment}
\poeml \v{7}The \divine{Lord} is good--- \\
\poemll    a refuge in troubled times.\fnote{\fbackref{1:7} Or \fbib{in a day of trouble}} \\
\poeml He knows those who are confiding in him. \\
\poeml \v{8}But with an overwhelming deluge he will bring utter desolation to Nineveh,\fnote{\fbackref{1:8} Lit. \fbib{to its place}} \\
\poemll    and his enemies he will pursue with darkness. \\
\poeml \v{9}What are you scheming against the \divine{Lord}? \\
\poemll    He will bring about utter desolation--- \\
\poemlll       adversity will not strike twice! \\
\poeml \v{10}Indeed, while tangled as by a thorn bush, \\
\poemll    while drunken as by a strong drink, \\
\poemlll       the Ninevites\fnote{\fbackref{1:10} Lit. \fbib{they}} will be burned like dry straw. \\
\poeml \v{11}Someone has left you who plans evil against the \divine{Lord}. \\
\poemll    He is a demonic counselor.\fnote{\fbackref{1:11} Or \fbib{a worthless counselor}; Lit. \fbib{a counselor of Belial}}
\passage{The \divine{Lord}'s Rebuke to Assyria}
\poeml \v{12}This is what the \divine{Lord} says: \\
\poeml ``No matter how strong they are,\fnote{\fbackref{1:12} The Heb. lacks \fbib{they are}} \\
\poemll    and no matter how numerous, \\
\poemlll       they will surely be annihilated\fnote{\fbackref{1:12} Lit. \fbib{be cut down}} and pass away. \\
\poeml Though I have afflicted you,\fnote{\fbackref{1:12} The Heb. lacks \fbib{you}} \\
\poemll    I will do so no more. \\
\poeml \v{13}Now I will break off Assyria's\fnote{\fbackref{1:13} Lit. \fbib{his}} yoke from you, \\
\poemll    and tear apart your shackles.'' \\
\poeml \v{14}Now this is what the Lord has decreed about you, Nineveh:\fnote{\fbackref{1:14} The Heb. lacks \fbib{Nineveh}} \\
\poemll    ``There will be no more children born\fnote{\fbackref{1:14} Lit. \fbib{sown}} to carry on your name. \\
\poeml I will cut out the graven and molten images from the temples of your gods. \\
\poemll    I myself will dig your grave, \\
\poemlll       because you are vile.''
\passage{The Sure and Certain Deliverance of Judah}
\poeml \v{15}\fnote{\fbackref{1:15} This verse is 2:1 in MT}Look! There on the mountains! \\
\poemll    The feet of the one who brings good news, \\
\poemlll       who broadcasts a message of peace. \\
\poeml Judah, celebrate your solemn festivals \\
\poemll    and keep your vows, \\
\poeml because the wicked will never again invade you. \\
\poemll    Nineveh\fnote{\fbackref{1:15} Lit. \fbib{It}} will be\fnote{\fbackref{1:15} Or \fbib{has been}} completely eliminated!
\end{poetry}
\labelchapt{2}
\passage{The Coming Invasion of Nineveh}

\chapt{2}
\v{1}\fnote{\fbackref{2:1} This verse is 2:2 in MT, and so throughout the chapter}You are being attacked by advancing forces!

\begin{poetry}
\poeml Guard your rampart! \\
\poemlll       Watch your roads!\fnote{\fbackref{2:1} Lit. \fbib{the road}} \\
\poeml Prepare yourselves!\fnote{\fbackref{2:1} Lit. \fbib{Strengthen the loins!}} \\
\poemll    Marshall your defenses!\fnote{\fbackref{2:1} Lit. \fbib{Marshall power}} \\
\poeml \v{2}For the \divine{Lord} will restore the glory of Jacob, \\
\poemll    just as he will restore\fnote{\fbackref{2:2} The Heb. lacks \fbib{will restore}} the glory of Israel, \\
\poeml although plunderers have devastated them, \\
\poemll    vandalizing their vine branches. \\
\poeml \v{3}The shields deployed by\fnote{\fbackref{2:3} Lit. \fbib{shield of}} Israel's\fnote{\fbackref{2:3} Lit. \fbib{its}} elite forces are scarlet, \\
\poemll    their valiant men are clothed in crimson. \\
\poeml When they are prepared, \\
\poemll    the polished armament on their chariots will shine, \\
\poemlll       and lances will be brandished about ferociously.\fnote{\fbackref{2:3} The Heb. lacks \fbib{ferociously}} \\
\poeml \v{4}Their chariots storm through the streets, \\
\poemll    jostling each other along broad avenues. \\
\poeml They look like torches, \\
\poemll    as they dart around like lightning. \\
\poeml \v{5}He will remember to summon\fnote{\fbackref{2:5} The Heb. lacks \fbib{to summon}} his finest troops. \\
\poemll    They will stumble on their way, \\
\poemlll       hurrying over to Nineveh's\fnote{\fbackref{2:5} The Heb. lacks \fbib{Nineveh's}} wall. \\
\poeml Their defensive shield is in place. \\
\poeml \v{6}The river gates will be opened, \\
\poemll    and the palace will collapse. \\
\poeml \v{7}It has been determined: \\
\poemll    The woman\fnote{\fbackref{2:7} I.e. Nineveh personified as a woman} is unveiled and sent away, \\
\poemlll       her servant girls mourn. \\
\poeml Beating their breasts, \\
\poemll    they whimper like doves. \\
\poeml \v{8}Nineveh is a reservoir whose water is draining away. \\
\poemll    ``Wait! Wait!'' they cry,\fnote{\fbackref{2:8} The Heb. lacks \fbib{they cry}} \\
\poemlll       yet not even one person\fnote{\fbackref{2:8} The Heb. lacks \fbib{person}} looks back. \\
\poeml \v{9}Take the silver! Take the gold! \\
\poemll    There is no end to the treasure--- \\
\poemlll       fabulous riches of every imagination. \\
\poeml \v{10}Nineveh\fnote{\fbackref{2:10} Lit. \fbib{She}; i.e. Nineveh personified as a woman} is devastated, deserted, and desolate. \\
\poemll    Her heart melts, her knees knock. \\
\poeml Every stomach is upset, \\
\poemll    every face grows pale.\fnote{\fbackref{2:10} Lit. \fbib{gathers blackness}; cf. Joel 2:6b}
\passage{Nineveh: the Lion's Den Destroyed}
\poeml \v{11}Where is this lion's den? \\
\poemll    Where is the place where the young lions fed, \\
\poeml where the lion and its mate walked with their young, \\
\poemll    the place where they feared nothing? \\
\poeml \v{12}This lion renders its prey to pieces to feed its whelps, \\
\poemll    and strangles enough prey\fnote{\fbackref{2:12} The Heb. lacks \fbib{prey}} for its mate, \\
\poeml filling its lairs with prey \\
\poemll    and its dens with rendered flesh. \\
\poeml \v{13}``I am against you,'' declares the \divine{Lord} of the Heavenly Armies, \\
\poemll    ``and I will send your chariots up in smoke. \\
\poeml A sword will devour your young lions, \\
\poemll    I will eliminate your prey from the earth, \\
\poemlll       and the voice of your messengers will no longer be heard.''
\end{poetry}
\labelchapt{3}
\passage{The Coming Judgment of Nineveh}

\chapt{3}
\v{1}Woe to this city, contaminated with shed blood,

\begin{poetry}
\poeml all full of lies and robberies--- \\
\poemlll       it is\fnote{\fbackref{3:1} The Heb. lacks \fbib{it is}} never without victims! \\
\poeml \v{2}The crack of whips \\
\poemll    and the clamor of wheels! \\
\poeml The galloping horses \\
\poemll    and the bounding chariots! \\
\poeml \v{3}The cavalry attacks--- \\
\poemll    the flashing sword \\
\poemlll       and the glittering spear! \\
\poeml Many are the slain--- \\
\poemll    so many casualties!--- \\
\poeml No end to bodies, \\
\poemll    and the soldiers\fnote{\fbackref{3:3} Lit. \fbib{They}} trip over the corpses. \\
\poeml \v{4}Innumerable are the harlotries of this well-favored whore, \\
\poemll    this mistress of witchcraft, \\
\poeml who enslaves nations through her fornication \\
\poemll    and families through her sorcery.
\passage{God's Decree against Nineveh}
\poeml \v{5}``Look, I am against you,'' declares the \divine{Lord} of the Heavenly Armies, \\
\poemll    ``so I will pull up your dress over your face. \\
\poeml I will show your nakedness to the nations, \\
\poemll    and your disgrace to the kingdoms. \\
\poeml \v{6}I will hurl abominable filth upon you, \\
\poemll    making you look foolish, \\
\poemlll       and making an example of you. \\
\poeml \v{7}It will be that everyone who looks at you will run away, saying, \\
\poemll    `Nineveh has been violently overthrown! \\
\poemlll       Who will mourn for her? \\
\poemll    Where will I find anyone to comfort you?'\,''
\passage{Thebes: an Example of God's Justice}
\poeml \v{8}``Are you any better than Thebes,\fnote{\fbackref{3:8} Lit. \fbib{than No-Amon}; i.e. Thebes, capital of southern Egypt (cf. Jer 46.25)} \\
\poemll    which sits by the upper Nile, surrounded by water? \\
\poeml The sea was her defense, \\
\poemll    the waters her wall of protection.\fnote{\fbackref{3:8} The Heb. lacks \fbib{of protection}} \\
\poeml \v{9}Sudan\fnote{\fbackref{3:9} Lit. \fbib{Cush}} was her source of strength, along with Egypt--- \\
\poemll    there were no limits. \\
\poeml Put and the Libyans were her allies. \\
\poeml \v{10}But she, too, was exiled--- \\
\poemll    she went into captivity! \\
\poeml Her young children were dashed to pieces \\
\poemll    at every crossroad of their streets, \\
\poeml and her famous citizens were sold by lottery, \\
\poemll    while all of her aristocrats were put in chains. \\
\poeml \v{11}You will also become drunk. \\
\poemll    You will disappear, \\
\poemlll       trying to hide from your enemies. \\
\poeml \v{12}All your defenses are like fig trees with ripe early fruit--- \\
\poemll    when shaken, it falls right into the devourer's mouth. \\
\poeml \v{13}Look at your people---\fnote{\fbackref{3:13} I.e. Nineveh's army} \\
\poemll    they are women! \\
\poeml Your borders stand wide open to your enemies, \\
\poemll    while fire consumes the bars of your gates.''
\passage{The Futility of Avoiding God's Judgment}
\poeml \v{14}``Draw water, because a siege is coming!\fnote{\fbackref{3:14} The Heb. lacks \fbib{coming}} \\
\poemll    Strengthen your fortresses! \\
\poeml Make the clay good and strong! \\
\poemll    Mix the mortar! \\
\poemlll       Go get your brick molds!\fnote{\fbackref{3:14} I.e. this verse appears to be affirming the uselessness of constructing a defense against God's coming judgment.} \\
\poeml \v{15}In that place fire will consume you, \\
\poemll    the sword will cut you down, \\
\poemlll       consuming you as locusts do. \\
\poeml Multiply yourself like locusts, \\
\poemll    increase like swarming grasshoppers. \\
\poeml \v{16}You added to your inventory of businessmen--- \\
\poemll    they number more than the stars of heaven. \\
\poeml The creeping locust sheds its skin and flies away. \\
\poeml \v{17}Your imperial guards are like the swarming grasshopper; \\
\poemll    your marshals are like hordes of grasshoppers, \\
\poemlll       settling in the stone walls on a chilly day. \\
\poeml The sun rises, and they flee away; \\
\poemll    no one knows where they went. \\
\poeml \v{18}Hey king of Assyria! Your shepherds are asleep \\
\poemll    and your nobles are lying down! \\
\poeml Your people lie scattered on the mountains, \\
\poemll    and there is no one to gather them together. \\
\poeml \v{19}There is no healing for your injury--- \\
\poemll    your wound is fatal. \\
\poeml Everyone who hears about you will applaud, \\
\poemll    because who hasn't escaped your endless evil?''\end{poetry}
