\bookheader{Ezekiel}
\labelbook{Ezek}

\bookpretitle{The Book of the Prophet}
\booktitle{Ezekiel}

\labelchapt{1}
\passage{An Introduction to Ezekiel's Visions}

\chapt{1}
\v{1}On the fifth day of the fourth month of the thirtieth year of the exile to Babylon,\fnote{The Heb. lacks \fbib{of the exile to Babylon}} while I was among the captives on the bank of\fnote{The Heb. lacks \fbib{the bank of}} the Chebar River, heaven opened up and I saw visions from God.
\passage{The Vision of the Fire Cloud}

\v{2}On the fifth day\fnote{The Heb. lacks \fbib{day}} of the month in the fifth year of King Jehoiachin's imprisonment in exile, \v{3}a message from\fnote{Lit. \fbib{the word of}} the \divine{Lord} came directly to Buzi's son Ezekiel,\fnote{The Heb. name \fbib{Ezekiel} means \fbib{My strengthener is God}} the priest, by the Chebar River in the land of the Chaldeans.\fnote{I.e. Aramaic speaking people of southern Mesopotamia; or magi-astrologers; and so throughout the book} The hand of the \divine{Lord} rested upon him there.

\v{4}I was amazed to see a wind storm blow\fnote{Lit. \fbib{come}} in from the north, consisting of\fnote{The Heb. lacks \fbib{consisting of}} a massive cloud and fire that was flashing back and forth, surrounded by bright light. From deep within the cloud,\fnote{Lit. \fbib{within it}} something was shining that appeared to have a color like bronze that had been placed in fire until it glowed.
\passage{The Vision of the Four Beings}

\v{5}Deep inside it, the likenesses of four living beings were visible. Their appearances were similar to human forms, \v{6}except that they each had four faces, four pairs of wings,\fnote{Or \fbib{four wings}} \v{7}and straight legs. Their feet resembled calves' hooves, but they gleamed like polished bronze. \v{8}From under their wings there were human hands on their four sides.

Now as to their four faces and four pairs of wings, \v{9}their pairs of wings overlapped each other. They moved in straight directions without turning their faces around as they moved. \v{10}The form of their faces was human, but each of the four also had the face of a lion to the right, the face of an ox to the left, and the face of an eagle behind them.\fnote{The Heb. lacks \fbib{behind them}} \v{11}That's what their faces were like. Their wings spread out above and around them, one pair overlapping another, with one pair covering themselves. \v{12}Each moved in straight directions. Wherever they decided\fnote{Lit. \fbib{wherever their spirit was}} to go, they went without turning themselves.

\v{13}Now, in the midst of the living beings there was something that\fnote{So LXX; MT reads \fbib{These living beings}} appeared to glow like coals kindled by a fire,\fnote{Or \fbib{appeared like glowing coals of fire}} like torches that moved back and forth between the living beings. The fire was dazzling, and lightning flashed from the fire. \v{14}The living beings moved around, in appearance resembling lightning.
\passage{The Vision of the Wheels}

\v{15}As I observed the living beings, I noticed one wheel on the earth beside each being---that is, for the four of them.\fnote{Lit. \fbib{of their faces}} \v{16}Their wheels and their construction details looked like gold-colored beryl.\fnote{Lit. \fbib{like tarshish}; i.e. a semi-precious stone similar to beryl or yellow jasper} Each wheel was identical in form to the others,\fnote{Lit. \fbib{the four wheels}} and they appeared to have been constructed and designed as if one wheel were within another. \v{17}Whenever the four moved, no matter which of four directions, they moved without turning around.

\v{18}Their wheel rims were ornate\fnote{Lit. \fbib{lofty}; or \fbib{high}} and terrifying. They were full of eyes that surrounded the four of them. \v{19}Whenever the living beings moved, the wheels moved, too. Whenever the living beings rose from the earth, the wheels rose also. \v{20}Whatever direction these spirits went, the wheels would be lifted up along with them, because the wheels were alive.\fnote{Lit. \fbib{the spirit of the living beings resided in the wheels}} \v{21}They moved around whenever they wanted to move around,\fnote{Lit. \fbib{In their moving they moved}} and they stood still whenever they wanted to stand still;\fnote{Lit. \fbib{and in their standing they stood}} and whenever they rose from the earth, the wheels remained close beside them, because the wheels were also alive.\fnote{Lit. \fbib{the spirit of the living beings resided also in the wheels}}
\passage{The Vision of the Wings}

\v{22}There was spread out over the heads of the living beings what looked like a canopy,\fnote{Or \fbib{expanse}} in outward appearance resembling ice, \v{23}and underneath the canopy, their wings spread out straight over their heads toward each other. They each also had two wings with which they covered themselves, one wing covering its body on one side and one wing covering itself on the other side.

\v{24}I also heard the sound of their wings, like the sound of roaring\fnote{Or \fbib{of an abundant amount of}} water, like the voice of the Almighty, or like a boisterous crowd within an army camp. Whenever they stopped flying, they lowered their wings. \v{25}A sound came from above the canopy that was spread out over their heads. Whenever they stood still, they lowered their wings. \v{26}From above the canopy that was spread out over their heads, there appeared to be something reminiscent of a throne, resembling sapphire\fnote{Or \fbib{of lapis lazuli}} in form.
\passage{The Vision of the Glory of God}

There was the likeness of the appearance of a human being seated on the likeness of the throne high above. \v{27}I noticed that from what appeared to look like his waist upward there was something that looked like metal that glowed as if it were immersed in fire. Below this there was something resembling fire, with a radiant light surrounding him. \v{28}The appearance of the radiant light resembled that of a rainbow shining in a cloud on a rainy day. This was what the appearance of the form of the glory of the \divine{Lord} resembled. When I saw all of this,\fnote{The Heb. lacks \fbib{all of this}} I fell flat on my face. Then I heard a voice speaking.
\labelchapt{2}
\passage{Ezekiel's Commission to Prophesy}

\chapt{2}
\v{1}``Son of Man,'' the \divine{Lord} said,\fnote{The Heb. lacks \fbib{the \divine{Lord} said}} ``get up on your feet. I want to talk to you.'' \v{2}Even while he was speaking to me, the Spirit entered me, set me on my feet, and I listened to the voice that had been speaking to me.

\v{3}``Son of Man, I'm sending you to that rebellious people, the Israelis, who have rebelled against me the same way their ancestors did. And they're still rebels\fnote{The Heb. lacks \fbib{And they're still rebels}} to this very day! \v{4}They're stubborn\fnote{Lit. \fbib{They're children of hard faces}} and strong willed. I'm sending you to them to tell them what the \divine{Lord} says. \v{5}Whether this rebellious group\fnote{Lit. \fbib{house}} listens to you or not, at least\fnote{The Heb. lacks \fbib{at least}} they'll realize that a prophet had appeared in their midst!

\v{6}``Now as for you, Son of Man, never be afraid of them or of anything they have to say, because being with them will be like settling down to live among briers, thorn bushes, and scorpions! Don't be afraid of anything they have to say, and don't be awed by their appearance, since they are a rebellious group.\fnote{Lit. \fbib{house}} \v{7}You are to tell them whatever I have to say to them, whether they listen or not, since they are rebellious.''
\passage{The Vision of the Edible Scroll}

\v{8}``Son of Man, you are to listen to what I tell you. You are never to be rebellious like they are: a rebellious group.\fnote{Lit. \fbib{house}} Now, open your mouth and eat what I'm giving you{\ldots}''

\v{9}As I watched, all of a sudden there was a hand being stretched out in my direction! And there was a scroll \v{10}being unrolled right in front of me! Written on both sides were lamentations, mourning, and cries of grief.\fnote{The Heb. lacks \fbib{of grief}}
\labelchapt{3}
\passage{Ezekiel's Commission to Prophesy}

\chapt{3}
\v{1}Then he told me, ``Son of Man, eat! Eat what you see\fnote{Lit. \fbib{find}}---this scroll---and then go talk to the house of Israel.'' \v{2}So I opened my mouth and he fed me\fnote{Lit. \fbib{he caused me to eat}} the scroll.

\v{3}Then he told me, ``Son of Man, fill your stomach and digest this scroll that I'm giving you.'' So I ate it, and it was like sweet honey in my mouth.

\v{4}Then he told me, ``Son of Man, go to the house of Israel and tell them what I have to say to them, \v{5}because you're not going to a people whose speech you cannot understand or whose language is difficult to speak. Instead, you're going to the house of Israel. \v{6}This isn't a large group of people whose speech is unintelligible to you or whose language is difficult for you to comprehend. Frankly, if I had sent you to that kind of people,\fnote{Lit. \fbib{to them}} they would certainly have listened to you! \v{7}But the house of Israel won't listen to you, since they weren't willing to listen to me. That's because the entire house of Israel is hard-headed and hard-hearted. \v{8}So pay attention! I'm going to make you just as obstinate\fnote{Lit. \fbib{making your face hard against their faces}} and unyielding as they are.\fnote{\fbib{and your forehead as hard as their foreheads}} \v{9}I'm making you harder than flint---like diamond! So you are not to fear them or be intimidated by how they look at you,\fnote{The Heb. lacks \fbib{at you}} since they're a rebellious group.''
\passage{Ezekiel is Commissioned to Speak}

\v{10}Next, he told me, ``Son of Man, take to heart every word that I'm telling you. Listen carefully, \v{11}then go immediately\fnote{Lit. \fbib{walk and go}} to the exiles; that is, to your people's descendants, and tell them, `This is what the Lord \divine{God} says{\ldots}' whether they listen or not.''\fnote{Lit. \fbib{or fail}}

\v{12}Then the Spirit lifted me up and I heard a great earthquake behind me and the glory of the \divine{Lord} arose from his place, \v{13}accompanied by the sound of the wings of the living creatures gently touching each other and with the sound of the wheels emanating from the front, accompanied by a great earthquake.
\passage{Ezekiel Addresses the Israelis}

\v{14}Then the Spirit lifted me up and carried me away. I went bitterly with an angry attitude as the hand of the \divine{Lord} rested on me. \v{15}I came to the exiles at Tel-abib by the Chebar River and sat down among them for seven days, appalled. \v{16}At the end of the seven days, this message from the \divine{Lord} came to me: \v{17}``Son of Man,'' he said,\fnote{The Heb. lacks \fbib{he said}} ``I've appointed you to be a watchman\fnote{cf. 2Sam 18:24; 2King 9:17} over the house of Israel. Therefore when you hear a message that comes from me, you are to warn them for me.

\v{18}``So when I say to a wicked person, `You're about to die,' if you don't warn or instruct that wicked person that his behavior\fnote{Lit. \fbib{ways}} is wicked so he can live, that wicked person will die in his sin, but I'll hold you responsible for his death.\fnote{Lit. \fbib{but I'll seek his blood from your hand}} \v{19}If you warn the wicked person, and he doesn't repent of his wickedness or of his wicked behavior,\fnote{Lit. \fbib{ways}} he'll die in his sin, but you will have saved your own life.

\v{20}``When a righteous man abandons his righteousness to practice unrighteousness, I'll set a stumbling block before him. He'll die. If you don't warn him, he'll die in his sin and the righteous deeds that he had practiced won't be remembered, but you'll be held responsible for his death.\fnote{Lit. \fbib{but I'll seek his blood from your hand}} \v{21}If you warn the righteous person, so that he\fnote{Lit. \fbib{righteous person}.} doesn't commit sin, then he'll live, since he had been warned. And you will have saved your life.''
\passage{Ezekiel Sees God in the Valley}

\v{22}The hand of the \divine{Lord} was upon me, and he told me, ``Get up! Go to the valley, and I'll speak with you there.'' \v{23}So I got up, went to the valley, and there was the glory of the \divine{Lord}, standing there just like\fnote{Lit. \fbib{there like the glory of the \divine{Lord} that}} I had seen at the Chebar River. So I fell on my face.

\v{24}The Spirit entered me, rested on me, caused me to stand on my feet, and then he spoke to me. This is what he had to say: ``Go barricade yourself in your house. \v{25}Now pay attention! They're going to bind you with ropes, tying you up right in their midst, so you won't be able to circulate freely among them. \v{26}Meanwhile, I'll make your tongue stick to the roof of your mouth so that you'll be mute and unable to reprove them, since they're a rebellious group.\fnote{Lit. \fbib{house}} \v{27}But when I speak with you, I'll open your mouth so you can say to them, `This is what the Lord \divine{God} says:

\begin{poetry}
\poeml ``As for those who will listen, \\
\poemll    `Let them listen,' \\
\poeml but as for those who refuse, \\
\poemll    `Let them refuse,' \\
\poemlll       since they're a rebellious group.''\,'\,''\fnote{Lit. \fbib{house}}
\end{poetry}
\labelchapt{4}
\passage{The Vision of the Brick}

\chapt{4}
\v{1}``And now Son of Man, you are to take a brick,\fnote{Or \fbib{tile}} set it in front of you, and inscribe on it the outline of\fnote{The Heb. lacks \fbib{the outline of}} the city---that is, Jerusalem.\fnote{I.e. a symbolic map of the city} \v{2}You are to lay siege against it, build a rampart around it, set a bulwark against it, encircle it with a berm, set up camps against it, and place battering rams around it. \v{3}Then you are to take a flat, iron plate and set it up as an iron wall between you and the city.

``Next, you are to turn toward it, oppose\fnote{Lit. \fbib{it, set your face against}} it, and place it under siege, because you are to lay siege to it. All of this will serve as a sign to the house of Israel.

\v{4}``Now as for you, you are to sleep\fnote{Lit. \fbib{lay}} on your left side, symbolically\fnote{The Heb. lacks \fbib{symbolically}} bearing the punishment\fnote{Or \fbib{iniquity}} of the house of Israel while you're counting the days you'll be sleeping on your left side\fnote{Lit. \fbib{on it}} to bear symbolically\fnote{The Heb. lacks \fbib{symbolically}} the punishment for\fnote{Or \fbib{the iniquity of}} their sin. \v{5}I've assigned you to sleep this way for 390 days, representing the years they've been sinning,\fnote{I.e. one year for each day} as you bear symbolically\fnote{The Heb. lacks \fbib{symbolically}} the punishment of the house of Israel. \v{6}When you have completed this, you are to sleep\fnote{Lit. \fbib{lay}} on your right side, symbolically\fnote{The Heb. lacks \fbib{symbolically}} bearing the iniquity of Judah for 40 days. Each day that I've assigned to you represents one year. \v{7}After this, you are to turn toward the rampart of Jerusalem and oppose\fnote{Lit. \fbib{and to set your face against}} it with your bare arms, because I'm going to prophesy about it. \v{8}Look! I'll tie you up\fnote{Lit. \fbib{I'll set ropes on you}.} so that you're unable to turn from one side to the other until you've completed your siege.''
\passage{Ezekiel's Menu}

\v{9}``Furthermore, you are to take some wheat, barley, beans, lentils, millet, and spelt, and mix them together in one container. Then you are to make bread from these grains sufficient to supply you through the time during which you'll be sleeping on your side. You are to eat it for 390 days. \v{10}The food that you'll be eating is to consist of portions weighing 20 shekels,\fnote{I.e. about eight ounces; a shekel weighed about 0.4 ounces} to be consumed daily at regular intervals.\fnote{Lit. \fbib{it from time to time}} \v{11}You are to measure one sixth of one hin\fnote{I.e. about a pint and a half} of water each time you drink it. \v{12}You are to eat it as barley cakes and bake it right in front of them, using dried human dung for cooking fuel.''\fnote{The Heb. lacks \fbib{for cooking fuel}}

\v{13}Then the \divine{Lord} said, ``This is how the Israelis will be eating unclean food among the nations, where I'll be sending them.''

\v{14}``Now, Lord \divine{God},'' I replied, ``I've never been defiled, ever since I was young until now. I haven't eaten an animal that died on its own or was torn by beasts, and no unclean meat has ever entered my mouth!''

\v{15}``Very well,'' he responded. ``I'll allow you to substitute cow's dung for human dung. Cook your food\fnote{Lit. \fbib{bread}} over that.''

\v{16}He also told me, ``Son of Man, look! I'm about to disrupt the source\fnote{Lit. \fbib{staff}} of bread in Jerusalem. As a result, they'll ration bread by weight while their terror continues to grow and they'll ration drinking water while their horror continues to mount! \v{17}Indeed, they'll need bread and water, but everyone will be panic-stricken as they waste away in their iniquity.''
\labelchapt{5}
\passage{Ezekiel Shaves with a Sword}

\chapt{5}
\v{1}``Now as for you, Son of Man, you are to go find a sharp sword and use it like a barber's razor. You are to cut your hair and beard. Then you are to take a weighing scale and divide your shaved hair into three parts.\fnote{The Heb. lacks \fbib{your shaved hair into three parts}} \v{2}You are to burn a third of it in the middle of the city when you've finished your siege. Next, you are to take another third of it and beat it with your sword. Last, you are to scatter the remaining third to the wind, after which I'll unsheathe my sword and pursue them. \v{3}You are to preserve a few strands of hair and hide them in the folds\fnote{Lit. \fbib{wings}} of your garment. \v{4}Then you are to take a few strands, throw them in the fire, and incinerate them. A fire will proceed to the house of Israel from there.''
\passage{Jerusalem's Desolation Predicted}

\v{5}``This is what the Lord \divine{God} says, `This is Jerusalem. I placed her in the center of nations, with many\fnote{The Heb. lacks \fbib{many}} nations surrounding her. \v{6}But she rebelled against my ordinances and my statutes. She practiced more evil than all the nations and territories around her. They rejected my ordinances and didn't live by\fnote{Lit. \fbib{didn't walk in}} my statutes.'

\v{7}``Therefore this is what the Lord \divine{God} says: `Because you're more disrespectful than the nations that surround you, you didn't follow my statutes or follow my ordinances. You didn't even follow the ordinances of the surrounding nations!'

\v{8}``Therefore this is what the Lord \divine{God} says: `Watch out! I---that's right, even I---am against you. I'll carry out my sentence among you right in front of the nations. \v{9}In fact, I'm going to do what I've never done before and what I'll never again do, because of all of your loathsome behavior: \v{10}Fathers will eat their children in your midst. After this, your sons will eat their fathers as I carry out my sentence against you and scatter your survivors to the winds!'

\v{11}``Therefore, as sure as I live,'' declares the Lord \divine{God}, ``because you've defiled my sanctuary with every loathsome thing and every abomination, I'll restrain myself, and I'll show neither pity nor compassion.\fnote{Lit. \fbib{and my eyes won't show pity and I won't have compassion}} \v{12}A third of you will die by pestilence, starving because of the famine in your midst. Another third will die violently by the violence of war\fnote{Lit. \fbib{will fall by the sword}} around you. The final third I'll scatter to the wind as I unsheathe my sword to pursue them.

\v{13}``Only then will I stop being angry---my burning in anger. Then they'll know that I've spoken out in my arduous anger. Only then will my burning anger\fnote{The Heb. lacks \fbib{anger}} against them be exhausted. \v{14}I'm also going to turn you into a waste and an object of insult among the nations that surround you and in front of every person who passes by. \v{15}As a result, Jerusalem\fnote{MT reads \fbib{it}; DSS 11QEzek reads \fbib{you}} will become an insult, an object of taunt, an example of chastisement, and a useless waste to all the nations that surround you when I carry out my sentence against you in my anger, my burning rage, and my burning rebukes. I, the \divine{Lord}, have spoken it. \v{16}I'll send arrows of severe famine in their direction, meant for destruction, which I'll shoot, intending to destroy them. I'll make you have more and more famines that will attack you, and I'll disrupt your source of food.\fnote{Lit. \fbib{your staff of bread}}

\v{17}``I'll send famine and wild beasts against you that will rob you of your children.\fnote{Lit. \fbib{will make you childless}} Pestilence and bloodshed will devastate you when\fnote{Lit. \fbib{because}} I'll declare war on\fnote{Lit. \fbib{I'll bring the sword against}} you. I, the \divine{Lord}, have spoken.''
\labelchapt{6}
\passage{Prophecy against the Mountains of Israel}

\chapt{6}
\v{1}The \divine{Lord} continued with his message to me. \v{2}``Son of Man,'' he said, ``turn your face to oppose the mountains of Israel and prophesy against them. \v{3}Tell the mountains of Israel to listen as the Lord \divine{God} speaks. This is what the Lord \divine{God} has to say to the mountains, hills, streams, and the valleys: `Look! I'm about to bring my sword against you. I'm going to destroy your high places. \v{4}Your altars will become desolate and your sun pillars will be shattered. I'll throw your slain down right in front of your idols. \v{5}I'll place the corpses of the Israelis in front of their idols. I'll scatter your bones around your altar. \v{6}In all the places where you live, the cities will be desolate. The high places will also be desolate so that your altars will be laid waste, bearing the punishment appropriate to them.\fnote{The Heb. lacks \fbib{appropriate to them}} Your idols will be shattered, your sun pillars will be hewn down, and your works will be obliterated. \v{7}The fatally wounded among you will fall, and at that time you'll know that I am the \divine{Lord}. \v{8}I'll leave a remnant among you---those who will escape the sword when I'll have scattered you throughout the earth. \v{9}Your survivors will remember me among the nations where they'll be taken captives. I've been crushed by their unfaithful\fnote{Lit. \fbib{whoring}} hearts that have turned against me. \v{10}Then they'll know that I am the \divine{Lord}. I didn't declare this evil that's intended for them\fnote{\fbib{evil to do to them}} without a reason.'\,''

\v{11}This is what the Lord \divine{God} says: ``Clap your hands and stamp your feet! Say, `Oh, no!' Because of all the detestable evil that has come from Israel's house, they'll fall by the sword, famine, and pestilence. \v{12}The one who lives far away will die by pestilence and the one who is near will die violently.\fnote{Lit. \fbib{will fall by the sword}} The survivors and their surveillance details will die by famine as I exhaust my rage against them.

\v{13}``You'll learn\fnote{Or \fbib{know}} that I am the \divine{Lord}, when the fatally wounded will be among their idols, around their altars, on every hill, on top of the mountains, under every luxuriant tree, and under all the full-grown\fnote{Lit. \fbib{under every high}} foliage---every place where they've offered fragrant aromas to all their idols. \v{14}I'll stretch out my hands to strike\fnote{Lit. \fbib{hands against}} them and send devastation to the land, from the wilderness of Diblah, throughout all their dwelling places. Then they'll know that I am the \divine{Lord}.''
\labelchapt{7}
\passage{The End has Come}

\chapt{7}
\v{1}This message from the \divine{Lord} arrived for me: \v{2}``Son of Man, this is what Lord \divine{God} says to the land of Israel: `It's over! All four corners of the land are out of time! \v{3}Your time is up! I'm sending my anger against you to judge you according to how you live your lives,\fnote{Lit. \fbib{to your ways}} and I'm going to pay you back with the consequences of all your detestable practices. \v{4}I won't be showing pity on you and I won't be showing compassion. I'm going to turn your own lifestyles against you while your detestable practices remain among you. Then you'll learn\fnote{Or \fbib{know}} that I am the \divine{Lord}.'\,''
\passage{One Bad Thing after Another}

\v{5}``This is what the Lord \divine{God} says: `It's one evil event after another!

```Look out! It's coming!

\v{6}```The end is coming!

```The end is here!

```And it's looking in your direction!\fnote{Lit. \fbib{looking for you}}

```Look out! It's arrived!

\v{7}```Your doom has come to you, you who live in the land. The time has arrived, and the day of confusion is near. There will be no shouts of joy on the mountains. \v{8}Very soon now, I'll pour out my burning anger on you. I'll complete expressing my anger at you, judge you according to your behavior, and repay you for all your detestable practices. \v{9}I won't be showing pity or compassion. I'll repay you according to your behavior while your detestable practices remain among you. And you'll know that I, the \divine{Lord}, have been attacking you.'\,''\fnote{The Heb. lacks \fbib{you}}
\passage{The Harvest Approaches}

\v{10}``Look out! The day!

``Look out! It's coming!

``Doom has blossomed.

``Arrogance has sprouted!

\v{11}``Violence has matured into a branch that is wicked. No one will survive from that vast crowd, from their wealthy people, or from the famous among them.

\v{12}``The time has come!

``The day has arrived.\fnote{Lit. \fbib{reached}} Don't let the buyer rejoice, nor the seller lament, because wrath is coming to attack the entire multitude. \v{13}The seller won't regain what he has sold while the crowd remains\fnote{Lit. \fbib{while they're}} alive, because the vision concerning the entire multitude won't be annulled. No person will be able to survive because of the sin in his life.

\v{14}``They've sounded the alarm,\fnote{Lit. \fbib{They've blown the trumpet}} and everyone is prepared, but no one is marching for battle, since I'm angry at the entire multitude. \v{15}The sword lurks outside, but pestilence and famine are on the prowl inside the house. Whoever is in the field will die by violence,\fnote{Lit. \fbib{by the sword}} while famine and pestilence will devour those in the city. \v{16}Fugitives will escape to the mountains like doves fleeing through the valleys, all of them moaning because of their own iniquity. \v{17}Every hand will be limp. Every knee will glisten with sweat.''\fnote{Lit. \fbib{water}}
\passage{The Coming Terror}

\v{18}``They'll clothe themselves with sackcloth, terror will overcome them, shame will cover their faces, and baldness will spread over their entire heads. \v{19}They'll fling their silver into the streets, and their gold will be cast away as impure. Their silver and gold won't be able to deliver them during the time\fnote{Lit. \fbib{day}} of the \divine{Lord}'s wrath. They won't be able to satisfy their appetites or fill their stomachs, because their iniquity has tripped them up.''
\passage{The Temple Defiled}

\v{20}``As for his beautiful ornament,\fnote{I.e. the temple in Jerusalem} he set it up in majesty, but they made detestable images and loathsome idols. Therefore, I'll give them something loathsome--- \v{21}I'll give it as plunder into the control of strangers and as the spoils of war to the wicked who will invade the land to profane it. \v{22}I'll turn my face away from them so that they'll defile my treasured place. Robbers will enter and profane it!

\v{23}``Forge a chain, because the land is full of bloody judgment and the city is filled with violence. \v{24}Therefore, I'm bringing the worst of the nations, who will take possession of their houses. I'll cause the pride of the mighty to cease, and their sanctuaries will be profaned.

\v{25}``When destruction comes, they'll seek peace, but there will be none to be found. \v{26}Disaster upon disaster will come, followed by rumor after rumor. They'll seek an oracle from the prophet, but the Law will be gone from the priests, and counsel from the elders.

\v{27}``The king will mourn, the prince will be clothed with desolation,\fnote{Or \fbib{with torn garments}} and the hands of the people of the land will tremble. I'll deal with them according to their behavior and I will judge them by how they've judged. Then they'll learn\fnote{Or \fbib{know}} that I am the \divine{Lord}.''
\labelchapt{8}
\passage{The Vision of Jerusalem}

\chapt{8}
\v{1}In the sixth year, on the fifth day of the sixth month, I had just sat down in my house, with the elders of Judah seated in front of me. All of a sudden, the hand of the Lord \divine{God} touched me \v{2}and I saw a likeness comparable to the appearance of a man. From his thighs downward there was the appearance of fire, and from his waist upward, there was the appearance of brightness that looked like brass.

\v{3}The form of a hand reached out and took me by the hair of my head. Then the Spirit lifted me up between the earth and sky, brought me toward Jerusalem, and in visions that came from God took me through the doors of the inner gate that faced north, where an image that provoked God's jealous anger had been erected.

\v{4}All of a sudden, the glory of the God of Israel was there! It looked like what I had seen back in the valley. \v{5}Then he told me, ``Son of Man, look up toward the north.''

So I looked off toward the north. Suddenly, off toward the north, facing the gate that led to the altar, the image that provoked God's jealousy was standing near the entrance.

\v{6}Then the Spirit\fnote{Lit. \fbib{Then he}} told me, ``Son of Man, don't you see what they're doing? The house of Israel practices awful, detestable things here, so I'm going far away from my sanctuary. But you're about to see things even more detestable than these.''
\passage{Idol Worship in the Temple}

\v{7}Then the Spirit\fnote{Lit. \fbib{Then he}} brought me to the entrance of the court. As I watched, all of a sudden, there was a\fnote{Lit. \fbib{one}} hole in the wall! \v{8}Then he told me, ``Son of Man, dig through the wall!'' So I dug into the wall. That's when I uncovered an entrance!

\v{9}Then he told me, ``Go on through that entrance, so you may see the wicked, detestable things that they're committing here.''

\v{10}So I entered, looked around, and there was every form of crawling thing, loathsome animals, and all kinds of idols from the house of Israel carved all around the wall. \v{11}I saw 70 men from the elders of the house of Israel standing among them, including Shaphan's son Jaazaniah. Each man held a censer in his hand. As the scent of the cloud of incense ascended, \v{12}the Spirit\fnote{Lit. \fbib{he}} asked me, ``Do you see, Son of Man, what the elders of Israel's house are doing in secret, each in the chamber of his own carved idol? They keep saying, `God doesn't see us. The \divine{Lord} has abandoned the land.'\,''

\v{13}Then the Spirit\fnote{Lit. \fbib{Then he}} told me, ``You're about to see even more detestable practices that they're doing!''
\passage{Women Weeping for Tammuz}

\v{14}Then he brought me to the entrance of the gate to the \divine{Lord}'s Temple, which faced the north. That's where I saw women seated, weeping for Tammuz. \v{15}Then he asked me, ``Do you see this, Son of Man? You're about to see even more detestable practices than these.''
\passage{Sun Worship in the Temple}

\v{16}Then he brought me to the inner court of the \divine{Lord}'s Temple. There, at the entrance to the \divine{Lord}'s Temple, between the porch and the altar, were 25 men, with their backs toward the \divine{Lord}'s Temple and facing the east, prostrating themselves to the sun.

\v{17}``Do you see this, Son of Man?'' he asked me. ``Is it an insignificant thing for Judah's house to commit the detestable things that they're doing here? They've filled the land with violence and turned away from me, causing me to become angry again. Look how they're sniffing with their noses!\fnote{So MT; i.e. using flora to create or sustain an altered state during idolatrous worship; LXX reads \fbib{with contempt}} \v{18}I'm going to deal with them in rage and anger. I'll show neither pity nor compassion. They'll cry loudly directly in my ears, but I won't listen to them.''
\labelchapt{9}
\passage{The Vision of the Executioners}

\chapt{9}
\v{1}Then the Spirit\fnote{Lit. \fbib{Then he}} shouted right in my ears with a loud voice! ``Come forward,'' he said, ``you executioners of the city, and bring your weapon of destruction in your\fnote{Lit. \fbib{his}} hand!''

\v{2}All of a sudden, I noticed six men approaching from the direction of the upper gate, which faces north. Each of them held a destructive weapon in his hand. Among them there was one man, clothed in linen, who was equipped with a writing set\fnote{I.e., a case containing ink and writing implements} at his side. They went in and presented themselves beside the bronze altar. \v{3}Then the glory that is Israel's God arose from the cherubim on which he had been seated and settled on the threshold of the Temple. He called out to the man dressed in linen who wore the writing case at his side.

\v{4}The \divine{Lord} told him, ``Go throughout the city of Jerusalem and put a mark on the foreheads of everyone who sighs and moans over all of the loathsome things that are happening in it.''

\v{5}As I continued to listen, he also told the others, ``Follow him through the city and start killing. Don't spare anyone you see, and don't show pity of any kind. \v{6}You are to execute old men, young men, young women, little children, and women. But don't touch anyone who has been marked. Begin at my Holy Place!'' And so they started with the elders who were in standing in front of the Temple.

\v{7}``Desecrate my Temple,'' he told them, ``and fill its courtyard with the dead!'' So they went out and began striking down people throughout the city.
\passage{Ezekiel Intercedes for Israel}

\v{8}While they were out carrying out the executions, I was left alone. So I fell on my face and cried out, ``O Lord \divine{God}, are you going to destroy all of the survivors of Israel when you pour out your anger on Jerusalem?''

\v{9}``The house of Israel and Judah is guilty---and theirs is a stubborn guilt, at that!'' he replied to me. ``The land is filled with blood, and the city overflows with injustice, because they keep saying, `The \divine{Lord} has abandoned the land,' and `The \divine{Lord} isn't watching.' \v{10}So as for me, I'm not going to show pity, and I won't look in their direction with mercy. I'm repaying them for what they have done.''

\v{11}Then I noticed the man dressed in linen who wore the writing case by his side as he brought back this message: ``I've done as you have commanded me.''
\labelchapt{10}
\passage{The Vision of God's Throne}

\chapt{10}
\v{1}As I continued to watch, there on the expanse above the heads of the cherubim was a massive\fnote{The Heb. lacks \fbib{massive}} sapphire stone that resembled a throne in form and appearance. \v{2}The \divine{Lord}\fnote{Lit. \fbib{He}} spoke to the man who was clothed in white linen, telling him, ``Go between the whirling wheels, under the cherubim, and fill your hands with burning coals from among the cherubim. Then scatter them\fnote{The Heb. lacks \fbib{them}} over the city.'' So he entered as I watched.\fnote{Lit. \fbib{entered in my sight}}

\v{3}Now the cherubim were standing on the south\fnote{Lit. \fbib{right side}} side of the entrance to the Temple, when the man entered and a cloud filled the inner court. \v{4}The glory of the \divine{Lord} rose above the cherub and moved to the threshold of the Temple. A cloud filled the Temple and the court was filled with the brilliance of the \divine{Lord}'s glory. \v{5}The sound of the wings of the cherubim, reminiscent of the voice of the Sovereign God when he speaks, could be heard as far as the outer court.

\v{6}He issued this order to the man who was clothed in white linen: ``Take fire from within the whirling wheels, among the cherubim.'' So he went and stood beside the wheels.
\passage{Ezekiel's Vision of the Cherubim}

\v{7}Then a cherub stretched out his hand to the fire, which was among the cherubim, took some of the fire, and placed it in the hands of the one clothed in white linen, who took it and left. \v{8}There appeared to be human hands under the wings of the cherubim.

\v{9}As I continued to watch, I observed four wheels beside the cherubim, one wheel beside each cherub.\fnote{Lit. \fbib{cherub and another wheel beside another cherub.}} The wheels resembled beryl stone. \v{10}In appearance, the four wheels looked like they consisted of a wheel within a wheel. \v{11}Whenever they moved, they proceeded without turning around as they moved, but they followed in the direction where their head was facing, without looking around as they moved.

\v{12}Their entire bodies, backs, hands, and wings were filled with eyes around, including each of their four wheels. \v{13}The wheels whose sound I was hearing were called ``the whirling wheels''. \v{14}Each had four faces. The first one was the face of a cherub, the second the face of a man, the third the face of a lion, and the fourth the face of an eagle.

\v{15}The cherubim arose. These were the same beings that I had seen at the Chebar River. \v{16}When the cherubim moved, the wheels went alongside them. But when the cherubim started to ascend, beating their wings to rise above the earth, the wheels beside them didn't turn. \v{17}When they stood still, the wheels stood still. When they rose up, the wheels rose up, too, because they were alive.\fnote{Lit. \fbib{because the spirit of the living beings resided in the wheels}}

\v{18}Then the glory of the \divine{Lord} moved away from the threshold of the Temple and stood over the cherubim. \v{19}The cherubim lifted their wings and rose above the earth while I watched. They went out, along with their wheels, and stood at the entrance to the east gate of the \divine{Lord}'s Temple as the glory of Israel's God remained above, covering them.

\v{20}These were the living beings that I had seen under the God of Israel on the bank of the Chebar River. I knew that they were cherubim. \v{21}Each one had four faces. Each one had four wings, and the form of human hands could be seen under their wings. \v{22}As to the likeness of their faces, they were like what I had seen on the bank of the Chebar River. They each moved straight ahead.
\labelchapt{11}
\passage{The Vision of the Eastern Gate}

\chapt{11}
\v{1}The Spirit lifted me up and brought me to the east facing gate of the \divine{Lord}'s Temple. At the entrance of the gate I saw 25 men. Included among them were Azzur's son Jaazaniah and Benaiah's son Pelatiah, who were princes of the people.

\v{2}Then he told me, ``Son of Man, these men are plotting evil and are giving wicked advice in this city. \v{3}They keep saying, `The right time to build families\fnote{Lit. \fbib{houses}} hasn't yet arrived. The city is the pot and we are the meat.' \v{4}Therefore you are to prophesy against them. Prophesy, Son of Man!''
\passage{God Rebukes those who Plot Evil}

\v{5}Just then the Spirit of the \divine{Lord} took control of\fnote{Lit. \fbib{\divine{Lord} fell on}} me and told me, ``You are to say, `This is what the \divine{Lord} says: ``You've said, O house of Israel, that I know what goes through your mind.\fnote{Or \fbib{spirit}} \v{6}You've increased the number of fatally wounded in this city and you've filled your streets with the dead.''

\v{7}`Therefore this is what the Lord \divine{God} says, ``The corpses that you've laid out in your midst are the meat, and this city is the cooking pot. But you'll be taken out from the middle of it. \v{8}You've feared the sword,\fnote{I.e. execution during military invasion; and so throughout the chapter} but I'm bringing violent death in your direction,''\fnote{Lit. \fbib{bringing the sword against you}} declares the Lord \divine{God}. \v{9}``I'm bringing you out from the middle of it and I'm going to deliver you into the hands of strangers, because I'm going to carry out my sentence against you. \v{10}You're going to die violently,\fnote{Lit. \fbib{to fall by the sword}} and I'll judge you as far as the borders of Israel. Then you'll learn\fnote{Or \fbib{know}} that I am the \divine{Lord}. \v{11}This city won't be your cooking pot and neither will you be the meat in it, because I'm going to judge you as far as the borders of Israel. \v{12}Then you'll learn\fnote{Or \fbib{know}} that I am the \divine{Lord}, because you didn't live by my statues or obey my ordinances. Instead, you obeyed the ordinances of the nations around you.''\,'\,''
\passage{Ezekiel Reacts to Pelatiah's Death}

\v{13}While I was prophesying, Benaiah's son Pelatiah died, so I fell on my face and cried out with a loud voice. ``Ah, Lord \divine{God},'' I said, ``are you going to put an end to the survivors within Israel?''

\v{14}Then this message came to me from the \divine{Lord}: \v{15}``Son of Man, your brothers, your other relatives, your fellow exiles,\fnote{So LXX and Syr; MT reads \fbib{your redeemers}} and the entire house of Israel are the people to whom the inhabitants of Jerusalem have said, `They've abandoned the \divine{Lord}. This land was given to us for an inheritance.'\,''
\passage{The Future Hope of Israel}

\v{16}``Therefore you are to say, `This is what the Lord \divine{God} says, ``Although I've removed them far away to live among the nations, and although I've scattered them throughout the earth, yet I've continued to be their sanctuary, even for the short time that they will be living in the lands to which they've gone.''\,'

\v{17}``Therefore you are to say, `This is what the Lord \divine{God} says, ``I'm going to gather you from among the nations, assembling you from the lands among which you have been dispersed. I'll give you the land of Israel. \v{18}When they return from there and cast away all of their loathsome things and detestable practices, \v{19}then I'll give them a united heart, placing a new spirit within them.\fnote{So LXX, Syriac, Targums, and Vulgate; MT reads \fbib{you} (pl)} I'll remove their stubborn heart\fnote{Lit. \fbib{heart from their flesh}} and give them a heart that's sensitive to me.\fnote{Lit. \fbib{heart of flesh}} \v{20}When they live by my statutes and keep my ordinances by observing them, then they'll be my people and I will be their God. \v{21}But to those whose hearts delight in loathsome things and detestable practices, I'll bring the consequences of their behavior crashing down on their own heads,'' declares the Lord \divine{God}.'\,''
\passage{The Cherubim Leave}

\v{22}Then the cherubim arose, with their wheels alongside, and the glory of Israel's God remained above and over them. \v{23}The glory of the \divine{Lord} went up from the middle of the city and stood on the mountain, east of the city. \v{24}Then in a vision from the Spirit of God, the Spirit lifted me up and brought me to the exiles in Chaldea. At that point, the vision that I had been observing ended. \v{25}Later, I spoke to the exiles concerning everything the \divine{Lord} had spoken that I had witnessed.
\labelchapt{12}
\passage{Ezekiel Packs for Exile}

\chapt{12}
\v{1}This message came to me from the \divine{Lord}: \v{2}``Son of Man, you live in a rebellious house that has eyes to see, but they can't see, and ears to hear, but they can't hear, since they're a rebellious house.

\v{3}``So now, Son of Man, you are to prepare your luggage for a trip into exile, and then you are to leave during the daytime so they see you leaving. Leave from your place to another while they're watching. Then perhaps they'll realize that they're a rebellious house.

\v{4}``Bring out your luggage, like you're packing to go into exile, and do this during the daytime while they're watching you.\fnote{Lit. \fbib{before their eyes}; and so through v. 7} Later that evening, leave while they're watching you like someone heading into exile. \v{5}While they continue to watch, dig a hole for yourself in the wall and enter through it.

\v{6}``While they're watching, carry your luggage\fnote{Lit. \fbib{carry it}} on your shoulder and go out in total\fnote{Lit. \fbib{thick}} darkness. Cover your face so that you won't see the land, because I'm using you as a sign to Israel's house.''

\v{7}I did just as I was commanded. I brought out the luggage as if it were luggage for exile. I did this during the day. Then in the evening I dug a hole in the wall with my hand and brought the luggage out in total\fnote{Lit. \fbib{thick}} darkness and carried it out on my shoulder while they were watching.
\passage{The Meaning of the Message}

\v{8}The next morning, this message came to me from the \divine{Lord}: \v{9}``Son of Man, didn't the house of Israel, that rebellious house, ask you, `What are you doing?' \v{10}Answer them, `This is what the Lord \divine{God} says, ``This oracle concerns the prince of Jerusalem and the whole of Israel's house that is in their midst. \v{11}Tell them, `I'm a sign for you. Just as I enacted it,\fnote{Lit. \fbib{did}} it's going to happen to them. They'll go into exile and captivity. \v{12}Then the prince, who will be one of them, will carry his luggage\fnote{The Heb. lacks \fbib{his luggage}} on his shoulder in the dark and will go out. They'll dig a hole in the wall for him to go through. His face will be covered so that he won't be able to see the land with his eyes. \v{13}But I'll throw my net over him. As a result, he'll be captured with my net, and with it I'll bring him to Babel, the land of the Chaldeans. He won't see it, though he'll die there. \v{14}I'll scatter every attendant who surrounds him, along with his entire army, to every wind. When I unsheathe my sword to pursue them, \v{15}they'll learn\fnote{Or \fbib{know}} that I am the \divine{Lord}, when I've dispersed them among the nations and scattered them throughout the earth.''\,'\,''
\passage{The Purpose of the Surviving Remnant}

\v{16}``But I'll preserve\fnote{Or \fbib{retain}} a few people out of the violent death,\fnote{Lit. \fbib{the sword}} famine, and pestilence, so they can recount their detestable practices among the nations when they'll go there. Then they'll know that I am the \divine{Lord}.''
\passage{The Coming Devastation}

\v{17}This message came to me from the \divine{Lord}: \v{18}``Son of Man, eat your bread with trembling and drink your water with quivering and anxiety. \v{19}Then tell the people of the land, `This is what the \divine{Lord} says to the inhabitants of Jerusalem, to Israel's land: ``They'll eat their food in anxiety and drink their water in trepidation, because their land will be desolate in its entirety due to all the violence committed by all who live in it. \v{20}The towns that are inhabited will lie in ruins, because the land will be devastated. Then they'll learn\fnote{Or \fbib{know}} that I am the \divine{Lord}.''\,'\,''
\passage{The Coming Fulfillment of Visions}

\v{21}Later, this message came to me from the \divine{Lord}: \v{22}``Son of Man, what's this proverb you have concerning Israel's land that says, `The days pass slowly and every vision ends in nothing.'?\fnote{Lit. \fbib{vision is destroyed}} \v{23}Therefore you are to tell them, `This is what the Lord \divine{God} says, ``I'm about to put an end to use of this proverb in Israel. It will never be used again as a proverb in Israel. Instead, tell them that the days are drawing near when every vision will be fulfilled. \v{24}There will no longer be worthless visions and flattering divinations in the midst of Israel's house. \v{25}Because I am the \divine{Lord}, I'll speak and the message that I communicate will be accomplished without delay. While you continue to be a rebellious house, I'll speak the message and then fulfill it,'' declares the Lord \divine{God}.'\,''
\passage{The Imminent Fulfillment}

\v{26}Later, this message came to me from the \divine{Lord}: \v{27}``Son of Man, pay attention! The house of Israel keeps on saying, `The vision that he's talking about concerns the distant future. He's prophesying concerning times that are far in the future!' \v{28}Therefore tell them, `This is what the Lord \divine{God} says, ``None of my messages will be delayed any longer. Any message that I speak will be fulfilled,'' declares the Lord \divine{God}.'\,''
\labelchapt{13}
\passage{A Prophecy against Prophets}

\chapt{13}
\v{1}This message came to me from the \divine{Lord}: \v{2}``Son of Man, prophesy against the prophets of Israel, who even now are prophesying, and tell those prophets that keep on prophesying according to what they wish would happen,\fnote{Lit. \fbib{prophesying from their heart}} `Listen to what the \divine{Lord} says.'\,''

\v{3}``This is what the Lord \divine{God} says, `How terrible it will be for the false prophets who walk according to their own wrong inclinations\fnote{Lit. \fbib{spirit}} and see nothing. \v{4}Israel, your prophets have become like foxes among ruins. \v{5}You didn't go up to repair\fnote{The Heb. lacks \fbib{repair}} the breaches in the walls and you didn't build the walls so Israel's house would be able to endure battle on the Day of the \divine{Lord}. \v{6}Instead, they crafted\fnote{Lit. \fbib{they have seen}} false prophecies and divination.

```They say, ``{\ldots}declares the \divine{Lord},'' even though the \divine{Lord} didn't send them. And they hope for the fulfillment of their message. \v{7}You've crafted\fnote{Lit. \fbib{seen}} a false prophesy and spoken deceptive divination, haven't you? But then you say, ``{\ldots}declares the \divine{Lord},'' although I haven't spoken a single word.

\v{8}```Therefore this is what the Lord \divine{God} says, ``Because you've spoken falsehood and deceptions, I am therefore opposing\fnote{Lit. \fbib{against}} you,'' declares the Lord \divine{God}. \v{9}My hand will oppose the prophets who see false visions and speak deceptive divinations. They won't be included with the council of my people, nor will they be entered into the registry of Israel's house or enter Israel's land. Then you'll know that I am the Lord \divine{God}, \v{10}because they've truly caused my people to stray saying, ``Peace,'' but there's no peace.'\,''
\passage{Metaphor of the Whitewashed Wall}

``When someone builds a wall, they coat it with whitewash. \v{11}Tell those who coat it with whitewash that it will fall. It will be washed off by the rain. Great hailstones will fall and a stormy wind will strip it off.\fnote{Lit. \fbib{rip it open}} \v{12}Look! When the wall collapses, won't it be said of you, `Where's the coat of paint that you spread all over the wall?'

\v{13}``Therefore this is what the Lord \divine{God} says, `In my burning anger, I'll rip it open with a windstorm. In my anger, I'll rinse it off with rain, and put an end to it with a hailstorm in my destructive rage. \v{14}I'll tear down the wall that you've smeared with whitewash, level it to the ground, and tear out its foundation. Then it will collapse---and you'll perish with it! Then you'll know that I am the \divine{Lord}.

\v{15}```That's how I'll vent my anger on the wall and on the ones who coated it with whitewash. And I'll say to you, ``The wall is gone and so are those who coated it.''\fnote{Lit. \fbib{Those who coated it are not.}} \v{16}The prophets of Israel prophesied about Jerusalem and saw visions of peace concerning her, yet there's no peace,'\,'' declares the Lord \divine{God}.
\passage{A Rebuke to Israel's Women}

\v{17}``And now, Son of Man, turn toward and oppose\fnote{Lit. \fbib{Man, set your face against}} the women\fnote{Lit. \fbib{daughters}} of your people who prophesy according to their own wrong inclinations\fnote{Lit. \fbib{spirit}} and prophesy against them. \v{18}Tell them, `This is what the Lord \divine{God} says, ``How terrible it will be for those women who sew magical bracelets on all their wrists and make one-size-fits all headbands,\fnote{Or \fbib{veils}} in order to entrap their souls. Will you hunt for the souls of my people and remain alive? \v{19}You've profaned me among my people for a handful of barley and a morsel of bread. You're causing people to die who shouldn't have to die, and you're causing people to live who shouldn't survive, when you deceive my people who tend to listen to lies.''

\v{20}```Therefore, this is what the Lord \divine{God} says, ``Watch out! I'm opposing your amulets with which you hunt souls as one would swat at a flying insect.\fnote{Lit. \fbib{flying thing}} I'll tear them off your arms and then deliver those people, whom you've hunted like birds. \v{21}I'll also tear off your headbands\fnote{Or \fbib{veils}} and deliver my people from your grip so that they won't be under your control anymore. Then you'll know that I am the \divine{Lord}.

\v{22}`````Because you've dismayed the heart of the righteous---whom I never intended to dismay---with lies, and because you've encouraged\fnote{Lit. \fbib{you've strengthened the hand of}} the wicked so that he wouldn't abandon his evil behavior and by doing so live, \v{23}you'll no longer see false visions or again practice divination, because I'm going to deliver my people from your power. Then you'll know that I am the \divine{Lord}.''\,'\,''
\labelchapt{14}
\passage{A Prophecy against Idolatry}

\chapt{14}
\v{1}Later, some men from the elders of Israel came to visit me. After they had sat down in my presence, \v{2}this message came to me from the \divine{Lord}.

\v{3}``Son of Man, these men have taken idols into their hearts. They've placed the stumbling block that is their own iniquity right in front of their faces. Should I be consulted by them at all? \v{4}Therefore, speak up and tell them, `This is what the Lord \divine{God} says, ``Every person from Israel's house who follows his idols and sets the stumbling block that is his own sin in front of his face, and then consults a prophet, I the \divine{Lord} will answer him according to how many idols he embraces. \v{5}I'll do this in order to capture the hearts of Israel's house who have become alienated from me due to all of their idols.''\,'\,''
\passage{An Exhortation to Turn Away}

\v{6}``Therefore you are to tell Israel's house, `This is what the Lord \divine{God} says, ``Turn away! Turn away from your idols, and abandon your detestable practices! \v{7}For when a native Israeli or a resident alien abandons me to set up idols in his heart behind my back, and then places the stumbling block of his iniquity right in front of his own face, then approaches a prophet to inquire of me on behalf of his own self-interest, I, the \divine{Lord} will answer him myself. \v{8}I'm determined to oppose that person\fnote{Lit. \fbib{man}} and make him an example. Proverbs will be written about him\fnote{The Heb. lacks \fbib{will be written abut him}} when I eliminate him from my people. Then you'll know that I am the \divine{Lord}.''\,'\,''
\passage{On False Prophets}

\v{9}``Now as to the prophet, if through deceit he delivers a message, I the \divine{Lord} have deceived that prophet! I'll reach out in opposition to him and exterminate him from among my people Israel. \v{10}They'll bear the consequences of their guilt, and the prophet will be just as guilty as the one who seeks that prophet's guidance. \v{11}Then Israel's house won't wander away from me again, nor will they defile themselves again with all their transgressions. They'll become my people and I'll be their God,'' declares the Lord \divine{God}.
\passage{On Noah, Daniel, and Job}

\v{12}This message came to me from the \divine{Lord}: \v{13}``Son of Man, when a nation\fnote{Lit. \fbib{land}} sins against me by a treacherous act,\fnote{Lit. \fbib{nation acts treacherously by a treacherous act}} I'll reach out to oppose it, destroying its source of food,\fnote{Lit. will break \fbib{in pieces its staff of bread.}} by sending famine against it, and by destroying both people and beast within it. \v{14}Though these three men, Noah, Daniel,\fnote{cf. Eze 28:3} and Job lived in that land, they would only save their own lives on account of their righteousness,'' declares the Lord \divine{God}.

\v{15}``If I were to make wild animals pass throughout the land, so that they kill its residents\fnote{Lit. \fbib{children}} and it were to become desolate because no one will travel through it due to those wild animals,\fnote{Lit. \fbib{in the face of living beings}} \v{16}then even though these three men were in it, as I live,'' says the Lord \divine{God}, ``they wouldn't be able to deliver even their sons or daughters. They would only save themselves, but the land would become desolate.

\v{17}``Or if I were to bring war to\fnote{Lit. \fbib{bring a sword against}} that land and say, `Hey, sword! Pass throughout the land so I can destroy both man and beasts in it,' \v{18}though these three men lived there, as I live,'' declares the Lord \divine{God}, ``they couldn't deliver their own sons and daughters. They would only save themselves.

\v{19}``Or if I were to send a pestilence against that land and pour out my anger in it with bloodshed, destroying both man and beast in it, \v{20}even though Noah, Daniel, and Job were among them, as I live'' says the Lord \divine{God}, ``they couldn't save their own sons or daughters. They would only save their own souls due to their own righteousness.''

\v{21}This is what the Lord \divine{God} says, ``I'm sending four of my most destructive judgments---military invasion,\fnote{Lit. \fbib{judgments---the sword}} famine, wild animals, and pestilence---into Jerusalem to destroy both human beings and livestock in it. \v{22}But look! There will be a remnant who escapes, a few sons and daughters to be brought out. Look! They'll come out to you and you'll see how they've lived and what they've done, and you'll be comforted concerning the catastrophe that I brought on Jerusalem, including everything that I brought against her. \v{23}They'll comfort you when you see how they've lived and what they've done, because you'll know for certain that I haven't done anything that I've done against them without any reason,''\fnote{Or \fbib{cause}} declares the Lord \divine{God}.
\labelchapt{15}
\passage{A Message about Vines}

\chapt{15}
\v{1}This message came to me from the \divine{Lord}: \v{2}``Son of Man, how does wood from a vine compare to a branch taken from any of the trees in the forest? \v{3}Is wood ever taken from it to make anything practical? Can it even be made into a peg to hang something on? \v{4}After all, it's useful only for kindling a fire, isn't it? And once you've burnt up the ends and charred through the middle of it, is it useful for anything else? \v{5}If it was useless before it was burned, now that it's been burned and charred through, it's even more useless!

\v{6}Therefore this is what the Lord \divine{God} says: ``Just as the wood from a grape vine is removed from the forest and used for kindling fires, I'm giving the inhabitants of Jerusalem over \v{7}to punishment. They may have escaped one fire, but the coming fire will burn them up completely, and they will know that I am the \divine{Lord}, when I set myself in opposition to\fnote{Lit. \fbib{set my face against}} them \v{8}and dedicate the land to desolation because of their unfaithful unbelief,'' declares the Lord \divine{God}.
\labelchapt{16}
\passage{A Prophecy Confronting Jerusalem}

\chapt{16}
\v{1}This message came to me from the \divine{Lord}: \v{2}``Son of Man, make known to Israel her detestable practices. \v{3}You are to declare, `This is what the Lord \divine{God} says to Jerusalem: ``Your birth place\fnote{Lit. \fbib{Your origin and birth}} was the territory that belonged to the Canaanites. Your father was an Ammonite and your mother was a Hittite. \v{4}Now as to your birth, on the day you were born your umbilical cord wasn't cut. You weren't washed with water to clean you, and nobody rubbed you with salt. And it's certain that you weren't wrapped in strips of cloth. \v{5}Nobody pitied you to do any of these things for you, and nobody showed you any compassion. You were tossed outside on the ground, because you\fnote{Lit. \fbib{your soul}} were detested from the day you were born.

\v{6}`````When I passed by you, I saw you kicking around, covered in your own blood. That's when I told you, `Live!'---while you were wallowing in your blood. I commanded you to live, even as you lay there in your own blood. \v{7}I made you increase like sprouting grain\fnote{Lit. \fbib{like that which sprouts}} in the field. As a result, you multiplied greatly. Eventually, you reached the age when young women start wearing jewelry. Your breasts were formed, your hair had grown, but you were still bare and naked.''\,'\,''
\passage{God's Betrothal to Jerusalem}

\v{8}``When I passed by you again, I looked at you, and noticed that it was your proper time for love. I spread my cloak\fnote{Lit. \fbib{wings}} over you to cover your nakedness. I made a solemn promise to you and entered into a covenant with you,'' declares the Lord \divine{God}. ``You belong to me. \v{9}I bathed you with water, rinsed your own blood from you, and anointed you with oil. \v{10}Then I covered you with embroidered clothing, clothed your feet with leather sandals, wrapped\fnote{Lit. \fbib{bound}} you with fine linen, and dressed you in silk. \v{11}I adorned you with jewels, placing bracelets on your hand and necklaces on your neck. \v{12}I put a ring in your nose, earrings in your ears, and a crown encrusted with jewels on your head. \v{13}You were adorned with gold, silver, clothing of fine linen, silk, and embroidery. You ate food made from the finest flour, honey, and olive oil. You were exceedingly beautiful, attaining royal status. \v{14}Your fame\fnote{Lit. \fbib{name}} spread throughout the nations because of your beauty. You were perfectly beautiful due to my splendor with which I endowed you,'' declares the Lord \divine{God}.
\passage{Jerusalem's Arrogant Unfaithfulness}

\v{15}``But you trusted in your beauty. You did what whores do, as a result of your fame. You passed out your sexual favors\fnote{Lit. \fbib{adulteries}} to anyone who passed by, giving yourself to anyone. \v{16}You took some of your clothes and made gaily-colored high places and prostituted yourself all around them---something which had never happened before nor will ever happen again.

\v{17}``You also took your fine jewelry---including my gold and my silver that I had given you. Then you made for yourself male images and had sex with them! \v{18}You took your embroidered gowns and made clothes to cover them. Then you offered my olive oil and incense to them.

\v{19}``Not only that, you took the food I gave you---my fine flour, olive oil, and honey with which I fed you, and you offered\fnote{Or \fbib{set}} them to those gods\fnote{The Heb. lacks \fbib{to those gods}} in order to appease them.\fnote{Lit. \fbib{gods as a pleasing aroma}} That's exactly what happened,'' says the Lord \divine{God}. \v{20}``Then you took your sons and daughters whom you bore for me and sacrificed them for your idols to eat. As though your prostitutions were an insignificant thing, \v{21}you also slaughtered my sons and offered them to idols, incinerating them in fire.\fnote{The Heb. lacks \fbib{in fire}} \v{22}Throughout all of your detestable practices and immorality, you never did remember your earlier life when you were bare, naked, and wallowing in your own blood.''
\passage{The Unfaithfulness of God's People}

\v{23}``How terrible! How terrible it will be for all of your wickedness!'' declares the Lord \divine{God}. \v{24}``You built raised mounds and high places for yourself on every plaza. \v{25}At every street corner you made your beauty abhorrent when you made yourself available for sex to\fnote{Lit. \fbib{you parted your legs for}} anyone who was passing by. By doing this, you kept on committing more and more immorality. \v{26}Then you committed immorality with your neighbors, the Egyptians,\fnote{Lit. \fbib{sons of Egypt}} with perverted lust,\fnote{Lit. \fbib{with great flesh}} and by doing so you fornicated even more, provoking me to anger.

\v{27}``Therefore, look out! I've reached out to oppose you. I withdrew your rations and delivered you\fnote{Lit. \fbib{soul}} to those Philistine women who hate you. Even they were embarrassed at your wicked ways! \v{28}You committed immorality with the Assyrians,\fnote{Lit. \fbib{sons of Assyria}} because you still weren't satisfied. You committed immorality with them, but you still weren't satisfied. \v{29}You committed even more immorality with that land of the merchants, the Chaldeans. But you weren't satisfied even with these!

\v{30}``How weak is your heart,'' declares the Lord \divine{God}, ``when you committed all of these deeds, the acts of an imperious whore! \v{31}When you built your mound on every street corner and constructed your high place at every plaza, you weren't like a common prostitute, in that you've insulted the wages of a prostitute \v{32}who commits adultery, preferring a stranger over her husband!

\v{33}``All prostitutes receive gifts, but you give your gifts to all your lovers, then you bribe them to come to you from everywhere to get your sexual favors!\fnote{Lit. \fbib{your fornications}} \v{34}You're different from other women when you commit immorality---no one can match you in that!\fnote{Lit. \fbib{in how you commit immorality}} After all, you pay fees, but no fee is given to you. You're certainly different!''
\passage{The Coming Punishment}

\v{35}``Therefore listen to this message from the \divine{Lord}, you whore! \v{36}This is what the Lord \divine{God} says: `Because your lust has been poured out and your nakedness has been uncovered by your acts of fornication with your lovers, and because of all your detestable idols and the blood of your sons, whom you offered to them, \v{37}therefore, watch out! I'm about to gather all your lovers from whom you've received your pleasure, everyone whom you've loved, and those whom you've hated. I'll gather them together to oppose you from every side, and they'll uncover your nakedness in their presence. Then they'll see you completely naked. \v{38}I'll judge you with the same standards by which I issue verdicts against a woman who commits adultery and murder.\fnote{Lit. \fbib{and one who sheds blood}} I'll avenge the blood you've shed with impassioned wrath.\fnote{Lit. \fbib{wrath and passion}}

\v{39}``I'll also deliver you into their control, and they'll break down your mounds, tear down your high places, strip off your clothes, remove your fine jewels, and then they'll leave you stark naked! \v{40}They'll bring a mob against you to stone you to death\fnote{Lit. \fbib{death with stones}} and cut you into pieces with their swords. \v{41}Then they'll burn your houses and carry out my sentence\fnote{Lit. \fbib{and execute judgment}} against you in the sight of many women.

``That's how I'll make you stop your prostitution so you won't pay any prostitute's fees anymore. \v{42}I'll stop being angry with you, and I'll cease being jealous.\fnote{Lit. \fbib{I'll turn aside my jealousy from you}} I'll be calm and not be indignant anymore. \v{43}Because you didn't remember the time when you were young, but instead you provoked me to anger because of all these things, watch out! I'm going to bring your behavior back to haunt\fnote{The Heb. lacks \fbib{haunt}} you!'' declares the Lord \divine{God}. ``Didn't you do this wicked thing, in addition to all your other\fnote{The Heb. lacks \fbib{other}} detestable practices?''
\passage{Like Mother, Like Daughter}

\v{44}``Now, everyone who likes proverbs will quote this proverb about you, `Like mother, like daughter.' \v{45}You're the daughter of your mother, who loathed her husband and children. You're the sister of your sisters, who loathed their husbands and children.

``Your mother was a Hittite and your father was an Amorite. \v{46}Your elder sister was Samaria. She and her daughters lived in the north,\fnote{Lit. \fbib{lived on your left}} while your younger sister who lived in the south\fnote{Lit. \fbib{lived on your right}} with her daughters was Sodom. \v{47}It wasn't just that you lived like they did and committed their detestable practices, but in just a little while your behavior led you to become more corrupt than they were!''
\passage{Sins of Sodom}

\v{48}``As I live,'' declares the Lord \divine{God}, ``your sister Sodom and her daughters didn't do what you and your daughters have done. \v{49}Look! This was the sin of your sister Sodom and her daughters: Pride, too much food, undisturbed peace, and failure to help\fnote{Lit. \fbib{to strengthen the hand of}} the poor and needy. \v{50}In their arrogance, they committed detestable practices in my presence, so when I saw it, I removed them. \v{51}Samaria didn't commit half of your sins---you practiced more detestable deeds than they did! You've caused your sister to be more righteous than you, because of the detestable practices that you've committed. \v{52}So now, bear your own shame as you mediate for your sisters. The sins that you've committed are more detestable than theirs. That makes them more righteous than you. Indeed, be ashamed and bear your reproach, because you've made your sisters to be more righteous than you.''
\passage{A Change in Circumstances}

\v{53}``I'll bring them back from their captivity---that is, from the captivity of Sodom and her daughters, along with the captivity of Samaria and her daughters and the captivity of your captives among them. \v{54}But you'll continue to bear your own reproach and be humiliated for everything that you've done. You'll be a comfort to them. \v{55}Your sister Sodom and her daughters will be restored to their former status. Samaria and her daughters will be restored to their former status. Then you and your daughters will be restored to your former status.

\v{56}``When you were being so arrogant, you never once mentioned your sister Sodom \v{57}before your wickedness was revealed. Now you've become an object of derision to the inhabitants\fnote{Lit. \fbib{daughters}} of Aram and its neighbors, including the Philistines\fnote{Lit. \fbib{the daughters of the}}---all those around you who despise you. \v{58}You are to bear the punishment of your wickedness and detestable practices,'' declares the \divine{Lord}, \v{59}``since the Lord \divine{God} says, `I'll deal with you according to what you've done, when you despised your oath by breaking the covenant.

\v{60}```Meanwhile, as for me, I'll remember my covenant with you from when you were young, because I'll establish an eternal covenant with you. \v{61}Then you'll remember your behavior and be ashamed when you greet your sisters---your elder sister and your younger sister. I'll give them to you as daughters, but not on account of my covenant with you. \v{62}I'll establish my covenant with you, and then you'll know that I am the \divine{Lord}. \v{63}Then you will remember, be ashamed, and you won't open your mouth anymore due to humiliation when I will have made atonement for you for everything that you've done,' declares the Lord \divine{God}.''
\labelchapt{17}
\passage{The Parable of the Eagle}

\chapt{17}
\v{1}This message came to me from the \divine{Lord}: \v{2}``Son of Man, compose a riddle and relate a parable to Israel's house. \v{3}Tell them, `This is what the Lord \divine{God} says, ``A massive eagle with gigantic wings, long pinions, and full, multi-colored plumage came to Lebanon and took away the top of the cedar.\fnote{I.e. a genus of coniferous evergreen in the family \fbib{Pinaceae}; and so throughout the book} \v{4}He plucked off the top of its shoot, brought it to a land of merchants, and set it down in a city full of traders. \v{5}Then the eagle took a seed from the land and planted it in fertile ground. He planted it like a willow tree next to abundant waters. \v{6}It flourished and became a low, spreading vine. Its branches turned toward him, and its roots spread under him to become a vine that put out shoots and spread out its branches.

\v{7}`````All of a sudden, there was another eagle with gigantic wings and thick plumage. The vine stretched its roots hungrily toward him and spread its branches out to him in order to be watered on the terraces where it was planted. \v{8}It was transplanted into good soil\fnote{Or \fbib{ground}} near abundant water, and it produced branches and bore fruit, becoming a magnificent vine.''\,'

\v{9}``Tell them, `This is what the Lord \divine{God} says, ``Will it prosper? Won't he pull up its roots, and strip it bare so all its fresh foliage dries up? It won't be by great strength or by a great army that it will be uprooted. \v{10}Look! Because it's a transplanted vine, won't it wither when the east wind hits it? It will surely wither in the terraces where it had started to sprout.''\,'\,''
\passage{The Meaning of the Parable}

\v{11}This message came to me from the \divine{Lord}: \v{12}``Tell my\fnote{Lit. \fbib{the}} rebellious house, `Don't you know what these things mean? Look! The king of Babylon came to Jerusalem, captured her king and princes, and took them with him to Babylon. \v{13}Then he took one of the royal descendants, made a covenant with him, and put him under an oath of loyalty, taking the leaders of the land captive \v{14}in order to humiliate the kingdom so it wouldn't be able to return to power, but would still be able to continue as long as he keeps his covenant. \v{15}But he rebelled against the king of Babylon\fnote{Lit. \fbib{against him}} by sending his messengers to Egypt to obtain horses and a large army. Will he succeed? Or will the one who did this escape? Will he break the covenant, but still be delivered?'\,''
\passage{God will Punish the King}

\v{16}``As long as I live,'' declares the Lord \divine{God}, ``in Babylon, that place where the king has enthroned him, whose oath he despised so as to break his covenant, he'll die with him. \v{17}Pharaoh, with his massive army and large battalions won't protect him when mounds and siege walls are built to destroy many people.\fnote{Lit. \fbib{souls}} \v{18}He despised the oath he had made and broke the covenant. Look! Because he willingly submitted,\fnote{Lit. \fbib{He has given his hand}} yet he has done all these things, he won't escape.

\v{19}Therefore, this is what the Lord \divine{God} says, ``As long as I live, because he despised my oath and broke my covenant, he's going to suffer the consequences.\fnote{Lit. \fbib{covenant, I'll bring it upon his head}} \v{20}I'll spread my net over him so that he'll be caught in my snare. I'll bring him to Babylon and carry out my sentence there because of his treachery toward me. \v{21}The fugitives of his troops will die by the sword, and the survivors will be scattered to the four\fnote{Lit. \fbib{to all the}} winds. Then you'll know that I, the \divine{Lord}, have spoken.''
\passage{The Transplanted Vine}

\v{22}This is what the Lord \divine{God} says, ``I'm also going to take a shoot from the top of a cedar and plant it. I'll pluck off its delicate twigs and transplant it on a high and lofty mountain. \v{23}I'll transplant it on Israel's land, and it will grow branches, bear fruit, and become a majestic cedar. All sorts\fnote{Lit. \fbib{wing}} of birds will rest under it, and they'll settle down in the shade of its branches. \v{24}Then all the trees of the fields will know that I, the \divine{Lord}, bring down the lofty tree and exalt the lowly tree. I dry up the green\fnote{Or \fbib{fresh}} tree and cause the dry tree to bud. I the \divine{Lord} have spoken this, and I will fulfill it.''
\labelchapt{18}
\passage{The Outdated Proverb}

\chapt{18}
\v{1}This message came to me from the \divine{Lord}: \v{2}``Why do you cite this proverb when you talk about Israel's land: `The fathers eat sour grapes but it's their children's teeth that have become numb.' \v{3}As long as I live,'' declares the \divine{Lord}, ``you won't use this proverb about Israel anymore. \v{4}Look! Every living soul belongs to me---the father's as well as the son's.\fnote{Lit. \fbib{As the soul of the father, so the soul of the son belongs to me}.} So pay attention! The person who keeps on sinning is going to die.''
\passage{Standards of Righteous Behavior}

\v{5}``If a person is righteous, and practices what's lawful and right, \v{6}if he doesn't eat at mountain shrines, and doesn't look to the idols that have been erected in Israel's house, if he doesn't defile his neighbor's wife or approach a woman during her time of menstrual separation, \v{7}if he doesn't oppress anyone, but instead returns the debtor's security for his debt, if he doesn't rob anyone, but instead shares his food with the hungry and gives clothes to those who are naked, \v{8}if he doesn't lend with usury or exact interest, but instead refuses to participate in\fnote{Lit. \fbib{instead withdraws his hand from}} what is unjust, if he administers true justice between people,\fnote{Lit. \fbib{between man and man}} \v{9}if he lives his life\fnote{Lit. \fbib{he walks}} consistent with my statutes and keeps my ordinances by practicing what's true, then he's righteous and will certainly live,'' declares the Lord \divine{God}.
\passage{Standards of Unrighteous Behavior}

\v{10}``Now suppose that person produces a son who's violent, a murderer, and practices any of these things, \v{11}even though the father\fnote{Lit. \fbib{though he}} hasn't done any of these things. The son who eats at mountain shrines, defiles his neighbor's wife, \v{12}oppresses the afflicted and the poor, robs others, doesn't return security for a debt, looks to idols, does detestable things, \v{13}loans with usury, and exacts interest; will he live? He certainly will not! He has done all these detestable practices. He will certainly die, and his guilt will be his own fault.''\fnote{Lit. \fbib{his blood will be on him}}
\passage{Personal Accountability for Sin}

\v{14}``Now suppose that he produced a son who practiced all of his father's sins, but then that son\fnote{The Heb. lacks \fbib{that son}} began to fear me and stopped doing all of these things. \v{15}That is, suppose he doesn't eat at the mountain shrines, doesn't look to the idols of Israel's house, doesn't defile his neighbor's wife, \v{16}doesn't oppress anyone, doesn't take possession of a debtor's pledge, or doesn't steal, but instead shares his food with the hungry, gives clothes to those who are naked, \v{17}doesn't refuse to help the afflicted, or refuses to loan with usury or exact interest, but instead follows my ordinances and lives his life consistent with my statutes. He won't die because of his father's sin, will he? No! He'll certainly live. \v{18}As for his father, watch out! If he wrongfully oppressed or robbed his brother and did what wasn't good among his people, he'll die because of\fnote{Lit. \fbib{die in}} his own sin.''
\passage{The Person who Sins will Die}

\v{19}``Yet you keep asking, `Why wouldn't the son bear the punishment of his father's sin?' Because the son has done what was lawful and right, and has kept all my statutes and obeyed them, he's certainly going to live. \v{20}The soul who sins dies. The son won't bear the punishment of his father's sin and the father won't bear the punishment of his son's sin. The righteous deeds of that righteous person will be attributed to him, while the wicked deeds of the wicked person will be charged against him. \v{21}But if the wicked person turns from all his sins, which he did and keeps my statutes, then he'll live. He won't die. \v{22}None of the transgressions that he had committed will be held\fnote{Lit. \fbib{remembered}} against him. Because of the righteous deeds that he had done, he'll live.

\v{23}``I don't take delight in the death of the wicked, do I?'' asks the Lord \divine{God}. ``Shouldn't I rather delight\fnote{The Heb. lacks \fbib{shouldn't I rather delight}} when he turns from his wicked ways and lives? \v{24}But when the righteous person abandons his righteous deeds and commits evil, detestable practices, as wicked people do, he won't live, will he? None of the righteous acts that he had done will be remembered. He'll die in his treacherous unfaithfulness and sins that he had committed.''
\passage{Accusing God of Unrighteousness}

\v{25}``Yet you keep saying, `The \divine{Lord} isn't being consistent with his standards.' Pay attention, you house of Israel: Is my behavior really inconsistent with my standards? Isn't it your behavior that isn't just?

\v{26}``When a righteous person turns from his righteous deeds and does evil, he'll die because of that evil. He'll die because of his unrighteous acts that he committed. \v{27}When a wicked person quits\fnote{Or \fbib{abandons}} his wicked behavior\fnote{Lit. \fbib{ways that he had committed}} and does what's just and right, he'll be enabled to live.\fnote{Lit. \fbib{he makes his soul come alive}} \v{28}Because he reconsidered his transgression and turned away from everything that he had been doing, he'll certainly live and not die. \v{29}Yet Israel's house keeps saying, `The \divine{Lord} isn't being consistent with his standards.' Is it my behavior that's inconsistent with my standards?\fnote{The Heb. has \fbib{adjusted to the standard}} Is it not your behavior that's inconsistent with my standards?''\fnote{The Heb. has \fbib{adjusted to the standard}}
\passage{A Command to Repent}

\v{30}``Therefore, Israel, I'm going to judge you according to the behavior of each and every one of you,'' declares the Lord \divine{God}. ``So repent and turn from all your sins so that sin won't keep on being a stumbling block for you. \v{31}Stop your transgressing---the deeds by which you've rebelled---and then make for yourselves a new heart and a new spirit. Why should you die, you house of Israel? \v{32}I don't take pleasure in the death of anyone who dies,'' declares the \divine{Lord}. ``So repent, so you may live!''
\labelchapt{19}
\passage{A Prophecy against Israel's Nobles}

\chapt{19}
\v{1}``Now as for you, publish\fnote{Lit. \fbib{sound forth}} this mourning psalm about Israel's leaders. \v{2}Tell them:

\begin{poetry}
\poeml `What a lioness your mother was among lions! \\
\poemll    She reared her cubs in the midst of fierce young males. \\
\poeml \v{3}She raised one cub in particular, \\
\poemll    teaching that fierce lion to become a hunter-prowler--- \\
\poemlll       to eat human beings. \\
\poeml \v{4}The nations heard about him. \\
\poemll    He had become caught in their trap.\fnote{Lit. \fbib{pit}} \\
\poeml They brought him with hooks \\
\poemll    to the land of Egypt. \\
\poeml \v{5}When she learned that her plans had been frustrated \\
\poemll    and that her hopes were dashed, \\
\poeml she took another of her cubs \\
\poemll    and turned him into a fierce lion. \\
\poeml \v{6}He prowled around among the lions, \\
\poemll    became a strong, young lion, \\
\poeml and learned to become a hunter-prowler--- \\
\poemll    to eat human beings. \\
\poeml \v{7}He raped\fnote{Lit. \fbib{knew}} the women, \\
\poemll    devastating their towns. \\
\poeml The land was made desolate, \\
\poemll    and all the while the land was filled \\
\poemlll       with the sound of his roaring. \\
\poeml \v{8}The surrounding nations attacked. \\
\poemll    They tossed their net over him, \\
\poemlll       and he was caught in their trap.\fnote{Lit. \fbib{pit}} \\
\poeml \v{9}They imprisoned him in a cage with hooks \\
\poemll    and brought him to the king of Babel. \\
\poeml Then they placed him in their dungeon \\
\poemll    where his voice would no longer be heard \\
\poemlll       on the mountains of Israel. \\
\poeml \v{10}`Your mother was like a vine \\
\poemll    entwining a pomegranate,\fnote{So LXX; MT reads \fbib{in your blood}, misreading the Heb. \fbib{a pomegranate}} \\
\poeml planted by water, full of fruit, \\
\poemll    and full of branches \\
\poemlll       because it had been watered generously. \\
\poeml \v{11}Strong were its boughs, \\
\poemll    suitable for use in the scepter of a ruler. \\
\poeml It reached to the clouds, \\
\poemll    noticeable because of its height \\
\poemlll       and its abundant branches. \\
\poeml \v{12}Yet in anger it was uprooted \\
\poemll    and cast down to the earth. \\
\poeml An east wind desiccated its fruit; \\
\poemll    its strong branches broke off and withered, \\
\poemlll       and a fire consumed them. \\
\poeml \v{13}Now it is planted in the desert, \\
\poemll    in a dry and thirsty land! \\
\poeml \v{14}Fire had burned through its branches, \\
\poemll    consuming its shoots and fruits. \\
\poeml No strong branches remain in it, \\
\poemll    and there is no scepter to rule!'
\end{poetry}

``This is a lamentation, and it is to be used in mourning.''
\labelchapt{20}
\passage{A Prophecy against Israel's Elders}

\chapt{20}
\v{1}On the seventh year, on the tenth day\fnote{The Heb. lacks \fbib{day}} of the fifth month, men came from the elders of Israel to seek the \divine{Lord}. They sat down in front of me.

\v{2}``Son of Man,'' the \divine{Lord} told me, \v{3}``Tell the elders of Israel, `This is what the Lord \divine{God} asks, ``Did you come to inquire of me? As long as I live, I won't let myself be sought by you,'' declares the Lord \divine{God}.'

\v{4}``Will you judge them? Son of Man, will you indeed judge them? Teach them about the detestable things that their ancestors did. \v{5}Tell them, `This is what the Lord \divine{God} says, ``The day I chose Israel, when I made my commitment\fnote{Lit. \fbib{When I lifted my hand}, and so throughout.} to the descendants of Jacob's house, I revealed myself to them in the land of Egypt and I made my promise to them with the words, `I am the \divine{Lord} your God.' \v{6}That day I promised to bring them out of the land of Egypt to the land that I had explored for them---a land flowing with milk and honey. It's the most beautiful of all lands. \v{7}Then I told them, `Each of you are to abandon your detestable practices.\fnote{Lit. \fbib{practices before your eyes}} You are not to defile yourselves with Egypt's idols. I am the \divine{Lord} your God.'\,''\,'\,''
\passage{A Brief History of Israel's Rebellion}

\v{8}``But they rebelled against me and weren't willing to obey me. None of them abandoned their detestable practices\fnote{Lit. \fbib{practices before their eyes}} or their Egyptian idols. So I said, `I'll pour out my anger on them, extending my fury in the middle of the land of Egypt.' \v{9}I did this so my reputation\fnote{Lit. \fbib{name}} might not be tarnished among the nations where they were living, among whom I made myself known in their presence when I brought them out of the land of Egypt. \v{10}I brought them out of the land of Egypt to bring them to the wilderness \v{11}where I gave them my statutes and revealed my ordinances to them, which if a person\fnote{Lit. \fbib{man}} observes, he'll live by them. \v{12}Also, I instituted\fnote{Lit. \fbib{gave}} my Sabbath for them as a sign between me and them, so they would know that I am the \divine{Lord}, who has set them apart.''
\passage{Israel Rebels in the Wilderness}

\v{13}``But the house of Israel rebelled against me in the wilderness. They didn't live by\fnote{Lit. \fbib{walk in}} my statutes. They despised my ordinances, which if a person observes, he'll live by them. They greatly profaned my Sabbaths. So I said I would pour out my anger on them and bring them to an end in the wilderness. \v{14}I did this so my reputation wouldn't be tarnished among the nations in whose presence I had brought them out.

\v{15}``Moreover, I solemnly swore to them in the wilderness that I wouldn't bring them to the land that I had given them---a land flowing with milk and honey, the most beautiful of all lands---\v{16}because they kept on rejecting my ordinances. They didn't live life consistent with my statutes, they profaned my Sabbaths, and their hearts followed\fnote{Lit. \fbib{walked}} their idols. \v{17}Even then, I\fnote{Lit. \fbib{my eyes}} looked on them with compassion and didn't completely destroy them in the wilderness. \v{18}I told their children in the wilderness, `You are not to follow the statutes of your ancestors, observe their ordinances, or be defiled by their idols. \v{19}I am the \divine{Lord} your God. You are to follow my statutes, observe my ordinances, and keep them. \v{20}You are to make my Sabbaths holy, and you are to let them serve as a sign between you and me, so that you may know that I am the \divine{Lord} your God.'\,''
\passage{More of Israel's Rebellion}

\v{21}``But they rebelled against me. They didn't live according to my statutes, observe my ordinances, or practice them, by which a person will live. They also kept profaning my Sabbaths. So I said that I was going to pour out my anger on them and in my anger I'm going to bring about a complete end to them in the wilderness. \v{22}But I withdrew my decision\fnote{Lit. \fbib{hand}} so my reputation wouldn't be tarnished among the nations before whose eyes I brought them out.

\v{23}``Furthermore, I solemnly swore in the wilderness to disperse them among the nations and scatter them to other\fnote{The Heb. lacks \fbib{other}} lands \v{24}because they didn't observe my ordinances. Instead, they rejected my statutes, profaned my Sabbaths, and worshipped\fnote{Lit. \fbib{Their eyes went after}} their ancestors' idols. \v{25}So I gave them statutes that weren't good and ordinances by which they could not live. \v{26}I made them unclean because of their offerings, so they made all their firstborn\fnote{Lit. \fbib{their first to open the womb}} to pass through the fire, so that I could make them astonished. Then they'll know that I am the \divine{Lord}.''
\passage{The Blasphemy of Israel's Ancestors}

\v{27}``Therefore, Son of Man, you are to speak to the children of Israel and tell them, `This is what the Lord \divine{God} says: ``Your ancestors also blasphemed me in their treacherous behavior against me. \v{28}I brought them to the land that I had promised to give them. But whenever they saw any high hill and or any leafy tree, they slaughtered their sacrifices there and presented their offerings that provoked my anger. There they presented their pleasing aromas and poured out their drink offering. \v{29}So I asked them, `What is the high place to which you're going?' That's why the name of the place has been called Bamah\fnote{The Heb. name \fbib{Bamah} means \fbib{What is?}} to this day.''\,'

\v{30}``Therefore you are to say to Israel's house, `This is what the Lord \divine{God} says: ``Will you defile yourselves like your ancestors did by acting as a prostitute, consistent with their horrible deeds? \v{31}When you present your gifts and make your sons pass through the fire, you continue to defile yourselves with your idols to this day. Should I be inquired of by you, you house of Israel? As I live,'' declares the \divine{Lord}, ``I certainly won't be inquired of by you.'' \v{32}The thing that you're imagining\fnote{Lit. \fbib{that is coming upon your spirits}} is never going to happen, since you're thinking, ``We'll be like the nations, like the clans of other\fnote{The Heb. lacks \fbib{other}} lands who serve gods made from wood and stone.''\,'\,''
\passage{The Coming Discipline of Israel}

\v{33}``As I live,'' declares the Lord \divine{God}, ``with my powerful hand and outstretched arm, along with my wrath poured out, I'll reign as king over you. \v{34}I'll bring you out from the peoples and gather you from the lands where you were scattered. With a powerful hand, with an outstretched arm, and with wrath poured out, \v{35}I'll bring you into the wilderness of the nations. I'll judge you right there, face to face. \v{36}Just as I judged your ancestors in the wilderness in the land of Egypt, so I'll judge you,'' declares the \divine{Lord}. \v{37}``I'll cause you to pass under the rod until I will have brought you into the bond of the covenant. \v{38}I'll eliminate the rebels among you, along with those who are transgressing against me. I'll bring them out of the land where you've lived, but they won't be able to enter the land of Israel. Then you'll know that I am the \divine{Lord}.''
\passage{The Coming Regathering of Israel}

\v{39}And now, you house of Israel, this is what the Lord \divine{God} says, ``Go ahead and serve your idols, both now and later, but later you'll listen to me, and you won't profane my sacred name again by your offerings and idols. \v{40}For on my holy mountain, on Israel's high mountains,'' declares the Lord \divine{God}, ``the whole of Israel's house---all of it---will serve me there in the land. I'll accept them there. And there I'll demand your offerings, the first fruits of your portions of all your sacred things.

\v{41}``When I will have brought you from among the people and have gathered you from the lands where you were scattered, I'll accept you as a pleasing aroma. I'll reveal my holiness among you, and the entire world will see it. \v{42}Then you'll know that I, the \divine{Lord}, brought you to the land of Israel, to the land that I promised to give to your ancestors. \v{43}You'll remember all your practices and evil actions by which you've become defiled. You'll loathe yourselves\fnote{Lit. \fbib{your souls in your own sight}} because of all the evil things you've done. \v{44}Then you'll know that I am the \divine{Lord}, when I will have dealt with you for the benefit of my own reputation and not according to your evil attitudes or corrupt practices, you house of Israel,'' declares the Lord \divine{God}.
\passage{Coming Judgment on the South}

\v{45}\fnote{This v. is 21:1 1n MT}This message came to me from the \divine{Lord}: \v{46}``Son of Man, turn to the south and oppose it, talking toward the south. \v{47}Prophesy against the forest of the Negev,\fnote{I.e. southern regions of the Sinai peninsula; cf. Josh 10:40} `Listen to this message from the \divine{Lord}. This is what the Lord \divine{God} says: ``Look out! I'm about to ignite a fire and set it against you. It will devour every tree---whether green or dry---that lives in you. This powerful flame will not be extinguishable, and the entire surface from south to north will be scorched by it. \v{48}Then everyone\fnote{Lit. \fbib{Then all flesh}} will see that I, the \divine{Lord}, have kindled it, because it won't be extinguished.''\,'\,''
\passage{Ezekiel's Complaint to God}

\v{49}Then I said, ``O Lord \divine{God}! They're saying about me, `Isn't he one to propound parables?'\,''
\labelchapt{21}
\passage{A Prophecy against Jerusalem}

\chapt{21}
\v{1}\fnote{This v. is 21:6 in MT}This message came to me from the \divine{Lord}: \v{2}``Son of Man, look toward Jerusalem, preach\fnote{Lit. \fbib{Drop a word}} against its sanctuaries, and prophesy against Israel's land. \v{3}Declare to Israel, `This is what the \divine{Lord} says: ``Watch out! I'm against you! I'm going to unsheathe my sword to kill both the righteous and the wicked among you. \v{4}Since I'm going to kill both the righteous and the wicked among you, I'll be unsheathing my sword against everyone from south to north. \v{5}Then everyone will know that I am the \divine{Lord}, who unsheathed my sword, and who won't have to unsheathe it again.''\,'

\v{6}``And now, Son of Man, you are to start groaning until you're sick to your stomach.\fnote{Lit. \fbib{until your loins break}} You are to groan bitterly right in front of them.\fnote{Lit. \fbib{bitterly before their eyes}} \v{7}When they'll ask you, `Why are you groaning?' you are to say, `Because of the news that just arrived. Every heart will melt with fear, every hand will grow limp, every spirit will grow faint, and every knee will glisten with sweat.' Look! It has come and it will be fulfilled,'' declares the Lord \divine{God}.
\passage{God's Sword and Scepter}

\v{8}This message came to me from the \divine{Lord}: \v{9}``Son of Man, prophesy and say, `This is what the Lord \divine{God} says:

\begin{poetry}
\poeml `A sword! \\
\poemll    A sword is being sharpened. \\
\poemlll       It's also being polished. \\
\poeml \v{10}It's being sharpened for slaughter, \\
\poemll    and being polished to gleam like lightning.' \\
\poeml ``We shouldn't be rejoicing, should we, \\
\poemll    while my Son's scepter, the sword, \\
\poemlll       is despising\fnote{The verb \fbib{despising} requires the Heb. antecedent \fbib{the sword}, which The Heb. lacks} every tree?\fnote{I.e. every living human being in Israel} \\
\poeml \v{11}It's intended to be polished \\
\poemll    so it can be grasped in the hand. \\
\poeml The sword is sharpened. \\
\poemll    It's polished for placement \\
\poemlll       into the hand of the executioner.'' \\
\poeml \v{12}`Cry and wail, you Son of Man! \\
\poemll    It's headed against my people. \\
\poeml It's also against all the princes of Israel, \\
\poemll    who have been tossed to the sword, \\
\poemlll       along with my people. \\
\poeml So it's time to grieve like you mean it,\fnote{Lit. \fbib{So strike your thigh}} \\
\poeml \v{13}because testing is sure to come.
\end{poetry}

`In putting the sword to the test along with the scepter, it won't keep on rejecting, will it?' declares the Lord \divine{God}.''
\passage{A Double and Triple Judgment}

\begin{poetry}
\poeml \v{14}``Now, Son of Man, prophesy \\
\poemll    as you strike your hands together. \\
\poeml Let the sword that fatally wounds be doubled and tripled. \\
\poemll    That great, slaughtering sword closes in on them. \\
\poeml I've set in place a slaughtering sword \\
\poemll    at the entrance to all their gates, \\
\poeml \v{15}so that their hearts melt \\
\poemll    and the number of those who stumble increase. \\
\poeml I've set in place a slaughtering sword \\
\poemll    at the entrance to all their gates. \\
\poeml Oh, no! It's made like lightning. \\
\poemll    It's drawn to slaughter. \\
\poeml \v{16}Be sharp! \\
\poemll    Attack on the right, \\
\poeml or parry to your left, \\
\poemll    wherever you point your sword.\fnote{Lit. \fbib{face}} \\
\poeml \v{17}I will also clap my hands; \\
\poemll    then my anger will subside.\fnote{Lit. \fbib{rest}} \\
\poemlll       I, the \divine{Lord} have spoken it.''
\end{poetry}
\passage{Two Pathways to Invasion}

\v{18}This message came to me from the \divine{Lord}: \v{19}``Now, Son of Man, demarcate two pathways by which the sword of Babylon's king will arrive in the land. Both pathways will lead from a single land.

``Make a sign,\fnote{Lit. \fbib{hand}} carving it out and installing it at the junction on the way to the city. \v{20}Set it to point one way for bringing the sword against Rabbah, the descendants of Ammon, and the other way against Judah and fortified Jerusalem.

\v{21}``Meanwhile, Babylon's king is standing at the fork of the road,\fnote{Lit. \fbib{at the point of departure}} where he can head in either of two directions, and that's where he is practicing divination. Shaking his arrows, he's asking questions of his teraphim while he examines livers. \v{22}On his right hand he is divining against Jerusalem, preparing to set up battering rams, preparing\fnote{Lit. \fbib{rams, to open the mouth}} for the slaughter, getting ready to sound the alarm for battle,\fnote{Lit. \fbib{to shout}} setting the battering rams in place at the gates, building siege mounds, and erecting a siege wall. \v{23}In their view, it will seem to be a false prophecy, but because they swore allegiance, he'll make them remember their guilt as he takes them captive.''
\passage{Imminent Invasion}

\v{24}Therefore this is what the Lord \divine{God} says: ``Because you remembered your sins when your transgressions were uncovered, your sins are visibly evident in everything you've done. And since you've remembered them, you'll be taken captive.

\begin{poetry}
\poeml \v{25}``So now, you dishonored and wicked prince of Israel, \\
\poemll    whose day has come \\
\poemlll       in this time of final punishment,
\end{poetry}

\begin{poetry}
\poeml \v{26}This is what the Lord \divine{God} says: \\
\poeml `Remove your turban! \\
\poemll    Take off your crown! \\
\poeml Things aren't going to remain \\
\poemll    as they used to be. \\
\poeml What is lowly will be exalted, \\
\poemll    and what is lofty will be brought low. \\
\poeml \v{27}A ruin! A ruin! \\
\poemll    I'm bringing about ruin!' \\
\poeml But this also will not happen \\
\poemll    until he who has authority over it arrives, \\
\poemlll       because I'll give it to him.''
\end{poetry}
\passage{A Rebuke to Ammon}

\v{28}And now Son of Man, prophesy to the Ammonites that this is what the Lord \divine{God} says to the Ammonites about their approaching humiliation:

\begin{poetry}
\poeml ``A sword! A sword is being drawn for slaughter. \\
\poemll    It's polished to flash like lightning. \\
\poeml \v{29}When they see empty visions about you \\
\poemll    while they're divining lies for you, \\
\poeml to lay you on the necks of the wicked, \\
\poemll    who are fatally wounded, \\
\poeml whose days have come, \\
\poemll    their time for punishment. \\
\poeml \v{30}Return it to its scabbard.
\end{poetry}

\begin{poetry}
\poeml ``At the place where you were formed, \\
\poemll    in the land of your origin, \\
\poemlll       there is where I'll judge you. \\
\poeml \v{31}I'm going to pour out my indignation all over you. \\
\poemll    I'll blow my fierce wrath against you \\
\poeml and deliver you into the control of brutal men \\
\poemll    who are skilled at destruction. \\
\poeml \v{32}You'll be fuel\fnote{Lit. \fbib{food}} for the conflagration. \\
\poemll    Your blood will be spilled\fnote{The Heb. lacks \fbib{spilled}} throughout the land. \\
\poeml You won't be remembered anymore, \\
\poemll    now that I, the \divine{Lord}, have spoken.''
\end{poetry}
\labelchapt{22}
\passage{A Prophecy against Jerusalem}

\chapt{22}
\v{1}This message came to me from the \divine{Lord}: \v{2}``Now, Son of Man, will you truly judge that blood-stained city? Then make her aware of all of her detestable practices.

\v{3}``You are to say, `This is what the Lord \divine{God} says: ``The city keeps on shedding blood within her, hastening her time to be judged. She crafts idols that defile her.

\v{4}`````You're guilty because of the blood that you've shed. You were defiled by the idols that you've made. You've caused your judgment day to draw near and have even come to the end of\fnote{The Heb. lacks \fbib{the end of}} your life. Therefore, I've made you an object of derision among the nations and to other\fnote{The Heb. lacks \fbib{other}} lands. \v{5}Those who are both near and far away from you will scoff at you, because your reputation will be infamous and full of turmoil.

\v{6}`````Watch out! Each of the princes of Israel has misused his authority to shed blood. \v{7}They've treated mothers and fathers among you with contempt. They've oppressed the foreigner among you. They've maltreated the orphan and the widow among you.

\v{8}`````You have despised my sacred things and profaned my Sabbaths. \v{9}Slanderous men live among you, intent on shedding blood. They've eaten at the top of mountain shrines. They've crafted plans to do evil things among you. \v{10}They've revealed the nakedness of their father in your midst. They've humiliated those among you who were unclean due to their impurity. \v{11}One of you commits detestable practices with his neighbor's wife. Another sexually defiles his daughter-in-law. Another humiliates his sister, his own father's daughter. \v{12}They take bribes among you to shed blood. You've taken usury and exacted interest. You've gained control over your neighbor through extortion. And you've forgotten me,'' declares the Lord \divine{God}.

\v{13}``Watch out! I'm vehemently angry about\fnote{Lit. \fbib{I'm going to strike my hands}} the unjust gains that you've earned, and about the blood that has been shed among you. \v{14}Can your heart stand up to this? Can your hands remain strong when I deal with you? I, the \divine{Lord}, have spoken and will fulfill this. \v{15}I'm going to disperse you among the nations and scatter you to other lands. I'm going to put an end to your uncleanness. \v{16}When you've been defiled in the sight of the nations, then you'll know that I am the \divine{Lord}.''\,'\,''
\passage{God's Purging Fire}

\v{17}This message came to me from the \divine{Lord}: \v{18}``Son of Man, Israel has become like dross to me. All of them are like remnants of bronze, tin, iron, and lead in a furnace---the dross left over from smelting silver. \v{19}Therefore this is what the Lord \divine{God} says, `Because all of you have become dross, watch out! I'm going to gather all of you at the center of Jerusalem, \v{20}just like a smelter gathers all the silver, bronze, lead, and tin to the center of a furnace and injects fire in order to melt it, I'm going to gather you in my anger and rage, make you settle down---and then I'm going to melt you down. \v{21}Indeed, I'm going to gather you together and exhale the fire of my fury, and then you'll be melted from the inside out \v{22}like melting silver at the center of a furnace. When you've been melted from the center out, then you'll know that I am the \divine{Lord}. I'll pour out my anger on you.'\,''
\passage{God Rebukes Prophets and Priests}

\v{23}This message came to me from the \divine{Lord}: \v{24}``Son of Man, tell her,\fnote{I.e. Israel personified as a woman} `You're a land that hasn't been purified, one that hasn't been rained on in the day of indignation. \v{25}There's a conspiracy of prophets within her, and like a roaring lion tearing its prey, they've devoured people, and confiscated treasures, and taken precious things. They've added to the population of widows within her. \v{26}Her priests have violated my Law and profaned my sacred things. They didn't differentiate between what's sacred and what's common. They didn't instruct others to discern clean from unclean things. They refused to keep\fnote{Lit. \fbib{refused their eyes from}} my Sabbaths.

```I'm constantly being profaned among them. \v{27}Her princes within her are like wolves tearing their prey apart. They shed blood, destroying souls, and make unjust gain.

\v{28}```Her prophets whitewashed all of these things through false visions and lying divinations. They kept on saying, ``This is what the Lord \divine{God} says{\ldots}'', when the \divine{Lord} has not spoken. \v{29}The people of the land were vigorously oppressive and took possession of plunder by violence. They've afflicted the poor and the needy and unjustly treated the foreigner. \v{30}I sought for a man among them to build the wall and stand in the breach in my presence on behalf of the land so that it won't be destroyed, but I found no one, \v{31}so I poured my indignation over them. With my fierce anger, I've consumed them. I brought the consequences of\fnote{The Heb. lacks \fbib{the consequences of}} their behavior upon them,'\fnote{Lit. \fbib{upon their heads}} declares the Lord \divine{God}.''
\labelchapt{23}
\passage{Introducing Oholah and Oholibah}

\chapt{23}
\v{1}This message came to me from the \divine{Lord}: \v{2}``Son of Man, here are two sisters who are daughters from the same mother. \v{3}They committed sexual immorality in Egypt. They did this\fnote{Lit. \fbib{They committed sexual immorality}} in their youth. There, their breasts were caressed. Their virgin breasts were fondled. \v{4}The older one was named Oholah\fnote{The Heb. name \fbib{Oholah} means \fbib{she worships at a tent shrine}} and her sister was named Oholibah.\fnote{The Heb. name \fbib{Oholibah} means \fbib{she is a tent shrine}} They belonged to me and gave birth to sons and daughters. Now as to their real identities, Oholah refers to Samaria and Oholibah to Jerusalem.''
\passage{The Sins of Samaria}

\v{5}``Oholah committed sexual immorality while she belonged to me. She lusted for Assyria's warriors, \v{6}who were clothed in blue---including governors and commanders. All of them were desirable young men---horsemen mounted on horses. \v{7}She bestowed her sexual favors\fnote{Lit. \fbib{her sexual immorality}} on them---all of them, the best of the Assyrians---and with whomever she lusted for.

``She defiled herself with all their idols. \v{8}She never abandoned the immorality that she practiced in Egypt during her youth, where they laid down with her and fondled her virgin breasts, lavishing her with all kinds of favors. \v{9}Therefore, I turned her over to the control\fnote{Lit. \fbib{hands}} of her lovers, that is, into the control\fnote{Lit. \fbib{hands}} of the Assyrians for whom she lusted. \v{10}They stripped her naked, took away her sons and daughters, and executed her with a sword. She became an object of ridicule\fnote{Lit. \fbib{became a name}} among other nations\fnote{Lit. \fbib{among the women}} when they punished her.''
\passage{The Sins of Jerusalem}

\v{11}``Her sister Oholibah saw this, but she was more corrupt in her lust and sexual immorality than her sister had been in her own sexual immorality. \v{12}She lusted after the Assyrians---governors, commanders, warriors clothed in gorgeous attire, cavalry mounted on their horses---all of them desirable young men. \v{13}I saw that she was defiled, because the two of them both were on the same\fnote{Lit. \fbib{one}} path.

\v{14}``She became even more sexually immoral when she saw the images of the Chaldean men who had been carved in red on their walls. \v{15}Girded with waistbands around their loins, with flowing turbans on their heads, all of them looked like chariot officers, similar to the Babylonians from Chaldea, where they had been born.

\v{16}``She lusted after them when she saw them, so she sent messengers to summon them from Chaldea. \v{17}The Babylonians came to her love nest\fnote{Lit. \fbib{best}} and defiled her with their sexual immorality. As a result, she was defiled by them. Even so, she turned away from them in disgust. \v{18}She displayed her immorality publicly and stripped herself naked, so I turned away in disgust from her, just as I had turned away in disgust from her sister.

\v{19}``Nevertheless, she became even more sexually immoral, even reminiscing about when she was young, when she kept on practicing sexual immorality in the land of Egypt. \v{20}She lusted after her paramours, whose genitals are\fnote{Lit. \fbib{whose flesh is}} like those of donkeys, and whose emissions are like those of horses. \v{21}Think about the wickedness that you practiced when you were young, when the Egyptians fondled your breasts, the breasts of your youth.''
\passage{God's Rebuke to Jerusalem}

\v{22}``Therefore, Oholibah, this is what the Lord \divine{God} says: `Look! I'm about to stir up your lovers against you, the ones from whom you've turned away in disgust. I'm going to bring them against you from every direction---\v{23}the Babylonians, all the Chaldeans, Pekod, Shoa, Koa, and all of the Assyrians with them. They're all desirable young men, governors, commanders, chariot officers, and famous men, all of them mounted on horses.

\v{24}```They'll invade you with weapons, chariots, wagons, and a vast army. They'll set themselves in place to attack you from every direction with large shields, small shields, and helmets. I'll turn over judgment to them, and they'll punish you according to their own standards.\fnote{Lit. \fbib{punishment}} \v{25}I'll expend my jealousy on you so they'll deal with you in anger. They'll cut off your noses and your ears. Your survivors will die violently.\fnote{Lit. \fbib{die by the sword}} They'll take your sons and daughters away from you, but your survivors will be incinerated. \v{26}They'll strip off your clothes and confiscate your jewelry.\fnote{Lit. \fbib{your articles of beauty}} \v{27}That's how I'll put an end to your obscene conduct and sexual immorality that you kept on practicing since the day you left\fnote{Lit. \fbib{practicing from}} the land of Egypt so that you won't look in Egypt's direction or even remember it anymore.'

\v{28}``This is what the Lord \divine{God} says, `I'm about to turn you over to the control\fnote{Lit. \fbib{hands}} of those you hate, to the control of those from whom you turned away in disgust. \v{29}They'll deal with you with hatred. They'll take away your productivity, leaving you naked and defenseless, so that the nakedness of your sexual immorality will be uncovered---your licentious sexual immorality. \v{30}These things will happen to you because of your sexual immorality that was patterned after what the nations do. You've been defiled by their idols. \v{31}You took the path of your sister, so I'll place her cup in your hand.'

\v{32}``This is what the Lord \divine{God} says: `You'll drink from your sister's cup, which is both large and deep. You'll become a laughing stock and an object of derision, since the cup is so full! \v{33}You'll be filled with drunkenness and grief. The cup that belongs to your sister Samaria is filled with horror and devastation, \v{34}but you'll drink from it and drain it completely. As for the vessel, you'll break it to pieces and you'll tear at your breasts, for I've spoken,' declares the Lord \divine{God}.

\v{35}``Therefore this is what the Lord \divine{God} says: `Because you abandoned me and threw me behind your back, you will bear the consequences of your obscene conduct and sexual immorality.'\,''
\passage{What Israel and Samaria Did}

\v{36}Then the \divine{Lord} spoke to me. ``Son of Man, speak out in judgment of both Oholah and Oholibah. Make their detestable practices widely known, \v{37}because they've committed adultery, and blood covers their hands. They've also committed adultery with their idols, making their sons born to me to pass through the fire\fnote{The Heb. lacks \fbib{fire}} as an offering to\fnote{Lit. \fbib{as food for}} them.

\v{38}``They've also done this to me: They defiled my sanctuary and profaned my Sabbaths, all at the same time!\fnote{The Heb. has \fbib{day}} \v{39}When they killed their sons as offerings to\fnote{Lit. \fbib{sons for}} their idols, they brought them to my sanctuary and defiled it.\fnote{I.e. with their corpses} Look what they've done with my Temple!

\v{40}``In addition, they sent messengers for men to come from afar. When they arrived, you bathed yourself for them, painted your eyes, adorned yourself with jewelry, \v{41}then sat down on an elegant bed. A table was arranged in front of it, on which you set out my incense and oil. \v{42}The sound of a carefree multitude accompanied her. Men from a multitude of peoples were coming---including Sabeans\fnote{Or \fbib{drunkards}} from the wilderness, adorned\fnote{Lit. \fbib{they put}} with bracelets on their hands and beautiful crowns on their heads.

\v{43}``After she had worn herself out by her adulterous behavior, I asked her, `Will they continue with their sexual immorality and with their prostitution?' \v{44}They've gone to her, like men do, to have sex with a prostitute. They\fnote{Lit. \fbib{He}} had sex with Oholah and Oholibah, those licentious women. \v{45}Righteous men will judge them with punishments fit for adulterers and for those who shed blood, because they're adulterers with blood on their hands.''
\passage{The Coming Invasion}

\v{46}This is what the Lord \divine{God} says: ``Bring an army\fnote{Lit. \fbib{company}} against them and deliver them over to terror and plunder. \v{47}Then the army will stone them with stones and cut them to pieces with their swords. They'll kill their sons and daughters and incinerate their houses. \v{48}I'll cause obscene conduct to stop throughout the land, because all the women will be admonished not to practice their obscene conduct. \v{49}You'll receive the consequences for your obscene conduct and bear the punishment for your sins of idolatry. Then you'll know that I am the Lord \divine{God}.''
\labelchapt{24}
\passage{God Brews His Judgment}

\chapt{24}
\v{1}In the ninth year, in the tenth month, and on the tenth day of the month, this message came to me from the \divine{Lord}:

\v{2}``Son of Man, write down the name of this day, this very day. The king of Babylon has laid siege to Jerusalem on this very day. \v{3}So compose a parable for the rebellious house. Tell them, `This is what the Lord God says:

\begin{poetry}
\poeml ``Prepare your pot for boiling! \\
\poemll    Set it in place. \\
\poemlll       Fill it up with water, too. \\
\poeml \v{4}Gather together the best pieces of meat on it--- \\
\poemll    including the thighs and the shoulders--- \\
\poemlll       and fill it with the choicest bones. \\
\poeml \v{5}Take the best bones from the flock, \\
\poemll    pile wood\fnote{The Heb. lacks \fbib{wood}} under the pot\fnote{Lit. \fbib{under it}} for the bones, \\
\poeml bring it to a boil, \\
\poemll    and then cook the bones in it.''\,'\,''
\end{poetry}
\passage{Woe to Jerusalem}

\v{6}``This is what the Lord \divine{God} says:

\begin{poetry}
\poeml `How terrible it is for that blood-filled city, \\
\poemll    to the pot whose rust remains in it, \\
\poemlll       whose rust won't come off. \\
\poeml Empty it one piece at a time. \\
\poemll    Don't let a lot fall on it. \\
\poeml \v{7}Her blood was in it. \\
\poemll    She poured it out onto bare rock. \\
\poeml She didn't pour it out on the ground, \\
\poemll    intending to cover it with dirt. \\
\poeml \v{8}In order to stir up my anger \\
\poemll    and in order to take vengeance, \\
\poeml I set the blood on a bare rock \\
\poemll    so that it cannot be covered.'
\end{poetry}

\v{9}``Therefore this is what the Lord \divine{God} says:

\begin{poetry}
\poeml `How terrible it is for that blood-filled city--- \\
\poemll    I'm also going to add to my\fnote{The Heb. lacks \fbib{my}} pile of kindling. \\
\poeml \v{10}Pile up the wood! \\
\poemll    Make the fire burn hot. \\
\poeml Boil the meat! \\
\poemll    Mix the seasonings. \\
\poeml Burn those bones to a crisp! \\
\poeml \v{11}Make the pot stand empty on the coals \\
\poemll    until its bronze glows red,\fnote{Lit. \fbib{its copper burns hot}} \\
\poeml its rust can be scoured off,\fnote{Or \fbib{is poured out}} \\
\poemll    and its dross completely removed. \\
\poeml \v{12}The pot\fnote{Lit. \fbib{She}} wearies me,\fnote{The Heb. lacks \fbib{me}} \\
\poemll    but its thick\fnote{Lit. \fbib{great}} rust won't come off, \\
\poemlll       even with fire. \\
\poeml \v{13}There is wickedness in your obscene conduct. \\
\poemll    Even though I've cleansed you, \\
\poemlll       you uncleanness cannot be washed away. \\
\poeml You cannot be cleansed again \\
\poemll    until my rage against you has subsided.'
\end{poetry}

\v{14}```I, the \divine{Lord} have spoken. It will happen, because I'm going to do it. I won't hold back, have compassion, or change my mind.\fnote{Or \fbib{repent}} They'll judge you according to your ways and deeds,' declares the Lord \divine{God}.''
\passage{The Death of Ezekiel's Wife}

\v{15}This message came to me from the \divine{Lord}: \v{16}``Son of Man, pay attention! I'm about to take away your most precious treasure\fnote{Lit. \fbib{away the desire of your eyes}} with a single, fatal stroke, but you are not to mourn, weep, nor even let tears well up in your eyes.\fnote{Or \fbib{tears come}} \v{17}You are to weep in silence, but you are not to participate in mourning rituals.\fnote{Lit. \fbib{to mourn the dead}} You are to keep your turban on your head and your sandals on your feet. You are not to cover your mouth\fnote{Lit. \fbib{moustache}} or eat what your comforters bring to you.''\fnote{Lit. \fbib{eat the food of men}}

\v{18}So I spoke to the people in the morning, and my wife died that evening. The next\fnote{The Heb. lacks \fbib{next}} morning, I did as I had been commanded.

\v{19}Then the people told me, ``Are you going to explain what these things that you're doing should mean to us?''

\v{20}So I responded, ``This message came to me from the \divine{Lord}: \v{21}`Tell the house of Israel that this is what the Lord \divine{God} says: ``Look! I'm about to profane my sanctuary, the source of your proud strength, the desire of your eyes, and the object of your affection. Your sons and daughters, whom you've left behind, will die by the sword. \v{22}That's why you will soon be doing what I've just done. You are not to cover your mouth\fnote{Lit. \fbib{moustache}} or eat what your comforters bring to you.\fnote{Lit. \fbib{eat the food of men}} \v{23}Your turbans will be on your heads and your sandals will be on your feet. You won't mourn or weep. Instead, you'll waste away in your sins. Every one of you will groan to his relative. \v{24}That's how Ezekiel will be an example for you. You'll be doing exactly what he has done. When it happens, then you'll know that I am the Lord \divine{God}.''\,'

\v{25}``And now, Son of Man, on the day that I take their strength, joy, and glory from them, those whom they love to watch, the focus of their affection---their sons and daughters--- \v{26}at that time,\fnote{Lit. \fbib{day}} a fugitive will come to you and will bring you the news.\fnote{Lit. \fbib{will make ears hear}} \v{27}Your mouth will freely speak to the fugitive. You won't be silent any longer. You'll be a sign to them. Then they'll know that I am the \divine{Lord}.''
\labelchapt{25}
\passage{A Message Condemning Ammon}

\chapt{25}
\v{1}This message came to me from the \divine{Lord}: \v{2}``Son of Man, turn your attention\fnote{Lit. \fbib{face}} to the descendants of Ammon and rebuke\fnote{Lit. \fbib{and prophesy against}} them. \v{3}Tell the Ammonites: `Listen to a message from\fnote{Lit. \fbib{to the word of}} the Lord \divine{God}! This is what the Lord \divine{God} says: ``Because you have said, `Aha!'\fnote{I.e. an expression of delight upon hearing that disaster has befallen another} about my sanctuary when it was desecrated, about the land of Israel when it became desolate, and about the households of Judah when they went into exile, \v{4}therefore you'd better look out! I'm going to turn you over to men\fnote{Lit. \fbib{to children}} from the East, who will dominate you. You will become their property. They will set up military encampments and permanent places\fnote{Lit. \fbib{tents}} in which to live among you, and then they'll eat your fruit and drink your milk. \v{5}I will turn Rabbah\fnote{I.e. the capital city of the Ammonites, located east of the Jordan River} into a pasture for camels, and Ammon will become a resting place for flocks of sheep. That's how they'll learn that I am the \divine{Lord}.''\,'\,''
\passage{Why God Condemned Ammon}

\v{6}``This is what the Lord \divine{God} says: `Because you've applauded, stamped your feet, and rejoiced with all sorts of malice in your heart\fnote{I.e. expressions of delight upon hearing that disaster has befallen another} against the land of Israel, \v{7}therefore you'd better\fnote{The Heb. lacks \fbib{you'd better}} watch out! I'm raising a clenched fist\fnote{Lit. \fbib{I'm stretching out my hand}} in your direction! I'm about to feed you to the surrounding\fnote{The Heb. lacks \fbib{surrounding}} nations as war plunder. I'm going to eliminate you as a nation and kill off those of you who survive to live in other\fnote{The Heb. lacks \fbib{other}} countries. I'm going to destroy you, and that's how you'll learn that I am the \divine{Lord}.'\,''
\passage{A Message Rebuking Moab and Seir}

\v{8}``This is what the Lord \divine{God} says: `Because Moab and Seir are claiming, ``Judah's citizens are\fnote{Lit. \fbib{Judah's household is}} just like every other\fnote{The Heb. lacks \fbib{other}} nation,'' \v{9}therefore you'd better watch out! I'm going to tear open Moab's flanks, starting with its frontier cities---the very glory of the nation!---including Beth-jeshimoth,\fnote{This city was originally intended to be owned by the tribe of Reuben. Cf. Num 33:49; Josh 12:3; 13:20. The name means \fbib{House of Destruction}.} Baal-meon,\fnote{This city was originally intended to be owned by the tribe of Reuben. Cf. Num 33:38; 1Chron 5:8. The name means \fbib{Lord of the Habitation}.} and Kiriathaim.\fnote{This city was originally intended to be owned by the tribe of Reuben. Cf. Num 32:37; Josh 13:19. The name means \fbib{Twin Cities}.} \v{10}I'm going to turn these cities\fnote{Lit. \fbib{give it}} over to men\fnote{Lit. \fbib{to children}} from the East, who will dominate you. You will become their property. As a result, Ammon will be forgotten as a nation. \v{11}I'm also going to punish Moab, and that's how they'll learn that I am the \divine{Lord}.'\,''
\passage{The Coming Destruction of Edom}

\v{12}``This is what the Lord \divine{God} says: `Because Edom has made it their practice to seek extraordinary vengeance against Judah's citizens,\fnote{Lit. \fbib{household}} and by doing so has incurred extraordinary guilt by taking revenge against them,' \v{13}therefore this is what the Lord \divine{God} says: `I'm going to raise my clenched fist\fnote{Lit. \fbib{to stretch out my hand}} in Edom's direction and eliminate every single human being and animal from Edom! I'm going to turn everything into a wasteland, starting with Teman, and Dedan will fall by violence!\fnote{Lit. \fbib{by the sword}} \v{14}I'm going to inundate Edom with\fnote{Lit. \fbib{give Edom}} my retribution, using my people Israel to carry it out! They'll deliver my anger, acting as an agent of my fury. Edom will come to know my vengeance,' declares the Lord \divine{God}.'\,''
\passage{A Message Condemning Philistia}

\v{15}``This is what the Lord \divine{God} says: `Because Philistia has made it their practice to carry out retribution, accompanied by extraordinary malice in their personal vendettas---vendettas that spring from their everlasting hostility--- \v{16}this is what the Lord \divine{God} says: ``Look out! I'm raising my clenched fist\fnote{Lit. \fbib{to stretch out my hand}} in Philistia's direction. I'm going to execute\fnote{Lit. \fbib{cut off}; the Heb. verb is the root upon which the Heb. term \fbib{Cherethites} is based.} the Cherethites\fnote{Lit. \fbib{executioners}; i.e. Philistines who originally served as bodyguards for King David, but by the date of this writing had become rogue mercenaries who harassed the territory of ancient Israel.} and destroy what's left of the coastline of the Mediterranean\fnote{The Heb. lacks \fbib{Mediterranean}} Sea. \v{17}I'll take vengeance on them, punishing them severely in my anger. They'll know that I am the \divine{Lord} when I take my vengeance on them.''\,'\,''
\labelchapt{26}
\passage{A Message Condemning Tyre}

\chapt{26}
\v{1}During the eleventh year, on the first day of the month of our captivity\fnote{The Heb. lacks \fbib{of our captivity}}, this message came to me from the \divine{Lord}: \v{2}``Son of Man, because Tyre has been saying about Jerusalem,

\begin{poetry}
\poeml `The international gateway is broken down! \\
\poemll    It's wide open to me! \\
\poeml I will be replenished, \\
\poemll    now that it lies in ruins!'
\end{poetry}

\v{3}``Therefore this is what the Lord \divine{God} says: `Watch out! I'm coming to get\fnote{Lit. \fbib{coming against}} you, Tyre! I'm about to bring many nations to attack you. They'll come in wave after wave, like the advancing tide,\fnote{Lit. \fbib{come like the sea brings waves}} \v{4}and will destroy the city walls of Tyre. After they break down her fortified towers, I'll scrape away the city's debris, right down to the bare bedrock, \v{5}and it will become a place where nets will be spread out right in the middle of the Mediterranean\fnote{The Heb. lacks \fbib{Mediterranean}} Sea. Because I have declared this to happen,' declares the Lord \divine{God}, `Tyre will be treated as the spoils of war by the invading\fnote{The Heb. lacks \fbib{invading}} nations. \v{6}Furthermore, her citizens\fnote{Lit. \fbib{daughters}} who live on the mainland will be executed with swords. That's how they'll learn that I am the \divine{Lord}.'\,''
\passage{Nebuchadnezzar's Invasion}

\v{7}``This is what the Lord \divine{God} says: `Watch out! I'm about to bring from the north King Nebuchadnezzar of Babylon, that king of kings. He'll come with horses, chariots, cavalry, and a vast army. \v{8}He'll execute your citizens who live on the mainland with swords. He'll build siege engines to attack you. Then he'll construct siege ramps against you and build huge shields to protect themselves\fnote{The Heb. lacks \fbib{to protect themselves}} against you.

\v{9}```He'll direct the blows of his battering rams against your walls and will breach your fortified towers with axes.\fnote{Or \fbib{swords}} \v{10}There will be so many horses that the dust raised by them will cover you completely. The walls of your city will tremble from the noise of Nebuchadnezzar's\fnote{Lit. \fbib{their}} cavalry, wagons, and chariots when they enter through the gates of your city, as men enter a city that has been breached.

\v{11}```Their horses will trample all the public places as he executes your inhabitants with swords. The most fortified of your pillars will be torn to the ground. \v{12}They will plunder your riches and loot your businesses. They'll tear down your walls and demolish your luxurious homes. They'll grab the stones, wood, and rubble from the destruction and dump it all into the Mediterranean\fnote{The Heb. lacks \fbib{Mediterranean}} Sea.

\v{13}```I'll silence the noise of your songs and the music of your harps won't be heard anymore. \v{14}I'll turn you into bare rock, and your city will become a place to spread nets. You will never be built again, because I the \divine{Lord} have decreed this,' declares the Lord \divine{God}.''
\passage{Terror at Tyre's Destruction}

\v{15}``This is what the Lord \divine{God} says to Tyre: `When your wounded citizens groan while the slaughter takes place among you, the people who live in the coastlands will tremble in terror as they hear about your fall, will they not? \v{16}That's when all the kings of the seafaring nations will abandon their thrones, strip off their fancy clothes, and collapse trembling on the ground. They'll be so frightened as they observe what has happened to you that they'll be unable to stop trembling. They will be utterly appalled at you! \v{17}They'll sing this mourning song for you:

\begin{poetry}
\poeml ``How lost you are, \\
\poemll    you inhabited city, \\
\poemlll       that was built in the middle of the sea! \\
\poeml How famous you were! \\
\poemll    How strong on the sea! \\
\poeml She and her inhabitants \\
\poemll    inflicted\fnote{Lit. \fbib{inhabitants who inflicted}} terror to everyone \\
\poemlll       who lived within her.'' \\
\poeml \v{18}`Now the coastland inhabitants \\
\poemll    will tremble on the day that you fall. \\
\poeml The coastland inhabitants, \\
\poemll    who make their living from\fnote{Lit. \fbib{who are by}} the sea, \\
\poemlll       will be terrified when you pass away!'
\end{poetry}

\v{19}``This is what the Lord \divine{God} says: `When I turn your city into a ghost town, when I flood you with deep water that covers you completely, \v{20}I'll make sure that you go straight to the Pit,\fnote{I.e. the place of punishment in the afterlife} into the lowest part of the earth, where you'll be with people who lived in ancient times. You'll keep company there with the dead, who have gone into the Pit.\fnote{I.e. the place of punishment in the afterlife} As a result, your city\fnote{Lit. \fbib{result, you}} won't be inhabited. Meanwhile, I will display my glory in the land of the living. \v{21}I'm going to send terrifying calamity in your direction, and you won't exist any longer. You might be sought after, but you'll never be found again,' declares the Lord \divine{God}.''
\labelchapt{27}
\passage{A Message Condemning Tyre}

\chapt{27}
\v{1}This message came to me from the \divine{Lord}: \v{2}``Son of Man, compose a mourning song for Tyre. \v{3}Tell Tyre, who lives at the gateway to the Mediterranean\fnote{The Heb. lacks \fbib{Mediterranean}} Sea, who serves as the international merchant to many coastal districts: `This is what the Lord \divine{God} says:

\begin{poetry}
\poeml ``Tyre, you've been claiming, \\
\poemll    ``I am beauty perfected.' \\
\poeml \v{4}You've set your national boundary in international waters. \\
\poemll    Your builders made you downright beautiful!''\,'\,''\fnote{Or \fbib{you perfectly beautiful}}
\passage{Tyre's Luxurious Sailing Vessels}
\poeml \v{5}`They brought in a ship \\
\poemll    made with pine planking from Senir, \\
\poeml configured with a mast carved from a cedar from Lebanon, \\
\poeml \v{6}equipped with oars \\
\poemll    made from oaks from Bashan, \\
\poeml with ivory-inlaid cypress wood\fnote{Lit. \fbib{daughter of Assyria}} decking \\
\poemll    imported from the coastlands of Cypress, \\
\poeml \v{7}with sails made with embroidered Egyptian linen, \\
\poemll    festooned with blue banners, \\
\poeml and with your sun shades made \\
\poemll    with purple cloth from Cypress. \\
\poeml \v{8}Your sailors were conscripted \\
\poemll    from Sidon and Arvad, \\
\poeml and your officers served aboard \\
\poemll    as pilots. \\
\poeml \v{9}The wise men and elders from Gebal accompanied you, \\
\poemll    serving as ship's carpenters. \\
\poeml All the maritime navies and their seaman also accompanied you \\
\poemll    to assist you in doing business internationally.''
\passage{Tyre's International Makeup}
\poeml \v{10}``Soldiers from Persia,\fnote{I.e. modern Iran} Lud,\fnote{I.e., fourth son of Shem, progenitor of the Lydians. The Heb. name \fbib{Lud} means \fbib{strife.}} and Libya,\fnote{Lit. \fbib{Put;} i.e. the third son of Ham (cf. Gen 10:6); the Heb. name means \fbib{bow}} \\
\poemll    served in your army. \\
\poemlll       They were your mighty soldiers. \\
\poeml Their helmets and shields adorned your barracks walls, \\
\poemll    and they won battle decorations for you. \\
\poeml \v{11}Mercenaries from Arvad and Helech \\
\poemll    stood guard duty on your walls, \\
\poemlll       while brave men manned your towers. \\
\poeml They hung their shields all around your walls--- \\
\poemll    just the right touch to perfect your interior decorating!''\fnote{Lit. \fbib{your beauty}}
\end{poetry}
\passage{Tyre's Trading Partners}

\v{12}`Tarshish was your business partner because of your phenomenal wealth. They traded silver, iron, tin, and lead for your merchandise. \v{13}Greece, Tubal,\fnote{I.e. a son of Noah's son Japheth; this people resided in what is now modern eastern Turkey} and Meshech\fnote{I.e. a son of Noah's son Japheth; this people resided in what is now modern Armenia} bartered with you, exchanging slaves and bronze vessels for your wares. \v{14}Beth-togarmah traded horses, war horses, and mules in exchange for what you had to sell. \v{15}Men from the low country south of Edom\fnote{Lit. \fbib{The descendants of Dedan}; i.e., a son of Cush (cf. Gen 10:7) or a grandson of Abraham through Keturah (Gen 25:3)} and many of the coastlands were your markets for ivory tusks and ebony that they brought to trade with you.

\v{16}``Aram was one of your customers because you had so much merchandise. They paid by trading turquoise, purple yarn, embroidered goods, Egyptian linen,\fnote{Lit. \fbib{byssus}; i.e. a white cotton cloth} coral, and rubies. \v{17}The territories of Judah and Israel were your clients, too. They traded wheat from their distribution centers,\fnote{Lit. \fbib{from Minnith}; perhaps a site in Ammon, east of the Jordan River} baked goods, honey, oil, and ointments for your merchandise.

\v{18}``Because you have so much to sell and are so rich, Damascus has been your trading partner, exchanging wine from Helbon, unbleached wool, \v{19}and casks of wine from Izal\fnote{Or \fbib{Uzal}, i.e., a city located in a wine growing region between Haran and the Tigris River} for your wrought iron, cassia wood, and aromatic reeds.

\v{20}``Dedan traded with you, exchanging riding blankets. \v{21}Arabia, including all the princes of Kedar, came to you, shopping for lambs, rams, and goats. \v{22}Traders from Sheba and Raamah paid for the best of what you had to offer with all types of spices, precious stones, and gold. \v{23}Haran, Canneh, Eden,\fnote{These cities are thought to have been located in ancient Mesopotamia} merchants from Sheba, Asshur, and Chilmad did business with you, \v{24}trading garments made into the finest blue and embroidered mantels, and also multi-colored carpets, ropes, and other merchandise. \v{25}Ocean-going fleets\fnote{Lit. \fbib{Ships of Tarshish}; i.e., a class of vessel capable of travelling the oceans} carried your merchandise.''
\passage{Tyre's Coming Storm}

\begin{poetry}
\poeml ``How filled you were! \\
\poemll    How glorious you were, \\
\poemlll       at home in the heart of the sea! \\
\poeml \v{26}But your rowers have brought you \\
\poemll    into dangerous waters. \\
\poeml The east wind has broken you \\
\poemll    in the heart of the ocean! \\
\poeml \v{27}Your wealth, your products, your merchandise \\
\poemll    your sailors, your pilots, \\
\poeml your tailors, your salesmen, \\
\poemll    all your mercenaries with you--- \\
\poeml your entire company with you--- \\
\poemll    will fall into the midst of the sea \\
\poemlll       on the day when you're overthrown! \\
\poeml \v{28}When your ships' captains cry out, \\
\poemll    the pasturelands along the coast will cry out! \\
\poeml \v{29}Everyone who handles an oar will abandon ship, \\
\poemll    they'll head straight for dry land, \\
\poeml \v{30}and they will cry so loud \\
\poemll    you won't be able to make yourself heard! \\
\poemlll       How bitterly they'll cry! \\
\poeml They'll throw dust on their heads \\
\poemll    and wallow in ashes. \\
\poeml \v{31}They'll shave their heads bald because of you. \\
\poemll    They'll dress themselves in sackcloth \\
\poeml and weep for you with deep bitterness of heart, \\
\poemll    with the most pitiful of mourning. \\
\poeml \v{32}In the depth of their despair \\
\poemll    they'll compose a lament for you. \\
\poeml This is what they'll say:
\end{poetry}

\begin{poetry}
\poeml `Who is like Tyre? \\
\poemll    Who is so silent in the midst of the sea?' \\
\poeml \v{33}Your merchandise went out over the oceans \\
\poemll    to satisfy many nations; \\
\poeml with the abundance of your wealth \\
\poemll    you enriched the kings of the earth.
\end{poetry}

\begin{poetry}
\poeml \v{34}``But now it's your time to be wrecked \\
\poemll    at the bottom of the sea! \\
\poeml Your products and your workers have sunk, \\
\poemll    and so have you! \\
\poeml \v{35}Everyone who lives by the sea \\
\poemll    is appalled at your destruction. \\
\poeml Their leaders are terrified--- \\
\poemll    their faces reflect their fears! \\
\poeml \v{36}Traders circulate among the people, hissing at you. \\
\poemll    What a horror you've become! \\
\poeml Now you will cease to exist \\
\poemll    forever and ever!''
\end{poetry}
\labelchapt{28}
\passage{Prophecy against Tyre}

\chapt{28}
\v{1}This message came to me from the \divine{Lord}: \v{2}``Son of Man, tell Tyre's Commander-in-Chief,\fnote{Lit. \fbib{Nagid}; i.e. a senior officer entrusted with dual roles of operational oversight and administrative authority} `This is what the Lord \divine{God} says:

\begin{poetry}
\poeml ``Because your heart is arrogant,\fnote{Lit. \fbib{tall}} \\
\poemll    and because you keep saying, \\
\poeml `I have taken my seat, \\
\poemll    I am a god, \\
\poemlll       seated in God's seat right in the middle\fnote{Lit. \fbib{heart}} of the sea,'\fnote{I.e., an allusion to Tyre's location on an island off the coast of Lebanon} \\
\poeml and because you're a man, \\
\poemll    and not a god, \\
\poeml even though you pretend \\
\poemll    that you have a god-like heart{\ldots} \\
\poeml \v{3}Look! You're wiser than Daniel, aren't you? \\
\poemll    No secret is too mysterious for you! \\
\poeml \v{4}Your wisdom and understanding \\
\poemll    brought you phenomenal wealth. \\
\poeml You've brought gold and silver \\
\poemll    into your treasuries. \\
\poeml \v{5}By your great wisdom, \\
\poemll    by your skills in trading \\
\poeml you have amassed wealth for yourself \\
\poemll    and your heart has become arrogant \\
\poemlll       because of your wealth.'' \\
\poeml \v{6}Therefore this is what the Lord \divine{God} says: \\
\poeml ``Because you've made your heart \\
\poemll    like that of God \\
\poeml \v{7}Therefore, look! \\
\poemll    I'm bringing foreigners in your direction, \\
\poemlll       the most terrifying of nations! \\
\poeml They will direct their violence\fnote{Lit. \fbib{swords}} \\
\poemll    against the grandeur \\
\poemlll       that you've created by your wisdom. \\
\poeml \v{8}They'll send you down to the Pit,\fnote{I.e. the place of punishment in the afterlife} \\
\poemll    and you'll die defiled in the depths of the sea. \\
\poeml \v{9}Is that when you'll say, `I'm God' \\
\poemll    to the face of those who will be killing you? \\
\poeml After all, you're a man, \\
\poemll    and have never been a god, \\
\poeml especially when you're under the control\fnote{Lit. \fbib{you're in the hand}} of those \\
\poemll    who will defile you! \\
\poeml \v{10}You will die a death fit for the uncircumcised \\
\poemll    at the hand of foreigners.'' \\
\poeml `for I have said it will be so,' \\
\poemll    declares the \divine{Lord}.''
\end{poetry}
\passage{A Rebuke for Tyre's King}

\v{11}Another message came to me from the \divine{Lord}, and this is what it said: \v{12}``Son of Man, start singing this lamentation for the king of Tyre. Tell him, `This is what the Lord \divine{God} says:

\begin{poetry}
\poeml ``You served as my\fnote{The Heb. lacks \fbib{my}} model, \\
\poemll    my example of complete wisdom \\
\poemlll       and perfect beauty. \\
\poeml \v{13}You used to be in Eden--- \\
\poemll    God's paradise! \\
\poeml You wore precious stones for clothing: \\
\poeml ruby,\fnote{Lit. \fbib{red}} topaz, diamond,\fnote{Or \fbib{emerald}} \\
\poemll    beryl,\fnote{Lit. \fbib{Tarshish}; i.e. a yellow stone} onyx, jasper, \\
\poemlll       sapphire,\fnote{Or \fbib{lapis lazuli}; a bright blue stone} turquoise, and carbuncle. \\
\poeml Your settings were crafted in gold, \\
\poemll    along with your engravings. \\
\poeml On the day of your creation \\
\poemll    they had been prepared! \\
\poeml \v{14}``You were the anointed cherub; \\
\poemll    having been set in place \\
\poeml on the holy mountain of God, \\
\poemll    you walked in the midst of fiery stones. \\
\poeml \v{15}You were blameless in your behavior\fnote{Lit. \fbib{ways}} \\
\poemll    from the day you were created \\
\poemlll       until wickedness was discovered in you. \\
\poeml \v{16}Since your vast business dealings \\
\poemll    filled you with violent intent\fnote{Lit. \fbib{with violence}} \\
\poeml from top to bottom,\fnote{Lit. \fbib{in your midst}} \\
\poemll    you sinned, \\
\poeml so I cast you away as defiled \\
\poemll    from the mountain of God. \\
\poeml I destroyed you, \\
\poemll    you guardian cherub, \\
\poemlll       from the midst of the fiery stones. \\
\poeml \v{17}Your heart grew arrogant because of your beauty; \\
\poemll    you annihilated your own wisdom \\
\poemlll       because of your splendor. \\
\poeml Then I threw you to the ground \\
\poemll    in the presence of kings, \\
\poemlll       giving them a good look at you! \\
\poeml \v{18}By all of your iniquity \\
\poemll    and unrighteous businesses \\
\poeml you defiled your sanctuaries, \\
\poemll    so I'm going to bring out fire from within you \\
\poeml and burn you to ashes on the earth \\
\poemll    before the whole watching world! \\
\poeml \v{19}Everyone who knows you \\
\poemll    throughout all the nations \\
\poeml will be appalled at your calamity \\
\poemll    and you will no longer exist forever.''\,'\,''
\end{poetry}
\passage{The Judgment of Sidon}

\v{20}Another message came to me from the \divine{Lord}, who had this to say:

\v{21}``Son of Man, turn your attention\fnote{Lit. \fbib{face}} to Sidon and prophesy against her.\fnote{I.e., the city personified as a woman} \v{22}Tell her:

\begin{poetry}
\poeml `Pay attention to me, Sidon! \\
\poemll    I'm against you, \\
\poemlll       and I'm going to glorify myself right in your midst.' \\
\poeml They'll learn that I am the \divine{Lord} \\
\poemll    when I carry out these punishments \\
\poemlll       and manifest my holiness in her midst. \\
\poeml \v{23}I'm going to send disease into that city\fnote{Lit. \fbib{disease into her}} \\
\poemll    and blood into her streets. \\
\poeml People will drop dead in her midst \\
\poemll    from the violence done to\fnote{Lit. \fbib{the sword brought against}} her from every side. \\
\poemlll       Then they'll learn that I am the Lord \divine{God}.''
\end{poetry}
\passage{The Future Regathering of Israel}

\v{24}``The house of Israel will never again suffer from painful briers and sharp thorn bushes that surround them on every side, and they will learn that I am the \divine{Lord}. \v{25}This is what the Lord \divine{God} says:

`When I gather the house of Israel from the nations to which I've scattered them, I will show them my holiness before the watching world, and they will live on the land that I gave to my servant Jacob. \v{26}They will live in safety in the land,\fnote{Lit. \fbib{in her}} building houses and planting vineyards. They'll live in safety while I judge everyone who maligns them among those who surround them. At that time they'll learn that I am the \divine{Lord} their God.'\,''
\labelchapt{29}
\passage{Prophecy against Egypt}

\chapt{29}
\v{1}In the tenth year, on the twelfth day of the tenth month, a message came to me from the \divine{Lord}, who had this to say:

\v{2}``Son of Man, turn your attention to Pharaoh, king of Egypt, and prophesy against him and the entire nation of Egypt. \v{3}Tell him that this is what the Lord \divine{God} says:

\begin{poetry}
\poeml `Watch out! I'm coming to get\fnote{Lit. \fbib{I'm against}} you, \\
\poemll    Pharaoh, king of Egypt! \\
\poeml You big monster! \\
\poemll    You lay in wait in the middle of your waterways\fnote{Or \fbib{Nile}; i.e. the Nile River, and so throughout the chapter} and say, \\
\poeml ``My waterways belong to me! \\
\poemll    I made them for myself!'' \\
\poeml \v{4}`So I'm going to plant a hook in your jaw \\
\poemll    and make the fish in your waterways grab hold of your scales. \\
\poeml I'll bring you up out of the middle of your waterways, \\
\poemll    along with all of the fish from your waterways that cling to your scales, \\
\poeml \v{5}Then I'll fling you out into the desert, \\
\poemll    you and all those fish in your waterways. \\
\poeml You'll fall out in the open fields; \\
\poemll    you'll never be reunited. \\
\poeml I'm giving you to the wild beasts of the earth \\
\poemll    and to the birds of the sky, \\
\poemlll       and they will dine on you! \\
\poeml \v{6}`Then everyone living in Egypt \\
\poemll    will know that I am the \divine{Lord}, \\
\poeml because they have been an unreliable ally\fnote{Lit. \fbib{been a staff made of reeds}} \\
\poemll    to the house of Israel. \\
\poeml \v{7}When they reached out to you for support, \\
\poemll    you tore their hands \\
\poemlll       and dislocated all of their shoulders. \\
\poeml When they tried to lean on you, \\
\poemll    they couldn't control their own bowels.'\fnote{Lit. \fbib{all of their loins became unstable}; so LXX; i.e. an involuntary physiological response from terror in the aftermath of their military defeat}
\end{poetry}

\v{8}``Therefore this is what the Lord \divine{God} says: `Look out! I'm bringing violent death\fnote{Lit. \fbib{bringing a sword}} in your direction! I'm going to kill every person and animal, \v{9}and the land of Egypt will be turned into a desolate ruin. Then you will know that I am the \divine{Lord}. Because Egypt said, ``The Nile is mine. I made it!'' \v{10}therefore watch out! I'm coming to get\fnote{Lit. \fbib{I'm against}} you! I'm going to attack your waterways, and then I'm going to make the land of Egypt a total wasteland from the Aswan\fnote{Lit. \fbib{Syene}, an Egyptian frontier town near the southern border with Ethiopia} fortress to the border of Ethiopia!\fnote{Lit. \fbib{Cush}; the Heb. name means \fbib{black}} \v{11}Neither man nor beast will walk through that area. It won't even be inhabited for 40 years. \v{12}I'll see to it that Egypt becomes a devastated land in the midst of devastated lands. Her cities deep inside her territories will be laid waste and desolate for 40 years. I will scatter Egypt among the nations and disperse them throughout the land.'\,''
\passage{Restoration of Egypt after Judgment}

\v{13}``Because this is what the \divine{Lord} says: `At the end of 40 years I'll gather the Egyptians from the people among whom they have been scattered. \v{14}I'll restore the economy\fnote{Lit. \fbib{fortunes}} of Egypt and return them to the land of Pathros,\fnote{I.e. southern Egypt} from which they originated, and there they will remain an insignificant kingdom, \v{15}the least significant of kingdoms. It will never again dominate other nations. I will make them so small that they will never again rule any nation. \v{16}Egypt will never again be a source of confidence to the nation\fnote{Lit. \fbib{house}} of Israel. Instead, Egypt will serve as a reminder of when they sinned by turning to Egypt for help. Then they'll know that I am the Lord \divine{God}.'\,''
\passage{Egypt Given to Nebuchadnezzar}

\v{17}On the first day of the first month of the twenty-seventh year of our captivity,\fnote{The Heb. lacks \fbib{of our captivity}} a message came to me from the \divine{Lord}, who had this to say:

\v{18}``Son of Man, King Nebuchadnezzar of Babylon made his army work very hard to attack Tyre. They tore their hair out and rubbed their shoulders raw! Despite all of that work trying to capture Tyre, neither he nor his army got paid from Tyre for all that! \v{19}Therefore this is what the Lord \divine{God} says: `I'm going to give the land of Egypt to King Nebuchadnezzar of Babylon. He's going to carry off her wealth, confiscate her war implements, and use it all to pay wages for his army! \v{20}I've given him the land of Egypt as a reward for attacking Tyre for me,' declares the Lord \divine{God}. \v{21}`When that day comes about, I'll strengthen Israel's military might, and I will give you an audience in their midst. Then they will know that I am the \divine{Lord}.'\,''
\labelchapt{30}
\passage{The Day of the \divine{Lord}}

\chapt{30}
\v{1}Another message came to me from the \divine{Lord}, who had this to say:

\v{2}``Son of Man, here's what you are to prophesy and announce,

\begin{poetry}
\poeml `This is what the Lord \divine{God} says: \\
\poeml ``Wail out loud! \\
\poemll    Oh no! The day! \\
\poeml \v{3}For comes now the day--- \\
\poemll    comes now the Day of the \divine{Lord}, \\
\poeml the day of clouds! \\
\poemll    The time of the gentiles\fnote{Or \fbib{nations}} is fulfilled!\fnote{Or \fbib{gentiles shall come to pass}; or \fbib{It shall be the time of judgment for the gentiles}} \\
\poeml \v{4}War\fnote{Lit. \fbib{The sword}} will come to Egypt, \\
\poemll    and Ethiopia will be in anguish \\
\poeml when the slain fall in Egypt, \\
\poemll    when her wealth is carried off, \\
\poemlll       and her foundations are demolished.
\end{poetry}

\v{5}``Ethiopia,\fnote{Lit. \fbib{Cush}; the Heb. name means \fbib{black}} Libya,\fnote{Lit. \fbib{Put}; the Heb. name means \fbib{bow}} descendants of\fnote{The Heb. lacks \fbib{descendants of}} Lud,\fnote{I.e., the fourth son of Shem, progenitor of the Lydians. The Heb. name \fbib{Lud} means \fbib{strife.}} all those who have mixed themselves,\fnote{So LXX; cf. Dan 2:43; MT reads \fbib{all the twilight}} and Libya\fnote{Lit. \fbib{Chub}; the Heb. means \fbib{horde}}---along with everyone in the land of Israel who is in league\fnote{Or \fbib{who have joined in a covenant}} with them---will die violently.''\,'\,''\fnote{Lit. \fbib{will fall by the sword}}
\passage{Continued Judgment on Egypt}

\begin{poetry}
\poeml \v{6}``This is what the \divine{Lord} says: \\
\poeml `Those who are supporting Egypt will fall; \\
\poemll    her majestic strength that she brought\fnote{The Heb. lacks \fbib{that she brought}} from the Aswan\fnote{Lit. \fbib{Syene}, an Egyptian frontier town near the southern border with Ethiopia} fortress will collapse \\
\poemlll       by the sword that invades her,' \\
\poeml declares the Lord \divine{God}. \\
\poeml \v{7}They'll remain desolate among desolate lands, \\
\poemll    their cities will be named among those that are ruined. \\
\poeml \v{8}They will know that I am the \divine{Lord} \\
\poemll    when I kindle my fire in Egypt \\
\poemlll       and all who help her are crushed.
\end{poetry}

\v{9}`When that happens, couriers will go out in ships to terrify Ethiopia\fnote{Lit. \fbib{Cush}; the Heb. name means \fbib{black}} in its complacency. Anguish will visit them as it will visit Egypt. Watch out! It's coming!'\,''
\passage{Foreigners will Invade Egypt}

\begin{poetry}
\poeml \v{10}``This is what the \divine{Lord} says: \\
\poeml `I'm putting an end to that gang from Egypt, \\
\poemll    and I'm going to use King Nebuchadnezzar of Babylon, to do it! \\
\poeml \v{11}He and his ruthless army with him will be brought \\
\poemll    to destroy the land. \\
\poeml They'll draw their swords and attack\fnote{Lit. \fbib{swords against}} Egypt, \\
\poemll    filling the land with the dead! \\
\poeml \v{12}I'll dry up their waterways, \\
\poemll    and evil men will sell off the land. \\
\poeml I'm going to make that land desolate, \\
\poemll    along with everything that's in it, \\
\poeml and I'm going to use foreigners to do it. \\
\poemll    I, the \divine{Lord} have spoken!'\,''
\passage{Destruction of Egypt's Gods}
\poeml \v{13}``This is what the Lord \divine{God} says: \\
\poeml `I will destroy the idols \\
\poemll    and put an end to the images that come from Memphis. \\
\poeml There will no longer be a prince from the land of Egypt, \\
\poemll    and I will terrify the land of Egypt. \\
\poeml \v{14}I'm going to turn Pathros into a desolation, \\
\poemll    set fire to Zoan,\fnote{I.e. the residence city of Egypt's Pharaoh at the time of the exodus (c. 1440 BCE)} \\
\poemlll       and judge Thebes.\fnote{Lit. \fbib{No}; i.e. the ancient capital of Egypt} \\
\poeml \v{15}I'll pour out my anger on Sin,\fnote{So MT; LXX reads \fbib{Sain}; i.e. Pelusium, a fortified city on Egypt's northeastern border} \\
\poemll    Egypt's strong fortress, \\
\poemlll       and I'll eliminate the gangs in Thebes. \\
\poeml \v{16}I'll set fire to Egypt, \\
\poemll    and Aswan\fnote{LXX reads \fbib{Syene}, an Egyptian frontier town near the southern border with Ethiopia; MT reads \fbib{Sin}} will writhe in agony. \\
\poeml Thebes will be demolished, \\
\poemll    and Memphis will face daily distress. \\
\poeml \v{17}The young men of On and Pi-beseth will die violently,\fnote{Lit. \fbib{will fall by the sword}} \\
\poemll    and their cities will be taken captive. \\
\poeml \v{18}It will be a dark day for Tahpanhes \\
\poemll    when I break the yokes of Egypt. \\
\poeml That's when her arrogant power will come to an end. \\
\poemll    She'll be covered by a cloud, \\
\poemlll       and her citizens\fnote{Lit. \fbib{daughters}} will go into captivity. \\
\poeml \v{19}I will judge Egypt, \\
\poemll    and they will learn that I am the \divine{Lord}.'\,''
\end{poetry}
\passage{Babylon's Victory}

\v{20}On the seventh day of the first month of the eleventh year of our captivity,\fnote{The Heb. lacks \fbib{of our captivity}} a message came to me from the \divine{Lord}. It had this to say: \v{21}``Son of Man, I've broken the arm of Pharaoh, king of Egypt. Look! It hasn't been set in a splint for healing or wrapped with a bandage so it could be strong enough to hold a sword! \v{22}Therefore this is what the Lord \divine{God} says:

`I'm coming to attack Pharaoh, king of Egypt, and I'm going to break both of his arms, the strong one and the wounded one. That will make him drop his sword. \v{23}I'm going to scatter Egypt throughout the surrounding\fnote{The Heb. lacks \fbib{surrounding}} nations and disperse them throughout the world. \v{24}I'm going to strengthen the military might\fnote{Lit. \fbib{the arm}} of the king of Babylon, put my own sword in his hand, and break Pharaoh's strength.\fnote{Lit. \fbib{arms}} Then Pharaoh\fnote{Lit. \fbib{he}} will groan like a dying man right in front of the king of Babylon.\fnote{Lit. \fbib{of him}} \v{25}When I strengthen the military might of Babylon, the military might of Pharaoh will fail, and then they will learn that I am the \divine{Lord} when I place my own sword in the hand of the king of Babylon. He will attack the land of Egypt. \v{26}When I scatter the Egyptians among the nations and disperse them throughout the world, they will learn that I am the \divine{Lord}.'\,''
\labelchapt{31}
\passage{Egypt Learns from Assyria's Demise}

\chapt{31}
\v{1}On the first day of the third month of the eleventh year of our captivity,\fnote{The Heb. lacks \fbib{of our captivity}} this message came to me from the \divine{Lord}: \v{2}``Son of Man, tell this to Pharaoh, king of Egypt and his gangs:

\begin{poetry}
\poeml `Who do you think you are? \\
\poemll    What makes you so great? \\
\poeml \v{3}Think about Assyria,\fnote{Lit. \fbib{Asshur}} \\
\poemll    that cedar of Lebanon, \\
\poeml beautiful with its branches, \\
\poemll    like a shady forest, \\
\poeml with an awesome height, \\
\poemll    its summit touches the clouds. \\
\poeml \v{4}Abundant water made it great, \\
\poemll    Subterranean rivers made it grow. \\
\poeml Rivers surrounded the area where it had been planted, \\
\poemll    and water channels nourished all the trees in the fields. \\
\poeml \v{5}That's why it grew taller than any of the trees in the fields. \\
\poemll    Its boughs flourished. \\
\poeml Its branches grew luxurious \\
\poemll    because all the water made it spread out well. \\
\poeml \v{6}The birds in the sky made nests in its boughs; \\
\poemll    all the beasts of the field gave birth under its branches. \\
\poemlll       All the great nations rested in its shade. \\
\poeml \v{7}`Beautiful because it was so great, \\
\poemll    with its long branches, \\
\poemlll       it was rooted in many bodies of water.\fnote{Lit. \fbib{many waters}} \\
\poeml \v{8}The cedars in God's garden could not compare to it; \\
\poemll    Fir trees could not match its boughs. \\
\poeml The plane tree\fnote{i.e. a species of trees that could readily be stripped of their bark; cf. Gen 30:37} never grew branches like it, \\
\poemll    and no tree in God's garden compares to its beauty. \\
\poeml \v{9}I made it beautiful, \\
\poemll    including all of its branches; \\
\poeml all the trees in God's garden of Eden envied it!'\,''
\end{poetry}
\passage{Assyria's Fall Due to Arrogance}

\v{10}``Therefore this is what the Lord \divine{God} says: `Because of its towering height, with its summit reaching into the clouds, and because it was haughty in its position,\fnote{Lit. \fbib{heart}} \v{11}I turned it over to the leader of those\fnote{Lit. \fbib{to the hand of the leader of the}} nations, who dealt with it thoroughly. I have driven it away because of its wickedness. \v{12}Foreign dictators have trimmed it down to size and abandoned it. Its branches have fallen off on mountains and in all the valleys. Its boughs have broken off in all the ravines of the land. All the nations of the earth have moved out of its shade and abandoned it. \v{13}All the birds in the sky will live among its ruins, and the wild animals\fnote{Lit. \fbib{the animals of the field}} will forage among its branches. \v{14}As a result, none of its watered trees will grow tall, their tops will never reach to the clouds, and they'll never grow so high again, because all of them have been appointed\fnote{Lit. \fbib{given}} to death in the world beneath where human beings go, that is, down to the Pit.'\,''\fnote{I.e. the realm of eternal punishment in the afterlife}
\passage{Fear at Assyria's Fall}

\v{15}``This is what the Lord \divine{God} says: `On the day that it descended into Sheol,\fnote{I.e. the realm of the afterlife} I shut down its water supplies, covered over its deep water, and shut down its rivers. As a result its abundant water sources dried up, and I caused Lebanon to mourn for it. All the trees of the field wilted because of it. \v{16}I made the nations tremble when they heard that Assyria\fnote{Lit. \fbib{he}} was falling, descending into Sheol\fnote{I.e. the realm of the afterlife} to join those who go down into the Pit.\fnote{I.e. the realm of punishment in the afterlife} Then all of the trees of Eden in the world below were comforted, including the choicest and best of Lebanon, all of whom were well-watered. \v{17}They also went down with it into Sheol,\fnote{I.e. the realm of the afterlife} to those who had been killed violently\fnote{Lit. \fbib{killed by the sword}} and to those who had trusted in its strength by living in its shadow among the nations. \v{18}So tell me now, which of the trees of Eden compares to you in glory or greatness? Nevertheless, you'll be brought down, along with those trees of Eden, to the earth below. You'll lie in the middle of the uncircumcised, with those who have been killed in war.\fnote{Lit. \fbib{killed by the sword}} Pharaoh and all his gang will be just like this!' declares the Lord \divine{God}.''
\labelchapt{32}
\passage{Another Prophecy about Egypt}

\chapt{32}
\v{1}On the first day of the twelfth month of the twelfth year of our captivity,\fnote{The Heb. lacks \fbib{of our captivity}} a message came to me from the \divine{Lord}, who had this to say:

\v{2}``Son of Man, start singing this lamentation about Pharaoh, king of Egypt. Tell him,

\begin{poetry}
\poeml `You may have called yourself a lion among nations, \\
\poemll    but you're a monster at sea. \\
\poeml You thrash about in your rivers, \\
\poemll    muddy the water with your feet, \\
\poemlll       and relieve yourself in the rivers.' \\
\poeml \v{3}``This is what the Lord \divine{God} says: \\
\poemll    `I'm coming fishing for you! \\
\poeml Right in the sight of many nations \\
\poemll    they'll haul you up in my dragnet. \\
\poeml \v{4}I'll fling you up onto the land; \\
\poemll    I'll haul you into the field, \\
\poeml I'll make every carrion-eating bird come to dine on you, \\
\poemll    and I'll make all the scavenging animals gorge themselves on you. \\
\poeml \v{5}I'll cover the mountains with your flesh \\
\poemll    and fill their valleys with your rotting carcass.\fnote{So MT; the Syr. Peshittas read \fbib{your maggots}} \\
\poeml \v{6}I'll drench the land with your blood, \\
\poemll    right up to the mountains, \\
\poeml and the ravines will overflow \\
\poemll    with blood that comes from you! \\
\poeml \v{7}When I extinguish your lights, \\
\poemll    I'll cover the heavens \\
\poemlll       and darken their stars. \\
\poeml I'll cover the sun with a cloud \\
\poemll    and the moon won't reflect its light. \\
\poeml \v{8}I'll darken the bright lights in the sky above you \\
\poemll    and bring darkness to your territory,' \\
\poemlll       declares the Lord \divine{God}.
\end{poetry}

\v{9}```I'll bring distress to the hearts of many nations when I destroy you among nations whose territories you have not known. \v{10}I'll make many nations be appalled at you, and their kings will be terrified because of you when I brandish my sword right in their face. They will all tremble from fear for their own safety\fnote{Lit. \fbib{soul}} on the day that you fall!'

\v{11}``This is what the Lord \divine{God} says: `The army\fnote{Lit. \fbib{sword}} of the king of Babylon will attack you. \v{12}I'm going to make your gangs die using the weapons of valiant warriors, all of whom are ruthless people.

\begin{poetry}
\poeml `They will devastate the majesty of Egypt, \\
\poemll    destroying all of its hordes. \\
\poeml \v{13}I'm going to destroy all of its livestock \\
\poemll    along its many riverbanks. \\
\poeml Human feet won't muddy the rivers anymore, \\
\poemll    nor will the hooves of livestock stir up the water. \\
\poeml \v{14}That's when I'll make their waterways flow smoothly, \\
\poemll    and their rivers flow like olive oil,'\fnote{I.e. the rivers will be undisturbed by human activity} \\
\poemlll       declares the Lord \divine{God}.'' \\
\poeml \v{15}`When I turn the land of Egypt into a desolation, \\
\poemll    and the land is emptied of everything that used to fill it, \\
\poeml when I strike everyone who lives there, \\
\poemll    they will learn that I am the \divine{Lord}.'
\end{poetry}

\v{16}``This has been a lamentation. They will chant it, and the citizens\fnote{Lit. \fbib{daughters}} of the nations will chant it, too. They'll chant it about Egypt and about all of its hordes.''
\passage{Babylon's Invasion of Egypt}

\v{17}On the fifteenth day of the first\fnote{The Heb. lacks \fbib{the first}} month of the twelfth year of our captivity,\fnote{The Heb. lacks \fbib{of our captivity}} a message from the \divine{Lord} came to me, and this is what it said: \v{18}``Son of Man, mourn about the hordes of Egypt. Bring them down---that is, her and the citizens\fnote{Lit. \fbib{daughters}} of those majestic\fnote{Or \fbib{powerful}} nations---whose destiny is the deep part of the Pit.\fnote{I.e. the realm of punishment in the afterlife}

\begin{poetry}
\poeml \v{19}``So who's more beautiful than you? \\
\poemll    You'll be buried with the uncircumcised.\fnote{I.e. as one who does not honor God, and so throughout the chapter}
\end{poetry}

\v{20}``They'll die along with others who are killed violently.\fnote{Lit. \fbib{killed with the sword}, and so throughout the chapter} Egypt has been given over to violence,\fnote{Lit. \fbib{to the sword}, and so throughout the chapter} which will carry off both it and its hordes.''
\passage{Egypt Condemned by the Dead}

\v{21}``Mighty leaders will address them and those who assist them right out of the middle of Sheol:\fnote{I.e. the realm of the afterlife} `They've come down and will lie still, these uncircumcised people who have died violently.'\fnote{Lit. \fbib{by the sword}, and so throughout the chapter} \v{22}Assyria will be there, along with all of those who keep company with her,\fnote{I.e. the nation personified as a woman} all of them killed violently. \v{23}Her grave will be set in the remotest part of the Pit,\fnote{I.e. the realm of punishment in the afterlife} surrounded by those who accompanied her. All of them will have been killed, executed violently, who spread terror throughout the land of the living.

\v{24}``Elam will be there. Its hordes will surround Elam's\fnote{Lit. \fbib{its}} grave. All of them have been killed. They died\fnote{Lit. \fbib{fell}} violently, and they have descended uncircumcised into the world below after having spread terror throughout the land of the living. They will bear the shame of those who descend to the Pit.\fnote{I.e. the realm of punishment in the afterlife} \v{25}They have prepared a bed for her and for her hordes that surround her graves. All of them are uncircumcised, having been killed violently, because they had spread terror throughout the land of the living. They will bear the shame of those who descend to the Pit\fnote{I.e. the realm of punishment in the afterlife} and will take their place among the dead.

\v{26}``Meshech and Tubal will be there, along with all of the hordes that surround her grave. Every one of them is uncircumcised, killed violently, because they spread terror throughout the land of the living. \v{27}They won't be buried with dead warriors from ancient times, who went straight to Sheol,\fnote{I.e. the realm of the afterlife} buried with their war weapons, with their swords placed under their heads and their shields laid on top of their bones, since they spread terror throughout the land of the living. \v{28}You'll be broken, and you'll lie down with the uncircumcised who died violently.

\v{29}``Edom will be there, along with its kings and princes who despite all their power have been killed violently. They, too, are lying dead, along with the uncircumcised; that is, with those who descend into the Pit.\fnote{I.e. the realm of punishment in the afterlife}

\v{30}``All of the princes from the North are there, along with the Sidonians, who have gone down in shame to join those who have been killed because of all the terror they caused by their military might. They lie dead, uncircumcised, with those who have been killed violently. They will bear their shame, along with those who descend into the Pit.\fnote{I.e. the realm of punishment in the afterlife}

\v{31}``When Pharaoh sees them, he will take comfort in his hordes. Pharaoh and all his army will die violently,'' says the Lord \divine{God}, \v{32}``because he spread terror throughout the land of the living. Therefore he'll be laid to rest among the uncircumcised, who have been killed violently; that is, Pharaoh and all of his hordes,'' declares the Lord \divine{God}.
\labelchapt{33}
\passage{Warnings for Watchmen}

\chapt{33}
\v{1}This message came to me from the \divine{Lord}: \v{2}``Son of Man, speak to your nation's children and tell them: `If I bring war\fnote{Lit. \fbib{bring a sword}} to a land, and the people of that land appoint one of their conscripted men\fnote{Or \fbib{of the men from their border}} to serve as a sentinel, \v{3}and if he notices that violence\fnote{Lit. \fbib{that a sword}} is approaching and sounds an alarm to warn the people, \v{4}then if anyone who hears the sound of the alarm does not heed the warning, when the sword arrives and destroys him, his shed blood will remain his own responsibility.\fnote{Lit. \fbib{will be on him}} \v{5}After all, he heard the alarm sounding, but did not heed the warning, so his shed blood will remain his own responsibility.\fnote{Lit. \fbib{will be on him}} If he had heeded the warning, he would have saved himself.\fnote{Lit. \fbib{have rescued his soul}} \v{6}If that sentinel notices that violence is approaching, but does not sound an alarm, then because the nation does not take warning and the sword arrives and destroys their lives because of their guilt, I'll seek retribution for their shed blood from the\fnote{Lit. \fbib{from the hand of the}; i.e. as if the sentinel were responsible for the death} one who was acting as sentinel.'\,''
\passage{Warning for Ezekiel}

\v{7}``Now as for you, Son of Man, I've established you as a sentinel for the house of Israel. So whenever you hear a message from me,\fnote{Lit. \fbib{from my mouth}} you are to warn the people\fnote{Lit. \fbib{warn them}} from me. \v{8}If I should say to a certain wicked person, `You wicked man, you're certainly about to die,' but you don't warn him to turn from his wicked behavior,\fnote{Lit. \fbib{ways}} he'll die in his guilt, but I'll seek retribution for his bloodshed from you.\fnote{Lit. \fbib{from your hand}; i.e. as if the sentinel were responsible for the death} \v{9}However, if you warn the wicked to turn from his behavior\fnote{Lit. \fbib{way}} and he does not do so, he will die in his guilt, and you will have saved yourself.''\fnote{Lit. \fbib{have rescued your soul}}
\passage{God Hates the Death of the Wicked}

\v{10}```Now, Son of Man, tell this to the house of Israel:

`You keep saying, ``Our crimes and sins burden us so much that we're rotting away, so how can we keep on living?''\,'

\v{11}``Tell them, `As certainly as I'm alive and living,' declares the Lord \divine{God}, `I receive no pleasure in the death of the wicked. Instead, my pleasure is that the wicked repent from their behavior\fnote{Lit. \fbib{way}} and live. Turn back! Turn back, all of you, from your wicked behavior! Why do you have to die, you house of Israel?'\,''
\passage{Human Effort is Useless in Sustaining Righteousness}

\v{12}``And now, Son of Man, say this to your people: `The righteousness of the righteous won't save them when they keep on committing crimes against me, the wickedness of the wicked won't keep them from remaining away\fnote{Lit. \fbib{from stumbling}} when they're turning from their wickedness, and no righteous person will keep on living by their righteousness when they sin.'

\v{13}``If I tell the righteous person that he will certainly live, if he trusts in his own righteousness and commits evil, none of his righteousness will be remembered, and he will die because of the wrong that he commits.

\v{14}``If I tell the wicked person that he will certainly die, if he turns from his sin and acts with justice and righteousness, \v{15}returning what has been placed as collateral for a loan, paying back what he has taken, following the regulations that promote life, and committing no iniquity, he will certainly live, and not die. \v{16}None of the sins that he has committed will be remembered against him. Since he did what is just and right, he will certainly live.

\v{17}``Nevertheless, your people's children keep saying, `Living life the Lord's way\fnote{Lit. \fbib{saying, `The way of}} isn't right,' when all the while it is their way of living\fnote{The Heb. lacks \fbib{of living}} that isn't right. \v{18}When a righteous man forsakes his own righteousness and commits evil acts, he will die because of those acts, \v{19}and when the wicked turn away from their wickedness and do what is just and right, he will certainly live because of that. \v{20}``And yet you keep saying, `Living life\fnote{Lit. \fbib{saying, `The way of}} the Lord's way isn't right,' But I will judge every one of you according to the way you live, you house of Israel!''
\passage{False Reliance on Abraham's Heritage}

\v{21}On the fifth day of the tenth month of the twelfth year of our captivity, a fugitive who had escaped from Jerusalem came and informed me, ``The city has been destroyed.''

\v{22}Now the hand of the \divine{Lord} had been touching me the evening before that fugitive arrived, so the \divine{Lord} had given me something to say by the time the messenger\fnote{Lit. \fbib{time he}} arrived the next morning. He opened my mouth and I no longer had nothing to say to him.\fnote{The Heb. lacks \fbib{to him}} \v{23}As a result, this message came to me from the \divine{Lord}:

\v{24}``Son of Man, those who are living among these ruins of the land of Israel keep saying, `Abraham was only one man, but he was able to possess the land! As for us, we're a multitude, and the land has been given to us as an inheritance.' \v{25}So tell them, `This is what the Lord \divine{God} says: ``You keep eating flesh along with its blood, you keep looking to your idols, and you keep shedding blood, and you're going to take possession of the land? \v{26}You keep trusting in your weapons, you continue to commit loathsome deeds, men keep defiling their neighbors' wives, and you're going to take possession of the land?'

\v{27}``Tell them this: `This is what the Lord \divine{God} says: ``As certainly as I'm alive and living, those who live in the wastelands are certain to die violently,\fnote{Lit. \fbib{die by the sword}} I'll give those who die in the open fields to the wild animals for food, and whoever takes refuge in caves and fortified places will die of diseases. \v{28}Then I'll turn the land into a desolate ruin and her arrogant strength will come to an abrupt end. The mountains of Israel will become so desolate that no one will be able to travel over them.'' \v{29}`Then they'll learn that I am the \divine{Lord}, when I've turned their land into a desolate wasteland because of all of the loathsome deeds that they've committed.'\,''
\passage{The Disobedient Exiles of Babylon}

\v{30}``Now as for you, Son of Man, your nation's children keep gathering together to talk about you beside the walls and at the doorway to their houses. Everyone tells one another, `Please come! Let's go hear what the \divine{Lord} has to say!' \v{31}Then they come to you as a group, sit down right in front of you as if they were my people, hear your words---and then they don't do what you say---\fnote{The Heb. lacks \fbib{what you say}} because they're seeking only their own desires,\fnote{Lit. \fbib{because their lust is in their mouths}} they pursue ill-gotten profits, and they keep following their own self-interests. \v{32}As far as they are concerned, you sing romantic songs with a beautiful voice and play a musical instrument well. They'll listen to what you have to say, but they won't put it into practice! \v{33}When all of this comes about---and you can be sure that it will!---they'll learn that a prophet has been in their midst.''
\labelchapt{34}
\passage{Israel's False Shepherds}

\chapt{34}
\v{1}A message came from the \divine{Lord} for me, and it had this to say: \v{2}``Son of Man, prophesy against Israel's shepherds. Tell those shepherds, `This is what the Lord \divine{God} says:

``Woe to you shepherds of Israel who have been feeding yourselves and not the sheep. Shouldn't shepherds feed the sheep? \v{3}You're eating the best parts,\fnote{Lit. \fbib{the fat}} clothing yourselves with the wool, and slaughtering the home-grown sheep without having fed the sheep! \v{4}You haven't strengthened the weak, treated the sick, set broken bones, regathered the scattered, or looked for the lost. Instead, you've dominated them with brutal force and ruthlessness.

\v{5}``Since they have no shepherd, they have been scattered around and have become prey for all sorts of wild animals. How scattered they are! \v{6}My sheep have gone wandering on all of the mountains, on all of the hills, and throughout every high place in the whole world, with no one to look for them or go out in search of them.

\v{7}``Therefore listen to what the \divine{Lord} says, you shepherds: \v{8}`As certainly as I'm alive and living, my sheep have truly become victims, food for all of the wild animals because there are no shepherds. My shepherds did not go searching for my flock. Instead, the shepherds fed themselves, and my flock they would not feed!'

\v{9}``Therefore, you shepherds, listen to what the \divine{Lord} says: \v{10}`This is what the Lord \divine{God} says: ``Watch out, I'm coming after you shepherds! I'm going to demand my sheep back from them\fnote{Lit. \fbib{from their hand}} and fire them as shepherds. The shepherds won't be shepherds anymore when I snatch my flock right out of their mouths so they can't be eaten by them anymore.''\,'\,''
\passage{The Coming True Shepherd}

\v{11}``This is what the \divine{Lord} says: `Watch me! I'm going to search for my flock. I'll watch over them myself. \v{12}Just as a shepherd looks after his flock during the day time while he is with them, so also I'm going to watch over my sheep, delivering them from every place where they've been scattered during the times of gloom and doom.\fnote{Lit. \fbib{cloud}} \v{13}I'm going to bring them out from foreign\fnote{Lit. \fbib{the}} nations and from foreign lands. Then I'll bring them to their own land and feed them in Israel---on the mountains, in their valleys, and in all of their settlements throughout the land. \v{14}I'll feed them in excellent pastures, and even the very heights of Israel's mountains will serve as verdant pastures for them in which they'll rest and feed---yes, even on the fertile mountains of Israel! \v{15}I will feed my sheep and give them rest,' declares the Lord \divine{God}. \v{16}`I'm going to seek both the lost as well as the scattered, and bring them both back so their broken bones can be set and the sick can be healed. But in righteousness I'll exterminate the fat and the stiff-necked.'\,''
\passage{God's Message to His Sheep}

\v{17}``Now as for you, my flock, this is what the Lord \divine{God} says: `Watch out! I'm going to judge between one sheep and another, and between the rams and the goats. \v{18}Is it such an insignificant thing to you that you're feeding in good pastures but trampling down the other pastures with your feet? Or that as you're drinking from the clear streams you're muddying the rest with your feet? \v{19}My flock is grazing on what you've been treading down with your feet and they're drinking what you've been making muddy with your feet!'

\v{20}``Therefore this is what the Lord \divine{God} says to them: `Watch me! I'm going to judge between the fat sheep and the lean sheep, \v{21}since you've been bumping aside all the weaker sheep with your backsides and shoulders, butting them with your horns until they're scattered around outside. \v{22}That's how I'll save my sheep so they won't be plundered any longer. I'm going to judge between one sheep and another.'\,''
\passage{God's Shepherd: His Servant David}

\v{23}```Then I'll install one shepherd for them---my servant David---and he will feed them, will be there for them, and will serve as their shepherd. \v{24}I, the \divine{Lord}, will be their God, and my servant David will rule among them as Prince.' I, the \divine{Lord}, have spoken this.

\v{25}``I'm going to enter into a covenant with them, one of peace, and I'll eliminate wild beasts from the land so they can live securely in the wilderness and sleep in the forests. \v{26}I'm going to make\fnote{Lit. \fbib{give}} them and everything that surrounds my hill\fnote{I.e. Mount Zion} a blessing. I'll send down the rain! At the appropriate time there will be a rainstorm of blessing! \v{27}I'll bring fruit to the trees in the orchards, the land will yield its produce, they will live securely on their land, and they will learn that I am the \divine{Lord}, when I break the bar that has been their yoke and deliver them from the control of those who have enslaved them. \v{28}They will no longer be plundered by the nations, and wild animals will no longer devour them. They will settle down confidently, with nothing to frighten them. \v{29}I'm going to prepare for them the best of gardening spots. They will no longer live as victims in a land of starvation, nor will they have to bear the insults of the international community. \v{30}That's when they'll learn that I, the \divine{Lord} their God, am with them, and that they, the house of Israel, are my people,' declares the Lord \divine{God}. \v{31}`And as for you, my sheep, the flock that I'm pasturing, you are mankind, and I am your God,' declares the Lord \divine{God}.''
\labelchapt{35}
\passage{Prophecy against Mount Seir}

\chapt{35}
\v{1}A message came to me from the \divine{Lord} and it went like this: \v{2}``Son of Man, turn your attention\fnote{Lit. \fbib{face}} toward Mount Seir\fnote{This mountain, the modern \fbib{Jebel esh-sher\'{a}}, is located in the mountain range that extends south of the Dead Sea toward the Gulf of Aqaba, and is bordered by the Arabah Valley to the west.} and begin to prophesy against it. \v{3}Tell them,\fnote{Lit. \fbib{him}; i.e. the city personified as a single person} `This is what the Lord \divine{God} says:

\begin{poetry}
\poeml ``Watch out! I'm coming to get you, Mount Seir! \\
\poemll    I'm stretching out my hand to strike you, \\
\poemlll       and I'm going to turn you into a desolate wasteland. \\
\poeml \v{4}I'm going to turn your cities into ghost towns, \\
\poemll    and you will become a ruin. \\
\poeml Then you will learn \\
\poemll    that I am the \divine{Lord}.
\end{poetry}

\v{5}``Because of your undying hatred, you kept on making the Israelis experience abuse\fnote{Lit. \fbib{to the hand of the sword}} during the time of their calamity, even when they were in their final stages\fnote{Lit. \fbib{time}} of punishment, \v{6}therefore as I'm alive and living,'' declares the Lord \divine{God}, ``I'm turning you over to bloodshed,\fnote{The Heb. word \fbib{blood} sounds like \fbib{Edom}, the territory south of the Dead Sea in which Mt. Seir, the modern \fbib{Jebel esh-sher\'{a}}, is located} and bloodshed will certainly overtake you, since you never have hated shedding blood. That's why bloodshed will certainly pursue you. \v{7}I'm turning Mount Seir over to ruin and desolation. I'm going to eliminate everyone who comes and goes, \v{8}and I'll fill that\fnote{Lit. \fbib{his}} mountain with the dead. Those who die by violence\fnote{Lit. \fbib{by the sword}} will cover your hills, and fill your valleys and all your ravines! \v{9}I will turn you into an everlasting wasteland, and your cities will never be inhabited. Then you'll learn that I am the \divine{Lord}!

\v{10}``Because you have claimed, `These two nations and these two lands are going to belong to me, and we will take possession of them, even though the \divine{Lord} is there,' \v{11}therefore as I'm alive and living'' declares the Lord \divine{God}, ``I'm going to deal with you as your anger deserves. When I judge you, I'll treat you like you did the Israelis\fnote{Lit. \fbib{did them}}---that is, with the same kind of envy that motivated your constant hatred of them. \v{12}That's how you'll know that I, the \divine{Lord}, have heard every loathsome, reviling thing that you've had to say against the mountains of Israel, such as, `They're desolate, and we'll eat them for dinner!' \v{13}Not only that, you've arrogantly reviled me many times over, and I've heard every word!

\v{14}``So this is what the Lord \divine{God} says: `Just as the earth rejoices, I'm going to turn you into a desolate wasteland. \v{15}Just as you rejoiced when Israel's inheritance became desolate, I'm going to do the same thing to you. Mount Seir, you and Edom---all of you---will become a desolate wasteland.' Then they will learn that I am the \divine{Lord}.''
\labelchapt{36}
\passage{Prophecy to Israel's Mountains}

\chapt{36}
\v{1}``Now as for you, Son of Man, prophesy to Israel's mountains and tell them, `Listen to this message from the \divine{Lord}, you mountains of Israel: \v{2}``This is what the Lord \divine{God} says: `The enemy has been saying about you, ``Good! The ancient heights are back in our possession!''\,'\,''\,'

\v{3}``Therefore this is what you are to prophesy: `Here's what the Lord \divine{God} says, ``You've been turned into a desolate wasteland and crushed by everyone who surrounds you for a very, very good reason. You've become the property of all the other nations and you've become the object of gossip and whispering campaigns of the nations.''\,'\,''

\v{4}``Therefore listen to what the Lord \divine{God} has to say, you mountains of Israel: `This is what the Lord \divine{God} says to the mountains, to the hills, to the waterways, to the valleys, to the desolate wastelands, and to the abandoned cities that have become an object of derision to the mountains that surround you: \v{5}```Because this is what the Lord \divine{God} says: ``Motivated by my burning zealousness, I have spoken against the rest of the surrounding nations, including Edom, who confiscated my land, taking possession of it with joyful enthusiasm and with animal-like malice, in order to plunder Israel's\fnote{Lit. \fbib{her}} pastures.' \v{6}``Therefore prophesy concerning the land of Israel and speak to its mountains, hills, ravines, and valleys. Tell them, `This is what the Lord \divine{God} says: ``Pay attention! In my zealous anger I'm speaking because you've had to endure the mockery of the world's nations.''\,'\,''

\v{7}``Therefore this is what the Lord \divine{God} says: ``I hereby raise my hand to swear this oath: the nations that surround you will have their own mockery to endure! \v{8}But you mountains of Israel are going to sprout branches and bear fruit for my people Israel, because they'll be coming soon.'\,''
\passage{The Future of Israel's Mountains}

\v{9}``Watch me! I'm on your side! I'll be turning in your direction, and you mountains\fnote{The Heb. lacks \fbib{mountains}} will be plowed and planted. \v{10}I'm going to make the entire house of Israel grow---every single member\fnote{Lit. \fbib{every human being}}---and their cities will be inhabited with all the ruins rebuilt. \v{11}I'll make both the population and the livestock increase throughout your territories. They'll increase and be fruitful. I'll make your territories to be settled like you were in the past, and you will be treated better than you ever were before. At that time you will know that I am the \divine{Lord}.

\v{12}``I'll lead my people, my nation of Israel, across you mountains,\fnote{The Heb. lacks \fbib{mountains}} and they will take possession of you again, and you'll be their inheritance once more. Never again will you leave them robbed of children.

\v{13}``This is what the Lord \divine{God} says: `Because some have been saying to you, ``You mountains\fnote{The Heb. lacks \fbib{mountains}} have been devouring human beings and leaving people childless,'' \v{14}therefore you will no longer devour human beings or leave their nation childless,' declares the Lord \divine{God}. \v{15}`I won't let you hear other people mock you, and no nation will ever make you childless again,' declares the Lord \divine{God}.''
\passage{Israel's Past Punishments}

\v{16}This message came to me from the \divine{Lord}: \v{17}``Son of Man, when the house of Israel was living on their own land, they defiled it with their lifestyles\fnote{Lit. \fbib{ways}} and behavior; they were as disqualified to be with me as a menstruating woman is to you.\fnote{The Heb. lacks \fbib{is to you}} \v{18}So I poured out my anger on them because of all the bloodshed throughout the land and because they had defiled it with their idolatry. \v{19}Then I scattered them among the nations, dispersing them to other lands, just as their lifestyles and behavior deserved. That's how I judged them. \v{20}Nevertheless, when they arrived in those nations, they continued to profane my holy name. It was said about them, `These are the \divine{Lord}'s people, even though they've left his land.' \v{21}I've been concerned about my holy reputation,\fnote{Lit. \fbib{name}; and so throughout the chapter} which the house of Israel has been defiling throughout all of the nations where they've gone.''
\passage{The Coming Renewal of Israel}

\v{22}``Therefore tell the house of Israel, `This is what the Lord \divine{God} says: ``I'm not about to act for your sake, you house of Israel, but for the sake of my holy reputation, which you have been defiling throughout all of the nations where you've gone. \v{23}I'm going to affirm\fnote{Or \fbib{consecrate}} my great reputation that has been defiled among the nations (that is, that you have defiled in their midst), and those people will learn that I am the \divine{Lord},'' declares the Lord \divine{God}, ``when I affirm my holiness in front of their very eyes. \v{24}I'm going to remove you from the nations, gather you from all of the territories, and bring you all back to your own land. \v{25}I'll sprinkle pure water on you all, and you'll be cleansed from your impurity and from all of your idols.''

\v{26}`````I'm going to give you a new heart, and I'm going to give you a new spirit within all of your deepest parts. I'll remove that rock-hard heart of yours\fnote{Lit. \fbib{heart from your flesh}} and replace it with one that's sensitive to me.\fnote{Lit. \fbib{with a heart of flesh}} \v{27}I'll place my spirit within you, empowering you to live according to\fnote{Lit. \fbib{to walk in}} my regulations and to keep my just decrees. \v{28}You'll live in the land that I gave to your ancestors, you'll be my people, and I will be your God. \v{29}In addition, I'll deliver you from everything that makes you unclean. I'll call out to the grain you plant, ordering it to produce abundant yields, and I will never bring famine in your direction. \v{30}I'll increase the yields of your fruit trees and crops so that you'll never again experience the disgrace of famine that occurs in other nations. \v{31}Then you'll remember your lifestyles and practices that were not good. You'll hate yourselves as you look at your own iniquities and loathsome practices. \v{32}I won't be doing any of this for your sake,'' declares the Lord \divine{God}, ``so keep that in mind. Be ashamed and frustrated because of your behavior, you house of Israel!''\,'\,''
\passage{The Restoration of Israel's Cities}

\v{33}``This is what the Lord \divine{God} says: `At the same time\fnote{Lit. \fbib{On the day}} that I cleanse you from all of your guilt, I'll make your cities become inhabited again and the desolate wastelands will be rebuilt. \v{34}The desolate fields will be cultivated, replacing the former wasteland that everyone who passed by in times past\fnote{The Heb. lacks \fbib{in times past}} had noticed. \v{35}They will say, ``This wasteland has become like the garden of Eden, and what used to be desolate ruins are now fortified and inhabited.'' \v{36}Then the surviving people that live around you will learn that I, the \divine{Lord}, have rebuilt these ruins and have cultivated these pastures that used to be desolate wastelands. I, the \divine{Lord}, have spoken this, and I'm going to bring it about!'

\v{37}``This is what the Lord \divine{God} has to say: `I'm going to allow the house of Israel to ask anything they want from me, including this: I'm going to increase their population as a shepherd increases his flock. \v{38}The desolate cities will be filled with flocks of human beings, just like Jerusalem used to be filled with flocks of sheep during the times of the appointed festivals. Then they will know that I am the \divine{Lord}.'\,''
\labelchapt{37}
\passage{The Vision of the Valley of Bones}

\chapt{37}
\v{1}The \divine{Lord} laid his hand on me and brought me out by the Spirit of the \divine{Lord} to the middle of a valley that was filled with bones. \v{2}He led me here and there throughout\fnote{Lit. \fbib{me over them all around all around}} the valley, and I was amazed to see that the surface of the entire valley was covered with myriads of very dry bones! \v{3}The \divine{Lord}\fnote{Lit. \fbib{He}} asked me, ``Son of Man, will these bones ever live?''

``Lord \divine{God},'' I replied, ``you know the answer to that!''\fnote{The Heb. lacks \fbib{the answer to that}}

\v{4}Then the \divine{Lord} told me, ``Prophesy to these bones. Tell them: `You dry bones, listen to the message from the \divine{Lord}: ``\v{5}This is what the Lord \divine{God} says to you\fnote{Lit. \fbib{these}} dry bones! `Pay attention! I'm bringing my Spirit into you right now, and you're going to live! \v{6}I'm going to grow tendons on you, regenerate your flesh, cover you with skin, and make you breathe again so that you can come back to life and learn that I am the \divine{Lord}.'\,''\,'\,''
\passage{The Bones are Raised to Life}

\v{7}So I prophesied, just as I had been ordered to do so. Immediately there was a noise and a rattling---and then all of a sudden the bones came together by themselves! Each bone came together, all of them attached together!\fnote{Lit. \fbib{together, one to another}} \v{8}As I continued to watch, I saw tendons growing on the bones,\fnote{Lit. \fbib{on them}} and muscles growing and covering them, and then skin covered the flesh from above. But the bodies weren't breathing. \v{9}Then he ordered me, ``Prophesy to the Spirit, Son of Man. Tell the Spirit, `This is what the Lord \divine{God} says: ``Come from the four winds, you Spirit, and breathe into these people who have been killed, so they will live.''\,'\,'' \v{10}So I prophesied as I had been ordered, breath entered them, and they began to live. They stood on their own feet as a vast, united army.
\passage{The Vision is Interpreted for Ezekiel}

\v{11}``These bones represent the entire house of Israel,'' the \divine{Lord}\fnote{Lit. \fbib{Israel,'' he}} explained to me. ``Look how they keep saying, `Our bones are dried up, and our future is lost. We've been completely eliminated!' \v{12}``Therefore prophesy to them, and tell them, `Watch me! I'm going to open your graves, lift you out of those graves, and bring my people back into the land of Israel. \v{13}Then you'll learn that I am the \divine{Lord}, when I've opened your graves and caused you to come up out of them, my people. \v{14}I'm going to place my Spirit in you all, and you will live. I'll place you all into your land, and you'll learn that I, the \divine{Lord}, have been speaking and doing this,' declares the \divine{Lord}.'\,''
\passage{The Future Union of Israel and Judah}

\v{15}A message came to me from the \divine{Lord}, and this is what it was: \v{16}``Now as for you, Son of Man, grab a stick of wood for yourself and write on it these words:

\begin{poetry}
\poeml `\divine{For Judah and the Israelis, his companions}'
\end{poetry}

``Then grab another stick and write on it:

\begin{poetry}
\poeml `\divine{For Joseph, the stick of Ephraim, and all the house of Israel, his companions}'
\end{poetry}

\v{17}``Then join them together end-to-end so that they become a single baton in your hand. \v{18}When the descendants of your people ask you, `Would you please explain to us what you mean by this?' \v{19}you are to tell them, `This is what the \divine{Lord} says: ``Watch me! I'm taking the baton that represents Joseph, which Ephraim is holding in his hand, along with his companions the tribes of Israel, and I'm going to join them with the baton that represents Judah. I'm making them a single baton, that is, a complete baton in my hand.''

\v{20}``The batons on which you engrave your writing are to remain right in front of them in your hand. \v{21}Then tell them, `This is what the Lord \divine{God} says: ``Watch me take the Israelis out of the nations where they've gone and return them from every direction. I'm going to bring them back into their own land. \v{22}I'm going to make them a united people in the land, on the mountains of Israel, and I'll set a single king to rule over them. They'll never again be two separate people. They'll never again be divided into two kingdoms. \v{23}They will never again defile themselves with their idols, with other loathsome things, or with any of their sins. Instead, I will deliver them from all of the places where they have sinned, and then I'll cleanse them. They will be my people and I will be their God.''\,'\,''
\passage{David's Rule as King}

\v{24}`````My servant King David will be there for them, and one shepherd will be appointed for them. They will live according to my decrees, keep my regulations, and practice them. \v{25}They will live in the land that I gave to my servant Jacob and on which your ancestors lived. They will live in that land, along with their children and grandchildren, forever. David my servant will be their everlasting leader. \v{26}I'll make a secure covenant\fnote{Or \fbib{a covenant of peace}} with them, one that will last forever. I will establish them, make them increase in population,\fnote{The Heb. lacks \fbib{in population}} and will place my sanctuary in their midst forever. \v{27}I will pitch my tent among them and will be their God. They will be my people, \v{28}and the nations will learn that I, the \divine{Lord}, am the sanctifier of Israel when I place my sanctuary in their midst forever.''\,'\,''
\labelchapt{38}
\passage{The Prophecy against Gog}

\chapt{38}
\v{1}This message from the \divine{Lord} came to me: \v{2}``Son of Man, turn your attention toward Gog,\fnote{I.e. a mountain tribe north of Assyria, and so through chapter 39} from the land of Magog,\fnote{I.e. a son of Noah's son Japheth; the area includes what is now modern eastern Turkey} leader of the head\fnote{Or \fbib{of Rosh,}} of Meshech,\fnote{I.e. a son of Noah's son Japheth; this people resided in what is now modern Armenia} and of Tubal.\fnote{I.e. a son of Noah's son Japheth; the area includes what is now modern eastern Turkey} Prophesy this against him: \v{3}`This is what the Lord \divine{God} says: ``Watch out! I'm coming after you, Gog, leader of the head\fnote{Or \fbib{of Rosh,}} of Meshech,\fnote{I.e. a son of Noah's son Japheth; this people resided in what is now modern Armenia} and of Tubal.\fnote{I.e. a son of Noah's son Japheth; this people resided in what is now modern eastern Turkey} \v{4}I'm going to turn you around, put hooks into your jaws, and bring you out---you and your whole army---along with your horses and cavalry riders, all of them richly attired, a magnificent company replete with buckler and shield, and all of them wielding battle swords. \v{5}Persia,\fnote{I.e. the area includes what is now modern Iran} Cush,\fnote{I.e. this area includes what is now modern Ethiopia and Somalia} and Libya\fnote{Lit. \fbib{Put}; the Heb. name means \fbib{bow}} will be accompanying them, all of them equipped with shields and helmets. \v{6}Gomer\fnote{I.e. a son of Noah's son Jepheth; the area encompasses what is now modern Turkey, Iran, Afghanistan, and Iraq.} with all its troops, and the household of Togarmah\fnote{I.e. named after Gomer, the region encompasses what is now Armenia} from the remotest parts of the north with all its troops---many people will accompany you. \v{7}Be prepared. Yes, prepare yourself---you and all of your many battalions that have gathered together around you to protect you.

\v{8}`````Many days from now---in the latter years---you will be summoned to a land that has been restored from violence.\fnote{Lit. \fbib{from the sword}} You will be gathered from many nations to the mountains of Israel, which formerly had been a continuous waste, but which will be populated with people who have been brought back from the nations. All of them will be living there securely. \v{9}You'll arise suddenly, like a tornado, coming like a windstorm\fnote{Or \fbib{cloud}} to cover the land, you and all your soldiers with you, along with many nations.''\,'\,''
\passage{The Invasion Strategy}

\v{10}``This is what the Lord \divine{God} says: `This is what's going to happen on the very day that you begin your invasion: You'll be thinking,\fnote{Lit. \fbib{thinking in your heart}} making evil plans, \v{11}and boasting, ``I'm going to invade a land comprised of open country\fnote{I.e. territory characterized by a lack of military fortifications} that is at rest, its people\fnote{The Heb. lacks \fbib{its people}} living confidently, all of whose inhabitants will be living securely, with neither fortification nor bars on their doors. \v{12}I'm going to confiscate anything I can put my hands on. I'll attack the restored ruins and the people who have been gathered together from the nations, who are acquiring livestock and other goods, and who live at the center of the world's attention.''\fnote{The Heb. lacks `\fbib{s attention}} \v{13}`Businessmen based in Sheba,\fnote{I.e. what is now southwest Saudi Arabia} Dedan,\fnote{I.e. what is now southern Saudi Arabia} Tarshish,\fnote{I.e. a city accessible from the Red Sea to which ships based on the Elanitic Gulf could sail} and all of its growling lions will ask you, ``Are you coming for war spoils? Have you assembled your armies to carry off silver and gold, and to gather lots of war booty?''\,'\,''
\passage{God's Rebuke to Gog}

\v{14}``Therefore, Son of Man, prophesy to Gog and tell him, `This is what the Lord \divine{God} says: ``When the day comes when my people are living securely, won't you be aware of it? \v{15}You'll come in from your home\fnote{Lit. \fbib{place}} in the remotest parts of the north. You'll come with many nations, all of them riding along on horses. You'll be a huge, combined army. \v{16}You'll come up to invade my people Israel like a storm cloud to cover the land. In the last days, Gog, I'll bring you up to invade my land so that the world will learn to know me when I show them how holy I am before their very eyes.''\,'\,''
\passage{A Prediction for the Distant Future}

\v{17}``This is what the Lord \divine{God} says: `Surely you're the one about whom I spoke years ago in the writings\fnote{Lit. \fbib{ago by the hand}} of my servants, Israel's prophets, aren't you? They predicted back then that I would bring you up after many years, didn't they? \v{18}So it will be that on that day, when Gog\fnote{I.e. a mountain tribe north of Assyria} invades the land of Israel,' declares the Lord \divine{God}, `my zeal will ignite my anger. \v{19}Because of my zeal and burning anger, at that time\fnote{Lit. \fbib{day}} there will be a massive earthquake throughout the land of Israel. \v{20}I'm going to shake the fish in the sea, the birds in the sky, the wild beasts, all the creatures that crawl on the earth, and every single human being who lives on the surface of the earth. Mountains will collapse, as will their mountain passages, and every wall will fall to the ground. \v{21}Then I'll call for war against Gog\fnote{I.e. a mountain tribe north of Assyria} on top of every mountain,' declares the Lord \divine{God}, `and every weapon of war will be turned against their fellow soldier. \v{22}I'll judge them with disease and bloodshed. I'll shower him, his soldiers, and the vast army that accompanies him with a torrential flood, hailstones, fire, and sulfur. \v{23}I will exalt myself and demonstrate my holiness, making myself known to many people, who will learn that I am the \divine{Lord}.'\,''
\labelchapt{39}
\passage{The Destruction of Gog}

\chapt{39}
\v{1}``Now as for you, Son of Man, prophesy against Gog\fnote{I.e. a mountain tribe north of Assyria} and tell him, `This is what the Lord \divine{God} has to say: ``Watch out, Gog, you leader of the head\fnote{Or \fbib{of Rosh,}} of Meshech and of Tubal! \v{2}I'm going to turn you around, drag you along to your destruction,\fnote{So LXX; MT reads \fbib{around, lead you}} and bring you up from the farthest parts of the north, and carry you to the mountains of Israel. \v{3}There I will strike your bow from your left hand and your arrows from your right, causing your fall. \v{4}You will collapse on the mountains of Israel, along with all of your soldiers and the nations that have accompanied you. There the raptors, vultures,\fnote{Or \fbib{carrion feeders}} and wild animals will feed on you. \v{5}You will fall dead in the open fields, because I have ordered this to happen,' declares the Lord \divine{God}. \v{6}`I'm also going to incinerate Magog, along with those who are settled down and at home in the islands. That's when they'll learn that I am the \divine{Lord}. \v{7}I'll make my holiness and reputation\fnote{Lit. \fbib{name}} known in the midst of my people Israel, and I won't let my holiness be profaned anymore. The nations will learn that I, the \divine{Lord}, am holy in the midst of Israel. \v{8}Pay attention! It's coming and will certainly happen,' declares the Lord \divine{God}. `This will be the day about which I've spoken!'\,''\,'\,''
\passage{The Aftermath of the Battle}

\v{9}``After all this happens, the people who live in the cities of Israel will be kindling fires for seven years, using small shields, large shields, bows, arrows, clubs, personal weapons, and spears to do so. \v{10}They won't need to cut down trees from the fields nor gather firewood from the forests, because they will light fires with the weapons as they plunder the plunderers and loot the looters!'' declares the Lord \divine{God}. \v{11}``When all of this happens, I'm going to set aside a grave site for Gog in Israel's Traveler's Valley,\fnote{Lit. \fbib{in the Crossover Valley}; or \fbib{Israel, the valley where people cross}} near the approach\fnote{I.e. to the north, as one travels from Jerusalem} to the Dead Sea. She\fnote{MT does not identify the woman} will block off everyone who tries to bypass it. There they will bury Gog, and rename the area `Valley of Gog's Gang'.\fnote{Lit. \fbib{Gog's Crowd'} and so in v. 15} \v{12}The house of Israel will be burying them for seven months in order to purify the land. \v{13}Everyone in the land will be involved in the burials, and this will serve as a reminder for them that I have glorified myself,'' declares the Lord \divine{God}. \v{14}``Men will be assigned to travel continuously throughout the land, exploring for seven full months as they go about burying the bodies that remain from the battle\fnote{The Heb. lacks \fbib{from the battle}} on the surface of the ground, so that the land may be sterilized. \v{15}As scouts go searching throughout the land, whenever they see someone's bones, they will place a sign beside the remains until the remains have been buried in the Valley of Gog's Gang. \v{16}They'll also name the city that is there `Hamonah,'\fnote{The Heb. name means \fbib{The Crowds}} as they purify the land.''
\passage{An Invitation to Dine on Human Flesh}

\v{17}``Now as for you, Son of Man, this is what the Lord \divine{God} has to say: `Tell all of the birds and wild beasts, ``Come! Gather together and participate in the sacrifice that I'm going to make for you. This great sacrifice will take place on the mountains of Israel, where you'll be eating flesh and drinking blood. \v{18}You'll eat the flesh of mighty men and drink the blood of the world's princes, drinking the blood of these rams, lambs, goats, bulls, all of them fattened as if they're from Bashan, fit for slaughter! \v{19}You'll eat until you're fat and satiated. You'll drink blood until you're drunk from the sacrifice that I'm going to make for all of you. \v{20}You'll be fully satiated at my table, dining on\fnote{The Heb. lacks \fbib{dining on}} horse flesh, horsemen, elite soldiers, and every kind of warrior,'' declares the Lord \divine{God}. \v{21}`I'm going to display my glory among the people, and every nation will see the judgment that I administer by my own hand among them. \v{22}The house of Israel will learn that I am the \divine{Lord} their God from that day forward! \v{23}The nations will also learn that because of Israel's sin the house of Israel went into captivity, since they were unfaithful in their behavior toward me. As a result, I hid my presence from them, turned them over to the control of their enemies, and they died by violence.\fnote{Lit. \fbib{they fell by the sword}} \v{24}It was because of their defilement and transgression that I treated them this way by hiding my presence from them.'\,''
\passage{The Final Restoration of Israel}

\v{25}``Therefore this is what the Lord \divine{God} has to say: `I'm going to restore the resources of Jacob and show mercy to the entire house of Israel. I'll be zealous for my own reputation\fnote{Lit. \fbib{name}} and for my holiness. \v{26}They'll forget their shame and all of their unfaithfulness by which they behaved so unfaithfully toward me. They will live on their land in confidence, not in fear. \v{27}When I bring them back from the nations and gather them together from the lands that belong to their enemies, I will demonstrate my holiness through them right in front of the eyes of the world,\fnote{Lit. \fbib{of many peoples}} \v{28}and they will learn that I am the \divine{Lord} their God, who sent them into exile and who gathered them back to their land. I will not leave any of them remaining in exile. \v{29}I'll no longer hide my presence from them again when I pour out my Spirit on the house of Israel,' declares the Lord \divine{God}.''
\labelchapt{40}
\passage{The Vision of Jerusalem}

\chapt{40}
\v{1}At the beginning of year 25 of our captivity, on the tenth day of the fourteenth year after the destruction of Jerusalem\fnote{Lit. \fbib{of the city}}---on that very day---the \divine{Lord} grabbed me in his hand and took me there. \v{2}God brought me in a series of visions to the land of Israel and placed me on top of a very high mountain, where to the south there was something that looked like the outline of a city. \v{3}That's where he took me. All of a sudden, there was a man whose appearance resembled glowing\fnote{The Heb. lacks \fbib{glowing}} bronze! He had a measuring reed and line in his hand as he stood in the city gate. \v{4}This is what the man told me: ``Son of Man, watch carefully,\fnote{Lit. \fbib{watch with your own eyes}} listen closely,\fnote{Lit. \fbib{listen with your own ears}} and remember\fnote{Lit. \fbib{and put in your heart}} everything I'm going to be showing you, because you've been brought here to be shown what you're about to see. Be sure that you tell the house of Israel everything that you observe.''
\passage{Measuring the Temple Grounds}

\v{5}All of a sudden, we were at the exterior wall that completely surrounded the Temple. The man whom I had observed held a measuring reed that was six cubits\fnote{I.e. about 10.5 feet, given the designated measurement in royal cubits, about 21 inches} long as measured in cubits that were a cubit and a handbreadth\fnote{I.e. the royal cubit, which measured about 21 inches} long. As he measured the thickness of the wall, he measured out one reed.\fnote{I.e. about 10.5 feet; the reed was six royal cubits} Its height was also one reed.\fnote{I.e. about 10.5 feet, ; the reed was six royal cubits} \v{6}Then he went over to the gate that faced toward the east, ascended its steps, and measured its thresholds. One threshold measured one reed\fnote{I.e. about 10.5 feet; the reed was six royal cubits} and the other one measured one reed.\fnote{I.e. about 10.5 feet; the reed was six royal cubits} \v{7}Each guardhouse\fnote{Or \fbib{alcove}; and so throughout the chapter} measured one reed\fnote{I.e. about 10.5 feet; the reed was six royal cubits} long and one reed\fnote{I.e. about 10.5 feet; the reed was six royal cubits} wide, and the distance\fnote{The Heb. lacks \fbib{the distance}} between each guardhouse was five cubits.\fnote{I.e. about 8.75 feet, given the designated measurement in royal cubits, about 21 inches} The threshold of the gate near the vestibule facing away from the Temple entrance\fnote{The Heb. lacks \fbib{entrance}} measured one reed.\fnote{I.e. about 10.5 feet; the reed was six royal cubits}

\v{8}Next, he measured the vestibule of the gate facing away from the Temple entrance at one reed.\fnote{I.e. about 10.5 feet; the reed was six royal cubits} \v{9}He measured the vestibule of the gate inside at eight cubits\fnote{I.e. about 14 feet; the royal cubit was 21 inches} and the doorjambs at two cubits.\fnote{I.e. about 42 inches; the royal cubit was 21 inches} (The vestibule at the gate faced away from the Temple.) \v{10}Gate guardhouses stood facing east, numbering three on each side,\fnote{Lit. \fbib{three from here and there}} each of them of equal size\fnote{Lit. \fbib{from here one measurement}} to the door jamb; that is, having the same\fnote{The Heb. lacks \fbib{that is, having the same}} measurement on each side.\fnote{Lit. \fbib{measurement from here and from here}} \v{11}He measured the width of the gateway at ten cubits,\fnote{I.e. about 17.5 feet; the royal cubit was 21 inches} and the length of the gate at thirteen cubits.\fnote{I.e. about 22.75 feet; the royal cubit was 21 inches}

\v{12}The retaining\fnote{Lit. \fbib{border}; or \fbib{barrier}} wall in front of the guardhouses measured one cubit\fnote{I.e. about 21 inches; the royal cubit was 21 inches} wide. It stood one cubit\fnote{I.e. about 21 inches; the royal cubit was 21 inches} from the wall to the guardhouses, which were six cubits\fnote{I.e. about 10.5 feet; the royal cubit was 21 inches} square.\fnote{Lit. \fbib{were six cubits from here and six cubits from here.}} \v{13}He measured the gate from the roof of the guardhouses to the roof of another\fnote{The Heb. lacks \fbib{of another}} at 25 cubits\fnote{I.e. about 43.75 feet; the royal cubit was 21 inches} from doorway to opposite doorway. \v{14}Then he measured\fnote{Lit. \fbib{made}} the open air porch\fnote{So LXX; MT reads \fbib{the jamb}} at 60 cubits\fnote{I.e. about 106.75 feet; the royal cubit was 21 inches} from the doorjamb of the courtyard that encompassed the gate. \v{15}The distance from the front entrance gate to the vestibule of the inner gate measured 50 cubits.\fnote{I.e. about 87.5 feet; the royal cubit was 21 inches} \v{16}Latticed windows faced the guardhouses, their side pillars within the gate all around, and also for the porches. Windows were placed all around inside, and the side pillars were engraved with palm trees.
\passage{The Outer Court}

\v{17}Next, he brought me into the outer court, where chambers and a paved area had been constructed all around the courtyard, with 30 chambers facing the pavement. \v{18}The pavement to the side\fnote{Or \fbib{The lower pavement}} of the gates corresponded to the length of the gates. \v{19}He also measured the width from the front lower gate to the front of the exterior inner court at 100 cubits\fnote{I.e. about 175 feet; the royal cubit was 21 inches} to the east and to the north.
\passage{The North Facing Outer Court}

\v{20}Next, he measured the length and width of the outer north-facing gate to the courtyard. \v{21}It was equipped\fnote{The Heb. lacks \fbib{was equipped}} with three guardhouses on each side. Its side pillars and porches had measurements identical to the first gate: 50 cubits\fnote{I.e. about 87.5 feet; the royal cubit was 21 inches} long and 25 cubits\fnote{I.e. about 43.75 feet; the royal cubit was 21 inches} wide. \v{22}Its windows, porches, and palm tree ornaments had measurements identical to the east-facing gate. Reached by seven ascending\fnote{The Heb. lacks \fbib{ascending}} steps, its porch lay\fnote{The Heb. lacks \fbib{lay}} to the front of the steps. \v{23}From a gate that stood opposite the northern gate he measured 100 cubits,\fnote{I.e. about 175 feet; the royal cubit was 21 inches} as well as from the eastern gate.
\passage{The South Facing Gate}

\v{24}Then he led me toward the south, where there was a gate with side pillar and porch measurements identical to the others. \v{25}The gate and its porches contained windows all around, identical to the other windows. The length of the porch\fnote{The Heb. lacks \fbib{of the porch}} was 50 cubits\fnote{I.e. about 87.5 feet; the royal cubit was 21 inches} and its width was 25 cubits.\fnote{I.e. about 43.75 feet; the royal cubit was 21 inches} \v{26}Seven steps led up to it, with a porch in front of them. Palm tree ornaments were engraved on its side pillars, one on each side. \v{27}The inner court contained a south-facing gate measuring 100 cubits\fnote{I.e. about 175 feet; the royal cubit was 21 inches} from gate to gate toward the south.
\passage{The Inner Southern Court}

\v{28}Next, he brought me to the inner courtyard by way of the south-facing gate. He measured the south-facing gate as having measurements identical to the others. \v{29}The measurements of its guardhouses, its side pillars, and its porches were identical to the others. The gate and its porches contained windows all around. The length of the porch\fnote{The Heb. lacks \fbib{of the porch}} was 50 cubits\fnote{I.e. about 87.5 feet; the royal cubit was 21 inches} and its width was 25 cubits.\fnote{I.e. about 43.75 feet; the royal cubit was 21 inches} \v{30}Porches lay all around, measuring 25 cubits\fnote{I.e. about 43.75 feet; the royal cubit was 21 inches} long and five cubits\fnote{I.e. about 8.75 feet; the royal cubit was 21 inches} wide, \v{31}leading to the outer courtyard. Palm tree ornaments were engraved on its side pillars. The stairway leading to it contained eight steps.
\passage{The Inner Eastern Court}

\v{32}Then he brought me into the inner east-facing courtyard, where he measured the gate, identical to the others. \v{33}The measurement of its guardhouses, side pillars, and porches was identical to the others. The gate and its porches contained windows all around. The length of the porch\fnote{The Heb. lacks \fbib{of the porch}} was 50 cubits\fnote{I.e. about 87.5 feet; the royal cubit was 21 inches} and its width was 25 cubits,\fnote{I.e. about 43.75 feet; the royal cubit was 21 inches} \v{34}leading to the outer courtyard. Palm tree ornaments were engraved on its side pillars. The stairway leading to it contained eight steps.
\passage{The North Facing Gate}

\v{35}Next, he brought me to the north-facing gate, where he measured the gate, identical to the others. \v{36}The measurement of its guardhouses, side pillars, and porches was identical to the others. The gate and its porches contained windows all around. The length of the porch\fnote{The Heb. lacks \fbib{of the porch}} was 50 cubits\fnote{I.e. about 87.5 feet; the royal cubit was 21 inches} and its width was 25 cubits,\fnote{I.e. about 43.75 feet; the royal cubit was 21 inches} \v{37}leading to the outer courtyard. Palm tree ornaments were engraved on its side pillars. The stairway leading to it contained eight steps. \v{38}There was a chamber with a doorway by the side pillars next to the gate where they prepare\fnote{Lit. \fbib{rinse}} the burnt offerings.

\v{39}In the porch leading in front of the gate there were two tables on either side for slaughtering burnt offerings, sin offerings, and guilt offerings, \v{40}and on the outer side, approaching the northern gateway, there were two tables, as well as two tables on the opposite side of the porch in front of the gate. \v{41}In that way, there were four tables on each side in front of the gate, for a total of eight tables for use in slaughtering the offerings.\fnote{The Heb. lacks \fbib{the offerings}}

\v{42}There were four tables carved from stone for the burnt offering, each one and a half cubits\fnote{I.e. about 31.5 inches; the royal cubit was 21 inches} long, one and a half cubits\fnote{I.e. about 31.5 inches; the royal cubit was 21 inches} wide, and one cubit\fnote{I.e. about 21 inches; the royal cubit was 21 inches} high, on which the instruments are laid for slaughtering burnt offerings and sacrifices. \v{43}Double hooks, a single handbreadth\fnote{I.e. about 3 inches} in length, were installed all around in this portion of\fnote{The Heb. lacks \fbib{this portion of}} the temple area.
\passage{The Inner Gate}

\v{44}From outside leading into the inner gate there were chambers for the choir. One was beside the north gate facing the south, and another was at the south gate facing the north. \v{45}The angel\fnote{Lit. \fbib{He}} told me, ``This south-facing chamber is for the priests who maintain the Temple, \v{46}and the north-facing chamber is for the priests who maintain the altar. These are Zadok's descendants, who, as descendants of Levi approach the \divine{Lord} to minister directly to him.'' \v{47}He measured the court in the form of a square at 100 cubits\fnote{I.e. about 175 feet; the royal cubit was 21 inches} long and 100 cubits\fnote{I.e. about 175 feet; the royal cubit was 21 inches} wide. The altar stood in front of the Temple.
\passage{The Temple Porch}

\v{48}Next, he brought me to the Temple porch and measured the side pillars at five cubits\fnote{I.e. about 8.75 feet; the royal cubit was 21 inches} on each side. The width of the gate measured three cubits\fnote{I.e. about 5.25 feet; the royal cubit was 21 inches} on each side. \v{49}The porch was 20 cubits\fnote{I.e. about 35 feet; the royal cubit was 21 inches} long and eleven cubits\fnote{I.e. about 19.25 feet; the royal cubit was 21 inches} wide. The stairway by which it was ascended was equipped with columns attached to its side pillars, one on each side.
\labelchapt{41}
\passage{The Vision of the Temple}

\chapt{41}
\v{1}Next he brought me to the Temple and measured its door jambs at six cubits\fnote{I.e. about 10.5 feet; the royal cubit was 21 inches} wide on each side of the structure.\fnote{Lit. \fbib{tent}} \v{2}The entrance was ten cubits\fnote{I.e. about 17.5 feet; the royal cubit was 21 inches} wide and its door jambs were five cubits\fnote{I.e. about 8.75 feet; the royal cubit was 21 inches} wide on each side. He measured the length of the nave at 40 cubits\fnote{I.e. about 35 feet; the royal cubit was 21 inches} and its width at 20 cubits.\fnote{I.e. about 70 feet; the royal cubit was 21 inches}

\v{3}Then he went inside and measured the door jambs at two cubits\fnote{I.e. about 42 inches; the royal cubit was 21 inches} wide and the doorway at six cubits\fnote{I.e. about 10.5 feet; the royal cubit was 21 inches} high. The doorway was seven cubits\fnote{I.e. about 12.25 feet; the royal cubit was 21 inches} wide. \v{4}He measured its length at 20 cubits,\fnote{I.e. about 35 feet; the royal cubit was 21 inches} its width at 20 cubits\fnote{I.e. about 35 feet; the royal cubit was 21 inches} in front of the structure,\fnote{I.e. the separation between the Holy Place and the most holy area} and then he told me, ``This is the most holy area.''

\v{5}Next, he measured the Temple walls at six cubits\fnote{I.e. about 10.5 feet; the royal cubit was 21 inches} high and the width of the side chambers at four cubits\fnote{I.e. about seven feet; the royal cubit was 21 inches} around all four sides of the Temple. \v{6}The side chambers consisted of three stories, each above the other, with 30 chambers in each story. The side chambers extended out from the wall that faced the inside of the chambers where the chambers were fastened together, but the chamber walls were not fastened directly into the Temple walls themselves. \v{7}The side chambers surrounding the Temple were wider at each successive story, because the surrounding structure ascended by proportional increments as it rose, ascending to the highest story by going up successively from the lowest.

\v{8}I observed a raised platform that surrounded the Temple, and the foundations of the side chambers were a full six cubits\fnote{I.e. about 10.5 feet; the royal cubit was 21 inches} deep. \v{9}The outer wall of the side chambers was five cubits\fnote{I.e. about 8.75 feet; the royal cubit was 21 inches} thick, and there was an empty space between the Temple's side chambers \v{10}and its outer chambers 20 cubits\fnote{I.e. about 35 feet; the royal cubit was 21 inches} in width, surrounding the Temple on each side. \v{11}The side chamber doorway facing the free space contained a single north-facing doorway and a second south-facing doorway. The width of the free space was five cubits\fnote{I.e. about 8.75 feet; the royal cubit was 21 inches} all around the perimeter.\fnote{The Heb. lacks \fbib{the perimeter}} \v{12}The building that faced the west side of the courtyard was 70 cubits\fnote{I.e. about 122.5 feet; the royal cubit was 21 inches} wide, and the building's wall was five cubits\fnote{I.e. about 8.75 feet; the royal cubit was 21 inches} thick all around. It was 90 cubits\fnote{I.e. about 157.5 feet; the royal cubit was 21 inches} long.
\passage{The Temple}

\v{13}Then he measured the Temple. It was 100 cubits\fnote{I.e. about 175 feet; the royal cubit was 21 inches} long, and the courtyard, its building, and its walls were 100 cubits\fnote{I.e. about 175 feet; the royal cubit was 21 inches} long. \v{14}The front of the Temple and its east-facing courtyard were each\fnote{The Heb. lacks \fbib{each}} 100 cubits\fnote{I.e. about 175 feet; the royal cubit was 21 inches} long. \v{15}Next, he measured 100 cubits\fnote{I.e. about 175 feet; the royal cubit was 21 inches} as the length of the structure toward the front of the courtyard that stood behind it, where it housed a gallery on each side of it. Then he measured the Temple and the inner porticos\fnote{Or \fbib{porches}} of the courtyard, \v{16}the thresholds, the shielded\fnote{Or \fbib{latticed}} windows, and the surrounding three-storied galleries that stood opposite. From the ground to the shielded\fnote{Or \fbib{latticed}} windows, they were paneled with wood all around, \v{17}including up to the doorway, up to the Temple (both within and without) and all around both sides of the inner wall, according to his measurement. \v{18}There were carved cherubim and palm trees, alternating with a palm tree between a cherub, and each cherub had two faces, \v{19}with a human face looking\fnote{The Heb. lacks \fbib{looking}} toward the palm tree on one side and a young lion's face looking\fnote{The Heb. lacks \fbib{looking}} toward the palm tree on the other side. These carvings extended all the way around the Temple, \v{20}from the ground to above the doorway, as well as on the walls of the main sanctuary.

\v{21}The door posts of the main sanctuary were square. Each door post was identical in appearance to the others. \v{22}The altar was made of wood, three cubits\fnote{I.e. about 5.25 feet; the royal cubit was 21 inches} high and two cubits\fnote{I.e. about 42 inches; the royal cubit was 21 inches} long. Its corners, base, and sides were of wood. He told me, ``This table stands in the \divine{Lord}'s presence.''

\v{23}The nave and the sanctuary each were equipped with double doors. \v{24}Each door had two sections mounted on hinges,\fnote{Lit. \fbib{two swinging sections}} for a total of two sections for one door and two sections for the other. \v{25}The doors of the nave had carvings engraved on them, consisting of cherubim and palm trees identical to those on the walls. The front of the exterior porch was equipped with a wooden threshold. \v{26}Shielded windows and palm trees were visible\fnote{The Heb. lacks \fbib{visible}} on both sides; that is, on the sides of the porch, the side chambers of the Temple, and on its thresholds.
\labelchapt{42}
\passage{The Vision of the Outer Court}

\chapt{42}
\v{1}Then he brought me to the outer, north-facing courtyard into the chamber that stood opposite the structure that was facing north. \v{2}It stood 100 cubits\fnote{I.e. about 175 feet; the royal cubit was 21 inches} long and 50 cubits\fnote{I.e. about 87.5 feet; the royal cubit was 21 inches} wide, with a door in the middle.\fnote{Lit. \fbib{north}} \v{3}Opposite the 20 cubits\fnote{I.e. about 35 feet; the royal cubit was 21 inches} wide inner court, and opposite the paved area that comprised the outer court, there were three stories of galleries that faced each other. \v{4}In front of the chambers there was an inner walkway ten cubits wide and 100 cubits\fnote{So with LXX Syr; MT reads \fbib{and one cubit}} wide, the openings to which were on the\fnote{Or \fbib{which faced}} north. \v{5}The upper chambers were narrower, since the galleries required more space than did the lower and middle portions of the building. \v{6}The three part structure had no columns, unlike the courts, which is why the upper chambers were offset from the ground upward, more so than the lower and middle chambers.

\v{7}The outer wall by the side of the chambers toward the outer court and facing the chambers was 50 cubits\fnote{I.e. about 87.5 feet; the royal cubit was 21 inches} long. \v{8}While the chambers in the outer court were 50 cubits\fnote{I.e. about 87.5 feet; the royal cubit was 21 inches} in length, the chambers facing the Temple were 100 cubits\fnote{I.e. about 175 feet; the royal cubit was 21 inches} long. \v{9}Below these chambers, as one might enter from the outer court, was the east side entrance. \v{10}There were chambers built into the thick part of the wall of the court facing the east; that is, facing the separate area toward the front of the building, \v{11}with a passageway in front of them, similar in appearance to the chambers that were on the north, proportional to their length and width, with all of their exits according to their arrangements and doorways. \v{12}Corresponding to the chamber doorways facing the south was an opening at the beginning of the passage; that is, the passage in front of the corresponding part of the wall facing east as one might enter.
\passage{The Place for Holy Things}

\v{13}Then he told me, ``The north and south chamber, which are opposite the courtyard, are consecrated areas where the priests who approach the \divine{Lord} will eat consecrated offerings and lay the consecrated grain offerings, sin offerings, and guilt offerings, because the area is holy. \v{14}When the priest enters, they will not enter the outer court from the sanctuary without having removed their garments worn during their time of ministry, because they are holy. They will put on different clothes, and then they will approach the area reserved for the people.''

\v{15}After he had finished measuring the inner temple, he brought me out through the east-facing gate and measured it all around. \v{16}He measured the east side at 500 reeds,\fnote{I.e. about one mile, and so through vs. 20} according to the length of the measuring stick, \v{17}the north side at 500 reeds, according to the length of the measuring stick, \v{18}the south side at 500 reeds, according to the length of the measuring stick, \v{19}and the west side at 500 reeds, according to the length of the measuring stick. \v{20}He measured a wall that encompassed all four sides, 500 hundred long and 500 wide, dividing between the sacred and common areas.
\labelchapt{43}
\passage{The Vision of the Gates}

\chapt{43}
\v{1}Next, he brought me to the east-facing gate, \v{2}and the glory of the God of Israel was coming from the east. His voice sounded like roaring\fnote{Lit. \fbib{many}} water, and the land shimmered from his glory. \v{3}His appearance in the vision that I was having was similar to what I observed in the vision where he had come to destroy the city, and also like the visions that I saw by the Chebar River. I fell flat on my face \v{4}while the glory of the \divine{Lord} entered the Temple through the east-facing gate. \v{5}Just then, the Spirit lifted me up and carried me into the inner courtyard, where the glory of the \divine{Lord} was filling the Temple! \v{6}I heard someone speaking to me from the Temple, and a man appeared, standing beside me!
\passage{God to Live among His People}

\v{7}``Son of Man,'' the Lord \divine{God} told\fnote{Lit. \fbib{he told}} me, ``This is where my throne is, where I place the soles of my feet, and where I will live among the Israelis forever. The house of Israel will no longer defile my holy name---neither they nor their kings---by their unfaithfulness, by the lifeless idols\fnote{Lit. \fbib{the corpses}} of their kings on their funeral mounds,\fnote{Or \fbib{their high places}} \v{8}by their setting up their threshold too close to my threshold and their door post too close to my door post, with a wall between me and them. After all, they have defiled my holy name by the loathsome things that they did, so I devoured them in my anger. \v{9}But now let them send their unfaithfulness---that is, the lifeless idols\fnote{Lit. \fbib{the corpses}} of their kings---far away from me, and I will live among them forever.''
\passage{Ezekiel Describes the Temple}

\v{10}``And now, Son of Man, describe the Temple to the house of Israel. They will be ashamed because of their sin. They will measure its pattern. \v{11}If they are ashamed of everything that they've done, you are to reveal to them the design of the Temple, its structure, its exits and entrances, its plans, its ordinances, and all of its regulations. Write it down where they can see it and remember all of its designs and regulations, so they will implement them. \v{12}This is to be the regulation for the Temple: the entire area on top of the mountain is to be considered wholly consecrated. This is to be the law of the Temple.''
\passage{The Altar}

\v{13}``Here are the measurements of the altar in cubits that were a cubit and a handbreadth\fnote{I.e. the royal cubit, which measured about 21 inches} long: its base is a cubit\fnote{I.e. about 21 inches; the royal cubit was about 21 inches} long and a cubit\fnote{I.e. about 21 inches; the royal cubit was about 21 inches} wide, and its border around the edge at one handbreadth\fnote{I.e. about three inches} is to be the height of the altar. \v{14}From the base on the ground to its lower edge is to be two cubits,\fnote{I.e. about 3.5 feet; the royal cubit was about 21 inches} with its width to be one cubit.\fnote{I.e. about 21 inches; the royal cubit was about 21 inches} From the lesser ledge to the larger edge is to be four cubits.\fnote{I.e. about seven feet; the royal cubit was about 21 inches} Its width is to be one cubit.\fnote{I.e. about 21 inches; the royal cubit was about 21 inches} \v{15}The hearth is to be four cubits high,\fnote{The Heb. lacks \fbib{high}} and four horns are to extend upwards from the hearth. \v{16}The hearth is to be twelve cubits\fnote{The Heb. lacks \fbib{cubits}} long and twelve cubits\fnote{The Heb. lacks \fbib{cubits}} wide; that is, it will be a four-sided square. \v{17}It is to have a ledge fourteen cubits\fnote{The Heb. lacks \fbib{cubits}} long by fourteen cubits\fnote{The Heb. lacks \fbib{cubits}} wide around the four sides. Its border is to be half a cubit\fnote{I.e. about 10.5 inches; the royal cubit was about 21 inches} and its base is to be a cubit\fnote{I.e. about 21 inches; the royal cubit was about 21 inches} all around, with its steps facing east.''
\passage{The Offerings}

\v{18}Then he told me, ``This is what the Lord \divine{God} says: `These are the regulations for the altar, starting the day that it is constructed for presenting burnt offerings and sprinkling blood. \v{19}You are to present to the Levitical priests, Zadok's descendants, who will approach me to serve me, a young bull for a sin offering,' declares the Lord \divine{God}. \v{20}You are to take some of its blood and put it on the four horns of the altar,\fnote{The Heb. lacks \fbib{of the altar}} on the four corners of its ledge, and on the border that surrounds it, thus cleansing it and making an atonement for it. \v{21}You are also to present a bull for a sin offering, incinerating it in the appointed place at the Temple, outside the sanctuary.

\v{22}`The second day following commencement of offerings,\fnote{The Heb. lacks \fbib{following commencement of offerings}} you are to offer a male goat without defect for a sin offering to cleanse the altar the same way they cleansed it with the bull. \v{23}After you've finished the cleansing, you are to present a young bull without defect and a ram from the flock without defect. \v{24}You are to present them in the \divine{Lord}'s presence, and the priests are to throw salt on them and then present them as a burnt offering to the \divine{Lord}.

\v{25}`Every day for a week, you are to prepare a goat for a sin offering, a young bull, and a ram from the flock, each\fnote{The Heb. lacks \fbib{each}} without defect. \v{26}For a seven day period they are to make atonement for the altar, purifying it and consecrating it. \v{27}When they will have completed this period,\fnote{Lit. \fbib{completed these days}} starting the next day,\fnote{Lit. \fbib{period, from the eighth day following,}} the priests are to offer your burnt offerings on the altar, along with your peace offerings, and I will accept you,' declares the Lord \divine{God}.''
\labelchapt{44}
\passage{The Vision of the Outer Gates}

\chapt{44}
\v{1}Then the Lord \divine{God}\fnote{Lit. \fbib{Then he}} brought me back through the east-facing outer gate of the sanctuary. But it was shut. \v{2}The \divine{Lord} told me, ``This gate is to remain shut. It will not be opened. No man is to enter through it, because the \divine{Lord} God of Israel entered through it, so it is to remain shut. \v{3}The Regent\fnote{The Heb. lacks \fbib{Regent}; and so through chapter 48} Prince\fnote{I.e. a ruler who will govern with a king's authority in the name of one holding higher supremacy; and so through chapter 48} will be seated there,\fnote{Lit. \fbib{will sit in it}} as Regent Prince, and will dine in the \divine{Lord}'s presence, entering through the portico of the gate and exiting through it also.''
\passage{The Front of the Temple}

\v{4}Then he brought me through the north-facing gate to the front of the Temple. As I looked, the glory of the \divine{Lord} filled the \divine{Lord}'s Temple, and I fell flat on my face! \v{5}Then the \divine{Lord} told me, ``Son of Man, watch carefully,\fnote{Lit. \fbib{watch with your own eyes}} listen closely,\fnote{Lit. \fbib{listen with your own ears}} and remember\fnote{Lit. \fbib{and put in your heart}} everything I'm going to be telling you about all the statutes pertaining to the \divine{Lord}'s Temple and all of its laws. Pay careful attention to the entrance to the Temple, along with all of the exits to the sanctuary.''
\passage{A Rebuke to the Rebellious}

\v{6}``You are to tell the Resistance---that is, the house of Israel, `This is what the Lord \divine{God} says: ``I've had enough of all of your loathsome behavior, you house of Israel! \v{7}You kept on bringing in foreigners, those who were uncircumcised in heart and flesh, to profane my sanctuary by being inside my Temple, and by doing so you've emptied my covenant, all the while offering my food---the fat and the blood---in addition to all of the other loathsome things you've done.\fnote{The Heb. lacks \fbib{you've done}} \v{8}Furthermore, you haven't paid attention to the requirements for my holy things. Instead, you placed foreigners in charge of my sanctuary.''\,'

\v{9}``This is what the Lord \divine{God} says, `No foreigner who is both uncircumcised in heart and flesh, of all the foreigners who are among the Israelis is to enter my sanctuary. \v{10}But the descendants of Levi, who went far away from me when Israel abandoned me, who left me to follow their idols, are to bear the punishment of their iniquity. \v{11}Nevertheless, they are to serve in my sanctuary, overseeing the gates of the Temple, taking care of the Temple, slaughtering the burnt offerings and the sacrifices presented for the people, standing in the presence of the people, and ministering to them. \v{12}Because they kept serving them in the presence of their idols, becoming a sin-filled stumbling block to the house of Israel,' declares the Lord \divine{God}.

```I have sworn to them that they are to bear the consequences of their iniquity. \v{13}They are not to come near me to serve me as a priest, nor approach any of my holy things, including the most holy things. Instead, they are to bear the shame of the loathsome things that they have done. \v{14}Nevertheless, I will appoint them to take care of my Temple, including all of its service and everything that is to be done inside of it.'\,''
\passage{Levitical Ordinances}

\v{15}``The descendants of Zadok, Levitical priests who took care of my sanctuary when the Israelis wandered away from me, are to come near me to minister to me. They are to stand before me to offer the fat and the blood to me,'' declares the Lord \divine{God}. \v{16}``They are to enter my sanctuary, approach my table to minister to me, and carry out my requirements. \v{17}Whenever they enter at the gates of the inner court, they are to be clothed with linen garments. They are not to wear wool when they are ministering within the gates of the inner courtyard or in the Temple. \v{18}Linen turbans are to be on their heads, and they are to wear linen undergarments. Also, they are not to clothe themselves with anything that makes them perspire.

\v{19}``When they enter the outer courtyard, that is, the outer courtyard where the people are, they are to take off their garments in which they were ministering, lay them in the consecrated chambers, and put on different garments so they will not transfer\fnote{Or \fbib{transmit}} holiness to the people through their garments. \v{20}Also, they are not to shave their heads nor let their hair grow long. Instead, they are certainly to trim the hair on\fnote{The Heb. lacks \fbib{the hair on}} their heads. \v{21}None of the priests are to drink wine after entering the inner courtyard. \v{22}They are not to marry\fnote{Lit. \fbib{take}} a widow or a divorced woman. Instead, they are to marry\fnote{Lit. \fbib{take}} virgins from the descendants of the house of Israel, or a widow who is the widow of a priest.''
\passage{Duties of Ministry}

\v{23}``They are to teach my people how to discern\fnote{The Heb. lacks \fbib{how to discern}} what is holy in contrast to what is common, showing them how to discern between what is unclean and clean. \v{24}When disputes arise, they are to serve as a judge, adjudicating matters according to my ordinances. They are to enforce my laws, my statutes, all of my appointed festivals, and they are to sanctify my Sabbaths. \v{25}They are not to come in contact with a dead body, so they don't defile themselves, except in the case of their father, mother, son, daughter, brother, or for an unmarried sister, on whose behalf they may defile themselves. \v{26}After he is cleansed from that contact,\fnote{The Heb. lacks \fbib{from that contact}} he is to not to minister for seven days. \v{27}On the day that he returns to the sanctuary's inner court to minister, he is to offer his own sin offering,'' declares the Lord \divine{God}.
\passage{Ministerial Inheritances}

\v{28}``Now with respect to the priests'\fnote{Lit. \fbib{to their}} inheritances, I am to be their inheritance, and you are to give them no possession in Israel, since I am their possession. \v{29}They are to eat the grain offerings, sin offering, and guilt offering. Everything consecrated in Israel is to belong to them. \v{30}The first portion of all the first fruits of every kind and every offering of any kind is to be for the priests. You are to give the priest the first portion of your grain. As a result a blessing will rest on your household. \v{31}However, the priests are not to eat any bird or animal that has died a natural death or that has been torn apart.''
\labelchapt{45}
\passage{Israel's Future Temple Park}

\chapt{45}
\v{1}``When you divide the land for an inheritance, you are to present a Terumah\fnote{Lit. \fbib{Gift}; i.e. a special section of Israel's land to be dedicated to the \fbib{}\divine{Lord} as a national temple park; cf. Eze 48:8ff} to the \divine{Lord}, a consecrated portion of the land 25,000 cubits\fnote{The Heb. lacks \fbib{cubit}; if the unit of measurement is royal cubits, the length is about 8.29 miles.} long and 20,000\fnote{So LXX; MT reads \fbib{10,000}} cubits\fnote{LXX and MT lack \fbib{cubit}; if the unit of measurement is 20,000 royal cubits, the length is about 6.6 miles.} wide. Everything within this area is to be treated as holy. \v{2}A Holy Place is to be dedicated from this area in the form of a square measuring 500 by 500 cubits,\fnote{The Heb. lacks \fbib{cubits}} with a 50 cubit\fnote{I.e. about 87.5 feet; the royal cubit was about 21 inches} buffer zone\fnote{Lit. \fbib{cubit open space}} surrounding it. \v{3}From this area a measure is to be made 25,000 cubits\fnote{The Heb. lacks \fbib{cubits}; if the unit of measurement is royal cubits, the length is about 8.29 miles.} long and 10,000 cubits\fnote{The Heb. lacks \fbib{cubits}} wide, which is to contain the sanctuary, the holiest of holy objects. \v{4}It is to be a holy portion of the land, set aside\fnote{The Heb. lacks \fbib{set aside}} for the priests who serve the sanctuary, who approach the \divine{Lord} to serve him. It is to be a place for their houses, as well as the Holy Place of the sanctuary. \v{5}An area 25,000 cubits\fnote{The Heb. lacks \fbib{cubits}; if the unit of measurement is royal cubits, the length is about 8.29 miles.} long by 10,000 cubits wide is to be set aside\fnote{The Heb. lacks \fbib{to be set aside}} for use by the Levite\fnote{I.e. the ministry formerly held by the descendants of Levi} servants of the Temple, 20 parcels\fnote{Or \fbib{chambers}; so with MT; LXX reads \fbib{temple, cities}} for their residential properties. \v{6}The land allocation for the city is to be set at 5,000 cubits\fnote{The Heb. lacks \fbib{cubits}; if the unit of measurement is royal cubits, the length is about 1.66 miles.} wide and 25,000 cubits\fnote{The Heb. lacks \fbib{cubits}; if the unit of measurement is royal cubits, the length is about 8.29 miles.} long, adjacent to the sanctuary district, reserved for the entire house of Israel.''
\passage{The Portion for the Regent Prince}

\v{7}``The Regent Prince is to have a portion\fnote{The Heb. lacks \fbib{a portion}} on both sides of the consecrated allotment for the sanctuary and the city's land allotment, adjacent to both on the west\fnote{Lit. \fbib{the sea side facing the sea}} and the east sides, comparable in length to one of the portions from the west\fnote{Lit. \fbib{the sea side}} border to the east border. \v{8}This property in Israel is to belong to the Regent Prince,\fnote{Lit. \fbib{to him}} so my regent princes will no longer mistreat my nation. The remaining portion of the land is to be allotted to the house of Israel, that is, to its tribes.''
\passage{An Exhortation to Honest Business}

\v{9}``This is what the Lord \divine{God} says, `Enough of you, you regent princes of Israel! Abandon your violence and destruction. Practice what is just and right instead! Stop confiscating property from my people!' declares the Lord \divine{God}. \v{10}`You're to use an honest scale, an honest dry measure,\fnote{Lit. \fbib{honest ephah}} and an honest liquid measure!\fnote{Lit. \fbib{bath}} \v{11}The ephah and the bath are to be of equal volume; that is, the bath is to contain one tenth of an omer and the ephah one tenth of an omer. The omer is to be the standard on which their volume measurement is to be based. \v{12}The shekel\fnote{A shekel weighed about 0.4 ounces} is to weigh 20 gerahs. The mina\fnote{Or \fbib{maneh}; the Babylonian standard was equivalent to 1/60\textsuperscript{th} of a talent, with a talent weighing about 75 pounds} is to be comprised of three coins weighing\fnote{The Heb. lacks \fbib{comprised of three coins weighing}} 20, 25, and fifteen shekels, respectively.'\,''
\passage{Weight Standards for Offerings}

\v{13}``Here are the standards for presenting offerings: a sixth of an ephah that is based on the standard omer of wheat, and a sixth of an ephah based on the standard omer of barley. \v{14}The olive oil quota is to be based on the bath, measured at ten baths to each omer, which is equal to one kor. \v{15}The sheep quota is to be one from each flock of 200 taken from the pastures of Israel. From all of these you are to present grain offerings, burnt offerings, and peace offerings, to make atonement for them,'' declares the Lord \divine{God}.

\v{16}``The entire nation living in the land is to present this offering to the Regent Prince in Israel. \v{17}The Regent Prince is to provide the burnt offerings, grain offerings, and drink offerings at the festivals, on the New Moons and Sabbaths, and at all of the prescribed festivals of the house of Israel. He is to provide the grain offerings, burnt offerings, and peace offerings in order to make atonement for the house of Israel.''

\v{18}``This is what the Lord \divine{God} says, `On the first day of the first month, you are to present a young bull without defect in order to cleanse the sanctuary. \v{19}The priest is to place some of the blood from the sin offering on the door posts of the Temple, on the four corners of the ledge around the altar, and on the posts of the gate leading to the inner court. \v{20}You are also to do this on the seventh day of the month, to make atonement for any person who wanders away or who sins through ignorance in order to make atonement for the Temple.

\v{21}```On the fourteenth day of the first month, you are to observe the Passover as a festival for seven days. Unleavened bread is to be eaten. \v{22}On that day, the Regent Prince is to provide, both for himself and for all the people who live in the land, a bull for a sin offering. \v{23}Each day during the seven days of the festival, he is to provide a burnt offering to the \divine{Lord}, consisting of seven bulls and seven rams without defect, offered each day throughout the seven days, along with a male goat offered each day as a sin offering.

\v{24}```The Regent Prince\fnote{Lit. \fbib{He}} is also to present a grain offering consisting of an ephah with each bull and an ephah with each ram, along with a hin of olive oil mixed with an ephah of grain. \v{25}On the fifteenth day of the seventh month, during a seven day festival, the Regent Prince\fnote{Lit. \fbib{festival, he}} is to present these as daily sin offerings, burnt offerings, and grain offerings mixed with oil.'\,''
\labelchapt{46}
\passage{Regulations for the Inner Court}

\chapt{46}
\v{1}``This is what the Lord \divine{God} says: `The inner, east-facing courtyard is to remain shut during the six working days of the week,\fnote{The Heb. lacks \fbib{of the week}} but on the Sabbath day it is to be opened, as well as on the day of the New Moon. \v{2}The Regent Prince is to enter through the portico of the gate from outside and is to stand at the doorframe of the gate where the priests are to present the Regent Prince's\fnote{Lit. \fbib{his}} burnt offerings and peace\fnote{Or \fbib{fellowship}; and so throughout the chapter} offerings. Then the Regent Prince\fnote{Lit. \fbib{Then he}} is to worship at the threshold of the gate and go out. The gate is not to be closed until evening. \v{3}The people who live\fnote{The Heb. lacks \fbib{who live}} in the land are to worship at the doorway of the gate on the Sabbaths and New Moons in the \divine{Lord}'s presence.'\,''
\passage{Sabbath Offerings by the Regent Prince}

\v{4}```The burnt offering that the Regent Prince is to present to the \divine{Lord} on the Sabbath day is to consist of six lambs without defect, a ram without defect, \v{5}a grain offering with the ram consisting of an ephah, a grain offering with the lambs consisting of whatever amount he brings with him, and a hin of oil with each ephah of grain.\fnote{The Heb. lacks \fbib{of grain}} \v{6}Furthermore, each New Moon there is to be a young bull presented without defect, six male lambs, and a ram without defect. \v{7}The Regent Prince\fnote{Lit. \fbib{He}} is to present an ephah\fnote{I.e. five gallons in volume} of grain\fnote{The Heb. lacks \fbib{of grain}} along with the bull, an ephah\fnote{I.e. five gallons in volume} of grain\fnote{The Heb. lacks \fbib{of grain}} along with the ram, a grain offering---consisting of as much\fnote{The Heb. lacks \fbib{a grain offering---consisting of as much}} as he is able to give---and a hin\fnote{I.e. about a gallon} of olive oil with each ephah\fnote{I.e. five gallons in volume} of grain.\fnote{The Heb. lacks \fbib{of grain}}

\v{8}```The Regent Prince is to enter through the portico of the gate and is to leave the same way he came in. \v{9}When the people who live\fnote{The Heb. lacks \fbib{who live}} in the land come into the \divine{Lord}'s presence during the festivals, whoever enters through the northern gate is to leave through the southern gate, and whoever enters through the southern gate is to leave through the northern gate. No one is to leave by the same route that he enters, but instead is to go straight out. \v{10}The Regent Prince is to enter when they are coming in, and he is to leave when they go out.'\,''
\passage{Daily Offerings by the Regent Prince}

\v{11}```The grain offering for the festivals and appointed festivals is to include an ephah\fnote{I.e. five gallons in volume} with a bull, an ephah\fnote{I.e. five gallons in volume} with a ram, and as much grain with the lambs as the Regent Prince\fnote{Lit. \fbib{as he}} brings with him, along with a hin\fnote{I.e. about a gallon} of oil with each ephah. \v{12}Whenever the Regent Prince presents a voluntary offering, burnt offering, or peace offering, he is to present it voluntarily to the \divine{Lord}, and the east-facing gate is to be opened for him. He is to provide his burnt offering and peace offering as he does on the Sabbath. When he leaves, the gate is to be shut behind him. \v{13}He is to present a one year old lamb without defect for a burnt offering to the \divine{Lord} in the morning every day. \v{14}In addition, he is to present a grain offering with it every morning, consisting of a sixth of an ephah\fnote{I.e. about 5/6 of a gallon} mixed with one third of a hin\fnote{I.e. about 1/6 of a gallon} of oil. This grain offering is to be offered to the \divine{Lord} as a permanent ordinance. \v{15}They are to present the lamb offering, the grain offering, and the oil every morning as an ongoing\fnote{Or \fbib{perpetual}} burnt offering.'\,''
\passage{Gifts by the Regent Prince}

\v{16}``This is what the Lord \divine{God} says: `If the Regent Prince gives a gift to someone,\fnote{Lit. \fbib{to a man among his sons}} it is to remain with the man's descendants as their own inheritance. \v{17}But if he gives a gift to any of his servants, it is to belong to the servant\fnote{Lit. \fbib{to him}} until the Year of Release, at which time it is to be returned to the Regent Prince. His inheritance is to belong only to his sons. \v{18}The Regent Prince is not to appropriate the nation's inheritance nor take advantage of them by taking their property from them. Instead, he is to provide an inheritance for his sons from his own possessions so that my people will not be separated from their possessions.'\,''
\passage{The Place Where Offerings are Boiled}

\v{19}Then the angel\fnote{Lit. \fbib{Then he}} brought me in through an entrance beside the gate into the north-facing chambers dedicated to the priests. As I looked toward the rear\fnote{Lit. \fbib{north}} of the far western end, I saw a place \v{20}about which he said, ``This is where the priests will be boiling the guilt and sin offerings and baking the grain offerings so they don't bring them through the outer courtyard, thus diminishing the people's holiness.''\fnote{Or \fbib{thus transmitting holiness to the people}} \v{21}Then he brought me out to the exterior courtyard and led me across to each of the four corners of the courtyard. There in each corner was an enclosed area set aside, \v{22}all of them the same size; that is, each was 40 cubits\fnote{I.e. about 70 feet; a royal cubit was about 21 inches} long and 30 cubits\fnote{I.e. about 52.5 feet; a royal cubit was about 21 inches} wide. \v{23}A low wall\fnote{Lit. \fbib{A row}} built of masonry surrounded each courtyard, with boiling places set in rows in the wall. \v{24}He told me, ``This is where\fnote{Lit. \fbib{is the house}} the ministers of the Temple will be preparing\fnote{Lit. \fbib{boiling}} the sacrifices that will be presented by the people.
\labelchapt{47}
\passage{The Vision of the Temple River}

\chapt{47}
\v{1}After this, he brought me back to the doorway to the Temple. To my amazement, there was water flowing out toward the east from beneath the threshold of the Temple! (The Temple faced eastward.) The water flowed down from beneath the right side of the Temple,\fnote{I.e. the right side as one inside the temple faced toward the east.} that is, from the south-facing side where the altar was located. \v{2}Then he brought me out through the north gateway and around to the one outside that faces toward the east. To my amazement, water was trickling out from that part of\fnote{The Heb. lacks \fbib{that part of}} the south side, too!

\v{3}As the man went out toward the east, he carried a measuring line in his hand. He measured out 1,000 cubits\fnote{I.e. about 1,750 feet; a royal cubit was about 21 inches} as he led me through water that was ankle-deep. \v{4}Then he measured out another 1,000 cubits,\fnote{The Heb. lacks \fbib{cubits}} where he led me through water that was knee-deep. And then he measured out another 1,000 cubits,\fnote{The Heb. lacks \fbib{cubits}} where the water was waist-deep. \v{5}When he had measured out another 1,000 cubits,\fnote{The Heb. lacks \fbib{cubits}} the water had become deep enough that I wasn't able to ford it. Instead, I would have had to swim through it.

\v{6}Then, as he was bringing me back along the river bank, he asked me, ``Son of Man, did you see all of this?'' \v{7}As we were coming back, I was amazed to see that there were many, many trees lining both banks of the river. \v{8}He told me, ``This river flows toward the eastern territories all the way down into the Arabah,\fnote{I.e. Israel's southern desert areas including from the Sea of Galilee to the Red Sea through the Dead Sea} and from there its water flows toward the Dead\fnote{The Heb. lacks \fbib{Dead}} Sea, where the sea water turns fresh. \v{9}It will support all kinds of living creatures that will thrive abundantly wherever the river flows. There will be a great many fish, because this water will flow there and turn the salt water fresh. As a result, everything will live wherever the river flows. \v{10}A day will come when fishermen will line its banks---from En-gedi to En-eglaim\fnote{I.e. a city probably located on the northwest shore of the Dead Sea, not far from En-gedi, perhaps modern `Ain Feshka} there will be plenty of room to spread out nets. There will be all sorts of species of fish, as abundant as the fish that live in the Mediterranean\fnote{Lit. \fbib{Great}} Sea. There will be lots of them!

\v{11}``The river delta\fnote{The Heb. lacks \fbib{delta}} will consist of swamps and marshes that will remain a salt water wetland preserve. \v{12}Lining each side of the river banks, all sorts of species of fruit trees will be growing. Their leaves will never wither and their fruit will never fail. They will bear fruit every month, because the water that nourishes them will be flowing from the sanctuary. Their fruit will be for food and their leaves will contain substances that promote healing.''
\passage{Israel's Future Borders Delineated}

\v{13}This is what the Lord \divine{God} says: ``This is to be the territorial border by which you apportion the land for an inheritance among the twelve tribes of Israel, with Joseph double-portioned. \v{14}Apportion it for their inheritances, distributing everything equally as if you were distributing things to your own\fnote{The Heb. lacks \fbib{own}} brother, which is how I promised to give it to your ancestors. This way, the land will fall to you as an inheritance.

\v{15}``This is to be the\fnote{The Heb. lacks \fbib{is to be the}} border for the land: on the north side, from the Mediterranean\fnote{Lit. \fbib{Great}} Sea by the Hethlon Road to the entrance of Zedad, \v{16}Hamath, Berothah, Sibraim (which lies between the border of Damascus and the border of Hamath), and Hazer-hatticon, which is on the border of Hauran. \v{17}The border is to proceed from the Mediterranean\fnote{The Heb. lacks \fbib{Mediterranean}} Sea to Hazer-enan (a border of Damascus), and on the north facing north is to be the border of Hamath. This is to be the north side.

\v{18}``The eastern extremity is to proceed from between Hauran and\fnote{Lit. \fbib{Hauran and then between}} Damascus, then between Gilead, and then through the land of Israel---the Jordan River.\fnote{The Heb. lacks \fbib{River}} You are to measure from the northern border to the Dead\fnote{Lit. \fbib{Eastern}} Sea. This is to be the eastern perimeter.

\v{19}``You are to determine the southern extremity running from Tamar as far as the waters of Meribath-kadesh, then from there proceeding to the Wadi,\fnote{I.e. the Wadi of Egypt, a seasonal stream or river that channels water during rain seasons but is dry at other times; ancient Israel's southwestern-most border} and then to the Mediterranean\fnote{Lit. \fbib{Great}} Sea. This is to be the southern\fnote{Or \fbib{Negev}} perimeter.

\v{20}``The western\fnote{Lit. \fbib{sea}; i.e. the border running along the Mediterranean Sea} perimeter is to be the Mediterranean\fnote{Lit. \fbib{Great}} Sea, from the southernmost border to a location opposite the entrance to Hamath. This is to be the western\fnote{Lit. \fbib{sea}; i.e. the border running along the Mediterranean Sea} perimeter.

\v{21}``You are to apportion this land among yourselves according to the tribes of Israel, \v{22}dividing it by lottery among yourselves and among the foreigners who live among you and bear children among you. You are to treat them like native-born Israelis. Among you they,\fnote{Or \fbib{Israelis among you. They}} too, are to be allotted an inheritance with the tribes of Israel. \v{23}Furthermore, you are to provide the foreigner's inheritance there in the tribe within which he remains,'' declares the Lord \divine{God}.
\labelchapt{48}
\passage{Regulations for Israel's Northern Land Divisions}

\chapt{48}
\v{1}``These are the names of the tribes from the northernmost extremity westward\fnote{Lit. \fbib{to the sea}; i.e. in the direction of the Mediterranean Sea} along the road from Hethlon to the entrance of Hamath,\fnote{Or \fbib{to Lebo-hamath}} Hazar-enan (a border of Damascus) northward to the coast\fnote{Lit. \fbib{to the sea}; i.e. in the direction of the Mediterranean Sea} of Hamath. The perimeter is to run\fnote{The Heb. lacks \fbib{is to run}} east-to-west;\fnote{Lit. \fbib{from east to the Sea}; and so throughout the chapter} the tribe of\fnote{The Heb. lacks \fbib{The tribe of}; and so throughout the chapter} Dan with one portion;\fnote{The Heb. lacks \fbib{portion}; and so throughout the list} \v{2}running along the border of the tribe of Dan from the eastern perimeter to the western perimeter, the tribe of Asher with one portion; \v{3}running along the border of the tribe of Asher from the eastern perimeter to the western perimeter, the tribe of Naphtali with one portion; \v{4}running along the border of the tribe of Naphtali from the eastern perimeter to the western perimeter, the tribe of Manasseh with one portion; \v{5}running along the border of the tribe of Manasseh from the eastern perimeter to the western perimeter, the tribe of Ephraim with one portion; \v{6}running along the border of the tribe of Ephraim from the eastern perimeter to the western perimeter, the tribe of Reuben with one portion; \v{7}and running along the border of the tribe of Reuben from the eastern perimeter to the western perimeter, the tribe of Judah with one portion.''
\passage{Israel's National Temple Allotment}

\v{8}``Running along the border of the tribe of Judah from the eastern perimeter to the western perimeter you are to set apart the Terumah,\fnote{Lit. \fbib{Gift}; i.e. a special section of Israel's land to be dedicated to the \fbib{}\divine{Lord} as a national temple park, and so throughout the chapter; cf. Eze 45:1-7} 25,000 units\fnote{The Heb. lacks \fbib{units}; i.e. the measuring unit is unspecified throughout the chapter} wide, with its east-west length equal to one of the other apportionments, from the eastern perimeter to the western perimeter, with the Temple in the middle of it. \v{9}The Terumah that you are to give to the \divine{Lord} is to be 25,000 units wide.''\fnote{The Heb. lacks \fbib{wide}}
\passage{Allotments for the Priests}

\v{10}``The holy Terumah is to be reserved for these, the priests: Toward the north, 25,000 units in length;\fnote{The Heb. lacks \fbib{in length}} toward the west, 10,000 units in width; toward the east, 10,000 units in width; and toward the south, 25,000 units in length. The \divine{Lord}'s sanctuary is to be in its midst. \v{11}It is to be for use by\fnote{The Heb. lacks \fbib{It is to be for use by}} priests from the descendants of Zadok, who have observed the things with which I charged them and who did not wander astray when the Israelis went astray, just as the descendants of Levi wandered astray. \v{12}It is to be a Terumah for them from the allotment of the land, a Most Holy Place, adjoining the border of the descendants of Levi.''

\v{13}``Alongside the border of the priests, the descendants of Levi are to be allotted 25,000 units in length and 10,000 units in width. The entire length is to be 25,000 units and its width 10,000 units. \v{14}Furthermore, they are not to sell or exchange any part of it, nor transfer these first fruits\fnote{Or \fbib{this choice portion}} of the land, because it is holy to the \divine{Lord}.

\v{15}``The rest, 5,000 units wide and 25,000 units along its front, will serve as a common portion for use by the city for housing and open spaces, since the city is to be in its midst. \v{16}These are to be its dimensions: the north side, 4,500 units; the south side, 4,500 units; the east side, 4,500 units; and the west side 4,500 units. \v{17}The city is to have urban areas set aside: on the north 250 units; on the south, 250 units, on the east, 250 units; and on the west, 250 units.

\v{18}``The remainder of the length that borders the holy Terumah is to be 10,000 units long eastward and 10,000 units westward. It is to lie adjacent to the holy Terumah. Its harvest will produce food for those who work in the city. \v{19}The city workers who cultivate it are to come from all the tribes of Israel. \v{20}The entire Terumah is to be\fnote{The Heb. lacks \fbib{is to be}} 25,000 units by 25,000 units---you are to reserve it as a holy Terumah in the form of a square within the city limits.''
\passage{The Allotment for the Regent Prince}

\v{21}``Now the remainder of the allotment\fnote{The Heb. lacks \fbib{of the allotment}} on either side of the holy Terumah is to be for the Regent Prince and for city property\fnote{I.e. public property}---adjoining the 25,000 units along the eastern border and adjoining the 25,000 units along the western border, and parallel to the allotments. These are to be for the Regent Prince. The holy Terumah and the sanctuary of the Temple is to stand in the middle of it. \v{22}Except for what belongs to the descendants of Levi and the city property, which will stand in the middle of what belongs to the Regent Prince, whatever is between the border of Judah and the border of Benjamin is to belong to the Regent Prince.''
\passage{The Allotment for the Tribes}

\v{23}``Now as to the rest of the tribes: from the east side to the west side, Benjamin is to retain one portion.\fnote{Lit. \fbib{Benjamin, one}; and so through v. 27} \v{24}Adjacent to the border of Benjamin running from east to west, Simeon is to retain one portion. \v{25}Adjacent to the border of Simeon running from east to west, Issachar is to retain one portion. \v{26}Adjacent to the border of Issachar running from east to west, Zebulun is to retain one portion. \v{27}Adjacent to the border of Zebulun running from east to west, Gad is to retain one portion. \v{28}Adjacent to the border of Gad to the south and extending toward the south, the border is to proceed from Tamar to the waters of Meribath-kadesh, then to the Wadi\fnote{I.e. a seasonal stream or river that channels water during rain seasons but is dry at other times; ancient Israel's southwestern-most border} of Egypt,\fnote{I.e. ancient Israel's southwestern-most border} and from there to the Mediterranean\fnote{Lit. \fbib{Great}} Sea. \v{29}This is the land that you are to allocate by lottery to the tribes of Israel as their inheritance, and these are their respective divisions,'' declares the Lord \divine{God}.
\passage{The Gates of the City}

\v{30}``These are the exits to the city: On the north side, 4,500 units by measurement, \v{31}are to be the gates of the city. Named after the tribes of Israel, three gates are to serve the north site: one named the Reuben Gate, one named the Judah Gate, and one named the Levi Gate. \v{32}On the east side, 4,500 units by measurement,\fnote{The Heb. lacks \fbib{by measurement}} there are to be three gates: one named the Joseph Gate, one named the Benjamin Gate, and one named the Dan Gate. \v{33}On the south side, 4,500 units by measurement, there are to be three gates: one named the Simeon Gate, one named the Issachar Gate, and one named the Zebulun Gate. \v{34}On the west side, 4,500 units by measurement,\fnote{The Heb. lacks \fbib{by measurement}} there are to be three gates: one named the Gad Gate, one named the Asher Gate, and one named the Naphtali Gate. \v{35}A perimeter is to measure 18,000 units, and the name of the city from that time on is to be:

\begin{poetry}
\poeml `\divine{The Lord is There}.'\,''\end{poetry}
