\bookheader{Proverbs}
\labelbook{Prov}

\bookpretitle{The Book of}
\booktitle{Proverbs}

\labelchapt{1}
\passage{Introduction and Purpose}

\chapt{1}
\v{1}The proverbs of David's son Solomon, king of Israel.

\begin{poetry}
\poeml \v{2}These proverbs are\fnote{\fbackref{1:2} The Heb. lacks \fbib{These proverbs are}} for gaining\fnote{\fbackref{1:2} Or \fbib{knowing}} wisdom and discipline;\fnote{\fbackref{1:2} Or \fbib{instruction}} \\
\poemll    for understanding words of insight; \\
\poeml \v{3}for acquiring the discipline\fnote{\fbackref{1:3} Or \fbib{instruction}} that produces wise behavior, \\
\poemll    righteousness, justice, and upright living;\fnote{\fbackref{1:3} Lit. \fbib{and uprightness}} \\
\poeml \v{4}for giving prudence to the na\"{i}ve, \\
\poemll    and knowledge and discretion to the young. \\
\poeml \v{5}Let the wise listen and increase their\fnote{\fbackref{1:5} The Heb. lacks \fbib{their}} learning; \\
\poemll    let the person of understanding receive guidance \\
\poeml \v{6}in understanding proverbs, clever sayings, \\
\poemll    words of the wise, and their riddles.
\passage{The Major Theme}
\poeml \v{7}The fear of the \divine{Lord} is the beginning of knowledge, \\
\poemll    but fools despise wisdom and discipline.\fnote{\fbackref{1:7} Or \fbib{instruction}}
\passage{The Minor Theme}
\poeml \v{8}My son, listen to your father's instruction, \\
\poemll    and do not let go of your mother's teaching. \\
\poeml \v{9}They will be a graceful wreath for your head \\
\poemll    and a chain for your neck.
\passage{Avoid Evil Counsel}
\poeml \v{10}My son, if sinners entice you, \\
\poemll    do not consent. \\
\poeml \v{11}If they say, ``Come with us! \\
\poemll    Let's lie in wait for blood; \\
\poemlll       let's ambush some innocent person for no reason at all. \\
\poeml \v{12}Let's swallow them alive like Sheol,\fnote{\fbackref{1:12} I.e. the realm of the dead; possibly an allusion to the rebellion of Korah (cf. Num 16:33)} \\
\poemll    and whole like those who go down into the Pit.\fnote{\fbackref{1:12} I.e. the place of punishment in the afterlife} \\
\poeml \v{13}We'll find all kinds of valuable wealth, \\
\poemll    and we'll fill our houses with spoil. \\
\poeml \v{14}Throw your lot in with us, \\
\poemll    and all of us will have one purse.'' \\
\poeml \v{15}My son, do not go along with them,\fnote{\fbackref{1:15} Lit. \fbib{in the way with them}} \\
\poemll    and keep your feet away from their paths! \\
\poeml \v{16}For they\fnote{\fbackref{1:16} Lit. \fbib{For their feet}} run toward evil; \\
\poemll    these enticers\fnote{\fbackref{1:16} Lit. \fbib{they}} shed blood without hesitation.\fnote{\fbackref{1:16} Lit. \fbib{blood quickly}} \\
\poeml \v{17}Look, it is useless to spread a net in full view of\fnote{\fbackref{1:17} Lit. \fbib{in the eyes of}} all the birds, \\
\poeml \v{18}but these people\fnote{\fbackref{1:18} Lit. \fbib{they}} lie in wait for their own blood.\fnote{\fbackref{1:18} The Heb. lacks \fbib{their own}} \\
\poemlll       They ambush only themselves. \\
\poeml \v{19}Such is the way of all those who seek illicit gain--- \\
\poemll    it takes away the lives of those who possess it.
\passage{The Benefits of Choosing Wisdom}
\poeml \v{20}Wisdom cries out in the street; \\
\poemll    she raises her voice in the public squares. \\
\poeml \v{21}She calls out at the busiest part\fnote{\fbackref{1:21} Lit. \fbib{head}} of the noisy streets,\fnote{\fbackref{1:21} So MT; LXX Syr Targ read \fbib{and on top of the walls}} \\
\poemll    and at the entrance to the gates of the city she utters her words: \\
\poeml \v{22}``You na\"{i}ve ones, how long will you love naivet\'{e}? \\
\poemll    And how long will scoffers delight in scoffing \\
\poemlll       or fools hate knowledge?'' \\
\poeml \v{23}Return to my correction! \\
\poemll    Look, I will pour out my spirit on you, \\
\poemlll       and I will make my words known to you.
\passage{The Consequences of Refusing Wisdom}
\poeml \v{24}``Because I called out to you and you refused to respond---\fnote{\fbackref{1:24} Lit. \fbib{you refused}} \\
\poemll    I appealed,\fnote{\fbackref{1:24} Lit. \fbib{I stretched out my hand}} but no one paid attention--- \\
\poeml \v{25}because\fnote{\fbackref{1:25} The Heb. lacks \fbib{because}} you neglected all my advice \\
\poemll    and did not want my correction, \\
\poeml \v{26}I will laugh at your calamity. \\
\poemll    I will mock when what you fear\fnote{\fbackref{1:26} Lit. \fbib{when your fear}} comes, \\
\poeml \v{27}when what you dread comes like a storm, \\
\poemll    and your calamity comes on like a whirlwind, \\
\poemlll       when distress and anguish come upon you. \\
\poeml \v{28}``Then they will call out to me, \\
\poemll    but I will not answer; \\
\poeml they will seek me diligently, \\
\poemll    but they will not find me. \\
\poeml \v{29}``Because they hated knowledge \\
\poemll    and did not choose the fear of the \divine{Lord}; \\
\poeml \v{30}they did not want my advice, \\
\poemll    and they rejected all my correction. \\
\poeml \v{31}They will eat the fruit\fnote{\fbackref{1:31} I.e. experience the consequences} of their way, \\
\poemll    and they will be filled with their own devices. \\
\poeml \v{32}Indeed, the waywardness\fnote{\fbackref{1:32} So MT; DSS 4QProv\textsuperscript{a} reads \fbib{narrow-mindedness}; lit. \fbib{the pull of}; LXX reads \fbib{Because they would wrong the na\"{i}ve, they will be murdered}} of the na\"{i}ve will kill them, \\
\poemll    and the complacency of fools will destroy them. \\
\poeml \v{33}``But the person who listens to me will live safely \\
\poemll    and will be secure from the fear of evil.''
\end{poetry}
\labelchapt{2}
\passage{The Benefits of Embracing Wisdom}

\begin{poetry}
\poeml \chapt{2}
\v{1}My son, if you accept my words, \\
\poeml and treasure my instructions\fnote{\fbackref{2:1} Lit. \fbib{instructions within you}}--- \\
\poeml \v{2}making your ear attentive to wisdom, \\
\poemll    and turning your heart to understanding--- \\
\poeml \v{3}if, indeed, you call out for insight \\
\poemll    and raise your voice for understanding, \\
\poeml \v{4}if you seek it like silver \\
\poemll    and search for it like hidden treasure, \\
\poeml \v{5}then you will understand the fear of the \divine{Lord} \\
\poemll    and learn to know God. \\
\poeml \v{6}For the \divine{Lord} gives wisdom, \\
\poemll    and from his mouth come knowledge and understanding. \\
\poeml \v{7}He stores up sound wisdom for the upright \\
\poemll    and is a shield to those who walk in integrity--- \\
\poeml \v{8}guarding the paths of the just \\
\poemll    and protecting the way of his faithful ones. \\
\poeml \v{9}Then you will understand what is right, just, \\
\poemll    and upright---every good path. \\
\poeml \v{10}For wisdom will enter your heart, \\
\poemll    and knowledge will be pleasant to your soul. \\
\poeml \v{11}Discretion\fnote{\fbackref{2:11} Or \fbib{Wise planning}} will protect you; \\
\poemll    understanding will watch over you, \\
\poeml \v{12}delivering you from the way of evil, \\
\poemll    from men who speak perverse things, \\
\poeml \v{13}and from those who abandon the right\fnote{\fbackref{2:13} Lit. \fbib{straight} or \fbib{upright}} path \\
\poemll    to travel along the ways of darkness; \\
\poeml \v{14}who delight in doing evil, \\
\poemll    and rejoice in the perverseness of evil; \\
\poeml \v{15}whose paths are crooked \\
\poemll    and who are devious in their ways, \\
\poeml \v{16}delivering you from the adulteress, \\
\poemll    from the immoral\fnote{\fbackref{2:16} Lit. \fbib{foreign}; i.e. one whose values are foreign to God's Law} woman with her seductive words, \\
\poeml \v{17}someone who abandoned the companion of her youth \\
\poemlll       and forgot the covenant of her God. \\
\poeml \v{18}For her house leads down to death, \\
\poemll    and her paths down to the realm of the dead. \\
\poeml \v{19}None who go to her return, \\
\poeml nor do they reach the paths of life. \\
\poeml \v{20}This is how you will walk in the way of good men \\
\poemll    and will keep to the paths of the righteous. \\
\poeml \v{21}For the upright will live in the land, \\
\poemll    and people of integrity will remain in it. \\
\poeml \v{22}But the wicked will be cut off from the land, \\
\poemll    and the treacherous will be uprooted from it.
\end{poetry}
\labelchapt{3}
\passage{The Blessings of Trusting God}

\begin{poetry}
\poeml \chapt{3}
\v{1}My son, don't forget my instruction, \\
\poeml and keep my commandments carefully in mind.\fnote{\fbackref{3:1} Lit. \fbib{Let your heart keep my commandments}} \\
\poeml \v{2}For they will add length to your days, years to your life, \\
\poemll    and abundant peace to you. \\
\poeml \v{3}Do not let gracious love and truth leave you. \\
\poemll    Bind them around your neck, \\
\poemlll       write them on the tablet of your heart, \\
\poeml \v{4}and find favor and a good reputation\fnote{\fbackref{3:4} Lit. \fbib{good judgment} or \fbib{sense}} with God and men. \\
\poeml \v{5}Trust in the \divine{Lord} with all your heart, \\
\poemll    and do not depend on your own understanding. \\
\poeml \v{6}In all your ways acknowledge\fnote{\fbackref{3:6} Or \fbib{know}} him, \\
\poemll    and he will make your paths straight. \\
\poeml \v{7}Do not be wise in your own opinion. \\
\poemll    Fear the \divine{Lord} and turn away from evil. \\
\poeml \v{8}This will bring healing to your body, \\
\poemll    and refreshment to your bones. \\
\poeml \v{9}Honor the \divine{Lord} with your wealth \\
\poemll    and with the first\fnote{\fbackref{3:9} Or \fbib{best}} of all your produce, \\
\poeml \v{10}so your barns will be filled with abundance, \\
\poemll    and your vats will burst open with new wine. \\
\poeml \v{11}My son, do not reject the \divine{Lord}'s discipline, \\
\poemll    and do not despise his correction, \\
\poeml \v{12}because the \divine{Lord} corrects the person he loves, \\
\poemll    just as a father corrects\fnote{\fbackref{3:12} The Heb. lacks \fbib{corrects}} the son he delights in.\fnote{\fbackref{3:12} So MT; LXX reads \fbib{loves, and he punishes every son he accepts}}
\passage{Wisdom More Valuable than Riches}
\poeml \v{13}How joyful is the man who finds wisdom, \\
\poemll    and the man who gains understanding, \\
\poeml \v{14}because her profit is better than the profit of silver, \\
\poemll    and her yield than fine gold. \\
\poeml \v{15}She is more precious than rubies, \\
\poemll    and nothing you desire compares with her. \\
\poeml \v{16}Long life is in her right hand, \\
\poemll    and in her left are riches and honor. \\
\poeml \v{17}Her ways are pleasant ways, \\
\poemll    and all her paths are peaceful. \\
\poeml \v{18}She is a tree of life for those who embrace her, \\
\poemll    and whoever clutches her tightly will be joyful. \\
\poeml \v{19}By wisdom the \divine{Lord} laid the earth's foundations, \\
\poemll    and by understanding he set the heavens in place. \\
\poeml \v{20}By his knowledge the depths broke open, \\
\poemll    and the clouds drip with dew.
\passage{Benefits of Wisdom}
\poeml \v{21}My son, do not let wisdom\fnote{\fbackref{3:21} The Heb. lacks \fbib{wisdom}} leave your sight. \\
\poemll    Carefully observe sound judgment and discernment, \\
\poeml \v{22}and they will be life to you \\
\poemll    and a graceful ornament\fnote{\fbackref{3:22} Lit. \fbib{grace}} for your neck. \\
\poeml \v{23}Then you will travel safely on your way, \\
\poemll    and your foot will not stumble. \\
\poeml \v{24}When you sit\fnote{\fbackref{3:24} So LXX; MT reads \fbib{lie}} down, you will not be afraid; \\
\poemll    when you lie down, your sleep will be pleasant.\fnote{\fbackref{3:24} Or \fbib{sweet}} \\
\poeml \v{25}Do not be afraid of sudden disaster,\fnote{\fbackref{3:25} Lit. \fbib{terror}} \\
\poemll    or the devastation that comes to the wicked. \\
\poeml \v{26}Indeed, the \divine{Lord} will be your confidence, \\
\poemll    and he will keep your foot from being caught.
\passage{Wisdom in Action}
\poeml \v{27}Do not withhold good from those to whom it is due, \\
\poemll    when it is in your power to act. \\
\poeml \v{28}Do not say to your neighbor, \\
\poemll    ``Go, and come back. \\
\poemlll       I will pay you\fnote{\fbackref{3:28} Lit. \fbib{it}} tomorrow,'' \\
\poeml when you have cash\fnote{\fbackref{3:28} The Heb. lacks \fbib{cash}} with you. \\
\poeml \v{29}Do not plan to harm your neighbor, \\
\poemll    when he is living peacefully\fnote{\fbackref{3:29} Or \fbib{securely}} beside you. \\
\poeml \v{30}Do not bring a lawsuit against a person for no reason, \\
\poemll    when he has done you no harm. \\
\poeml \v{31}Do not envy a violent man, \\
\poemll    and do not emulate his lifestyle.\fnote{\fbackref{3:31} Lit. \fbib{ways}} \\
\poeml \v{32}Indeed, a perverse man is utterly disgusting\fnote{\fbackref{3:32} Lit. \fbib{an abomination}} to the \divine{Lord}, \\
\poemll    but he takes the upright into his confidence.\fnote{\fbackref{3:32} Lit. \fbib{but his secret counsel is with the upright}} \\
\poeml \v{33}The \divine{Lord}'s curse is on the house of the wicked, \\
\poemll    but he blesses the dwelling of the righteous. \\
\poeml \v{34}Though God\fnote{\fbackref{3:34} Lit. \fbib{he}} scoffs at scoffers, \\
\poemll    he gives grace to the humble. \\
\poeml \v{35}The wise will inherit honor, \\
\poemll    but he holds fools up for ridicule.
\end{poetry}
\labelchapt{4}
\passage{Diligently Pursue Wisdom}

\begin{poetry}
\poeml \chapt{4}
\v{1}Listen, children,\fnote{\fbackref{4:1} Lit. \fbib{sons}} to your father's instruction, \\
\poeml and pay attention in order to gain understanding. \\
\poeml \v{2}I give you sound teaching, \\
\poeml so do not abandon my instruction.\fnote{\fbackref{4:2} Or \fbib{law}} \\
\poeml \v{3}When I was a son to my father, \\
\poemll    not yet strong\fnote{\fbackref{4:3} Lit. \fbib{delicate}} and an only son to my mother, \\
\poeml \v{4}he taught me and told me, \\
\poemll    ``Let your heart fully embrace what I have to say;\fnote{\fbackref{4:4} Lit. \fbib{embrace my words}} \\
\poemlll       keep my commandments and live! \\
\poeml \v{5}Get wisdom! Get understanding! \\
\poemll    Do not forget or turn aside from the words of my mouth! \\
\poeml \v{6}Do not abandon her, and she will protect you. \\
\poemll    Love her, and she will watch over you. \\
\poeml \v{7}Wisdom is of utmost importance, therefore get wisdom, \\
\poemll    and with all your effort work to acquire understanding. \\
\poeml \v{8}Prize her and she will exalt you. \\
\poemll    Indeed, if you embrace her, she will honor you. \\
\poeml \v{9}She will place on your head a graceful garland; \\
\poemll    she will present to you a crown of beauty.'' \\
\poeml \v{10}Listen, my son: accept my words, \\
\poemll    and you'll live a long, long time.\fnote{\fbackref{4:10} Lit. \fbib{and the years of your life will be many}} \\
\poeml \v{11}I have directed you in the way of wisdom, \\
\poemll    and I have led you along straight\fnote{\fbackref{4:11} Or \fbib{upright}} paths. \\
\poeml \v{12}When you walk, your step will not be hindered, \\
\poemll    and when you run, you will not stumble. \\
\poeml \v{13}Hold on to instruction, do not let it go! \\
\poemll    Guard wisdom,\fnote{\fbackref{4:13} Lit. \fbib{her}} because she is your life!
\passage{Avoiding the Ways of the Wicked}
\poeml \v{14}Do not enter the path of the wicked, \\
\poemll    or go along the way of evil men. \\
\poeml \v{15}Avoid it! Don't travel on it! \\
\poemll    Turn away from it, and pass on by. \\
\poeml \v{16}For they cannot sleep unless they are doing evil, \\
\poemll    and they are robbed of their sleep unless they cause someone to stumble. \\
\poeml \v{17}For they eat the bread of wickedness, \\
\poemll    and they drink the wine of violence. \\
\poeml \v{18}The path of the righteous is like the light of dawn \\
\poemll    that grows brighter until the full light of day. \\
\poeml \v{19}But the way of the wicked is like deep darkness, \\
\poemll    and they do not know what they are stumbling over.
\passage{Remembering the Counsel of a Wise Father}
\poeml \v{20}My son, pay attention to my words, \\
\poemll    and listen closely\fnote{\fbackref{4:20} Lit. \fbib{turn your ear}} to what I say. \\
\poeml \v{21}Do not let them out of your sight; \\
\poemll    keep them within your heart. \\
\poeml \v{22}For they are life to those who find them, \\
\poemll    and healing to their whole body.\fnote{\fbackref{4:22} Lit. \fbib{flesh}} \\
\poeml \v{23}Above everything else\fnote{\fbackref{4:23} Lit. \fbib{Above all watching}} guard your heart, \\
\poemll    because from it flow the springs of life. \\
\poeml \v{24}Never talk deceptively \\
\poemll    and don't keep company with people whose speech is corrupt.\fnote{\fbackref{4:24} Lit. \fbib{keep corrupt lips far from you}} \\
\poeml \v{25}Let your eyes look directly ahead; \\
\poemll    fix your gaze straight in front of you. \\
\poeml \v{26}Carefully measure\fnote{\fbackref{4:26} Lit. \fbib{Weigh}} the paths for your feet, \\
\poemll    and all your ways will be established. \\
\poeml \v{27}Do not turn to the right or to the left; \\
\poemll    turn your foot away from evil.
\end{poetry}
\labelchapt{5}
\passage{Warning against Sexual Immorality}

\begin{poetry}
\poeml \chapt{5}
\v{1}My son, pay attention to my wisdom, \\
\poeml and listen closely to my insight, \\
\poeml \v{2}so you may carefully practice\fnote{\fbackref{5:2} Lit. \fbib{guard}} discretion \\
\poemll    and your lips preserve knowledge. \\
\poeml \v{3}For the lips of an adulteress drip honey, \\
\poemll    and her speech\fnote{\fbackref{5:3} Lit. \fbib{palate}} is smoother than oil. \\
\poeml \v{4}But in the end she is as bitter as wormwood,\fnote{\fbackref{5:4} \fbib{Wormwood} is a plant with an extremely bitter taste.} \\
\poemll    and as sharp as a double-edged sword. \\
\poeml \v{5}Her feet go down to death; \\
\poemll    her steps lead to Sheol.\fnote{\fbackref{5:5} I.e. the realm of the dead} \\
\poeml \v{6}You aren't thinking about\fnote{\fbackref{5:6} Or \fbib{She does not consider}} where her life is headed; \\
\poemll    her steps wander, but you do not realize\fnote{\fbackref{5:6} Or \fbib{she does not realize}} it. \\
\poeml \v{7}Now, children,\fnote{\fbackref{5:7} Or \fbib{sons}} listen to me. \\
\poemll    Don't turn away from what I am saying.\fnote{\fbackref{5:7} Lit. \fbib{from the words of my mouth}} \\
\poeml \v{8}Keep\fnote{\fbackref{5:8} Lit. \fbib{Keep your path}} far away from her, \\
\poemll    and don't go near the entrance to her house, \\
\poeml \v{9}so that you don't give your honor to others, \\
\poemll    and waste your best years;\fnote{\fbackref{5:9} Lit. \fbib{and your years to the cruel}} \\
\poeml \v{10}so that strangers don't enrich themselves at your expense,\fnote{\fbackref{5:10} Lit. \fbib{don't satisfy themselves with your strength}} \\
\poemll    and your work won't end up the possession of foreigners.\fnote{\fbackref{5:10} Lit. \fbib{won't go into a foreigner's house}} \\
\poeml \v{11}You will cry out in anguish when your end comes, \\
\poemll    when your flesh and body are consumed, \\
\poeml \v{12}and you will say, ``How I hated instruction,\fnote{\fbackref{5:12} Or \fbib{discipline}} \\
\poemll    and my heart rejected correction! \\
\poeml \v{13}I did not obey my teachers \\
\poemll    and did not listen\fnote{\fbackref{5:13} Lit. \fbib{incline my ear}} to my instructors. \\
\poeml \v{14}Now I am at the point of utter disaster \\
\poemll    in\fnote{\fbackref{5:14} Lit. \fbib{in the midst of}} the assembly and in the congregation.''
\passage{The Delights of Marital Faithfulness}
\poeml \v{15}Drink water from your own cistern, \\
\poemll    and fresh\fnote{\fbackref{5:15} Lit. \fbib{flowing}} water from your own well. \\
\poeml \v{16}Should your springs flow outside, \\
\poemll    or streams of water in the street? \\
\poeml \v{17}They should be for you alone \\
\poemll    and not for strangers who are with you. \\
\poeml \v{18}Let your fountain be blessed \\
\poemll    and enjoy the wife of your youth. \\
\poeml \v{19}Like a loving deer, a beautiful doe, \\
\poemll    let her breasts satisfy you all the time. \\
\poemlll       Be constantly intoxicated by her love. \\
\poeml \v{20}Why should you be intoxicated by an adulteress, my son, \\
\poemll    and embrace the bosom of a foreign woman? \\
\poeml \v{21}Indeed, what a man does is\fnote{\fbackref{5:21} Lit. \fbib{Indeed, a man's ways are}} always in the \divine{Lord}'s presence,\fnote{\fbackref{5:21} Lit. \fbib{in front of the \divine{Lord}'s eyes}} \\
\poemll    and he weighs all his paths. \\
\poeml \v{22}The wicked person's iniquities will capture him, \\
\poemll    and he will be held with the cords of his sin. \\
\poeml \v{23}He will die for lack of discipline, \\
\poemll    and he goes astray because of his great folly.
\end{poetry}
\labelchapt{6}
\passage{The Folly of Guaranteeing Loans}

\begin{poetry}
\poeml \chapt{6}
\v{1}My son, if you guarantee a loan for your neighbor, \\
\poeml if you have agreed to a deal\fnote{\fbackref{6:1} Lit. \fbib{have clapped your hands}; i.e. have shaken hands} with a stranger, \\
\poeml \v{2}trapped by your own words, \\
\poemll    and caught by your own words, \\
\poeml \v{3}then do this, my son, and deliver yourself, \\
\poemll    because you have come under your neighbor's control.\fnote{\fbackref{6:3} Lit. \fbib{into the hands of your neighbor}} \\
\poeml Go, humble yourself! \\
\poemll    Plead passionately with your neighbor! \\
\poeml \v{4}Don't allow yourself to sleep \\
\poemll    or even to close your eyes. \\
\poeml \v{5}Deliver yourself like a gazelle from a hunter's hand,\fnote{\fbackref{6:5} So MT; LXX Syr Targ read \fbib{from the hunter}; or \fbib{a noose}} \\
\poemll    or like a bird from a fowler's hand.
\passage{The Folly of Laziness}
\poeml \v{6}Go to the ant, you lazy man! \\
\poemll    Observe its ways and become wise. \\
\poeml \v{7}It has no commander, \\
\poemll    officer, or ruler, \\
\poeml \v{8}but prepares its provisions in the summer \\
\poemll    and gathers its food in the harvest. \\
\poeml \v{9}How long will you lie down, lazy man? \\
\poemll    When will you get up from your sleep? \\
\poeml \v{10}A little sleep, a little slumber, \\
\poemll    a little folding of the hands to rest, \\
\poeml \v{11}and your poverty will come on you like a bandit \\
\poemll    and your desperation like an armed man.
\passage{The Folly of Causing Strife}
\poeml \v{12}A worthless man, a wicked man, \\
\poemll    goes around with devious speech, \\
\poeml \v{13}winking with his eyes, making signs\fnote{\fbackref{6:13} Lit. \fbib{scraping}} with\fnote{\fbackref{6:13} The Heb. lacks \fbib{with}} his feet, \\
\poemll    pointing with his fingers, \\
\poeml \v{14}planning evil with a perverse mind,\fnote{\fbackref{6:14} Or \fbib{heart}} \\
\poemll    continually stirring up discord. \\
\poeml \v{15}Therefore, disaster will overtake him suddenly. \\
\poemll    He will be broken in an instant, \\
\poemlll       and he will never recover.
\passage{What God Hates}
\poeml \v{16}Here are six things that the \divine{Lord} hates--- \\
\poemll    seven, in fact,\fnote{\fbackref{6:16} The Heb. lacks \fbib{in fact}} are detestable to him:\fnote{\fbackref{6:16} Lit. \fbib{to his soul}} \\
\poeml \v{17}Arrogant eyes, \\
\poemll    a lying tongue, \\
\poemlll       and hands shedding innocent blood; \\
\poeml \v{18}a heart crafting evil plans, \\
\poemll    feet running swiftly to wickedness, \\
\poeml \v{19}a false witness snorting lies, \\
\poemll    and someone sowing quarrels between brothers.
\passage{Parental Counsel about Immorality}
\poeml \v{20}Keep your father's commands, my son, \\
\poemll    and never forsake your mother's rules,\fnote{\fbackref{6:20} Or \fbib{laws}} \\
\poeml \v{21}by binding them to your heart continuously, \\
\poemll    fastening them around your neck. \\
\poeml \v{22}During your travels wisdom\fnote{\fbackref{6:22} Lit. \fbib{wisdom; i}.e. wisdom personified as a woman} will lead you; \\
\poemll    she will watch over you while you rest; \\
\poeml and when you are startled from your sleep, \\
\poemll    she will commune with you. \\
\poeml \v{23}Because the command is a lamp \\
\poemll    and the Law a light, \\
\poemlll       rebukes that discipline are a way of life--- \\
\poeml \v{24}to protect you from the evil\fnote{\fbackref{6:24} So MT; LXX reads \fbib{married}} woman, \\
\poemll    from the words of the seductive woman. \\
\poeml \v{25}Do not focus on her beauty in your mind, \\
\poemll    nor allow her to take you prisoner with her flirting eyes, \\
\poeml \v{26}because the price of a whore is a loaf of bread, \\
\poemll    and an adulterous woman stalks a man's precious life. \\
\poeml \v{27}Can a man scoop fire into his bosom \\
\poemll    without burning his clothes? \\
\poeml \v{28}Can a man walk on hot coals \\
\poemll    without scorching his feet? \\
\poeml \v{29}So also is it with someone who has sex with his neighbor's wife; \\
\poemll    anyone touching her will not remain unpunished. \\
\poeml \v{30}A thief isn't despised \\
\poemll    if he steals to meet his needs\fnote{\fbackref{6:30} Lit. \fbib{to refresh his soul}} when he is hungry, \\
\poeml \v{31}but when he is discovered, \\
\poemll    he must restore seven-fold, \\
\poemlll       forfeiting the entire value of his house. \\
\poeml \v{32}Whoever commits adultery with a woman is out of his mind; \\
\poemll    by doing so he corrupts his own soul. \\
\poeml \v{33}He will receive a beating and dishonor, \\
\poemll    and his shame won't disappear, \\
\poeml \v{34}because jealousy incites\fnote{\fbackref{6:34} The Heb. lacks \fbib{incites}} a strong man's rage, \\
\poemll    and he will show no mercy when it's time for revenge. \\
\poeml \v{35}He will not consider any payment, \\
\poemll    nor will he be willing to accept it,\fnote{\fbackref{6:35} The Heb. lacks \fbib{to accept it}} \\
\poemlll       no matter how large the bribe.
\end{poetry}
\labelchapt{7}
\passage{On Avoiding the Immoral Woman}

\begin{poetry}
\poeml \chapt{7}
\v{1}My son, guard what I say \\
\poeml and treasure my commands. \\
\poeml \v{2}Keep my commands and you'll live. \\
\poemll    Guard\fnote{\fbackref{7:2} The Heb. lacks \fbib{Guard}} my teaching as you do your eyesight. \\
\poeml \v{3}Strap them to your fingers \\
\poemll    and engrave them on the tablet of your heart. \\
\poeml \v{4}Say to wisdom, ``You're my sister!'' \\
\poemll    and call understanding your close relative, \\
\poeml \v{5}so they can keep you from an adulterous woman, \\
\poemll    from the immoral woman with her seductive words.
\passage{A Father's Warning}
\poeml \v{6}For from a window in my house \\
\poemll    I peered through the lattice work, \\
\poeml \v{7}and I noticed among the na\"{i}ve--- \\
\poemll    that is, I discerned among the youths--- \\
\poemlll       a senseless young man. \\
\poeml \v{8}Proceeding down the street near her corner, \\
\poemll    he makes his way toward her house \\
\poeml \v{9}at twilight, during the evening, \\
\poemll    even during the darkest part of the night. \\
\poeml \v{10}Look! A woman makes her way to meet him, \\
\poemll    dressed as a prostitute \\
\poemlll       and intending to entrap him. \\
\poeml \v{11}She is brazen and defiant--- \\
\poemll    her feet don't remain at home. \\
\poeml \v{12}Now she is in the street, now in the plazas, \\
\poemll    she lurks near every corner. \\
\poeml \v{13}So she grabs hold of him and kisses him, \\
\poemll    with a brazen face she speaks to him, \\
\poeml \v{14}``I have given\fnote{\fbackref{7:14} The Heb. lacks \fbib{given}} my peace offerings, \\
\poemll    and today I fulfilled my vows. \\
\poeml \v{15}Therefore, I've come out to meet you, \\
\poemll    I've looked just for you, \\
\poemlll       and I found you! \\
\poeml \v{16}I've decorated my bed with new coverings--- \\
\poemll    embroidered linen from Egypt. \\
\poeml \v{17}I've perfumed my bed \\
\poemll    with myrrh, aloes, and cinnamon. \\
\poeml \v{18}Come, let's make love until dawn; \\
\poemll    let's comfort ourselves with love, \\
\poeml \v{19}because my husband isn't home. \\
\poemll    He left on a long trip. \\
\poeml \v{20}He took a fist full of cash \\
\poemll    and he'll return home in a month.'' \\
\poeml \v{21}She leads him astray with great persuasion; \\
\poemll    with flattering lips she seduces him. \\
\poeml \v{22}All of a sudden he follows her \\
\poemll    like an ox fit for slaughter \\
\poemlll       or like a fool fit for a trap\fnote{\fbackref{7:22} So MT; LXX reads \fbib{a dog fit for chains}} \\
\poeml \v{23}until an arrow pierces his liver. \\
\poemll    As a bird darts into a snare, \\
\poemlll       he doesn't realize his fatal decision.\fnote{\fbackref{7:23} Lit. \fbib{realize it is his life}} \\
\poeml \v{24}So listen to me, my sons, \\
\poemll    and pay attention to what I have to say. \\
\poeml \v{25}Don't be led astray by her lifestyle,\fnote{\fbackref{7:25} Lit. \fbib{ways}} \\
\poemll    and don't imitate her behavior.\fnote{\fbackref{7:25} Lit. \fbib{paths}} \\
\poeml \v{26}For many are the victims whom she has conquered, \\
\poemll    and many are her slain. \\
\poeml \v{27}Her house leads to Sheol,\fnote{\fbackref{7:27} I.e. the realm of the dead} \\
\poemll    descending to death's catacombs.
\end{poetry}
\labelchapt{8}
\passage{Wisdom Calls for an Audience}

\begin{poetry}
\poeml \chapt{8}
\v{1}Isn't wisdom calling out; \\
\poeml isn't understanding raising her voice? \\
\poeml \v{2}On top of the highest places along the road \\
\poemll    she stands where the roads meet. \\
\poeml \v{3}Beside the gates, at the city entrance--- \\
\poemll    at the entrance to the portals she cries aloud: \\
\poeml \v{4}``I'm calling to you, men! \\
\poemll    What I have to say pertains\fnote{\fbackref{8:4} Lit. \fbib{My voice is}} to all mankind! \\
\poeml \v{5}Understand prudence, you na\"{i}ve people; \\
\poemll    and gain an understanding heart, you foolish ones. \\
\poeml \v{6}Listen, because I have noble things to say, \\
\poemll    and what I have to say\fnote{\fbackref{8:6} Lit. \fbib{my open lips}} will reveal what is right. \\
\poeml \v{7}For my mouth speaks the truth--- \\
\poemll    wickedness is detestable to me. \\
\poeml \v{8}Everything I have to say is just; \\
\poemll    there isn't anything corrupt or perverse in my speech.\fnote{\fbackref{8:8} Lit. \fbib{words}} \\
\poeml \v{9}Everything I say is sensible to someone who understands, \\
\poemll    and correct to those who have acquired knowledge. \\
\poeml \v{10}Grab hold of my instruction in lieu of money \\
\poemll    and knowledge instead of the finest gold, \\
\poeml \v{11}because wisdom is better than precious gems\fnote{\fbackref{8:11} Or \fbib{rubies}} \\
\poemll    and nothing you desire can compare to it.''
\passage{The Way of Wisdom}
\poeml \v{12}``I, wisdom, am related to\fnote{\fbackref{8:12} Lit. \fbib{wisdom, live with}} prudence. \\
\poemll    I know how to be discreet. \\
\poeml \v{13}The fear of the \divine{Lord} is to hate evil. \\
\poemll    Pride, arrogance, an evil lifestyle, \\
\poemlll       and perverted speech I despise. \\
\poeml \v{14}Counsel belongs to me, \\
\poemll    along with sound judgment. \\
\poeml I am understanding. \\
\poemll    Power belongs to me. \\
\poeml \v{15}Kings reign by me, \\
\poemll    and rulers dispense justice through me. \\
\poeml \v{16}By me leaders rule, as do noble officials \\
\poemll    and all who govern justly.\fnote{\fbackref{8:16} So MT; LXX reads \fbib{and tyrants rule the earth}} \\
\poeml \v{17}I love those who love me, \\
\poemll    and those who seek me will find me. \\
\poeml \v{18}Wealth and honor accompany me, \\
\poemll    as do enduring wealth and righteousness. \\
\poeml \v{19}My fruit is better than gold, \\
\poemll    better\fnote{\fbackref{8:19} The Heb. lacks \fbib{better}} than even refined gold, \\
\poemlll       and my benefit surpasses the purest silver. \\
\poeml \v{20}I walk on the way of righteousness, \\
\poemll    along paths that are just, \\
\poeml \v{21}I bequeath wealth to those who love me, \\
\poemll    and I will fill their treasuries.''
\passage{The Agelessness of Wisdom}
\poeml \v{22}``The \divine{Lord} made me as he began his planning,\fnote{\fbackref{8:22} Lit. \fbib{ways}} \\
\poemll    before his ancient activity commenced. \\
\poeml \v{23}From eternity I was appointed, \\
\poemll    from the beginning, \\
\poemlll       from before there was land. \\
\poeml \v{24}When there were no ocean depths, \\
\poemll    I brought them\fnote{\fbackref{8:24} The Heb. lacks \fbib{them}} to birth \\
\poemlll       at a time when there were no springs. \\
\poeml \v{25}Before the mountains were shaped, \\
\poemll    before there were hills, \\
\poemlll       I was bringing them\fnote{\fbackref{8:25} The Heb. lacks \fbib{them}} to birth. \\
\poeml \v{26}Even though he had not made the earth, nor the fields, \\
\poemll    nor the world's first grains of dust, \\
\poeml \v{27}when he crafted the heavens, \\
\poemll    I was there--- \\
\poemlll       when he marked out a circle on the face of the deep, \\
\poeml \v{28}when he made the clouds from above, \\
\poemll    when the springs of the depths were established, \\
\poeml \v{29}when he set a boundary for the sea \\
\poemll    so the waters would not exceed his limits,\fnote{\fbackref{8:29} Lit. \fbib{command}} \\
\poemlll       when he marked out the foundations of the earth. \\
\poeml \v{30}Then I was with him, his master craftsman--- \\
\poemll    I was his delight\fnote{\fbackref{8:30} So LXX; MT reads \fbib{was filled with delight}} daily, \\
\poemlll       continuously rejoicing in his presence, \\
\poeml \v{31}rejoicing in his inhabitable world \\
\poemll    and taking delight in mankind.''
\passage{The Exhortation of Wisdom}
\poeml \v{32}``So listen to me, children! \\
\poemll    Blessed are those who obey me. \\
\poeml \v{33}Listen to instruction and be wise. \\
\poemll    Don't ignore it. \\
\poeml \v{34}Blessed is the person who listens to me, \\
\poemll    watching daily at my gates, \\
\poemlll       waiting at my doorways--- \\
\poeml \v{35}because those who find me find life \\
\poemll    and gain favor from the \divine{Lord}. \\
\poeml \v{36}But whoever sins against me destroys himself; \\
\poemll    everyone who hates me loves death.''
\end{poetry}
\labelchapt{9}
\passage{Wisdom's Invitation}

\begin{poetry}
\poeml \chapt{9}
\v{1}Wisdom\fnote{\fbackref{9:1} I.e. \fbib{wisdom} personified as a woman} has built her house; \\
\poeml she has hewn out her seven pillars. \\
\poeml \v{2}She has prepared her food,\fnote{\fbackref{9:2} Or \fbib{meat}} \\
\poemll    she has spiced\fnote{\fbackref{9:2} Or \fbib{mixed}} her wine, \\
\poemlll       and she also has set her dining table. \\
\poeml \v{3}She has sent out her young women, \\
\poemll    while calling out from the heights of the city, \\
\poeml \v{4}``Let whoever is na\"{i}ve, turn in here.'' \\
\poemll    To anyone lacking sense, she says, \\
\poeml \v{5}``Come! Eat my food, \\
\poemll    and drink the wine that I have mixed. \\
\poeml \v{6}Leave your na\"{i}ve ways, and live. \\
\poemll    Walk in the path of understanding.''
\passage{Wisdom Extends Life}
\poeml \v{7}Whoever corrects a mocker invites only insult,\fnote{\fbackref{9:7} Lit. \fbib{insult to himself}} \\
\poemll    and whoever rebukes the wicked will himself become stained. \\
\poeml \v{8}Don't rebuke a mocker or he will hate you. \\
\poemll    Rebuke a wise person, and he will love you. \\
\poeml \v{9}Counsel a wise man, \\
\poemll    and he will be wiser still; \\
\poeml teach a righteous man, \\
\poemll    and he will add to his learning. \\
\poeml \v{10}The fear of the \divine{Lord} is where wisdom begins, \\
\poemll    and knowing holiness\fnote{\fbackref{9:10} Or \fbib{knowing holy ones}} demonstrates understanding. \\
\poeml \v{11}For because of me you will live a long life, \\
\poemll    and years will be added to your life. \\
\poeml \v{12}If you are wise, \\
\poemll    your wisdom will assist you. \\
\poeml If you mock, \\
\poemll    you alone will be held responsible.
\passage{Folly's Entrapment}
\poeml \v{13}The foolish woman is loud, \\
\poemll    undisciplined, and without knowledge. \\
\poeml \v{14}She sits at the entrance of her house, \\
\poemll    on a seat high above the city. \\
\poeml \v{15}She calls out to those passing by on the road, \\
\poemll    who are minding their own business,\fnote{\fbackref{9:15} Or \fbib{are going straight on their way}} \\
\poeml \v{16}``Whoever is na\"{i}ve, turn in here!'' \\
\poemll    And to anyone lacking sense, she says, \\
\poeml \v{17}``Stolen waters are sweet, \\
\poemll    and food eaten in secret is delicious.'' \\
\poeml \v{18}But he does not realize that the dead lurk there, \\
\poemll    and her invited guests wind up in the depths of Sheol.\fnote{\fbackref{9:18} I.e. the realm of the dead}
\end{poetry}
\labelchapt{10}
\passage{Solomon's Sayings}

\chapt{10}
\v{1}The proverbs of Solomon.

\begin{poetry}
\poeml A wise son brings joy to his father, \\
\poemll    but a foolish son grieves his mother. \\
\poeml \v{2}Nothing good comes from ill-gotten wealth, \\
\poemll    but righteousness delivers from death. \\
\poeml \v{3}The \divine{Lord} won't cause the righteous to hunger, \\
\poemll    but he will reject what the wicked crave. \\
\poeml \v{4}Lazy hands bring poverty, \\
\poemll    but hard-working hands lead to wealth. \\
\poeml \v{5}Whoever harvests during summer acts wisely, \\
\poemll    but the son who sleeps during harvest is disgraceful.
\passage{The Righteous and Wicked Compared}
\poeml \v{6}Blessings come\fnote{\fbackref{10:6} The Heb. lacks \fbib{come}} upon the head of the righteous, \\
\poemll    but the words\fnote{\fbackref{10:6} Lit. \fbib{mouth}} of the wicked conceal violence. \\
\poeml \v{7}The reputation\fnote{\fbackref{10:7} Lit. \fbib{memorial}} of the righteous leads to blessing, \\
\poemll    but the name of the wicked will rot. \\
\poeml \v{8}The wise person\fnote{\fbackref{10:8} Lit. \fbib{wise in heart}} accepts commands, \\
\poemll    but the chattering fool will be brought down. \\
\poeml \v{9}Whoever walks in integrity lives prudently,\fnote{\fbackref{10:9} Lit. \fbib{lives in safety}} \\
\poemll    but whoever perverts his way of life will be exposed. \\
\poeml \v{10}Those who wink their eyes\fnote{\fbackref{10:10} I.e. Those whose looks communicate insincerity} are trouble makers, \\
\poemll    and the mocking fool will be brought down.\fnote{\fbackref{10:10} So MT; LXX reads \fbib{makers, but the one who reproves publicly makes peace}} \\
\poeml \v{11}What the righteous say\fnote{\fbackref{10:11} Lit. \fbib{The mouth of the righteous}} is a flowing fountain,\fnote{\fbackref{10:11} Lit. \fbib{a fountain of life}} \\
\poemll    but what the wicked say\fnote{\fbackref{10:11} Lit. \fbib{but the mouth of the wicked}} conceals violence. \\
\poeml \v{12}Hatred awakens contention, \\
\poemll    but love covers all transgressions. \\
\poeml \v{13}Wisdom characterizes the speech\fnote{\fbackref{10:13} Lit. \fbib{Wisdom is found on the lips}} of the discerning, \\
\poemll    but the rod is for the backs of those lacking discernment. \\
\poeml \v{14}Those who are wise store up knowledge, \\
\poemll    but when the fool speaks,\fnote{\fbackref{10:14} Lit. \fbib{but the mouth of the fool}} destruction is near. \\
\poeml \v{15}The rich hide within the fortress that is their wealth, \\
\poemll    but the poor are dismayed due to their poverty. \\
\poeml \v{16}Honorable wages lead\fnote{\fbackref{10:16} The Heb. lacks \fbib{lead}} to life; \\
\poemll    the salaries of the wicked, to retribution. \\
\poeml \v{17}Whoever heeds correction is on the pathway to life, \\
\poemll    but someone who ignores exhortation goes astray. \\
\poeml \v{18}Whoever conceals hatred is a deceitful liar, \\
\poemll    and whoever spreads slander is a fool. \\
\poeml \v{19}Transgression is at work where people talk too much, \\
\poemll    but anyone who holds his tongue is prudent. \\
\poeml \v{20}What the righteous person says\fnote{\fbackref{10:20} Lit. \fbib{The tongue of the righteous}} is like precious silver; \\
\poemll    the thoughts of the wicked are compared to small things. \\
\poeml \v{21}What the righteous person says\fnote{\fbackref{10:21} Lit. \fbib{The lips of the righteous}} nourishes many, \\
\poemll    but fools die because they lack discerning\fnote{\fbackref{10:21} The Heb. lacks \fbib{discerning}} hearts. \\
\poeml \v{22}The blessing of the \divine{Lord} establishes wealth, \\
\poemll    and difficulty does not accompany it. \\
\poeml \v{23}Just as the fool considers wickedness his joy, \\
\poemll    so is wisdom to the discerning man. \\
\poeml \v{24}What the wicked fears will come about, \\
\poemll    but the longing of the righteous will be granted. \\
\poeml \v{25}When the storm ends, the wicked vanish,\fnote{\fbackref{10:25} Lit. \fbib{wicked are no more}} \\
\poemll    but the righteous person is forever firm. \\
\poeml \v{26}As vinegar is to the mouth\fnote{\fbackref{10:26} Lit. \fbib{teeth}} and smoke to the eyes, \\
\poemll    so is the lazy person to those who send him. \\
\poeml \v{27}Fearing the \divine{Lord} prolongs life, \\
\poemll    but the wicked will not live long. \\
\poeml \v{28}What the righteous hope for brings joy, \\
\poemll    but the expectation of the wicked dies. \\
\poeml \v{29}To the upright, the way of the \divine{Lord} is a place of safety, \\
\poemll    but it's a place of ruin to those who practice evil. \\
\poeml \v{30}The righteous will never be overthrown, \\
\poemll    but the wicked will never inhabit the land. \\
\poeml \v{31}The words of the righteous overflow with wisdom, \\
\poemll    but the perverse tongue will be cut out. \\
\poeml \v{32}Righteous lips know what is prudent, \\
\poemll    but the words of the wicked are perverse.
\end{poetry}
\labelchapt{11}
\passage{The Value of Righteousness}

\begin{poetry}
\poeml \chapt{11}
\v{1}The \divine{Lord} hates false scales, \\
\poeml but he delights in accurate weights. \\
\poeml \v{2}When pride appears, disgrace accompanies it, \\
\poemll    but humility is present with wisdom. \\
\poeml \v{3}The integrity of the righteous guides them, \\
\poemll    but the hypocrisy of the treacherous destroys them. \\
\poeml \v{4}Wealth won't help in the time of judgment,\fnote{\fbackref{11:4} Lit. \fbib{the day of wrath}} \\
\poemll    but righteousness will deliver from death. \\
\poeml \v{5}The righteousness of the innocent creates a level path, \\
\poemll    but the wicked fall by their wickedness. \\
\poeml \v{6}The righteousness of the upright delivers them, \\
\poemll    but the treacherous are trapped by their evil desires. \\
\poeml \v{7}When a wicked person dies, his hope vanishes;\fnote{\fbackref{11:7} So MT; LXX reads \fbib{When a righteous man dies, his hope does not perish}} \\
\poemll    and what he\fnote{\fbackref{11:7} So MT; LXX reads \fbib{what the ungodly}} expected from his scheming comes to nothing. \\
\poeml \v{8}The righteous person is delivered from trouble; \\
\poemll    it comes upon the wicked instead. \\
\poeml \v{9}By what he says, the godless person can destroy his neighbor, \\
\poemll    but through knowledge the righteous escape. \\
\poeml \v{10}The city rejoices when the righteous prosper, \\
\poemll    and when the wicked perish there is jubilation. \\
\poeml \v{11}Through the blessing of the righteous a city is built up, \\
\poemll    but what the wicked say tears it down. \\
\poeml \v{12}Whoever belittles his neighbor lacks sense, \\
\poemll    but the discerning man controls his comments. \\
\poeml \v{13}Whoever spreads gossip betrays secrets, \\
\poemll    but the trustworthy person\fnote{\fbackref{11:13} Lit. \fbib{trustworthy in spirit}} keeps a confidence. \\
\poeml \v{14}A nation falls through a lack of guidance, \\
\poemll    but victory comes through the counsel of many.\fnote{\fbackref{11:14} Or \fbib{through much planning}} \\
\poeml \v{15}Securing a loan for a stranger will bring suffering, \\
\poemll    but by refusing to do so, one remains safe. \\
\poeml \v{16}A gracious woman attains honor,\fnote{\fbackref{11:16} So MT; LXX reads \fbib{honor for her husband}} \\
\poemll    but\fnote{\fbackref{11:16} So MT; LXX reads \fbib{but a seat of dishonor is for the woman who hates justice}} ruthless men attain\fnote{\fbackref{11:16} So MT; LXX reads \fbib{justice. The deficient shrink from wealth, but the diligent support themselves with}} wealth. \\
\poeml \v{17}A gracious man benefits himself, \\
\poemll    but the cruel person damages himself. \\
\poeml \v{18}Evil people earn deceptive wages, \\
\poemll    but those who plant righteousness are truly rewarded. \\
\poeml \v{19}Genuine righteousness leads to life, \\
\poemll    but whoever pursues evil will die. \\
\poeml \v{20}Devious minds are abhorrent to the \divine{Lord}, \\
\poemll    but those whose ways are innocent are his delight. \\
\poeml \v{21}Be sure of this:\fnote{\fbackref{11:21} Lit. \fbib{Hand to hand}} the wicked will not go unpunished, \\
\poemll    but the descendants of the righteous will go free. \\
\poeml \v{22}Like a gold ring in a pig's snout \\
\poemll    is a beautiful woman without discretion. \\
\poeml \v{23}The desire of the righteous is to seek good, \\
\poemll    but the hope of the wicked results in wrath. \\
\poeml \v{24}Those who give freely gain even more; \\
\poemll    others hold back what they owe, becoming even poorer. \\
\poeml \v{25}A generous person will prosper, \\
\poemll    and anyone who gives water will receive a flood in return. \\
\poeml \v{26}People will curse whoever withholds grain, \\
\poemll    but blessing will come to whoever is selling. \\
\poeml \v{27}The person seeking good will find favor, \\
\poemll    but anyone who searches for evil---it will find him! \\
\poeml \v{28}The person who trusts in his wealth will fall, \\
\poemll    but the righteous will flourish like green leaves. \\
\poeml \v{29}Whoever troubles his household will inherit the wind, \\
\poemll    and the fool will be a servant to the wise. \\
\poeml \v{30}The fruit of the righteous is\fnote{\fbackref{11:30} So MT; LXX reads \fbib{From the fruit of righteousness grows}} a tree of life, \\
\poemll    and the one who wins people is wise.\fnote{\fbackref{11:30} So MT; LXX reads \fbib{life, but the souls of those who practice evil are cut off prematurely}} \\
\poeml \v{31}If the righteous receive what they are due here on earth, \\
\poemll    how much more will the wicked and the sinner.
\end{poetry}
\labelchapt{12}
\passage{Wisdom and Wickedness Contrasted}

\begin{poetry}
\poeml \chapt{12}
\v{1}The person who loves correction loves knowledge, \\
\poeml but anyone who hates a rebuke is stupid. \\
\poeml \v{2}The good person will gain favor from the \divine{Lord}, \\
\poemll    but the man who plots evil will be condemned by him. \\
\poeml \v{3}A person doesn't gain security by wickedness, \\
\poemll    but the righteous won't be uprooted. \\
\poeml \v{4}A virtuous woman is a crown to her husband, \\
\poemll    but a wife\fnote{\fbackref{12:4} Lit. \fbib{but she}} who puts him to shame is like bone cancer.\fnote{\fbackref{12:4} Lit. \fbib{decay}} \\
\poeml \v{5}The plans of the righteous are just, \\
\poemll    but the advice of the wicked is deceitful. \\
\poeml \v{6}The words of the wicked lead to\fnote{\fbackref{12:6} Lit. \fbib{wicked lie in wait for}} bloodshed, \\
\poemll    but the speech of the upright delivers them. \\
\poeml \v{7}After they're overthrown, the wicked won't be found, \\
\poemll    but the house of the righteous stands firm. \\
\poeml \v{8}A man is praised because of his wise words, \\
\poemll    but the perverted mind\fnote{\fbackref{12:8} Lit. \fbib{heart}} will be despised. \\
\poeml \v{9}It's better to be unimportant, yet have a servant, \\
\poemll    than to pretend to be important, but lack food. \\
\poeml \v{10}The righteous person looks out for the welfare of his livestock, \\
\poemll    but even\fnote{\fbackref{12:10} The Heb. lacks \fbib{even}} the compassion of the wicked is cruel. \\
\poeml \v{11}Whoever tills his soil will have a lot to eat, \\
\poemll    but anyone who pursues fantasies lacks sense.\fnote{\fbackref{12:11} Lit. \fbib{heart}} \\
\poeml \v{12}The wicked desires what evil people gain, \\
\poemll    but the foundation\fnote{\fbackref{12:12} Or \fbib{root}} of the righteous is productive. \\
\poeml \v{13}An evil man's sinful speech ensnares him, \\
\poemll    but the righteous person escapes from trouble. \\
\poeml \v{14}By his fruitful speech a man can remain satisfied, \\
\poemll    and a man's handiwork will reward him. \\
\poeml \v{15}The lifestyle of the fool is right in his own opinion, \\
\poemll    but wise is the man who listens to advice. \\
\poeml \v{16}The anger of a fool becomes readily apparent, \\
\poemll    but the prudent person overlooks an insult. \\
\poeml \v{17}The truth teller speaks what is right, \\
\poemll    but the false witness speaks what is\fnote{\fbackref{12:17} The Heb. lacks \fbib{speaks what is}} deceitful. \\
\poeml \v{18}Some speak rashly like the cutting of a sword, \\
\poemll    but what the wise say promotes healing. \\
\poeml \v{19}A truthful saying\fnote{\fbackref{12:19} Lit. \fbib{lips}} is trusted forever, \\
\poemll    but the liar\fnote{\fbackref{12:19} Lit. \fbib{the lying tongue}} only for a moment. \\
\poeml \v{20}Deceit is at home\fnote{\fbackref{12:20} The Heb. lacks \fbib{at home}} in the heart of those who plan evil, \\
\poemll    but those who promote peace rejoice. \\
\poeml \v{21}No harm overwhelms the righteous, \\
\poemll    but the wicked overflow with trouble. \\
\poeml \v{22}Deceitful speech is reprehensible to the \divine{Lord}, \\
\poemll    but those who act faithfully are his delight. \\
\poeml \v{23}A prudent man keeps what he knows to himself,\fnote{\fbackref{12:23} The Heb. lacks \fbib{to himself}} \\
\poemll    but the hearts of fools shout forth their foolishness. \\
\poeml \v{24}The diligent will take control, \\
\poemll    but the lazy will be put to forced labor. \\
\poeml \v{25}A person's anxiety weighs down his heart, \\
\poemll    but an appropriate word is encouraging. \\
\poeml \v{26}The righteous person is cautious with respect to his neighbor, \\
\poemll    but the lifestyle of the wicked leads them astray. \\
\poeml \v{27}The lazy person does not roast what he has hunted, \\
\poemll    but diligence is one's most important possession. \\
\poeml \v{28}In the pathway to righteousness there is life, \\
\poemll    and in that lifestyle there is no death.
\end{poetry}
\labelchapt{13}
\passage{Who is a Wise Son?}

\begin{poetry}
\poeml \chapt{13}
\v{1}A wise son heeds\fnote{\fbackref{13:1} The Heb. lacks \fbib{heeds}} a father's correction, \\
\poeml but a mocker does not listen to rebuke. \\
\poeml \v{2}From the fruit of his words a man receives benefit,\fnote{\fbackref{13:2} Lit. \fbib{man eats good things}} \\
\poemll    but the treacherous crave violence. \\
\poeml \v{3}Anyone who guards his words protects his life; \\
\poemll    anyone who talks too much\fnote{\fbackref{13:3} Lit. \fbib{who opens wide his lips}} is ruined. \\
\poeml \v{4}The lazy person craves, yet receives nothing, \\
\poemll    but the desires of the diligent are satisfied. \\
\poeml \v{5}A righteous person hates deceit, \\
\poemll    but the wicked person is shameful and disgraceful. \\
\poeml \v{6}Righteousness protects the blameless, \\
\poemll    but wickedness brings down\fnote{\fbackref{13:6} So MT DSS 4QProv\textsuperscript{b}; LXX reads \fbib{but sins ruin the wicked}} the sinner. \\
\poeml \v{7}One person pretends to be wealthy, but has nothing; \\
\poemll    another pretends to be poor, yet is rich. \\
\poeml \v{8}The life of a wealthy man may be held for ransom, \\
\poemll    but whoever is poor receives no threats. \\
\poeml \v{9}The light of the righteous shines, \\
\poemll    but the lamp of the wicked is extinguished. \\
\poeml \v{10}Arrogance only brings quarreling, \\
\poemll    but those receiving advice are wise. \\
\poeml \v{11}Wealth gained dishonestly dwindles away, \\
\poemll    but whoever works diligently increases his prosperity.\fnote{\fbackref{13:11} The Heb. lacks \fbib{his prosperity}} \\
\poeml \v{12}Delayed hope makes the heart ill, \\
\poemll    but fulfilled longing is a tree of life. \\
\poeml \v{13}Anyone who despises a word of advice will pay for it, \\
\poemll    but whoever heeds a command will be rewarded. \\
\poeml \v{14}What the wise have to teach is a fountain of life \\
\poemll    and causes someone to avoid the snares of death. \\
\poeml \v{15}Good understanding produces grace, \\
\poemll    but the lifestyle of the treacherous never changes.\fnote{\fbackref{13:15} So MT; LXX Syr read \fbib{grace, and to know the Law is the sign of a sound mind, but the path of scorners ends in destruction}} \\
\poeml \v{16}Every sensible person acts from knowledge, \\
\poemll    but a fool demonstrates folly. \\
\poeml \v{17}An evil messenger stumbles into trouble, \\
\poemll    but a faithful envoy brings healing. \\
\poeml \v{18}Poverty and shame are for those who ignore correction, \\
\poemll    but whoever listens to instruction gains honor. \\
\poeml \v{19}Fulfilled longing is sweet to the soul, \\
\poemll    but avoiding evil is detestable to the fool. \\
\poeml \v{20}Whoever keeps company with the wise becomes wise, \\
\poemll    but the companion of fools suffers harm. \\
\poeml \v{21}Disaster pursues the sinful, \\
\poemll    but good will reward the righteous. \\
\poeml \v{22}A good person leaves an inheritance to his grandchildren, \\
\poemll    but the wealth of the wicked is reserved for the righteous. \\
\poeml \v{23}The field of the poor may produce much food, \\
\poemll    but it can be swept away through injustice. \\
\poeml \v{24}Whoever does not discipline\fnote{\fbackref{13:24} Lit. \fbib{Whoever spares the rod}} his son hates him, \\
\poemll    but whoever loves him is diligent to correct him. \\
\poeml \v{25}A righteous person eats to his heart's content, \\
\poemll    but the stomach of the wicked remains hungry.
\end{poetry}
\labelchapt{14}
\passage{How Wise People Live}

\begin{poetry}
\poeml \chapt{14}
\v{1}Every wise woman builds up her household, \\
\poeml but the foolish one tears it down with her own hands. \\
\poeml \v{2}Someone whose conduct is upright fears the \divine{Lord}, \\
\poemll    but whoever is devious in his ways despises him. \\
\poeml \v{3}What a fool says brings\fnote{\fbackref{14:3} Lit. \fbib{The mouth of the fool}} a rod to his back, \\
\poemll    but the words of the wise protect them. \\
\poeml \v{4}Where there are no oxen, the feeding trough is clean, \\
\poemll    but profits come through the strength of the ox. \\
\poeml \v{5}A trustworthy witness does not deceive, \\
\poemll    but a false witness spews lies. \\
\poeml \v{6}A mocker seeks wisdom and finds\fnote{\fbackref{14:6} The Heb. lacks \fbib{finds}} none, \\
\poemll    but learning comes easily to someone who understands. \\
\poeml \v{7}Stay away from a foolish man, \\
\poemll    for you will not find competent advice. \\
\poeml \v{8}The wisdom of the prudent helps him know how to live, \\
\poemll    but a fool's stupidity deceives him. \\
\poeml \v{9}Fools make fun of guilt, \\
\poemll    but among the upright there are good intentions. \\
\poeml \v{10}The heart knows its own bitterness--- \\
\poemll    an outsider cannot share in its joy. \\
\poeml \v{11}The house of the wicked will be destroyed, \\
\poemll    but the tent of the upright will flourish. \\
\poeml \v{12}There is a pathway that seems right to a man, \\
\poemll    but in the end it's a road to death. \\
\poeml \v{13}Even in laughter there may be heartache, \\
\poemll    and at the end of joy there may be grief. \\
\poeml \v{14}The faithless one will pay for his behavior,\fnote{\fbackref{14:14} Lit. \fbib{ways}} \\
\poemll    but a good man will be rewarded\fnote{\fbackref{14:14} The Heb. lacks \fbib{will be rewarded}} for his. \\
\poeml \v{15}An unthinking person believes everything, \\
\poemll    but the prudent one thinks before acting.\fnote{\fbackref{14:15} Lit. \fbib{one considers his steps}} \\
\poeml \v{16}The wise person fears and turns away from evil, \\
\poemll    but a fool is reckless and overconfident. \\
\poeml \v{17}A quick tempered person does foolish things, \\
\poemll    and a devious man is hated. \\
\poeml \v{18}The na\"{i}ve inherit folly, \\
\poemll    but the careful are crowned with knowledge. \\
\poeml \v{19}Evil men will bow down in the presence of good men \\
\poemll    and the wicked at the gates of the righteous. \\
\poeml \v{20}The poor person is shunned by his neighbor, \\
\poemll    but many are the friends of the wealthy. \\
\poeml \v{21}Whoever despises his neighbor sins, \\
\poemll    but whoever shows kindness to the poor will be happy. \\
\poeml \v{22}Won't those who plot evil go astray? \\
\poemll    But gracious love and truth are for those who plan what is good. \\
\poeml \v{23}In hard work there is always profit, \\
\poemll    but too much chattering\fnote{\fbackref{14:23} Lit. \fbib{word of lips}} leads to poverty. \\
\poeml \v{24}The crown of the wise is their wealth, \\
\poemll    but the stupidity of fools is just that---stupidity! \\
\poeml \v{25}A truthful witness saves lives, \\
\poemll    but the person who lies is deceitful. \\
\poeml \v{26}Rock-solid security is found\fnote{\fbackref{14:26} The Heb. lacks \fbib{is found}} in the fear of the \divine{Lord}, \\
\poemll    and within it one's children find refuge. \\
\poeml \v{27}The fear of the \divine{Lord} is a fountain of life, \\
\poemll    enabling anyone to escape the snares of death. \\
\poeml \v{28}A large population is a king's glory, \\
\poemll    but a shortage of people is a ruler's ruin. \\
\poeml \v{29}Being slow to get angry compares to great understanding \\
\poemll    as being quick-tempered compares to stupidity. \\
\poeml \v{30}A tranquil mind brings life to one's body, \\
\poemll    but jealousy causes one's bones to rot. \\
\poeml \v{31}Whoever oppresses the poor defies their Creator, \\
\poemll    but whoever is kind to the needy honors them. \\
\poeml \v{32}The wicked person is thrown down by his own wrongdoing, \\
\poemll    but the righteous person has a place of safety in death.\fnote{\fbackref{14:32} So MT DSS 4QProv\textsuperscript{b}; LXX reads \fbib{in his own piety}} \\
\poeml \v{33}Wisdom is at rest in the mind of the discerning--- \\
\poemll    even fools know this.\fnote{\fbackref{14:33} So MT; LXX reads \fbib{but in the heart of fools it is not discerned}} \\
\poeml \v{34}Righteousness makes a nation great, \\
\poemll    but sin diminishes\fnote{\fbackref{14:34} So DSS 4QPro\textsuperscript{b} LXX; MT reads \fbib{sin is a disgrace to}} any people. \\
\poeml \v{35}The king approves the wise servant, \\
\poemll    but he is angry at anyone who acts shamefully.
\end{poetry}
\labelchapt{15}
\passage{How to Live Wisely}

\begin{poetry}
\poeml \chapt{15}
\v{1}A gentle response diverts anger, \\
\poeml but a harsh statement incites fury. \\
\poeml \v{2}The wise speak, presenting\fnote{\fbackref{15:2} Lit. \fbib{The tongues of the wise present}} knowledge appropriately, \\
\poemll    but fools spout foolishness. \\
\poeml \v{3}The eyes of the \divine{Lord} are in every place, \\
\poemll    observing both the evil and the good. \\
\poeml \v{4}A gentle statement\fnote{\fbackref{15:4} Lit. \fbib{tongue}} is a tree of life, \\
\poemll    but perverted speech shatters the spirit. \\
\poeml \v{5}A fool rejects his father's instructions, \\
\poemll    but anyone who respects\fnote{\fbackref{15:5} Lit. \fbib{keeps}} reproof acts sensibly. \\
\poeml \v{6}The righteous house is itself\fnote{\fbackref{15:6} The Heb. lacks \fbib{itself}} a great treasure, \\
\poemll    but within the revenue of the wicked calamity is at work. \\
\poeml \v{7}What the wise have to say disseminates\fnote{\fbackref{15:7} Lit. \fbib{The lips of the wise spread}} knowledge, \\
\poemll    but it's not in the heart of fools to do so. \\
\poeml \v{8}The sacrifice of the wicked is detestable to the \divine{Lord}, \\
\poemll    but the prayer of the upright is his delight. \\
\poeml \v{9}The lifestyle of the wicked is detestable to the \divine{Lord}, \\
\poemll    but he loves those who ardently pursue righteousness. \\
\poeml \v{10}Severe punishment awaits anyone who wanders off the path--- \\
\poemll    anyone who despises reproof will die. \\
\poeml \v{11}Since Sheol\fnote{\fbackref{15:11} I.e. the realm of the dead} and Abaddon\fnote{\fbackref{15:11} I.e. the realm of destruction in the afterlife} lie open in the \divine{Lord}'s presence, \\
\poemll    how much more the hearts of human beings! \\
\poeml \v{12}The arrogant mocker never loves the one who corrects him; \\
\poemll    he will not inquire of\fnote{\fbackref{15:12} Lit. \fbib{not go to}} the wise. \\
\poeml \v{13}A happy heart enlightens the face, \\
\poemll    but a sad heart reflects a broken spirit. \\
\poeml \v{14}A discerning mind seeks knowledge, \\
\poemll    but the mouth of fools feeds on stupidity. \\
\poeml \v{15}The entire life\fnote{\fbackref{15:15} Lit. \fbib{All the days}} of the afflicted seems disastrous, \\
\poemll    but a good heart feasts continuously.
\passage{On Contentment and Other Good Things of Life}
\poeml \v{16}Better is a little accompanied by fear of the \divine{Lord} \\
\poemll    than abundant wealth with turmoil. \\
\poeml \v{17}A vegetarian meal\fnote{\fbackref{15:17} Lit. \fbib{A meal of herbs}} served with love is better \\
\poemll    than a big, thick steak\fnote{\fbackref{15:17} Lit. \fbib{a fattened ox}} with a plateful of\fnote{\fbackref{15:17} The Heb. lacks \fbib{a plateful of}} animosity. \\
\poeml \v{18}The quickly angered man stirs up contention, \\
\poemll    but anyone who controls his temper calms a dispute. \\
\poeml \v{19}The lifestyle of the lazy is like a thorny hedge, \\
\poemll    but the path taken by the upright is an open highway. \\
\poeml \v{20}A wise son makes a father glad, \\
\poemll    but a foolish man despises his mother. \\
\poeml \v{21}Stupidity is the delight of the senseless, \\
\poemll    but an understanding man walks uprightly. \\
\poeml \v{22}Plans fail without advice, \\
\poemll    but with many counselors they are confirmed. \\
\poeml \v{23}An appropriate answer brings joy to a person, \\
\poemll    and a well-timed word is a good thing. \\
\poeml \v{24}The way of life leads upward for the wise \\
\poemll    so he may avoid Sheol\fnote{\fbackref{15:24} I.e. the realm of the dead} below. \\
\poeml \v{25}The house of the proud the \divine{Lord} will demolish, \\
\poemll    but he will protect the widow's boundary line. \\
\poeml \v{26}To the \divine{Lord} evil plans are detestable, \\
\poemll    but pleasant words are pure. \\
\poeml \v{27}Those who are greedy for unjust gain bring trouble into their homes, \\
\poemll    but the person who hates bribes will live. \\
\poeml \v{28}The mind of the righteous thinks before speaking, \\
\poemll    but the wicked person spews out evil. \\
\poeml \v{29}The \divine{Lord} is far away from the wicked, \\
\poemll    but he hears the prayers of the righteous. \\
\poeml \v{30}Bright eyes\fnote{\fbackref{15:30} Or \fbib{A cheerful look}} encourage the heart; \\
\poemll    good news nourishes the body.\fnote{\fbackref{15:30} Lit. \fbib{bones}} \\
\poeml \v{31}Whoever listens to a life-giving rebuke \\
\poemll    will be at home among the wise. \\
\poeml \v{32}Whoever ignores instruction hates himself, \\
\poemll    but anyone who heeds reproof gains understanding.\fnote{\fbackref{15:32} Lit. \fbib{heart}} \\
\poeml \v{33}The fear of the \divine{Lord} teaches wisdom, \\
\poemll    and humility precedes honor.
\end{poetry}
\labelchapt{16}
\passage{Wisdom's Blessings}

\begin{poetry}
\poeml \chapt{16}
\v{1}People do the planning,\fnote{\fbackref{16:1} Lit. \fbib{Preparations of the heart belong to human beings}} \\
\poeml but the end result\fnote{\fbackref{16:1} Or \fbib{the response of the tongue}} is from the \divine{Lord}. \\
\poeml \v{2}Everything a person does seems pure in his own opinion, \\
\poemll    but the \divine{Lord} weighs intentions. \\
\poeml \v{3}Entrust your work to the \divine{Lord}, \\
\poemll    and your planning will succeed. \\
\poeml \v{4}The \divine{Lord} made everything answerable to him, \\
\poemll    including the wicked at the time of trouble.\fnote{\fbackref{16:4} Lit. \fbib{evil}} \\
\poeml \v{5}The \divine{Lord} detests those who are proud; \\
\poemll    truly they will not go unpunished. \\
\poeml \v{6}Iniquity is atoned for by gracious love and truth, \\
\poemll    and through fear of the \divine{Lord} people\fnote{\fbackref{16:6} The Heb. lacks \fbib{people}} turn from evil. \\
\poeml \v{7}When a person's ways please the \divine{Lord}, \\
\poemll    even his enemies will be at peace with him. \\
\poeml \v{8}A little gain\fnote{\fbackref{16:8} The Heb. lacks \fbib{gain}} with righteousness is better \\
\poemll    than great income without justice. \\
\poeml \v{9}A person plans his way, \\
\poemll    but the \divine{Lord} directs his steps. \\
\poeml \v{10}When a king is ready to speak officially,\fnote{\fbackref{16:10} Lit. \fbib{king speaks an oracle}} \\
\poemll    what he says should not err with respect to justice. \\
\poeml \v{11}Honest scales and balances are from the \divine{Lord}; \\
\poemll    he made all the weights in the bag. \\
\poeml \v{12}Kings detest wrongdoing, \\
\poemll    for through righteousness the throne is established. \\
\poeml \v{13}Kings take pleasure in righteous speech; \\
\poemll    they treasure a person who speaks what is upright. \\
\poeml \v{14}The king's wrath results in a death sentence, \\
\poemll    but whoever is wise will appease him. \\
\poeml \v{15}When a king is pleased,\fnote{\fbackref{16:15} Lit. \fbib{a king's face lightens}} there is life, \\
\poemll    and his favor is like a cloud that brings spring rain. \\
\poeml \v{16}How much better than gaining gold is the acquisition of wisdom, \\
\poemll    the attainment of wisdom better than silver! \\
\poeml \v{17}The road of the upright circumvents evil, \\
\poemll    and whoever watches how he lives\fnote{\fbackref{16:17} Lit. \fbib{watches his path}} preserves his life. \\
\poeml \v{18}Pride precedes destruction; \\
\poemll    an arrogant spirit appears before a fall. \\
\poeml \v{19}Better to be humble among the poor, \\
\poemll    than to share what is stolen with the proud. \\
\poeml \v{20}Whoever listens to a word of instruction prospers, \\
\poemll    and anyone who trusts in the \divine{Lord} is blessed. \\
\poeml \v{21}The wise-hearted person is told to be discerning, \\
\poemll    and that pleasant speech promotes instruction. \\
\poeml \v{22}Anyone who has understanding is a fountain of life, \\
\poemll    but foolishness brings punishment to fools. \\
\poeml \v{23}A wise person's thoughts\fnote{\fbackref{16:23} Lit. \fbib{heart}} control his words, \\
\poemll    and his speech promotes instruction. \\
\poeml \v{24}Pleasant words are honey from a honeycomb--- \\
\poemll    sweet to the soul and healing for the body.\fnote{\fbackref{16:24} Lit. \fbib{bone}}
\passage{Advice to the Wise}
\poeml \v{25}There is a road that seems right for a man to travel,\fnote{\fbackref{16:25} The Heb. lacks \fbib{to travel}} \\
\poemll    but in the end it's the road to death. \\
\poeml \v{26}The appetite of the laborer motivates him; \\
\poemll    indeed, his hunger drives him on. \\
\poeml \v{27}A worthless person concocts evil gossip\fnote{\fbackref{16:27} The Heb. lacks \fbib{gossip}}--- \\
\poemll    his lips are like a burning fire. \\
\poeml \v{28}A deceitful man stirs dissension, \\
\poemll    and anyone who gossips separates friends. \\
\poeml \v{29}A violent man entices his companion \\
\poemll    and leads him on a path that is not good. \\
\poeml \v{30}Whoever winks knowingly\fnote{\fbackref{16:30} Lit. \fbib{with his eyes}} is plotting\fnote{\fbackref{16:30} So MT; LXX Syr Targ Vg read \fbib{winks with his eyes considers}} deceit; \\
\poemll    anyone who purses his lips is bent towards evil. \\
\poeml \v{31}Gray hair is a crown of glory; \\
\poemll    it is obtained by following\fnote{\fbackref{16:31} The Heb. lacks \fbib{following}} a righteous path. \\
\poeml \v{32}Whoever controls his temper is better than a warrior, \\
\poemll    and anyone who has control of his spirit is better \\
\poemlll       than someone who captures a city. \\
\poeml \v{33}The dice is cast into someone's lap, \\
\poemll    but the outcome is from the \divine{Lord}.
\end{poetry}
\labelchapt{17}
\passage{More Words of Wisdom}

\begin{poetry}
\poeml \chapt{17}
\v{1}Dry crumbs in peace\fnote{\fbackref{17:1} Lit. \fbib{quiet}} are better \\
\poeml than a full meal\fnote{\fbackref{17:1} Lit. \fbib{house full of meat}} with strife. \\
\poeml \v{2}A prudent servant will rule in place of a disgraceful son \\
\poemll    and will share in the inheritance among brothers. \\
\poeml \v{3}The crucible is for silver \\
\poemll    and the furnace for gold--- \\
\poemlll       but the \divine{Lord} assays hearts. \\
\poeml \v{4}Whoever practices evil pays attention to wicked speech, \\
\poemll    and the liar listens to malicious talk. \\
\poeml \v{5}Whoever mocks the poor shows contempt for their maker, \\
\poemll    and whoever is happy about disaster \\
\poemlll       will not go unpunished. \\
\poeml \v{6}Grandchildren are the crown of the aged, \\
\poemll    and the pride of children is their parents. \\
\poeml \v{7}Appropriate speech is inconsistent with the fool; \\
\poemll    how much more are deceitful statements\fnote{\fbackref{17:7} Lit. \fbib{lips}} with a prince! \\
\poeml \v{8}A bribe works wonders\fnote{\fbackref{17:8} Lit. \fbib{A gift is a stone of favor}} in the eyes of its giver; \\
\poemll    wherever he turns he prospers. \\
\poeml \v{9}Anyone who overlooks\fnote{\fbackref{17:9} Lit. \fbib{covers}} an offense promotes love, \\
\poemll    but someone who gossips separates close friends. \\
\poeml \v{10}A rebuke is more effective with a man of understanding \\
\poemll    than a hundred lashes to a fool. \\
\poeml \v{11}A rebellious person seeks evil; \\
\poemll    a cruel emissary will be sent to oppose him. \\
\poeml \v{12}It's better to meet a mother bear who has lost her cubs \\
\poemll    than a fool in his stupidity. \\
\poeml \v{13}The person who repays good with evil \\
\poemll    will never see\fnote{\fbackref{17:13} The Heb. lacks \fbib{will see}} evil leave his home. \\
\poeml \v{14}Starting a quarrel is like spilling water--- \\
\poemll    so drop the dispute before it escalates. \\
\poeml \v{15}Exonerating the wicked and condemning the righteous \\
\poemll    are both detestable to the \divine{Lord}. \\
\poeml \v{16}What is this? A fool has enough money to buy wisdom, \\
\poemll    but is senseless?\fnote{\fbackref{17:16} Lit. \fbib{but has no heart}} \\
\poeml \v{17}A friend loves at all times, \\
\poemll    and a brother is there\fnote{\fbackref{17:17} Lit. \fbib{born}} for times of trouble. \\
\poeml \v{18}A man who lacks sense\fnote{\fbackref{17:18} Lit. \fbib{heart}} cosigns a loan,\fnote{\fbackref{17:18} Lit. \fbib{sense strikes the palm}} \\
\poemll    becoming a guarantor for his neighbor. \\
\poeml \v{19}The person who loves transgression loves strife; \\
\poemll    the person who builds a high gate invites destruction. \\
\poeml \v{20}The person whose mind\fnote{\fbackref{17:20} Lit. \fbib{heart}} is perverse does not find good, \\
\poemll    and anyone with perverted speech falls into trouble. \\
\poeml \v{21}The man who fathers a fool does so to his sorrow--- \\
\poemll    the father of a fool has no joy. \\
\poeml \v{22}A joyful heart is good medicine, \\
\poemll    but a broken spirit drains one's strength.\fnote{\fbackref{17:22} Lit. \fbib{spirit dries the bones}} \\
\poeml \v{23}The wicked man takes a bribe in secret \\
\poemll    in order to pervert the course of justice. \\
\poeml \v{24}A person with understanding has wisdom as his objective, \\
\poemll    but a fool looks only\fnote{\fbackref{17:24} The Heb. lacks \fbib{only}} to earthly goals. \\
\poeml \v{25}A foolish son brings grief to his father \\
\poemll    and bitterness to his mother.\fnote{\fbackref{17:25} Lit. \fbib{to the one who bore him}} \\
\poeml \v{26}Furthermore, it isn't good to fine the righteous, \\
\poemll    or to beat an official because of his uprightness. \\
\poeml \v{27}Whoever controls what he says is knowledgeable; \\
\poemll    anyone who has a calm spirit is a man of understanding. \\
\poeml \v{28}Even a fool is thought to be wise when he remains silent; \\
\poemll    he is thought to be prudent when he keeps his mouth shut.
\end{poetry}
\labelchapt{18}
\passage{How Fools Talk}

\begin{poetry}
\poeml \chapt{18}
\v{1}Whoever isolates himself pursues selfish ends; \\
\poeml he resists all sound advice. \\
\poeml \v{2}A fool finds no satisfaction in trying to understand, \\
\poemll    for he would rather express his own opinion. \\
\poeml \v{3}When an evil person comes, contempt also comes, \\
\poemll    along with dishonor and disgrace. \\
\poeml \v{4}The words a man says are as deep waters--- \\
\poemll    a fountain of wisdom is an overflowing stream. \\
\poeml \v{5}It's not good to be partial towards an evil person, \\
\poemll    thereby depriving the righteous of justice. \\
\poeml \v{6}A fool's words\fnote{\fbackref{18:6} Lit. \fbib{lips}} bring strife, \\
\poemll    and his mouth invites fighting. \\
\poeml \v{7}A fool's mouth is his unraveling, \\
\poemll    and his lips entrap himself. \\
\poeml \v{8}The words of a gossip are like choice morsels \\
\poemll    as they descend to the innermost parts of the body.
\passage{Avoiding Fools and Their Foolishness}
\poeml \v{9}Whoever is lazy regarding his work \\
\poemll    is also a brother to the master of destruction. \\
\poeml \v{10}The name of the \divine{Lord} is a strong tower; \\
\poemll    a righteous person rushes to it and is lifted up above the danger.\fnote{\fbackref{18:10} The Heb. lacks \fbib{above the danger}} \\
\poeml \v{11}The wealth of a rich person is his fortified city; \\
\poemll    in his own imagination, it is like a high wall. \\
\poeml \v{12}Before a man's downfall, his mind\fnote{\fbackref{18:12} Lit. \fbib{heart}} is arrogant, \\
\poemll    but humility precedes honor. \\
\poeml \v{13}Whoever answers before listening \\
\poemll    is both foolish and shameful. \\
\poeml \v{14}A man's spirit can sustain him during his illness, \\
\poemll    but who can bear a crushed spirit? \\
\poeml \v{15}The mind\fnote{\fbackref{18:15} Lit. \fbib{heart}} of a discerning person gains knowledge, \\
\poemll    while the ears of wise people seek out knowledge. \\
\poeml \v{16}A person's gift opens doors for him, \\
\poemll    bringing him access to important people. \\
\poeml \v{17}The first to put forth his case seems right, \\
\poemll    until someone else steps forward and cross-examines him. \\
\poeml \v{18}Casting dice settles a dispute, \\
\poemll    deciding between strong contenders. \\
\poeml \v{19}An offended brother is more unyielding than a fortified city, \\
\poemll    and his disputes are like the bars of a fortress. \\
\poeml \v{20}The positive words that a man speaks\fnote{\fbackref{18:20} Lit. \fbib{words from a man's mouth}} fill his stomach; \\
\poemll    he will be satisfied with what his lips produce. \\
\poeml \v{21}The power of the tongue is life and death--- \\
\poemll    those who love to talk\fnote{\fbackref{18:21} Lit. \fbib{love it}} will eat what it produces. \\
\poeml \v{22}Whoever finds a wife finds what is good, \\
\poemll    and receives favor from the \divine{Lord}. \\
\poeml \v{23}The poor person pleads for mercy, \\
\poemll    but the wealthy man responds harshly. \\
\poeml \v{24}A man with many friends can still be ruined, \\
\poemll    but a true friend sticks closer than a brother.
\end{poetry}
\labelchapt{19}
\passage{The Priorities of Life Contrasted}

\begin{poetry}
\poeml \chapt{19}
\v{1}A poor man who walks blamelessly is better \\
\poeml than a fool who speaks perversely. \\
\poeml \v{2}Furthermore, it isn't good to be ignorant,\fnote{\fbackref{19:2} Lit. \fbib{good for an ignorant soul}} \\
\poemll    and whoever rushes into things\fnote{\fbackref{19:2} Lit. \fbib{whoever hurries with his feet}} misses the mark. \\
\poeml \v{3}A man's foolishness ruins his life,\fnote{\fbackref{19:3} Lit. \fbib{way}} \\
\poemll    yet his heart rages against the \divine{Lord}. \\
\poeml \v{4}Wealth brings many friends, \\
\poemll    but a poor man is deserted by his friend. \\
\poeml \v{5}A witness to lies will not go unpunished; \\
\poemll    the teller of falsehoods will not escape. \\
\poeml \v{6}Many curry favor of an official; \\
\poemll    everyone is a friend of the gift giver. \\
\poeml \v{7}All the relatives of a poor person shun him--- \\
\poemll    how much more do his friends avoid him! \\
\poeml Though he runs after them pleading, \\
\poemll    they aren't around. \\
\poeml \v{8}Whoever obtains wisdom loves himself, \\
\poemll    and whoever treasures understanding will prosper. \\
\poeml \v{9}A witness to lies will not go unpunished; \\
\poemll    the teller of falsehoods will perish. \\
\poeml \v{10}It's not fitting for a fool to live in luxury; \\
\poemll    neither is it for a servant to rule over princes. \\
\poeml \v{11}A person's discretion makes him slow to anger, \\
\poemll    and it is to his credit that he ignores an offence. \\
\poeml \v{12}The king's anger is like the roaring of a lion, \\
\poemll    but his goodwill is like dew on the grass. \\
\poeml \v{13}A father's ruin is a foolish son, \\
\poemll    and a wife's quarreling is like\fnote{\fbackref{19:13} The Heb. lacks \fbib{like}} dripping water that never stops. \\
\poeml \v{14}A house and self-sufficiency are a father's inheritance, \\
\poemll    but from the \divine{Lord} comes an insightful wife. \\
\poeml \v{15}Laziness puts one to sleep, \\
\poemll    and an idle person will go hungry. \\
\poeml \v{16}Whoever obeys a commandment keeps himself safe,\fnote{\fbackref{19:16} Lit. \fbib{keeps his soul}} \\
\poemll    but someone who is contemptuous in conduct\fnote{\fbackref{19:16} Lit. \fbib{in his way}} will die. \\
\poeml \v{17}Whoever is kind to the poor is lending to the \divine{Lord}--- \\
\poemll    the benefit of his gift will return to him in abundance. \\
\poeml \v{18}Discipline your son while there is still hope--- \\
\poemll    but don't set your heart on his destruction. \\
\poeml \v{19}The person who has great anger must pay the consequences, \\
\poemll    because if you rescue him, you will have to do it again. \\
\poeml \v{20}Listen to advice and accept discipline, \\
\poemll    and you'll be wise for the rest of your life.\fnote{\fbackref{19:20} The Heb. lacks \fbib{of your life}} \\
\poeml \v{21}Many plans occupy the mind\fnote{\fbackref{19:21} Lit. \fbib{heart}} of a man, \\
\poemll    but the \divine{Lord}'s purposes will prevail.\fnote{\fbackref{19:21} Or \fbib{will be established}} \\
\poeml \v{22}Human beings long for grace, \\
\poemll    and it's better to be poor than a man of deceit. \\
\poeml \v{23}The fear of the \divine{Lord} leads\fnote{\fbackref{19:23} The Heb. lacks \fbib{leads}} to life; \\
\poemll    whoever is satisfied with it will rest, \\
\poemlll       untouched by evil. \\
\poeml \v{24}The lazy person buries his hand in his dish \\
\poemll    and doesn't bother to bring it back to his mouth. \\
\poeml \v{25}If you scourge a scoffer, \\
\poemll    the simple person may learn to be discreet; \\
\poeml rebuke a discerning man \\
\poemll    and he will gain understanding. \\
\poeml \v{26}Whoever mistreats his father \\
\poemll    and alienates his mother \\
\poemlll       is a son who brings both shame and disrespect. \\
\poeml \v{27}My son, if you stop listening to instruction, \\
\poemll    you will stray from the principles of knowledge. \\
\poeml \v{28}A corrupt witness\fnote{\fbackref{19:28} I.e. a worthless person} mocks justice, \\
\poemll    and the wicked person feeds on iniquity. \\
\poeml \v{29}Condemnation is appropriate for mockers, \\
\poemll    just as beatings are for the backs of fools.
\end{poetry}
\labelchapt{20}
\passage{Advice on How to Live}

\begin{poetry}
\poeml \chapt{20}
\v{1}Wine causes mocking, and beer causes fights; \\
\poeml everyone led astray by them lacks wisdom. \\
\poeml \v{2}A king's anger is like a lion's roar; \\
\poemll    anyone who angers him forfeits his life. \\
\poeml \v{3}Avoiding strife brings a man honor, \\
\poemll    but every fool is quarrelsome. \\
\poeml \v{4}A lazy person doesn't plow in the proper\fnote{\fbackref{20:4} The Heb. lacks \fbib{proper}} season; \\
\poemll    he looks for a harvest, but there is nothing. \\
\poeml \v{5}The intentions of a person's heart are deep waters, \\
\poemll    but a discerning person reveals them. \\
\poeml \v{6}Many claim ``I'm a loyal person!''\fnote{\fbackref{20:6} Lit. \fbib{claim to be people of gracious love}} \\
\poemll    but who can find someone who truly is? \\
\poeml \v{7}The righteous person lives a life of integrity; \\
\poemll    happy are his children who follow him! \\
\poeml \v{8}A king sits on a throne of justice, \\
\poemll    sifting out all sorts of evil with his glance. \\
\poeml \v{9}Who can say, ``My intentions are pure; \\
\poemll    I am clean from any sin?'' \\
\poeml \v{10}False\fnote{\fbackref{20:10} Or \fbib{Diverse}} weights and measures--- \\
\poemll    the \divine{Lord} surely detests both of them. \\
\poeml \v{11}Even a child is known by his actions, \\
\poemll    whether his deeds are pure and right. \\
\poeml \v{12}The ear that hears and the eye that sees--- \\
\poemll    the \divine{Lord} surely made them both. \\
\poeml \v{13}Do not love sleep or you'll become poor, \\
\poemll    keep your eyes open and you'll have plenty of food. \\
\poeml \v{14}``This is bad, bad,'' says whoever is buying--- \\
\poemll    but then he brags as he walks away after the sale.\fnote{\fbackref{20:14} The Heb. lacks \fbib{after the sale}} \\
\poeml \v{15}There is an abundance of gold and precious stones, \\
\poemll    but lips of knowledge are a rare jewel. \\
\poeml \v{16}Take the garment of anyone who puts up collateral for a stranger; \\
\poemll    hold it in pledge if he does it for an unfamiliar woman. \\
\poeml \v{17}Bread gained by deceit is sweet to a man, \\
\poemll    but later his mouth will be full of gravel. \\
\poeml \v{18}Make plans by seeking advice; \\
\poemll    make war by obtaining guidance. \\
\poeml \v{19}Whoever spreads gossip betrays confidences; \\
\poemll    so don't get involved with someone who talks too much. \\
\poeml \v{20}Whoever curses his father or mother, \\
\poemll    his lamp will be extinguished in the deepest darkness. \\
\poeml \v{21}An inheritance quickly obtained at the beginning \\
\poemll    will not be blessed at the end. \\
\poeml \v{22}Don't say ``I'll avenge that wrong!'' \\
\poemll    Wait on the \divine{Lord} and he will deliver you. \\
\poeml \v{23}The \divine{Lord} detests differing weights, \\
\poemll    and dishonest scales are not good. \\
\poeml \v{24}A man's steps are directed by the \divine{Lord}; \\
\poemll    how then can anyone understand his own way? \\
\poeml \v{25}It is a trap for a person to declare quickly, ``This is sacred,'' \\
\poemll    and only later to have second thoughts about the vows. \\
\poeml \v{26}A wise king sifts the wicked, \\
\poemll    crushing them with the threshing wheel. \\
\poeml \v{27}A person's spirit is the lamp of the \divine{Lord}; \\
\poemll    it searches throughout one's innermost being. \\
\poeml \v{28}Gracious love and truth preserve a king; \\
\poemll    through love his throne is made secure. \\
\poeml \v{29}The glory of young men is their strength; \\
\poemll    and the splendor of elders is their gray hair. \\
\poeml \v{30}Blows that wound clean away evil; \\
\poemll    such beatings cleanse\fnote{\fbackref{20:30} The Heb. lacks \fbib{cleanse}} the innermost being.
\end{poetry}
\labelchapt{21}
\passage{Thoughts on the Sovereignty of God}

\begin{poetry}
\poeml \chapt{21}
\v{1}A king's heart is a water stream that the \divine{Lord} controls; \\
\poeml he directs it wherever he pleases. \\
\poeml \v{2}Every man's lifestyle is proper in his own view, \\
\poemll    but the \divine{Lord} weighs the heart. \\
\poeml \v{3}To do what is right and just \\
\poemll    is more acceptable to the \divine{Lord} than sacrifice.
\end{poetry}

\v{4}A proud attitude,\fnote{\fbackref{21:4} Lit. \fbib{heart}} accompanied by\fnote{\fbackref{21:4} Lit. \fbib{proud heart and}} a haughty look, is sin;

\begin{poetry}
\poemll    they reveal\fnote{\fbackref{21:4} Lit. \fbib{sin; the lamp of}} wicked people. \\
\poeml \v{5}Plans of the persistent surely lead to productivity, \\
\poemll    but all who are hasty will surely become poor. \\
\poeml \v{6}A fortune gained by deceit\fnote{\fbackref{21:6} Lit. \fbib{by a lying tongue}} \\
\poemll    is a fleeting vapor and a deadly snare.\fnote{\fbackref{21:6} So MT; LXX reads \fbib{is pursuing worthlessness into deadly snares}} \\
\poeml \v{7}Devastation caused by the wicked will drag them away \\
\poemll    because they refuse to do what is just. \\
\poeml \v{8}The conduct\fnote{\fbackref{21:8} Lit. \fbib{way}} of a guilty man is perverse, \\
\poemll    but the behavior of the pure is upright. \\
\poeml \v{9}It's better to live in a corner on the roof \\
\poemll    than to share a house with a contentious woman. \\
\poeml \v{10}The soul of the wicked craves evil; \\
\poemll    he extends no mercy to his neighbor. \\
\poeml \v{11}When a mocker is punished, the fool gains wisdom; \\
\poemll    but when the wise is instructed, he receives knowledge. \\
\poeml \v{12}The righteous God\fnote{\fbackref{21:12} The Heb. lacks \fbib{God}} considers the house of the wicked, \\
\poemll    bringing the wicked to ruin. \\
\poeml \v{13}Whoever refuses to hear the cry of the poor \\
\poemll    will also cry himself, but he won't be answered. \\
\poeml \v{14}Privately given gifts pacify wrath, \\
\poemll    and payments made secretly\fnote{\fbackref{21:14} Lit. \fbib{made under the cloak}} appease\fnote{\fbackref{21:14} The Heb. lacks \fbib{appease}} great anger. \\
\poeml \v{15}Administering justice brings joy to the righteous, \\
\poemll    but terror to those who practice iniquity. \\
\poeml \v{16}Whoever wanders from the path of understanding \\
\poemll    will end up where the dead\fnote{\fbackref{21:16} Lit. \fbib{the departed spirits}} are gathered. \\
\poeml \v{17}Pleasure lovers become poor; \\
\poemll    loving wine and oil doesn't bring riches. \\
\poeml \v{18}The wicked are ransom for the righteous, \\
\poemll    and the unfaithful for the upright. \\
\poeml \v{19}It's better to live in the wilderness \\
\poemll    than to live with a contentious and irritable woman. \\
\poeml \v{20}Precious treasures and oil are found\fnote{\fbackref{21:20} So MT; LXX reads \fbib{A desirable treasure will rest}} where the wise live, \\
\poemll    but a foolish man devours them. \\
\poeml \v{21}Whoever pursues righteousness and gracious love \\
\poemll    finds life, righteousness, and honor. \\
\poeml \v{22}A wise man attacks the city of the mighty, \\
\poemll    bringing down the fortress in which they trust. \\
\poeml \v{23}Whoever watches his mouth and tongue \\
\poemll    keeps himself from trouble. \\
\poeml \v{24}The names ``Proud,'' ``Arrogant,'' and ``Mocker'' \\
\poemll    fit whoever acts with presumptuous conceit. \\
\poeml \v{25}What the lazy person craves will kill him, \\
\poemll    because his hands refuse to work. \\
\poeml \v{26}All day long he continues to crave, \\
\poemll    while the righteous person gives without holding back. \\
\poeml \v{27}What the wicked person sacrifices is detestable--- \\
\poemll    how much more when he offers it with vile motives! \\
\poeml \v{28}A false witness will perish, \\
\poemll    but whoever listens will testify successfully.\fnote{\fbackref{21:28} Lit. \fbib{testify forever}} \\
\poeml \v{29}The wicked man puts up a bold appearance, \\
\poemll    but the upright thinks about what he is doing.\fnote{\fbackref{21:29} Lit. \fbib{about his ways}} \\
\poeml \v{30}No wisdom, insight, or counsel \\
\poemll    can prevail\fnote{\fbackref{21:30} The Heb. lacks \fbib{can prevail}} against the \divine{Lord}. \\
\poeml \v{31}The horse may be prepared for the day of battle, \\
\poemll    but to the \divine{Lord} goes the victory.
\end{poetry}
\labelchapt{22}
\passage{Advice for Everyday Life}

\begin{poetry}
\poeml \chapt{22}
\v{1}A good reputation is more desirable than great wealth, \\
\poeml and favorable acceptance more than silver and gold. \\
\poeml \v{2}The rich and the poor have this in common--- \\
\poemll    the \divine{Lord} created both of them. \\
\poeml \v{3}The prudent person sees trouble ahead and hides, \\
\poemll    but the na\"{i}ve continue on and suffer the consequences. \\
\poeml \v{4}The reward of humility is the fear of the \divine{Lord}, \\
\poemll    along with wealth, honor, and life. \\
\poeml \v{5}Thorns and snares lie in the path of the perverse person, \\
\poemll    but whoever is cautious stays far away from them. \\
\poeml \v{6}Train a child in the way appropriate for him, \\
\poemll    and when he becomes older, he will not turn from it. \\
\poeml \v{7}The wealthy rule over the poor, \\
\poemll    and anyone who borrows is a slave to the lender. \\
\poeml \v{8}Whoever sows wickedness reaps trouble, \\
\poemll    and the anger he uses for a weapon\fnote{\fbackref{22:8} Lit. \fbib{rod}} will be destroyed. \\
\poeml \v{9}Whoever is generous\fnote{\fbackref{22:9} Lit. \fbib{A good eye}} will be blessed, \\
\poemll    for he shares his food with the poor. \\
\poeml \v{10}Throw out the mocker and strife departs, too;\fnote{\fbackref{22:10} The Heb. lacks \fbib{too}} \\
\poemll    furthermore, quarrels\fnote{\fbackref{22:10} Or \fbib{litigation}} and discord will end. \\
\poeml \v{11}Whoever loves purity\fnote{\fbackref{22:11} Lit. \fbib{purity of heart}} and gracious speech \\
\poemll    will gain the king as his friend. \\
\poeml \v{12}The \divine{Lord} watches over anyone with knowledge, \\
\poemll    but he ruins the plans\fnote{\fbackref{22:12} Lit. \fbib{words}} of the unfaithful. \\
\poeml \v{13}The lazy person says, ``There is a lion outside! \\
\poemll    I will be killed in the street!'' \\
\poeml \v{14}The mouth of an immoral woman is a deep pit; \\
\poemll    a man experiencing the \divine{Lord}'s wrath will fall into it. \\
\poeml \v{15}A child's heart has a tendency to do wrong, \\
\poemll    but the rod of discipline removes it far away from him. \\
\poeml \v{16}Whoever oppresses the poor to enrich himself \\
\poemll    and whoever gives gifts to the wealthy \\
\poemlll       will yield only loss.
\passage{Sayings of the Wise}
\poeml \v{17}Pay attention and listen to the words of the wise, \\
\poemll    and apply your heart to my teaching, \\
\poeml \v{18}for it is pleasant when you treasure them within you \\
\poemll    and have them ready on your lips. \\
\poeml \v{19}As a result, your trust will be in the \divine{Lord}, \\
\poemll    that's why I'm teaching you today, even you. \\
\poeml \v{20}Have I not written for you 30 sayings \\
\poemll    containing counsel and knowledge, \\
\poeml \v{21}to teach you true and reliable advice, \\
\poemll    so you can give truthful answers to those who sent you? \\
\poeml \v{22}Don't rob the poor person because he is poor, \\
\poemll    and don't crush the helpless in court,\fnote{\fbackref{22:22} Lit. \fbib{gate}} \\
\poeml \v{23}for the \divine{Lord} will plead their case \\
\poemll    and ruin the lives of those who ruin them. \\
\poeml \v{24}Don't make friends with a hot-tempered man, \\
\poemll    and do not associate with someone who is easily angered, \\
\poeml \v{25}or you may learn his ways \\
\poemll    and find yourself caught in a trap. \\
\poeml \v{26}Don't be one of those who make promises \\
\poemll    to guarantee loans for debts. \\
\poeml \v{27}If you don't have the ability to pay, \\
\poemll    why should your very bed be taken from under you? \\
\poeml \v{28}Don't remove an ancient boundary stone \\
\poemll    that was set up by your ancestors. \\
\poeml \v{29}Do you see a man skilled in his work? \\
\poemll    He will work for kings, not unimportant people.
\end{poetry}
\labelchapt{23}
\passage{Things to Avoid in Life}

\begin{poetry}
\poeml \chapt{23}
\v{1}Whenever you sit down to dine with a ruler, \\
\poeml carefully think about what is before you. \\
\poeml \v{2}Put a knife to your own throat, \\
\poemll    if you have a big appetite.\fnote{\fbackref{23:2} Lit. \fbib{a master of an appetite}} \\
\poeml \v{3}Don't crave his delicacies, \\
\poemll    because the meal is deceptive. \\
\poeml \v{4}Don't exhaust yourself acquiring wealth; \\
\poemll    be smart enough to stop. \\
\poeml \v{5}When you fix your gaze on it, it's gone, \\
\poemll    for it sprouts wings for itself \\
\poemlll       and flies to the sky like an eagle. \\
\poeml \v{6}Don't consume food provided by a miserly\fnote{\fbackref{23:6} Lit. \fbib{by the evil eyed}} person, \\
\poemll    and don't desire his delicacies, \\
\poeml \v{7}for as he thinks within himself, so he is. \\
\poemll    ``Eat and drink!'' he'll say to you, \\
\poemlll       but his heart won't be with you. \\
\poeml \v{8}You'll vomit up what little you've eaten, \\
\poemll    and your compliments will have been wasted. \\
\poeml \v{9}Don't speak when a fool is listening, \\
\poemll    because he'll despise your wise words. \\
\poeml \v{10}Don't move ancient boundaries \\
\poemll    or invade fields belonging to orphans; \\
\poeml \v{11}for strong is their Redeemer \\
\poemll    who will take up their case against you. \\
\poeml \v{12}Learn diligently, \\
\poemll    and listen to words of knowledge. \\
\poeml \v{13}Don't withhold discipline from a child; \\
\poemll    if you punish him with a rod, \\
\poemlll       he won't die. \\
\poeml \v{14}Punish him with a rod, \\
\poemll    and you will rescue his soul from Sheol.\fnote{\fbackref{23:14} I.e. the realm of the dead.}
\passage{On Listening to Your Parents}
\poeml \v{15}My son, if your heart is wise, \\
\poemll    my own heart will greatly rejoice. \\
\poeml \v{16}My innermost being will be glad \\
\poemll    when your lips speak what is right. \\
\poeml \v{17}Never let yourself envy sinners; \\
\poemll    instead, remain\fnote{\fbackref{23:17} The Heb. lacks \fbib{remain}} in fear of the \divine{Lord} every day, \\
\poeml \v{18}for there is surely a future life, \\
\poemll    and what you hope for will not be cut off. \\
\poeml \v{19}Listen, my son, and be wise, \\
\poemll    commit yourself to live God's\fnote{\fbackref{23:19} Lit. \fbib{live in the}} way. \\
\poeml \v{20}Don't associate with heavy drinkers \\
\poemll    or dine with gluttons, \\
\poeml \v{21}because drunks and gluttons tend to become poor, \\
\poemll    and drowsiness will clothe them in rags. \\
\poeml \v{22}Listen to the one who fathered you, \\
\poemll    and don't despise your mother in her old age. \\
\poeml \v{23}Purchase truth, but don't sell it; \\
\poemll    store up\fnote{\fbackref{23:23} The Heb. lacks \fbib{store up}} wisdom, instruction, and understanding. \\
\poeml \v{24}The father of a righteous person will greatly rejoice; \\
\poemll    whoever fathers a wise son will be glad because of him. \\
\poeml \v{25}Let your father and mother rejoice; \\
\poemll    make the one who gave birth to you happy. \\
\poeml \v{26}Give me your heart, my son, \\
\poemll    and keep your eyes fixed on my ways, \\
\poeml \v{27}because a prostitute is a deep pit, \\
\poemll    and the adulterous\fnote{\fbackref{23:27} Lit. \fbib{foreign}} woman a narrow well. \\
\poeml \v{28}Surely she lies in wait like a bandit, \\
\poemll    increasing those who are faithless among mankind.
\passage{On Sobriety}
\poeml \v{29}Who has woe? Who has grief? \\
\poemll    Who has contention? Who has complaints? \\
\poeml Who has wounds without cause? \\
\poemll    Who has bloodshot eyes? \\
\poeml \v{30}Those who linger over their wine, \\
\poemll    who consume mixed drinks. \\
\poeml \v{31}Don't stare into red wine, \\
\poemll    when it sparkles in the cup \\
\poemlll       and goes down smoothly. \\
\poeml \v{32}Eventually it will bite like a snake \\
\poemll    and sting like a serpent. \\
\poeml \v{33}Your eyes will see strange things, \\
\poemll    and with slurred words you'll speak what you really believe. \\
\poeml \v{34}You will be like someone who lies down in the sea, \\
\poemll    or like someone who sleeps on top of a mast. \\
\poeml \v{35}``They struck me,'' you will say,\fnote{\fbackref{23:35} The Heb. lacks \fbib{you will say}} \\
\poemll    ``but I never felt it. \\
\poeml They beat me, \\
\poemll    but I never knew it \\
\poeml When will I wake up? \\
\poemll    I want another drink.''
\end{poetry}
\labelchapt{24}
\passage{Benefits of Wisdom}

\begin{poetry}
\poeml \chapt{24}
\v{1}Don't be envious of wicked men \\
\poeml or wish you were with them, \\
\poeml \v{2}because they\fnote{\fbackref{24:2} Lit. \fbib{because their hearts}} plan violence, \\
\poemll    and they are always talking\fnote{\fbackref{24:2} Lit. \fbib{and their lips talk}} about trouble. \\
\poeml \v{3}By wisdom a house is built; \\
\poemll    it is made secure through understanding. \\
\poeml \v{4}By knowledge its rooms are furnished \\
\poemll    with all sorts of expensive and beautiful goods. \\
\poeml \v{5}A wise man is strong,\fnote{\fbackref{24:5} So MT; LXX reads \fbib{Being wise is better than being strong}} \\
\poemll    and a knowledgeable man grows in strength. \\
\poeml \v{6}For through wise counsel you will wage your war, \\
\poemll    and victory lies in an abundance of advisors. \\
\poeml \v{7}Wisdom lies beyond reach of the fool; \\
\poemll    he has nothing to say in court.\fnote{\fbackref{24:7} Lit. \fbib{in the gate}} \\
\poeml \v{8}The person who plans on doing evil \\
\poemll    will be called a schemer. \\
\poeml \v{9}To devise folly is sin, \\
\poemll    and people detest a scoffer. \\
\poeml \v{10}If you grow weary when times are troubled, \\
\poemll    your strength is limited.\fnote{\fbackref{24:10} Or \fbib{undersized}} \\
\poeml \v{11}Rescue those who are being led away to death, \\
\poemll    and save those who stumble toward slaughter. \\
\poeml \v{12}If you say, ``Look here, we didn't know about this,'' \\
\poemll    doesn't God,\fnote{\fbackref{24:12} Lit. \fbib{he}} who examines motives,\fnote{\fbackref{24:12} Lit. \fbib{examines the heart}} discern it? \\
\poeml Doesn't the one who guards your soul \\
\poemll    know about it? \\
\poeml Won't he repay each person \\
\poemll    according to what he has done? \\
\poeml \v{13}My son, eat honey, because it's good for you;\fnote{\fbackref{24:13} The Heb. lacks \fbib{for you}} \\
\poemll    indeed, drippings from the honeycomb are sweet to your taste; \\
\poeml \v{14}Keep in mind that wisdom is like that for your soul; \\
\poemll    if you find it, there will be a future for you, \\
\poemlll       and what you hope for won't be cut short. \\
\poeml \v{15}Don't lie in wait like an outlaw \\
\poemll    to attack where the righteous live; \\
\poeml \v{16}for though a righteous man falls seven times, \\
\poemll    he will rise again, \\
\poemlll       but the wicked stumble into calamity. \\
\poeml \v{17}Don't rejoice when your enemy falls; \\
\poemll    don't let yourself be glad when he stumbles. \\
\poeml \v{18}Otherwise the \divine{Lord} will observe and disapprove, \\
\poemll    and he will turn his anger away from him. \\
\poeml \v{19}Don't be anxious about those who practice evil, \\
\poemll    and don't be envious of the wicked. \\
\poeml \v{20}For the wicked man has no future; \\
\poemll    the lamp of the wicked will be extinguished. \\
\poeml \v{21}My son, fear both the \divine{Lord} and the king, \\
\poemll    and don't keep company with rebels. \\
\poeml \v{22}They will be destroyed suddenly, \\
\poemll    and who knows what kind of punishment will come from these two?
\passage{Sayings of the Wise}
\poeml \v{23}Here are some more proverbs from wise people: \\
\poeml It isn't good to show partiality in judgment. \\
\poeml \v{24}Whoever says to the wicked, ``You're in the right,'' \\
\poemlll       will be cursed by people and hated by nations. \\
\poeml \v{25}But as for people who rebuke the wicked;\fnote{\fbackref{24:25} The Heb. lacks \fbib{the wicked}} \\
\poemll    a good blessing will fall upon them. \\
\poeml \v{26}A kiss on the lips--- \\
\poemll    that's what someone who gives an honest answer deserves.\fnote{\fbackref{24:26} The Heb. lacks \fbib{deserves}} \\
\poeml \v{27}First do your outside work, \\
\poemll    preparing your land for yourself. \\
\poemlll       After that, build your house. \\
\poeml \v{28}Don't testify against your neighbor without a cause, \\
\poemll    and don't lie when you speak.\fnote{\fbackref{24:28} Lit. \fbib{don't deceive with your lips}} \\
\poeml \v{29}Don't say, ``I'll do to him like he did to me, \\
\poemll    I'll be sure to pay him back for what he did.'' \\
\poeml \v{30}I went by the field belonging to a lazy man, \\
\poemll    by a vineyard belonging to a senseless person. \\
\poeml \v{31}There it was, overgrown with thistles, \\
\poemll    the ground covered with thorns, \\
\poemlll       its stone wall collapsed. \\
\poeml \v{32}As I observed, I thought about it; \\
\poemll    I watched, and learned a lesson: \\
\poeml \v{33}``A little sleep! A little slumber! \\
\poemll    A little folding of my hands to rest!'' \\
\poeml \v{34}Then your poverty will come upon you like a robber, \\
\poemll    your need like an armed bandit.
\end{poetry}
\labelchapt{25}
\passage{More Proverbs from Solomon}

\chapt{25}
\v{1}Here are some more proverbs by Solomon, which the men of Hezekiah, king of Judah, transcribed.

\begin{poetry}
\poeml \v{2}It is the glory of God to conceal a matter, \\
\poemll    and the glory of kings to investigate a matter. \\
\poeml \v{3}Just as the heavens are high \\
\poemll    and earth is deep, \\
\poemlll       so the heart of a king is unfathomable. \\
\poeml \v{4}Purge the dross from the silver, \\
\poemll    and material for\fnote{\fbackref{25:4} The Heb. lacks \fbib{material for}} a vessel comes forth for the silversmith. \\
\poeml \v{5}Purge the wicked from the king's presence, \\
\poemll    and his throne will be established in righteousness. \\
\poeml \v{6}Don't magnify yourself in the presence of a king, \\
\poemll    and don't pretend to be in the company of famous men, \\
\poeml \v{7}for it is better that it be told you, ``Come up here,'' \\
\poemll    than for you to be placed lower \\
\poemlll       in the presence of an official. \\
\poeml What you've seen with your own eyes, \\
\poeml \v{8}don't be in a hurry to argue in court. \\
\poeml Otherwise, what will you do later on \\
\poemll    when your neighbor humiliates you? \\
\poeml \v{9}Instead, take up the matter with your neighbor, \\
\poemll    and don't betray another person's confidence. \\
\poeml \v{10}Otherwise, anyone who hears will make you ashamed, \\
\poemll    and your bad reputation will never leave you. \\
\poeml \v{11}Like golden apples set in silver \\
\poemll    is a word spoken at the right time. \\
\poeml \v{12}Like a gold earring and a necklace of pure gold \\
\poemll    is a wise reprover to a listening ear. \\
\poeml \v{13}Like cold snow during harvest time \\
\poemll    is a faithful messenger to those who send him; \\
\poemlll       he refreshes his masters. \\
\poeml \v{14}Like clouds and winds without rain \\
\poemll    is the man who brags \\
\poemlll       about gifts he never gave. \\
\poeml \v{15}Through patience a ruler may be persuaded; \\
\poemll    a gentle word\fnote{\fbackref{25:15} Lit. \fbib{tongue}} can break a bone. \\
\poeml \v{16}If you find some honey, \\
\poemll    eat only what you need. \\
\poeml Take too much, \\
\poemll    and you'll vomit. \\
\poeml \v{17}Seldom set foot in your neighbor's home; \\
\poemll    otherwise, he'll grow weary and hate you. \\
\poeml \v{18}A club, a sword, and a sharp arrow--- \\
\poemll    that's what a man is who lies about his neighbor. \\
\poeml \v{19}A bad tooth and an unsteady foot--- \\
\poemll    that's what confidence in an unreliable man is like \\
\poemlll       in a time of trouble. \\
\poeml \v{20}Taking your coat off when it's cold \\
\poemll    or pouring vinegar on soda--- \\
\poemlll       that's what singing songs does to a heavy heart. \\
\poeml \v{21}If your enemy hungers, give him food to eat; \\
\poemll    and if he thirsts, give him water to drink. \\
\poeml \v{22}For you'll be piling burning coals of shame\fnote{\fbackref{25:22} The Heb. lacks \fbib{of shame}} on his head \\
\poemll    and the \divine{Lord} will reward you. \\
\poeml \v{23}The north wind brings rain, \\
\poemll    and a backbiting tongue an angry look. \\
\poeml \v{24}It's better to live in a corner on the roof \\
\poemll    than in a house with a contentious woman. \\
\poeml \v{25}Cold water to someone who is thirsty\fnote{\fbackref{25:25} Or \fbib{tired}}--- \\
\poemll    that's what good news from a distant land is. \\
\poeml \v{26}A muddied spring or a polluted well--- \\
\poemll    that's what a righteous person is \\
\poemlll       who compromises with the wicked. \\
\poeml \v{27}To eat too much honey isn't good; \\
\poemll    and neither is it honorable to seek one's own glory. \\
\poeml \v{28}Like a city with breached walls \\
\poemll    is a man without self-control.
\end{poetry}
\labelchapt{26}
\passage{On Fools}

\begin{poetry}
\poeml \chapt{26}
\v{1}Like snowfall in summer or rain at harvest time, \\
\poeml so honor is inappropriate for a fool. \\
\poeml \v{2}Like a fluttering sparrow \\
\poemll    or a swallow in flight, \\
\poemlll       a curse without cause will not alight. \\
\poeml \v{3}A whip is for the horses, \\
\poemll    a bridle is for the donkey, \\
\poemlll       a rod is for the back of fools. \\
\poeml \v{4}Don't answer a fool according to his foolishness, \\
\poemll    or you will be just like him. \\
\poeml \v{5}Answer a fool according to his foolishness, \\
\poemll    or he will think himself to be wise. \\
\poeml \v{6}Whoever sends a message by the hand of a fool \\
\poemll    cuts off his own\fnote{\fbackref{26:6} The Heb. lacks \fbib{his own}} feet and drinks violence. \\
\poeml \v{7}Useless legs to the lame--- \\
\poemll    that's what a proverb quoted by a fool is. \\
\poeml \v{8}Tying a stone to a sling--- \\
\poemll    that's what giving honor to a fool is. \\
\poeml \v{9}A thorn in the hand of a drunkard--- \\
\poemll    that's what a proverb quoted by a fool is. \\
\poeml \v{10}An archer who shoots at anyone--- \\
\poemll    is like someone who hires a fool or anyone who passes by. \\
\poeml \v{11}A dog that returns to its vomit \\
\poemll    is like a fool who reverts to his folly. \\
\poeml \v{12}Do you see a man who is wise in his own opinion? \\
\poemll    There's more hope for a fool than for him.
\passage{On Laziness}
\poeml \v{13}The lazy person claims, ``There is a lion in the road! \\
\poemll    There's a lion in the streets!'' \\
\poeml \v{14}The door turns on its hinges--- \\
\poemll    as does the lazy person on his bed. \\
\poeml \v{15}The lazy person buries his hand in the dish, \\
\poemll    but he's too tired to bring it to his mouth again. \\
\poeml \v{16}The lazy person is wiser in his own opinion \\
\poemll    than seven men who can give an appropriate response. \\
\poeml \v{17}Picking up a dog by the ears--- \\
\poemll    that's what someone is like who\fnote{\fbackref{26:17} Lit. \fbib{who, as he is passing by,}} meddles in another's fight. \\
\poeml \v{18}Like the maniac who shoots \\
\poemll    fiery darts and deadly arrows--- \\
\poeml \v{19}that's what someone is like who lies to his neighbor \\
\poemll    and then says, ``I was joking, wasn't I?''
\passage{On Gossip and Backbiting}
\poeml \v{20}Without wood, the fire goes out. \\
\poemll    Without a gossip, contention stops. \\
\poeml \v{21}Charcoal is to hot coals \\
\poemll    as wood is to fire; \\
\poemlll       so also a quarrelsome man fuels strife. \\
\poeml \v{22}The words of a gossip are like delicate morsels; \\
\poemll    they sink down deep within. \\
\poeml \v{23}A clay vessel plated with a thin veneer of silver--- \\
\poemll    that's what smooth\fnote{\fbackref{26:23} So LXX; MT reads \fbib{burning}} lips with a wicked heart are. \\
\poeml \v{24}Someone who hates hides behind his words, \\
\poemll    harboring deceit within himself. \\
\poeml \v{25}Though he speaks graciously, don't believe him, \\
\poemll    for there are seven detestable things in his heart. \\
\poeml \v{26}Though malice disguises itself with deception, \\
\poemll    its evil will be exposed publicly. \\
\poeml \v{27}Whoever digs a pit will fall into it, \\
\poemll    and the stone will come back \\
\poemlll       on whoever starts it rolling. \\
\poeml \v{28}A lying tongue hates its victims, \\
\poemll    and a flattering mouth causes ruin.
\end{poetry}
\labelchapt{27}
\passage{General Counsel}

\begin{poetry}
\poeml \chapt{27}
\v{1}Never brag about the day to come, \\
\poeml because you don't know what it\fnote{\fbackref{27:1} Lit. \fbib{what a day}} might bring. \\
\poeml \v{2}Let someone else praise you, \\
\poemll    not your own mouth; \\
\poemlll       a stranger, and never your own lips. \\
\poeml \v{3}Rocks are heavy, \\
\poemll    and sand is weighty, \\
\poemlll       but a fool's provocation outweighs them both. \\
\poeml \v{4}Wrath can be fierce and anger overwhelms \\
\poemll    but who can stand up to jealousy? \\
\poeml \v{5}An open rebuke is better \\
\poemll    than unspoken love. \\
\poeml \v{6}Wounds from someone who loves are trustworthy, \\
\poemll    but kisses from an enemy speak volumes.\fnote{\fbackref{27:6} Lit. \fbib{enemy are profuse}} \\
\poeml \v{7}The person\fnote{\fbackref{27:7} Lit. \fbib{soul}} who is full spurns honey, \\
\poemll    but to a hungry person even the bitter seems sweet. \\
\poeml \v{8}Like a bird that strays from its nest \\
\poemll    is a man who wanders away from his home.\fnote{\fbackref{27:8} Lit. \fbib{place}} \\
\poeml \v{9}Ointments and perfume encourage the heart; \\
\poemll    in a similar way, a friend's advice is sweet to the soul.\fnote{\fbackref{27:9} So MT; LXX reads \fbib{heart; but through misfortune the soul is torn apart}} \\
\poeml \v{10}Never abandon your friend nor your father's friend, \\
\poemll    and don't go to your brother's house in times of trouble. \\
\poeml A neighbor who is near is better \\
\poemll    than a brother who lives far away. \\
\poeml \v{11}Be wise, my son, and make me happy, \\
\poemll    so I can reply to anyone who insults me. \\
\poeml \v{12}Those who are prudent see danger and take refuge, \\
\poemll    but the na\"{i}ve continue on and suffer the consequences. \\
\poeml \v{13}Take the coat of anyone who puts up security for a stranger; \\
\poemll    hold it in pledge if he cosigns for an immoral woman. \\
\poeml \v{14}A friend's loud blessing early in the morning \\
\poemll    will be thought of as a curse. \\
\poeml \v{15}A continual dripping on a rainy day \\
\poemll    and a contentious wife are alike. \\
\poeml \v{16}Trying to keep her in check is like stopping a wind storm \\
\poemll    or grabbing oil with your right hand. \\
\poeml \v{17}Iron sharpens iron; \\
\poemll    so a man sharpens a friend's character.\fnote{\fbackref{27:17} Lit. \fbib{countenance}} \\
\poeml \v{18}Whoever nurtures the fig tree will eat its fruit, \\
\poemll    and whoever obeys\fnote{\fbackref{27:18} Lit. \fbib{guards}} his master will be honored. \\
\poeml \v{19}Just as water reflects the face, \\
\poemll    so the heart reflects the person. \\
\poeml \v{20}Sheol\fnote{\fbackref{27:20} I.e. the realm of the dead} and Abaddon\fnote{\fbackref{27:20} I.e. the realm of destruction in the afterlife} are never satiated, \\
\poemll    and neither are human eyes. \\
\poeml \v{21}As the crucible tests\fnote{\fbackref{27:21} The Heb. lacks \fbib{tests}} silver, \\
\poemll    and the furnace assays\fnote{\fbackref{27:21} The Heb. lacks \fbib{assays}} gold; \\
\poemlll       so praise received tests\fnote{\fbackref{27:21} The Heb. lacks \fbib{tests}} a man. \\
\poeml \v{22}Though you crush a fool in a mortar and pestle \\
\poemll    as someone might crush grain, \\
\poemlll       his stupidity still won't leave him. \\
\poeml \v{23}Keep well informed of the condition of your flocks \\
\poemll    and pay attention to your herds, \\
\poeml \v{24}because riches don't endure forever, \\
\poemll    and crowns don't last from one generation to the next. \\
\poeml \v{25}When the grass disappears, \\
\poemll    and new growth appears, \\
\poemlll       the mountain spices will be harvested, \\
\poeml \v{26}the lambs will supply your clothing, \\
\poemll    and your goats the price of a field. \\
\poeml \v{27}You will have enough goat's milk to drink \\
\poemll    and to supply your household needs, \\
\poemlll       as well as sustenance for your servant girls.
\end{poetry}
\labelchapt{28}
\passage{Contrasting Good and Evil}

\begin{poetry}
\poeml \chapt{28}
\v{1}The wicked flee, though no one pursues, \\
\poeml but the righteous are bold like a lion. \\
\poeml \v{2}When a land transgresses, \\
\poemll    it gains a succession of leaders, \\
\poeml but with an understanding and knowledgeable man, \\
\poemll    its stability endures. \\
\poeml \v{3}A poor man who oppresses the weak \\
\poemll    is like a rainstorm that destroys all\fnote{\fbackref{28:3} Lit. \fbib{that leaves no}} the crops. \\
\poeml \v{4}Those who forsake the Law praise the wicked, \\
\poemll    but whoever keeps it\fnote{\fbackref{28:4} Lit. \fbib{keeps the Law}} fights them. \\
\poeml \v{5}Evil men don't understand justice, \\
\poemll    but whoever seeks the \divine{Lord} understands it all. \\
\poeml \v{6}It's better to be poor and live a blameless life \\
\poemll    than to be rich but crooked in one's lifestyle. \\
\poeml \v{7}Whoever keeps the Law is a discerning son, \\
\poemll    but whoever keeps company with gluttons \\
\poemlll       brings shame to his father. \\
\poeml \v{8}Whoever gains wealth by charging exorbitant\fnote{\fbackref{28:8} Lit. \fbib{charging interest upon}} interest \\
\poemll    collects it for someone who is kind to the poor. \\
\poeml \v{9}If someone quits\fnote{\fbackref{28:9} Lit. \fbib{turns away from}} listening to the Law \\
\poemll    even his prayer is detestable. \\
\poeml \v{10}Whoever misleads the upright along an evil way \\
\poemll    will himself fall into his own pit, \\
\poemlll       but the blameless will inherit what is good. \\
\poeml \v{11}The rich man may be wise in his own opinion; \\
\poemll    but a discerning, poor man sees through him. \\
\poeml \v{12}When the righteous are victorious, there is great glory, \\
\poemll    but when the wicked arise, men hide themselves. \\
\poeml \v{13}Whoever hides his transgressions will not succeed, \\
\poemll    but whoever confesses and forsakes them will find mercy. \\
\poeml \v{14}Blessed is the man who always fears the \divine{Lord},\fnote{\fbackref{28:14} The Heb. lacks \fbib{the \divine{Lord}}} \\
\poemll    but whoever hardens his heart will fall into disaster. \\
\poeml \v{15}A roaring lion and a charging bear--- \\
\poemll    that's what a wicked tyrant is over poor people. \\
\poeml \v{16}A Commander-in-Chief\fnote{\fbackref{28:16} Lit. \fbib{Nagid}; i.e. a senior officer entrusted with dual roles of operational oversight and administrative authority} who is a cruel oppressor lacks understanding, \\
\poemll    but whoever hates unjust gain will live longer.\fnote{\fbackref{28:16} Lit. \fbib{will lengthen his days}} \\
\poeml \v{17}A guilty man tormented by bloodshed \\
\poemll    will be a lifelong fugitive; \\
\poemlll       let no one support him. \\
\poeml \v{18}Whoever lives blamelessly will be delivered, \\
\poemll    but whoever is perverted will fall without warning. \\
\poeml \v{19}Whoever works his farmland will have abundant food, \\
\poemll    but whoever chases fantasies will become very poor. \\
\poeml \v{20}The faithful man will prosper with blessings, \\
\poemll    but whoever is in a hurry to get rich \\
\poemlll       will not escape punishment. \\
\poeml \v{21}To show partiality isn't good, \\
\poemll    yet for a piece of bread the valiant will transgress. \\
\poeml \v{22}The miser\fnote{\fbackref{28:22} Lit. \fbib{The man with an evil eye}} is in a hurry to get wealthy, \\
\poemll    but he isn't aware that poverty will catch up with him. \\
\poeml \v{23}Whoever rebukes a man will later on find more favor \\
\poemll    than someone who flatters with his words.\fnote{\fbackref{28:23} Lit. \fbib{tongue}} \\
\poeml \v{24}Whoever steals from his father or mother \\
\poemll    but claims, ``It's no sin,'' \\
\poemlll       is a companion to someone who demolishes. \\
\poeml \v{25}An arrogant\fnote{\fbackref{28:25} Or \fbib{greedy}} man stirs up dissension, \\
\poemll    but anyone who trusts in the \divine{Lord} prospers. \\
\poeml \v{26}Whoever trusts in himself is foolish, \\
\poemll    but whoever lives wisely will be kept safe. \\
\poeml \v{27}Whoever gives to the poor will never lack, \\
\poemll    but whoever shuts his eyes to their poverty\fnote{\fbackref{28:27} The Heb. lacks \fbib{to their poverty}} will be cursed. \\
\poeml \v{28}When the wicked rise to power, people hide themselves, \\
\poemll    but when the wicked\fnote{\fbackref{28:28} Lit. \fbib{when they}} perish, the righteous increase.
\end{poetry}
\labelchapt{29}
\passage{Advice on Life and Justice}

\begin{poetry}
\poeml \chapt{29}
\v{1}After many rebukes, the stiff-necked man \\
\poeml will be broken incurably, without any warning. \\
\poeml \v{2}As the righteous grow powerful,\fnote{\fbackref{29:2} The Heb. lacks \fbib{powerful}} people rejoice; \\
\poemll    but when the wicked rule, people groan. \\
\poeml \v{3}The man who loves wisdom brings joy to his father, \\
\poemll    but anyone who consorts with immoral women squanders his wealth. \\
\poeml \v{4}A king brings stability to a land through justice, \\
\poemll    but a man who takes bribes brings it to ruin. \\
\poeml \v{5}A strong man who flatters his neighbor \\
\poemll    is setting a trap where he walks.\fnote{\fbackref{29:5} Lit. \fbib{trap for his footsteps}}
\passage{The Wicked and Righteous Contrasted}
\poeml \v{6}An evil man is trapped by transgression, \\
\poemll    but the righteous person sings and rejoices. \\
\poeml \v{7}The righteous person is concerned about the poor; \\
\poemll    but the wicked don't understand what they need to know.\fnote{\fbackref{29:7} Lit. \fbib{understand knowledge}} \\
\poeml \v{8}Scornful men enflame a city, \\
\poemll    but the wise defuse anger. \\
\poeml \v{9}When a wise man has a dispute with a fool, \\
\poemll    the fool either rages or laughs without relief. \\
\poeml \v{10}Bloodthirsty men hate the innocent person, \\
\poemll    but the upright show concern for his life. \\
\poeml \v{11}The fool vents all his feelings,\fnote{\fbackref{29:11} Lit. \fbib{spirit}} \\
\poemll    but the wise person keeps them to himself.\fnote{\fbackref{29:11} The Heb. lacks \fbib{to himself}} \\
\poeml \v{12}When a ruler is listening to lies, \\
\poemll    all of his officials tend to become wicked. \\
\poeml \v{13}The poor man and the oppressor have this in common:\fnote{\fbackref{29:13} Lit. \fbib{oppressor meet together}} \\
\poemll    the \divine{Lord} gave them both eyes with which to see.\fnote{\fbackref{29:13} Lit. \fbib{\divine{Lord} lights the eyes of both}} \\
\poeml \v{14}When a king faithfully administers justice to the poor, \\
\poemll    his throne will be established forever. \\
\poeml \v{15}The rod and rebuke bestow wisdom, \\
\poemll    but an undisciplined child\fnote{\fbackref{29:15} Lit. \fbib{but a child left alone}} brings shame to his mother. \\
\poeml \v{16}As the wicked grow powerful,\fnote{\fbackref{29:16} The Heb. lacks \fbib{powerful}} transgression increases, \\
\poemll    but the righteous will observe their downfall. \\
\poeml \v{17}Correct your son and he will comfort you; \\
\poemll    he will also delight your soul. \\
\poeml \v{18}Without prophetic vision, people abandon restraint, \\
\poemll    but those who obey the Law are happy.
\passage{Dangerous Behaviors}
\poeml \v{19}By mere words a servant will not be corrected; \\
\poemll    even though he understands, \\
\poemlll       there will be no response. \\
\poeml \v{20}Do you see a man who speaks hastily? \\
\poemll    There is more hope for a fool than for him. \\
\poeml \v{21}If you pamper a servant from his childhood, \\
\poemll    later on he'll become ungrateful. \\
\poeml \v{22}An angry man stirs up arguments, \\
\poemll    and a hot-tempered man causes many transgressions. \\
\poeml \v{23}A person's pride will bring about his downfall, \\
\poemll    but the humble in spirit will gain honor. \\
\poeml \v{24}A thief's accomplice hates himself; \\
\poemll    though testifying under oath,\fnote{\fbackref{29:24} Lit. \fbib{though he hears the oath}} he reveals nothing. \\
\poeml \v{25}Fearing any human being is a trap, \\
\poemll    but confiding in the \divine{Lord} keeps anyone safe. \\
\poeml \v{26}Many seek a ruler's favor,\fnote{\fbackref{29:26} Lit. \fbib{face}} \\
\poemll    but justice for a man comes from the \divine{Lord}. \\
\poeml \v{27}The unjust man is detestable to the righteous, \\
\poemll    and whoever lives blamelessly is detestable to the wicked.
\end{poetry}
\labelchapt{30}
\passage{The Oracle}

\chapt{30}
\v{1}A discourse by the faithful collector.\fnote{\fbackref{30:1} Or \fbib{by Jakeh's son Agur}}

\begin{poetry}
\poeml This is what this valiant man declared to the God with me, \\
\poemll    to the God with me, who then prevailed:\fnote{\fbackref{30:1} Or \fbib{declared to Ithiel, to Ithiel, and Ucal}} \\
\poeml \v{2}Surely I am beyond the senselessness of any man; \\
\poemll    I do not perceive things\fnote{\fbackref{30:2} The Heb. lacks \fbib{things}} the way human beings do. \\
\poeml \v{3}I never acquired wisdom, \\
\poemll    but I know what the Holy One knows. \\
\poeml \v{4}Who has ascended to heaven, \\
\poemll    and then descended? \\
\poemlll       Who has collected the wind in his hands? \\
\poeml Who has wrapped up waters in a garment? \\
\poemll    Who has established all the farthest points of the earth? \\
\poeml What is his name, \\
\poemll    and what is his son's name? \\
\poemlll       Surely you know! \\
\poeml \v{5}Everything God says is pure; \\
\poemll    he is a shield for those who take refuge in him. \\
\poeml \v{6}Don't add to his words, \\
\poemll    or he will rebuke you, \\
\poemlll       and you will be shown to be a liar.
\passage{On Contentment in Life}
\poeml \v{7}God,\fnote{\fbackref{30:7} The Heb. lacks \fbib{God}} I have asked you for two things--- \\
\poemll    don't refuse me before I die--- \\
\poeml \v{8}Keep deception and lies far away from me, \\
\poemll    and give me neither poverty nor wealth. \\
\poeml Feed me with food that I need for today,\fnote{\fbackref{30:8} Or \fbib{that is appropriate for me}} \\
\poeml \v{9}so that I don't become overfed and deny you by saying, \\
\poemlll       ``Who is the \divine{Lord}?'' \\
\poeml or so that I don't become poor and steal, \\
\poemll    and then misuse the name of my God.
\passage{On Different Kinds of People}
\poeml \v{10}Don't lie about a servant to his master, \\
\poemll    or that servant\fnote{\fbackref{30:10} Lit. \fbib{or he}} will curse you and you will pay for it. \\
\poeml \v{11}Some people\fnote{\fbackref{30:11} Lit. \fbib{A} \fbib{generation}} curse their fathers \\
\poemll    and won't bless their mothers. \\
\poeml \v{12}Some people\fnote{\fbackref{30:12} Lit. \fbib{A generation}} view themselves as pure, \\
\poemll    but haven't been cleansed from their own filth. \\
\poeml \v{13}Some people\fnote{\fbackref{30:13} Lit. \fbib{A generation}}---what an arrogant look they have!--- \\
\poemll    raise their eyebrows haughtily. \\
\poeml \v{14}Some people\fnote{\fbackref{30:14} Lit. \fbib{A generation}} have swords for teeth \\
\poemll    and knives for fangs \\
\poeml to devour the afflicted from the earth \\
\poemll    and the needy from among mankind. \\
\poeml \v{15}The leech has two daughters: \\
\poemll    ``Give'' and ``Give''. \\
\poeml Three things will never be satisfied; \\
\poemll    four will never say ``Enough''--- \\
\poeml \v{16}The afterlife,\fnote{\fbackref{30:16} Lit. \fbib{Sheol}; i.e. the realm of the dead} the barren womb, \\
\poemll    earth that still demands water, \\
\poemlll       and fire---they never say, ``Enough''. \\
\poeml \v{17}The eye that mocks a father \\
\poemll    and looks with a disobedient attitude at\fnote{\fbackref{30:17} Lit. \fbib{and despises obedience to}} a mother--- \\
\poeml the valley ravens will pluck it out; \\
\poemll    and vultures will eat it.
\passage{What Causes Wonder}
\poeml \v{18}Three things cause wonder for me; \\
\poemll    four are beyond my understanding: \\
\poeml \v{19}The way an eagle flies in the sky, \\
\poemll    the way of a serpent on a rock, \\
\poeml the way of a ship on the high seas, \\
\poemll    and the way of a man with a young woman. \\
\poeml \v{20}This is what an immoral woman is like: \\
\poemll    she eats, wipes her mouth, then says \\
\poemlll       ``I've done nothing wrong.'' \\
\poeml \v{21}Under three things the earth trembles, \\
\poemll    under four it cannot remain steady: \\
\poeml \v{22}Under a slave when he becomes a king, \\
\poemll    a fool when he is overfed, \\
\poeml \v{23}an unloved woman when she finds a husband, \\
\poemll    and a servant girl who inherits from her mistress. \\
\poeml \v{24}Four things on earth are small, \\
\poemll    but they are very, very wise: \\
\poeml \v{25}Ants aren't a strong species,\fnote{\fbackref{30:25} Lit. \fbib{people}} \\
\poemll    yet they prepare their food in the summer. \\
\poeml \v{26}The rock badgers aren't a strong species\fnote{\fbackref{30:26} Lit. \fbib{people}} either, \\
\poemll    yet they build their dens in the rocks. \\
\poeml \v{27}Locusts have no king, \\
\poemll    but they all swarm in ranks. \\
\poeml \v{28}Spiders can be caught by the hand, \\
\poemll    yet they're found in kings' palaces. \\
\poeml \v{29}Three things are stately in procession, \\
\poemll    four which are stately in their gait: \\
\poeml \v{30}The lion, mighty among the beasts, \\
\poemll    retreats before nothing. \\
\poeml \v{31}The strutting rooster, as well as the goat, \\
\poemll    and a king with his army. \\
\poeml \v{32}If you've foolishly exalted yourself \\
\poemll    or if you've plotted evil, \\
\poemlll       put your hand over your mouth. \\
\poeml \v{33}Just as whipping milk produces butter, \\
\poemll    and twisting the nose causes bleeding, \\
\poemlll       so also stirring up anger produces contention.
\end{poetry}
\labelchapt{31}
\passage{Counsel from King Lemuel's Mother}

\begin{poetry}
\poeml \chapt{31}
\v{1}The words of King Lemuel--- \\
\poeml a pronouncement with which his mother encouraged him. \\
\poeml \v{2}No,\fnote{\fbackref{31:2} Or \fbib{What}} my son! \\
\poemll    No,\fnote{\fbackref{31:2} Or \fbib{What}} my son whom I conceived!\fnote{\fbackref{31:2} Lit. \fbib{son of my womb?}} \\
\poemlll       No,\fnote{\fbackref{31:2} Or \fbib{What}} my son to whom I gave birth!\fnote{\fbackref{31:2} Lit. \fbib{son of my vows}} \\
\poeml \v{3}Never devote all your energy to sex,\fnote{\fbackref{31:3} Lit. \fbib{women}} \\
\poemll    or dedicate your life\fnote{\fbackref{31:3} Lit. \fbib{ways}} to destroying kings. \\
\poeml \v{4}It is not for kings, Lemuel--- \\
\poemll    Not for kings to drink wine \\
\poemlll       or for rulers to desire liquor. \\
\poeml \v{5}Otherwise, they may drink and forget what has been ordained, \\
\poemll    perverting justice for all the oppressed. \\
\poeml \v{6}Give liquor to someone who is perishing, \\
\poemll    and wine to someone who is deeply depressed.\fnote{\fbackref{31:6} Lit. \fbib{one whose soul is bitter}} \\
\poeml \v{7}Let him drink, forget his poverty, \\
\poemll    and remember his troubles no more. \\
\poeml \v{8}Speak for those who cannot speak; \\
\poemll    seek justice for all those on the verge of destruction.\fnote{\fbackref{31:8} Lit. \fbib{all sons of destruction}} \\
\poeml \v{9}Speak up, judge righteously, \\
\poemll    and defend the rights of the afflicted and oppressed.
\passage{The Honorable Woman}
\poeml \v{10}Who can find a capable wife? \\
\poemll    Her value far exceeds the finest jewels. \\
\poeml \v{11}Her husband has full confidence in her; \\
\poemll    as a result, he lacks nothing of value. \\
\poeml \v{12}She will bring good to him---never evil--- \\
\poemll    every day of her life. \\
\poeml \v{13}She seeks out wool and flax, \\
\poemll    working with eager hands. \\
\poeml \v{14}She is like a seagoing ship \\
\poemll    that brings her food from far away. \\
\poeml \v{15}She rises while it is still night, \\
\poemll    preparing meals for her family \\
\poemlll       and providing for her women servants. \\
\poeml \v{16}She evaluates a field and purchases it; \\
\poemll    from the proceeds she plants a vineyard. \\
\poeml \v{17}She clothes herself with fortitude, \\
\poemll    and fortifies her arms with strength. \\
\poeml \v{18}She is confident that her profits are sufficient. \\
\poemll    Her lamp does not go out at night. \\
\poeml \v{19}She works with her own hands on her clothes\fnote{\fbackref{31:19} Lit. \fbib{staff}}--- \\
\poemll    her hands work the sewing spindle. \\
\poeml \v{20}She reaches out to the poor, \\
\poemll    opening her hands to those in need. \\
\poeml \v{21}She is unafraid of winter's effect on\fnote{\fbackref{31:21} Lit. \fbib{of the snow for}} her household, \\
\poemll    because all of them are warmly clothed.\fnote{\fbackref{31:21} Lit. \fbib{are clothed in red}} \\
\poeml \v{22}She creates her own bedding, \\
\poemll    using fine linen and violet cloth. \\
\poeml \v{23}Her husband is well known;\fnote{\fbackref{31:23} Lit. \fbib{is known in the gates}} \\
\poemll    he sits among the leaders of the land. \\
\poeml \v{24}She designs and sells linen garments, \\
\poemll    supplying accessories\fnote{\fbackref{31:24} Or \fbib{belts}} to clothiers. \\
\poeml \v{25}Strength and dignity are her garments; \\
\poemll    she smiles about the future. \\
\poeml \v{26}She speaks wisely, \\
\poemll    teaching with gracious love. \\
\poeml \v{27}She looks discretely to the affairs of her household, \\
\poemll    and she is never lazy.\fnote{\fbackref{31:27} Lit. \fbib{she does not eat the food of laziness}} \\
\poeml \v{28}Her children stand up and encourage her, \\
\poemll    as does her husband, who praises her: \\
\poeml \v{29}``Many women have done wonderful things,'' he says,\fnote{\fbackref{31:29} The Heb. lacks \fbib{he says}} \\
\poemll    ``but you surpass all of them!'' \\
\poeml \v{30}Charm is deceitful and beauty fades; \\
\poemll    but a woman who fears the \divine{Lord} will be praised. \\
\poeml \v{31}Reward her for her work--- \\
\poemll    let her actions result in public praise.\fnote{\fbackref{31:31} Lit. \fbib{in praise in the gates}}\end{poetry}
