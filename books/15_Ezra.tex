\bookheader{Ezra}
\labelbook{Ezra}

\bookpretitle{The Book of}
\booktitle{Ezra}

\labelchapt{1}
\passage{An Edict to Rebuild the Temple}
\passageinfo{(2 Chronicles 36:22-23)}

\chapt{1}
\v{1}During the first year of Cyrus, king of Persia, in fulfillment of the message from the \divine{Lord} spoken through Jeremiah, the \divine{Lord} prompted\fnote{\fbackref{1:1} Lit. \fbib{stirred up the spirit of}} Cyrus, king of Persia, to make this proclamation throughout his entire kingdom, which was also released in written form:

\v{2}\divine{An Official Statement}

\divine{from}\fnote{\fbackref{1:2} Lit. \fbib{Thus says}} \divine{Cyrus, King of Persia}

\begin{poetry}
\poeml All of the kingdoms of the earth have been given to me by the \divine{Lord} God of Heaven, and he specifically charged me to build a temple\fnote{\fbackref{1:2} Or \fbib{house}, and so throughout the book} for him in Jerusalem, which is in Judah. \v{3}Therefore, who among the \divine{Lord}'s\fnote{\fbackref{1:3} Lit. \fbib{among all of his}} people trusts in his God? Whoever among this group wishes to do so may travel to Jerusalem of Judah to rebuild the Temple of the \divine{Lord}\fnote{\fbackref{1:3} Lit. \fbib{of his \divine{Lord}}} God of Israel, the God of Jerusalem. \v{4}Furthermore, everyone who wishes to repatriate\fnote{\fbackref{1:4} Lit. \fbib{who remains}} from any territory where he now resides is to receive assistance from his fellow residents in the form of silver, gold, equipment, and pack animals, in addition to voluntary offerings for the Temple of the God of Jerusalem.
\end{poetry}

\v{5}In response, the heads of the families\fnote{\fbackref{1:5} Lit. \fbib{fathers}} of Judah and Benjamin, the priests and descendants of Levi, and all those who had been prompted\fnote{\fbackref{1:5} Lit. \fbib{all whose spirit had been stirred up}} by God, prepared to travel to rebuild the Temple of the \divine{Lord}, which was in Jerusalem. \v{6}So all of their neighbors equipped the travelers\fnote{\fbackref{1:6} Lit. \fbib{strengthened their hands}} with silver, gold, equipment, pack animals, and valuable goods, in addition to voluntary offerings.
\passage{Temple Instruments Returned}

\v{7}King Cyrus also brought out from storage\fnote{\fbackref{1:7} The Heb. lacks \fbib{from storage}} the service instruments from the Temple of the \divine{Lord}, which Nebuchadnezzar had taken from Jerusalem and had placed in the temple of his gods.\fnote{\fbackref{1:7} LXX \fbib{his god}} \v{8}Cyrus, king of Persia, had them brought out to Mithredath the Treasurer, had them inventoried, and had them placed in care of\fnote{\fbackref{1:8} Lit. \fbib{Treasurer, and numbered them to}} Sheshbazzar,\fnote{\fbackref{1:8} I.e. Zerubbabel; \fbib{Sheshbazzar} is the Persian equivalent (cf. 2:2)} governor of Judah. \v{9}Here is a partial inventory:\fnote{\fbackref{1:9} Lit. \fbib{This was their number}}

Gold dishes: 30

Silver dishes: 1,000

Sacrificial knives: 29

\v{10}Gold bowls: 30

Silver bowls of another kind:\fnote{\fbackref{1:10} Lit. \fbib{of a second}} 410

Miscellaneous instruments: 1,000

\v{11}The complete inventory of gold and silver vessels totaled 5,400. Sheshbazzar\fnote{\fbackref{1:11} I.e. Zerubbabel; \fbib{Sheshbazzar} is the Persian equivalent (cf. 2:2)} brought them all to Jerusalem, along with the exiles from Babylon.
\labelchapt{2}
\passage{A List of Those who Returned}
\passageinfo{(Nehemiah 7:6-73)}

\chapt{2}
\v{1}Here is a list\fnote{\fbackref{2:1} Cf. Neh 7:6} of descendants of the province of Judah\fnote{\fbackref{2:1} The Heb. lacks \fbib{of Judah}} who returned from the captivity, from those who had been exiled. Nebuchadnezzar, king of Babylon, had taken them to Babylon. They came back to Jerusalem and Judah, each one to his town, \v{2}along with Zerubbabel, Jeshua, Nehemiah, Seraiah, Reelaiah,\fnote{\fbackref{2:2} MT of Neh 7:7 lacks \fbib{Seraiah, Reelaiah}} Mordecai, Bilshan, Mispar,\fnote{\fbackref{2:2} Cf. Neh 7:7 \fbib{Mispereth}} Bigvai, Rehum,\fnote{\fbackref{2:2} Cf. Neh 7:7 \fbib{Nehum}} and Baanah. Here is the enumeration of:

The Men of Israel:

\v{3}Descendants of\fnote{\fbackref{2:3} Lit. \fbib{Sons of}; and so throughout the chapter} Parosh: 2,172

\v{4}Descendants of Shephatiah: 372

\v{5}Descendants of Arah: 775\fnote{\fbackref{2:5} Cf. Neh 7:10 \fbib{652}}

\v{6}Descendants of Pahath-moab; that is, through Jeshua and Joab: 2,812\fnote{\fbackref{2:6} Cf. Neh 7:11 \fbib{2,818}}

\v{7}Descendants of Elam: 1,254

\v{8}Descendants of Zattu: 945\fnote{\fbackref{2:8} Cf. Neh 7:13 \fbib{845}}

\v{9}Descendants of Zaccai: 760

\v{10}Descendants of Bani:\fnote{\fbackref{2:10} Cf. Neh 7:13 \fbib{Binnui}} 642\fnote{\fbackref{2:10} Cf. Neh 7:13 \fbib{648}}

\v{11}Descendants of Bebai: 623\fnote{\fbackref{2:11} Cf. Neh 7:16 \fbib{628}}

\v{12}Descendants of Azgad: 1,222\fnote{\fbackref{2:12} Cf. Neh 7:17 \fbib{2,322}}

\v{13}Descendants of Adonikam: 666\fnote{\fbackref{2:13} Cf. Neh 7:18 \fbib{667}}

\v{14}Descendants of Bigvai: 2,056\fnote{\fbackref{2:14} Cf. Neh 7:19 \fbib{2,067}}

\v{15}Descendants of Adin: 454\fnote{\fbackref{2:15} Cf. Neh 7:20 \fbib{655}}

\v{16}Descendants of Ater through Hezekiah: 98

\v{17}Descendants of Bezai: 323\fnote{\fbackref{2:17} Cf. Neh 7:22 \fbib{328}}

\v{18}Descendants of Jorah:\fnote{\fbackref{2:18} Cf. Neh 7:24 \fbib{Hariph}} 112

\v{19}Descendants of Hashum: 223\fnote{\fbackref{2:19} Cf. Neh 7:22 \fbib{328}}

\v{20}Descendants of Gibbar:\fnote{\fbackref{2:20} Cf. Neh 7:25 \fbib{Gibeon}} 95

\v{21}Descendants of exiles from\fnote{\fbackref{2:21} The Heb. lacks \fbib{exiles from}; and so through v. 35} Bethlehem: 123

\v{22}People from\fnote{\fbackref{2:22} Lit. \fbib{Men of}; and so in vv. 23, 27, and 28} Netophah: 56\fnote{\fbackref{2:22} Cf. Neh 7:26, where the combined total is \fbib{188}}

\v{23}People from Anathoth: 128

\v{24}Descendants of exiles from Azmaveth:\fnote{\fbackref{2:24} Cf. Neh 7:28 \fbib{Beth-azmaveth}} 42

\v{25}Descendants of exiles from Kiriath-arim;\fnote{\fbackref{2:25} Cf. Neh 7:29 \fbib{Kiriath-jearim}} that is, Chephirah and Beeroth: 743

\v{26}Descendants of exiles from Ramah and Geba: 621

\v{27}People from Michmas: 122

\v{28}People from Bethel and Ai: 223\fnote{\fbackref{2:28} Cf. Neh 7:32 \fbib{123}}

\v{29}Descendants of exiles from Nebo: 52

\v{30}Descendants of exiles from Magbish: 156

\v{31}Descendants of exiles from the other Elam: 1,254

\v{32}Descendants of exiles from Harim: 320

\v{33}Descendants of exiles from Lod, Hadid, and Ono: 725\fnote{\fbackref{2:33} Cf. Neh 7:37 \fbib{721}}

\v{34}Descendants of exiles from Jericho: 345

\v{35}Descendants of exiles from Senaah: 3,630\fnote{\fbackref{2:35} Cf. Neh 7:38 3,930}

\v{36}The Priests:

Descendants of Jedaiah from the household of Jeshua: 973

\v{37}Descendants of Immer: 1,052

\v{38}Descendants of Pashhur: 1,247

\v{39}Descendants of Harim: 1,017

\v{40}The Descendants of Levi:

Descendants of Jeshua and Kadmiel; that is, descendants of Hodaviah:\fnote{\fbackref{2:40} Cf. Neh 7:43 \fbib{Hodevah}} 74

\v{41}The Singers:

Descendants of Asaph: 128\fnote{\fbackref{2:41} Cf. Neh 7:44 \fbib{148}}

\v{42}The Descendants of the Gatekeepers:

Descendants of Shallum, Ater, Talmon, Akkub, Hatita, and Shobai, totaling: 139\fnote{\fbackref{2:42} Cf. Neh 7:45 \fbib{138}}

\v{43}The Temple Servants:\fnote{\fbackref{2:43} Heb. \fbib{Nethinim}; i.e. a division of special assistants to the descendants of Levi, originally appointed by King David; and so throughout the book; cf. Ezra 2:58; 2:70; 7:7,24; 8:17,20.}

Descendants of Ziha, Hasupha, and Tabbaoth.

\v{44}Descendants of Keros, Siaha,\fnote{\fbackref{2:44} Cf. Neh 7:47 \fbib{Sia}} and Padon.

\v{45}Descendants of Lebanah, Hagabah, and Akkub.\fnote{\fbackref{2:45} Cf. Neh 7:48 \fbib{Shalmai}}

\v{46}Descendants of Hagab, Shalmai, and Hanan.

\v{47}Descendants of Giddel, Gahar, and Reaiah.

\v{48}Descendants of Rezin, Nekoda, and Gazzam.

\v{49}Descendants of Uzza, Paseah, and Besai.

\v{50}Descendants of Asnah,\fnote{\fbackref{2:50} Cf. Neh 7:52 \fbib{Besai}} Meunim, and Nephusim.

\v{51}Descendants of Bakbuk, Hakupha, and Harhur.

\v{52}Descendants of Bazluth, Mehida, and Harsha.

\v{53}Descendants of Barkos, Sisera, and Temah.

\v{54}Descendants of Neziah and Hatipha.

\v{55}The Descendants of Solomon's Servants:

Descendants of Sotai, Hassophereth,\fnote{\fbackref{2:55} Cf. Neh 7:57 \fbib{Sophereth}} and Peruda.\fnote{\fbackref{2:55} Cf. Neh 7:57 \fbib{Perida}}

\v{56}Descendants of Jaalah,\fnote{\fbackref{2:56} Cf. Neh 7:58 \fbib{Jaala}} Darkon, and Giddel.

\v{57}Descendants of Shephatiah, Hattil, Pochereth-hazzebaim, and Ami.\fnote{\fbackref{2:57} Cf. Neh 7:59 \fbib{Ammon}}

\v{58}All of the Temple Servants and descendants of Solomon's servants numbered 392.
\passage{Non-Documented Persons}
\passageinfo{(Nehemiah 7:61-69)}

\v{59}Here is a list of returnees from Tel-melah, Tel-harsha, Cherub, Addan, and Immer who could not prove their ancestry and lineage from Israel:

\v{60}Descendants of Delaiah, Tobiah, and Nekoda: 652\fnote{\fbackref{2:60} Cf. Neh 7:62 \fbib{642}}

\v{61}Descendants of the Priests:

Descendants of Habaiah, Hakkoz,\fnote{\fbackref{2:61} Cf. Neh 7:63 \fbib{Koz}} and Barzillai, who married one of the daughters of Barzillai from Gilead and took that name.

\v{62}These people searched for their ancestral registrations but they couldn't be located. Accordingly, they were assigned an ``unclean'' status and couldn't be priests. \v{63}Governor Zerubbabel\fnote{\fbackref{2:63} The Heb. lacks \fbib{Zerubbabel}} also ruled that they shouldn't eat anything holy until a priest arose with Urim and Thummim.\fnote{\fbackref{2:63} I.e. a high priest to whom God would reveal his will through the jewel-encrusted breastplate that he wore; cf. Exod 28:30, Neh 7:65}

\v{64}The entire assembly numbered 42,360, \v{65}not including 7,337 male and female servants, along with 200\fnote{\fbackref{2:65} Cf. Neh 7:66 \fbib{245}} singing men and women. \v{66}In addition, they had 736 horses, 245 mules, \v{67}435 camels, and 6,720 donkeys.
\passage{Gifts for the Temple}
\passageinfo{(Nehemiah 7:70-73)}

\v{68}When they arrived at the Temple of the \divine{Lord} in Jerusalem, some of the heads of the families contributed toward building the Temple of God on its former site. \v{69}They contributed to the treasury for this work in accordance with their ability: 61,000 golden drachma, 5,000 units\fnote{\fbackref{2:69} Lit. \fbib{mina}} of silver, and 100 priestly robes. \v{70}As a result, the priests, descendants of Levi, certain people, the singers, door-keepers, and the Temple Servants were able to settle in their original cities, with the rest of the Israelis in their cities.
\labelchapt{3}
\passage{Initial Offering Ceremonies}
\passageinfo{(Nehemiah 7:72)}

\chapt{3}
\v{1}Seven months after the Israelis had settled in their cities, they all gathered together in Jerusalem as a united body.\fnote{\fbackref{3:1} Lit. \fbib{together as one man in Jerusalem}} \v{2}Then Jozadak's son Jeshua and his brothers got up, along with Shealtiel's son Zerubbabel and his brothers. They built an altar of the God of Israel in order to offer burnt offerings, as prescribed by the Law of Moses, the man of God.

\v{3}Even though they feared the people in neighboring regions, they rebuilt the altar where it had stood before.\fnote{\fbackref{3:3} Lit. \fbib{altar on its bases}} They offered burnt offerings on it to the \divine{Lord}---burnt offerings both in the morning and in the evening. \v{4}They also observed the Festival of Tents\fnote{\fbackref{3:4} Or \fbib{Shelters}} as has been prescribed, offering a specific number of daily burnt offerings in accordance with the ordinance of each day. \v{5}After that, they offered\fnote{\fbackref{3:5} The Heb. lacks \fbib{they offered}} all of the continual burnt offerings and the New Moon sacrifices\fnote{\fbackref{3:5} Lit. \fbib{the moons}} for all of the designated festivals of the \divine{Lord} that were being consecrated, along with all the voluntary offerings that were dedicated to the \divine{Lord}. \v{6}They began to offer burnt offerings to the \divine{Lord} from the first day of the seventh month, even though the foundation of the Temple of the \divine{Lord} had not yet been laid.
\passage{Construction Begins on the Temple}

\v{7}They paid masons and carpenters in cash.\fnote{\fbackref{3:7} Lit. \fbib{silver}} They paid\fnote{\fbackref{3:7} The Heb. lacks \fbib{They paid}} the residents of Sidon and Tyre with food, drink, and oil, for them to bring cedar trees by sea from Lebanon to Joppa in accordance with the order they had obtained from Cyrus, king of Persia.

\v{8}Two years and two months after arriving at the site of the Temple of God in Jerusalem, Shealtiel's son Zerubbabel, Jozadak's son Jeshua, the relatives of the priests and descendants of Levi, and everyone else who had left the Babylonian\fnote{\fbackref{3:8} The Heb. lacks \fbib{Babylonian}} captivity for Jerusalem appointed descendants of Levi who were 20 years old and older to oversee the work of the \divine{Lord}'s Temple.

\v{9}At this time Jeshua, along with his children and relatives, and Kadmiel, with his children and the descendants of Judah, joined the family of Henadad with his children and relatives, and the descendants of Levi in overseeing the work on the Temple of God.
\passage{The Temple Foundation is Laid}

\v{10}After the builders laid the foundation for the \divine{Lord}'s Temple, the priests stood in their ministerial robes with trumpets and the descendants of Levi (who were also descendants of Asaph) with cymbals to praise the \divine{Lord}, according to instructions prepared by\fnote{\fbackref{3:10} Lit. \fbib{\divine{Lord} according to the hand of}} David, king of Israel. \v{11}And they sang in unison\fnote{\fbackref{3:11} Or \fbib{sang by antiphonal courses}} to one another, giving thanks to the \divine{Lord}:

\begin{poetry}
\poeml ``He is good, \\
\poemll    and his gracious love to Israel endures forever.''
\end{poetry}

And all the people shouted out loudly in praise to the \divine{Lord} when the foundation of the \divine{Lord}'s Temple was laid.
\passage{Remembering the Former Temple}

\v{12}Now a number of the priests, the Levities, and the leading officials of the elders---who were very\fnote{\fbackref{3:12} The Heb. lacks \fbib{very}} elderly---had seen the former Temple with their own eyes. When they observed the foundation of the Temple being laid, they wept with a loud voice, while the rest of them shouted for joy. \v{13}As a result, the people couldn't distinguish between the noise coming from the shouts of joy and the noise coming from the weeping people, because everyone\fnote{\fbackref{3:13} Lit. \fbib{the people}} was shouting loudly and could be heard a long way off.
\labelchapt{4}
\passage{A Plot to Hinder the Work}

\chapt{4}
\v{1}When the enemies of Judah and Benjamin learned that the descendants of the Babylonian\fnote{\fbackref{4:1} The Heb. lacks \fbib{Babylonian}} captivity had built their Temple to the \divine{Lord}, the God of Israel, \v{2}they approached Zerubbabel and the heads of the families\fnote{\fbackref{4:2} Lit. \fbib{fathers}} with this message: ``Let's build along with you, because, like you, we seek your God, as do you, and we've been making sacrifices to him since the reign of Esarhaddon, king of Assyria, who brought us here.''

\v{3}But Zerubbabel, Jeshua, and the rest of the heads of the families\fnote{\fbackref{4:3} Lit. \fbib{fathers}} of Israel replied, ``You have no part in our plans for\fnote{\fbackref{4:3} The Heb. lacks \fbib{plans for}} building a temple to our God, because we alone will build to the \divine{Lord}, the God of Israel, in accordance with the decree issued by King Cyrus, king of Persia.''
\passage{The Plot Succeeds---for a While}

\v{4}After this, the non-Israeli inhabitants\fnote{\fbackref{4:4} Lit. \fbib{the people}} of the land undermined\fnote{\fbackref{4:4} Lit. \fbib{weakened the hands of}} the people of Judah, harassing them in their construction work \v{5}by bribing their consultants in order to frustrate their plans throughout the reign of Cyrus, king of Persia until Darius became king.\fnote{\fbackref{4:5} Lit. \fbib{until the reign of Darius, king of Persia}}

\v{6}At the beginning of the reign of Ahasuerus, they lodged a formal accusation against the inhabitants of Judah and Jerusalem. \v{7}While Artaxerxes was king of Persia, Bishlam, Mithredath, Tabeel, and the rest of their co-conspirators wrote in the Aramaic language and script to King Artaxerxes of Persia.

Aramaic:\fnote{\fbackref{4:7} From this point through 6:18, the text of MT is in Aramaic.}

\v{8}Governor Rehum and Shimshai the scribe wrote a letter concerning Jerusalem to King Artaxerxes as follows:

\v{9}From Governor Rehum

Shimshai the scribe

The rest of their colleagues---

Judges, envoys, officials, Persians, the people of Erech, the Babylonians, the people of Susa (that is, the Elamites) \v{10}and many other nations whom the great and honorable Osnappar deported and resettled in Samaria and in the rest of the province beyond the Euphrates\fnote{\fbackref{4:10} The Aram. lacks \fbib{Euphrates}} River.

\v{11}This is the text of the letter they sent.

\begin{poetry}
\poeml To: King Artaxerxes \\
\poeml From: Your servants, the men of the province beyond the Euphrates\fnote{\fbackref{4:11} The Aram. lacks \fbib{Euphrates}} River. \\
\poeml \v{12}May the king be advised that the Jews who came from you to us have reached Jerusalem and are rebuilding a rebellious and wicked city, having completed its walls and repaired its foundations. \\
\poeml \v{13}May the king be further advised that if this city is rebuilt and its walls erected, its citizens\fnote{\fbackref{4:13} Lit. \fbib{erected, they}} will refuse to pay tributes, taxes, and tariffs, thereby restricting royal revenues. \\
\poeml \v{14}Now, because we are royal employees\fnote{\fbackref{4:14} Lit. \fbib{we received salt from the palace}} and are committed to preserving the reputation of the king, we have written to the king and have declared its contents to be true,\fnote{\fbackref{4:14} Lit. \fbib{and certified to the king}} \v{15}urging\fnote{\fbackref{4:15} The Aram. lacks \fbib{urging}} that a search may be made in the official registers of your predecessors.\fnote{\fbackref{4:15} Lit. \fbib{fathers}} You will discover in the registers that\fnote{\fbackref{4:15} Lit. \fbib{books and will know}} this city is a rebellious city, that it is damaging to both kings and provinces, that it has been moved to sedition from time immemorial, and that because of this it was destroyed. \\
\poeml \v{16}We certify to the king that if this city is rebuilt and its walls completed, you will lose your land holdings in the province beyond the Euphrates\fnote{\fbackref{4:16} The Aram. lacks \fbib{Euphrates}} River.
\end{poetry}
\passage{The Response of Ahasuerus}

\v{17}The king replied:

\begin{poetry}
\poeml To: Governor Rehum, Shimshai the scribe, and their colleagues living in Samaria, and the remainder living beyond the Euphrates\fnote{\fbackref{4:17} The Aram. lacks \fbib{Euphrates}} River. \\
\poeml Greetings:\fnote{\fbackref{4:17} Lit. \fbib{Peace, and now.}} \\
\poeml \v{18}The memorandum you sent to us has been read and carefully considered.\fnote{\fbackref{4:18} Lit. \fbib{been read plainly before me}} \v{19}Pursuant to my edict, an investigation has been undertaken. It is noted that this city has fomented rebellion against kings from time immemorial, and that rebellion and sedition has occurred in it. \\
\poeml \v{20}Powerful kings have reigned over Jerusalem, including ruling over all lands beyond the Euphrates\fnote{\fbackref{4:20} The Aram. lacks \fbib{Euphrates}} River. Furthermore, taxes, tribute, and tolls have been paid to them. \\
\poeml \v{21}Accordingly, issue an order to force these men to cease their work\fnote{\fbackref{4:21} The Aram. lacks \fbib{their work}} so that this city is not rebuilt until you receive further notice from me. \\
\poeml \v{22}Be diligent and take precautions so that you do not neglect your responsibility in this matter. Why should the kingdom sustain any more damage?
\end{poetry}
\passage{Reconstruction Ceases}

\v{23}As soon as a copy of the letter from King Artaxerxes was read to Rehum, to Shimshai the scribe, and to their colleagues, they traveled quickly to Jerusalem and compelled the Jews to cease by force of arms. \v{24}As a result, work on the Temple of God in Jerusalem ceased and did not begin again until the second year of the reign of King Darius of Persia.
\labelchapt{5}
\passage{Rebuilding Efforts Begin Again}
\passageinfo{(Haggai 1:1; Zechariah 1:1)}

\chapt{5}
\v{1}At that time, the prophets Haggai and Iddo's son Zechariah prophesied specifically to the Jews in Judah and Jerusalem in the name of the God of Israel. \v{2}So Shealtiel's son Zerubbabel and Jozadak's son Jeshua restarted construction of the Temple of God in Jerusalem. And the prophets of God were there supporting them.
\passage{Government Interference}

\v{3}Right about then, Trans-Euphrates\fnote{\fbackref{5:3} Lit. \fbib{Beyond the River}} Governor Tattenai, Shethar-bozenai, and their colleagues approached and challenged them. They asked, ``Who authorized you to build this Temple and to reconstruct this wall?'' \v{4}In answer, we responded with a list of\fnote{\fbackref{5:4} Lit. \fbib{responded thus: ``What are}} the names of the men who were building the structure. \v{5}But God watched over the Jewish leaders, who could not be forced to stop working until Darius received a report and responded in reply.
\passage{A Memorandum}

\v{6}Here is a copy of the letter that Trans-Euphrates\fnote{\fbackref{5:6} Lit. \fbib{Beyond the River}} Governor Tattenai, Shethar-bozenai, and his colleagues the Trans-Euphrates Persians sent to King Darius. \v{7}The letter sent to him was written like this:

\begin{poetry}
\poeml To: King Darius: \\
\poeml Greetings!\fnote{\fbackref{5:7} Lit. \fbib{All peace!}} \\
\poeml \v{8}This is to inform\fnote{\fbackref{5:8} Lit. \fbib{Let it be known to}} the king that we traveled to the Temple of the great God in the Judean province, which is being built with large stones and reinforced with wooden beams in its walls. The work proceeds diligently and is in capable hands.\fnote{\fbackref{5:8} Lit. \fbib{and prospers in their hands}} \\
\poeml \v{9}We asked the elders, ``Who authorized you to build this Temple and to reinforce these walls?'' \v{10}We also asked them their names so that we could certify the identities\fnote{\fbackref{5:10} Lit. \fbib{could write the names}} of their leaders to you. \\
\poeml \v{11}In answer they responded, ``We are servants of the God of heaven and earth, and are rebuilding the Temple that was built many years ago by a great king of Israel. \v{12}But because our predecessors provoked the God of Heaven to become angry, he handed them over to the control\fnote{\fbackref{5:12} Lit. \fbib{hand}} of King Nebuchadnezzar of Babylon, the Chaldean who destroyed this Temple and transported the people to Babylon. \\
\poeml \v{13}However, during King Cyrus' first year---that same King Cyrus of Babylon---issued a decree to reconstruct this Temple of God. \v{14}He delivered into the care of Sheshbazzar (whom he appointed governor) the gold and silver utensils that Nebuchadnezzar had taken from the Jerusalem Temple and brought into the Babylonian temple. \\
\poeml \v{15}``And Cyrus\fnote{\fbackref{5:15} Lit. \fbib{he}} told him, `Take these utensils, go to Jerusalem, and carry them to the Temple, after the Temple of God has been built\fnote{\fbackref{5:15} Lit. \fbib{temple, and let the temple of God be built}} in its appropriate place.' \\
\poeml \v{16}``Then this very same Sheshbazzar arrived and laid the foundations for the Temple of God in Jerusalem. Since that time until now the Temple has been under construction and is not yet completed.'' \\
\poeml \v{17}Accordingly, with your approval we suggest that\fnote{\fbackref{5:17} Lit. \fbib{Accordingly, if it seems good to the king, let}} a search be conducted within the king's treasury at Babylon to verify\fnote{\fbackref{5:17} The Aram. lacks \fbib{to verify}} whether or not King Cyrus ever issued such a decree to reconstruct this Temple of God in Jerusalem. Then please notify us concerning the king's pleasure in this matter.
\end{poetry}
\labelchapt{6}
\passage{King Darius Verifies the Decree}

\chapt{6}
\v{1}Then King Darius issued an order to search the Hall of Records where the Babylonian archives were stored. \v{2}The following was found written on a scroll in Ecbatana at the summer\fnote{\fbackref{6:2} The Aram. lacks \fbib{summer}} palace of the province of Media:

\begin{poetry}
\poeml \v{3}\divine{Date}: First year of Cyrus the King \\
\poeml \divine{From}: King Cyrus \\
\poeml \divine{Subject}: The Temple of God in Jerusalem \\
\poeml Let the Temple be rebuilt where they offered sacrifices. Let the foundations thereof be laid with a height of 60 cubits\fnote{\fbackref{6:3} I.e. about 90 feet; a cubit was about eighteen inches} and a width of 60 cubits,\fnote{\fbackref{6:3} I.e. about 90 feet; a cubit was about eighteen inches} \v{4}constructed\fnote{\fbackref{6:4} The Aram. lacks \fbib{constructed}} with three layers of foundation\fnote{\fbackref{6:4} Lit. \fbib{heavy}} stone interlaced with a row of new timber, the expenses for which are to be paid from the king's treasury. \\
\poeml \v{5}Furthermore, let the gold and silver utensils from the Temple of God (that Nebuchadnezzar took from the Temple in Jerusalem and carried off to Babylon) be brought back to the Temple at Jerusalem and restored to their respective places in the Temple of God.
\passage{King Darius Confirms the Decree}
\poeml \v{6}To: Tattenai, Trans-Euphrates Governor, Shethar-bozenai, and your colleagues living beyond the Euphrates\fnote{\fbackref{6:6} The Aram. lacks \fbib{Euphrates}} River. \\
\poeml Stay away from there! \\
\poeml \v{7}Leave the work on this Temple of God alone! \\
\poeml Let the Jewish governor and the Jewish leaders build this Temple of God on its site. \\
\poeml \v{8}Furthermore, I hereby decree what you are to do for the Jewish leaders who are building this Temple of God: you are to pay the expenses of these men out of the king's assets from taxes collected\fnote{\fbackref{6:8} The Aram. lacks \fbib{collected}} beyond the River so that they are not hindered. \\
\poeml \v{9}And be sure that you don't fail to provide their daily needs---including young bulls, rams, and lambs for the burnt offerings of the God of Heaven, along with wheat, salt, wine, and oil, as the priests in Jerusalem tell you--- \v{10}so they may approach the God of Heaven with fragrant sacrifices and pray for the life of this king and his sons. \\
\poeml \v{11}I hereby also decree that whoever shall alter the wording of this edict, let his residence be torn down for timber to build a gallows,\fnote{\fbackref{6:11} The Aram. lacks \fbib{a gallows}} hang\fnote{\fbackref{6:11} Or \fbib{impale}} him on it, and turn his home into an outhouse. \v{12}And may the God who causes his Name to rest there destroy any king or people who might attempt\fnote{\fbackref{6:12} Lit. \fbib{shall put their hand out}} to destroy this Temple of God in Jerusalem. \\
\poeml I, Darius, have issued this decree. Let it be carried out quickly.
\end{poetry}

\v{13}Because of what King Darius had mandated, Tattenai, the Trans-Euphrates Governor, Shethar-bozenai, and their colleagues carried out his orders quickly.
\passage{Progress and Completion}

\v{14}And so the Jewish leaders continued their building, and prospered because of the prophecies of Haggai the prophet and Iddo's son Zechariah. They completed the rebuilding in accordance with the commandment from the God of Israel and the edicts of Cyrus, Darius, and Artaxerxes, king of Persia. \v{15}The Temple was completed on the third day of the month Adar during the sixth year of the reign of King Darius.

\v{16}The Israelis---the priests, the descendants of Levi, and the other related descendants who had returned from captivity---celebrated with joy at the dedication of the Temple of God. \v{17}At the dedication offering of the Temple of God, they presented 100 bulls, 200 rams, and 400 lambs, along with a sin offering of twelve male goats for the entire nation of Israel according to the number of the tribes of Israel.

\v{18}Furthermore, they established the priests in their divisions and the descendants of Levi in their positions for the service of God conducted at Jerusalem, as is proscribed in the Book of Moses.
\passage{The First Post-Captivity Passover}
\passageinfo{(Deuteronomy 16:1-8)}

\v{19}\fnote{\fbackref{6:19} At this point, the text of MT reverts to Heb.}The former exiles\fnote{\fbackref{6:19} Lit. \fbib{The sons of the captivity}} observed the Passover on the fourteenth day of the first month \v{20}because the priests and descendants of Levi had purified themselves together---all of them were pure---and they killed the Passover lamb\fnote{\fbackref{6:20} The Heb. lacks \fbib{lamb}} for every former exile,\fnote{\fbackref{6:20} Lit. \fbib{for all of the sons of the captivity}} for their relatives the priests, and for themselves.

\v{21}So the Israelis who had returned from captivity ate the Passover with all who had consecrated themselves from the uncleanness of the nations of the land in order to seek the \divine{Lord} God of Israel. \v{22}Then they observed the Festival of Unleavened Bread for seven days with joy, because the \divine{Lord} had made them glad, turning the heart of the king of Assyria toward them and strengthening them for their work on the Temple of God, the God of Israel.
\labelchapt{7}
\passage{Ezra's Return to Jerusalem}
\passageinfo{(Ezra 2:1-70)}

\chapt{7}
\v{1}After all of this, during the reign of King Artaxerxes of Persia, Seraiah's son Ezra (who was the grandson of Azariah, son of Hilkiah, \v{2}son of Shallum, son of Zadok, son of Ahitub, \v{3}son of Amariah, son of Azariah, son of Meraioth, \v{4}son of Zerahiah, son of Uzzi, son of Bukki, \v{5}son of Abishua, son of Phinehas, son of Eleazar, son of Aaron the chief priest) \v{6}left\fnote{\fbackref{7:6} Lit. \fbib{Ezra himself left}} Babylon. He was a skillful scribe of the Law of Moses that the \divine{Lord} God of Israel had given. And the king granted him everything he had requested because the hand of the \divine{Lord} his God was upon him. \v{7}Some of the descendants of Israel also left for Jerusalem, including the priests, the descendants of Levi, the singers, the gatekeepers, and the Temple Servants, during the seventh year of king Artaxerxes.

\v{8}He arrived in Jerusalem during the fifth month of the seventh year of the king's reign.\fnote{\fbackref{7:8} Lit. \fbib{seventh of the king}} \v{9}On the first day\fnote{\fbackref{7:9} The Heb. lacks \fbib{day}} of the first month he left Babylon and he arrived in Jerusalem on the first day\fnote{\fbackref{7:9} The Heb. lacks \fbib{day}} of the fifth month, since the beneficent hand of his God was upon him. \v{10}For Ezra had set his heart to seek the Law of the \divine{Lord}, to obey it, and to teach God's\fnote{\fbackref{7:10} The Heb. lacks \fbib{God's}} statutes and judgments in Israel.
\passage{The Letter from King Artaxerxes}

\v{11}Here is a copy of the letter that King Artaxerxes gave to Ezra, the priest-scribe, a scholar\fnote{\fbackref{7:11} Or \fbib{scribe}} in matters concerning the commandments of the \divine{Lord} and concerning his statutes pertaining to Israel:

\begin{poetry}
\poeml \v{12}From:\fnote{\fbackref{7:12} At this point, the text of MT changes to Aramaic through verse 26.} Artaxerxes, King of Kings \\
\poeml To: Ezra, the Priest, a scholar\fnote{\fbackref{7:12} Or \fbib{scribe}} in matters concerning the laws of the God of Heaven \\
\poeml Greetings!\fnote{\fbackref{7:12} Lit. \fbib{Perfect and so forth}} \\
\poeml \v{13}I hereby decree that all of the people of Israel--- along with their priests and descendants of Levi in my kingdom---who are determined to return to Jerusalem with you may do so. \v{14}You have authority to act for the king and for his Council of Seven to conduct an inquiry concerning Judah and Jerusalem in accordance with the Law of your God, which is in your possession. \v{15}You are carrying silver and gold that the King and his advisors have freely given to the God of Israel, whose Temple is in Jerusalem, \v{16}together with all of the silver and gold that you can raise in the province of Babylon, plus the freewill offerings given by the people and the priests, contributed for the Temple of their God, which is in Jerusalem. \\
\poeml \v{17}Accordingly, you are to exercise due diligence to utilize this money to purchase bulls, rams, lambs, grain offerings, and drink offerings, and to offer them upon the altar of the Temple of your God, which is in Jerusalem. \\
\poeml \v{18}Furthermore, the balance remaining of the silver and gold may be used for whatever other purpose you and your people desire, as long as such use is consistent with the will of your God. \\
\poeml \v{19}Furthermore, you are to deliver to the God of Jerusalem the vessels for the service of the Temple of your God that have been given to you. \\
\poeml \v{20}Furthermore, provide from the royal treasury whatever else may be needed for the Temple of your God. \\
\poeml \v{21}I, Artaxerxes, in my capacity as king,\fnote{\fbackref{7:21} Lit. \fbib{And I, even I, Artaxerxes the King}} hereby decree to all royal treasuries beyond the Euphrates\fnote{\fbackref{7:21} The Aram. lacks \fbib{Euphrates}} River that whatever Ezra the priest-scribe of the Law of the God of Heaven, may require of you are to be performed with all due diligence, \v{22}up to 100 silver talents,\fnote{\fbackref{7:22} I.e. about 7,500 pounds; a talent weighed about 75 pounds} 100 measures of wheat, 100 measures of wine, 100 measures of oil, and salt without limitation. \v{23}Whatever is commanded by the God of Heaven is to be done with all due diligence for the Temple of the God of Heaven, or wrath will come against the king's realm and his sons. \\
\poeml \v{24}Furthermore, we decree that with respect to any of the priests, descendants of Levi, singers, gatekeepers, Temple Servants, or other servants of this Temple of God, it is not to be lawful to impose any tribute, tax, or toll on them. \\
\poeml \v{25}And you, Ezra, in accordance with the wisdom given to you by your God, are to appoint magistrates and judges to administer justice to all the people beyond the Euphrates\fnote{\fbackref{7:25} The Aram. lacks \fbib{Euphrates}} River. All of them are to know the laws of your God, and you are to instruct those who do not know them. \v{26}Whoever refuses to practice the law of your God and the law of the king is to see judgment executed quickly, whether to death, banishment, confiscation of goods, or imprisonment.
\passage{Ezra's Response to the Letter}
\poeml \v{27}Blessed be the \divine{Lord} God of our ancestors, \\
\poemll    who placed this decree\fnote{\fbackref{7:27} The Heb. lacks \fbib{decree}} into the king's heart \\
\poemlll       to beautify the Temple of the \divine{Lord} in Jerusalem \\
\poeml \v{28}and who showed gracious love to me before the king, \\
\poemlll       before his advisors, \\
\poemlll       and before all of the king's mighty officials.
\end{poetry}

And I was strengthened because the hand of the \divine{Lord} my God was upon me. So I gathered together the leaders of Israel to go with me.
\labelchapt{8}
\passage{Ezra's List of Family Leaders}

\chapt{8}
\v{1}These are the leaders of the families listed among those who left Babylon with me during the reign of King Artaxerxes: \v{2}From Phinehas's descendants: Gershom. From Ithamar's descendants: Daniel. From David's descendants: Hattush. \v{3}From Shecaniah's descendants and\fnote{\fbackref{8:3} The Heb. lacks \fbib{and}} from Parosh's descendants: Zechariah, along with 150 men whose genealogies had been certified. \v{4}From Pahath-moab's descendants: Zerahiah's son Eliehoenai and 200 men with him. \v{5}From Zattu's descendants: Jahaziel's son Shecaniah and 300 men with him. \v{6}From Adin's descendants: Jonathan's son Ebed and 50 men with him. \v{7}From Elam's descendants: Athaliah's son Jeshaiah and 70 men with him. \v{8}From Shephatiah's descendants: Michael's son Zebadiah and 80 men with him. \v{9}From Joab's descendants: Jehiel's son Obadiah and 218 men with him. \v{10}From Bani's descendants:\fnote{\fbackref{8:10} So LXX. The Heb. lacks \fbib{Bani}} Josiphiah's son Shelomith and 160 men with him. \v{11}From Bebai's descendants: Bebai's son Zechariah and 28 men with him. \v{12}From Azgad's descendants: Hakkatan's son Johanan and 110 men with him. \v{13}From Adonikam's later descendants: Eliphelet, Jeuel, Shemaiah, and 60 men with him. \v{14}From Bigvai's descendants: Uthai, Zabbud,\fnote{\fbackref{8:14} So MT, but \fbib{qere} directs that the name be read \fbib{Zaccur}, perhaps due to confusion with the nearly identical Heb. word for \fbib{men} (\fbib{Zecarim})} and 70 men with him.
\passage{Ezra Calls the Leaders to Fast}

\v{15}I gathered them together at the river that flows toward Ahava,\fnote{\fbackref{8:15} I.e. about 80 miles northwest of Babylon (cf. 2Kings 17:24)} where we camped three days. Afterwards, I examined the people and the priests, but found no descendants of Levi there. \v{16}So I sent for Eliezer, Ariel, Shemaiah, Elnathan, Jarib, Elnathan, Nathan, Zechariah, and Meshullam, who were senior leaders, as well as for Joiarib and Elnathan, who were men of discernment. \v{17}I told them to go see Iddo, a leader of Casiphia, and tell him and his relatives (administrators of Casiphia) to bring us men who could serve in the Temple of our God. \v{18}By the grace\fnote{\fbackref{8:18} Lit. \fbib{the good hand}} of our God they brought back a discerning man from the descendants of Mahli, a descendant of Israel's son Levi, along with Sherebiah and eighteen of his sons and brothers; \v{19}Hashabiah and Jeshaiah from the descendants of Merari and 20 of his brothers and their sons; \v{20}220 descendants of the Temple Servants whom David and the leaders had appointed to serve the descendants of Levi, listed by name.

\v{21}Then I called for a fast there at the Ahava River so we could humble ourselves before our God and seek from him an appropriate way for us and our little ones to live, and how we should guard our personal wealth,\fnote{\fbackref{8:21} Lit. \fbib{and for our wealth}} \v{22}because I was ashamed to ask the king for a contingent of soldiers and cavalry to protect us from enemies we might encounter\fnote{\fbackref{8:22} The Heb. lacks \fbib{we might encounter}} on the way. After all, we had told the king, ``The hand of our God seeks the good of all who seek him,\fnote{\fbackref{8:22} Lit. \fbib{God upon all who seek him to the good}} but his power and anger are against everyone who forsakes him.'' \v{23}So we fasted and asked our God about this, and he listened to us.
\passage{Ezra Delegates Responsibilities}

\v{24}Next I selected twelve of the chief priests---Sherebiah, Hashabiah, and ten of their brothers with them--- \v{25}and divided between them the silver, the gold, the vessels, and the offering for the Temple of our God which the king had offered, along with his advisors, his senior officials, and all of Israel assembled there. \v{26}I divided among them\fnote{\fbackref{8:26} Lit. \fbib{divided upon their hand}} 650 silver talents,\fnote{\fbackref{8:26} I.e. about 48,750 pounds; a talent weighed about 75 pounds} silver utensils weighing 100 talents,\fnote{\fbackref{8:26} I.e. about 7,500 pounds; a talent weighed about 75 pounds} 100 talents\fnote{\fbackref{8:26} I.e. about 7,500 pounds; a talent weighed about 75 pounds} of gold, \v{27}20 gold basins weighing 1,000 darics\fnote{\fbackref{8:27} I.e. about 15.6 pounds; a daric weighed about one quarter of an ounce} each, and two vessels made of polished brass, as valuable as gold.

\v{28}I told them, ``You are consecrated\fnote{\fbackref{8:28} Or \fbib{holy}} to the \divine{Lord}, and the vessels are also consecrated.\fnote{\fbackref{8:28} Or \fbib{holy}} The silver and the gold are a freely given offering to the \divine{Lord} God of your ancestors. \v{29}Guard and protect them until you disperse them to the chief priests, the descendants of Levi, and to the family leaders of Israel at Jerusalem in the chambers of the Temple of the \divine{Lord}.''

\v{30}So the priests and descendants of Levi took possession of the silver, the gold, and the vessels in order to bring them to Jerusalem, to the Temple of our God.

\v{31}Then we left the Ahava River for Jerusalem on the twelfth day\fnote{\fbackref{8:31} The Heb. lacks \fbib{day}} of the first month. Our God's protection\fnote{\fbackref{8:31} Lit. \fbib{hand}} was with us, and he delivered us from the enemy's power\fnote{\fbackref{8:31} Lit. \fbib{hand}} and from ambush along the way.
\passage{Ezra's Delegation Arrives in Jerusalem}

\v{32}We arrived in Jerusalem and remained there three days. \v{33}On the fourth day the silver, the gold, and the vessels were distributed at the Temple of our God into the care\fnote{\fbackref{8:33} Lit. \fbib{hand}} of Uriah's son Meremoth the priest, Phinehas' son Eleazar, Jeshua's son Jozabad, and Binnui's son Noadiah, the descendants of Levi. \v{34}Distribution was according to inventory\fnote{\fbackref{8:34} Lit. \fbib{By number}} and weight, with all weights being recorded at that time.

\v{35}The descendants of those who had been taken into captivity and who had returned from captivity offered burnt offerings to the God of Israel: twelve bulls for all of Israel, 96 rams, 77 lambs, and twelve male goats as a sin offering---all of them burnt offerings to the \divine{Lord}. \v{36}Then they delivered copies of\fnote{\fbackref{8:36} The Heb. lacks \fbib{copies of}} the king's orders to the king's officers, and governors on this side of the Euphrates\fnote{\fbackref{8:36} The Heb. lacks \fbib{Euphrates}} River. The orders were in support of the people and God's Temple.
\labelchapt{9}
\passage{Ezra's Reaction to Foreign Marriages}
\passageinfo{(Nehemiah 13:23)}

\chapt{9}
\v{1}After these things occurred, certain\fnote{\fbackref{9:1} The Heb. lacks \fbib{certain}} officials approached me and said ``The people of Israel, the priests, and the descendants of Levi have not separated themselves from the people of the lands or from the detestable behavior of the Canaanites, the Hittites, the Perizzites, the Jebusites, the Ammonites, the Moabites, the Egyptians, and the Amorites, \v{2}because they and their sons have married foreign women.\fnote{\fbackref{9:2} Lit. \fbib{married some of their daughters}} As a result, the holy people\fnote{\fbackref{9:2} Lit. \fbib{seed}} have mingled themselves among the people who live in these lands. As a matter of fact, the senior officials and the rulers have been foremost in this sin.''

\v{3}When I heard this, I tore both my garment and robe, plucked hair from both my head and my beard, and collapsed in shock! \v{4}Then everyone who trembled at the words of the God of Israel gathered together as a group because of the sin committed by those who had been led astray. As for me, I remained seated, in shock, until the evening sacrifice.
\passage{Ezra's Prayer of Repentance}

\v{5}At the time of the evening sacrifice, I arose from my discouragement. Still in my torn garment and robe, I fell to my knees with my hands outstretched to the \divine{Lord} my God, \v{6}and said,

\begin{poetry}
\poeml ``My God, I am too ashamed and hurt to turn to you, because we're in our iniquities over our heads. Furthermore, my God, our sins have grown as high as the heavens. \v{7}We have lived in great sin from the days of our ancestors even until today, and because of those iniquities we, our kings, and our priests have been delivered over to foreign kings, for execution, for captivity, for plunder, and for humiliation, as is the case\fnote{\fbackref{9:7} The Heb. lacks \fbib{is the case}} today. \v{8}Though now, for a moment, grace has been shown\fnote{\fbackref{9:8} The Heb. lacks \fbib{shown}} from the \divine{Lord} our God, leaving a few survivors to escape, and providing us a secure hold in his Holy Place, so that our God might enlighten us and give us relief from our servitude. \v{9}Even though we are slaves, our God has not abandoned us in our slavery. Instead, he has extended gracious love to us in the presence of the kings of Persia, to grant us revival, to set up the Temple of our God, to repair its damage, and to give us a protective wall for Judah and Jerusalem. \\
\poeml \v{10}Now, our God, what can we say besides this? Because we have abandoned your commandments \v{11}that you gave in the writings\fnote{\fbackref{9:11} Lit. \fbib{you ordered by the hand}} of your servants, the prophets: \\
\poeml `The land you are entering to possess is a morally unclean land due to the moral uncleanness of the people who live in there---along with their abominations---that has filled it from one end to the other with their impurities. \v{12}So, therefore, do not give your daughters in marriage to their sons, nor marry their daughters to your sons, and under no circumstances are you to seek their well-being or their wealth, so that you may remain strong, enjoying the best things the land has to give, and so that you may establish an inheritance for your children forever.'\fnote{\fbackref{9:12} This quotation from the prophets is not contained in MT} \\
\poeml \v{13}``After all that has happened to us because of our evil behavior, and because of our great sin---considering that you our God have punished us less than our iniquities deserve\fnote{\fbackref{9:13} The Heb. lacks \fbib{deserve}} and have given us this deliverance--- \v{14}should we violate your commandments by intermarrying with the nations\fnote{\fbackref{9:14} Lit. \fbib{peoples}} who practice these abominations? Would you not be angry with us until you had consumed us, with not even a remnant surviving\fnote{\fbackref{9:14} The Heb. lacks \fbib{surviving}} to escape? \\
\poeml \v{15}\divine{Lord} God of Israel, you are just: As a result, we remain here today delivered. Look at us! Because of our sin, we cannot stand in your presence as a result of everything that has happened.''
\end{poetry}
\labelchapt{10}
\passage{The People Gather with Ezra}

\chapt{10}
\v{1}Now while Ezra was praying and confessing in tears, having prostrated himself to the ground before the Temple of God, a very large crowd of Israelis---men, women, and children---gathered around him. Indeed, the people were crying bitterly.

\v{2}Jehiel's son Shecaniah, one of Elam's descendants, responded to Ezra: ``We have sinned against our God by marrying foreign wives from the people of the land, but even now there is hope in Israel, despite this. \v{3}So let's make a promise to our God by which we divorce\fnote{\fbackref{10:3} Or \fbib{expel}} all of these foreign\fnote{\fbackref{10:3} The Heb. lacks \fbib{of these foreign}} wives---as well as those born to them---in accordance with the counsel of our Lord and of those who tremble at our God's command. Furthermore, let it be done according to the Law. \v{4}So get up---it's your responsibility! We're with you. Be strong, and get to work.''\fnote{\fbackref{10:4} Or \fbib{and do it}}
\passage{The People Agree to Dissolve Their Marriages}

\v{5}So Ezra got up and made the chief priests, the descendants of Levi, and all of Israel vow to carry out everything they promised. And so they agreed.\fnote{\fbackref{10:5} Lit. \fbib{swore}} \v{6}Ezra arose in front of the Temple of God to visit the apartment of Eliashib's son Jehohanan. While there, he neither ate nor drank because he was in mourning over the sins of those who had returned from exile. \v{7}Then they sent word throughout Judah and Jerusalem to everyone who had returned from the exile, to gather together in Jerusalem. \v{8}Whoever would not come within three days would forfeit his assets and be separated from the community of the returning exiles, just as the high officials and elders had advised.

\v{9}Less than three days later, all of the men of Judah and Benjamin gathered together on the twentieth day\fnote{\fbackref{10:9} The Heb. lacks \fbib{day}} of the ninth month. Everyone sat in the plaza of the Temple of God, trembling because of everything that was happening, and also because it was raining heavily. \v{10}Ezra the priest stood up and spoke to them, ``You have sinned by marrying foreign wives, thereby increasing the transgressions of Israel. \v{11}Now confess this to the \divine{Lord} God of your ancestors, and separate yourselves from the people who live in the land and from foreign wives.''

\v{12}At this, the entire community responded with a loud cry, ``We will do just as you've spoken! \v{13}However, many people are involved, and it's raining heavily. Furthermore, this is not just a matter of a day or two of work, because many of us have sinned in this. \v{14}So let's have our officials remain on behalf of the whole community. Then all who have married foreign wives are to come appear at specific times before\fnote{\fbackref{10:14} Lit. \fbib{with}} the elders and judges of each city until the fierce anger of our God has been turned away from us in this matter.''

\v{15}Only Asahel's son Jonathan and Tikvah's son Jahzeiah opposed this, and they were supported by Meshullam and Shabbethai the descendant of Levi.
\passage{The People Carry Out Their Promise}

\v{16}So those who had returned from exile did this. Ezra the priest and leaders of certain ancestral groups listed by name devoted themselves to examine the situation on the first day of the tenth month. \v{17}By the first day of the first month they concluded their investigation of all of the men who had married foreign wives.
\passage{Those who Married Foreign Women}

\v{18}Here is a list of priestly descendants who were found to have married foreign women. From Jeshua's descendants:\fnote{\fbackref{10:18} Lit. \fbib{Children of}; and so through v. 43} Jozadak's son and his brothers Maaseiah, Eliezer, Jarib, and Gedaliah. \v{19}Pleading guilty, they promised to divorce their wives. Then they offered a ram from their flocks for their offense.

\v{20}From Immer's descendants: Hanani and Zebadiah. \v{21}From Harim's descendants: Maaseiah, Elijah, Shemaiah, Jehiel, and Uzziah. \v{22}From Pashhur's descendants: Elioenai, Maaseiah, Ishmael, Nethanel, Jozabad, and Elasah. \v{23}From the descendants of Levi: Jozabad, Shimei, Kelaiah (that is, Kelita), Pethahiah, Judah, and Eliezer. \v{24}From the singers: Eliashib. From the gatekeepers: Shallum, Telem, and Uri.

\v{25}From the Israelis: Parosh's descendants: Ramiah, Izziah, Malchijah, Mijamin, Eleazar, Malchijah, and Benaiah. \v{26}From Elam's descendants: Mattaniah, Zechariah, Jehiel, Abdi, Jeremoth, and Elijah. \v{27}From Zattu's descendants: Elioenai, Eliashib, Mattaniah, Jeremoth, Zabad, and Aziza. \v{28}From Bebai's descendants: Jehohanan, Hananiah, Zabbai, and Athlai. \v{29}From Bani's descendants: Meshullam, Malluch, Adaiah, Jashub, Sheal, and Jeremoth. \v{30}From Pahath-moab's descendants: Adna, Chelal, Benaiah, Maaseiah, Mattaniah, Bezalel, Binnui, and Manasseh. \v{31}From Harim's descendants: Eliezer, Isshijah, Malchijah, Shemaiah, Shimeon, \v{32}Benjamin, Malluch, and Shemariah.

\v{33}From Hashum's descendants: Mattenai, Mattattah, Zabad, Eliphelet, Jeremai, Manasseh, and Shimei. \v{34}From Bani's descendants: Maadai, Amram, Uel, \v{35}Benaiah, Bedeiah, Cheluhi, \v{36}Vaniah, Meremoth, Eliashib, \v{37}Matanza, Maternai, Jas, \v{38}Ban\'{i}, Binai, Shihezi, \v{39}Shelemiah, Nathan, Adaiah, \v{40}Machnadebai, Shashai, Sharai, \v{41}Azarel, Shelemiah, Shemariah, \v{42}Shallum, Amariah, and Joseph. \v{43}From Nebo's descendants: Jeiel, Mattithiah, Zabad, Zebina, Jaddai, Joel, and Benaiah.

\v{44}All of these had married foreign wives, and some of them had children by them.
