\bookheader{Colossians}
\labelbook{Col}

\bookpretitle{The Letter from Paul to the}
\booktitle{Colossians}

\labelchapt{1}
\passage{Greetings from Paul}

\chapt{1}
\v{1}From:\fnote{\fbackref{1:1} The Gk. lacks \fbib{From}} Paul, an apostle of the Messiah\fnote{\fbackref{1:1} Or \fbib{Christ}} Jesus by the will of God, and Timothy our brother.

\v{2}To: The holy\fnote{\fbackref{1:2} Or \fbib{to the saints}} and faithful brothers in Colossae who are in union with the Messiah.\fnote{\fbackref{1:2} Or \fbib{Christ}}

May grace and peace from God our Father\fnote{\fbackref{1:2} Other mss. read \fbib{from God our Father and the Lord Jesus, the Messiah}} be yours!
\passage{Paul's Prayer for the Colossians}

\v{3}We give thanks to God, the Father of our Lord Jesus, the Messiah,\fnote{\fbackref{1:3} Or \fbib{Christ}} praying always for you, \v{4}because we have heard about your faith in the Messiah\fnote{\fbackref{1:4} Or \fbib{Christ}} Jesus and the love that you have for all the saints, \v{5}based on the hope laid up for you in heaven. Some time ago you heard about this hope\fnote{\fbackref{1:5} Lit. \fbib{about it}} through the word of truth, the gospel \v{6}that has come to you. Just as it is bearing fruit and spreading all over the world, so it has been doing\fnote{\fbackref{1:6} The Gk. lacks \fbib{it has been doing}} among you from the day you heard it and came to know the grace of God in truth. \v{7}You learned about this gospel\fnote{\fbackref{1:7} Lit. \fbib{Just as you learned}} from Epaphras, our dear fellow servant, who is a faithful minister of the Messiah\fnote{\fbackref{1:7} Or \fbib{Christ}} on your\fnote{\fbackref{1:7} Other mss. read \fbib{our}} behalf. \v{8}He has told us about your love in the Spirit.
\passage{The Messiah is Above All}

\v{9}For this reason, since the day we heard about this, we have not stopped praying for you and asking that you may be filled with the full knowledge of God's\fnote{\fbackref{1:9} Lit. \fbib{his}} will with respect to all spiritual wisdom and understanding, \v{10}so that you might live in a manner worthy of the Lord and be fully pleasing to him\fnote{\fbackref{1:10} Lit. \fbib{to all pleasing}} as you bear fruit while doing all kinds of good things and growing in the full knowledge of God. \v{11}You are being strengthened with all power according to his glorious might, so that you might patiently endure everything with joy \v{12}and might thank the Father, who has enabled us\fnote{\fbackref{1:12} Other mss. read \fbib{you}} to share in the saints' inheritance in the light. \v{13}God\fnote{\fbackref{1:13} Lit. \fbib{He}} has rescued us from the power of darkness and has brought us into the kingdom of the Son whom he loves, \v{14}through whom we have redemption, the forgiveness of sins.
\passage{The Centrality of Jesus}

\begin{poetry}
\poeml \v{15}The Son\fnote{\fbackref{1:15} Lit. \fbib{He}} is the image of the invisible God, \\
\poemll    the firstborn over all creation. \\
\poeml \v{16}For by him all things in heaven and on earth were created, \\
\poemll    things visible and invisible, \\
\poemlll       whether they are kings,\fnote{\fbackref{1:16} Lit. \fbib{thrones}} lords, rulers, or powers. \\
\poeml All things have been created through him and for him. \\
\poeml \v{17}He himself existed before anything else did, \\
\poemll    and he holds all things together. \\
\poeml \v{18}He is also the head of the body, \\
\poemll    which is the church. \\
\poeml He is the beginning, the firstborn from the dead, \\
\poemll    so that he himself might have first place in everything. \\
\poeml \v{19}For God\fnote{\fbackref{1:19} Lit. \fbib{he}} was pleased to have \\
\poemll    all of his divine essence\fnote{\fbackref{1:19} Lit. \fbib{all of the fullness}} inhabit him. \\
\poeml \v{20}Through the Son,\fnote{\fbackref{1:20} Lit. \fbib{Through him}} God\fnote{\fbackref{1:20} Lit. \fbib{he}} also reconciled all things to himself, \\
\poemll    whether things on earth or things in heaven, \\
\poeml thereby making peace \\
\poemll    through the blood of his cross.
\end{poetry}

\v{21}You who were once alienated with a hostile attitude, doing evil,\fnote{\fbackref{1:21} Lit. \fbib{in evil deeds}} \v{22}he has now reconciled by the death of his physical body, so that he may present you holy, blameless, and without fault before him. \v{23}However, you must remain firmly established and steadfast in the faith, without being moved from the hope of the gospel that you heard, which has been proclaimed to every creature under heaven and of which I, Paul, have become a servant.\fnote{\fbackref{1:23} Or \fbib{minister}}
\passage{Paul's Service in the Church}

\v{24}Now I am rejoicing while suffering for you as I complete in my flesh whatever remains of the Messiah's\fnote{\fbackref{1:24} Or \fbib{Christ's}} sufferings on behalf of his body, which is the church. \v{25}I became its servant\fnote{\fbackref{1:25} Or \fbib{minister}} as God commissioned me to work for you, so that I may complete my ministry of\fnote{\fbackref{1:25} The Gk. lacks \fbib{the ministry of}} the word of God. \v{26}This secret was hidden throughout the ages and generations but has now been revealed to his saints, \v{27}to whom God wanted to make known the glorious riches of this secret among the gentiles---which is the Messiah\fnote{\fbackref{1:27} Or \fbib{Christ}} in you, our glorious hope. \v{28}It is he whom we proclaim as we admonish and wisely teach everyone, so that we may present everyone mature\fnote{\fbackref{1:28} Or \fbib{complete}} in the Messiah.\fnote{\fbackref{1:28} Or \fbib{Christ}} \v{29}I work hard and struggle to do this, using the energy that he powerfully provides in me.
\labelchapt{2}

\chapt{2}
\v{1}For I want you to know how much I struggle for you, for those in Laodicea, and for all who have never seen me face to face.\fnote{\fbackref{2:1} Lit. \fbib{my face in the flesh}} \v{2}Because they are united in love, I pray\fnote{\fbackref{2:2} The Gk. lacks \fbib{I pray}} that their hearts may be encouraged by all the riches that come from a complete understanding of the full knowledge of the Messiah,\fnote{\fbackref{2:2} Or \fbib{Christ}} who is\fnote{\fbackref{2:2} The Gk. lacks \fbib{who is}} the mystery of God. \v{3}In him are stored all the treasures of wisdom and knowledge. \v{4}I say this so that no one will mislead you with nice-sounding rhetoric. \v{5}For although I am physically absent, I am with you in spirit, rejoicing to see how stable you are and how firm your faith in the Messiah\fnote{\fbackref{2:5} Or \fbib{Christ}} is.
\passage{Fullness of Life}

\v{6}So then, just as you have received the Messiah\fnote{\fbackref{2:6} Or \fbib{Christ}} Jesus the Lord, continue to live dependent on him. \v{7}For you have been rooted in him and are being built up and strengthened in the faith, just as you were taught, while you continue to be thankful. \v{8}See to it that no one enslaves you through philosophy and empty deceit according to human tradition, according to the basic principles of the world,\fnote{\fbackref{2:8} Or \fbib{the elemental spirits of the universe}} and not according to the Messiah,\fnote{\fbackref{2:8} Or \fbib{Christ}} \v{9}because all the essence\fnote{\fbackref{2:9} Lit. \fbib{all of the fullness}} of deity inhabits him in bodily form. \v{10}And you have been filled by him, who is the head of every ruler and authority. \v{11}In union with him you were also circumcised with a circumcision performed without human\fnote{\fbackref{2:11} The Gk. lacks \fbib{human}} hands by stripping off the corrupt nature by the circumcision performed by the Messiah.\fnote{\fbackref{2:11} Or \fbib{Christ}} \v{12}When you were buried with the Messiah\fnote{\fbackref{2:12} Lit. \fbib{with him}} in baptism, you were also raised with him through faith in the power of God, who raised him from the dead. \v{13}Even when you were dead because of your offenses and the uncircumcision of your flesh, God\fnote{\fbackref{2:13} Lit. \fbib{he}} made you alive with him when he forgave us all of our offenses, \v{14}having erased the charges that were brought against us, along with their obligations that were hostile to us. He took those charges away when he nailed them to the cross. \v{15}And when he had disarmed the rulers and the authorities, he made a public spectacle of them, triumphing over them in the cross.\fnote{\fbackref{2:15} Lit. \fbib{in it}}

\v{16}Therefore, let no one judge you in matters of food and drink or with respect to a festival, a New Moon, or Sabbath days.\fnote{\fbackref{2:16} Lit. \fbib{or Sabbaths}} \v{17}These are a shadow of the things to come, but the reality\fnote{\fbackref{2:17} Or \fbib{substance}} belongs to the Messiah.\fnote{\fbackref{2:17} Or \fbib{Christ}} \v{18}Let no one who delights in humility and the worship of angels cheat you out of the prize by rejoicing about what he has seen.\fnote{\fbackref{2:18} Other mss. read \fbib{what he has not seen}} Such a person is puffed up for no reason by his carnal mind. \v{19}He does not hold on to the head, from whom the whole body, which is nourished and held together by its joints and ligaments, grows as God enables it.
\passage{The New Life in the Messiah}

\v{20}If you have died with the Messiah\fnote{\fbackref{2:20} Or \fbib{Christ}} to the basic principles of the world,\fnote{\fbackref{2:20} Or \fbib{the elemental spirits of the universe}} why are you submitting to its decrees as though you still lived in the world? \v{21}``Don't handle this! Don't taste or touch that!'' \v{22}All of these things will be destroyed as they are used, because they are based on human commands and teachings. \v{23}These things have the appearance of wisdom in promoting self-made religion, humility, and harsh treatment of the body, but they have no value against self-indulgence.
\labelchapt{3}
\passage{Keep Focusing on the Messiah}

\chapt{3}
\v{1}Therefore, if you have been raised with the Messiah,\fnote{\fbackref{3:1} Or \fbib{Christ}} keep focusing on the things that are above, where the Messiah\fnote{\fbackref{3:1} Or \fbib{Christ}} is seated at the right hand of God. \v{2}Keep your minds on things that are above, not on things that are on the earth. \v{3}For you have died, and your life has been safely guarded by the Messiah\fnote{\fbackref{3:3} Or \fbib{Christ}} in God. \v{4}When the Messiah,\fnote{\fbackref{3:4} Or \fbib{Christ}} who is\fnote{\fbackref{3:4} The Gk. lacks \fbib{who is}} your\fnote{\fbackref{3:4} Other mss. read \fbib{our}} life, is revealed, then you, too, will be revealed with him in glory.

\v{5}So put to death your worldly impulses:\fnote{\fbackref{3:5} Lit. \fbib{the parts that are on the earth}} sexual sin, impurity, passion, evil desire, and greed (which is idolatry). \v{6}It is because of these things that the wrath of God is coming on those who are disobedient.\fnote{\fbackref{3:6} Lit. \fbib{on the sons of disobedience}} \v{7}You used to behave like them as you lived among them. \v{8}But now you must also get rid of anger, wrath, malice, slander, obscene speech, and all such sins. \v{9}Do not lie to one another, for you have stripped off the old nature with its practices \v{10}and have clothed yourselves with the new nature, which is being renewed in full knowledge, consistent with the image of the one who created it. \v{11}In him\fnote{\fbackref{3:11} Lit. \fbib{it,} \fbib{\v{11}where}} there is no Greek or Jew, circumcised or uncircumcised, barbarian, Scythian,\fnote{\fbackref{3:11} I.e. uncivilized person} slave, or free person. Instead, the Messiah\fnote{\fbackref{3:11} Or \fbib{Christ}} is all and in all.

\v{12}Therefore, as God's chosen ones, holy and loved, clothe yourselves with compassion, kindness, humility, meekness,\fnote{\fbackref{3:12} Or \fbib{gentleness}} and patience. \v{13}Be tolerant of one another and forgive each other if anyone has a complaint against another. Just as the Lord\fnote{\fbackref{3:13} Other mss. read \fbib{the Messiah}} has forgiven you, you also should forgive.\fnote{\fbackref{3:13} Lit. \fbib{so you also}} \v{14}Above all, clothe yourselves with\fnote{\fbackref{3:14} The Gk. lacks \fbib{clothe yourselves with}} love, which ties everything together in unity. \v{15}Let the peace of the Messiah\fnote{\fbackref{3:15} Or \fbib{Christ}} also rule in your hearts, to which you were called in one body, and be thankful. \v{16}Let the word of the Messiah\fnote{\fbackref{3:16} Or \fbib{Christ}; other mss. read \fbib{of God}; still other mss. read \fbib{of the Lord}} inhabit you richly with wisdom, teaching and admonishing one another with psalms, hymns, and spiritual songs, and singing to God with thankfulness in your hearts. \v{17}And whatever you do, whether by speech or action, do everything in the name of the Lord Jesus, giving thanks to God the Father through him.
\passage{Family Duties}

\v{18}Wives, submit yourselves to your husbands, as is appropriate for those who belong to the Lord. \v{19}Husbands, love your wives, and do not be harsh with\fnote{\fbackref{3:19} Or \fbib{bitter toward}} them.

\v{20}Children, obey your parents in everything, for this is pleasing to the Lord. \v{21}Fathers, do not make your children resentful. Otherwise, they'll become discouraged.

\v{22}Slaves, obey your earthly masters in everything, not only while being watched in order to please them, but with a sincere heart, fearing the Lord. \v{23}Whatever you do, work at it wholeheartedly as though you were doing it\fnote{\fbackref{3:23} The Gk. \fbib{lacks though you were doing it}} for the Lord and not merely for people. \v{24}You know that it is from the Lord that you will receive the inheritance as a reward. It is the Lord Messiah\fnote{\fbackref{3:24} Or \fbib{Christ}} whom you are serving! \v{25}For the person who does what is wrong will be paid back for what he has done without favoritism.
\labelchapt{4}

\chapt{4}
\v{1}Masters, treat your slaves justly and fairly, because you know that you also have a Master in heaven.
\passage{Closing Exhortations}

\v{2}Devote yourselves to prayer. Be alert\fnote{\fbackref{4:2} Lit. \fbib{Be alert in it}} and thankful when you pray. \v{3}At the same time also pray for us---that God would open before us a door for the word so that we may tell the secret about the Messiah,\fnote{\fbackref{4:3} Or \fbib{Christ}} for which I have been imprisoned. \v{4}May I reveal it as clearly as I should!\fnote{\fbackref{4:4} Lit. \fbib{as I should speak}}

\v{5}Behave wisely toward outsiders, making the best use of your time. \v{6}Let your speech always be gracious, seasoned with salt, so that you may know how you ought to answer everyone.
\passage{Greetings from Paul and His Fellow Workers}

\v{7}Tychicus will tell you everything that has happened to me. He is a dear brother, a faithful minister, and a fellow servant in the Lord. \v{8}I am sending him to you for this very reason, so that you may know how we are doing and that he may encourage your hearts. \v{9}He is coming with Onesimus, that faithful and dear brother, who is one of you. They will tell you everything that is happening here.

\v{10}Aristarchus, my fellow prisoner, sends his greetings, as does Mark, the cousin of Barnabas. You have received instructions about him. If he comes to you, welcome him. \v{11}Jesus, who is called Justus, also greets you. These are the only ones of the circumcision who are fellow workers for the kingdom of God. They have been an encouragement to me. \v{12}Epaphras, who is one of you, a servant\fnote{\fbackref{4:12} Or \fbib{slave}} of the Messiah\fnote{\fbackref{4:12} Or \fbib{Christ}} Jesus, sends you his greetings. He is always wrestling in his prayers for you, so that you may stand mature,\fnote{\fbackref{4:12} Or \fbib{complete}} completely convinced of the entire will of God. \v{13}For I can testify on his behalf that he has a deep concern for you and for those in Laodicea and in Hierapolis. \v{14}Luke, the beloved physician, and Demas greet you. \v{15}Give my greetings to the brothers in Laodicea, especially to Nympha and the church that is in her house. \v{16}When this letter has been read among you, have it read also in the church of the Laodiceans, and be sure to read the one from Laodicea. \v{17}Tell Archippus, ``See that you complete the ministry you have received from the Lord.''
\passage{Final Greeting}

\v{18}This greeting is written with my own signature\fnote{\fbackref{4:18} Lit. \fbib{hand}}---``Paul.'' Remember that I remain imprisoned. May grace be with you! Amen.\fnote{\fbackref{4:18} Other mss. lack \fbib{Amen}}
