\bookheader{2 Corinthians}
\labelbook{2Cor}

\bookpretitle{The Letter from Paul Called}
\booktitle{Second Corinthians}

\labelchapt{1}
\passage{Paul Greets the Church in Corinth}

\chapt{1}
\v{1}From:\fnote{\fbackref{1:1} The Gk. lacks \fbib{From}} Paul, an apostle of the Messiah\fnote{\fbackref{1:1} Or \fbib{Christ}} Jesus by the will of God, and Timothy our brother.

To: God's church in Corinth, and to all the holy people\fnote{\fbackref{1:1} Or \fbib{the saints}} throughout Achaia.

\v{2}May grace and peace from God our Father and the Lord Jesus, the Messiah,\fnote{\fbackref{1:2} Or \fbib{Christ}} be yours!
\passage{The God of All Comfort}

\v{3}Blessed be the God and Father of our Lord Jesus, the Messiah!\fnote{\fbackref{1:3} Or \fbib{Christ}} He is our merciful Father and the God of all comfort, \v{4}who comforts us in all our suffering, so that we may be able to comfort others in all their suffering, as we ourselves are being comforted by God. \v{5}For as the Messiah's\fnote{\fbackref{1:5} Or \fbib{Christ's}} sufferings overflow into us, so also our comfort overflows through the Messiah.\fnote{\fbackref{1:5} Or \fbib{Christ}} \v{6}If we suffer, it is for your comfort and salvation. If we are comforted, it is for your comfort when you patiently endure the same sufferings that we are suffering. \v{7}Our hope for you is unshaken, because we know that as you share our sufferings, you also share our comfort.
\passage{How God Rescued Paul}

\v{8}For we do not want you to be ignorant, brothers, about the suffering we experienced in Asia. We were so crushed beyond our ability to endure that we even despaired of living. \v{9}In fact, we felt that we had received a death sentence so we would not rely on ourselves but on God, who raises the dead. \v{10}He has rescued us from a terrible death, and he will continue to rescue us. Yes, he is the one on whom we have set our hope, and he will rescue us again, \v{11}as you also help us by your prayers for us. Then many people will thank God\fnote{\fbackref{1:11} The Gk. lacks \fbib{God}} on our behalf because of the favor shown us through the prayers of many.
\passage{Paul's Reason for Boasting}

\v{12}For this is what we boast about: Our conscience testifies that we have conducted ourselves in the world with pure motives and godly sincerity, without earthly wisdom but with God's grace---especially toward you. \v{13}For what we are writing you is nothing more than what you can read and also understand. I hope you will understand completely, \v{14}just as you have already understood us partially, so that on the Day of our\fnote{\fbackref{1:14} Other mss. read \fbib{the}} Lord Jesus we can be your reason to boast, even as you are ours.
\passage{Why Paul's Visit Was Postponed}

\v{15}Because I was confident, I planned to come to you first so you might receive a double blessing. \v{16}I planned to leave you in order to go\fnote{\fbackref{1:16} Lit. \fbib{To go through you}} to Macedonia, and then come back to you from Macedonia, and let you send me on to Judea.

\v{17}When I planned this, I did not do it lightly, did I? Are my plans so fickle\fnote{\fbackref{1:17} Lit. \fbib{according to the flesh}} that I can say ``Yes'' and ``No''\fnote{\fbackref{1:17} Lit. ``\fbib{Yes, yes'' and ``No, no''}} at the same time? \v{18}As certainly as God is faithful, we haven't talked to you with mixed messages like that.\fnote{\fbackref{1:18} Lit. \fbib{faithful, our word to you is not ``Yes'' and ``No''}} \v{19}For God's Son, Jesus the Messiah,\fnote{\fbackref{1:19} Or \fbib{Christ}} who was preached among you by us---by me, Silvanus, and Timothy---was not ``Yes'' and ``No.'' But with him it is always ``Yes.'' \v{20}For all God's promises are ``Yes'' in him. And so through him we can say ``Amen,''\fnote{\fbackref{1:20} Lit. \fbib{through him is the ``Amen''}} to the glory of God. \v{21}Now the one who makes us---and you as well---secure in union with the Messiah\fnote{\fbackref{1:21} Or \fbib{Christ}} and has anointed us is God, \v{22}who has placed his seal on us and has given us the Spirit in our hearts as a down payment.

\v{23}I call upon God as a witness on my behalf that it was in order to spare you that I did not return to Corinth. \v{24}It is not that we are trying to rule over your faith, but rather to work with you for your joy, because you have been standing firm in the faith.
\labelchapt{2}
\passage{Paul's Painful Visit}

\chapt{2}
\v{1}Now\fnote{\fbackref{2:1} Other mss. read \fbib{For}} I decided not to pay you another painful visit. \v{2}After all, if I were to grieve you, who should make me happy but the person I am making sad? \v{3}This is the very reason I wrote you, so that when I did come I might not be made sad by those who should have made me happy. For I had confidence that all of you would share the joy that I have. \v{4}I wrote to you out of great sorrow and anguish of heart---along with many tears---not to make you sad but to let you know how much love I have for you.
\passage{Forgive the Person who Sinned}

\v{5}But if anyone has caused grief, he didn't cause me any grief. To some extent---I don't want to emphasize this too much---it has affected\fnote{\fbackref{2:5} The Gk. lacks \fbib{much---it has affected}} all of you. \v{6}This punishment by the majority is severe enough for such a man. \v{7}So forgive and comfort him, or else he will drown in his excessive grief. \v{8}That's why I'm urging you to assure him of your love. \v{9}I had also written to you to see if you would stand the test and be obedient in every way. \v{10}When you forgive someone, I do, too. Indeed, what I have forgiven---if there was anything to forgive---I did\fnote{\fbackref{2:10} The Gk. lacks \fbib{I did}} in the presence of the Messiah\fnote{\fbackref{2:10} Or \fbib{Christ}} for your benefit, \v{11}so that we may not be outsmarted by Satan. After all, we are not unaware of his intentions.
\passage{Paul's Anxiety and Relief}

\v{12}When I went to Troas on behalf of the gospel of the Messiah,\fnote{\fbackref{2:12} Or \fbib{Christ}} the Lord opened a door for me, \v{13}but my spirit could not find any relief, because I couldn't find Titus, my brother. So I said goodbye to them and went on to Macedonia.

\v{14}But thanks be to God! He always leads us triumphantly by the Messiah\fnote{\fbackref{2:14} Or \fbib{Christ}} and through us spreads everywhere the fragrance of knowing him. \v{15}To God we are the aroma of the Messiah\fnote{\fbackref{2:15} Or \fbib{Christ}} among those who are being saved and among those who are being lost. \v{16}To some people we are a deadly fragrance,\fnote{\fbackref{2:16} Lit. \fbib{a fragrance of death to death}} while to others we are a living fragrance.\fnote{\fbackref{2:16} Lit. \fbib{a fragrance of life to life}} Who is qualified for this? \v{17}At least we are not commercializing God's word like so many others. Instead, we speak with sincerity in the Messiah's\fnote{\fbackref{2:17} Or \fbib{Christ's}} name,\fnote{\fbackref{2:17} The Gk. lacks \fbib{name}} like people who are sent from God and are accountable to God.\fnote{\fbackref{2:17} Lit. \fbib{as from God and before God}}
\labelchapt{3}
\passage{Ministers of the New Covenant}

\chapt{3}
\v{1}Are we beginning to recommend ourselves again? Unlike some people, we do not need letters of recommendation to you or from you, do we? \v{2}You are our letter, written in our hearts and known and read by everyone. \v{3}You are demonstrating that you are the Messiah's\fnote{\fbackref{3:3} Or \fbib{Christ's}} letter, produced by our service, written not with ink but with the Spirit of the living God, not on tablets of stone but on tablets of human hearts.

\v{4}Such is the confidence that we have in God through the Messiah.\fnote{\fbackref{3:4} Or \fbib{Christ}} \v{5}By ourselves we are not qualified to claim that anything comes from us. Rather, our credentials come from God, \v{6}who has also qualified us to be ministers of a new covenant, which is not written but spiritual, because the written text\fnote{\fbackref{3:6} Lit. \fbib{what is written}} brings death, but the Spirit gives life.

\v{7}Now if the ministry of death that was inscribed in letters of stone came with such glory that the people of Israel could not gaze on Moses' face (because the glory was fading away from it), \v{8}will not the Spirit's ministry have even more glory? \v{9}For if the ministry of condemnation has glory, then the ministry of justification has an overwhelming glory. \v{10}In fact, that which once had glory lost its glory, because the other glory surpassed it. \v{11}For if that which fades away came\fnote{\fbackref{3:11} The Gk. lacks \fbib{came}} through glory, how much more does that which is permanent have glory?

\v{12}Therefore, since we have such a hope, we speak very boldly, \v{13}not like Moses, who kept covering his face with a veil to keep the people of Israel from gazing at the end of what was fading away. \v{14}However, their minds were hardened, for to this day the same veil is still there when they read the old covenant. Only in union with the Messiah\fnote{\fbackref{3:14} Or \fbib{Christ}} is that veil removed.\fnote{\fbackref{3:14} Lit. \fbib{is it removed}} \v{15}Yet even to this day, when Moses is read, a veil covers their hearts. \v{16}But whenever a person turns to the Lord, the veil is removed. \v{17}Now the Lord is the Spirit, and where the Lord's Spirit is, there is freedom. \v{18}As all of us reflect the glory of the Lord with unveiled faces, we are becoming more like him with ever-increasing glory by the Lord's Spirit.
\labelchapt{4}
\passage{Treasure in Clay Jars}

\chapt{4}
\v{1}Therefore, since we have this ministry through the mercy shown to us, we do not get discouraged. \v{2}Instead, we have renounced secret and shameful ways. We do not use trickery or pervert God's word. By clear statements of the truth we commend ourselves to everyone's conscience before God.

\v{3}So if our gospel is veiled, it is veiled to those who are dying.\fnote{\fbackref{4:3} Or \fbib{being destroyed}} \v{4}In their case, the god of this world has blinded the minds of those who do not believe to keep them from seeing the light of the glorious gospel of the Messiah,\fnote{\fbackref{4:4} Or \fbib{Christ}} who is the image of God.

\v{5}For we do not preach ourselves, but rather Jesus the Messiah\fnote{\fbackref{4:5} Or \fbib{Christ}} as Lord, and ourselves as merely your servants for Jesus' sake. \v{6}For God, who said, ``Let light shine out of darkness,''\fnote{\fbackref{4:6} Gen 1:3} has shone in our hearts to give us the light of the knowledge of God's glory in the face of Jesus\fnote{\fbackref{4:6} Other mss. lack \fbib{Jesus}} the Messiah.\fnote{\fbackref{4:6} Or \fbib{Christ}; other mss. read \fbib{of the Messiah Jesus}}

\v{7}But we have this treasure in clay jars to show that its extraordinary power comes from God and not from us. \v{8}In every way we're troubled but not crushed, frustrated but not in despair, \v{9}persecuted but not abandoned, struck down but not destroyed. \v{10}We are always carrying around the death of Jesus in our bodies, so that the life of Jesus may be clearly shown in our bodies. \v{11}While we are alive, we are constantly being handed over to death for Jesus' sake, so that the life of Jesus may be clearly shown in our mortal bodies. \v{12}And so death is at work in us, but life is at work\fnote{\fbackref{4:12} The Gk. lacks \fbib{is at work}} in you.

\v{13}Now since we have the same spirit of faith in keeping with this Scripture: ``I believed, and so I spoke,''\fnote{\fbackref{4:13} Ps 116:10} we also believe and therefore speak. \v{14}We know that the one who raised the Lord Jesus will also raise us with Jesus and present us to God\fnote{\fbackref{4:14} The Gk. lacks \fbib{to God}} together with you. \v{15}All this is for your sake so that, as his grace spreads, more and more people will give thanks and glorify God.
\passage{Life in an Earthly Tent}

\v{16}That's why we are not discouraged. No, even if outwardly we are wearing out, inwardly we are being renewed each and every day. \v{17}This light, temporary nature of our suffering is producing for us an everlasting weight of glory, far beyond any comparison, \v{18}because we do not look for things that can be seen but for things that cannot be seen. For things that can be seen are temporary, but things that cannot be seen are eternal.
\labelchapt{5}

\chapt{5}
\v{1}We know that if the earthly tent we live in is torn down, we have a building in heaven that comes from God, an eternal house not built by human\fnote{\fbackref{5:1} The Gk. lacks \fbib{human}} hands. \v{2}For in this one we sigh, since we long to put on our heavenly dwelling. \v{3}Of course, if we do put it on, we will not be found without a body.\fnote{\fbackref{5:3} Lit. \fbib{found naked}} \v{4}So while we are still in this tent, we sigh under our burdens, because we do not want to put it off but to put it on, so that our dying bodies may be swallowed up by life. \v{5}God has prepared us for this and has given us his Spirit as a guarantee.

\v{6}Therefore, we are always confident, and we know that as long as we are at home in this body we are away from the Lord. \v{7}For we live by faith, not by sight. \v{8}We are confident, then, and would prefer to be away from this body and to live with the Lord. \v{9}So whether we are at home or away from home, our goal is to be pleasing to him. \v{10}For all of us must appear before the judgment seat of the Messiah,\fnote{\fbackref{5:10} Or \fbib{Christ}} so that each of us may receive what he deserves for what he has done in his body, whether good or worthless.\fnote{\fbackref{5:10} Or \fbib{bad}}
\passage{The Messiah's Love Controls Us}

\v{11}Therefore, since we know what it means to fear the Lord, we try to persuade people. We ourselves are perfectly known to God. I hope we are also really known to your consciences. \v{12}We are not recommending ourselves to you again but are giving you a reason to be proud of us, so that you can answer those who are proud of outward things rather than inward character.\fnote{\fbackref{5:12} Lit. \fbib{rather than the heart}} \v{13}So if we were crazy, it was for God; if we are sane, it is for you. \v{14}The love of the Messiah\fnote{\fbackref{5:14} Or \fbib{Christ}} controls us, for we are convinced of this: that one person died for all people; therefore, all people have died. \v{15}He died for all people, so that those who live should no longer live for themselves but for the one who died and rose for them.

\v{16}So then, from now on we do not think of anyone from a human point of view.\fnote{\fbackref{5:16} Lit. \fbib{according to the flesh}} Even if we did think of the Messiah\fnote{\fbackref{5:16} Or \fbib{Christ}} from a human point of view,\fnote{\fbackref{5:16} Lit. \fbib{according to the flesh}} we don't think of him that way anymore. \v{17}Therefore, if anyone is in the Messiah,\fnote{\fbackref{5:17} Or \fbib{Christ}} he is a new creation. Old things have disappeared, and---look!---all things have become new!

\v{18}All of this comes from God, who has reconciled us to himself through the Messiah\fnote{\fbackref{5:18} Or \fbib{Christ}} and has given us the ministry of reconciliation, \v{19}for through the Messiah,\fnote{\fbackref{5:19} Or \fbib{Christ}} God was reconciling the world to himself by not counting their sins against them. He has committed his message of reconciliation to us. \v{20}Therefore, we are the Messiah's\fnote{\fbackref{5:20} Or \fbib{Christ's}} representatives, as though God were pleading through us. We plead on the Messiah's\fnote{\fbackref{5:20} Or \fbib{Christ's}} behalf: ``Be reconciled to God!'' \v{21}God\fnote{\fbackref{5:21} Lit. \fbib{He}} made the one who did not know sin to be sin for us, so that God's righteousness would be produced in us.\fnote{\fbackref{5:21} Lit. \fbib{that we might become God's righteousness in him}}
\labelchapt{6}
\passage{Workers with God}

\chapt{6}
\v{1}Since, then, we are working with God,\fnote{\fbackref{6:1} Lit. \fbib{working together}} we plead with you not to accept God's grace in vain. \v{2}For he says,

\begin{poetry}
\poeml ``At the right time I heard you, \\
\poeml and on a day of salvation I helped you.''\fnote{\fbackref{6:2} Isa 49:8}
\end{poetry}

Listen, now is really the ``right time''! Now is the ``day of salvation''!
\passage{We are God's Servants}

\v{3}We do not put an obstacle in anyone's way. Otherwise, fault may be found with our ministry. \v{4}Instead, in every way we demonstrate that we are God's servants by tremendous endurance in the midst of difficulties, hardships, and calamities; \v{5}in beatings, imprisonments, and riots; in hard work, sleepless nights, and hunger; \v{6}with purity, knowledge, patience, and kindness; with the Holy Spirit, genuine love, \v{7}truthful speech, and divine power; through the weapons of righteousness in the right and left hands; \v{8}through honor and dishonor; through ill repute and good repute; perceived\fnote{\fbackref{6:8} The Gk. lacks \fbib{perceived}} as deceivers and yet true, \v{9}as unknown and yet well-known, as dying and yet---as you see---very much alive, as punished and yet not killed, \v{10}as sorrowful and yet always rejoicing, as poor and yet enriching many, as having nothing and yet possessing everything.

\v{11}We have spoken frankly\fnote{\fbackref{6:11} Lit. \fbib{Our mouth is open}} to you, Corinthians. Our hearts are wide open. \v{12}We have not cut you off, but you have cut off your own feelings toward us. \v{13}Do us a favor---I ask you as my children---and open wide your hearts.
\passage{Relating with Unbelievers}

\v{14}Stop becoming\fnote{\fbackref{6:14} Or \fbib{Don't become}} unevenly yoked with unbelievers. What partnership can righteousness have with lawlessness? What fellowship can light have with darkness? \v{15}What harmony exists between the Messiah\fnote{\fbackref{6:15} Or \fbib{Christ}} and Beliar,\fnote{\fbackref{6:15} I.e. the devil} or what do a believer and an unbeliever have in common? \v{16}What agreement can a temple of God make with idols? For we\fnote{\fbackref{6:16} Other mss. read \fbib{you} (pl.)} are the temple of the living God, just as God said:

\begin{poetry}
\poeml ``I will live and walk among them. \\
\poemll    I will be their God, \\
\poemlll       and they will be my people.''\fnote{\fbackref{6:16} Lev 26:12; Ezek 37:27}
\end{poetry}

\v{17}Therefore,

\begin{poetry}
\poeml ``Get away from them \\
\poemll    and separate yourselves from them,'' \\
\poemlll       declares the Lord,\fnote{\fbackref{6:17} MT source citation reads \fbib{}\divine{Lord}} \\
\poeml ``and don't touch anything unclean. \\
\poemll    Then I will welcome you. \\
\poeml \v{18}I will be your Father, \\
\poemll    and you will be my sons and daughters,'' \\
\poemlll       declares the Lord\fnote{\fbackref{6:18} MT source citation reads \fbib{}\divine{Lord}} Almighty.\fnote{\fbackref{6:18} Isa 52:11; Ezek 20:34, 41; 2 Sam 7:8, 14}
\end{poetry}
\labelchapt{7}
\passage{Cleanse Yourselves in Holiness}

\chapt{7}
\v{1}Since we have these promises, dear friends, let's cleanse ourselves from everything that contaminates body and spirit by becoming mature in our holy fear of God.
\passage{Encouraged by the Corinthians}

\v{2}Make room for us in your hearts!\fnote{\fbackref{7:2} The Gk. lacks \fbib{in your hearts}} We have not treated anyone unjustly, harmed anyone, or cheated anyone. \v{3}I am not saying this to condemn you. I told you before that you are in our hearts to die together and to live together. \v{4}I have great confidence in you. I am very proud of you. I am very much encouraged. I am overjoyed in all our troubles.

\v{5}For even when we came to Macedonia, our bodies had no rest. We suffered in a number of ways. Outwardly there were conflicts, inwardly there were fears. \v{6}Yet God, who comforts those who feel miserable, comforted us by the arrival of Titus, \v{7}and not only by his arrival but also by the comfort he had received from you. He told us about your longing for me, your sorrow, and your eagerness to take my side, and this made me even happier.

\v{8}If I made you sad with my letter, I do not regret it, although I did regret it then. I see that the letter caused you sorrow, though only for a while. \v{9}Now I am happy, not because you had such sorrow, but because your sorrow led you to repent. For you were sorry in a godly way, and so you were not hurt by us in any way. \v{10}For having sorrow in a godly way results in repentance that leads to salvation and leaves no regrets. But the sorrow of the world produces death.

\v{11}See what great earnestness godly sorrow has produced in you! How ready you are to clear yourselves, how indignant, how alarmed, how full of longing and enthusiasm, how eager to seek justice! In every way you have demonstrated that you are innocent in this matter. \v{12}So, even though I wrote to you, it wasn't because of the man who did the wrong or because of the man who was hurt. Instead, I wrote to you so that your devotion to us might be made perfectly clear to you before God.

\v{13}This is what comforted us. In addition to our own comfort, we were even more delighted at the joy of Titus, because his spirit had been set at rest by all of you. \v{14}For if I have been doing some boasting about you to him, I have never been ashamed of it. Moreover, since everything we told you was true, our boasting to Titus has also proved to be true. \v{15}His affection for you is even greater as he remembers how obedient all of you were and how you welcomed him with fear and trembling. \v{16}I rejoice that I can have complete confidence in you.
\labelchapt{8}
\passage{The Collection for the Christians in Jerusalem}

\chapt{8}
\v{1}We want you to know, brothers, about God's grace that was given to the churches of Macedonia. \v{2}In spite of their terrible ordeal of suffering, their abundant joy and deep poverty have led them to be abundantly generous. \v{3}I can testify that by their own free will they have given to the utmost of their ability, yes, even beyond their ability. \v{4}They begged us earnestly for the privilege\fnote{\fbackref{8:4} Or \fbib{for the grace}} of participating in this ministry to the saints. \v{5}We did not expect that! They gave themselves to the Lord first and then to us, since this was God's will. \v{6}So we urged Titus to finish this work of kindness\fnote{\fbackref{8:6} Or \fbib{this grace}} among you in the same way that he had started it. \v{7}Indeed, the more your faith, speech, knowledge, enthusiasm, and love for us increase, the more we want you to be rich in this work of kindness.\fnote{\fbackref{8:7} Or \fbib{this grace}}

\v{8}I am not commanding you but testing the genuineness of your love by the enthusiasm of others. \v{9}For you know the grace of our Lord Jesus, the Messiah.\fnote{\fbackref{8:9} Or \fbib{Christ}} Although he was rich, for your sakes he became poor, so that you, through his poverty, might become rich.

\v{10}I am giving you my opinion on this matter because it will be helpful to you. Last year you were not only willing to do something, but had already started to do it. \v{11}Now finish what you began, so that your eagerness to do so may be matched by your eagerness\fnote{\fbackref{8:11} Lit. \fbib{matched as you contribute from what you have}} to complete it. \v{12}For if the eagerness is there, the gift\fnote{\fbackref{8:12} Lit. \fbib{it}} is acceptable according to what you have, not according to what you do not have.

\v{13}Not that others should have relief while you have hardship. Rather, it is a question of fairness. \v{14}At the present time, your surplus fills their need, so that their surplus may fill your need. In this way things are fair. \v{15}As it is written,

\begin{poetry}
\poeml ``The person who had much did not have too much, \\
\poemll    and the person who had little did not have too little.''\fnote{\fbackref{8:15} Exod 16:18}
\end{poetry}
\passage{Titus and His Companions}

\v{16}But thanks be to God, who placed in the heart of Titus the same dedication to you that I have. \v{17}He welcomed my request and eagerly went to visit you by his own free will. \v{18}With him we have sent the brother who is praised in all the churches for spreading the gospel.\fnote{\fbackref{8:18} Lit. \fbib{in the gospel}} \v{19}More than that, he has also been selected by the churches to travel with us while we are administering this work of kindness\fnote{\fbackref{8:19} Or \fbib{this grace}} for the glory of the Lord and as evidence of our eagerness to help.\fnote{\fbackref{8:19} The Gk. lacks \fbib{to help}} \v{20}We are trying to avoid any criticism of the way we are administering this great undertaking. \v{21}We intend to do what is right, not only in the sight of the Lord, but also in the sight of people.

\v{22}We have also sent with them our brother whom we have often tested in many ways and found to be dedicated. At present he is more dedicated than ever because he has so much confidence in you.

\v{23}As for Titus, he is my partner and fellow worker on your behalf. Our brothers, emissaries\fnote{\fbackref{8:23} Or \fbib{apostles}} from the churches, are bringing glory to the Messiah.\fnote{\fbackref{8:23} Or \fbib{Christ}} \v{24}Therefore, demonstrate to the churches that you love them and show them publicly why we boast about you.
\labelchapt{9}
\passage{Why Giving is Important}

\chapt{9}
\v{1}I do not need to write to you any further about the ministry to the saints. \v{2}For I know how willing you are, and I boast about you to the people of Macedonia, saying\fnote{\fbackref{9:2} The Gk. lacks \fbib{saying}} that Achaia has been ready since last year, and your enthusiasm has stimulated most of them. \v{3}Now I have sent the brothers so that our boasting about you in this matter may not prove to be an idle boast, and so that you may stand ready, just as I said. \v{4}Otherwise, if any Macedonians come with me and find out that you are not ready, we would be humiliated---to say nothing of you---in this undertaking. \v{5}Therefore, I thought it necessary to urge these brothers to visit you ahead of me, to make arrangements in advance for this gift you promised, and to have it ready as something given generously and not forced.

\v{6}Remember\fnote{\fbackref{9:6} Lit. \fbib{Now}} this: The person who sows sparingly will also reap sparingly, and the person who sows generously will also reap generously. \v{7}Each of you must give what you have decided in your heart, not with regret or under compulsion, since God loves a cheerful giver. \v{8}Besides, God is able to make every blessing of yours overflow for you, so that in every situation you will always have all you need for any good work. \v{9}As it is written,

\begin{poetry}
\poeml ``He scatters everywhere and gives to the poor; \\
\poemll    his righteousness lasts forever.''\fnote{\fbackref{9:9} Ps 112:9}
\end{poetry}

\v{10}Now he who supplies seed to the farmer and bread to eat will also supply you with seed and multiply it and enlarge the harvest that results from your righteousness. \v{11}In every way you will grow richer and become even more generous, and this will cause others to give thanks to God because of us, \v{12}since this ministry you render is not only fully supplying the needs of the saints, it is also overflowing with more and more prayers of thanksgiving to God. \v{13}Because your service in giving proves your love,\fnote{\fbackref{9:13} The Gk. lacks \fbib{your love}} you will be glorifying God as you obey what your confession of the Messiah's\fnote{\fbackref{9:13} Or \fbib{Christ's}} gospel demands,\fnote{\fbackref{9:13} The Gk. lacks \fbib{demands}} since you are generous in sharing with them and with everyone else. \v{14}And so in their prayers for you they will long for you because of God's exceptional grace that was shown to you. \v{15}Thanks be to God for his indescribable gift!
\labelchapt{10}
\passage{Paul's Authority to Speak Forcefully}

\chapt{10}
\v{1}Now I myself, Paul, plead with you with the gentleness and kindness of the Messiah\fnote{\fbackref{10:1} Or \fbib{Christ}}---I who am humble when I am face to face with you but forceful toward you when I am away! \v{2}I beg you that when I come I will not need to be courageous by daring to oppose some people who think that we are living according to the flesh. \v{3}Of course, we are living in the world,\fnote{\fbackref{10:3} Lit. \fbib{flesh}} but we do not wage war in a world-like\fnote{\fbackref{10:3} Lit. \fbib{fleshly}} way. \v{4}For the weapons of our warfare are not those of the world.\fnote{\fbackref{10:4} Lit. \fbib{flesh}} Instead, they have the power of God to demolish fortresses. We tear down arguments \v{5}and every proud obstacle that is raised against the knowledge of God, taking every thought captive in order to obey the Messiah.\fnote{\fbackref{10:5} Or \fbib{Christ}} \v{6}Once your obedience is complete, we will be ready to reprimand every type of disobedience.

\v{7}Look at the plain facts! If anyone is confident that he belongs to the Messiah,\fnote{\fbackref{10:7} Or \fbib{Christ}} he should remind himself of this: Just as he belongs to the Messiah,\fnote{\fbackref{10:7} Or \fbib{Christ}} so do we. \v{8}So if I boast a little too much about our authority, which the Lord gave us to build you up and not to tear you down, I will not be ashamed of it.

\v{9}I do not want you to think that I am trying to frighten you with my letters. \v{10}For someone is saying,\fnote{\fbackref{10:10} The Gk. lacks \fbib{For someone is saying}} ``His letters are impressive and forceful, but his bodily presence is weak and his speech contemptible.'' \v{11}Someone like this should take note of the following: What we say by letter when we are absent is what we will do when present!
\passage{Paul's Reason for Boasting}

\v{12}We would not dare put ourselves in the same class with, or compare ourselves to, those who recommend themselves. Whenever they measure themselves by their own standards or compare themselves among themselves, they show how foolish they are. \v{13}We will not boast about what cannot be evaluated. Instead, we will stay within the field that God assigned us, so as to reach even you. \v{14}For it is not as though we were overstepping our limits when we came to you. We were the first to reach you with the gospel of the Messiah.\fnote{\fbackref{10:14} Or \fbib{Christ}} \v{15}We are not boasting about work done by others that cannot be evaluated. On the contrary, we cherish the hope that your faith may continue to grow and enlarge our sphere of action among you until it overflows. \v{16}Then we can preach the gospel in the regions far beyond you without boasting about things already accomplished by someone else.

\v{17}``The person who boasts should boast in the Lord.''\fnote{\fbackref{10:17} Jer 9:24; MT source citation reads \fbib{}\divine{Lord}} \v{18}It is not the person who commends himself who is approved, but the person whom the Lord commends.
\labelchapt{11}
\passage{Paul Contrasts Himself with False Apostles}

\chapt{11}
\v{1}I wish you would tolerate a little of my foolishness. Yes, please tolerate me! \v{2}I am jealous of you with God's own jealousy, because I promised you in marriage to one husband, to present you as a pure virgin to the Messiah.\fnote{\fbackref{11:2} Or \fbib{Christ}} \v{3}However, I am afraid that just as the serpent deceived Eve by its tricks, so your minds may somehow be lured away from sincere and pure\fnote{\fbackref{11:3} Other mss. lack \fbib{and pure}} devotion to the Messiah.\fnote{\fbackref{11:3} Or \fbib{Christ}}

\v{4}For if someone comes along and preaches another Jesus than the one we preached, or should you receive a different spirit from the one you received or a different gospel from the one you accepted, you are all too willing to listen. \v{5}I do not think I'm inferior in any way to those ``super-apostles.'' \v{6}Even though I may be untrained as an orator, I am not so in the field of knowledge. We have made this clear to all of you in every possible way.

\v{7}Did I commit a sin when I humbled myself by proclaiming to you the gospel of God free of charge, so that you could be exalted? \v{8}I robbed other churches by accepting support from them in order to serve you. \v{9}When I was with you and needed something, I did not bother any of you, because our brothers who came from Macedonia supplied everything I needed. I kept myself from being a burden to you in any way, and I will continue to do so.

\v{10}As surely as the truth of the Messiah\fnote{\fbackref{11:10} Or \fbib{Christ}} is in me, my boasting will not be silenced in the regions of Achaia. \v{11}Why? Because I do not love you? God knows that I do!

\v{12}But I will go on doing what I'm doing in order to deny an opportunity to those people who want an opportunity to be recognized as our equals in the work they are boasting about. \v{13}Such people are false apostles, dishonest workers who are masquerading as apostles of the Messiah.\fnote{\fbackref{11:13} Or \fbib{Christ}} \v{14}And no wonder, since Satan himself masquerades as an angel of light. \v{15}So it is not surprising if his servants also masquerade as servants of righteousness. Their doom\fnote{\fbackref{11:15} Lit. \fbib{end}} will match their deeds!
\passage{Paul's Sufferings as an Apostle}

\v{16}I will say it again: No one should think that I am a fool. But if you do, then treat me like a fool so that I can also boast a little. \v{17}When I talk as a confident boaster, I am not talking with the Lord's authority but like a fool. \v{18}Since many people boast in a fleshly way, I will do it, too. \v{19}You are wise, so you will gladly be tolerant of fools. \v{20}You tolerate anyone who makes you his slaves, devours what you have, takes what is yours, orders you around, or slaps your face!

\v{21}I am ashamed to admit it, but we have been too weak for that. Whatever anyone else dares to claim---I am talking like a fool---I can claim it, too. \v{22}Are they Hebrews? So am I. Are they Israelis? So am I. Are they among Abraham's descendants? So am I. \v{23}Are they the Messiah's\fnote{\fbackref{11:23} Or \fbib{Christ's}} servants? I am insane to talk like this, but I am a far better one! I have been involved in far greater efforts, far more imprisonments, countless beatings, and have faced death more than once. \v{24}Five times I received from the Jews 40 lashes minus one. \v{25}Three times I was beaten with a stick, once I was pelted with stones, three times I was shipwrecked, and I drifted on the sea for a day and a night. \v{26}I have traveled extensively and have been endangered from rivers, robbers, my own people, and gentiles. I've also been in danger in the city, in the open country, at sea, from false brothers, \v{27}in toil and hardship, through many a sleepless night, through hunger, thirst, many periods of fasting, coldness, and nakedness. \v{28}Besides everything else, I have a daily burden because of my anxiety about all the churches. \v{29}Who is weak without me being weak, too? Who is caused to stumble without me becoming indignant?

\v{30}If I must boast, I will boast about the things that show how weak I am. \v{31}The God and Father of the Lord Jesus, who is blessed forever, knows that I am not lying. \v{32}In Damascus, the governor under King Aretas put guards around the city of Damascus to catch me, \v{33}but I was let down in a basket through an opening in the wall and escaped from him.
\labelchapt{12}
\passage{Paul's Thorn}

\chapt{12}
\v{1}I must boast, although it does not do any good. Let's talk about visions and revelations from the Lord. \v{2}I know a man who belongs to the Messiah.\fnote{\fbackref{12:2} Or \fbib{Christ}} Fourteen years ago---whether in his body or outside of his body, I do not know, but God knows---that man was snatched away to the third heaven. \v{3}I know that this man---whether in his body or outside of his body, I do not know, but God knows--- \v{4}was snatched away to Paradise and heard things that cannot be expressed in words, things that no human being has a right even to mention.

\v{5}I will boast about this man, but as for myself I will boast only about my weaknesses. \v{6}However, if I did want to boast, I would not be a fool, because I would be telling the truth. But I am not going to do it in order to keep anyone from thinking more of me than what he sees and hears about me.

\v{7}To keep me from becoming conceited because of the exceptional nature of these revelations, a thorn\fnote{\fbackref{12:7} Or \fbib{stake}} was given to me and placed in my body.\fnote{\fbackref{12:7} Lit. \fbib{was given to me in the flesh}} It was Satan's messenger to keep on tormenting me so that I would not become conceited.

\v{8}I pleaded with the Lord three times to take it away from me, \v{9}but he has told me, \red{``My grace is all you need, because my power is perfected in weakness.''} Therefore, I will most happily boast about my weaknesses, so that the Messiah's\fnote{\fbackref{12:9} Or \fbib{Christ's}} power may rest on me. \v{10}That is why I take such pleasure in weaknesses, insults, hardships, persecutions, and difficulties for the Messiah's\fnote{\fbackref{12:10} Or \fbib{Christ's}} sake, for when I am weak, then I am strong.
\passage{Concern for the Corinthians}

\v{11}I have become a fool. You forced me to be one. Really, I should have been commended by you, for I am not in any way inferior to your ``super-apostles,'' even if I am nothing. \v{12}The signs of an apostle were performed among you with utmost patience---signs, wonders, and powerful actions. \v{13}How were you treated worse than the other churches, except that I did not bother you for help? Forgive me for this wrong! \v{14}Now I'm ready to visit you for a third time, and I will not bother you for help. I do not want your things, but rather you yourselves. Children should not have to support\fnote{\fbackref{12:14} Lit. \fbib{to save up for}} their parents, but parents their children. \v{15}I will be very glad to spend my money and myself for you. Do you love me less because I love you so much?

\v{16}Granting that I have not been a burden to you, was I a clever schemer who trapped you by some trick? \v{17}I did not take advantage of you through any of the men I sent you, did I? \v{18}I encouraged Titus to visit you, and I sent along with him the brother you know so well. Titus didn't take advantage of you, did he? We conducted ourselves with the same spirit, didn't we? We took the very same steps, didn't we?

\v{19}Have you been thinking all along that we are trying to defend ourselves before you? We are speaking before God in the authority of\fnote{\fbackref{12:19} The Gk. lacks \fbib{the authority of}} the Messiah,\fnote{\fbackref{12:19} Or \fbib{Christ}} and everything, dear friends, is meant to build you up. \v{20}I am afraid that I may come and somehow find you not as I want to find you, and that you may find me not as you want to find me. Perhaps there will be quarreling, jealousy, anger, selfishness, slander, gossip, arrogance, and disorderly conduct. \v{21}I am afraid that when I come my God may again humble me before you and that I may have to grieve over many who formerly lived in sin and have not repented of their impurity, sexual immorality, and promiscuity that they once practiced.
\labelchapt{13}
\passage{Final Warnings}

\chapt{13}
\v{1}This will be the third time I am coming to you. ``Every accusation must be verified by two or three witnesses.''\fnote{\fbackref{13:1} Deut 19:15} \v{2}I have already warned those who sinned previously and all the rest. Although I am absent now, I am warning them as I did on my second visit: If I come back, I will not spare you, \v{3}since you want proof that the Messiah\fnote{\fbackref{13:3} Or \fbib{Christ}} is speaking through me. He is not weak in dealing with you but is making his power felt among you. \v{4}Though he was crucified in weakness, he lives by God's power. We are weak with him, but by God's power we will live for you.

\v{5}Keep examining yourselves to see whether you are continuing in the faith. Test yourselves! You know, don't you, that Jesus the Messiah\fnote{\fbackref{13:5} Or \fbib{Christ}} lives in you? Could it be that you are failing the test? \v{6}I hope you will realize that we haven't failed our test. \v{7}We pray to God that you will not do anything wrong---not to show that we have not failed the test, but so that you may do what is right, even if we seem to have failed. \v{8}For we cannot do anything against the truth, but only for the truth. \v{9}We are glad when we are weak and you are strong. That is what we are praying for---your maturity.

\v{10}For this reason I am writing this while I am away from you: When I come I do not want to be severe in using the authority the Lord gave me to build you up and not to tear you down.
\passage{Final Greetings and Benediction}

\v{11}Finally, brothers, goodbye. Keep on growing to maturity. Keep listening to my appeals. Continue agreeing with each other and living in peace. Then the God of love and peace will be with you. \v{12}Greet one another with a holy kiss.\fnote{\fbackref{13:12} People customarily greeted their friends with a kiss.} \v{13}All the saints greet you.

\v{14}May the grace of the Lord Jesus the Messiah,\fnote{\fbackref{13:14} Or \fbib{Christ}} the love of God, and the fellowship of the Holy Spirit be with all of you!
