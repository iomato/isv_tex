\bookheader{Jeremiah}
\labelbook{Jer}

\bookpretitle{The Book of the Prophet}
\booktitle{Jeremiah}

\labelchapt{1}
\passage{Introduction}

\chapt{1}
\v{1}The words of Hilkiah's son Jeremiah,\fnote{\fbackref{1:1} The Heb. name \fbib{Jeremiah} means \fbib{God has appointed me}} who was one of the priests at Anathoth in the territory of Benjamin. \v{2}This message from the \divine{Lord} came to him during the thirteenth year of the reign of\fnote{\fbackref{1:2} Lit. \fbib{him in the days of}} Ammon's son Josiah, the king of Judah, \v{3}and during the reign of\fnote{\fbackref{1:3} Lit. \fbib{and in the days of}} Josiah's son Jehoiakim, the king of Judah, and continued until the exile of Jerusalem in the fifth month, at\fnote{\fbackref{1:3} Lit. \fbib{until}} the end of the eleventh year of the reign of Josiah's son Zedekiah, the king of Judah.
\passage{Jeremiah's Call as a Prophet}

\v{4}This message from the \divine{Lord} came to me:

\begin{poetry}
\poeml \v{5}``I knew you before I formed you in the womb; \\
\poemll    I set you apart for me before you were born; \\
\poemlll       I appointed you to be a prophet to the nations.''
\end{poetry}

\v{6}I replied, ``Ah, \divine{Lord} God! Look, I don't know how to speak, because I'm only\fnote{\fbackref{1:6} The Heb. lacks \fbib{only}} a young man.''

\v{7}Then the \divine{Lord} told me, ``Don't say, `I'm only\fnote{\fbackref{1:7} The Heb. lacks \fbib{only}} a young man,' for you will go everywhere I send you, and you will speak everything I command you. \v{8}Don't be afraid of them, because I am with you to deliver you,'' declares the \divine{Lord}.

\v{9}The \divine{Lord} stretched out his hand, touched my mouth, and then told me, ``Look, I've put my words in your mouth. \v{10}See, today I've appointed you to prophesy about nations and kingdoms, to pull up and tear down, to destroy and overthrow, to build and to plant.''
\passage{The Visions of the Almond Branch and Boiling Pot}

\v{11}This message from the \divine{Lord} came to me, asking, ``What do you see, Jeremiah?''

I replied, ``I see an almond branch.''

\v{12}The \divine{Lord} told me, ``You have observed well, because I'm watching over\fnote{\fbackref{1:12} The Heb. word \fbib{almond} (\fbib{shaqed}) sounds like the Heb. word \fbib{watching} (\fbib{shoqed})} my message, to make sure it comes about.''

\v{13}This message from the \divine{Lord} came to me a second time: ``What do you see?''

I replied, ``I see a boiling pot, and its mouth is tilted away from the north.''

\v{14}Then the \divine{Lord} told me, ``From the north disaster will pour out on all who live in the land, \v{15}because I'm about to summon all the families and kingdoms from the north,'' declares the \divine{Lord}.

\begin{poetry}
\poeml ``They'll come and each one will set up his seat\fnote{\fbackref{1:15} Or \fbib{throne}} \\
\poemll    at the entrance of the gates of Jerusalem, \\
\poeml against all of its surrounding walls, \\
\poemll    and against all of the towns of Judah. \\
\poeml \v{16}``I'll pronounce my judgments against them \\
\poemll    because of all their wickedness. \\
\poeml They have forsaken me, \\
\poemll    they have burned incense to other gods, \\
\poeml and they have bowed down in worship \\
\poemll    to the works of their own\fnote{\fbackref{1:16} The Heb. lacks \fbib{own}} hands.''\fnote{\fbackref{1:16} I.e. idols}
\end{poetry}
\passage{The \divine{Lord}'s Assurance to Jeremiah}

\v{17}``As for you, get ready!\fnote{\fbackref{1:17} Lit. \fbib{gird up your loins}} Stand up and tell them everything that I've commanded you. Don't be frightened as you face them, or I'll frighten you right in front of them.

\v{18}``As for me, today I'm making you a fortified city, an iron pillar, and a bronze wall against the whole land---against the kings of Judah, against its princes, against its priests, and against the people of the land. \v{19}They'll fight against you, but they won't prevail against you, because I am with you,'' declares the \divine{Lord}, ``to deliver you.''
\labelchapt{2}
\passage{Israel's Initial Fidelity}

\chapt{2}
\v{1}This message from the \divine{Lord} came to me:

\begin{poetry}
\poeml \v{2}``Go and announce to Jerusalem:
\end{poetry}

\begin{poetry}
\poeml `This is what the \divine{Lord} says:
\end{poetry}

\begin{poetry}
\poeml ``I remember the loyal devotion of your youth, \\
\poemll    your love as a bride. \\
\poeml You followed me in the desert, \\
\poemll    in a land that was not planted. \\
\poeml \v{3}Israel was consecrated\fnote{\fbackref{2:3} Or \fbib{set apart}} to the \divine{Lord}, \\
\poemll    she was the first fruits\fnote{\fbackref{2:3} I.e. the first and best} of his produce. \\
\poeml All who devoured her became guilty \\
\poemll    and disaster came on them,'' \\
\poemlll       declares the \divine{Lord}.'\,''
\passage{Her Rejection of God's Love}
\poeml \v{4}Listen to this message from the \divine{Lord}, \\
\poemll    you descendants of Jacob \\
\poemlll       and all the families of the descendants of Israel. \\
\poeml \v{5}This is what the \divine{Lord} says:
\end{poetry}

\begin{poetry}
\poeml ``What did your ancestors find wrong with me \\
\poemll    that they left me, \\
\poeml and pursued worthless things,\fnote{\fbackref{2:5} I.e. idols or false gods} \\
\poemll    and so they became worthless? \\
\poeml \v{6}``They didn't ask, `Where is the \divine{Lord} \\
\poemll    who brought us up from the land of Egypt, \\
\poeml who led us through the wilderness, \\
\poemll    through the land of desert and pits, \\
\poeml through the land of dryness and deep darkness, \\
\poemll    a land that people don't pass through, \\
\poemlll       and where no one lives?' \\
\poeml \v{7}``I brought you into the fruitful land to eat its fruit \\
\poemll    and its good things. \\
\poeml But you came in, defiled my land, \\
\poemll    and made my inheritance into an abomination. \\
\poeml \v{8}``The priests didn't say, `Where is the \divine{Lord}?' \\
\poemll    and those handling the Law didn't know me. \\
\poeml The rulers transgressed against me, \\
\poemll    the prophets prophesied by Baal, \\
\poemlll       and they followed that which does not profit.\fnote{\fbackref{2:8} I.e. idols or false gods} \\
\poeml \v{9}``Therefore I'll again accuse you,'' \\
\poemll    declares the \divine{Lord}, \\
\poeml ``and I'll accuse your grandchildren.'' \\
\poeml \v{10}``Indeed, go over to the coasts of Cyprus and see, \\
\poemll    send to Kedar and pay very close attention. \\
\poemlll       See if there has ever been such a thing as this! \\
\poeml \v{11}Has a nation ever changed gods \\
\poemll    when they aren't even gods? \\
\poeml But my people have exchanged their glory \\
\poemll    for that which does not profit. \\
\poeml \v{12}Heavens, be appalled at this, \\
\poemll    be shocked, be utterly\fnote{\fbackref{2:12} Lit. \fbib{very}} devastated,'' \\
\poemlll       declares the \divine{Lord}. \\
\poeml \v{13}``Indeed, my people have committed two evils: \\
\poemll    they have forsaken me, the fountain of living water,\fnote{\fbackref{2:13} I.e. fresh water} \\
\poeml and they have dug cisterns for themselves, \\
\poemll    broken cisterns that cannot hold water.''
\passage{Consequences of Israel's Unfaithfulness}
\poeml \v{14}``Is Israel a slave, or was he born a servant?\fnote{\fbackref{2:14} Lit. \fbib{was he a home born servant} (cf. Exod 21:4)} \\
\poemll    Why then has he become plunder? \\
\poeml \v{15}Young lions roar at him, they cry out loudly. \\
\poemll    They have made his land into a wasteland, \\
\poeml and his cities are destroyed \\
\poemll    so they are without inhabitants. \\
\poeml \v{16}Also, people from Memphis and Tahpanhes\fnote{\fbackref{2:16} I.e. Egyptian cities} \\
\poemll    have broken\fnote{\fbackref{2:16} Or \fbib{shaved}} your skull. \\
\poeml \v{17}You have done this to yourselves, have you not, \\
\poemll    by forsaking the \divine{Lord} your God, when he \\
\poemlll       is the one who led you on the way? \\
\poeml \v{18}Now, what are you doing on the road to Egypt, \\
\poemll    to drink the waters of the Nile? \\
\poeml And what are you doing on the road to Assyria, \\
\poemll    to drink the waters of the Euphrates? \\
\poeml \v{19}Your wickedness will be punished, \\
\poemll    and you will be corrected due to your acts of apostasy. \\
\poeml Know and see that it's evil and bitter for you \\
\poemll    to forsake the \divine{Lord} your God, \\
\poeml but the fear of me is not in you,'' \\
\poemll    declares the Lord \divine{God} of the Heavenly Armies. \\
\poeml \v{20}``For long ago I broke your yoke \\
\poemll    and tore off your bonds, \\
\poeml But you said, `I won't serve you!' \\
\poemll    Instead, on every high hill \\
\poeml and under every green tree, \\
\poemll    you bend down to commit fornication. \\
\poeml \v{21}I planted you myself as a choice vine, \\
\poemll    from the very best seed.\fnote{\fbackref{2:21} Lit. \fbib{faithful seed}} \\
\poeml How did you turn against me \\
\poemll    into a degenerate and foreign vine? \\
\poeml \v{22}Though you wash yourself with lye \\
\poemll    and use much soap, \\
\poeml the stain of your guilt is still before me,'' \\
\poemlll       declares the Lord \divine{God}.
\passage{Israel's Passion for Sin}
\poeml \v{23}``How can you say, `I'm not defiled. \\
\poemll    I haven't gone after the Baals.'?\fnote{\fbackref{2:23} I.e. images of the Canaanite storm god} \\
\poeml Look at what you've done\fnote{\fbackref{2:23} Lit. \fbib{at your way}} in the valley. \\
\poemll    Know what you have done. \\
\poemlll       You are a swift young camel galloping aimlessly; \\
\poeml \v{24}a wild donkey accustomed to the desert, \\
\poemll    sniffing the wind in her passion. \\
\poeml When she's in heat, \\
\poemll    who can turn her away? \\
\poeml None of the males who pursue her need to tire themselves out, \\
\poemll    for in her month\fnote{\fbackref{2:24} I.e. at mating time} they'll find her.'' \\
\poeml \v{25}``Don't run until your feet are bare \\
\poemll    and your throat is dry.\fnote{\fbackref{2:25} Lit. \fbib{Hold back your feet from being bare and your throat from being dry}} \\
\poeml But you say, `It's hopeless! \\
\poemll    Because I love foreign gods,\fnote{\fbackref{2:25} Or \fbib{foreigners}} I'll go after them!'\,'' \\
\poeml \v{26}``As a thief is disgraced when he's caught, \\
\poemll    so the house of Israel is disgraced--- \\
\poemlll       they, their kings, their princes, their priests, and their prophets, \\
\poeml \v{27}who say to a tree, `You are my father,' \\
\poemll    and to a stone, `You gave birth to me.' \\
\poeml They have turned their back to me, \\
\poemll    but not their faces. \\
\poeml In the time of their trouble, they'll say, \\
\poemll    `Rise up! Deliver us!'\,'' \\
\poeml \v{28}``But where are your gods \\
\poemll    that you made for yourselves? \\
\poeml Let them rise up, if they can deliver you \\
\poemll    in the time of your trouble. \\
\poemlll       You have as many gods as you have towns, Judah. \\
\poeml \v{29}Why do you contend with me? \\
\poemll    You have rebelled against me,'' \\
\poemlll       declares the \divine{Lord}. \\
\poeml \v{30}``I've punished your children with no results,\fnote{\fbackref{2:30} Lit. \fbib{in vain}} \\
\poemll    they have accepted no discipline. \\
\poeml Your sword has devoured your prophets \\
\poemll    like a destroying lion.'' \\
\poeml \v{31}``You, generation, \\
\poemll    pay attention to\fnote{\fbackref{2:31} Or \fbib{see}} this message from the \divine{Lord}! \\
\poeml Am I the desert to Israel, \\
\poemll    or a land of gloom? \\
\poeml Why do my people say, `We're free to roam? \\
\poemll    We won't come to you anymore.' \\
\poeml \v{32}Will a young woman forget her wedding\fnote{\fbackref{2:32} The Heb. lacks \fbib{wedding}} ornaments, \\
\poemll    or a bride her attire? \\
\poeml But my people have forgotten me \\
\poemll    days without number. \\
\poeml \v{33}How well you perfect your techniques\fnote{\fbackref{2:33} Lit. \fbib{your way}} for seeking love. \\
\poemll    Therefore you can teach even the most immoral women\fnote{\fbackref{2:33} Lit. \fbib{the wicked women}} your techniques.\fnote{\fbackref{2:33} Lit. \fbib{your ways}} \\
\poeml \v{34}On your skirts is found the lifeblood of the innocent poor, \\
\poemll    even though you didn't catch them breaking in. \\
\poeml Yet despite all these things, \\
\poeml \v{35}you say, `I'm innocent. \\
\poemlll       Surely his anger has turned away from me.'\,'' \\
\poeml ``I'm about to bring charges against you\fnote{\fbackref{2:35} Lit. \fbib{enter into judgment with you}} \\
\poemll    because you say, `I haven't sinned.' \\
\poeml \v{36}Why do you go about changing your mind so much? \\
\poemll    You will also be disappointed\fnote{\fbackref{2:36} Or \fbib{put to shame}} by Egypt, \\
\poemlll       just as you were disappointed\fnote{\fbackref{2:36} Or \fbib{put to shame}} by Assyria. \\
\poeml \v{37}You will also go out from this place \\
\poemll    with your hands over your heads.\fnote{\fbackref{2:37} I.e. in a gesture indicating mourning} \\
\poeml For the \divine{Lord} has rejected those in whom you trust, \\
\poemll    and you won't prosper through them.''
\end{poetry}
\labelchapt{3}
\passage{God Contemplates Divorcing Israel}

\begin{poetry}
\poeml \chapt{3}
\v{1}``When a man divorces his wife, she leaves him and \\
\poeml becomes another man's wife, \\
\poeml will the first husband\fnote{\fbackref{3:1} Lit. \fbib{he}} return to her again? \\
\poemlll       The land would be deeply polluted, would it not? \\
\poeml Since you have committed fornication with many lovers, \\
\poemll    would you now return to me?'' \\
\poemlll       declares the \divine{Lord}. \\
\poeml \v{2}``Look up to the barren heights and see. \\
\poemll    Is there any place\fnote{\fbackref{3:2} The Heb. lacks \fbib{Is there anyplace}} where you have not been ravished? \\
\poeml You have sat beside the road, waiting\fnote{\fbackref{3:2} The Heb. lacks \fbib{waiting}} for them\fnote{\fbackref{3:2} I.e. as if waiting for a prostitute} \\
\poemll    like a nomad in the desert. \\
\poeml And you have polluted the land \\
\poemll    with your fornication and your wickedness. \\
\poeml \v{3}This is why the rain has been withheld \\
\poemll    and there are no spring showers. \\
\poeml Yet you have a harlot's look\fnote{\fbackref{3:3} Lit. \fbib{forehead}} \\
\poemll    and you refuse to be ashamed. \\
\poeml \v{4}Have you not just called out to me, \\
\poemll    `My father, you are the friend of my youth--- \\
\poeml \v{5}will he hold on to his anger forever, \\
\poemll    will he persist in his wrath to the end?' \\
\poeml Look, you have spoken and done evil things, \\
\poemll    and you have succeeded in it.''\fnote{\fbackref{3:5} The Heb. lacks \fbib{in it}}
\end{poetry}
\passage{The Example of Samaria}

\v{6}In the time of King Josiah the \divine{Lord} told me, ``Have you seen what unfaithful Israel did? She went up on every high hill and under every green tree, and she committed fornication there. \v{7}I thought,\fnote{\fbackref{3:7} Lit. \fbib{I said}} `After she has done all these things, she will return to me.' But she didn't return, and her treacherous sister Judah saw this. \v{8}I saw that even though I had sent unfaithful Israel away for all her adulteries and had given her a\fnote{\fbackref{3:8} The Heb. lacks \fbib{a}} divorce decree, her treacherous sister Judah didn't fear, and she, too, committed adultery. \v{9}She took her fornication so lightly that she polluted the land and committed adultery with stones and trees.\fnote{\fbackref{3:9} I.e. the Canaanite fertility gods were represented by stones and trees.} \v{10}Yet in all this her treacherous sister Judah didn't return to me with her whole heart, but rather deceptively,'' declares the \divine{Lord}.
\passage{A Call for Repentance}

\v{11}Then the \divine{Lord} told me, ``Unfaithful Israel has shown herself more righteous than treacherous Judah. \v{12}Go, proclaim these words to the north, and say,

\begin{poetry}
\poeml `Return, unfaithful Israel,' \\
\poemll    declares the \divine{Lord}. \\
\poeml `I won't look on you in anger, \\
\poemll    for I am gracious,'\fnote{\fbackref{3:12} I.e. characterized by gracious love} \\
\poemlll       declares the \divine{Lord}. \\
\poeml `I won't remain angry forever. \\
\poeml \v{13}`Only acknowledge your iniquity, \\
\poemll    that you have rebelled against the \divine{Lord} your God, \\
\poemll    and have scattered your favors to strangers \\
\poemlll       under every green tree. \\
\poeml But you haven't obeyed me,' \\
\poemlll       declares the \divine{Lord}.
\end{poetry}

\v{14}``Return, unfaithful people,''\fnote{\fbackref{3:14} Or \fbib{sons}} declares the \divine{Lord}, ``for I am your husband.\fnote{\fbackref{3:14} Or \fbib{master}} I'll take you, one from a city and two from a family, and I'll bring you to Zion. \v{15}I'll give you shepherds\fnote{\fbackref{3:15} I.e. leaders} after my own heart, and they'll shepherd you with knowledge and good sense.''

\v{16}``And in those days when you increase in numbers and multiply in the land,'' declares the \divine{Lord}, ``people will no longer say, `The Ark of the Covenant of the \divine{Lord},' and it won't come to mind, and they won't remember it or miss it, nor will it be made again. \v{17}At that time people will call Jerusalem, ``The Throne of the \divine{Lord},'' and all the nations will be gathered to it, to the name of the \divine{Lord}, to Jerusalem. They'll no longer stubbornly follow their own evil desires.\fnote{\fbackref{3:17} Lit. \fbib{follow the stubbornness of their evil hearts}} \v{18}In those days the house of Judah will walk with the house of Israel, and together they'll come to the land that I gave your ancestors as an inheritance.''
\passage{God's Desire for His People}

\v{19}``I said,

\begin{poetry}
\poeml `How I wanted to treat you like children, \\
\poemll    and give you a pleasant land, \\
\poemlll       the most beautiful inheritance of the nations.' \\
\poeml I said, `You will call me, my father, \\
\poemll    and won't turn back from following me.' \\
\poeml \v{20}Instead, like an unfaithful wife leaves her husband, \\
\poemll    so you have been unfaithful to me, house of Israel,'' \\
\poemlll       declares the \divine{Lord}.
\passage{Israel Cries for Help}
\poeml \v{21}``A voice is heard on the barren heights, \\
\poemll    the weeping and pleading of the children of Israel \\
\poeml because they have perverted their way. \\
\poemll    They have forgotten the \divine{Lord} their God.''
\passage{God Calls for Repentance}
\poeml \v{22}``Turn back, unfaithful people,\fnote{\fbackref{3:22} \fbib{Or sons}} \\
\poemll    and I'll heal your unfaithfulness.''
\passage{Israel Replies}
\poeml ``Look, we're coming to you \\
\poemll    because you are the \divine{Lord} our God. \\
\poeml \v{23}Truly the hills are a deception,\fnote{\fbackref{3:23} I.e. as a source of deliverance} \\
\poemll    and the mountains\fnote{\fbackref{3:23} I.e. where false gods were worshipped} are confusion. \\
\poemlll       Truly, in the \divine{Lord} our God is Israel's salvation.'' \\
\poeml \v{24}Since our youth the false gods have consumed \\
\poemll    the products of our ancestors' hard work, \\
\poeml their sheep and their cattle, \\
\poemll    their sons and their daughters. \\
\poeml \v{25}``Let us lie down in our shame, \\
\poemll    and let our humiliation cover us, \\
\poeml because both we and our ancestors have sinned \\
\poemll    against the \divine{Lord} our God from our youth \\
\poemlll       until this present time. \\
\poeml We haven't obeyed the \divine{Lord} our God.''
\end{poetry}
\labelchapt{4}
\passage{Instructions for True Repentance}

\begin{poetry}
\poeml \chapt{4}
\v{1}``Israel, if you return to me,'' \\
\poeml declares the \divine{Lord}, \\
\poeml ``Return to me, \\
\poemll    remove your detestable idols from my presence, \\
\poemlll       and don't waver. \\
\poeml \v{2}If you swear, `as surely as the \divine{Lord} lives,' \\
\poemll    in truth, in justice, and in righteousness, \\
\poeml then nations will be blessed\fnote{\fbackref{4:2} Or \fbib{bless themselves}} by him, \\
\poemll    and in him they will boast.'' \\
\poeml \v{3}For this is what the \divine{Lord} says \\
\poemll    to the men\fnote{\fbackref{4:3} Or \fbib{people}} of Judah and Jerusalem, \\
\poeml ``Break up your unplowed ground, \\
\poemll    and don't sow among thorns. \\
\poeml \v{4}Circumcise yourselves to the \divine{Lord} \\
\poemll    and remove the foreskin of your heart, \\
\poeml you men of Judah and residents of Jerusalem, \\
\poemll    or else my wrath will break out like fire \\
\poeml and burn with no one to put it out, \\
\poemll    because of your evil deeds.''
\passage{Warning of the Coming Disaster}
\poeml \v{5}Declare in Judah, make known in Jerusalem, by saying, \\
\poemll    ``Blow the trumpet in the land, cry out, and say, \\
\poeml `Gather together \\
\poemll    and let's go to the fortified cities!' \\
\poeml \v{6}Raise a standard in the direction of Zion. \\
\poemll    Flee! Don't stand around! \\
\poeml For I'm bringing calamity from the north, \\
\poemll    along with great destruction. \\
\poeml \v{7}A lion has gone up from his thicket, \\
\poemll    and a destroyer of nations has set out. \\
\poeml He has left his place \\
\poemll    to make your land a waste. \\
\poeml Your cities will be ruined, \\
\poemll    and without inhabitants. \\
\poeml \v{8}So, put on sackcloth, \\
\poemll    mourn and wail, \\
\poeml because the burning anger of the \divine{Lord} \\
\poemll    has not turned away from us.'' \\
\poeml \v{9}``On that day,'' declares the \divine{Lord}, \\
\poemll    ``the courage of the king and the leaders will fail. \\
\poeml The priests will be appalled \\
\poemll    and the prophets astounded.''
\end{poetry}

\v{10}Then I replied, ``Ah, Lord GOD, you have completely deceived this people and Jerusalem when you said, `You will have peace,' while the sword is at their\fnote{\fbackref{410} Lit. \fbib{the}} throat!''
\passage{The Scorching Wind of Judgment}

\v{11}At that time, it will be told this people and to Jerusalem, ``A scorching wind from the barren heights in the desert is coming\fnote{\fbackref{4:11} The Heb. lacks \fbib{is coming}} toward my people, and it's not for winnowing or cleansing. \v{12}A wind too strong for that is coming at my bidding.\fnote{\fbackref{4:12} Lit. \fbib{coming to me}} Now I'm judging them as I speak.''
\passage{The People's Response to Judgment}

\begin{poetry}
\poeml \v{13}Look, he comes up like clouds, \\
\poemll    and his chariots are like a whirlwind. \\
\poeml His horses are as swift as eagles. \\
\poemll    Woe to us---we're destroyed! \\
\poeml \v{14}Jerusalem, wash your evil from your heart \\
\poemll    so that you may be delivered. \\
\poeml How long will you harbor \\
\poemll    evil schemes within you? \\
\poeml \v{15}For a voice announces from Dan \\
\poemll    and declares disaster from Mount Ephraim.
\passage{The \divine{Lord} Speaks}
\poeml \v{16}``Tell the nations, `Here they come!'\fnote{\fbackref{4:16} The Heb. lacks \fbib{they come}} \\
\poemll    Proclaim to Jerusalem, \\
\poeml `The besieging forces are coming from a distant land. \\
\poemll    They cry out\fnote{\fbackref{4:16} I.e. battle cries} against the cities of Judah. \\
\poeml \v{17}They have surrounded her like those guarding a field \\
\poemll    because they have rebelled against me,'\,'' \\
\poemlll       declares the \divine{Lord}. \\
\poeml \v{18}``Your lifestyles and your actions \\
\poemll    have brought these things on you. \\
\poeml This is your calamity---it is indeed bitter, \\
\poemll    for it has reached your heart!''
\passage{Jeremiah's Lament for His People}
\poeml \v{19}``My anguish, my anguish! I writhe in pain. \\
\poemll    Oh, the aching\fnote{\fbackref{4:19} Lit. \fbib{walls}} of my heart! \\
\poeml My heart pounds within me; \\
\poemll    I cannot keep silent. \\
\poeml For I hear the sound of the trumpet,\fnote{\fbackref{4:19} I.e. the signal for the troops to attack} \\
\poemll    the alarm for war. \\
\poeml \v{20}Disaster upon disaster is proclaimed, \\
\poemll    for the entire land is devastated. \\
\poeml Suddenly, my tent is destroyed, \\
\poemll    in a moment my curtains. \\
\poeml \v{21}How long will I see the battle standard \\
\poemll    and hear the sound of the trumpet?
\passage{The \divine{Lord}'s Complaint about His People}
\poeml \v{22}``For my people are foolish, \\
\poemll    they don't know me. \\
\poeml They're stupid children, \\
\poemll    they have no understanding. \\
\poeml They're skilled at doing evil, \\
\poemll    but how to do good, they don't know.''
\passage{A Vision of Chaos}
\poeml \v{23}I looked at the earth, and it was formless and void,\fnote{\fbackref{4:23} Cf. Gen 1:2} \\
\poemll    at the heavens, and there was no light there. \\
\poeml \v{24}I looked at the mountains; they were quaking, \\
\poemll    and all the hills moved back and forth. \\
\poeml \v{25}I looked, and no people were there. \\
\poemll    All the birds of the sky had gone. \\
\poeml \v{26}I looked, and the fruitful land\fnote{\fbackref{4:26} Or \fbib{Carmel}} had become a desert. \\
\poemll    All its towns were broken down \\
\poeml because of the \divine{Lord}, \\
\poemll    because of his burning anger.
\end{poetry}

\begin{poetry}
\poeml \v{27}For this is what the \divine{Lord} says:
\end{poetry}

\begin{poetry}
\poeml ``The entire land will be devastated, \\
\poemll    but I won't completely destroy\fnote{\fbackref{4:27} Lit. \fbib{do}} it. \\
\poeml \v{28}Because of this, the land will mourn, \\
\poemll    and the heavens above will be dark. \\
\poeml Because I have spoken and decided, \\
\poemll    I won't turn back from doing it.''
\passage{A Lament for Zion}
\poeml \v{29}At the sound of the horseman and the archer \\
\poemll    the entire city flees. \\
\poeml Its residents go into the thickets and climb among the rocks. \\
\poemll    Every city is abandoned, and no one lives in them. \\
\poeml \v{30}You are ruined! What are you doing \\
\poemll    dressing in scarlet, \\
\poeml putting on golden ornaments, \\
\poemll    and highlighting your eyes with makeup? \\
\poeml You are making yourself beautiful in vain. \\
\poemll    Your lovers reject you--- \\
\poemlll       they're out to kill you. \\
\poeml \v{31}I heard a cry like that of a woman in labor, \\
\poemll    anguish like one giving birth to her firstborn, \\
\poeml the cry of the daughter of Zion gasping for air, \\
\poemll    stretching out her hand: \\
\poemll    ``Woe is me! I'm about to faint in front of killers!''
\end{poetry}
\labelchapt{5}
\passage{A Dialogue about Righteousness: The \divine{Lord} Speaks}

\chapt{5}
\v{1}``Wander through the streets of Jerusalem.

\begin{poetry}
\poeml Look and investigate;\fnote{\fbackref{5:1} Lit. \fbib{know}} \\
\poeml search through her squares \\
\poemll    and see whether you find anyone--- \\
\poeml even one person there---doing justice and seeking truth. \\
\poemll    Then I'll forgive them.\fnote{\fbackref{5:1} Lit. \fbib{her}; i.e. Judah} \\
\poeml \v{2}Although they say, `As surely as the \divine{Lord} lives,' \\
\poemll    still they are swearing falsely.''\fnote{\fbackref{5:2} Or \fbib{swearing to false gods}}
\passage{The Prophet Speaks}
\poeml \v{3}\divine{Lord}, don't your eyes look for truth? \\
\poemll    You struck\fnote{\fbackref{5:3} I.e. in discipline} them, but they didn't flinch.\fnote{\fbackref{5:3} Or \fbib{did not weaken}} \\
\poeml You brought them to an end, \\
\poemll    but they refused to receive discipline. \\
\poeml They made their faces harder than stone, \\
\poemll    and they refused to repent. \\
\poeml \v{4}Then I said, ``These are only the poor, \\
\poemll    they're foolish, \\
\poeml for they don't know the \divine{Lord}'s way, \\
\poemll    the requirement\fnote{\fbackref{5:4} Or \fbib{judgments}} of their God. \\
\poeml \v{5}Let me go to the leaders\fnote{\fbackref{5:5} Lit. \fbib{great ones}} and speak to them. \\
\poemll    For they know the \divine{Lord}'s way, \\
\poemlll       the requirement\fnote{\fbackref{5:5} Or \fbib{judgments}} of their God.''
\passage{The \divine{Lord} Answers}
\poeml ``But they, all together, have broken the yoke \\
\poemll    and torn off the restraints.\fnote{\fbackref{5:5} Or \fbib{cords}} \\
\poeml \v{6}Therefore a lion from the forest will attack them, \\
\poemll    a wolf from the desert will devastate them. \\
\poeml A leopard is watching their towns, \\
\poemll    and everyone who goes out of them \\
\poemlll       will be torn to pieces. \\
\poeml For their transgressions are many, \\
\poemll    and their apostasies numerous. \\
\poeml \v{7}Why should I forgive you? \\
\poemll    Your sons have forsaken me, \\
\poeml and you have sworn by those \\
\poemll    who aren't gods. \\
\poeml When I gave them enough food to satisfy them, \\
\poemll    they committed adultery \\
\poemll    and marched to the prostitute's house. \\
\poeml \v{8}They were well-fed, lusty stallions, \\
\poemll    each one neighing after his neighbor's wife. \\
\poeml \v{9}``Should I not punish them for these things?'' \\
\poemll    asks the \divine{Lord}, \\
\poeml ``And on a nation like this, \\
\poemll    should I not seek retribution?''
\passage{The People Reject God's Warning}
\poeml \v{10}``Go through her rows of vines and destroy them, \\
\poemll    but don't completely destroy them. \\
\poeml Strip away her branches, \\
\poemll    because they aren't the \divine{Lord}'s. \\
\poeml \v{11}For both the house of Israel and the house of Judah \\
\poemll    have been utterly unfaithful to me,'' \\
\poemlll       declares the \divine{Lord}. \\
\poeml \v{12}``They have lied about the \divine{Lord} \\
\poemll    by saying, `He wouldn't do that!\fnote{\fbackref{5:12} Lit. \fbib{Not he}} \\
\poeml Disaster won't come on us. \\
\poemll    We won't see sword and famine. \\
\poeml \v{13}The prophets are nothing but\fnote{\fbackref{5:13} The Heb. lacks \fbib{nothing but}} wind, \\
\poemll    and the word is not in them. \\
\poeml So may the disaster happen to them!'\,''\fnote{\fbackref{5:13} Lit. \fbib{so let it be done to them}}
\end{poetry}

\v{14}Therefore, this is what the \divine{Lord} God of the Heavenly Armies says:

\begin{poetry}
\poeml ``Because you people\fnote{\fbackref{5:14} Heb. \fbib{you} (pl.)} have said this, \\
\poemll    my words in your\fnote{\fbackref{5:14} Heb. \fbib{your} (masculine sing.); i.e. in Jeremiah's mouth} mouth will become a fire \\
\poemlll       and these people the wood. \\
\poemll    The fire\fnote{\fbackref{5:14} Lit. \fbib{It}} will destroy them. \\
\poeml \v{15}People of Israel, I'm now bringing \\
\poemll    a nation from far away to attack you,'' \\
\poemlll       declares the \divine{Lord}. \\
\poeml ``It is an enduring nation, \\
\poemll    an ancient nation, \\
\poeml a nation whose language you don't know. \\
\poemll    And you won't understand what they say. \\
\poeml \v{16}Their quiver is like an open grave, \\
\poemll    and all of them are powerful warriors. \\
\poeml \v{17}``They'll devour your harvest and your food. \\
\poemll    They'll devour your sons and your daughters. \\
\poemlll       They'll devour your vines and your fig trees. \\
\poeml With their swords they'll batter down \\
\poemll    your fortified cities in which you trust.
\end{poetry}

\v{18}``Yet even in those days,'' declares the \divine{Lord}, ``I won't destroy you completely. \v{19}When the people\fnote{\fbackref{5:19} Lit. \fbib{they}} ask, `Why has the \divine{Lord} our God done all this to us?' you are to say to them, `Just as you have forsaken me and served foreign gods in your land, so you will serve strangers in a land that is not yours.'\,''
\passage{The \divine{Lord} Warns a Stubborn People}

\begin{poetry}
\poeml \v{20}``Declare this to the descendants\fnote{\fbackref{5:20} Or \fbib{family}} of Jacob, \\
\poemll    and proclaim it in Judah: \\
\poeml \v{21}`Hear this, you foolish and stupid people: \\
\poemll    They have eyes, but don't see; \\
\poemlll       they have ears, but don't hear. \\
\poeml \v{22}`You don't fear me, do you?' declares the \divine{Lord}. \\
\poemll    `You don't tremble before me, do you? \\
\poeml I'm the one who put the sand as a boundary for the sea, \\
\poemll    a perpetual barrier that it cannot cross.\fnote{\fbackref{5:22} Or \fbib{statute that it cannot transgress}} \\
\poeml Though the waves toss, they cannot prevail against it, \\
\poemll    though they roar, they cannot cross it.' \\
\poeml \v{23}But these people have stubborn and rebellious hearts. \\
\poemll    They have turned aside and have gone away. \\
\poeml \v{24}They don't say to themselves, \\
\poemll    `Let's fear the \divine{Lord} our God, \\
\poeml who gives rain in its season, \\
\poemll    both the autumn and the spring rain. \\
\poeml He sets aside for us the weeks appointed \\
\poemll    for the harvest.' \\
\poeml \v{25}Your iniquities have turned these things away, \\
\poemll    and your sins have held back from you what is good. \\
\poeml \v{26}``Evil men are found among my people. \\
\poemll    They lie in wait like someone who traps birds. \\
\poeml They set a trap, \\
\poemll    but they do so to catch people. \\
\poeml \v{27}Like a cage full of birds, \\
\poemll    so their houses are filled with treachery. \\
\poeml This is how they have become prominent and rich, \\
\poeml \v{28}and have grown fat and sleek. \\
\poeml There is no limit\fnote{\fbackref{5:28} Lit. \fbib{pass over}; or \fbib{transgress}} to their evil deeds. \\
\poemll    They don't argue the case of the orphan to secure\fnote{\fbackref{5:28} Lit. \fbib{win}} justice. \\
\poemlll       They don't defend the rights of\fnote{\fbackref{5:28} Lit. \fbib{judge justly}} the poor. \\
\poeml \v{29}`Should I not punish them for this?'\fnote{\fbackref{5:29} Or \fbib{punish these people}} \\
\poemll    asks the \divine{Lord}. \\
\poeml `Should I not avenge myself \\
\poemll    on a nation like this?' \\
\poeml \v{30}``An appalling and horrible thing \\
\poemll    has happened in the land: \\
\poeml \v{31}The prophets prophesy falsely, \\
\poemll    the priests rule by their own authority, \\
\poeml and my people love it this way. \\
\poemll    But what will you do in the end?''
\end{poetry}
\labelchapt{6}
\passage{The Enemy Besieges Jerusalem}

\chapt{6}
\v{1}``Flee to safety, you people of Benjamin,

\begin{poetry}
\poeml leave Jerusalem. \\
\poeml Sound the trumpet in Tekoa, \\
\poemll    and raise a signal over Beth-haccerem! \\
\poeml For calamity and terrible destruction \\
\poemll    are turning toward you\fnote{\fbackref{6:1} The Heb. lacks \fbib{toward you}} from the north. \\
\poeml \v{2}I'll destroy the lovely and delicate \\
\poemll    Daughter of Zion.\fnote{\fbackref{6:2} I.e. Jerusalem} \\
\poeml \v{3}Shepherds and their flocks will come against her. \\
\poemll    They'll pitch their tents all around her, \\
\poemlll       and every one will tend his flock in his own place. \\
\poeml \v{4}Prepare for war against her. \\
\poemll    Get ready, let's attack at noon! \\
\poeml How terrible for us that the day is coming to an end,\fnote{\fbackref{6:4} Lit. \fbib{is turning}} \\
\poemll    and that the evening shadows are lengthening. \\
\poeml \v{5}Get ready, let's attack at night, \\
\poemll    and destroy her fortresses.''\fnote{\fbackref{6:5} Or \fbib{palaces}}
\end{poetry}
\passage{Instructions for the Attackers}

\begin{poetry}
\poeml \v{6}For this is what the \divine{Lord} of the Heavenly Armies says:
\end{poetry}

\begin{poetry}
\poeml ``Cut down trees and \\
\poemll    set up siege works against Jerusalem. \\
\poeml It is the city to be judged, \\
\poemll    and there is oppression throughout the entire city.\fnote{\fbackref{6:6} Lit. \fbib{through her}} \\
\poeml \v{7}As a well keeps its waters fresh,\fnote{\fbackref{6:7} Or \fbib{cool}} \\
\poemll    so the city\fnote{\fbackref{6:7} Lit. \fbib{she}} keeps her wickedness fresh.\fnote{\fbackref{6:7} Or \fbib{cool}} \\
\poeml Violence and destruction are heard in her, \\
\poemll    sickness and wounds are always before me. \\
\poeml \v{8}Be warned, Jerusalem, \\
\poemll    or I'll be alienated from you. \\
\poeml I'll make you desolate, \\
\poemll    a land not inhabited.''
\end{poetry}

\v{9}This is what the \divine{Lord} of the Heavenly Armies says:

\begin{poetry}
\poeml ``Let them glean the remnant of Israel \\
\poemll    as thoroughly as they would the vine. \\
\poeml Pass your hand over them like grape gatherers \\
\poemll    over the branches. \\
\poeml \v{10}To whom will I speak and give a warning \\
\poemll    so they'll listen? \\
\poeml Look, their ears are closed,\fnote{\fbackref{6:10} Lit. \fbib{uncircumcised}} \\
\poemll    and they cannot hear. \\
\poeml Look, this message from the \divine{Lord} is contemptible to them; \\
\poemll    they don't delight in it. \\
\poeml \v{11}I'm full of the wrath of the \divine{Lord}, \\
\poemll    and I'm tired of holding it back. \\
\poeml Pour it out on the children in the street \\
\poemll    and on the groups of young men gathered together. \\
\poeml Indeed, both husband and wife will be caught in it, \\
\poemll    the old and the very old. \\
\poeml \v{12}Their houses will be turned over to others--- \\
\poemll    their fields and wives together--- \\
\poeml when I stretch out my hand against \\
\poemll    those who live in the land,'' \\
\poemlll       declares the \divine{Lord}. \\
\poeml \v{13}``Indeed, from the least important to the most important, \\
\poemll    they're all greedy for dishonest gain. \\
\poeml From prophet to priest, \\
\poemll    they all act deceitfully. \\
\poeml \v{14}They treated my people's wound superficially, telling them, \\
\poemll    `Peace, peace,' but there is no peace. \\
\poeml \v{15}Were they ashamed because they did \\
\poemll    what was repugnant to God?\fnote{\fbackref{6:15} Lit. \fbib{committed an abomination}} \\
\poeml They were not ashamed at all--- \\
\poemll    they don't even know how to blush! \\
\poeml Therefore they'll fall with those who fall. \\
\poemll    When I punish them, they'll be brought down,'' \\
\poemlll       says the \divine{Lord}.
\end{poetry}
\passage{Israel Refuses to Repent}

\v{16}This is what the \divine{Lord} says:

\begin{poetry}
\poeml ``Stand beside the roads and watch. \\
\poemll    Ask for the ancient paths, where the good way is. \\
\poeml Walk in it and find rest for yourselves. \\
\poemll    But they said, `We won't walk in it!'\fnote{\fbackref{6:16} The Heb. lacks \fbib{in it}} \\
\poeml \v{17}I appointed watchmen over you. \\
\poemll    Listen for the sound of the trumpet. \\
\poeml But they said, `We won't listen!' \\
\poeml \v{18}Therefore, hear, nations, \\
\poemll    and know, congregation, \\
\poemlll       what will happen to them.\fnote{\fbackref{6:18} Or \fbib{what is among them}} \\
\poeml \v{19}Listen, earth! \\
\poemll    I'm about to bring calamity on this people, \\
\poemlll       on the fruit of their plans, \\
\poeml because they didn't listen to my words \\
\poemll    and they rejected my instruction.\fnote{\fbackref{6:19} Or \fbib{Law}} \\
\poeml \v{20}What good is frankincense \\
\poemll    that comes from Sheba\fnote{\fbackref{6:20} I.e. Yemen} to me, \\
\poemlll       or sweet cane from a distant country? \\
\poeml Your burnt offerings aren't acceptable, \\
\poemll    nor are your sacrifices pleasing to me.'' \\
\poeml \v{21}Therefore, this is what the \divine{Lord} says:
\end{poetry}

\begin{poetry}
\poeml ``I'm about to put stumbling blocks in front of this people, \\
\poemll    and fathers and sons will stumble over them together. \\
\poemlll       The neighbor and his friends will perish.''
\end{poetry}
\passage{The Invaders from the North}

\begin{poetry}
\poeml \v{22}This is what the \divine{Lord} says:
\end{poetry}

\begin{poetry}
\poeml ``Look, people are coming from a northern country. \\
\poemll    A great nation is stirring from the ends of the earth. \\
\poeml \v{23}They grab bow and spear; \\
\poemll    they're cruel and show no mercy. \\
\poeml Their sound roars like the sea \\
\poemll    as they ride on horses, \\
\poeml deployed like men ready for battle \\
\poemll    against you, daughter of Zion.'' \\
\poeml \v{24}We have heard the news about it, \\
\poemll    and our hands are limp. \\
\poeml Distress has seized us \\
\poemll    like a woman in labor. \\
\poeml \v{25}Don't go out into the field, \\
\poemll    and don't travel on the road, \\
\poeml because the enemy has a sword, \\
\poemll    and terror is on every side. \\
\poeml \v{26}Daughter of my people, put on sackcloth \\
\poemll    and roll in ashes. \\
\poeml Mourn with bitter wailing, \\
\poemll    as one mourns at the death of\fnote{\fbackref{6:26} The Heb. lacks \fbib{at the death of}} an only son. \\
\poeml For the destroyer will come on us suddenly.
\passage{People Rejected by the \divine{Lord}}
\poeml \v{27}``I've made you an assayer\fnote{\fbackref{6:27} I.e. one who tests metal for purity} of my people, \\
\poemll    as well as\fnote{\fbackref{6:27} The Heb. lacks \fbib{as well as} (cf. Jer 1:18)} a fortress. \\
\poeml You know how \\
\poemll    to test their way.'' \\
\poeml \v{28}All of them are very rebellious, \\
\poemll    going around as slanderers. \\
\poeml They're bronze and iron, \\
\poemll    and all of them are corrupt. \\
\poeml \v{29}The bellows blow fiercely to consume \\
\poemll    the lead with the fire. \\
\poeml The assayer\fnote{\fbackref{6:29} Lit. \fbib{He}} keeps on refining, \\
\poemll    but the impurities\fnote{\fbackref{6:29} Or \fbib{wicked}} aren't separated out. \\
\poeml \v{30}They're called reject silver, \\
\poemll    because the \divine{Lord} has rejected them.
\end{poetry}
\labelchapt{7}
\passage{Jeremiah's Temple Sermon: Judah's Idolatry}

\chapt{7}
\v{1}The message that came to Jeremiah from the \divine{Lord}: \v{2}``Stand at the gate of the \divine{Lord}'s Temple and proclaim this message there. Say, `Listen to this message from the \divine{Lord}, all you people of Judah who come through these gates to worship the \divine{Lord}.'\,''

\v{3}This is what the \divine{Lord} of the Heavenly Armies, the God of Israel, says:

\begin{poetry}
\poeml ``Change\fnote{\fbackref{7:3} Lit. \fbib{Make good}} your ways and your deeds, and I'll let you live in this place. \v{4}Don't trust deceptive words like these, and say, `The Temple of the \divine{Lord}, the Temple of the \divine{Lord}, the Temple of the \divine{Lord},' \v{5}but rather, truly change\fnote{\fbackref{7:5} Lit. \fbib{make good}} your ways and your deeds. If you truly practice justice between each person and his neighbor, \v{6}and if you don't oppress the alien, the orphan, and the widow, and don't shed an innocent person's blood in this place, and if you don't follow other gods to your own harm,\fnote{\fbackref{7:6} Or \fbib{disaster}} \v{7}then I'll let you dwell in this land, the land that I gave to your ancestors forever and ever. \\
\poeml \v{8}``Look, you're trusting in deceptive words that cannot benefit.\fnote{\fbackref{7:8} Or \fbib{profit}} \v{9}Will you steal, murder, commit adultery, swear by false gods, burn incense to Baal, follow other gods that you don't know, \v{10}and then come to stand before me in this house that is called by my name and say, `We're delivered' so we can continue to do all these things that are repugnant to God?\fnote{\fbackref{7:10} Lit. \fbib{all these abominations}} \v{11}Has this house that is called by my name become a hideout\fnote{\fbackref{7:11} Lit. \fbib{cave}} for bandits in your eyes? Look, I'm watching,'' declares the \divine{Lord}. \\
\poeml \v{12}``Go to my place that was in Shiloh, where I first caused my name to dwell. See what I did to it because of the evil of my people Israel. \v{13}Now, because you have done all these things,'' declares the \divine{Lord}, ``I spoke to you over and over again,\fnote{\fbackref{7:13} Lit. \fbib{getting up early and speaking}} but you didn't listen. I called to you, but you didn't answer. \v{14}Just as I did to Shiloh, I'll do to the house in which you trust and which is called by my name, the place that I gave to you and your ancestors. \v{15}I'll cast you out of my sight, just as I cast out all your brothers, all the descendants of Ephraim. \\
\poeml \v{16}``As for you, don't pray on behalf of this people, don't cry or offer a petition for them, and don't plead with me, for I won't listen to you. \v{17}Don't you see what they're doing in the cities of Judah and in the streets of Jerusalem? \v{18}The children gather wood, the fathers kindle the fire, and the women knead dough to make cakes for the Queen of Heaven,\fnote{\fbackref{7:18} I.e. the Near Eastern fertility goddess Ishtar} and they pour out liquid offerings to other gods in order to provoke me. \v{19}Are they provoking me?'' asks the \divine{Lord}. ``Is it not themselves, and to their own shame?'' \v{20}Therefore, this is what the Lord \divine{God} says: ``I'm about to pour out my anger and my wrath on this place, on people and animals, on the trees of the field, and on the fruit of the ground. It will burn, and it won't be put out.''
\end{poetry}

\v{21}This is what the \divine{Lord} of the Heavenly Armies, the God of Israel, says:

\begin{poetry}
\poeml ``Add your burnt offerings to your sacrifices and eat the meat. \v{22}Indeed, when I brought your ancestors out of the land of Egypt, I didn't speak or command them about burnt offering and sacrifice, \v{23}but I did give them this command:\fnote{\fbackref{7:23} Lit. \fbib{I commanded them this word}} `Obey me and I'll be your God, and you will be my people. Walk in all the ways that I command you so it will go well for you.' \v{24}But they didn't listen,\fnote{\fbackref{7:24} Or \fbib{obey}} nor did they pay attention.\fnote{\fbackref{7:24} Lit. \fbib{incline their ears}} They pursued their own plans,\fnote{\fbackref{7:24} Lit. \fbib{They walked in plans}} stubbornly following their own evil desires.\fnote{\fbackref{7:24} Lit. \fbib{following the stubbornness of their evil hearts}} They went backward and not forward. \v{25}From the day your ancestors left the land of Egypt to this present time, I've sent all my servants, the prophets, to you, again and again.\fnote{\fbackref{7:25} Lit. \fbib{daily getting up early and sending them}} \v{26}But they didn't listen to me, and they didn't pay attention.\fnote{\fbackref{7:26} Lit. \fbib{incline their ears}} They stiffened their necks, and they did more evil than their ancestors. \\
\poeml \v{27}``You will tell them all these things, but they won't listen to you. You will call out to them, but they won't answer you. \v{28}You will say to them, `This is the nation that wouldn't listen to the voice\fnote{\fbackref{7:28} I.e. wouldn't obey} of the \divine{Lord} its God and wouldn't accept correction. Truth has perished; it has been eliminated from their discussions.' \\
\poeml \v{29}``Cut off your hair and throw it away; \\
\poemll    let your lamentations rise on the barren heights, \\
\poeml because the \divine{Lord} has rejected and abandoned \\
\poemll    the generation that is subject to his wrath.\fnote{\fbackref{7:29} Lit. \fbib{generation of his wrath}}
\end{poetry}

\begin{poetry}
\poeml \v{30}``For the people of Judah have done evil in my eyes,'' declares the \divine{Lord}. ``They have put their detestable idols\fnote{\fbackref{7:30} Lit. \fbib{their detestable things}} in the house that is called by my name in order to defile it. \v{31}They have built high places at Topheth in the Valley of Ben-hinnom to burn their sons and daughters in the fire. I didn't command this, and it never entered my mind!
\end{poetry}

\v{32}``Therefore, the time is near,'' declares the \divine{Lord}, ``when it will no longer be called Topheth or the Valley of Ben-hinnom, but the Valley of Slaughter. They'll bury in Topheth because there is no other\fnote{\fbackref{7:32} The Heb. lacks \fbib{other}} place to do it.\fnote{\fbackref{7:32} The Heb. lacks \fbib{to do it}} \v{33}The dead bodies of these people will be food for the birds of the sky and for the animals of the land, and no one will disturb them. \v{34}In the towns of Judah and the streets of Jerusalem I'll bring an end to the sound of gladness and rejoicing, to the sounds of the bridegroom and bride, for the land will become a wasteland.''
\labelchapt{8}

\chapt{8}
\v{1}``At that time,'' declares the \divine{Lord}, ``the bones of the king of Judah, the bones of his officials, the bones of the priests, the bones of the prophets, and the bones of the residents of Jerusalem will be removed from their graves. \v{2}They'll be spread out to the sun, the moon, and all the stars of the heavens, which they loved and served,\fnote{\fbackref{8:2} Or \fbib{worshipped}} and which they followed, consulted, and worshipped. Their bones\fnote{\fbackref{8:2} Lit. \fbib{They}} won't be collected, nor will they be buried. They'll be like dung on the surface of the ground.

\v{3}``In all the places where the people\fnote{\fbackref{8:3} Lit. \fbib{they}} remain, where I've banished them, death will be chosen over life by all the remnant that remains of this evil family,'' declares the \divine{Lord} of the Heavenly Armies.
\passage{A Stubborn People}

\v{4}``You are to say to them, `This is what the \divine{Lord} says:

\begin{poetry}
\poeml ``Will a person fall down and then not get up? \\
\poemll    Will someone turn away\fnote{\fbackref{8:4} Or \fbib{repent}} and then not turn back again?\fnote{\fbackref{8:4} Or \fbib{not repent}} \\
\poeml \v{5}Why has this people turned away?\fnote{\fbackref{8:5} Lit. \fbib{people committed apostasy}} \\
\poemll    Why does Jerusalem continue in apostasy? \\
\poemlll       They hold on to deceit and refuse to repent. \\
\poeml \v{6}I've listened and I've heard, \\
\poemll    and what they say is not right. \\
\poeml No one repents of his evil and says, \\
\poemll    `What have I done?' \\
\poeml ``They all turn to their own course \\
\poemll    like a horse racing into battle. \\
\poeml \v{7}Even the stork in the sky knows its seasons, \\
\poemll    and the dove, the swallow, and the crane observe the time for migration. \\
\poeml But my people don't know \\
\poemll    the requirements\fnote{\fbackref{8:7} I.e. the behavior God expects of his people} of the \divine{Lord}. \\
\poeml \v{8}How can you say, `We're wise, \\
\poemll    and the Law of the \divine{Lord} is with us,' \\
\poeml when, in fact, the deceitful pen of the scribe has made it \\
\poemll    into something that deceives. \\
\poeml \v{9}The wise men will be put to shame. \\
\poemll    They'll be dismayed and taken captive. \\
\poeml Look, they have rejected the message from the \divine{Lord}! \\
\poemll    So what kind of wisdom do they have? \\
\poeml \v{10}Therefore, I'll give their wives to others, \\
\poemll    and their fields to new owners. \\
\poeml Indeed, from the least important to the most important, \\
\poemll    they're all greedy for dishonest gain. \\
\poeml From prophet to priest, \\
\poemll    they all act deceitfully. \\
\poeml \v{11}They have treated my people's\fnote{\fbackref{8:11} Lit. \fbib{of the daughter of my people}} wound \\
\poemll    superficially, telling them, `Peace, peace,' \\
\poemlll       when there is no peace. \\
\poeml \v{12}Are they ashamed because they have done \\
\poemll    what is repugnant to God?\fnote{\fbackref{8:12} Lit. \fbib{committed an abomination}} \\
\poeml They weren't ashamed at all; \\
\poemll    they don't even know how to blush! \\
\poeml Therefore they'll fall with those who fall. \\
\poemll    When I punish them, they'll be brought down,'' \\
\poemlll       says the \divine{Lord}. \\
\poeml \v{13}``I would have gathered them,'' \\
\poemll    declares the \divine{Lord}, \\
\poeml ``but there were no grapes on the vine, \\
\poemll    and no figs on the fig tree, \\
\poeml and their leaves were withered. \\
\poemll    What I've given them has been taken away.''\,'\,''
\passage{The People Respond}
\poeml \v{14}Why are we sitting here? \\
\poemll    Join together! Let's go to the fortified cities \\
\poemlll       and perish there! \\
\poeml For the \divine{Lord} our God has condemned us to perish \\
\poemll    and given us poisoned water to drink, \\
\poemlll       because we have sinned against him.\fnote{\fbackref{8:14} Lit. \fbib{the \divine{Lord}}} \\
\poeml \v{15}We waited for peace, but no good has come, \\
\poemll    for a time of healing, but instead there was terror.
\passage{The \divine{Lord}'s Warning}
\poeml \v{16}``The snorting of their horses is heard from Dan. \\
\poemll    At the neighing of their stallions, \\
\poemlll       the whole earth quakes. \\
\poeml They're coming to devour \\
\poemll    the land and all it contains, \\
\poemlll       the city and all who live in it. \\
\poeml \v{17}Look, I'll send snakes among you, \\
\poemll    vipers that cannot be charmed, \\
\poemlll       and they'll bite you.''
\passage{Jeremiah Mourns for His People}
\poeml \v{18}Incurable sorrow has overwhelmed me, \\
\poemll    my heart is sick within me. \\
\poeml \v{19}Listen! My people\fnote{\fbackref{8:19} Lit. \fbib{the daughter of my people}} cry \\
\poemll    from a distant land: \\
\poeml ``Is the \divine{Lord} no longer in Zion? \\
\poemll    Is her king no longer there?''
\passage{The \divine{Lord} Speaks}
\poeml ``Why did they provoke me to anger with their images, \\
\poemll    with their worthless foreign gods?''
\passage{The People Speak}
\poeml \v{20}The harvest is past, \\
\poemll    the summer has ended, \\
\poemlll       and we haven't been delivered.
\passage{The Prophet Mourns}
\poeml \v{21}Because my people\fnote{\fbackref{8:21} Lit. \fbib{the daughter of my people}} are crushed, I'm crushed. \\
\poemll    I mourn, and dismay overwhelms me. \\
\poeml \v{22}Is there no balm in Gilead? \\
\poemll    Is there no physician there? \\
\poemlll       So why is there no healing for my people?\fnote{\fbackref{8:22} Lit. \fbib{daughter of my people}}
\end{poetry}
\labelchapt{9}
\passage{The \divine{Lord}'s Sorrow for His People}

\begin{poetry}
\poeml \chapt{9}
\v{1}\fnote{\fbackref{9:1} Because this verse is 8:23 in MT, there is a one verse discrepancy between MT and the ISV throughout this chapter.}``Oh, that my head were a spring of water,\fnote{\fbackref{9:1} Lit. \fbib{were waters}} \\
\poeml and my eyes a fountain of tears, \\
\poeml for then I would cry day and night for those \\
\poemll    of my people\fnote{\fbackref{9:1} Lit. \fbib{for the daughter of my people}} who have been killed. \\
\poeml \v{2}Oh, that I had a lodging place for travelers in the desert, \\
\poemll    so that I could leave my people \\
\poemlll       and go away from them. \\
\poeml For all of them are adulterers, \\
\poemll    a band of traitors. \\
\poeml \v{3}They use their tongues like a bow. \\
\poemll    Lies rather than truth fly throughout\fnote{\fbackref{9:3} Lit. \fbib{prevail in}} the land. \\
\poeml They progress from one evil to another, \\
\poemll    and they don't know me,'' \\
\poemlll       declares the \divine{Lord}. \\
\poeml \v{4}``Beware of your neighbors, and don't trust \\
\poemll    any of your relatives. \\
\poeml For all of your relatives act deceitfully, \\
\poemll    and every friend goes around as a slanderer. \\
\poeml \v{5}People deceive their friends, \\
\poemll    and they don't tell the truth. \\
\poeml They have taught their tongues to tell lies. \\
\poemll    They exhaust themselves practicing evil.\fnote{\fbackref{9:5} Or \fbib{themselves with iniquity}} \\
\poeml \v{6}You yourself live in the midst of deception, \\
\poemll    and because they are deceived they do not know me,'' \\
\poemlll       declares the \divine{Lord}. \\
\poeml \v{7}Therefore, this is what the \divine{Lord} of the Heavenly Armies says:
\end{poetry}

\begin{poetry}
\poeml ``Look, I'm about to refine and test them. \\
\poemll    Because they're my people, what else can I do?\fnote{\fbackref{9:7} Lit. \fbib{because of the daughter of my people}} \\
\poeml \v{8}Their tongue is a deadly arrow \\
\poemll    that speaks deceit. \\
\poeml With his mouth a person says, `Peace,' to his friend, \\
\poemll    but inwardly he sets a trap for him. \\
\poeml \v{9}Should I not punish them for these things?''\fnote{\fbackref{9:9} Or \fbib{punish these people}} \\
\poemll    asks the \divine{Lord}, \\
\poemlll       ``and should I not avenge myself on a nation like this?'' \\
\poeml \v{10}I'll weep and mourn for the mountains, \\
\poemll    and lament for the desert pastures, \\
\poeml because they are desolate and no one passes through them. \\
\poemll    They don't hear the lowing of the cattle. \\
\poeml Both the birds of the sky and the animals have fled. \\
\poemll    They're gone! \\
\poeml \v{11}``I'll make Jerusalem a heap of ruins, \\
\poemll    a refuge for jackals. \\
\poeml I'll make the towns of Judah desolate, \\
\poemll    without inhabitants.''
\end{poetry}
\passage{The Reason for Judgment}

\v{12}Who is the wise person who understands this, and to whom has the \divine{Lord}\fnote{\fbackref{9:12} Lit. \fbib{the mouth of the \divine{Lord}}} spoken so that he may declare it? Why is the land destroyed, ruined like the desert, without anyone passing through it? \v{13}The \divine{Lord} said, ``It is because they have forsaken my Law that I gave them. They didn't obey me and didn't live according to it. \v{14}Instead, they followed their rebellious hearts and the Baals,\fnote{\fbackref{9:14} I.e. images of the Canaanite storm god} as their ancestors taught them.''

\v{15}Therefore, this is what the \divine{Lord} of the Heavenly Armies, the God of Israel, says: ``Look, I'll make these people eat wormwood\fnote{\fbackref{9:15} I.e. a bitter plant} and drink poisoned water. \v{16}I'll scatter them among nations that neither they nor their ancestors have known, and I'll pursue them with the sword until I've finished them off.''
\passage{A Call to Lament}

\v{17}This is what the \divine{Lord} of the Heavenly Armies says:

\begin{poetry}
\poeml ``Think about what I'm saying!\fnote{\fbackref{9:17} The Heb. lacks \fbib{about what I'm saying}} \\
\poemll    Indeed, call out the professional mourners!\fnote{\fbackref{9:17} Lit. \fbib{lamenting women}} \\
\poemlll       Send for the best of them to come. \\
\poeml \v{18}Let them hurry and lament for us. \\
\poemll    Let tears run down from our eyes, \\
\poemlll       and let our eyelids flow with water. \\
\poeml \v{19}For a sound of mourning is heard from Zion: \\
\poemll    `How we're ruined! \\
\poeml Our shame is very great, \\
\poemll    because we have left the land, \\
\poemlll       because our houses are torn down.'\,'' \\
\poeml \v{20}``Now, you women, hear the message from the \divine{Lord}; \\
\poemll    listen to what he has to say! \\
\poeml Teach your daughters how to mourn, \\
\poemll    let every woman teach\fnote{\fbackref{9:20} The Heb. lacks \fbib{teach}} her friend how to lament. \\
\poeml \v{21}For death comes up through our windows; \\
\poemll    it has come into our palaces \\
\poeml to eliminate children from the streets \\
\poemll    and young men from the town squares. \\
\poeml \v{22}Speak! `This is what the \divine{Lord} says: \\
\poeml ``The corpses of people will fall like dung \\
\poemll    on the surface of the field, \\
\poeml and like a row of cut grain behind \\
\poemll    the harvester when there is no one to gather it.''\,'\,''
\end{poetry}
\passage{True Wisdom and the Coming Judgment}

\v{23}This is what the \divine{Lord} says: ``The wise man is not to boast in his wisdom; the strong man is not to boast in his strength; and the rich man is not to boast in his riches. \v{24}Rather, let the one who boasts, boast in this: that he understands and knows me, for I am the \divine{Lord} who acts with gracious love, justice, and righteousness in the land. I delight in these things,'' declares the \divine{Lord}.

\v{25}``Look, days are coming,'' declares the \divine{Lord}, ``when I'll punish all who are circumcised only in the flesh:\fnote{\fbackref{9:25} Lit. \fbib{circumcised of foreskin}} \v{26}Egypt, Judah, Edom, the people of Ammon, Moab, all those who live in the desert and shave the corners of their beard;\fnote{\fbackref{9:26} Lit. \fbib{cut off of side}} indeed all the other\fnote{\fbackref{9:26} The Heb. lacks \fbib{other}} nations that are uncircumcised, and all the house of Israel that is uncircumcised of heart.''
\labelchapt{10}
\passage{The True God and Worthless Idols}

\chapt{10}
\v{1}Hear the message that the \divine{Lord} has spoken to you, house of Israel. \v{2}This is what the \divine{Lord} says:

\begin{poetry}
\poeml ``Don't learn the way of the nations, \\
\poemll    and don't be terrified by signs in the heavens, \\
\poemll    though the nations are terrified of them. \\
\poeml \v{3}For the practices\fnote{\fbackref{10:3} Or \fbib{customs, ordinances}} of the people are worthless. \\
\poemll    Indeed, a tree is cut down from the forest; \\
\poemlll       it's the work of the hands of a craftsman\fnote{\fbackref{10:3} Or \fbib{engraver}; i.e. a wood carver} with an ax. \\
\poeml \v{4}They decorate it with silver and gold. \\
\poemll    They secure it with nails and hammers \\
\poemlll       so it won't totter. \\
\poeml \v{5}Their idols\fnote{\fbackref{10:5} Lit. \fbib{They}} are like scarecrows in a cucumber field. \\
\poemll    They can't speak! \\
\poeml They must always be carried \\
\poemll    because they can't walk! \\
\poeml Don't be afraid of them \\
\poemll    because they can do no harm, \\
\poemlll       nor can they do any good.'' \\
\poeml \v{6}There is no one like you, \divine{Lord}. \\
\poemll    You are great, and your name is great and powerful. \\
\poeml \v{7}Who wouldn't fear you, king of the nations? \\
\poemll    This is what you deserve! \\
\poeml Indeed, among all the wise men of the nations, \\
\poemll    and throughout all their kingdoms, \\
\poemlll       there is no one like you! \\
\poeml \v{8}Everyone is stupid\fnote{\fbackref{10:8} I.e. like a beast} and senseless. \\
\poemll    They follow worthless instruction \\
\poemlll       from a piece of wood!\fnote{\fbackref{10:8} Lit. \fbib{it is worthless instruction from wood}} \\
\poeml \v{9}Beaten silver is brought from Tarshish, \\
\poemll    and gold from Uphaz. \\
\poeml The idols are\fnote{\fbackref{10:9} Lit. \fbib{It is}} the work of a craftsman\fnote{\fbackref{10:9} Or \fbib{engraver}; i.e. a wood carver} \\
\poemll    and of the hands of a goldsmith. \\
\poeml Their clothing is violet and purple. \\
\poemll    The idols\fnote{\fbackref{10:9} Lit. \fbib{They}} are all the work of skilled craftsmen. \\
\poeml \v{10}The \divine{Lord} is the true God; \\
\poemll    he's the living God and the everlasting king. \\
\poeml At his wrath the earth quakes, \\
\poemll    and the nations cannot endure his indignation.
\end{poetry}

\v{11}\fnote{\fbackref{10:11} This verse is in Aramaic, the language the exiles would speak in Babylon.}Tell this to them: ``The gods who\fnote{\fbackref{10:11} Lit. \fbib{the one who}} didn't make the heavens and the earth will perish from the earth and from these heavens.''
\passage{A Hymn of Praise to God}

\begin{poetry}
\poeml \v{12}The \divine{Lord} is\fnote{\fbackref{10:12} The Heb. lacks \fbib{The \divine{Lord} is}} the one who made \\
\poemll    the world by his power, \\
\poeml who established the earth by his wisdom \\
\poemll    and stretched out the heavens by his understanding. \\
\poeml \v{13}When his voice sounds there is thunder \\
\poemll    from the waters of heaven, \\
\poeml and he makes clouds rise up from \\
\poemll    the ends of the earth. \\
\poeml He makes lightning for the rain \\
\poemll    and brings wind out of his storehouses. \\
\poeml \v{14}Everyone is stupid\fnote{\fbackref{10:14} I.e. like a beast} and without knowledge. \\
\poemll    Every goldsmith is put to shame by his idols, \\
\poemlll       for his images are false.\fnote{\fbackref{10:14} Lit. \fbib{deception}} \\
\poeml There is no life\fnote{\fbackref{10:14} Or \fbib{breath}} in them. \\
\poeml \v{15}They're worthless, a work of mockery, \\
\poemll    and when the time of punishment comes,\fnote{\fbackref{10:15} Lit. \fbib{at the time of their punishment}} \\
\poemlll       they'll perish. \\
\poeml \v{16}The Portion of Jacob\fnote{\fbackref{10:16} I.e. \fbib{The Portion of Jacob} is a name for the \fbib{\divine{Lord}}} is not like these. \\
\poemll    He made everything, \\
\poeml and Israel is the tribe of his inheritance. \\
\poemll    The \divine{Lord} of the Heavenly Armies is his name.
\passage{The Coming Captivity of Judah}
\poeml \v{17}You who live under siege, \\
\poemll    Gather up your bundle\fnote{\fbackref{10:17} I.e. your possessions} from the ground.\fnote{\fbackref{10:17} Or \fbib{land}} \\
\poeml \v{18}For this is what the \divine{Lord} says: \\
\poeml ``I'm going to throw out the inhabitants \\
\poemll    of the land at this time, \\
\poeml and I'll bring distress on them \\
\poemll    so they'll experience\fnote{\fbackref{10:18} Lit. \fbib{find}} it.'' \\
\poeml \v{19}Woe is me because of my injury. \\
\poemll    My wound is severe. \\
\poeml I said, ``Truly this is my sickness, \\
\poemll    and I must bear it. \\
\poeml \v{20}My tent is destroyed, \\
\poemll    and all my tent cords are broken. \\
\poeml My sons have gone away from me, \\
\poemll    they no longer live. \\
\poeml There is no one to pitch my tent again \\
\poemll    and set up my curtains. \\
\poeml \v{21}Because the shepherds are stupid\fnote{\fbackref{10:21} I.e. like a beast} \\
\poemll    and don't seek\fnote{\fbackref{10:21} Or \fbib{inquire of}} the \divine{Lord}, \\
\poeml therefore, they don't prosper, \\
\poemll    and their flock is scattered. \\
\poeml \v{22}The sound of a report, it's coming now! \\
\poemll    There is a great commotion from a land in the north \\
\poeml to make the towns of Judah desolate, \\
\poemll    a refuge for jackals.''
\passage{Jeremiah's Prayer}
\poeml \v{23}\divine{Lord}, I know that a person's life is not his to control,\fnote{\fbackref{10:23} Or \fbib{does not belong to him}} \\
\poemll    nor does a person establish his way in life.\fnote{\fbackref{10:23} Or \fbib{step as he walks}} \\
\poeml \v{24}\divine{Lord}, correct me, but with justice, \\
\poemll    not with anger. \\
\poemlll       Otherwise, you'll bring me to nothing. \\
\poeml \v{25}Pour out your anger on the nations \\
\poemll    that don't acknowledge you, \\
\poemlll       and on the families that don't call on your name. \\
\poeml For they have devoured Jacob; \\
\poemll    they have devoured and consumed him; \\
\poemlll       they have devastated his habitation.
\end{poetry}
\labelchapt{11}
\passage{The Broken Covenant}

\chapt{11}
\v{1}This is the message that came to Jeremiah from the \divine{Lord}: \v{2}``Listen to the words of this covenant, and convey them to the people of Judah and the residents of Jerusalem. \v{3}You are to say to them, `This is what the \divine{Lord} God of Israel says: ``Cursed is the person who does not listen to the words of this covenant \v{4}which I commanded to your ancestors on the day I brought them out of the land of Egypt, from the iron furnace. I said, `Obey me and do everything\fnote{\fbackref{11:4} Lit. \fbib{according to all}} that I commanded you. Then you will be my people and I'll be your God.' \v{5}As a result, I'll fulfill the oath that I made with your ancestors to give them a land flowing with milk and honey, just as is the case today.''\,'\,''

Then I answered, ``So be it,\fnote{\fbackref{11:5} Heb. \fbib{amen}} \divine{Lord}.''

\v{6}The \divine{Lord} told me, ``Proclaim all these words in the towns of Judah and in the streets of Jerusalem. You are to say, `Listen to the words of this covenant and do them. \v{7}For I've diligently warned your ancestors from the day I brought them out of the land of Egypt until now, regularly warning them,\fnote{\fbackref{11:7} Lit. \fbib{getting up early and warning}} saying, ``Obey me!'' \v{8}But they would not listen or turn their ear, and each of them stubbornly followed his own evil desires.\fnote{\fbackref{11:8} Lit. \fbib{follow the stubbornness of their evil hearts}} So I brought on them all the consequences\fnote{\fbackref{11:8} Lit. \fbib{words}} of this covenant that I commanded them to fulfill, but they did not.'\,''

\v{9}The \divine{Lord} told me, ``Conspiracy has been found among the people of Judah and the residents of Jerusalem. \v{10}They have turned back to the iniquities of their ancestors of old\fnote{\fbackref{11:10} Lit. \fbib{their first ancestors}} who refused to listen to my words. They followed other gods to serve them. The house of Israel and the house of Judah broke my covenant which I made with their ancestors.''

\v{11}Therefore, this is what the \divine{Lord} says: ``I'm about to bring disaster on them from which they won't be able to escape. They'll cry out to me, but I won't listen to them. \v{12}The towns of Judah and the residents of Jerusalem will go and cry out to the gods to whom they burn incense, but they'll be no help at all to them\fnote{\fbackref{11:12} Or \fbib{won't save them at all}} in the time of their disaster. \v{13}Judah, you have as many gods as you have towns, and you have set up as many altars to the shameful idols as there are streets in Jerusalem. You burn incense to Baal on these altars.

\v{14}``Jeremiah,\fnote{\fbackref{11:14} Lit. \fbib{You}} don't pray for this people and don't cry or pray for them. I won't listen when they cry out to me because of their disaster.

\begin{poetry}
\poeml \v{15}``What right does my beloved have in my house, \\
\poemll    when she has carried out many evil schemes? \\
\poeml Can sacrificial\fnote{\fbackref{11:15} Lit. \fbib{holy}} flesh turn disaster away from you, \\
\poemll    so you can rejoice?'' \\
\poeml \v{16}The \divine{Lord} once called you a green olive tree, \\
\poemll    with beautiful shape and fruit. \\
\poeml With a great roaring sound, he has set fire to it \\
\poemll    and its branches will be destroyed.
\end{poetry}

\v{17}The \divine{Lord} of the Heavenly Armies who planted you has called for disaster on you because of the evil of the house of Israel and the house of Judah, has provoked me by burning incense to Baal.''
\passage{Jeremiah's Life is Threatened}

\begin{poetry}
\poeml \v{18}The \divine{Lord} made it known to me, \\
\poemll    and so I understood. \\
\poemlll       Then you showed me their malicious deeds. \\
\poeml \v{19}I was like a gentle lamb \\
\poemll    led to the slaughter. \\
\poeml I didn't know that they had devised schemes \\
\poemll    against me. They told themselves,\fnote{\fbackref{11:19} The Heb. lacks \fbib{They told themselves}} \\
\poeml ``Let's destroy the tree with its fruit. \\
\poemll    Let's eliminate him from the land of the living, \\
\poemlll       so his name won't be remembered again.'' \\
\poeml \v{20}\divine{Lord} of the Heavenly Armies, the righteous judge, \\
\poemll    the one who tests feelings and the heart, \\
\poeml let me see your vengeance on them, \\
\poemll    for I've committed my cause to you.
\end{poetry}

\v{21}Therefore, this is what the \divine{Lord} says about the men of Anathoth who seek to kill you, all the while threatening you, ``Don't prophesy in the name of the \divine{Lord} so you won't die by our hand!'' \v{22}Therefore, this is what the \divine{Lord} of the Heavenly Armies says: ``I'm about to punish them. The young men will die by the sword. Their sons and daughters will die by famine. \v{23}Not one of them will be left,\fnote{\fbackref{11:23} Lit. \fbib{A remnant won't be to them}} for I'll bring disaster on the men of Anathoth when I punish them.''
\labelchapt{12}
\passage{Jeremiah's Complaint about Justice}

\begin{poetry}
\poeml \chapt{12}
\v{1}You are righteous, \divine{Lord}, \\
\poeml even when I bring a complaint to you. \\
\poeml But I want to discuss justice with you. \\
\poemll    Why does the way of the wicked prosper, \\
\poemlll       while all who are treacherous are at ease? \\
\poeml \v{2}You plant them and they take root, \\
\poemll    they grow and bear fruit. \\
\poeml ``You are near to us,'' they say with their mouths, \\
\poemll    but the truth is that you're far from their hearts. \\
\poeml \v{3}You know me, \divine{Lord}. \\
\poemll    You see me and test my thoughts\fnote{\fbackref{12:3} Or \fbib{heart}} toward you. \\
\poeml Pull the wicked\fnote{\fbackref{12:3} Lit. \fbib{them}} out like sheep for slaughter; \\
\poemll    set them apart for the day of butchering.\fnote{\fbackref{12:3} Or \fbib{killing}} \\
\poeml \v{4}How long will the land mourn \\
\poemll    and the vegetation of every field dry up? \\
\poeml Because of the wickedness of those who live in it, \\
\poemll    animals and birds are swept away. \\
\poemlll       For they say, ``He does not see our future.''
\passage{God's Reply to Jeremiah}
\poeml \v{5}Indeed, if you run with others on foot, \\
\poemll    and they tire you out, \\
\poemlll       how can you compete with horses? \\
\poeml You are secure\fnote{\fbackref{12:5} Or \fbib{You trust}} in a land at peace, \\
\poemll    but how will you do in the thicket of the Jordan? \\
\poeml \v{6}Indeed, even your brothers and your father's family \\
\poemll    conspire against you. \\
\poeml Even they cry out after you loudly. \\
\poemll    Don't believe them, even though they speak friendly words to you. \\
\poeml \v{7}I'll forsake my house, \\
\poemll    I'll abandon my inheritance. \\
\poeml I'll give the beloved of my heart \\
\poemll    into the hand of her enemies. \\
\poeml \v{8}My inheritance has become like a lion in the forest to me. \\
\poemll    She roars at me; therefore, I hate her. \\
\poeml \v{9}Is my inheritance like a speckled bird of prey to me? \\
\poemll    Are the other\fnote{\fbackref{12:9} The Heb. lacks \fbib{other}} birds of prey all around her coming against her? \\
\poeml Go, gather all the wild animals and \\
\poemll    bring them to devour it. \\
\poeml \v{10}Many shepherds will destroy my vineyard. \\
\poemll    They'll trample down my portion. \\
\poeml They'll turn my pleasant portion \\
\poemll    into a desolate desert. \\
\poeml \v{11}They'll make it into a desolate place, \\
\poemll    and, desolate, it will cry out in mourning to me. \\
\poeml The whole land will be desolate \\
\poemll    because no one takes it to heart. \\
\poeml \v{12}On all the barren heights in the desert \\
\poemll    destroyers will come. \\
\poeml Indeed, a sword of the \divine{Lord} will devour from \\
\poemll    one end of the land to the other. \\
\poeml There will be no peace\fnote{\fbackref{12:12} Or \fbib{safety}} for any person.\fnote{\fbackref{12:12} Lit. \fbib{for all flesh}} \\
\poeml \v{13}They have sown wheat, \\
\poemll    but they have harvested thorns. \\
\poeml They have tired themselves out, \\
\poemll    but they don't show a profit. \\
\poeml Now be disappointed about your harvest \\
\poemll    because of the fierce anger of the \divine{Lord}.
\end{poetry}
\passage{God's Word about Judah's Neighbors}

\v{14}This is what the \divine{Lord} says about all the wicked neighbors who strike out against the land\fnote{\fbackref{12:14} Lit. \fbib{inheritance}} I've given to my people Israel as their inheritance:\fnote{\fbackref{12:14} Lit. \fbib{to inherit}} ``I'm about to uproot them from their land, and I'll uproot the house of Judah from among them. \v{15}After I've uprooted them, I'll again have compassion on them. I'll return each one of them to his inheritance, and each one to his own land. \v{16}If they have learned the ways of my people well, to swear by my name: `As surely as the \divine{Lord} lives,' just as they once taught my people to swear by Baal, then they'll be built up among my people. \v{17}But if they don't listen, then I'll completely uproot that nation and destroy it,'' declares the \divine{Lord}.
\labelchapt{13}
\passage{Jeremiah's Linen Belt}

\chapt{13}
\v{1}This is what the \divine{Lord} told me: ``Go and buy a linen belt for yourself, and put it around your waist.\fnote{\fbackref{13:1} Or \fbib{loins}} But don't let it get wet.'' \v{2}So I bought the belt according to the \divine{Lord}'s instruction, and put it around my waist.

\v{3}Then this message from the \divine{Lord} came to me a second time: \v{4}Take the belt that you bought and that is around your waist. Get up and go to the Euphrates,\fnote{\fbackref{13:4} Or \fbib{Perath}} and hide it there in a crevice in the rock.'' \v{5}So I went and hid it at the Euphrates,\fnote{\fbackref{13:5} Or \fbib{Perath}} just as the \divine{Lord} had commanded me.

\v{6}After a long time,\fnote{\fbackref{13:6} Lit. \fbib{At the end of many days}} the \divine{Lord} told me, ``Arise, go to the Euphrates,\fnote{\fbackref{13:6} Or \fbib{Perath}} and get the belt that I commanded you to hide there.'' \v{7}I went to the Euphrates and dug it up. I got the belt from the place where I had hidden it. The belt was ruined! It was not good for anything.

\v{8}Then this message from the \divine{Lord} came to me: \v{9}``This is what the \divine{Lord} says: `In the same way I'll ruin the pride of Judah and the pride of Jerusalem. \v{10}This evil people that refuses to listen to my words, that stubbornly pursues their own desires,\fnote{\fbackref{13:10} Lit. \fbib{that walks in the stubbornness of their hearts}} and that follows other gods to serve and worship them, will be like this belt that is not good for anything. \v{11}For just as the belt clings tightly to a person's waist, so I've made all the people\fnote{\fbackref{13:11} Lit. \fbib{house}} of Israel and all the people\fnote{\fbackref{13:11} Lit. \fbib{house}} of Judah cling tightly to me,' declares the \divine{Lord}. `I did this\fnote{\fbackref{13:11} The Heb. lacks \fbib{I did this}} so that they would be my people, name, praise, and glory. But they wouldn't listen.'
\passage{The Wineskins}

\v{12}``This is what you're to tell them: `This is what the \divine{Lord} God of Israel says: ``Every wineskin is to be filled with wine.''\,' When they say to you, `Don't we know very well that every wineskin is to be filled with wine?', \v{13}then say to them, `This is what the \divine{Lord} says: ``I'm about to make all the inhabitants of this land drunk---the kings who sit on David's throne, the priests, the prophets, and all the residents of Jerusalem. \v{14}I'll smash them against each other, even fathers against their sons,''\fnote{\fbackref{13:14} Lit. \fbib{fathers and sons together}} declares the \divine{Lord}. ``I'll have no pity, mercy, or compassion when I destroy them.''\,'\,''

\begin{poetry}
\poeml \v{15}Listen and pay attention!\fnote{\fbackref{13:15} Lit. \fbib{give ear}} \\
\poemll    Don't be proud, for the \divine{Lord} has spoken. \\
\poeml \v{16}Give glory to the \divine{Lord} your God \\
\poemll    before he brings darkness, \\
\poeml before your feet stumble on the \\
\poemll    mountains at twilight. \\
\poeml You hope for light, \\
\poemll    but he turns it into deep darkness. \\
\poemlll       He changes it into heavy gloom. \\
\poeml \v{17}If you don't listen, I'll cry secretly \\
\poemll    because of your pride. \\
\poeml My eyes will cry bitterly, flowing tears, \\
\poemll    because the \divine{Lord}'s flock has been taken captive. \\
\poeml \v{18}Say to the king and the queen mother,\fnote{\fbackref{13:18} I.e. the king's mother} \\
\poemll    ``Come take a lowly seat, \\
\poeml because your beautiful crowns have fallen off your heads.'' \\
\poeml \v{19}The towns in the Negev\fnote{\fbackref{13:19} I.e. the southern regions of the Sinai peninsula; cf. Josh 10:40} will be closed up, \\
\poemll    and there will be no one to open them. \\
\poeml All Judah will be taken into exile \\
\poemll    and be completely exiled. \\
\poeml \v{20}``Look up and see those who are coming from the north. \\
\poemll    Where is the flock that was given to you--- \\
\poemlll       your beautiful sheep? \\
\poeml \v{21}What will you say when the \divine{Lord}\fnote{\fbackref{13:21} Lit. \fbib{he}} \\
\poemll    appoints over you as your head \\
\poemlll       those whom you taught to be your allies?\fnote{\fbackref{13:21} I.e. the Babylonians} \\
\poeml Pain will seize you like that seizing a woman \\
\poemll    about to give birth, will it not? \\
\poeml \v{22}When you say to yourselves, \\
\poemll    `Why have all these things happened to me?' \\
\poeml It's because of the extent of your iniquity \\
\poemll    that your skirt has been lifted up, \\
\poemlll       and your heels have suffered violence.\fnote{\fbackref{13:22} I.e. you have been violated} \\
\poeml \v{23}Can an Ethiopian change his skin, \\
\poemll    or a leopard his spots? \\
\poeml Then you who are trained to do evil \\
\poemll    will also be able to do good. \\
\poeml \v{24}I'll scatter them like chaff \\
\poemll    blown away by a desert wind. \\
\poeml \v{25}``This is your fate, \\
\poemll    the portion I've measured out for you,'' \\
\poemlll       declares the \divine{Lord}, \\
\poeml ``because you have forgotten me \\
\poemll    and have trusted in false gods.\fnote{\fbackref{13:25} Or \fbib{deception}} \\
\poeml \v{26}I'll also pull your skirt up over your face, \\
\poemll    so your shame will be seen, \\
\poeml \v{27}I've seen your detestable behavior: \\
\poemll    your adulteries, your passionate neighing, \\
\poemlll       your lewd immorality on the hills in the field. \\
\poeml How terrible it will be for you, Jerusalem! \\
\poemll    You are unclean. How much longer will this go on?''
\end{poetry}
\labelchapt{14}
\passage{A Terrible Drought in the Land}

\chapt{14}
\v{1}This is\fnote{\fbackref{14:1} The Heb. lacks \fbib{This is}} this message from the \divine{Lord} that came\fnote{\fbackref{14:1} The Heb. lacks \fbib{that came}} to Jeremiah concerning the drought:

\begin{poetry}
\poeml \v{2}``Judah mourns, and her gates languish. \\
\poemll    The people\fnote{\fbackref{14:2} Lit. \fbib{They}} mourn for the land, \\
\poemll    and the cry of Jerusalem goes up. \\
\poeml \v{3}Their nobles send their young people for water. \\
\poemll    They go to the cisterns, but they find no water. \\
\poeml They return with their vessels empty. \\
\poemll    They're disappointed\fnote{\fbackref{14:3} Or \fbib{ashamed}} and dismayed, \\
\poemlll       and they cover their heads in shame.\fnote{\fbackref{14:3} The Heb. lacks \fbib{in shame}} \\
\poeml \v{4}The ground is cracked, \\
\poemll    because there has been no rain in the land. \\
\poeml The farmers are disappointed,\fnote{\fbackref{14:4} Or \fbib{ashamed}} \\
\poemll    and they cover their heads in shame.\fnote{\fbackref{14:4} The Heb. lacks \fbib{in shame}} \\
\poeml \v{5}Even the doe in the field gives birth \\
\poemll    and then abandons her young\fnote{\fbackref{14:5} The Heb. lacks \fbib{her young}} \\
\poemlll       because there is no grass. \\
\poeml \v{6}Wild donkeys stand on the barren hills. \\
\poemll    They pant for air like jackals. \\
\poeml Their eyesight fails \\
\poemll    because there is no vegetation.''
\passage{The People Cry for Help}
\poeml \v{7}\divine{Lord}, even though our iniquities testify against us, \\
\poemll    do something for the sake of your name. \\
\poeml Indeed, our apostasies are many, \\
\poemll    and we have sinned against you. \\
\poeml \v{8}Hope of Israel, \\
\poemll    its deliverer in time of trouble, \\
\poeml why are you like a stranger\fnote{\fbackref{14:8} Or \fbib{resident alien}} in the land, \\
\poemll    like a traveler who sets up his tent for a night? \\
\poeml \v{9}Why are you like a man taken by surprise, \\
\poemll    like a strong man who can't deliver? \\
\poeml You are among us, \divine{Lord}, \\
\poemll    and your name is the one by which we're called. \\
\poemlll       Don't abandon us!
\passage{God Responds to the Prophet}
\poeml \v{10}This is what the \divine{Lord} says to these people: \\
\poemll    ``Yes, they do love to wander, \\
\poemlll       and they haven't restrained their feet. \\
\poeml So the \divine{Lord} won't accept them now. \\
\poemll    He will remember their iniquity \\
\poemlll       and punish their sin.''
\end{poetry}

\v{11}Then the \divine{Lord} told me, ``Don't pray for the welfare of these people. \v{12}Although they fast, I won't listen to their cry, and although they offer burnt offerings and grain offerings, I won't accept them. Instead, I'll put an end to them with the sword, with famine, and with a plague.''

\v{13}Then I said, ``Ah, Lord GOD, look! The prophets are saying to them, `You won't see the sword and you won't experience famine. Rather, I'll give you lasting peace in this place.'\,''

\v{14}Then the \divine{Lord} told me, ``The prophets are prophesying lies\fnote{\fbackref{14:14} Or \fbib{deception}} in my name. I didn't send them, I didn't command them, and I didn't speak to them. They're proclaiming\fnote{\fbackref{14:14} Lit. \fbib{prophesying}} to you false visions, worthless predictions,\fnote{\fbackref{14:14} or \fbib{divination}} and the delusions of their own minds. \v{15}Therefore, this is what the \divine{Lord} says about the false prophets who prophesy in my name, `There will be no sword and famine in this land' (though I haven't sent them): `By the sword and by famine these prophets will be finished off! \v{16}The people to whom they have prophesied will be thrown out into the streets of Jerusalem because of the famine and the sword. There will be no one to bury them, their wives, their sons, or their daughters. I'll pour out on them the\fnote{\fbackref{14:16} Lit. \fbib{their}} judgment they deserve.'\,''\fnote{\fbackref{14:16} The Heb. lacks \fbib{they deserve}}

\v{17}``And deliver\fnote{\fbackref{14:17} Lit. \fbib{speak}} this message to them:

\begin{poetry}
\poeml `Let tears run down my face,\fnote{\fbackref{14:17} Lit. \fbib{Let my eyes run down with tears}} \\
\poemll    night and day, and don't let them stop, \\
\poeml because my virgin daughter---my people--- \\
\poemll    will be broken with a powerful blow, \\
\poemlll       with a severe wound. \\
\poeml \v{18}If I go out into the field, \\
\poemll    I see those slain by the sword! \\
\poeml If I go into the city, \\
\poemll    I see the ravages of the famine! \\
\poeml Indeed, both prophet and priest \\
\poemll    ply their trade in the land, \\
\poemlll       but they don't know anything.'\,''\fnote{\fbackref{14:18} The Heb. lacks \fbib{anything}}
\passage{The People Plead to the \divine{Lord}}
\poeml \v{19}Have you completely rejected Judah? \\
\poemll    Do you despise Zion? \\
\poeml Why have you struck us, \\
\poemll    so that there is no healing for us? \\
\poeml We hoped for peace, but no good came, \\
\poemll    for a time of healing, but there was only terror. \\
\poeml \v{20}We acknowledge, \divine{Lord}, our wickedness, \\
\poemll    the guilt of our ancestors. \\
\poeml Indeed, we have sinned against you. \\
\poeml \v{21}For the sake of your name\fnote{\fbackref{14:21} I.e. your reputation} don't despise us. \\
\poemll    Don't dishonor your glorious throne. \\
\poemlll       Remember, don't break your covenant with us! \\
\poeml \v{22}Can any of the worthless gods of the nations make it rain? \\
\poemll    Can the heavens themselves bring forth showers? \\
\poeml Aren't you the one who does this,\fnote{\fbackref{14:22} The Heb. lacks \fbib{who does this}} \\
\poemll    \divine{Lord} our God? \\
\poeml So we hope in you, \\
\poemll    for you are the one who does all these things.
\end{poetry}
\labelchapt{15}
\passage{The Destiny of the Judged}

\chapt{15}
\v{1}Then the \divine{Lord} told me, ``Even if Moses and Samuel were standing before me, I wouldn't be favorably disposed toward this people. Send them out of my presence! Let them go!

\v{2}``When they say to you, `Where can we go?', say to them, `This is what the \divine{Lord} says:

\begin{poetry}
\poeml ``Those destined for death, \\
\poemll    to death will go;\fnote{\fbackref{15:2} The Heb. lacks \fbib{will go}} \\
\poeml those destined for the sword, \\
\poemll    to the sword will go;\fnote{\fbackref{15:2} The Heb. lacks \fbib{will go}} \\
\poeml and those destined for captivity, \\
\poemll    to captivity will go.\fnote{\fbackref{15:2} The Heb. lacks \fbib{will go}}
\end{poetry}

\v{3}``I'll appoint four kinds of judgment for them,'' declares the \divine{Lord}: ``the sword to kill, the dogs to drag off, the birds of the sky to devour, and the animals of the land to destroy. \v{4}I'll make them a horrifying sight to all the kingdoms of the earth because of what Hezekiah's son Manasseh, king of Judah, did in Jerusalem.

\begin{poetry}
\poeml \v{5}``Who will have pity on you, Jerusalem, \\
\poemll    and who will grieve for you? \\
\poeml Who will go out of his way \\
\poemll    to ask about your welfare? \\
\poeml \v{6}You have deserted me,'' declares the \divine{Lord}. \\
\poemll    ``You keep going backward. \\
\poeml I'll reach out my hand and destroy you. \\
\poemll    I'm tired of showing compassion. \\
\poeml \v{7}I'll winnow\fnote{\fbackref{15:7} Winnowing is the process of separating husks from the grain.} them with a winnowing fork \\
\poemll    in the gates of the land. \\
\poeml I'll make them childless. \\
\poemll    I'll destroy my people, \\
\poemlll       for they didn't change their ways. \\
\poeml \v{8}I'll make their\fnote{\fbackref{15:8} Lit. \fbib{her}} widows more numerous \\
\poemll    than the sand of the sea. \\
\poeml At noontime I'll send a destroyer \\
\poemll    against the mother\fnote{\fbackref{15:8} Lit. \fbib{send against them}} of a young man. \\
\poeml I'll cause terror and anguish \\
\poemll    to come to her unexpectedly. \\
\poeml \v{9}``The woman who gave birth to seven will grow faint, \\
\poemll    her life will expire. \\
\poeml Her sun will set while it's still day. \\
\poemll    She will be disgraced and humiliated. \\
\poeml I'll kill the rest of them with swords \\
\poemll    in the presence of their enemies,'' \\
\poemlll       declares the \divine{Lord}.
\passage{Jeremiah's Complaint}
\poeml \v{10}How terrible for me, my mother, \\
\poemll    that you gave birth to me, \\
\poeml a man of strife and contention for the whole land! \\
\poemll    I've neither lent nor borrowed, \\
\poemlll       yet everyone curses me.
\end{poetry}
\passage{God's Answer to Jeremiah's Complaint}

\v{11}The \divine{Lord} said,

\begin{poetry}
\poeml ``Have I not set you free \\
\poemll    for a good purpose? \\
\poeml Have I not intervened for you with your enemies \\
\poemll    in times of trouble and times of distress? \\
\poeml \v{12}``Can anyone break iron--- \\
\poemll    iron from the north---or bronze? \\
\poeml \v{13}``I'll give away your wealth and your treasures \\
\poemll    as plunder, for free, \\
\poemlll       because of all your sins throughout your territory. \\
\poeml \v{14}I'll make you serve your enemies \\
\poemll    in a land you don't know, \\
\poeml for my anger has started a fire \\
\poemll    that will burn against you.''
\passage{Jeremiah's Revised Complaint}
\poeml \v{15}You are aware--- \\
\poemll    \divine{Lord}, remember me, \\
\poeml pay attention to me, \\
\poemll    and vindicate me in front of those who pursue me. \\
\poeml You are patient--- \\
\poemll    don't take me away. \\
\poemlll       Know that I suffer insult because of you! \\
\poeml \v{16}Your words were found, and I consumed them. \\
\poemll    Your words were joy and my hearts delight, \\
\poeml because I bear your name,\fnote{\fbackref{15:16} Lit. \fbib{your name is called on me}} \\
\poemll    \divine{Lord} God of the Heavenly Armies. \\
\poeml \v{17}I didn't sit in the company of those who have fun, \\
\poemll    and I didn't rejoice. \\
\poeml Because of your hand on me,\fnote{\fbackref{15:17} The Heb. lacks \fbib{on me}} I sat alone, \\
\poemll    for you filled me with indignation. \\
\poeml \v{18}Why is my pain unending and my wound incurable, \\
\poemll    refusing to be healed?
\passage{God's Answer to Jeremiah's Revised Complaint}
\poeml You are like a deceptive brook, \\
\poemll    whose waters cannot be depended on. \\
\poeml \v{19}Therefore, this is what the \divine{Lord} says: \\
\poemll    ``If you repent, I'll take you back \\
\poemlll       and you will stand before me. \\
\poeml If you speak what is worthwhile,\fnote{\fbackref{15:19} Lit. \fbib{if worthwhile things come out}} \\
\poemll    instead of what is worthless, \\
\poemlll       then you will be my spokesman.\fnote{\fbackref{15:19} Lit. \fbib{my mouth}} \\
\poeml People\fnote{\fbackref{15:19} Lit. \fbib{They}} will turn to you, \\
\poemll    but you aren't to turn to them. \\
\poeml \v{20}I'll make you a fortified wall of bronze to this people. \\
\poemll    They'll fight against you, \\
\poeml but they won't prevail against you, \\
\poemll    for I am with you to save you \\
\poemlll       and deliver you,'' \\
\poeml \v{21}So I'll deliver you from the hand of the wicked, \\
\poemll    and redeem you from the grasp of the ruthless.''
\end{poetry}
\labelchapt{16}
\passage{The \divine{Lord}'s Instruction to His Prophet}

\chapt{16}
\v{1}This message from the \divine{Lord} came to me: \v{2}``You are not to take a wife, nor are you to have sons or daughters in this place.''

\v{3}For this is what the \divine{Lord} says about the sons and daughters who are born in this place, about their mothers who give birth to them, and about their fathers who father them in this land: \v{4}``They'll die of deadly diseases. People won't mourn for them, nor will they be buried. They'll be dung on the surface of the ground, and they'll come to an end with the sword and with famine. Their dead bodies will be food for the birds of the sky and the animals of the land.''

\v{5}For this is what the \divine{Lord} says: ``Don't go to a house where there is mourning, don't go to lament, nor to express sorrow to them. For I've taken my peace away from this people,'' declares the \divine{Lord}, ``as well as gracious love and compassion. \v{6}Both the most and the least important people\fnote{\fbackref{16:6} Lit. \fbib{great and small}; i.e. adults and children} will die in this land, and they won't be buried. People won't mourn for them. They won't cut themselves,\fnote{\fbackref{16:6} A Canaanite mourning practice forbidden by Deut. 14:1} nor will they shave their heads for them.\fnote{\fbackref{16:6} A common mourning practice in the ancient world} \v{7}They won't break bread\fnote{\fbackref{16:7} The Heb. lacks \fbib{bread}} for the mourner to be consoled for the dead. They won't give anyone the cup of consolation to drink for his father or\fnote{\fbackref{16:7} The Heb. lacks \fbib{father or}} mother. \v{8}Don't go to a banquet to sit with people\fnote{\fbackref{16:8} Lit. \fbib{with them}} to eat and drink.'' \v{9}For this is what the \divine{Lord} of the Heavenly Armies, the God of Israel, says: ``In this place I'm about to bring an end to the sounds of happiness and rejoicing, the sounds of the bridegroom and the bride. I'll do it in front of your eyes and in your time.

\v{10}``When you speak all these words to this people, they'll say to you, `Why has the \divine{Lord} pronounced all this disaster against us? What is our iniquity, and what is the sin that we have committed against the \divine{Lord} our God?' \v{11}Then say to them, `It is because your ancestors abandoned me,' declares the \divine{Lord}. `They followed other gods, served them, worshipped them, abandoned me, and didn't keep my Law. \v{12}You have done even more evil than your ancestors, and each one of you is stubbornly following his own evil desires,\fnote{\fbackref{16:12} Lit. \fbib{following the stubbornness of their evil heart}} refusing to listen to me. \v{13}I'll throw you out of this land into a land neither you nor your ancestors have known. There you will serve other gods day and night, and I'll show you no favor.'

\v{14}``Therefore, days are coming,'' declares the \divine{Lord}, ``when it will no longer be said, `As surely as the \divine{Lord} lives, who brought up the Israelis from the land of Egypt.' \v{15}Rather it will be said, `As surely as the \divine{Lord} lives, who brought the Israelis up from the land of the north and from all the lands to which the \divine{Lord}\fnote{\fbackref{16:15} Lit. \fbib{he}} had banished them.' I'll bring them back to their land, which I gave to their ancestors.

\v{16}``I'm about to send many fishermen,'' declares the \divine{Lord}, ``and they'll catch them. Afterwards, I'll send for many hunters and they'll hunt for them on every mountain and hill and in the crevices of the rocks. \v{17}For I am watching all their ways; they are not hidden from my sight.\fnote{\fbackref{16:17} Lit. \fbib{from before me}} Their iniquity is not concealed from my eyes. \v{18}First I'll repay them double for their iniquity and their sin, because they have polluted my land with the dead bodies of their detestable images, and they have filled my inheritance with their abominations.''\fnote{\fbackref{16:18} Or \fbib{their abominable idols}}

\begin{poetry}
\poeml \v{19}\divine{Lord}, my strength and my stronghold, \\
\poemll    my refuge in a time of difficulty, \\
\poeml to you the nations will come, \\
\poemll    and from the ends of the earth they'll say, \\
\poeml ``Surely our ancestors inherited deception,\fnote{\fbackref{16:19} I.e. false gods or idols} \\
\poemll    things that are worthless, \\
\poemlll       and in which there is no profit.'' \\
\poeml \v{20}Can a person make a god for himself? \\
\poemll    They are not gods! \\
\poeml \v{21}Therefore, I'm about to make them understand; \\
\poemll    this time I'll make them understand \\
\poeml my power and strength, \\
\poemll    so they'll understand that my name is the \divine{Lord}.
\end{poetry}
\labelchapt{17}
\passage{Judah's Sin and Its Consequence}

\begin{poetry}
\poeml \chapt{17}
\v{1}The sin of Judah is engraved \\
\poeml with an iron stylus. \\
\poeml It is inscribed with a diamond point \\
\poeml on the tablet of their heart \\
\poemlll       and on the horns of their\fnote{\fbackref{17:1} Lit. \fbib{your}} altars. \\
\poeml \v{2}When their sons remember, \\
\poemll    they remember their altars\fnote{\fbackref{17:2} Lit. \fbib{it is their altars}} and their Asherah poles\fnote{\fbackref{17:2} I.e. sacred poles representing the goddess Asherah} \\
\poemlll       beside green trees on the high hills. \\
\poeml \v{3}My mountain in the field, your wealth and your treasures \\
\poemll    I'll give as spoil; \\
\poeml along with your high places as the price of your sin \\
\poemll    throughout your territory. \\
\poeml \v{4}You will let go of your inheritance \\
\poemll    which I gave you, \\
\poeml and I'll make you serve your enemies \\
\poemll    in a land that you don't know. \\
\poeml For with my anger you have started a fire \\
\poemll    that will burn forever.
\passage{Two Ways Contrasted}
\poeml \v{5}This is what the \divine{Lord} says: \\
\poeml ``Cursed is the person who trusts in mankind, \\
\poemll    who makes flesh his strength, \\
\poemlll       and whose heart turns away from the \divine{Lord}. \\
\poeml \v{6}He will be like a bush in the desert, \\
\poemll    and he won't see when good comes. \\
\poeml He will dwell in parched places in the wilderness,\fnote{\fbackref{17:6} Or \fbib{desert}} \\
\poemll    a land of salt, without inhabitants. \\
\poeml \v{7}Blessed is the person who trusts in the \divine{Lord}, \\
\poemll    making the \divine{Lord} his trust. \\
\poeml \v{8}He will be like a tree planted by the water \\
\poemll    that sends out its roots by a stream. \\
\poeml He won't fear when the heat comes, \\
\poemll    and his leaves will be green. \\
\poeml In a year of drought he won't be concerned, \\
\poemll    nor will he stop producing fruit.''
\passage{The Deceitfulness of the Human Heart}
\poeml \v{9}``The heart is more deceitful than anything. \\
\poemll    It is incurable--- \\
\poemlll       who can know it? \\
\poeml \v{10}I am the \divine{Lord} who searches the heart, \\
\poemll    who tests the inner depths \\
\poeml to give to each person \\
\poemll    according to what he deserves,\fnote{\fbackref{17:10} Lit. \fbib{according to his way}} \\
\poemlll       according to the fruit of his deeds. \\
\poeml \v{11}As a partridge gathers together eggs \\
\poemll    that it didn't lay, \\
\poeml so is a person who amasses wealth unjustly. \\
\poemll    In the middle of his life\fnote{\fbackref{17:11} Lit. \fbib{at half of his days}} it will leave him, \\
\poemlll       and in the end he will prove to be a fool.
\passage{The \divine{Lord}: The Hope of Israel}
\poeml \v{12}A glorious throne exalted from the beginning \\
\poemll    is the place of our sanctuary. \\
\poeml \v{13}\divine{Lord}, you are the hope of Israel; \\
\poemll    all who forsake you will be put to shame. \\
\poeml Those who turn aside from you\fnote{\fbackref{17:13} Lit. \fbib{me}} will be \\
\poemll    written in the dust,\fnote{\fbackref{17:13} Or \fbib{recorded in the underworld}} \\
\poeml because they have forsaken the \divine{Lord}, \\
\poemll    the spring of living water.
\passage{The Prophet's Call for Help and Justice}
\poeml \v{14}Heal me, \divine{Lord}, and I'll be healed; \\
\poemll    deliver me, and I'll be delivered, \\
\poemlll       because you are my praise. \\
\poeml \v{15}Look, they're saying to me, \\
\poemll    ``Where is the message from the \divine{Lord}? \\
\poemlll       Let it come about!'' \\
\poeml \v{16}I haven't run away from being your shepherd,\fnote{\fbackref{17:16} Lit. \fbib{from shepherding after you}} \\
\poemll    and I haven't longed for the day of sickness.\fnote{\fbackref{17:16} I.e. the day of judgment} \\
\poeml You know what comes out from my lips, \\
\poemll    it's open before you.\fnote{\fbackref{17:16} Lit. \fbib{it is in front of you}} \\
\poeml \v{17}Don't be a terror to me. \\
\poemll    You are my refuge in a day of trouble. \\
\poeml \v{18}Let those who pursue me be put to shame, \\
\poemll    but don't put me to shame. \\
\poeml Let them be terrified, \\
\poemll    but don't let me be terrified. \\
\poeml Bring the day of judgment\fnote{\fbackref{17:18} Or \fbib{disaster}} on them, \\
\poemll    and destroy them with double destruction!
\end{poetry}
\passage{A Test Case: Keeping the Sabbath}

\v{19}The \divine{Lord} told me, ``Go, stand in the gate of the people,\fnote{\fbackref{17:19} Lit. \fbib{gate of the sons of the people}} where the kings of Judah come in and go out, and in the other gates of Jerusalem as well. \v{20}Say to them, `Kings of Judah, all Judah, and all the residents of Jerusalem entering these gates, hear this message from the \divine{Lord}. \v{21}This is what the \divine{Lord} says: ``Be careful! On the Sabbath day, don't carry any load or bring anything through the gates of Jerusalem. \v{22}Don't bring any load out of your houses on the Sabbath day, nor are you to do any work. You are to consecrate\fnote{\fbackref{17:22} I.e. set it apart} the Sabbath day, just as I commanded your ancestors. \v{23}But they didn't listen, nor did they pay attention.\fnote{\fbackref{17:23} Lit. \fbib{incline the ear}} They were determined\fnote{\fbackref{17:23} Lit. \fbib{stiffened their necks}} not to listen and not to accept instruction.\fnote{\fbackref{17:23} Or \fbib{discipline}} \v{24}If you listen to me carefully,'' declares the \divine{Lord}, ``and don't bring a load through the gates of this city on the Sabbath day, and you consecrate the Sabbath day and don't do any work on it, \v{25}then kings and princes, sitting on the throne of David will come through the gates of this city. They, their princes, the men of Judah, and the residents of Jerusalem will come riding in chariots and on horses, and this city will be inhabited forever. \v{26}They'll come from the cities of Judah, from the places around Jerusalem, from the territory of Benjamin, from the Shephelah,\fnote{\fbackref{17:26} I.e. the verdant central lowlands of Israel; cf. Josh 10:40} from the hill country, and from the Negev,\fnote{\fbackref{17:26} I.e. the southern regions of the Sinai peninsula; cf. Josh 10:40} bringing burnt offerings, sacrifices, grain offerings, and incense, and bringing thanksgiving offerings to the \divine{Lord}'s Temple. \v{27}But if you don't listen to me, to consecrate the Sabbath day and not carry any load as you enter the gates of Jerusalem on the Sabbath day, then I'll start a fire in its gates. It will consume the palaces of Jerusalem and won't be extinguished.''\,'\,''
\labelchapt{18}
\passage{The Potter's House and the Ruined Vessel}

\chapt{18}
\v{1}The message that came to Jeremiah from the \divine{Lord}: \v{2}``Arise and go down to the potter's house, and there I'll allow you to hear my words.'' \v{3}So I went down to the potter's house, and there he was doing work at the potter's wheel. \v{4}But the vessel he was working on with the clay was ruined in the potter's hand. So he remade it into another vessel that seemed appropriate to him.

\v{5}Then this message from the \divine{Lord} came to me: \v{6}``Israel, can't I deal with you like this potter?'' declares the \divine{Lord}. ``Look, Israel, like clay in the potter's hand, so are you in my hand. \v{7}At one moment I may speak about a nation or a kingdom to uproot it, pull it down, or destroy it. \v{8}But if that nation about which I spoke turns from its evil way, I'll change my mind about the disaster that I had planned for it. \v{9}At another moment I may speak about a nation or kingdom to build it or plant it. \v{10}But if that nation does evil in my eyes by not obeying me, I'll change my mind about the good that I said I would bring on it.

\v{11}``Now say to the people of Judah and to the residents of Jerusalem, `This is what the \divine{Lord} says: ``Look, I'm designing a disaster just for you, and I'm making plans against you. Each one of you must repent from his evil way. Make your ways and deeds right.''\,' \v{12}But they'll say, `It's useless! We will follow our plans and each of us will pursue his own evil desires.'\fnote{\fbackref{18:12} Lit. \fbib{will do the stubbornness of his evil heart}}

\v{13}``Therefore, this is what the \divine{Lord} says:

\begin{poetry}
\poeml `Ask the nations. \\
\poemll    Who has ever heard of anything like this? \\
\poeml You have done a most horrible thing, \\
\poemll    virgin Israel. \\
\poeml \v{14}Does the snow of Lebanon \\
\poemll    ever vanish from its rocky slopes?\fnote{\fbackref{18:14} Or \fbib{Do rocks ever leave the slopes, or the snow from Lebanon?}} \\
\poeml Or does the cold water from a foreign land \\
\poemll    ever cease to flow? \\
\poeml \v{15}Yet my people have forgotten me, \\
\poemll    and they burn incense to worthless idols \\
\poeml that make them stumble in their journey \\
\poemll    on the ancient paths. \\
\poeml They walk on trails, \\
\poemll    on a way that is not built up. \\
\poeml \v{16}They make their land into a desolate place, \\
\poemll    an object of lasting scorn.\fnote{\fbackref{18:16} Lit. \fbib{hissing}} \\
\poeml All who pass by will be appalled \\
\poemll    and will shake their heads.\fnote{\fbackref{18:16} I.e. in surprise at the magnitude of the destruction} \\
\poeml \v{17}`Like the east wind, I'll scatter them \\
\poemll    before the enemy. \\
\poeml I'll show them my back and not my face, \\
\poemll    on the day of their downfall.'\,''
\end{poetry}
\passage{Jeremiah Reacts to the Plot against Him}

\v{18}Then they said, ``Come, let's make up a plot against Jeremiah. After all, the priest's instruction, the wise man's counsel, and the prophet's message won't be destroyed.\fnote{\fbackref{18:18} Lit. \fbib{perish}} So let's verbally attack him. Pay no attention to anything he says!''

\begin{poetry}
\poeml \v{19}\divine{Lord}, pay attention to me. \\
\poemll    Listen to the voice of my accusers! \\
\poeml \v{20}Should good be repaid with evil? \\
\poemll    Yet they have dug a pit to take my life.\fnote{\fbackref{18:20} Or \fbib{a pit for me}} \\
\poeml Remember! I stood before you \\
\poemll    and spoke good on their behalf \\
\poemlll       in order to turn your wrath away from them. \\
\poeml \v{21}Therefore, make their children undergo famine, \\
\poemll    and deliver them over to death in time of war.\fnote{\fbackref{18:21} Lit. \fbib{to the power of the sword.}} \\
\poeml May their women be childless widows! \\
\poemll    May their men be slaughtered!\fnote{\fbackref{18:21} Lit. \fbib{killed dead}} \\
\poeml May their young men be slain \\
\poemll    by the sword in battle! \\
\poeml \v{22}Let a cry be heard from their houses because you \\
\poemll    have brought a raiding party against them suddenly. \\
\poeml For they have dug a pit to capture me \\
\poemll    and have set\fnote{\fbackref{18:22} Or \fbib{hidden}} traps for my feet. \\
\poeml \v{23}But you, \divine{Lord}, know all their plots to kill me. \\
\poemll    Don't forgive their iniquity, \\
\poemll    and don't erase their sin from your sight. \\
\poeml Let them stumble before you. \\
\poemll    When it's time for you to be angry, act against them!
\end{poetry}
\labelchapt{19}
\passage{The Lesson of the Broken Jug}

\chapt{19}
\v{1}This is what the \divine{Lord} says: ``Go and buy a potter's clay jug. Take along\fnote{\fbackref{19:1} The Heb. lacks \fbib{along}} some of the elders of the people and some of the elders of the priests. \v{2}Go out to the Valley of Hinnom\fnote{\fbackref{19:2} Lit. \fbib{Valley of Hinnom's son}} at the entrance to the Potsherd Gate, and there proclaim the words that I'm telling you.

\v{3}``You are to say, `Hear this message from the \divine{Lord}, you kings of Judah and residents of Jerusalem!

```This is what the \divine{Lord} of the Heavenly Armies, the God of Israel, says: ``I'm about to bring a disaster on this place that will make the ears of all who hear about it tingle. \v{4}For they have forsaken me and have treated this place as foreign. In it they have burned incense to other gods that neither they, their ancestors, nor the kings of Judah knew. They have also filled this place with the blood of innocent people. \v{5}They built the high places\fnote{\fbackref{19:5} I.e. the places where Canaanite gods were worshipped} for Baal to burn their children in the fire as a burnt offering to Baal---something I didn't command, didn't say, nor did it ever enter my mind!

\v{6}`````Therefore, days are coming,'' declares the \divine{Lord}, ``when this place will no longer be called Topheth, or the Valley of Hinnom, but rather the Valley of Slaughter. \v{7}I'll shatter\fnote{\fbackref{19:7} Or \fbib{nullify}; MT word for \fbib{shatter} sounds like MT word for \fbib{jug} in v. 1} the counsel of Judah and Jerusalem in this place, and I'll make them fall by the sword before their enemies and at the hands of those seeking their lives. I'll give their dead bodies as food to the birds of the sky and to the animals of the land. \v{8}I'll make this city into a desolate place and an object of scorn.\fnote{\fbackref{19:8} Lit. \fbib{hissing}} All who pass by it will be astonished and will scoff\fnote{\fbackref{19:8} Lit. \fbib{hiss}; i.e. hissing was an expression of contempt} because of all its wounds. \v{9}I'll cause them to eat the flesh of their sons and daughters,\fnote{\fbackref{19:9} Lit. \fbib{sons and the flesh of their daughters}} and people will eat the flesh of their neighbors in the siege and in the distress to which their enemies and those seeking their lives will subject them.''\,'\,''

\v{10}``Then you are to break the jug in front of the men who have come with you, \v{11}and say to them, `This is what the \divine{Lord} of the Heavenly Armies says: ``In this same way I'll break this people and this city, just as someone breaks a potter's vessel which he then cannot put back together again. They'll bury corpses\fnote{\fbackref{19:11} The Heb. lacks \fbib{corpses}} in Topheth until there is no more room to bury anyone.\fnote{\fbackref{19:11} The Heb. lacks \fbib{anyone}} \v{12}This is what I'll do to this place and its residents,'' declares the \divine{Lord}, ``making this city like Topheth. \v{13}The houses of Jerusalem and the houses of the kings of Judah will be polluted like Topheth, as will be all the houses on whose roofs people\fnote{\fbackref{19:13} Lit. \fbib{they}} burned incense to all the host of heaven and poured out liquid offerings to other gods.''\,'\,''

\v{14}Then Jeremiah went from Topheth where the \divine{Lord} had sent him to prophesy. He stood in the courtyard of the \divine{Lord}'s Temple, saying to all the people, \v{15}``This is what the \divine{Lord} of the Heavenly Armies, the God of Israel, says: `I'm about to bring on this city and all its towns all the disaster that I declared against it because they were determined\fnote{\fbackref{19:15} Lit. \fbib{they stiffened their neck}} not to obey my message.'\,''
\labelchapt{20}
\passage{Jeremiah Denounced}

\chapt{20}
\v{1}When the priest Pashhur, Immer's son, who was the officer in charge\fnote{\fbackref{20:1} Lit. \fbib{the Nagid}; i.e. a senior officer entrusted with dual roles of operational oversight and administrative authority} of the \divine{Lord}'s Temple heard Jeremiah prophesying these words, \v{2}Pashhur struck Jeremiah the prophet and put him in the stocks that were at the upper Benjamin Gate of the Temple. \v{3}The next day, Pashhur released Jeremiah from the stocks, and Jeremiah told him, ``The \divine{Lord} has not named you Pashhur, but rather Magor-missabib.\fnote{\fbackref{20:3} The Heb. name \fbib{Magor-missabib} means \fbib{terror on every side}; cf. v. 10} \v{4}For this is what the \divine{Lord} says: `Look, I'm going to make you a terror to yourself and to all your loved ones. They'll fall by the sword of their enemies, and your eyes will see it. I'll give all Judah into the hand of the king of Babylon. He will take them into exile to Babylon, and he will execute them with swords. \v{5}I'll turn over all the wealth of this city, all its possessions, all its valuables, and all the treasures of the kings of Judah right into the hands of their enemies, and they'll plunder them, capture them, and take them to Babylon. \v{6}You, Pashhur, and all those living in your house will go into captivity. You will go to Babylon and there you will die. There you and all your loved ones\fnote{\fbackref{20:6} Or \fbib{friends, colleagues}} to whom you have falsely prophesied will be buried.'\,''
\passage{Jeremiah's Complaint to the \divine{Lord}}

\begin{poetry}
\poeml \v{7}You deceived me, \divine{Lord}, \\
\poemll    and I've been deceived. \\
\poeml You overpowered me, \\
\poemll    and you prevailed. \\
\poeml I've become a laughing stock all day long, \\
\poemll    and everyone mocks me. \\
\poeml \v{8}Indeed, as often as I speak, I cry out, \\
\poemll    and shout, ``Violence and destruction!'' \\
\poeml For this message from the \divine{Lord} has caused me \\
\poemll    constant\fnote{\fbackref{20:8} Lit. \fbib{all day long}} reproach and derision. \\
\poeml \v{9}When I say, ``I won't remember the \divine{Lord}\fnote{\fbackref{20:9} Lit. \fbib{him}}, \\
\poemll    nor will I speak in his name anymore, \\
\poeml then there is this burning fire in my heart. \\
\poemll    It is bound up in my bones, \\
\poeml I grow weary of trying to hold it in, \\
\poemll    and I cannot do it! \\
\poeml \v{10}Indeed, I hear many people whispering, \\
\poemll    ``Terror on every side.\fnote{\fbackref{20:10} I.e. in mockery of the prophet's statement \fbib{Magor-missabib} in v. 3} \\
\poeml Denounce him, let's denounce him!'' \\
\poemll    All my close friends watch my steps and say, \\
\poeml ``Perhaps he will be deceived, \\
\poemll    and we can prevail against him \\
\poemlll       and take vengeance on him.'' \\
\poeml \v{11}But the \divine{Lord} is with me like a fearsome warrior. \\
\poemll    Therefore, those who pursue me will stumble \\
\poemlll       and won't prevail. \\
\poeml They'll be put to great shame, \\
\poemll    when they don't succeed. \\
\poemlll       Their everlasting disgrace won't be forgotten. \\
\poeml \v{12}\divine{Lord} of the Heavenly Armies, \\
\poemll    who tests the righteous, \\
\poemll    who sees the inner motives\fnote{\fbackref{20:12} Lit. \fbib{the liver}} and the heart, \\
\poeml let me see you take vengeance on them, \\
\poemll    for I've committed my case to you. \\
\poeml \v{13}Sing to the \divine{Lord}, \\
\poemll    give praise to the \divine{Lord}! \\
\poeml For he saves the life of the poor \\
\poemll    from the hand of the wicked.
\passage{Jeremiah Curses the Day of His Birth}
\poeml \v{14}Let the day on which I was born be cursed. \\
\poemll    Don't let the day on which my mother gave birth to me be blessed. \\
\poeml \v{15}Cursed is the person who brought \\
\poemll    the good news to my father, \\
\poeml ``A baby boy has been born to you,'' \\
\poemll    making him very happy. \\
\poeml \v{16}May that man be like the cities that \\
\poemll    the \divine{Lord} overthrew without compassion. \\
\poeml Let him hear a cry in the morning, \\
\poemll    and a battle cry at noon, \\
\poeml \v{17}because he didn't kill me in the womb, \\
\poemll    so that my mother would have been my grave \\
\poemlll       and her womb forever pregnant. \\
\poeml \v{18}Why did I ever come out of the womb \\
\poemll    to see trouble and sorrow, \\
\poemlll       and to finish my life living in shame?
\end{poetry}
\labelchapt{21}
\passage{Zedekiah's Request for a Miracle}

\chapt{21}
\v{1}The message that came to Jeremiah from the \divine{Lord} when King Zedekiah sent to him Malchijah's son Pashhur and Maaseiah's son Zephaniah the priest: \v{2}Please inquire of the \divine{Lord} on our behalf, for Nebuchadnezzar king of Babylon is fighting against us. Perhaps the \divine{Lord} will do some of his miraculous acts\fnote{\fbackref{21:2} Lit. \fbib{according to all his miraculous acts}} for us, and Nebuchadnezzar\fnote{\fbackref{21:2} Lit. \fbib{he}} will depart from us.''

\v{3}Jeremiah told them, ``This is what you are to say to Zedekiah, \v{4}`This is what the \divine{Lord} God of Israel says: ``I'm about to turn against you the weapons of war that are in your hands and with which you are fighting the king of Babylon and the Chaldeans who are besieging you outside the walls. I'll gather them into the center of this city. \v{5}Because of my anger, wrath, and great fury, I'll fight against you myself with an outstretched hand and a strong arm. \v{6}I'll strike down the residents of this city, both people and animals, and they'll die from a terrible plague. \v{7}Afterwards,'' declares the \divine{Lord}, ``I'll give King Zedekiah of Judah, his officials,\fnote{\fbackref{21:7} Or \fbib{servants}} and the people---those who are left in this city from the plague, the sword, and the famine---into the control of Nebuchadnezzar king of Babylon, right into the hand of their enemies and the hand of those who want to kill them. He'll execute them with swords and won't pity them. He won't spare them, nor will he have compassion on them.''\,'

\v{8}``You are to say to this people, `This is what the \divine{Lord} says: ``I'm about to set before you the way of life and the way of death. \v{9}Whoever stays in this city will die by the sword, by famine, and by the plague. But whoever goes out and surrenders to the Chaldeans who are besieging you will live. He will save his life as a spoil of war.\fnote{\fbackref{21:9} I.e. his life will be spared} \v{10}Indeed, I'm firmly decided---I'm sending calamity to this city, not good,'' declares the \divine{Lord}. ``It will be given into the hand of the king of Babylon, and he will set it on fire.''\,'
\passage{The Guilt of Judah's King}

\v{11}``To the house of the king of Judah say, `Hear this message from the \divine{Lord}.

\begin{poetry}
\poeml \v{12}This is what the \divine{Lord} says, house of David: \\
\poeml ``Judge appropriately every morning, \\
\poemll    and deliver those who have been robbed \\
\poemlll       from the oppressor, \\
\poeml so my anger does not break out like fire \\
\poemll    and burn with no one to put it out \\
\poemlll       because of your evil deeds. \\
\poeml \v{13}``Look, I'm against you, \\
\poemll    city dwelling in the valley, \\
\poeml rock of the plain,'' \\
\poemll    declares the \divine{Lord}, \\
\poeml ``those of you who say, `Who can come down against us \\
\poemll    and who can enter our habitations?' \\
\poeml \v{14}But I'll punish you according to \\
\poemll    what you have done,''\fnote{\fbackref{21:14} Lit. \fbib{to the fruit of your deeds}} \\
\poemlll       declares the \divine{Lord}. \\
\poeml ``I'll start a fire in her forest, \\
\poemll    and it will consume everything around her.''\,'\,''
\end{poetry}
\labelchapt{22}
\passage{Instructions for the Kings of Judah}

\chapt{22}
\v{1}This is what the \divine{Lord} says: ``Go down to the house of the king of Judah and tell him this: \v{2}`Listen to this message from the \divine{Lord}, king of Judah, you who sit on the throne of David---you, your officials,\fnote{\fbackref{22:2} Or \fbib{your servants}} and your people who enter these gates. \v{3}This is what the \divine{Lord} says: ``Uphold justice and righteousness. Deliver from their oppressor those who have been robbed. Don't mistreat or do violence to the alien, the orphan, or the widow, or shed the blood of innocent people in this place. \v{4}Rather, carefully obey this message,\fnote{\fbackref{22:4} Or \fbib{do this thing}} and then kings sitting for David on his throne and riding in chariots and on horses will enter the gates of this house. The king will enter along with his officials\fnote{\fbackref{22:4} Lit. \fbib{house, he, his officials}} and his people. \v{5}But if you don't listen to these words, I swear,'' declares the \divine{Lord}, ``that this house will become a ruin.''\,'\,'' \v{6}For this is what the \divine{Lord} says about the house of the king of Judah,

\begin{poetry}
\poeml ``You are like Gilead to me, \\
\poemll    like the summit of Lebanon. \\
\poeml Yet I'll surely make you a desert, \\
\poemll    towns where no one lives. \\
\poeml \v{7}I'll appoint people to destroy you--- \\
\poemll    men with their weapons.
\end{poetry}

They'll cut down some of your choice cedars\fnote{\fbackref{22:7} I.e. a genus of coniferous evergreen in the family \fbib{Pinaceae}; and so throughout the book}

\begin{poetry}
\poemll    and incinerate them.
\end{poetry}

\v{8}``Many nations will pass by this city and say to one another, `Why did the \divine{Lord} do this to this great city?' \v{9}Then people\fnote{\fbackref{22:9} Lit. \fbib{they'll say}} will respond, `It is\fnote{\fbackref{22:9} The Heb. lacks \fbib{It is}} because they have forsaken the covenant of the \divine{Lord} their God and have bowed down to other gods and served them.'

\begin{poetry}
\poeml \v{10}``Don't cry for the dead \\
\poemll    or grieve for them. \\
\poeml Weep bitterly for the one going away, \\
\poemll    because he won't return again \\
\poemlll       nor see the land of his birth.
\end{poetry}

\v{11}``For this is what the \divine{Lord} says about Josiah's son Shallum,\fnote{\fbackref{22:11} Shallum (also known as Jehoahaz) succeeded his father Josiah, but was removed by the Egyptians after three months and exiled to Egypt.} king of Judah, who reigned in place of his father Josiah: `He went out from this place and won't return to it again. \v{12}He will die in the place where they exiled him, and he won't ever\fnote{\fbackref{22:12} The Heb. lacks \fbib{ever}} see this land again.'\,''
\passage{An Oracle against Jehoiakim}

\begin{poetry}
\poeml \v{13}``How terrible for him who builds his house \\
\poemll    without righteousness, \\
\poeml and its upper rooms without justice, \\
\poemll    who makes his neighbor work for nothing, \\
\poemll    and does not pay him his wage. \\
\poeml \v{14}How terrible for\fnote{\fbackref{22:14} The Heb. lacks \fbib{How terrible for}} him who says, `I'll build a large \\
\poemll    house for myself with spacious upper rooms, \\
\poeml who cuts out windows for it, \\
\poemll    paneling it with cedar and painting it red.' \\
\poeml \v{15}Are you a king because you try to outdo \\
\poemll    everyone with cedar? \\
\poeml Your father ate and drank and upheld \\
\poemll    justice and righteousness, did he not? \\
\poemlll       And then it went well for him. \\
\poeml \v{16}He judged the case of the poor and needy. \\
\poemll    And then it went well for him. \\
\poemlll       Isn't this what it means to know me? \\
\poeml \v{17}But your eyes and heart are on nothing but \\
\poemll    your dishonest gain, \\
\poeml shedding the blood of innocent people, \\
\poemll    and practicing oppression and extortion.''
\end{poetry}

\v{18}Therefore, this is what the \divine{Lord} says about Josiah's son Jehoiakim, king of Judah,

\begin{poetry}
\poeml ``They won't lament for him with these words:\fnote{\fbackref{22:18} The Heb. lacks \fbib{with these words}} \\
\poemll    `How terrible, my brother, \\
\poemlll       How terrible, my sister!' \\
\poeml They won't lament for him with these words:\fnote{\fbackref{22:18} The Heb. lacks \fbib{with these words}} \\
\poemll    `How terrible, lord, \\
\poemlll       How terrible, your\fnote{\fbackref{22:18} The Heb. lacks \fbib{your}} majesty!' \\
\poeml \v{19}He will receive\fnote{\fbackref{22:19} Lit. \fbib{be buried with}} a donkey's burial, \\
\poemll    dragged out and thrown outside the gates of Jerusalem.''
\passage{An Oracle against Jerusalem}
\poeml \v{20}Go up to Lebanon and cry out, \\
\poemll    to Bashan and lift up your voice. \\
\poeml Cry out from Abarim, for all your lovers\fnote{\fbackref{22:20} I.e. \fbib{your allies}} \\
\poemll    have been crushed. \\
\poeml \v{21}I spoke to you when you were secure,\fnote{\fbackref{22:21} Or \fbib{prosperous}} \\
\poemll    but you said, ``I won't listen!'' \\
\poeml This has been your way since your youth, \\
\poemll    for you haven't obeyed me. \\
\poeml \v{22}The wind will shepherd\fnote{\fbackref{22:22} I.e. round them up and blow them away} all your shepherds,\fnote{\fbackref{22:22} I.e. leaders} \\
\poemll    and your lovers\fnote{\fbackref{22:22} I.e. \fbib{your allies}} will go into exile. \\
\poeml Indeed, you will then be ashamed and humiliated \\
\poemll    because of all your wickedness. \\
\poeml \v{23}You who live in Lebanon, \\
\poemll    who build your nest in the cedars, \\
\poeml how you will groan when pains come upon you, \\
\poemll    pain like that of a woman giving birth.
\end{poetry}
\passage{An Oracle against Jehoiachin}

\v{24}``As certainly as I'm alive and living,'' declares the \divine{Lord}, ``even if Jehoiakim's son King Jehoiachin\fnote{\fbackref{22:24} Lit. \fbib{Coniah}} of Judah were a signet ring on my right hand, I would pull you off \v{25}and give you to those who are trying to kill you, whom you fear---that is, to King Nebuchadnezzar of Babylon and the Chaldeans. \v{26}I'll hurl you and the mother who gave birth to you into another land where you were not born, and there you will die. \v{27}As for the land to which you\fnote{\fbackref{22:27} Lit. \fbib{they}} want to return, you\fnote{\fbackref{22:27} Lit. \fbib{they}} won't return there!

\begin{poetry}
\poeml \v{28}``Is this man Jehoiachin\fnote{\fbackref{22:28} Lit. \fbib{Coniah}} a despised and shattered jar, \\
\poemll    a vessel no one wants? \\
\poeml Why were he and his descendants hurled away, \\
\poemll    thrown into a land that they didn't know? \\
\poeml \v{29}Land, land, land, \\
\poemll    listen to this message from the \divine{Lord}! \\
\poeml \v{30}This is what the \divine{Lord} says: \\
\poeml `Write this man off as childless, \\
\poemll    a man who does not prosper in his lifetime.\fnote{\fbackref{22:30} Lit. \fbib{in his days}} \\
\poeml None of his descendants will succeed \\
\poemll    in sitting on the throne of David, \\
\poemlll       or ever ruling in Judah again.'\,''
\end{poetry}
\labelchapt{23}
\passage{A Righteous King for God's People}

\chapt{23}
\v{1}``How terrible for the shepherds\fnote{\fbackref{23:1} I.e. leaders} who are destroying and scattering the sheep of my pasture!'' declares the \divine{Lord}. \v{2}Therefore, this is what the \divine{Lord} God of Israel says about the shepherds who are shepherding my people, ``You have scattered my flock and driven them away. You haven't taken care of them, and now I'm about to take care of you\fnote{\fbackref{23:2} I.e. in judgment} because of your evil deeds,'' declares the \divine{Lord}. \v{3}``I'll gather the remnant of my flock from all the countries where I've driven them, and bring them back to their pasture where they'll be fruitful and increase in numbers. \v{4}I'll raise up shepherds over them, and they'll shepherd them. My flock\fnote{\fbackref{23:4} Lit. \fbib{they}} will no longer be afraid or terrified, and none will be missing,'' declares the \divine{Lord}.

\begin{poetry}
\poeml \v{5}``The time is coming,'' declares the \divine{Lord}, \\
\poemll    ``when I'll raise up a righteous branch for David. \\
\poeml He will be a king who rules wisely, \\
\poemll    and he will administer justice and righteousness in the land. \\
\poeml \v{6}In his time\fnote{\fbackref{23:6} Lit. \fbib{days}} Judah will be delivered \\
\poemll    and Israel will dwell in safety. \\
\poeml This is the name by which he will be known: \\
\poemll    `The \divine{Lord} Our Righteousness.'
\end{poetry}

\v{7}``Therefore, the time is coming,'' declares the \divine{Lord}, ``when people will no longer say, `As surely as the \divine{Lord} lives who brought up the Israelis from the land of Egypt,' \v{8}but they'll say,\fnote{\fbackref{23:8} The Heb. lacks \fbib{they'll say}} `As surely as the \divine{Lord} lives who brought the descendants of the Israelis from the land of the north and from all the lands where I had driven them and brought them into the land.'\fnote{\fbackref{23:8} The Heb. lacks \fbib{the land}} Then they'll live in their own land.''
\passage{An Oracle about False Prophets}

\v{9}Concerning the prophets:

\begin{poetry}
\poeml My heart is broken within me, \\
\poemll    and all my bones shake. \\
\poeml I'm like a drunk man, \\
\poemll    like a person overcome with wine, \\
\poeml because of the \divine{Lord}, \\
\poemll    and because of his holy words. \\
\poeml \v{10}Indeed, the land is full of adulterers. \\
\poemll    Indeed, the land mourns because of the curse; \\
\poemlll       the pastures of the wilderness have dried up. \\
\poeml The adulterers'\fnote{\fbackref{23:10} Lit. \fbib{Their}} lifestyles are evil, \\
\poemll    and they use\fnote{\fbackref{23:10} The Heb. lacks \fbib{they use}} their strength for what\fnote{\fbackref{23:10} The Heb. lacks \fbib{for what}} is not right. \\
\poeml \v{11}Indeed, both priest and prophet are ungodly. \\
\poemll    Even in my house I find evil,'' declares the \divine{Lord}. \\
\poeml \v{12}Therefore their way will be slippery. \\
\poemll    They'll be driven out into the darkness, \\
\poemlll       where they'll fall. \\
\poeml For I'll bring disaster on them, \\
\poemll    the year of their judgment,'' \\
\poemlll       declares the \divine{Lord}. \\
\poeml \v{13}``Among the prophets of Samaria I saw a disgusting thing, \\
\poemll    for they prophesied by Baal \\
\poemlll       and led my people Israel astray. \\
\poeml \v{14}Among the prophets of Jerusalem I saw a horrible thing, \\
\poemll    for they commit adultery and live a lie. \\
\poeml They strengthen the hands of those who do evil, \\
\poemll    so that no one repents of his evil. \\
\poeml All of them are like Sodom to me, \\
\poemll    and her\fnote{\fbackref{23:14} I.e. Jerusalem's} residents like Gomorrah.''
\end{poetry}

\v{15}Therefore, this is what the \divine{Lord} God of the Heavenly Armies says about the prophets,

\begin{poetry}
\poeml ``I'm about to make them eat wormwood\fnote{\fbackref{23:15} \fbib{Wormwood} is a plant with an extremely bitter taste} \\
\poemll    and drink poisoned water, \\
\poeml because godlessness has spread from the \\
\poemll    prophets of Jerusalem throughout the land.'' \\
\poeml \v{16}This is what the \divine{Lord} of the Heavenly Armies says: \\
\poeml ``Don't listen to the words of the prophets \\
\poemll    who are prophesying to you; \\
\poemlll       they're giving you false hopes. \\
\poeml They declare visions from their own minds--- \\
\poemll    they don't come from the \divine{Lord}!\fnote{\fbackref{23:16} Lit. \fbib{not from the mouth of the \divine{Lord}}} \\
\poeml \v{17}They keep on saying to those who despise me, \\
\poemll    `The \divine{Lord} has said, ``You will have peace.''\,' \\
\poeml To all who stubbornly follow their own desires\fnote{\fbackref{23:17} Lit. \fbib{walk in the stubbornness of their heart}} they say, \\
\poemll    `Disaster won't come upon you.' \\
\poeml \v{18}But who has stood in the \divine{Lord}'s council \\
\poemll    to see and hear his message? \\
\poemlll       Who has paid attention to his message and obeyed it?\fnote{\fbackref{23:18} Or \fbib{listened to}} \\
\poeml \v{19}Look, the storm of the \divine{Lord}'s wrath has gone forth, \\
\poemll    a whirling tempest, \\
\poeml and it will swirl down \\
\poemll    around the head of the wicked. \\
\poeml \v{20}The \divine{Lord}'s anger won't turn back \\
\poemll    until he has accomplished \\
\poemlll       what he intended to do. \\
\poeml In the future \\
\poemll    you will clearly understand it. \\
\poeml \v{21}I didn't send these prophets,\fnote{\fbackref{23:21} Lit. \fbib{prophets}} \\
\poemll    but they ran anyway. \\
\poeml I didn't speak to them, \\
\poemll    but they prophesied. \\
\poeml \v{22}If they had stood in my council \\
\poemll    and had delivered my words to my people, \\
\poeml then they would have turned them back \\
\poemll    from their evil way, \\
\poemlll       from their evil deeds.'' \\
\poeml \v{23}``Am I a God who is near,'' declares the \divine{Lord}, \\
\poemll    ``rather than a God who is far away? \\
\poeml \v{24}If a person hides himself in secret places, \\
\poemll    will I not see him?'' \\
\poemlll       declares the \divine{Lord}. \\
\poeml ``I fill the heavens and the earth, do I not?'' \\
\poemll    declares the \divine{Lord}.
\end{poetry}

\v{25}``I've heard what the prophets who prophesy lies in my name have said: `I had a dream; I had a dream.' \v{26}How long will this go on?\fnote{\fbackref{23:26} The Heb. lacks \fbib{will this go on}} Is there anything\fnote{\fbackref{23:26} The Heb. lacks \fbib{anything}} in the hearts of the prophets who prophesy lies, and who prophesy from the deceit that is in their hearts? \v{27}With their dreams that they relate to one another,\fnote{\fbackref{23:27} Lit. \fbib{each to his colleague}} they plan to make my people forget my name just as their ancestors forgot my name by embracing\fnote{\fbackref{23:27} The Heb. lacks \fbib{embracing}} Baal. \v{28}Let the prophet who has a dream relate the dream, but let whoever receives my message\fnote{\fbackref{23:28} Lit. \fbib{my word is with him}} speak my message truthfully. What does straw have in common with wheat?'' declares the \divine{Lord}. \v{29}``My message is like fire or like a hammer that shatters rock, is it not?'' declares the \divine{Lord}.

\v{30}``Therefore, look, I'm against the prophets,'' declares the \divine{Lord}, ``who steal my words from each other. \v{31}Look, I'm against the prophets,'' declares the \divine{Lord}, ``who use their tongues to issue a declaration.\fnote{\fbackref{23:31} I.e. a message that they claim came from God} \v{32}Look, I'm against those who prophesy based on false dreams,'' declares the \divine{Lord}, ``and relate them and lead my people astray with their lies and their recklessness. I didn't send them; I didn't command them, and they provide no benefit at all to these people,'' declares the \divine{Lord}.
\passage{The Oracle-Burden\fnote{\fbackref{23:33} An \fbib{oracle} is a message that claims to be a revelation from the \fbib{\divine{Lord}}. The same Heb. word means both \fbib{oracle} and \fbib{burden}, and this entire section invokes a word play between the two meanings of this Heb. word.} of the \divine{Lord}}

\v{33}``Jeremiah,\fnote{\fbackref{23:33} The Heb. lacks \fbib{Jeremiah}} when these people, the prophet, or a priest ask you,\fnote{\fbackref{23:33} I.e. the prophet Jeremiah; MT is masculine sing.} `What is the oracle\fnote{\fbackref{23:33} Or \fbib{burden}} of the \divine{Lord}?' say to them, `You are the burden,\fnote{\fbackref{23:33} Or \fbib{oracle}; MT reads \fbib{What oracle?}} and I'll cast you out,'\,'' declares the \divine{Lord}. \v{34}``As for the prophet, the priest, or the people who say, `I have\fnote{\fbackref{23:34} The Heb. lacks \fbib{I have}} an oracle of the \divine{Lord},' I'll judge that person and his household. \v{35}This is what you should say to one another and among yourselves,\fnote{\fbackref{23:35} Lit. \fbib{each to his neighbor and each to his brother}} `What has the \divine{Lord} answered?' or `What has the \divine{Lord} said?' \v{36}But you are to no longer mention\fnote{\fbackref{23:36} Or \fbib{remember}} the oracle of the \divine{Lord}, because the oracle is only for the person to whom the \divine{Lord} gives his message,\fnote{\fbackref{23:36} Lit. \fbib{for the person of his word}} and you have overturned the words of the living God, the \divine{Lord} of the Heavenly Armies, our God. \v{37}This is what you should say to the prophet, `What has the \divine{Lord} answered?' or `What has the \divine{Lord} said?' \v{38}Since you're saying, `We have an oracle of the \divine{Lord},'\fnote{\fbackref{23:38} The Heb. lacks \fbib{of the \divine{Lord}}} therefore this is what the \divine{Lord} says: He will answer your message with this message, `Burden\fnote{\fbackref{23:38} Or \fbib{Oracle}} of the \divine{Lord},' and I'll send you away with these words: `Don't say, ``Oracle of the \divine{Lord}.''\,' \v{39}Therefore I'll surely forget you and cast you and the city I gave you and your ancestors out of my presence. \v{40}I'll bring on you everlasting reproach and everlasting humiliation that won't ever\fnote{\fbackref{23:40} The Heb. lacks \fbib{ever}} be forgotten.''
\labelchapt{24}
\passage{Two Baskets of Figs}

\chapt{24}
\v{1}After Nebuchadnezzar, king of Babylon, had taken Jehoiakim's son Jeconiah,\fnote{\fbackref{24:1} I.e. Jehoiachin} king of Judah, along with the officials\fnote{\fbackref{24:1} Or \fbib{princes}} of Judah, the craftsmen, and the smiths from Jerusalem into exile, and had brought them to Babylon, the \divine{Lord} showed me two baskets of figs placed right in front of the Temple of the \divine{Lord}. \v{2}One basket contained very good figs like the first figs that ripen on the tree. The other basket contained very bad figs that were too bad to be eaten. \v{3}The \divine{Lord} told me, ``What do you see?''

I replied, ``Figs. The good figs are very good, and the bad figs are very bad. They're too bad to be eaten.''

\v{4}Then this message from the \divine{Lord} came to me: \v{5}``This is what the \divine{Lord} God of Israel says: `Like these good figs, so I'll regard as good the exiles of Judah whom I sent from this place to the land of the Chaldeans. \v{6}I'll look at them with good intentions, and I'll bring them back to this land. I'll build them up. I won't tear them down; I'll plant them and not rip them up. \v{7}I'll give them the ability\fnote{\fbackref{24:7} Lit. \fbib{them a heart}} to know me, for I am the \divine{Lord}. They will be my people, and I will be their God when they return to me with all their heart.

\v{8}```Like the bad figs that are too bad to be eaten---for this is what the \divine{Lord} says---so I'll give up on Zedekiah king of Judah, along with his officials, the remnant of Jerusalem that is left in this land, and those living in the land of Egypt. \v{9}I'll make them into a horrifying sight to all the kingdoms of the earth; into a cause for contempt, into a byword, into a taunt, and into a curse in all the places to which I drive them. \v{10}I'll send the sword, famine, and plague against them until they're completely destroyed from the land which I gave them and their ancestors.'\,''
\labelchapt{25}
\passage{The Irrevocable Judgment on Judah}

\chapt{25}
\v{1}This message from the \divine{Lord} came to Jeremiah concerning all the people of Judah in the fourth year of Josiah's son Jehoiakim, king of Judah. (This was also the first year of the reign of\fnote{\fbackref{25:1} The Heb. lacks \fbib{the reign of}} King Nebuchadnezzar of Babylon.) \v{2}This is what Jeremiah the prophet told all the people of Judah and all the residents of Jerusalem: \v{3}``From the thirteenth year of the reign of\fnote{\fbackref{25:3} The Heb. lacks \fbib{the reign of}} Ammon's son Josiah, the king of Judah, until the present time, for 23 years this message from the \divine{Lord} has come to me, and I've spoken to you again and again,\fnote{\fbackref{25:3} Lit. \fbib{getting up early and speaking}} but you haven't listened. \v{4}Again and again,\fnote{\fbackref{25:4} Lit. \fbib{getting up early and sending}} the \divine{Lord} sent all his servants, the prophets, to you, but you wouldn't listen or even turn your ears in my direction to hear. \v{5}They said, `Turn, each one of you, from your\fnote{\fbackref{25:5} Lit. \fbib{his}} evil habits\fnote{\fbackref{25:5} Lit. \fbib{ways}} and evil deeds, and live in the land that the \divine{Lord} gave to you and your ancestors forever and ever. \v{6}Don't follow other gods to serve and worship them. Don't provoke me with the idols\fnote{\fbackref{25:6} Lit. \fbib{works}} you make with your hands, and I won't bring disaster on you.' \v{7}But you didn't listen to me,'' declares the \divine{Lord}, ``so as to provoke me with the idols\fnote{\fbackref{25:7} Lit. \fbib{works}} you make with your hands to your own harm.

\v{8}``Therefore, this is what the \divine{Lord} of the Heavenly Armies says: `Because you haven't listened to my message, \v{9}I'm now sending for all the tribes from the north, declares the \divine{Lord}, and for my servant Nebuchadnezzar king of Babylon. I'll bring them against this land, against its inhabitants, and against all these surrounding nations. I'll utterly destroy them and make them an object of horror and scorn,\fnote{\fbackref{25:9} Lit. \fbib{hissing}; i.e. a sign of mocking and contempt} and an everlasting desolation. \v{10}I'll destroy the sounds of gladness and rejoicing from them, the sounds of the bridegroom and the bride, the sound of the hand mill and also the light of the lamp. \v{11}This entire land will be a desolation and a waste, and these nations will serve the king of Babylon for seventy years.

\v{12}`Then when the seventy years have passed, I'll judge the king of Babylon and that nation, declares the \divine{Lord}, I'll judge\fnote{\fbackref{25:12} The Heb. lacks \fbib{I'll judge}} the land of the Chaldeans for their iniquity and I'll make it a desolation forever. \v{13}I'll bring on that land all the things I spoke against it, all that is written in this book, which Jeremiah prophesied about the nations. \v{14}Indeed many nations and great kings will make slaves even of them, and I'll repay them according to their deeds, according to what they have done.'\,''
\passage{Judgment on the Nations}

\v{15}For this is what the \divine{Lord} God of Israel says to me, ``Take this cup of the wine of burning anger from my hand and make all the nations to whom I send you drink it. \v{16}They'll drink, stagger, and act like madmen because of the sword I'm sending among them.'' \v{17}So I took the cup from the \divine{Lord}'s hand, and I made all the nations to whom the \divine{Lord} sent me drink it: \v{18}Jerusalem, the cities of Judah, its kings and officials\fnote{\fbackref{25:18} Or \fbib{princes}} to make them into a ruin, an object of horror and scorn,\fnote{\fbackref{25:18} Lit. \fbib{hissing}; i.e. hissing was a sign of ridicule and contempt} and a curse, as it is this day; \v{19}Pharaoh, king of Egypt, his officials,\fnote{\fbackref{25:19} Or \fbib{servants}} his princes, and all his people; \v{20}all the various people;\fnote{\fbackref{25:20} Or \fbib{the mixed company}} all the kings of the land of Uz, all the kings of the land of the Philistines, Ashkelon, Gaza, Ekron, and what remains of Ashdod; \v{21}Edom, Moab, and the people of Ammon; \v{22}all the kings of Tyre, all the kings of Sidon, and all the kings of the coast lands that are beyond the sea; \v{23}Dedan, Tema, Buz, and those who shave the corners of their beards;\fnote{\fbackref{25:23} Lit. \fbib{cut off the side}} \v{24}all the kings of Arabia and all the kings of the various people\fnote{\fbackref{25:24} Or \fbib{the mixed company}} who live in the desert; \v{25}all the kings of Zimri, all the kings of Elam, and all the kings of Media; \v{26}all the kings of the north near and far, one after another, and all the kingdoms of the world on the face of the earth. The king of Sheshak\fnote{\fbackref{25:26} \fbib{Sheshak} is a cryptogram for Babylon} will drink after all the others.\fnote{\fbackref{25:26} Lit. \fbib{after them}}

\v{27}``You are to say to them, `This is what the \divine{Lord} of the Heavenly Armies, the God of Israel, says: ``Drink, get drunk, and vomit! Fall down and don't get up because of the sword I'm sending among you.''\,' \v{28}And if they refuse to take the cup from your hand to drink it, say to them, `This is what the \divine{Lord} of the Heavenly Armies says: ``You will surely drink it! \v{29}Look, I'm beginning to bring disaster on the city that is called by my name, and do you actually think you will avoid punishment? You won't avoid punishment because I'm summoning the sword against all those who live in the land,'' declares the \divine{Lord} of the Heavenly Armies.'\,''
\passage{The \divine{Lord} will Judge the Nations}

\v{30}``You are to prophesy all these things against them, and you are to say to them,

\begin{poetry}
\poeml `The \divine{Lord} roars from his high place, \\
\poemll    from his holy dwelling he lifts his voice. \\
\poeml He roars loudly against his flock,\fnote{\fbackref{25:30} Or \fbib{habitation}} \\
\poemll    and against all who live on the earth; \\
\poemlll       he shouts like those treading grapes.\fnote{\fbackref{25:30} The Heb. lacks \fbib{grapes}} \\
\poeml \v{31}A tumult reaches to the ends of the earth \\
\poemll    because the \divine{Lord} is bringing an indictment against the nations. \\
\poeml He judges all flesh. \\
\poemll    He has given the wicked over to the sword,' \\
\poemlll       declares the \divine{Lord}. \\
\poeml \v{32}`This is what the \divine{Lord} of the Heavenly Armies says: \\
\poemll    ``Look, disaster is going from nation to nation, \\
\poeml a great storm is being stirred up \\
\poemll    from the most distant parts of the earth.
\end{poetry}

\v{33}``Those slain by the \divine{Lord} on that day will extend\fnote{\fbackref{25:33} Lit. \fbib{will be}} from one end of the earth to the other. They won't be mourned for or gathered up or buried. They'll be like dung on the surface of the ground.

\begin{poetry}
\poeml \v{34}``Scream, you shepherds! Cry out! \\
\poemll    Roll in the dust, you leaders of the flock! \\
\poeml Indeed, the time for your slaughter \\
\poemll    and your dispersion has arrived, \\
\poemlll       and you will break like a choice vessel. \\
\poeml \v{35}Flight will be impossible\fnote{\fbackref{25:35} Lit. \fbib{will perish}} for the shepherds, \\
\poemll    as will be escape for the leaders of the flock. \\
\poeml \v{36}A sound---it's the cry of the shepherds \\
\poemll    and the scream of the leaders of the flock--- \\
\poemlll       because the \divine{Lord} is destroying their pastures. \\
\poeml \v{37}The peaceful meadows are silent \\
\poemll    because of the \divine{Lord}'s fierce anger. \\
\poeml \v{38}Like a lion, he has left his den.\fnote{\fbackref{25:38} Or \fbib{thicket}} \\
\poemll    Indeed, their land has become a waste \\
\poeml because of the anger of the oppressor \\
\poemll    and because of the \divine{Lord}'s\fnote{\fbackref{25:38} Lit. \fbib{his}} fierce anger.''
\end{poetry}
\labelchapt{26}
\passage{Jeremiah is Arrested}

\chapt{26}
\v{1}In the beginning of the reign of Josiah's son Jehoiakim, king of Judah, this message came from the \divine{Lord}: \v{2}``This is what the \divine{Lord} says: `Stand in the courtyard of the \divine{Lord}'s Temple and tell those from all the cities\fnote{\fbackref{26:2} Lit. \fbib{speak to all the cities}} of Judah who are coming to worship at the \divine{Lord}'s Temple everything that I've commanded you to say to them. Don't leave out a word! \v{3}Perhaps they'll listen, and each of them will repent from his evil way. Then I'll change my mind about the disaster I'm planning to bring on\fnote{\fbackref{26:3} Lit. \fbib{do to}} them because of their evil deeds. \v{4}Say to them, ``This is what the \divine{Lord} says: `If you don't listen to me to follow my Law which I've set before you, \v{5}and listen to the words of my servants, the prophets, whom I've sent to you over and over\fnote{\fbackref{26:5} Lit. \fbib{getting up early to send}}---but you wouldn't listen--- \v{6}then I'll make this house like Shiloh and make this city into a curse to all the nations of the earth.'\,''\,'\,''
\passage{Jeremiah Threatened with Death}

\v{7}The priests, the prophets, and all the people listened as Jeremiah spoke these words at the \divine{Lord}'s Temple. \v{8}As soon as Jeremiah finished saying everything that the \divine{Lord} had commanded him to say to all the people, the priests, the prophets, and all the people seized him, telling him as they did: ``You must certainly die! \v{9}Why have you prophesied in the name of the \divine{Lord} that this house will be like Shiloh, and this city will be without an inhabitant?'' Then all the people gathered around Jeremiah at the \divine{Lord}'s Temple.

\v{10}When the Judean officials\fnote{\fbackref{26:10} Or \fbib{princes}} heard all these things, they came up from the king's house to the \divine{Lord}'s Temple and sat in the doorway of the New Gate of the \divine{Lord}'s Temple.\fnote{\fbackref{26:10} The Heb. lacks \fbib{temple}} \v{11}The priests and prophets told the officials and all the people, ``A death sentence for this man, because he prophesied against this city, as you heard with your own ears!''

\v{12}Then Jeremiah spoke to all the officials and to all the people: ``The \divine{Lord} has sent me to prophesy all the things you heard against this house and against this city. \v{13}Now, change your habits\fnote{\fbackref{26:13} Lit. \fbib{ways}} and your deeds and obey the \divine{Lord} your God, and the \divine{Lord} will change his mind about the disaster that he told you about. \v{14}Look, I'm in your hands, so do with me what you think is good and right. \v{15}But know for certain that if you kill me, you will bring innocent blood on yourselves and on this city and its residents because the \divine{Lord} really did send me to you to say all these things for you to hear.''
\passage{The Elders Remember Micah's Similar Message}

\v{16}The officials and all the people told the priests and the prophets, ``No death sentence for this man because he has spoken to us in the name of the \divine{Lord} our God.''

\v{17}Some of the elders of the land got up and told all the assembled people, \v{18}``Micah of Moresheth prophesied during the reign\fnote{\fbackref{26:18} Lit. \fbib{time}} of Hezekiah king of Judah to all the people of Judah, `This is what the \divine{Lord} of the Heavenly Armies says:

\begin{poetry}
\poeml ``Zion will be a plowed field, \\
\poemll    and Jerusalem a ruin. \\
\poemlll       The Temple Mount will be a wooded hill.''\,'\fnote{\fbackref{26:18} Or \fbib{a wooded high place}}
\end{poetry}

\v{19}``Did Hezekiah king of Judah or anyone in Judah kill him? Didn't he fear the \divine{Lord} and seek the \divine{Lord}'s favor, and so the \divine{Lord} changed his mind about the disaster that he had spoken to them about. We're bringing great disaster on ourselves. \v{20}There was also a man named Uriah, Shemaiah's son from Kiriath-jearim, who prophesied in the \divine{Lord}'s name. He prophesied about this city and this land in words similar to those of Jeremiah. \v{21}King Jehoiakim, all his troops, and all the officials heard his words, and the king sought to kill him. Uriah heard about this and was afraid, so he fled and went to Egypt. \v{22}King Jehoiakim sent men to Egypt. He sent\fnote{\fbackref{26:22} The Heb. lacks \fbib{He sent}} Achbor's son Elnathan, along with a contingent of men\fnote{\fbackref{26:22} Lit. \fbib{Achbor and men with him}} into Egypt. \v{23}They brought Uriah out of Egypt and brought him to King Jehoiakim, who killed him with a sword. Then they threw his body into a common grave.\fnote{\fbackref{26:23} Lit. \fbib{a grave of the sons of the people}}''

\v{24}Yet because Shaphan's son Ahikam supported Jeremiah,\fnote{\fbackref{26:24} Lit. \fbib{the hand of Shaphan's son Ahikam was with Jeremiah}} he was not handed over to the people for them to kill.
\labelchapt{27}
\passage{Jeremiah Tells the Nations to Submit to Babylon}

\chapt{27}
\v{1}At the beginning of the reign of Josiah's son Jehoiakim, king of Judah, this message came to Jeremiah from the \divine{Lord}: \v{2}this is what the \divine{Lord} says to me: ``Make restraints and yokes for yourself and put them on your neck. \v{3}Then send messengers\fnote{\fbackref{27:3} Lit. \fbib{them}} to the king of Edom, the king of Moab, the king of the Ammonites, the king of Tyre, and the king of Sidon through the envoys\fnote{\fbackref{27:3} Or \fbib{messengers}} who come to Jerusalem to king Zedekiah of Judah. \v{4}Give them this order for their masters: `This is what the \divine{Lord} of the Heavenly Armies, the God of Israel, says, and this is what you are to say to your masters, \v{5}``By my great power and outstretched arm I made the earth, mankind, and the animals that are on the face of the earth, and I give it to whomever I see fit.\fnote{\fbackref{27:5} Or \fbib{to whoever is upright in my eyes}} \v{6}Now I've given all these lands to my servant, Nebuchadnezzar king of Babylon, and I've even given him the wild animals to serve him. \v{7}All the nations will serve him, his son, and his grandson until his country's time also comes, and then many nations and great kings will use him as a slave. \v{8}If a nation and kingdom does not serve him---King Nebuchadnezzar of Babylon---and does not put its neck under the yoke of the king of Babylon, I'll judge that nation with the sword, with famine, and with plague,'' declares the \divine{Lord}, ``until I've completely destroyed it by his hand. \v{9}You aren't to listen to your prophets, your diviners, your dreamers,\fnote{\fbackref{27:9} Lit. \fbib{your dreams}} your soothsayers, and your sorcerers who say to you, `Don't serve the king of Babylon.' \v{10}They're prophesying a lie to you in order to remove you far away from your land. I'll drive you out and you will perish. \v{11}But I'll let the nation that brings its neck under the yoke of the king of Babylon and serves him remain in its own land,'' declares the \divine{Lord}, ``and they'll work it and remain in it.''\,'\,''
\passage{Zedekiah Told to Submit to Babylon}

\v{12}I spoke to Zedekiah king of Judah using words like these: ``Bring your neck under the yoke of the king of Babylon. Serve him and his people, and you will live! \v{13}Why should you and your people die by the sword, by famine, and by plague as the \divine{Lord} has decreed about the nation that does not serve the king of Babylon? \v{14}Don't listen to the words of the prophets who say to you, `You won't serve the king of Babylon.' Indeed, they're prophesying a lie to you. \v{15}For I didn't send them,'' declares the \divine{Lord}, ``and they're falsely prophesying in my name, so I will drive both you and the prophets who prophesy to you out of the land.''
\passage{The People and Priests Told to Submit to Babylon}

\v{16}Then I spoke to the priests and all the people: ``This is what the \divine{Lord} says: `Don't listen to the words of the prophets who prophesy to you: ``The vessels from the Temple are about to be returned from Babylon very soon now.'' Indeed, they're prophesying a lie to you. \v{17}Don't listen to them! Serve the king of Babylon and you'll live. Why should this city become a ruin? \v{18}If they're prophets, and if they have a message from the \divine{Lord}, let them plead with the \divine{Lord} of the Heavenly Armies so that the utensils that remain in the \divine{Lord}'s Temple, in the house of the king of Judah, and in Jerusalem might not be taken to Babylon. \v{19}For this is what the \divine{Lord} of the Heavenly Armies says about the pillars, the bronze sea, the stands, and the rest of the vessels that remain in this city \v{20}that Nebuchadnezzar king of Babylon didn't take when he took Jehoiakim's son Jeconiah, king of Judah, and all the nobles of Judah and Jerusalem from Jerusalem into exile to Babylon--- \v{21}For this is what the \divine{Lord} of the Heavenly Armies, the God of Israel says about the vessels that remain in the \divine{Lord}'s Temple, in the house of the king of Judah, and in Jerusalem, \v{22}``They'll go into Babylon and there they'll remain until the time I take note of them,'' declares the \divine{Lord}. ``Then I'll bring them up and return them to this place.''\,'\,''
\labelchapt{28}
\passage{Jeremiah Challenges a False Prophet}

\chapt{28}
\v{1}In that same year, in the beginning of the reign of Zedekiah, king of Judah, in the fourth year and the fifth month, Azzur's son Hananiah, the prophet from Gibeon, told me at the \divine{Lord}'s Temple in front of the priests and all the people, \v{2}``This is what the \divine{Lord} of the Heavenly Armies, the God of Israel, says: `I've broken the yoke of the king of Babylon, \v{3}and within two years I'll bring back to this place all the vessels of the \divine{Lord}'s Temple that Nebuchadnezzar king of Babylon took from this place and carried to Babylon. \v{4}I'll bring back Jehoiakim's son Jeconiah, king of Judah, and all the exiles of Judah who went to Babylon to this place,' declares the \divine{Lord}, `for I'll break the yoke of the king of Babylon.'\,''

\v{5}The prophet Jeremiah spoke to the prophet Hananiah in front of the priests and all\fnote{\fbackref{28:5} Lit. \fbib{and in front of}} the people who were standing in the \divine{Lord}'s Temple. \v{6}The prophet Jeremiah said, ``May the \divine{Lord} truly do this thing! May the \divine{Lord} fulfill the words\fnote{\fbackref{28:6} Lit. \fbib{your words}} that you prophesied to bring back the vessels of the \divine{Lord}'s Temple and all the exiles from Babylon to this place. \v{7}But please listen to what I'm saying in your hearing and in the hearing of all the people. \v{8}The prophets who came before us\fnote{\fbackref{28:8} Lit. \fbib{before me and before you}} from ancient times prophesied war, famine, and plague against many lands and great kingdoms. \v{9}When a prophet prophesies peace, and what the prophet speaks comes about, he will be known as the prophet whom the \divine{Lord} has truly sent.''

\v{10}Then the prophet Hananiah took the yoke\fnote{\fbackref{28:10} Lit. \fbib{the bar of the yoke}} from the neck of Jeremiah the prophet and broke it. \v{11}Hananiah, in front of all the people, said, ``This is what the \divine{Lord} says: `In the same way, within two years, I'll break the yoke of Nebuchadnezzar king of Babylon from the neck of all the nations.'\,'' Then Jeremiah the prophet went on his way.

\v{12}This message from the \divine{Lord} came to Jeremiah after the prophet Hananiah had broken the yoke\fnote{\fbackref{28:12} Lit. \fbib{the bar of the yoke}} from the neck of Jeremiah the prophet: \v{13}``Go and say to Hananiah, `This is what the \divine{Lord} says: ``You have broken wooden yokes,\fnote{\fbackref{28:13} Lit. \fbib{the bars of the yoke}} but you have made iron yokes\fnote{\fbackref{28:13} Lit. \fbib{the bars of the yoke}} in their place.'' \v{14}For this is what the \divine{Lord} of the Heavenly Armies, the God of Israel, says: ``I've put an iron yoke on the necks of all these nations to serve Nebuchadnezzar king of Babylon. They'll serve him, and I've even given the wild animals to him.''\,'\,''

\v{15}The prophet Jeremiah told the prophet Hananiah, ``Listen, Hananiah! The \divine{Lord} didn't send you, and you are causing these people to trust in a lie. \v{16}Therefore, this is what the \divine{Lord} says: `I'm about to remove\fnote{\fbackref{28:16} Lit. \fbib{send you away}} you from the face of the earth. This year you will die because you have preached rebellion against the \divine{Lord}.'\,''

\v{17}So the prophet Hananiah died in the seventh month of that year.
\labelchapt{29}
\passage{Jeremiah's Letter to the Exiles}

\chapt{29}
\v{1}These are the words of the letter that the prophet Jeremiah sent from Jerusalem to the remaining elders among the exiles, to the priests, to the prophets, and to all the people whom Nebuchadnezzar had taken into exile from Jerusalem to Babylon, \v{2}after King Jeconiah, the queen mother, the palace officials,\fnote{\fbackref{29:2} Or \fbib{eunuchs}} the officials\fnote{\fbackref{29:2} Or \fbib{princes}} of Judah and Jerusalem, the craftsmen, and the smiths left Jerusalem. \v{3}The letter was sent by Shaphan's son Elasah and by Hilkiah's son Gemariah, whom King Zedekiah of Judah sent to Nebuchadnezzar king of Babylon in Babylon, and it said, \v{4}``This is what the \divine{Lord} of the Heavenly Armies, the God of Israel, says to all the exiles who were taken from Jerusalem into exile to Babylon, \v{5}`Build houses and live in them.\fnote{\fbackref{29:5} The Heb. lacks \fbib{in them}} Plant gardens and eat their produce. \v{6}Take wives and father sons and daughters. Take wives for your sons and give your daughters in marriage, so they may have sons and daughters. Increase in numbers there, don't decrease. \v{7}Seek the welfare of the city to which I've exiled you and pray to the \divine{Lord} for it, for your welfare depends on its welfare.'\fnote{\fbackref{29:7} Lit. \fbib{for in its welfare is your welfare}} \v{8}For this is what the \divine{Lord} of the Heavenly Armies, the God of Israel, says: `Don't let the prophets and diviners\fnote{\fbackref{29:8} Lit. \fbib{your prophets and your diviners}} who are among you deceive you, and don't listen to them when they tell you their dreams.\fnote{\fbackref{29:8} Lit. \fbib{to your dreams that you cause to be dreamed}} \v{9}Indeed, they're prophesying lies to you in my name. I didn't send them,' declares the \divine{Lord}.

\v{10}``For this is what the \divine{Lord} says: `When Babylon's seventy years are completed, I'll take note of you and will fulfill my good promises to you by bringing you back to this place. \v{11}For I know the plans that I have for you,' declares the \divine{Lord}, `plans for well-being, and not for calamity, in order to give you a future and a hope. \v{12}When you call out to me and come and pray to me, I'll hear you. \v{13}You will seek me and find me when you search for me with all your heart. \v{14}I'll be found by you,' declares the \divine{Lord}, `and I'll restore your security\fnote{\fbackref{29:14} Or \fbib{captivity}} and gather you from all the nations and all the places to which I've driven you,' declares the \divine{Lord}. `I'll bring you back to the place from which I sent you into exile.'

\v{15}``Indeed, you have said, `The \divine{Lord} has raised up prophets for us in Babylon.'

\v{16}``But this is what the \divine{Lord} says about the king who sits on David's throne, and about the people who live in this city---your brothers who didn't go with you into exile: \v{17}This is what the \divine{Lord} says: `I'm about to send the sword, famine, and plague on them, and I'll make them like rotten figs that cannot be eaten because they're so bad. \v{18}I'll pursue them with the sword, with famine, and with plague, and I'll make them a horrifying sight to all the kingdoms of the earth. I'll make them\fnote{\fbackref{29:18} The Heb. lacks \fbib{I'll make them}} a curse, an object of horror, and scorn,\fnote{\fbackref{29:18} Lit. \fbib{hissing}; i.e. hissing was a sign of mocking and contempt} and a desolation in all the nations to which I've driven them, \v{19}because they didn't listen to my words,' declares the \divine{Lord}. `When I sent my servants, the prophets, to you again and again,\fnote{\fbackref{29:19} Lit. \fbib{getting up early and sending}} you didn't listen,' declares the \divine{Lord}.

\v{20}``Now, all you exiles whom I sent from Jerusalem to Babylon, listen to this message from the \divine{Lord}! \v{21}This is what the \divine{Lord} of the Heavenly Armies, the God of Israel, says about Kolaiah's son Ahab and Maaseiah's son Zedekiah, who are prophesying lies to you in my name, `I'm about to give them into the domination\fnote{\fbackref{29:21} Lit. \fbib{hand}} of Nebuchadnezzar king of Babylon, and he will kill them before your eyes. \v{22}What happens to them will be the basis for a curse\fnote{\fbackref{29:22} Lit. \fbib{From them a curse will be taken}} for all the Judean exiles who are in Babylon. People will say,\fnote{\fbackref{29:22} Lit. \fbib{Saying}} ``May the \divine{Lord} make you like Zedekiah and Ahab, whom the \divine{Lord} roasted\fnote{\fbackref{29:22} MT word for \fbib{roasted} sounds like MT word for \fbib{curse}} in the fire, \v{23}because they did something stupid\fnote{\fbackref{29:23} Lit. \fbib{they committed folly}} in Israel. They committed adultery with their neighbors' wives, and in my name they spoke lies that I didn't command them. I'm the one who knows, and I'm a witness,'' declares the \divine{Lord}.'\,''
\passage{A Rebuke to Shemaiah}

\v{24}``You are to say to Shemaiah of Nehelam: \v{25}`This is what the \divine{Lord} of the Heavenly Armies, the God of Israel, says: ``Because you sent letters in your own name to all the people who are in Jerusalem, to Maaseiah's son Zephaniah the priest and to all the priests--- \v{26}The \divine{Lord} made you a priest instead of Jehoiada the priest to serve in the \divine{Lord}'s Temple as an official against every crazy prophet, and to put him in stocks and restraints. \v{27}And now, why didn't you rebuke Jeremiah from Anathoth who prophesies to you? \v{28}So he sent a message\fnote{\fbackref{29:28} The Heb. lacks \fbib{a message}} to us in Babylon: `The exile\fnote{\fbackref{29:28} Lit. \fbib{It}} will be long, so build houses and live in them.\fnote{\fbackref{29:28} The Heb. lacks \fbib{in them}} Plant gardens and eat their produce.'\,''\,'\,''

\v{29}Then Zephaniah the priest read this letter to Jeremiah the prophet, \v{30}and this message from the \divine{Lord} came to Jeremiah: \v{31}``Send a message to all the exiles: `This is what the \divine{Lord} says about Shemaiah from Nehelam, ``Because Shemaiah has prophesied to you, even though I didn't send him, and has made you trust a lie,'' \v{32}therefore, this is what the \divine{Lord} says: ``I'm about to judge Shemaiah from Nehelam along with his descendants. He won't have anyone related to him\fnote{\fbackref{29:32} The Heb. lacks \fbib{related to him}} living among these people. Nor will he see the good that I'll do for my people,'' declares the \divine{Lord}, ``because he advocated rebellion against the \divine{Lord}.''\,'\,''
\labelchapt{30}
\passage{A Message of Consolation}

\chapt{30}
\v{1}This message came from the \divine{Lord} to Jeremiah: \v{2}``This is what the \divine{Lord} God of Israel says: `Write all the words that I've spoken to you in a book. \v{3}Indeed, the time\fnote{\fbackref{30:3} Lit. \fbib{days}} will come,' declares the \divine{Lord}, `when I'll restore the security of my people Israel and Judah,' says the \divine{Lord}. `I'll bring them back to the land that I gave to their ancestors, and they'll possess it.'\,''

\v{4}These are the words that the \divine{Lord} spoke about Israel and Judah:

\begin{poetry}
\poeml \v{5}``Indeed, this is what the \divine{Lord} says: \\
\poeml `We have heard a sound of terror \\
\poemll    and of fear, and there is no peace. \\
\poeml \v{6}Ask about this and think about it--- \\
\poemll    Can a man give birth to a child? \\
\poeml Why then do I see every strong man \\
\poemll    with his hands on his thighs \\
\poeml like a woman giving birth, \\
\poemll    and all their faces have turned pale? \\
\poeml \v{7}Oh how terrible! That time\fnote{\fbackref{30:7} Lit. \fbib{day}} will be worse \\
\poemll    than any like it. \\
\poeml It will be a time of trouble for Jacob, \\
\poemll    but he will be rescued from it. \\
\poeml \v{8}On that day,' declares the \divine{Lord} \\
\poemll    of the Heavenly Armies, \\
\poeml `I'll break the yoke\fnote{\fbackref{30:8} Lit. \fbib{his yoke}} from your neck \\
\poemll    and will tear off your restraints.\fnote{\fbackref{30:8} Or \fbib{cords}} \\
\poemlll       Foreigners will no longer make you\fnote{\fbackref{30:8} Lit. \fbib{him} (i.e. Jacob)} serve them.\fnote{\fbackref{30:8} I.e. enslave you} \\
\poeml \v{9}Rather, they will serve the \divine{Lord} their God \\
\poemll    and David their king, \\
\poemlll       whom I will raise up for them. \\
\poeml \v{10}`My servant Jacob, don't be afraid,' declares the \divine{Lord}, \\
\poemll    `and Israel, don't be dismayed. \\
\poeml For I'll deliver you from a distant place \\
\poemll    and your descendants from the land of their captivity. \\
\poeml Jacob will return. He will be undisturbed and secure, \\
\poemll    and no one will cause him to fear. \\
\poeml \v{11}For I'll be with you to save you,' \\
\poemll    declares the \divine{Lord}. \\
\poeml `For I'll put an end to all the nations \\
\poemll    where I scattered you; \\
\poemlll       but I won't make an end of you. \\
\poeml I'll discipline you justly, \\
\poemll    but I certainly won't leave you unpunished.'
\passage{The Healing of Zion's Wounds}
\poeml \v{12}``For this is what the \divine{Lord} says: \\
\poeml `Your injury won't heal; \\
\poemll    your wound is severe. \\
\poeml \v{13}There is no one to plead your cause. \\
\poemll    There is no medicine for your sore;\fnote{\fbackref{30:13} Lit. \fbib{for a sore}} \\
\poemlll       no healing for you. \\
\poeml \v{14}All your lovers have forgotten you; \\
\poemll    they don't seek you. \\
\poeml Indeed, I've struck you down \\
\poemll    with the blow of an enemy, \\
\poemlll       with the punishment of a cruel foe\fnote{\fbackref{30:14} Lit. \fbib{cruel one}} \\
\poeml because your wickedness is great, \\
\poemll    and your sins are numerous. \\
\poeml \v{15}Why do you cry out because of your injury? \\
\poemll    Your wound won't heal. \\
\poeml Because your wickedness is severe, \\
\poemll    and your sins are numerous, \\
\poemlll       I've done all these things to you. \\
\poeml \v{16}In addition, all who devour you will be devoured, \\
\poemll    and all your oppressors---all of them--- \\
\poemlll       will go into captivity. \\
\poeml Those who plunder you will become plunder, \\
\poemll    and all who spoil you will become spoil. \\
\poeml \v{17}Indeed, I'll bring you healing, \\
\poemll    and I'll heal you of your wounds,' \\
\poemlll       declares the \divine{Lord}, \\
\poeml `because they have called you an outcast \\
\poemll    and have said,\fnote{\fbackref{30:17} The Heb. lacks \fbib{and have said}} ``It is Zion, no one cares for her!''\,'\,''\fnote{\fbackref{30:17} Or \fbib{seeks her}}
\passage{Jacob's Restoration}
\poeml \v{18}``This is what the \divine{Lord} says: \\
\poeml `I'm going to restore the fortunes of the tents of Jacob \\
\poemll    and have compassion on his dwellings. \\
\poeml A city will be rebuilt on its ruins \\
\poemll    and a palace\fnote{\fbackref{30:18} Or \fbib{fortress}} will sit on its rightful place. \\
\poeml \v{19}Thanksgiving and the sounds of laughter \\
\poemll    will come out of them. \\
\poeml I'll cause them to increase in numbers and not decrease. \\
\poemll    I'll honor them and not make them insignificant. \\
\poeml \v{20}Their\fnote{\fbackref{30:20} Lit. \fbib{his children} and so through v. 21} children will be as they were before, \\
\poemll    and their congregation will be established before me. \\
\poeml I'll punish all who oppress them. \\
\poeml \v{21}Their leader will be one of their own,\fnote{\fbackref{30:21} Lit. \fbib{of them}} \\
\poemll    and their ruler will come from among them. \\
\poeml I'll bring him near, and he will approach me, \\
\poemll    for who would otherwise dare to approach me?' \\
\poemlll       declares the \divine{Lord}. \\
\poeml \v{22}`You will be my people, \\
\poemll    and I'll be your God.'\,''
\passage{The Coming Judgment}
\poeml \v{23}Look, the storm of the \divine{Lord}! \\
\poemll    His\fnote{\fbackref{30:23} The Heb. lacks \fbib{His}} wrath has gone forth, a twisting storm. \\
\poemlll       It will swirl around the head of the wicked. \\
\poeml \v{24}The fierce anger of the \divine{Lord} won't turn back \\
\poemll    until he has accomplished and established the plan of his heart. \\
\poeml In the days to come, you will understand this.
\end{poetry}
\labelchapt{31}

\chapt{31}
\v{1}``At that time,'' declares the \divine{Lord}, ``I'll be the God of all the families of Israel, and they will be my people.''
\passage{The \divine{Lord} Promises Restoration}

\begin{poetry}
\poeml \v{2}This is what the \divine{Lord} says: \\
\poeml ``The people who survived the sword, \\
\poemll    found favor in the desert \\
\poemlll       while Israel was seeking rest.\fnote{\fbackref{31:2} Lit. \fbib{while going to find rest for him, Israel}; cf. Deut 28:65} \\
\poeml \v{3}The \divine{Lord} appeared to Israel\fnote{\fbackref{31:3} Lit. \fbib{to me}; i.e. here Jeremiah personifies Israel} from far away and said,\fnote{\fbackref{31:3} The Heb. lacks \fbib{and said}} \\
\poemll    ``I've loved you with an everlasting love, \\
\poemlll       therefore I've drawn you with gracious love. \\
\poeml \v{4}I'll again build you, and you will be rebuilt, \\
\poemll    Virgin Israel! \\
\poeml You will again take up your tambourines \\
\poemll    and go out to dance with those who are filled with joy. \\
\poeml \v{5}You will again plant vineyards on the hills of Samaria \\
\poemll    where planters had planted and defiled the fruit.\fnote{\fbackref{31:5} I.e. had used the fruit for inappropriate purposes} \\
\poeml \v{6}For there will be a day when the watchmen \\
\poemll    will call out on the hills of Ephraim, \\
\poemll    `Arise, let's go up to Zion to the \divine{Lord} our God.'\,''
\end{poetry}
\passage{The Blessings of Returning from Exile}

\begin{poetry}
\poeml \v{7}For this is what the \divine{Lord} says:
\end{poetry}

\begin{poetry}
\poeml ``Cry out with joy for Jacob \\
\poemll    and shout for the chief among the nations. \\
\poeml Announce, give praise, and say, \\
\poemll    `Lord, save your people, the remnant of Israel.' \\
\poeml \v{8}Look, I'm bringing them from the northern region,\fnote{\fbackref{31:8} Lit. \fbib{the land of the north}} \\
\poemll    and I'll gather them from the farthest parts of the earth. \\
\poeml among them will be the blind and the lame, \\
\poemll    together with the pregnant woman \\
\poemlll       and the woman in labor. \\
\poeml A large group will return here. \\
\poeml \v{9}They'll come crying, \\
\poemll    and I'll lead them as they pray for mercy.\fnote{\fbackref{31:9} Lit. \fbib{I'll lead them with prayer for mercy}} \\
\poeml I'll make them walk by streams of water, \\
\poemll    along a straight path on which they won't stumble. \\
\poeml For I am Israel's father, \\
\poemll    and Ephraim is my firstborn.'' \\
\poeml \v{10}Nations, listen to this message from the \divine{Lord}, \\
\poemll    and declare it in the distant coastlands. \\
\poeml Say, ``The one who scattered Israel will gather him \\
\poemll    and keep him as a shepherd keeps his flock.'' \\
\poeml \v{11}For the \divine{Lord} will deliver Jacob \\
\poemll    and redeem him from the hand of one stronger than he. \\
\poeml \v{12}They'll come and cry out with joy \\
\poemll    on the heights of Zion. \\
\poeml They'll be radiant over the \divine{Lord}'s goodness, \\
\poemll    over the grain, the new wine, the fresh oil, \\
\poemlll       and over the young of the flocks and herds. \\
\poeml Their lives will be like a well-watered garden. \\
\poemll    They'll never again grow faint.\fnote{\fbackref{31:12} I.e. from lack of food and drink} \\
\poeml \v{13}The virgins will rejoice with dancing, \\
\poemll    together with young men and old men. \\
\poeml For I'll turn their mourning into joy, \\
\poemll    and I'll comfort them and give them gladness \\
\poemlll       instead of sorrow. \\
\poeml \v{14}I'll give the priests abundant provisions,\fnote{\fbackref{31:14} Lit. \fbib{fatness}} \\
\poemll    and my people will be satisfied with my goodness,'' \\
\poemlll       declares the \divine{Lord}.
\passage{The End of Rachel's Mourning}
\poeml \v{15}This is what the \divine{Lord} says: \\
\poeml ``A voice is heard in Ramah, \\
\poemll    lamentation and bitter crying. \\
\poeml Rachel is crying, \\
\poemll    and she refuses to be comforted for her children, \\
\poemlll       because they are no longer alive.'' \\
\poeml \v{16}This is what the \divine{Lord} says: \\
\poeml ``Restrain your voice from crying, \\
\poemll    and your eyes from tears, \\
\poeml for there is a reward for your work,'' \\
\poemll    declares the \divine{Lord}. \\
\poemlll       ``They'll return from the enemy's land. \\
\poeml \v{17}There is hope for your future,'' \\
\poemll    declares the \divine{Lord}. \\
\poeml ``Your\fnote{\fbackref{31:17} The Heb. lacks \fbib{Your}} children will return to their own territory.''
\passage{Ephraim's Prayer and Confession}
\poeml \v{18}``I've certainly heard Ephraim \\
\poemll    shuddering with grief as they said,\fnote{\fbackref{31:18} The Heb. lacks \fbib{as they said}} \\
\poeml `You have disciplined me, \\
\poemll    and I'm disciplined like an untrained calf. \\
\poeml Restore me, and let me return,\fnote{\fbackref{31:18} Or \fbib{repent}} \\
\poemll    for you are the \divine{Lord} my God. \\
\poeml \v{19}Indeed, after I turned away, then I repented. \\
\poemll    And after I came to understand, \\
\poemlll       I slapped my forehead.\fnote{\fbackref{31:19} Lit. \fbib{thigh}; i.e. as a sign of remorse} \\
\poeml I was both ashamed and humiliated \\
\poemll    because I bear the disgrace of my youth.'\,''
\passage{God's Gracious Response}
\poeml \v{20}``Is Ephraim my dear son? \\
\poemll    Is he a darling child? \\
\poeml Indeed, as often as I've spoken about him, \\
\poemll    I surely still remember him. \\
\poeml Therefore I deeply yearn for him. \\
\poemll    I'll surely have great compassion on him,'' \\
\poemlll       declares the \divine{Lord}. \\
\poeml \v{21}Set up markers for yourselves. \\
\poemll    Erect signposts for yourselves. \\
\poeml Pay attention to the highway, \\
\poemll    to the road you traveled. \\
\poeml Return, virgin Israel, \\
\poemll    return to these cities of yours. \\
\poeml \v{22}How long will you go this way and that, \\
\poemll    rebellious daughter? \\
\poeml Indeed, the \divine{Lord} will create a new thing on the earth; \\
\poemll    a woman will protect\fnote{\fbackref{31:22} Lit. \fbib{surround}} a man.
\end{poetry}

\v{23}This is what the \divine{Lord} of the Heavenly Armies, the God of Israel, says: ``They'll again speak this message in the land of Judah and its towns when I restore their fortunes:\fnote{\fbackref{31:23} Or \fbib{return them from captivity}} `The \divine{Lord} bless you, righteous dwelling, holy mountain.' \v{24}Judah and all its towns will live together in the land,\fnote{\fbackref{31:24} Lit. \fbib{in it}} along with farmers and those who follow the flock. \v{25}I'll provide abundance for those who are weary, and fill all who are faint.'' \v{26}Then I awoke and looked around, and I had had a pleasant sleep.
\passage{Restoration and Responsibility}

\v{27}``Look, days are coming,'' declares the \divine{Lord}, ``when I'll sow the house of Israel and the house of Judah using people and animals as seed.\fnote{\fbackref{31:27} Lit. \fbib{with the seed of people and the seed of animals}} \v{28}Just as I've watched over them to pull up, tear down, overthrow, destroy, and bring disaster, so I'll watch over them to build and to plant,'' declares the \divine{Lord}. \v{29}``In those days people will no longer say, `The fathers have eaten sour grapes, but the children's teeth have been set on edge.' \v{30}Instead, each person will die for his own iniquity. Everyone who eats sour grapes will have his own\fnote{\fbackref{31:30} The Heb. lacks \fbib{own}} teeth set on edge.''
\passage{The New Covenant}

\v{31}``Look, days are coming,'' declares the \divine{Lord}, ``when I'll make a new covenant with the house of Israel and the house of Judah. \v{32}It won't be like the covenant I made with their ancestors on the day I took them by the hand to bring them out of the land of Egypt. They broke my covenant, although I was a husband to them,'' declares the \divine{Lord}. \v{33}``Rather, this is the covenant that I'll make with the house of Israel after those days,'' declares the \divine{Lord}. ``I'll put my Law\fnote{\fbackref{31:33} Or \fbib{instruction}} within them and will write it on their hearts. I'll be their God and they will be my people. \v{34}No longer will a person teach his neighbor or his relative: `Know the \divine{Lord}.' Instead, they'll all know me, from the least to the greatest of them,'' declares the \divine{Lord}. ``Indeed, I'll forgive their iniquity, and I'll remember their sin no more.''

\begin{poetry}
\poeml \v{35}This is what the \divine{Lord} says, \\
\poemll    who gives the sun for light by day, \\
\poeml the laws that govern the moon and stars for light by night, \\
\poemll    and who stirs up the sea so that its waves roar. \\
\poemlll       The \divine{Lord} of the Heavenly Armies is his name: \\
\poeml \v{36}``If these laws cease to function in my presence,'' \\
\poemll    declares the \divine{Lord}, \\
\poeml ``then the descendants of Israel will cease to be \\
\poemll    a nation in my presence for all time!'' \\
\poeml \v{37}This is what the \divine{Lord} says: \\
\poeml ``If the heavens could be measured above, \\
\poemll    or the foundations of the earth be searched out below, \\
\poeml then I also would reject all the descendants of Israel \\
\poemll    because of everything they have done,'' \\
\poemlll       declares the \divine{Lord}.
\end{poetry}

\v{38}``Look, days are coming,'' declares the \divine{Lord}, ``when the city of the \divine{Lord} will be rebuilt\fnote{\fbackref{31:38} Or \fbib{the city will be rebuilt for the \divine{Lord}}} from the Tower of Hananel to the Corner Gate. \v{39}A measuring line will go straight out from there to the hill of Gareb, and then it will turn to Goah. \v{40}The whole valley of dead bodies and ashes and all the fields as far as the Brook Kidron to the corner of the Horse Gate toward the east will be holy to the \divine{Lord}. It won't be uprooted or overthrown again, forever.''
\labelchapt{32}
\passage{Jeremiah Purchases a Field}

\chapt{32}
\v{1}This is\fnote{\fbackref{32:1} The Heb. lacks \fbib{This is}} the message that came to Jeremiah from the \divine{Lord} in the tenth year of the reign of\fnote{\fbackref{32:1} The Heb. lacks \fbib{of the reign of}} Zedekiah, king of Judah. It was the eighteenth year of the reign of\fnote{\fbackref{32:1} The Heb. lacks \fbib{of the reign of}} Nebuchadnezzar. \v{2}The army of the king of Babylon was then besieging Jerusalem, and Jeremiah the prophet was confined in the courtyard of the guard at the palace of the king of Judah \v{3}where Zedekiah had confined him. Zedekiah had said,\fnote{\fbackref{32:3} Lit. \fbib{him, saying}} ``Why did you prophesy and say these things? You said,\fnote{\fbackref{32:3} Lit. \fbib{Why did you prophesy, saying,}} `This is what the \divine{Lord} says: ``I'm about to give this city into the hand of the king of Babylon, and he will capture it. \v{4}Zedekiah, king of Judah, won't escape from\fnote{\fbackref{32:4} Lit. \fbib{from the hand of}} the Chaldeans, for he has surely been given over to the king of Babylon. He will speak to him face to face and look at him eye to eye. \v{5}The king of Babylon\fnote{\fbackref{32:5} Lit. \fbib{He}} will take Zedekiah to Babylon and there he will stay until I judge him,'' declares the \divine{Lord}. ``If you fight against the Chaldeans, you won't succeed.''\,'\,''

\v{6}Jeremiah replied, ``This message from the \divine{Lord} came to me: \v{7}`Look, Hanamel, your cousin,\fnote{\fbackref{32:7} Lit. \fbib{uncle's son}; and so throughout the chapter} is coming to you and will say, ``Buy my field in Anathoth for yourself, because the right of redemption to buy it belongs to you.''\,'

\v{8}``Then my cousin Hanamel came to me in the courtyard of the guard just as the \divine{Lord} had said, and he told me, `Please buy my field in Anathoth in the territory of Benjamin because you have the right to possess it, and the right to redeem it belongs to you. Buy it for yourself.' So I knew that this was a message from the \divine{Lord}.

\v{9}``Then I bought the field in Anathoth from my cousin Hanamel. I weighed out the silver for him---seventeen shekels\fnote{\fbackref{32:9} I.e. about 6.8 ounces; a shekel weighed about 0.4 ounces} of silver. \v{10}I signed the deed and sealed it. I called in witnesses and used scales to weigh out the silver. \v{11}Then I took the deed of purchase---both the sealed one\fnote{\fbackref{32:11} I.e. a private copy} with the terms and conditions and the open one\fnote{\fbackref{32:11} I.e. a public copy}---\v{12}and I gave the deed of purchase to Neriah's son Baruch, the grandson of Mahseiah, in the presence of my cousin Hanamel, in the presence of the witnesses who signed the deed of purchase, and in the presence of all the Judeans sitting in the courtyard of the guard. \v{13}In their presence, I instructed Baruch as follows: \v{14}`This is what the \divine{Lord} of the Heavenly Armies, the God of Israel, says: ``Take these deeds---both this sealed deed of purchase and this open deed---and put them in a clay pot so they'll last for a long time. \v{15}For this is what the \divine{Lord} of the Heavenly Armies, the God of Israel, says: `Houses, fields, and vineyards will again be bought in this land.'\,''\,'\,''
\passage{Jeremiah's Prayer}

\v{16}``After I had given the deed of purchase to Neriah's son Baruch, I prayed to the \divine{Lord}: \v{17}`\divine{Lord}! Look, you made the heavens and the earth with your great power and your outstretched arm. Nothing is too difficult for you! \v{18}You, the great God, the mighty one, show gracious love to thousands and repay the parents' iniquity to their children after them. The \divine{Lord} of the Heavenly Armies is his name. \v{19}You are\fnote{\fbackref{32:19} The Heb. lacks \fbib{You are}} great in regards to your purposes and mighty in regards to your works. Your eyes are open to everything that people do, and will reward each one according to their ways and just as their actions deserve.\fnote{\fbackref{32:19} Lit. \fbib{and according to the fruit of his deeds}} \v{20}You are the one who performed signs and wonders in the land of Egypt and continue to do so\fnote{\fbackref{32:20} The Heb. lacks \fbib{and continue to do so}} until this day, both in Israel and among the rest of humanity. You made a reputation for yourself that continues to this day.\fnote{\fbackref{32:20} Lit. \fbib{yourself as this day}} \v{21}By your strong hand and outstretched arm, and with great terror, you brought your people Israel out of the land of Egypt with signs and wonders. \v{22}And you gave them this land which you had promised their ancestors that you would give them---a land flowing with milk and honey. \v{23}They came and took possession of it, but they didn't obey you or walk according to your Law.\fnote{\fbackref{32:23} Or \fbib{instruction}} They didn't do what you commanded them to do, so you caused all this calamity to happen to them. \v{24}Look, the siege ramps have reached the city to take it. Because of the sword, famine, and plague, the city has been given over to the Chaldeans who are fighting against it. What you said has happened, and you are watching it occur!\fnote{\fbackref{32:24} Lit. \fbib{and look, you are watching}} \v{25}Lord, you have told me, ``Buy the field for yourself with money and call in witnesses,\fnote{\fbackref{32:25} Lit. \fbib{have witnesses witness}}'' even though the city is being given over to the Chaldeans.'\,''
\passage{Prophecy of Jerusalem's Fall}

\v{26}Then this message from the \divine{Lord} came to Jeremiah: \v{27}``Look, I am the \divine{Lord}, the God who rules over all flesh. Is anything too difficult for me?'' \v{28}Therefore, this is what the \divine{Lord} says: ``I'm about to give this city into the hands of the Chaldeans and Nebuchadnezzar, king of Babylon, and he will capture it. \v{29}The Chaldeans who are fighting against this city will come, set this city on fire, and burn it along with the houses on whose roofs incense was burned to Baal and liquid offerings were poured out to other gods in order to provoke me. \v{30}Indeed, the Israelis\fnote{\fbackref{32:30} Lit. \fbib{sons of Israel}; and so throughout the book} and Judeans\fnote{\fbackref{32:30} Lit. \fbib{sons of Judah}} have been doing only evil in my presence since their youth. Indeed, the Israelis have done nothing but provoke me by what they have made with their hands,'' declares the \divine{Lord}.

\v{31}``Indeed, this city has provoked me to anger and wrath from the day they built it until now, and so I'll remove it\fnote{\fbackref{32:31} Lit. \fbib{to remove it}} from my sight \v{32}because of all the evil that the Israelis and Judeans have done to provoke me. They, their kings, their officials, their priests, their prophets, the people of Judah, and those living in Jerusalem have done these things.\fnote{\fbackref{32:32} The Heb. lacks \fbib{have done these things}} \v{33}They have turned their backs to me rather than their faces. Even though I taught them, teaching them again and again,\fnote{\fbackref{32:33} Lit. \fbib{getting up early and teaching}} they didn't listen to accept correction. \v{34}They put their detestable idols in the house that is called by my name and defiled it. \v{35}They built the high places of Baal that are in the Hinnom Valley\fnote{\fbackref{32:35} Lit. \fbib{Valley of Hinnom's son}} in order to sacrifice their sons and daughters to Molech---something that I didn't command, nor did it ever enter my mind for them to require this utterly repugnant thing---and lead Judah into sin.''
\passage{A Promise of Restoration}

\v{36}``Now therefore,'' says the \divine{Lord} God of Israel, ``concerning this city about which you are saying, `It is being given into the control of the king of Babylon by sword, famine, and plague,' \v{37}I'm about to gather my people\fnote{\fbackref{32:37} Lit. \fbib{them}} from all the lands where I've driven them in my anger, wrath, and great indignation. I'll bring them back to this place and let them live in safety. \v{38}They'll be my people, and I'll be their God. \v{39}I'll give them one heart and one lifestyle\fnote{\fbackref{32:39} Lit. \fbib{way}} so they'll fear me always for their own good and for the good of\fnote{\fbackref{32:39} The Heb. lacks \fbib{the good of}} their descendants after them. \v{40}I'll make an everlasting covenant with them that I won't turn away from doing good for them.\fnote{\fbackref{32:40} Or \fbib{from them to do good}} I'll put the fear of me in their hearts so they won't turn away from me. \v{41}I'll rejoice over them to do good for them, and I'll faithfully plant them in this land with all my heart and soul.'

\v{42}``Indeed, this is what the \divine{Lord} says: `Just as I'm bringing all this great disaster on this people, so I'll bring on them all the good things that I'm promising concerning them. \v{43}Fields will be bought in this land about which you will say, ``It is a desolate place without people or animals. It is given into the hands of the Chaldeans.'' \v{44}People\fnote{\fbackref{32:44} Lit. \fbib{They}} will buy fields for money, sign deeds, seal them, and call witnesses in the land of Benjamin, in the areas around Jerusalem, in the towns of Judah, the towns of the hill country, the towns of the Shephelah,\fnote{\fbackref{32:44} I.e. the verdant central lowlands of Israel; cf. Josh 10:40} and the towns of the Negev,\fnote{\fbackref{32:44} I.e. the southern regions of the Sinai peninsula; cf. Josh 10:40} for I'll restore their fortunes,' declares the \divine{Lord}.''
\labelchapt{33}
\passage{Restoration of Judah and Jerusalem}

\chapt{33}
\v{1}This message from the \divine{Lord} came to Jeremiah a second time while he was still confined in the courtyard of the guard: \v{2}``This is what the \divine{Lord} says who made the earth, the \divine{Lord} who formed it in order to establish it---whose name is the \divine{Lord}--- \v{3}`Call to me and I'll answer you, and will tell you about great and hidden\fnote{\fbackref{33:3} Or \fbib{inaccessible}} things that you don't know.' \v{4}For this is what the \divine{Lord} God of Israel says about the houses of this city and the houses of the kings of Judah that were torn down to defend\fnote{\fbackref{33:4} The Heb. lacks \fbib{to defend}} against the siege ramps and the sword, \v{5}`The Chaldeans are coming to fight and to fill those houses with the dead bodies of the people that I've struck down in my anger and wrath, for I've hidden my face from this city because of all their wickedness.

\v{6}```Look, I'll bring restoration and healing to it, and I'll heal them. I'll reveal to them an abundance of peace and faithfulness. \v{7}I'll restore the security of Judah and Israel\fnote{\fbackref{33:7} Lit. \fbib{fortunes of Israel}} and rebuild them as they were at first. \v{8}I'll cleanse them from all their sin that they have committed against me, and I'll forgive all their sins that they committed against me and by which they rebelled against me. \v{9}Jerusalem\fnote{\fbackref{33:9} Lit. \fbib{It}} will be for me a name of joy, praise, and glory to all the nations of the earth that hear about all the good that I'm doing for them. They'll fear and tremble because of all the good and because of all the peace that I'm bringing to Jerusalem.'\fnote{\fbackref{33:9} Lit. \fbib{for it}}

\v{10}``This is what the \divine{Lord} says: `You are saying about this place, ``It is a ruin without people and without animals.'' Yet in the cities of Judah and the streets of Jerusalem which are desolate places without inhabitants and without animals, there will again be heard \v{11}the sounds of rejoicing and gladness, the sounds of the bridegroom and the bride, and the sounds of those saying,

\begin{poetry}
\poeml ``Give thanks to the \divine{Lord} of the Heavenly Armies, \\
\poemll    for the \divine{Lord} is good, \\
\poemlll       and his gracious love lasts forever,''
\end{poetry}

as they bring thanksgiving offerings to the \divine{Lord}'s Temple. For I'll restore the fortunes of the land as they were at first,' declares the \divine{Lord}.

\v{12}``This is what the \divine{Lord} of the Heavenly Armies says: `In this place that is now a ruin without people or animals, and in all its towns there will again be pasture for shepherds resting their flocks. \v{13}In the towns of the hill country, in the towns of the Shephelah,\fnote{\fbackref{33:13} I.e. the verdant central lowlands of Israel; cf. Josh 10:40} in the towns of the Negev,\fnote{\fbackref{33:13} I.e. the southern regions of the Sinai peninsula; cf. Josh 10:40} in the territory of Benjamin, in the areas around Jerusalem, and in the towns of Judah flocks will again pass under the hands of the one who counts them,' says the \divine{Lord}.''
\passage{The Righteous Branch and the Davidic Covenant}

\v{14}```Look, the time is coming,' declares the \divine{Lord}, `when I'll fulfill the good promise that I spoke concerning the house of Israel and Judah. \v{15}In those days and at that time I'll cause a righteous Branch to spring up for David, and he will uphold justice and righteousness in the land. \v{16}At that time Judah will be delivered and Jerusalem will dwell in safety. And this is the name people\fnote{\fbackref{33:16} Lit. \fbib{they}} will call it, ``The \divine{Lord} is Our Righteousness.''\,' \v{17}For this is what the \divine{Lord} says: `David will never be without\fnote{\fbackref{33:17} Lit. \fbib{there will never be cut off for David}} a man sitting on the throne of the house of Israel, \v{18}nor will the Levitical priests be without\fnote{\fbackref{33:18} \fbib{there will never be cut off for the Levitical priests}} a man offering up burnt offerings, bringing in grain offerings, and offering sacrifices continuously\fnote{\fbackref{33:18} Lit. \fbib{all the days}} before me.'\,''

\v{19}This message from the \divine{Lord} came to Jeremiah: \v{20}``This is what the \divine{Lord} says: `If you could break my covenant with the day and night\fnote{\fbackref{33:20} Lit. \fbib{and my covenant with the night}} so that day and night wouldn't occur at the proper time,\fnote{\fbackref{33:20} Lit. \fbib{at their time}} \v{21}then my covenant with my servant David might also be broken so that he wouldn't have a son sitting on his throne, and so also with my servants the Levitical priests. \v{22}As the heavenly bodies\fnote{\fbackref{33:22} Lit. \fbib{the hosts of the heavens}} cannot be counted, and the sands of the sea cannot be measured, so I'll multiply the descendants\fnote{\fbackref{33:22} Lit. \fbib{seed}} of my servant David and the descendants of Levi who serve me.'\,''

\v{23}This message from the \divine{Lord} came to Jeremiah: \v{24}``Haven't you noticed what these people have been saying?---`The \divine{Lord} rejected the two families that he had chosen!' They have contempt for my people and no longer consider them a nation. \v{25}This is what the \divine{Lord} says: `If I had not established my covenant for day and night and the laws that govern\fnote{\fbackref{33:25} Lit. \fbib{the ordinances of}} the heavens and earth, \v{26}then I might reject the descendants\fnote{\fbackref{33:26} Lit. \fbib{seed}; and so throughout the verse} of Jacob and my servant David by not taking some of his descendants as rulers over the descendants of Abraham, Isaac, and Jacob. Indeed, I'll restore their fortunes, and I'll have compassion on them.'\,''
\labelchapt{34}
\passage{A Message to Zedekiah}

\chapt{34}
\v{1}This is\fnote{\fbackref{34:1} The Heb. lacks \fbib{This is}} the message that came to Jeremiah from the \divine{Lord} while king Nebuchadnezzar of Babylon, all his army, all the kingdoms of the earth that were under his authority, along with all the people were fighting against Jerusalem and all its towns: \v{2}``This is what the \divine{Lord} God of Israel says: `Go and speak to king Zedekiah of Judah. Say to him, ``This is what the \divine{Lord} says: `Look, I'm giving this city into the hand of the king of Babylon, and he will set it on fire. \v{3}You won't escape from him. You will certainly be captured and given into his control.\fnote{\fbackref{34:3} Lit. \fbib{hand}} You will see the king of Babylon eye to eye, he will speak to you face to face, and you will go to Babylon.'\,''\,' \v{4}Yet, hear this message from the \divine{Lord}, king Zedekiah of Judah. This is what the \divine{Lord} says to you, `You won't die by the sword. \v{5}You will die peacefully, and as they burned fires\fnote{\fbackref{34:5} I.e. a memorial fire} for your ancestors,\fnote{\fbackref{34:5} Lit. \fbib{like the burning for your ancestors}} the former kings who were before you, so they'll burn fires\fnote{\fbackref{34:5} The Heb. lacks \fbib{fires}} for you, wailing, ``Oh how terrible, your majesty!''\,' For I've spoken the message,'' declares the \divine{Lord}.

\v{6}Then Jeremiah the prophet spoke all of this in Jerusalem to king Zedekiah of Judah, \v{7}while the army of the king of Babylon was fighting against Jerusalem and all the cities of Judah that were left, namely Lachish and Azekah. (They were the only fortified cities that remained among the cities of Judah.)
\passage{A Broken Agreement with Hebrew Servants}

\v{8}This is\fnote{\fbackref{34:8} The Heb. lacks \fbib{This is}} this message from the \divine{Lord} that came to Jeremiah from the \divine{Lord} after Zedekiah had made a covenant with all the people in Jerusalem proclaiming release for them. \v{9}Each person was to set free his male and female slaves who were Hebrews, so that no Jewish person would enslave his brother.\fnote{\fbackref{34:9} I.e. another Jewish person} \v{10}All the officials and all the people who had entered into the covenant agreed\fnote{\fbackref{34:10} Or \fbib{obeyed}} that each would set his male and female slaves free so that they\fnote{\fbackref{34:10} Lit. \fbib{he}} would not enslave them any longer. They obeyed and they released them. \v{11}But afterward they turned around and took back the male and female slaves that they had set free, and they forced them to become male and female slaves.

\v{12}Then this message from the \divine{Lord} came to Jeremiah from the \divine{Lord}: \v{13}``This is what the \divine{Lord} God of Israel says: `I made a covenant with your ancestors on the day I brought them out of the land of Egypt, out of the house of slavery. I told them: \v{14}``At the end of seven years, each of you is to set free your fellow Hebrew who has sold himself to you and has served you for six years. You are to send him out from you with no further obligation.'' But your ancestors didn't obey me or pay attention.\fnote{\fbackref{34:14} Lit. \fbib{incline their ears}} \v{15}You recently repented and did what was right in my eyes by proclaiming release for one another, and you made a covenant before me in the house that is called by my name. \v{16}But then you turned around and profaned my name when each of you took back his male and female slaves whom you had set free according to their desire, and you forced them to become male and female slaves.''\,'

\v{17}``Therefore, this is what the \divine{Lord} says: `You haven't obeyed me by each of you proclaiming a release for your brothers and neighbors. Now I'm going to proclaim a release for you,' declares the \divine{Lord}, `a release\fnote{\fbackref{34:17} The Heb. lacks \fbib{a release}} to the sword, to plague, and to famine, and I'll make you a horrifying sight to all the kingdoms of the earth. \v{18}I'll give over the men who transgressed my covenant, who haven't fulfilled the terms of the covenant that they made before me when they cut the calf in two and passed between its parts---\v{19}the officials of Judah, the officials of Jerusalem, the eunuchs,\fnote{\fbackref{34:19} Or \fbib{palace officials}} the priests, and all the people of the land who passed between the parts of the calf. \v{20}I'll give them to their enemies who are seeking to kill them, and their dead bodies will be food for the birds of the sky and the animals of the land. \v{21}I'll give Zedekiah, king of Judah, and his officials into the domination of their enemies, to those\fnote{\fbackref{34:21} Lit. \fbib{to the hands of those}} who are seeking to kill them, and to\fnote{\fbackref{34:21} Lit. \fbib{to the hands of the}} the army of the king of Babylon that is coming against them. \v{22}Look, I'm in command of them,' declares the \divine{Lord}, `and I'll bring them back to this city. They'll capture it and burn it with fire, and I'll turn the towns of Judah into desolate places without inhabitants.'\,''
\labelchapt{35}
\passage{The Example of the Rechabites}

\chapt{35}
\v{1}This is the message that came to Jeremiah from the \divine{Lord} during the reign\fnote{\fbackref{35:1} Lit. \fbib{days}} of Josiah's son Jehoiakim, king of Judah: \v{2}``Go to the house of the Rechabites and speak to them. Bring them into the \divine{Lord}'s Temple, to one of the offices, and offer them wine to drink.'' \v{3}So I took Jeremiah's son Jaazaniah (a descendant of Habazziniah), his brothers, all his sons, and the whole family of the Rechabites. \v{4}I brought them to the \divine{Lord}'s Temple to the office of the descendants of Igdaliah's son Hanan, the man of God, which was next to the office of the officials, and which was above the office of Shallum's son Maaseiah, the keeper of the threshold.

\v{5}I put containers full of wine and cups in front of the members of the Rechabite clan\fnote{\fbackref{35:5} Lit. \fbib{the sons of the house of the Rechabites}} and told them, ``Drink the wine!''

\v{6}But they said, ``We won't drink wine, because our ancestor, Rechab's son Jonadab commanded us: `You and your descendants are never to drink wine! \v{7}You aren't to build houses, you aren't to sow seeds, and you aren't to plant vineyards, or own them. Instead, you are to live in tents all your lives,\fnote{\fbackref{35:7} Lit. \fbib{your days}} so you will enjoy a long life in the land where you reside.'\fnote{\fbackref{35:7} I.e. living as resident aliens} \v{8}We have obeyed everything that our ancestor, Rechab's son Jonadab, commanded us. So we, our wives, our sons, and our daughters have drunk no wine all our lives,\fnote{\fbackref{35:8} Lit. \fbib{all our days}} \v{9}and have built no houses to live in. We don't have vineyards, fields, or seed. \v{10}We have lived in tents. We have obeyed and have done everything that our ancestor Jonadab commanded us. \v{11}Now when Nebuchadnezzar king of Babylon came up against the land, we said, `Come on! Let's go to Jerusalem because of the army of the Chaldeans and the army of Aram. And now we're living in Jerusalem.'\,''

\v{12}This message from the \divine{Lord} came to Jeremiah: \v{13}``This is what the \divine{Lord} of the Heavenly Armies, the God of Israel says: `Go and say to the people of Judah and the inhabitants of Jerusalem, ``Will you not accept correction by listening to what I say?'' declares the \divine{Lord}. \v{14}``But what Rechab's son Jonadab commanded his sons about not drinking wine is observed, and they haven't drunk wine until this day. Indeed, they obey the commands of their ancestor. But I've spoken to you again and again,\fnote{\fbackref{35:14} Lit. \fbib{getting up early and speaking}} and you haven't obeyed me. \v{15}I've sent you all my servants, the prophets, sending them again and again.\fnote{\fbackref{35:15} Lit. \fbib{getting up early and sending}} I've said, `Each of you turn from his evil behavior\fnote{\fbackref{35:15} Lit. \fbib{way}} and make your deeds right. Don't follow other gods to serve them. Then you will remain in the land that I gave to you and to your ancestors.' But you haven't paid attention\fnote{\fbackref{35:15} Lit. \fbib{inclined your ear}} and you haven't obeyed me. \v{16}Indeed the descendants of Rechab's son Jonadab have carried out the command of their ancestor that he gave them, but this people has not obeyed me.'' \v{17}Therefore, this is what the \divine{Lord} God of the Heavenly Armies, the God of Israel says: ``Look, I'm bringing on Judah and all the residents of Jerusalem all the disaster that I pronounced against them, because I spoke to them, but they didn't listen, and I called out to them, but they didn't answer.''\,'\,''

\v{18}Then Jeremiah told the house of the Rechabites, ``This is what the \divine{Lord} God of the Heavenly Armies, the God of Israel says: `Because you obeyed the commandment of your ancestor Jonadab, have observed all his commandments, and have done everything that he commanded you,' \v{19}therefore, this is what the \divine{Lord} God of the Heavenly Armies, the God of Israel says: `Rechab's son Jonadab won't lack a descendant\fnote{\fbackref{35:19} Cf. Neh 3:14} who serves me\fnote{\fbackref{35:19} Lit. \fbib{there won't be cut off from Rechab's son Jonadab one standing before me}} always.'\,''
\labelchapt{36}
\passage{Jeremiah's Scroll Read in the Temple}

\chapt{36}
\v{1}In the fourth year of the reign of\fnote{\fbackref{36:1} The Heb. lacks \fbib{of the reign of}} Josiah's son King Jehoiakim of Judah, this message came to Jeremiah from the \divine{Lord}: \v{2}``Take a scroll and write on it all the words that I've spoken to you about Israel, about Judah, and about all the nations, since I first spoke to you\fnote{\fbackref{36:2} Lit. \fbib{from the day I spoke to you}} in the time of Josiah until the present time. \v{3}Perhaps the house of Judah will hear about all the calamity that I'm planning to bring on them, and so each of them will turn from his wicked way and I'll forgive their iniquities and sins.''

\v{4}Jeremiah summoned Neriah's son Baruch and at Jeremiah's dictation, Baruch wrote on the scroll all the words of the \divine{Lord} that he had spoken to him.

\v{5}Jeremiah instructed Baruch, ``I'm confined and can't go to the \divine{Lord}'s Temple. \v{6}You go and read the words of the \divine{Lord} that you wrote at my dictation from the scroll. Read them\fnote{\fbackref{36:6} The Heb. lacks \fbib{Read them}} to\fnote{\fbackref{36:6} Lit. \fbib{scroll in the hearing of}} the people at the \divine{Lord}'s Temple on the fast day. Also read them to all the people of Judah who are coming from their towns. \v{7}Perhaps their pleas for help will come to the \divine{Lord}'s attention, and each of them will turn from his evil lifestyle in light of the great anger and wrath that the \divine{Lord} has declared against this people.'' \v{8}So Neriah's son Baruch did just as Jeremiah the prophet instructed him, reading the words of the \divine{Lord} from the scroll at the \divine{Lord}'s Temple.

\v{9}In the ninth month of the fifth year of the reign of\fnote{\fbackref{36:9} The Heb. lacks \fbib{of the reign of}} Josiah's son Jehoiakim, king of Judah, a fast was proclaimed in the \divine{Lord}'s presence in Jerusalem for all the people of Jerusalem, as well as all the people who were coming from the towns of Judah. \v{10}Baruch read the words of Jeremiah from the scroll to\fnote{\fbackref{36:10} Lit. \fbib{book in the hearing of}} all the people at the \divine{Lord}'s Temple. He did this\fnote{\fbackref{36:10} Heb. lacks \fbib{He did this}} from the office of Shaphan's son Gemariah the scribe, in the upper court at the entrance of the New Gate of the \divine{Lord}'s Temple.
\passage{Jeremiah's Scroll Read in the Palace}

\v{11}When Gemariah's son Micaiah, the grandson of Shaphan, heard all the words of the \divine{Lord} from the scroll, \v{12}he went down to the palace, to the scribe's office, where all the officials were sitting. Elishama the scribe, Shemaiah's son Delaiah, Achbor's son Elnathan, Shaphan's son Gemariah, Hananiah's son Zedekiah, and all the other officials were there. \v{13}Micaiah told them all the things that he had heard when Baruch read from the scroll to the people. \v{14}Then all the officials sent Nethaniah's son Jehudi, (who was also the grandson of Shelemiah and Cushi's great-grandson), to Baruch, who said, ``Take the scroll that you read to\fnote{\fbackref{36:14} Lit. \fbib{read in the hearing of}} the people and come.'' Neriah's son Baruch took the scroll with him and went to them.

\v{15}They told him, ``Please sit down and read it to us.''\fnote{\fbackref{36:15} Lit. \fbib{in our hearing}} So Baruch read it to them. \v{16}When they heard all the words, they turned to one another in fear, saying to Baruch, ``We must report all these things to the king.'' \v{17}Then they asked Baruch, ``Please tell us how you wrote all the words. Did Jeremiah dictate them all?''\fnote{\fbackref{36:17} Lit. \fbib{from his mouth}}

\v{18}Baruch answered them, ``Yes, Jeremiah dictated all these words to me, and I wrote them in the scroll with ink.''

\v{19}Then the officials told Baruch, ``Go, hide yourself, both you and Jeremiah, and don't let anyone know where you are.''
\passage{The King Burns Jeremiah's Scroll}

\v{20}The officials\fnote{\fbackref{36:20} Lit. \fbib{They}} went to the king in the courtyard, but they deposited the scroll in the office of Elishama the scribe. Then they reported everything written on the scroll\fnote{\fbackref{36:20} Lit. \fbib{reported all the words}} to the king. \v{21}The king sent Jehudi to get the scroll, and he took it from the office of Elishama the scribe. Jehudi read it to the king\fnote{\fbackref{36:21} Lit. \fbib{in the hearing of the king}} and to all the officials who were standing beside the king. \v{22}The king was sitting in the winter palace in the ninth month and a stove\fnote{\fbackref{36:22} Or \fbib{brazier}} was burning in front of him.\fnote{\fbackref{36:22} Or \fbib{a fire was burning in the stove in front of him}} \v{23}As Jehudi would read three or four columns, the king\fnote{\fbackref{36:23} Lit. \fbib{he}} would cut it with a scribe's knife and throw it into the fire which was in the stove, until all the scroll was burned\fnote{\fbackref{36:23} Lit. \fbib{until the completion of the scroll}} in the fire in the stove. \v{24}The king and all his officials\fnote{\fbackref{36:24} Or \fbib{servants}} who were listening to these words were not afraid, nor did they tear their garments. \v{25}Even though Elnathan, Delaiah, and Gemariah urged the king not to burn the scroll, he would not listen to them. \v{26}The king ordered his\fnote{\fbackref{36:26} Lit. \fbib{the king's}} son Jerahmeel, Azriel's son Seraiah, and Abdeel's son Shelemiah to get Baruch the scribe and Jeremiah the prophet, but the \divine{Lord} had hidden them.
\passage{Jeremiah Rewrites the Scroll}

\v{27}This message from the \divine{Lord} came to Jeremiah after the king burned the scroll containing the words that Baruch had written at Jeremiah's dictation: \v{28}``Go back, take another scroll and write on it all the original\fnote{\fbackref{36:28} Lit. \fbib{first}} words which were on the scroll that Jehoiakim, king of Judah, burned. \v{29}Concerning Jehoiakim, king of Judah, you are to say, `This is what the \divine{Lord} says: ``You burned this scroll, all the while saying, `Why did you write on it that the king of Babylon will definitely come, destroy this land, and eliminate both people and animals from it?'\,'' \v{30}Therefore, this is what the \divine{Lord} says concerning Jehoiakim, king of Judah, ``He will have no one to sit on the throne of David, and his corpse will be thrown out to rot during the heat of the day and the frost of the night. \v{31}I'll punish him, his descendants, and his officials\fnote{\fbackref{36:31} Or \fbib{servants}} for their iniquity. I'll bring on them, on the residents of Jerusalem, and on the men of Judah all the calamity about which I've warned them, but they would not listen.''\,'\,''

\v{32}Then Jeremiah took another scroll and gave it to Neriah's son Baruch the scribe. He wrote on it, at Jeremiah's dictation, all the words of the book that Jehoiakim king of Judah burned in the fire. He also added to them many similar words.
\labelchapt{37}
\passage{Zedekiah Consults Jeremiah}

\chapt{37}
\v{1}Josiah's son King Zedekiah reigned in place of Jehoiakim's son Coniah,\fnote{\fbackref{37:1} I.e. Jehoiachin} whom Nebuchadnezzar king of Babylon had made king of the land of Judah. \v{2}But neither he nor his officials nor the people of the land listened to the words of the \divine{Lord} that were spoken by\fnote{\fbackref{37:2} Lit. \fbib{that were in the hand of}} Jeremiah the prophet.

\v{3}King Zedekiah sent Shelemiah's son Jehucal and Maaseiah's son Zephaniah the priest to Jeremiah the prophet, asking him, ``Please pray to the \divine{Lord} our God for us.'' \v{4}Now Jeremiah was still\fnote{\fbackref{37:4} The Heb. lacks \fbib{still}} going in and out among the people since he had not yet been put in prison. \v{5}Pharaoh's army had come out of Egypt, and when the Chaldeans who were besieging Jerusalem heard the report about them, they withdrew from Jerusalem.

\v{6}Then this message from the \divine{Lord} came to Jeremiah the prophet: \v{7}``This is what the \divine{Lord} God of Israel says: `This is what you are to say to the king of Judah who sent you to me to inquire of me, ``Look, Pharaoh's army that has come to help will go back to its own land of Egypt, \v{8}and then the Chaldeans will come back to fight against this city, to capture it, and burn it with fire.''\,' \v{9}``This is what the \divine{Lord} says: `Don't deceive yourselves by saying, ``The Chaldeans will surely go away from us,'' `for they won't go. \v{10}Indeed, even if you defeated the entire Chaldean army that is fighting against you, and they had only wounded men left in their tents, they would get up and burn this city with fire.'\,''\,'\,''
\passage{Jeremiah Arrested for Treason}

\v{11}When the Chaldean army was leaving Jerusalem because of Pharaoh's army, \v{12}Jeremiah left Jerusalem to go to the territory of Benjamin to take possession of his property\fnote{\fbackref{37:12} The Heb. lacks \fbib{of his property}} there among the people. \v{13}He was in the Gate of Benjamin, and chief officer Irijah, Shelemiah's son and the grandson of Hananiah, was there. He arrested Jeremiah the prophet, accusing him: ``You are going over to the Chaldeans!''

\v{14}Jeremiah said, ``It's a lie! I'm not going over to the Chaldeans.'' But Irijah\fnote{\fbackref{37:14} Lit. \fbib{he}} would not listen to him, and he\fnote{\fbackref{37:14} Lit. \fbib{Irijah}} arrested Jeremiah and brought him to the officials. \v{15}The officials were angry with Jeremiah and beat him. They put him in jail in the house of Jonathan the scribe because they had made it into a prison. \v{16}So Jeremiah came into the cells in the dungeon\fnote{\fbackref{37:16} Lit. \fbib{cistern-house}} and remained there for a long time.\fnote{\fbackref{37:16} Lit. \fbib{for many days}}

\v{17}Then King Zedekiah sent for Jeremiah\fnote{\fbackref{37:17} The Heb. lacks \fbib{for Jeremiah}} and received him. The king questioned him secretly in his house: ``Is there a message from the \divine{Lord}?''

Jeremiah said, ``There is,'' and then he said, ``You will be given into the hand of the king of Babylon.'' \v{18}Then Jeremiah asked King Zedekiah, ``What offense have I committed against you, your officials, or these people that you have put me in prison? \v{19}Where are your prophets who prophesied to you, telling you: `The king of Babylon won't come against you or against this land'? \v{20}Now, please listen, your majesty,\fnote{\fbackref{37:20} Lit. \fbib{my lord the king}} and pay attention to what I'm asking you. Don't make me go back to the house of Jonathan the scribe, so I don't die there.''

\v{21}So King Zedekiah gave the order, and they assigned Jeremiah to the courtyard of the guard. Each day they gave him a loaf of bread from the bakers' street until all the bread in the city was gone. So Jeremiah remained in the courtyard of the guard.
\labelchapt{38}
\passage{Jeremiah is Arrested and Imprisoned}

\chapt{38}
\v{1}Mattan's son Shephatiah, Pashhur's son Gedaliah, Shelemiah's son Jucal, and Malchijah's son Pashhur heard the words that Jeremiah was speaking to all the people: \v{2}``This is what the \divine{Lord} says: `Whoever stays in this city will die by the sword, by famine, and by the plague, but the one who goes over to the Chaldeans will live. His life will be spared,\fnote{\fbackref{38:2} Lit. \fbib{He will have his life as a spoil of war}} and he will live.' \v{3}This is what the \divine{Lord} says: `This city will surely be given to the army of the king of Babylon, and he will capture it.'\,''

\v{4}Then the officials told the king, ``Let this man be put to death because he's undermining the efforts\fnote{\fbackref{38:4} Lit. \fbib{weakening the hands}} of the soldiers who remain in this city and that of all the people by speaking words like these to them. Indeed, this man is not seeking the well-being of this people, but rather their harm.''

\v{5}King Zedekiah said, ``Look, he's in your hands, and the king can do nothing to you.'' \v{6}So they threw Jeremiah into a cistern that belonged to the king's son Malchijah and was located in the courtyard of the guard. When they let Jeremiah down with ropes, because there was no water in the cistern---only mud---Jeremiah sank into the mud.
\passage{Jeremiah Rescued from the Cistern}

\v{7}Ebed-melech the Ethiopian, a eunuch\fnote{\fbackref{38:7} Or \fbib{official}} in the king's house, heard that Jeremiah had been put in the cistern. The king was sitting in the Benjamin Gate, \v{8}so Ebed-melech went out of the palace and spoke to the king: \v{9}``Your majesty,\fnote{\fbackref{38:9} Lit. \fbib{My lord the king}} these men have acted wickedly in all they have done to the prophet Jeremiah by throwing him into the cistern. He will die where he is because of the famine since there is no more bread in the city.''

\v{10}Then the king ordered Ebed-melech the Ethiopian:\fnote{\fbackref{38:10} Lit. \fbib{Cushite}} ``Thirty men are at your disposal. Take them with you and bring up Jeremiah the prophet from the cistern before he dies.'' \v{11}So Ebed-melech took the men with him and went to the palace, underneath the storeroom. He took worn out rags and worn out clothes from there, and using ropes he lowered them down to Jeremiah in the cistern.

\v{12}Ebed-melech the Ethiopian told Jeremiah, ``Put the worn out rags and clothes under your armpits under the ropes,'' and Jeremiah did as he said.\fnote{\fbackref{38:12} Lit. \fbib{did thus}} \v{13}They pulled Jeremiah with the ropes and brought him up from the cistern, but Jeremiah remained in the courtyard of the guard.
\passage{Zedekiah Again Seeks Advice from Jeremiah}

\v{14}King Zedekiah sent for Jeremiah the prophet and had him brought to him\fnote{\fbackref{38:14} Lit. \fbib{sent and took Jeremiah the prophet to him}} at the third entrance to the \divine{Lord}'s Temple. The king told Jeremiah, ``I'm going to ask you something, and don't hide anything from me.''

\v{15}Jeremiah told Zedekiah, ``When I tell you, you will surely put me to death, won't you? And when I give you advice, you don't listen to me.''

\v{16}Then King Zedekiah, in secret, swore an oath to Jeremiah: ``As surely as the \divine{Lord} lives, who gave us this life to live, I won't have you put to death, nor will I hand you over to these men who are seeking to kill you.''

\v{17}So Jeremiah told Zedekiah, ``This is what the \divine{Lord} God of the Heavenly Armies, the God of Israel, says: `If you will immediately surrender\fnote{\fbackref{38:17} Lit. \fbib{go out}} to the officers\fnote{\fbackref{38:17} Or \fbib{officials}} of the king of Babylon, then you will live, and this city won't be burned with fire. Both you and your family will live. \v{18}But if you don't surrender to the officers of the king of Babylon, then this city will be given to the Chaldeans, and they'll burn it with fire. You won't escape from their hands.'\,''

\v{19}Then King Zedekiah told Jeremiah, ``I'm afraid of the Judeans who have gone over to the Chaldeans. The Chaldeans\fnote{\fbackref{38:19} Lit. \fbib{They}} may turn me over to them,\fnote{\fbackref{38:19} Lit. \fbib{may give me into their hands}} and they may treat me harshly.''

\v{20}Jeremiah said, ``They won't turn you over. Obey the \divine{Lord} in what I'm telling you, and it will go well for you and you will live. \v{21}But if you refuse to surrender,\fnote{\fbackref{38:21} Lit. \fbib{to go out}} this is what the \divine{Lord} has shown me: \v{22}Look, all the women who are left in the house of the king of Judah will be brought out to the officers of the king of Babylon, and will say,

\begin{poetry}
\poeml `These friends of yours have mislead you \\
\poemll    and overcome you. \\
\poeml Your feet have sunk down into the mire, \\
\poemll    but they have turned away.'
\end{poetry}

\v{23}``They'll bring all your women and children out to the Chaldeans, and you won't escape from their hand. Indeed, you will be seized by the hand of the king of Babylon, and this city will be burned with fire.''

\v{24}Then Zedekiah told Jeremiah, ``Don't let anyone know about these words and you won't die. \v{25}If the officials hear that I've spoken with you, and they come to you and say,\fnote{\fbackref{38:25} Lit. \fbib{say to you}} `Tell us what you told the king, and what the king told you; don't hide it from us, and we won't put you to death,' \v{26}then you are to say to them, `I was presenting my request to the king that I not be taken back to the house of Jonathan to die there.'\,''

\v{27}When all the officials came to Jeremiah and questioned him, he replied to them exactly as the king had ordered him.\fnote{\fbackref{38:27} Lit. \fbib{according to all these words that the king ordered him}} So they stopped speaking with him because the conversation had not been overheard. \v{28}Jeremiah remained in the courtyard of the guard until the day Jerusalem was captured.
\labelchapt{39}
\passage{The Fall of Jerusalem and the Capture of Zedekiah}

\chapt{39}
\v{1}This is how Jerusalem was captured:\fnote{\fbackref{39:1}a in English is 38:28b in Hebrew} In the tenth month of the ninth year of the reign of\fnote{\fbackref{39:1} The Heb. lacks \fbib{of the reign of}} Zedekiah king of Judah, Nebuchadnezzar king of Babylon and all his army came to Jerusalem and laid siege to it. \v{2}On the ninth day of the fourth month, in the eleventh year of the reign of\fnote{\fbackref{39:2} The Heb. lacks \fbib{of the reign of}} Zedekiah, the wall of\fnote{\fbackref{39:2} The Heb. lacks \fbib{the wall of}} the city was breached. \v{3}All the officials of the king of Babylon came and sat in the Middle Gate, including\fnote{\fbackref{39:3} The Heb. lacks \fbib{including}} Nergal-sarri-usur, governor\fnote{\fbackref{39:3} Or \fbib{high official}} of Sinmagir,\fnote{\fbackref{39:3} I.e. a province of Babylon} Nabu-sarrussu-ukin the high official,\fnote{\fbackref{39:3} Lit. \fbib{the Rab-sa-resi}; a Babylonian title for a royal official} Nergal-sarri-user, the chief official,\fnote{\fbackref{39:3} Lit. \fbib{the Rab-mugi}, a Babylonian title for a royal official} and\fnote{\fbackref{39:3} Or \fbib{Gate: Nergal-sarezer, Samgar-nebu, Sarsekim, the high official, Nergal-sarezer, the chief official, and}} all the rest of the officials of the king of Babylon.

\v{4}When Zedekiah king of Judah and all the soldiers saw them, they fled and went out of the city at night through the king's garden through the gate between the two walls. Then he went out on the road toward the Arabah. \v{5}The Chaldean army pursued them and overtook Zedekiah on the plains of Jericho. When they seized him they brought him to Nebuchadnezzar king of Babylon at Riblah in the land of Hamath, and he passed judgment on him. \v{6}At Riblah, the king of Babylon executed Zedekiah's sons right\fnote{\fbackref{39:6} The Heb. lacks \fbib{right}} before his eyes. He\fnote{\fbackref{39:6} Lit. \fbib{The king of Babylon}} also executed all the nobles of Judah. \v{7}Then he put out Zedekiah's eyes and bound him with bronze fetters to take him to Babylon.

\v{8}The Chaldeans burned the palace and the houses of the people with fire, and they broke down the walls of Jerusalem. \v{9}Nebuzaradan, the captain of the Babylonian guard, took into exile in Babylon the rest of the people who remained in the city, those who had deserted to Nebuchadnezzar, and the rest of the people who remained. \v{10}Nebuzaradan the captain of the guard left in the land of Judah some of the poor people who did not have anything, and he gave them vineyards and fields on that day.
\passage{Jeremiah's Release from Prison}

\v{11}Nebuchadnezzar king of Babylon gave orders concerning Jeremiah through Nebuzaradan, the captain of the guard: \v{12}``Take him, look after him, and don't do anything to harm him. Rather, do for him whatever he tells you.'' \v{13}So Nebuzaradan, the captain of the guard, Nebushazban, the high official, Nergal-sar-ezer, the chief official, and all the officials of the king of Babylon sent for Jeremiah.\fnote{\fbackref{39:13} The Heb. lacks \fbib{for Jeremiah}} \v{14}They sent for Jeremiah\fnote{\fbackref{39:14} The Heb. lacks \fbib{for Jeremiah}} and took\fnote{\fbackref{39:14} The Heb. lacks \fbib{took}} him from the courtyard of the guard. They handed him over to Ahikam's son Gedaliah, the grandson of Shaphan, to take him home. So he remained among the people.
\passage{Ebed-melech Rewarded}

\v{15}This message from the \divine{Lord} came to Jeremiah while he was confined in the courtyard of the guard: \v{16}``Go and speak to Ebed-melech the Ethiopian: `This is what the \divine{Lord} of the Heavenly Armies, the God of Israel, says: ``Look, I'm going to fulfill my promise against this city for disaster rather than for good, and on that day it will happen before your eyes. \v{17}But I'll deliver you on that day,'' declares the \divine{Lord}. ``You won't be given into the hands of the men you fear. \v{18}For I'll surely deliver you, and you won't fall by the sword. Your life will be spared\fnote{\fbackref{39:18} Lit. \fbib{You will have your life as a spoil of war}} because you trusted me,'' declares the \divine{Lord}.'\,''
\labelchapt{40}
\passage{Jeremiah Chooses to Remain in Judah}

\chapt{40}
\v{1}This is\fnote{\fbackref{40:1} The Heb. lacks \fbib{This is}} the message that came to Jeremiah from the \divine{Lord} after Nebuzaradan the captain of the guard had released him from Ramah, when he was bound in chains, along with all the exiles from Jerusalem and Judah who were being taken into exile in Babylon.

\v{2}The captain of the guard took Jeremiah and told him, ``The \divine{Lord} your God has predicted this disaster on this place. \v{3}And now the \divine{Lord} has brought it about and has done just as he said. Because you people sinned against the \divine{Lord} and didn't obey him, this has happened to you. \v{4}Now, look, I've freed you today from the chains that were on your hands. If you want\fnote{\fbackref{40:4} Lit. \fbib{it is good in your eyes}} to come with me to Babylon, come, and I'll look after you. But if you don't want\fnote{\fbackref{40:4} Lit. \fbib{it is bad in your eyes}} to come with me to Babylon, don't.\fnote{\fbackref{40:4} Lit. \fbib{stop!}} Look, the whole land lies before you, so go wherever it seems good and right for you to go.''

\v{5}When he still did not respond, Nebuzaradan said,\fnote{\fbackref{40:5} The Heb. lacks \fbib{Nebuzaradan said}} ``Return to Ahikam's son Gedaliah, whom the king of Babylon has appointed over the cities of Judah, and remain with him among the people---or go wherever it seems right for you to go.'' Then the captain of the guard gave him an allowance of food and a gift and sent him off. \v{6}Jeremiah came to Ahikam's son Gedaliah at Mizpah, and he remained with him among the people who were left in the land.
\passage{Gedaliah and the Community in Judah}

\v{7}All the leaders of the forces who were in the field along with their men heard that the king of Babylon had appointed Ahikam's son Gedaliah over the men, women, children, and the poor of the land who had not been taken into exile in Babylon. \v{8}Those who came to Gedaliah at Mizpah included Nethaniah's son Ishmael, Jonathan, Kareah's son Jonathan, Tanhumeth's son Seraiah, Ephai's sons from Netophah; and Jezaniah, the son of a man from Maacah. They came along with\fnote{\fbackref{40:8} The Heb. lacks \fbib{came along with}} their men.

\v{9}Ahikam's son Gedaliah, the grandson of Shaphan, swore an oath to them and their men: ``Don't be afraid to serve the Chaldeans. Remain in the land and serve the king of Babylon, and things will go well for you. \v{10}As for me, I'll remain at Mizpah to represent you before\fnote{\fbackref{40:10} Lit. \fbib{to stand before}} the Chaldeans who come to us. As for you, gather wine, summer fruit, and oil. Put it in your containers and live in your cities that you have taken over.''

\v{11}All the Judeans who were in Moab, those with the people in Ammon, those in Edom, and those in all the other\fnote{\fbackref{40:11} The Heb. lacks \fbib{other}} countries also heard that the king of Babylon had left a remnant for Judah and that he had appointed Ahikam's son Gedaliah, the grandson of Shaphan, over them. \v{12}So all the Judeans returned from all the countries where they had been scattered. They came to the land of Judah, to Gedaliah at Mizpah, and they gathered wine and summer fruit in great abundance.
\passage{A Plot against Gedaliah}

\v{13}Kareah's son Jonathan and all leaders of the forces who were in the field came to Gedaliah at Mizpah. \v{14}They told him, ``Are you aware that Baalis, the king of the people of Ammon, has sent Nethaniah's son Ishmael to take your life?'' But Ahikam's son Gedaliah did not believe them.

\v{15}Then Kareah's son Jonathan spoke privately to Gedaliah at Mizpah: ``Let me go kill Nethaniah's son Ishmael, and no one will know. Why should he take your life? Otherwise\fnote{\fbackref{40:15} Lit. \fbib{Then}} all the Judeans who have gathered around you will be scattered, and the remnant of Judah will perish.''

\v{16}Ahikam's son Gedaliah replied to Kareah's son Jonathan, ``Don't do this! You're lying about Ishmael!''
\labelchapt{41}
\passage{Gedaliah is Assassinated}

\chapt{41}
\v{1}In the seventh month, Nethaniah's son Ishmael, the grandson of Elishama, a member of the royal family and one of the chief officers of the king, came to Ahikam's son Gedaliah at Mizpah, along with ten men. While they were dining together there at Mizpah, \v{2}Nethaniah's son Ishmael and the ten men with him got up and killed Ahikam's son Gedaliah, the grandson of Shaphan, with swords and killed the man whom the king of Babylon had appointed over the land. \v{3}Ishmael also struck down all the Judeans who were with him (that is, with Gedaliah) at Mizpah, along with the Chaldean soldiers who were found there.

\v{4}Now on the day after Gedaliah was killed, when as yet no one knew about it,\fnote{\fbackref{41:4} The Heb. lacks \fbib{about it}} \v{5}eighty men from Shechem, from Shiloh, and from Samaria came with their beards shaved, their clothes torn, and their bodies slashed. They had grain offerings and incense with them to present at the \divine{Lord}'s Temple.

\v{6}Nethaniah's son Ishmael went out from Mizpah to meet them, crying as he went. As he met them he told them, ``Come meet with Ahikam's son Gedaliah.'' \v{7}When they reached the middle of the city, Nethaniah's son Ishmael and the men who were with him slaughtered them and threw them into a cistern.\fnote{\fbackref{41:7} Lit. \fbib{slaughtered them to the middle of the cistern}}

\v{8}Ten men who were among\fnote{\fbackref{41:8} Lit. \fbib{found among}} them told Ishmael, ``Don't kill us because we have stores of wheat, barley, oil, and honey hidden in the field. So Ishmael stopped and did not kill them or their companions. \v{9}Ishmael threw the bodies of the men he killed on account of Gedaliah into the cistern that King Asa had made for protection against\fnote{\fbackref{41:9} Lit. \fbib{made on account of}} King Baasha of Israel. That is the same one Nethaniah's son Ishmael filled with those he killed. \v{10}Then Ishmael took captive all the rest of the people who were in Mizpah, including the king's daughters and all the rest of the people in Mizpah over whom Nebuzaradan the captain of the guard had appointed Ahikam's son Gedaliah. Nethaniah's son Ishmael took them captive and then set out to cross over to the Ammonites.
\passage{The Captives Rescued; Ishmael Escapes}

\v{11}Kareah's son Jonathan and all the military leaders who were with him heard about all the terrible things that Nethaniah's son Ishmael had done. \v{12}So they took all the men and went to fight Nethaniah's son Ishmael, and they found him at the large pool that is at Gibeon. \v{13}When all the people who were with Ishmael saw Kareah's son Jonathan and all the military leaders who were with him, they were glad. \v{14}All the people whom Ishmael had taken captive from Mizpah turned around and went back to Kareah's son Jonathan. \v{15}But Nethaniah's son Ishmael and eight other\fnote{\fbackref{41:15} The Heb. lacks \fbib{other}} men escaped from Jonathan and went to the Ammonites. \v{16}Kareah's son Jonathan and all the military leaders who were with him took all the rest of the people from Mizpah whom he had rescued\fnote{\fbackref{41:16} Lit. \fbib{brought back}} from Nethaniah's son Ishmael after he had killed Ahikam's son Gedaliah, including the young men, the soldiers, women, children, and eunuchs\fnote{\fbackref{41:16} Or \fbib{officials}} whom he had rescued from Gibeon. \v{17}They traveled and then stopped at Geruth Chimham near Bethlehem on their way to Egypt \v{18}because of the Chaldeans. They were afraid of the Chaldeans\fnote{\fbackref{41:18} Lit. \fbib{them}} because Nethaniah's son Ishmael had killed Ahikam's son Gedaliah, whom the king of Babylon had appointed over the land.
\labelchapt{42}
\passage{Jeremiah Asked to Pray for the People}

\chapt{42}
\v{1}Then all the military leaders, Kareah's son Jonathan, Hoshaiah's son Jezaniah, and all the people from the least to the greatest approached Jeremiah.\fnote{\fbackref{42:1} The Heb. lacks \fbib{Jeremiah}} \v{2}They told Jeremiah the prophet, ``Please listen to what we have to ask of you. Pray to the \divine{Lord} your God for us and for all these survivors. Indeed, only a few of us remain out of many, as you can see.\fnote{\fbackref{42:2} Lit. \fbib{as your eyes see us}} \v{3}Pray that the \divine{Lord} your God may inform us as to how we should live\fnote{\fbackref{42:3} Lit. \fbib{way we should walk}} and what we should do.''

\v{4}Jeremiah the prophet told them, ``I've heard, and I'm going to pray to the \divine{Lord} your God just as you have requested. Whatever the \divine{Lord} answers, I'll tell you. I won't withhold anything from you.''

\v{5}Then they told Jeremiah, ``May the \divine{Lord} be a true and faithful witness against us if we don't do everything that the \divine{Lord} your God tells us through you.\fnote{\fbackref{42:5} Lit. \fbib{sends to you for us}} \v{6}Whether it seems good or bad, we will obey the \divine{Lord} our God to whom we send you, so it may go well for us. Indeed, we will obey the \divine{Lord} our God.''
\passage{The \divine{Lord}'s Answer through Jeremiah}

\v{7}At the end of ten days a message from the \divine{Lord} came to Jeremiah. \v{8}So he called Kareah's son Jonathan, all the military leaders who were with him, and all the people from the least to the greatest. \v{9}He told them, ``This is what the \divine{Lord} God of Israel says, to whom you sent me to take your request:

\begin{poetry}
\poeml \v{10}`If you will just remain in this land, I'll build you up and not pull you down. I'll plant you and not uproot you, for I'm sorry about the disaster I've brought on you. \v{11}Don't be afraid of the king of Babylon as you have been.\fnote{\fbackref{42:11} Lit. \fbib{whom you fear}} Don't fear him,' declares the \divine{Lord}, `because I am with you to save you and deliver you from his control. \v{12}I'll show you compassion, so he will have compassion on you and return you to your land. \v{13}But if you disobey the \divine{Lord} your God by saying, ``We won't stay in this land,'' \v{14}and you also say, ``No, but we will go to the land of Egypt where we won't see war or hear the sound of the trumpet or hunger for bread, and there we will stay,'' \v{15}then hear this message from the \divine{Lord}, remnant of Judah: `This is what the \divine{Lord} of the Heavenly Armies, the God of Israel, says: ``If you are really determined\fnote{\fbackref{42:15} Lit. \fbib{really set your faces}} to go into Egypt, and you go there to settle, \v{16}the sword that you fear will overtake you there in the land of Egypt. The famine that you dread will pursue you into Egypt, and there you will die. \v{17}All the people who are determined to go into Egypt to settle there will die by the sword, by famine, and by the plague. No one will survive the disaster that I'll bring on them.'' \v{18}For this is what the \divine{Lord} of the Heavenly Armies, the God of Israel, says: `Just as my anger and my wrath were poured out on the inhabitants of Jerusalem, so my wrath will be poured out on you when you enter Egypt. You will be a curse and an object of horror, ridicule, and scorn, and you will never again see this place.' \v{19}The \divine{Lord} has told you, remnant of Judah, `Don't go to Egypt!' So be fully aware that I've warned you, today, \v{20}that you have deceived yourselves. Indeed, you yourselves sent me to the \divine{Lord} your God, saying, `Pray to the \divine{Lord} your God for us, and whatever the \divine{Lord} our God tells us we will do.' \v{21}I've told you today, but you haven't obeyed the \divine{Lord} your God in all that he sent me to tell\fnote{\fbackref{42:21} The Heb. lacks \fbib{tell}} you. \v{22}Now, be fully aware that you will die by the sword, by famine, and by plague in the place where you want to settle.''\fnote{\fbackref{42:22} Lit. \fbib{to go to settle}}
\end{poetry}
\labelchapt{43}
\passage{The Refugees Reject the \divine{Lord}'s Instruction}

\chapt{43}
\v{1}When Jeremiah had finished telling all the people all the words that the \divine{Lord} their God had sent him to tell them---that is, all these words---\v{2}Hoshaiah's son Azariah, Kareah's son Johanan, and all the arrogant men told Jeremiah, ``You're lying! The \divine{Lord} our God didn't send you to say, `Don't go to Egypt to settle there.' \v{3}Indeed, Neriah's son Baruch is inciting you against us in order to give us into the hands of the Chaldeans, to kill us, or to take us into exile to Babylon.''

\v{4}So Kareah's son Johanan, all the military leaders, and all the people did not obey the instructions given by\fnote{\fbackref{43:4} The Heb. lacks \fbib{the instructions given by}} the \divine{Lord} to remain in the land of Judah. \v{5}Kareah's son Johanan and all the military leaders took the entire remnant of Judah that had returned from all the nations where they had been scattered to settle in the land of Judah---\v{6}the young men, the women, the children, the daughters of the king, and everyone whom Nebuzaradan the captain of the guard had left with Ahikam's son Gedaliah, the grandson of Shaphan, along with Jeremiah the prophet and Neriah's son Baruch. \v{7}So they went into the land of Egypt, because they did not obey the \divine{Lord}, and they travelled as far as Tahpanhes.\fnote{\fbackref{43:7} \fbib{Tahpanhes} was a city in the delta region of Egypt.}
\passage{Nebuchadnezzar's Invasion of Egypt Predicted}

\v{8}Then this message from the \divine{Lord} came to Jeremiah in Tahpanhes: \v{9}``Take large stones in your hands, and, in the sight of the men of Judah, bury them in the mortar of the brickwork at the entrance of Pharaoh's house in Tahpanhes. \v{10}Then say to them, `This is what the \divine{Lord} of the Heavenly Armies, the God of Israel, says: ``I'm going to send for my servant Nebuchadnezzar king of Babylon. I'll take him and set his throne over these stones that I've buried, and he will spread his canopy over them. \v{11}He will come and attack the land of Egypt---those meant for death will be put to death, those meant for captivity will be taken captive, and those meant for the sword will be put to the sword. \v{12}He\fnote{\fbackref{43:12} So LXX; Heb. reads \fbib{I}} will set fire to the temples\fnote{\fbackref{43:12} Lit. \fbib{houses; and so throughout the section}} of the gods of Egypt. He will burn their idols\fnote{\fbackref{43:12} Lit. \fbib{them}} and take them captive. He will wrap himself with the land of Egypt like a shepherd wraps himself with a garment, and then he will leave from there in peace. \v{13}He will shatter the pillars of Heliopolis\fnote{\fbackref{43:13} Lit. \fbib{beth-shemesh}; i.e. \fbib{house of the sun}} in the land of Egypt and will burn the temples of the gods of Egypt with fire.''\,'\,''
\labelchapt{44}
\passage{Jeremiah Warns the Refugees in Egypt}

\chapt{44}
\v{1}This is the message that came to Jeremiah for all the Judeans who were living in the land of Egypt, who were living in Migdol, Tahpanhes, Memphis, and in the land of Pathros,\fnote{\fbackref{44:1} I.e. in southern Egypt} saying, \v{2}``This is what the \divine{Lord} of the Heavenly Armies, the God of Israel, says: `You have seen the disaster that I brought on Jerusalem and all the cities of Judah. Look, they're in ruins today, with no one living in them, \v{3}because of the\fnote{\fbackref{44:3} Lit. \fbib{their}} wickedness that they did, provoking me to anger by continuing to offer sacrifices and worship other gods that neither they nor you nor your ancestors had known. \v{4}Yet I sent all my servants the prophets to you again and again,\fnote{\fbackref{44:4} Lit. \fbib{getting up early and sending}} saying, ``Don't do this repulsive thing that I hate.'' \v{5}`But they didn't listen or pay attention\fnote{\fbackref{44:5} Lit. \fbib{turn their ears}} by turning from their wickedness and not offering sacrifices to other gods. \v{6}My wrath and my anger were poured out, and they burned in the cities of Judah and the streets of Jerusalem so that they have become a ruin and a desolate place, as is the case today.'

\v{7}``Now, this is what the \divine{Lord} of the Heavenly Armies, the God of Israel, says: `Why are you doing great harm to yourselves so as to cut off from Judah\fnote{\fbackref{44:7} Lit. \fbib{from the midst of Judah}} man and woman, child and infant from you, leaving yourselves without a remnant? \v{8}And why have you provoked me to anger by the works of your hands,\fnote{\fbackref{44:8} I.e. idols} by offering sacrifices to other gods in the land of Egypt where you have come to settle so that you cut yourselves off and become an object of ridicule and scorn among all the nations of the earth? \v{9}Have you forgotten the evil deeds of your ancestors, the evil deeds of the kings of Judah, the evil deeds of their\fnote{\fbackref{44:9} Lit. \fbib{his}} wives, your evil deeds, and the evil deeds of your wives, that they did in the land of Judah and the streets of Jerusalem? \v{10}To this day they haven't humbled themselves, they haven't shown reverence for the \divine{Lord},\fnote{\fbackref{44:10} The Heb. lacks \fbib{for the \divine{Lord}}} and they haven't lived according to my Law and my statutes that I set before them and before their ancestors.'

\v{11}``Therefore, this is what the \divine{Lord} of the Heavenly Armies, the God of Israel, says: `Look, I've determined to bring disaster on you\fnote{\fbackref{44:11} Lit. \fbib{have set my face against you for disaster}} and to cut off all Judah. \v{12}I'll take the remnant of Judah that determined to go to the land of Egypt to settle there, and all of them\fnote{\fbackref{44:12} The Heb. lacks \fbib{of them}} will come to an end in the land of Egypt. They'll fall by the sword, and they'll come to an end by famine. They'll become a curse, an object of horror, ridicule, and scorn. \v{13}I'll punish those who live in the land of Egypt just as I punished Jerusalem---with the sword, with famine, and with plague. \v{14}Of the remnant of Judah that came into the land of Egypt to settle there, no one will escape or survive to return to the land of Judah where they long to return and live.\fnote{\fbackref{44:14} Lit. \fbib{live there}} Indeed, they won't return, except for some\fnote{\fbackref{44:14} The Heb. lacks \fbib{for some}} refugees.'\,''
\passage{The Refugees Refuse to Repent}

\v{15}Then all the men who knew that their wives were offering sacrifices to other gods and all the women who were standing by---a large group, including all the people who were living in the land of Egypt in Pathros---answered Jeremiah: \v{16}``As for the message that you reported to us in the name of the \divine{Lord}, we won't listen to you! \v{17}Rather, we will keep doing everything that we said we would\fnote{\fbackref{44:17} Lit. \fbib{every word that comes out of our mouths}} by offering sacrifices to the Queen of Heaven\fnote{\fbackref{44:17} I.e. the Near Eastern fertility goddess Ishtar} and by pouring out liquid offerings to her just as we, our ancestors, our kings, and our leaders did in the cities of Judah and the streets of Jerusalem. Then we had plenty of bread, things went well for us, and we didn't experience disaster. \v{18}From the time we stopped offering sacrifices to the Queen of Heaven and pouring out liquid offerings to her, we have lacked everything, and we have been consumed\fnote{\fbackref{44:18} Lit. \fbib{have come to an end}} by the sword and famine. \v{19}Indeed, we\fnote{\fbackref{44:19} I.e. the women} are going to continue offering sacrifices to the Queen of Heaven and pouring out liquid offerings to her. And do you think we have made\fnote{\fbackref{44:19} Lit. \fbib{And have we made}} cakes to represent her or poured out liquid offerings for her without our husbands' approval?''\fnote{\fbackref{44:19} Lit. \fbib{apart from our husbands}}
\passage{Final Judgment Proclaimed}

\v{20}Then Jeremiah spoke a message to all the people, to the young men, to the women, and to all the people who were answering him: \v{21}``As for the sacrifices that you, your ancestors, your kings, your officials, and the people of the land offered in the cities of Judah and the streets of Jerusalem, the \divine{Lord} remembered them, did he not? And they came to his attention, did they not? \v{22}The \divine{Lord} could no longer bear it because of your evil deeds and the repulsive things that you did. So your land has become a ruin and an object of horror and ridicule without an inhabitant, as is the case today. \v{23}Because you offered sacrifices and sinned against the \divine{Lord}, you didn't obey the \divine{Lord} and didn't live according to his Law, his statutes, or his testimonies; therefore, this disaster has happened to you, as is the case today.''

\v{24}Then Jeremiah told all the people and all the women, ``All you people of Judah who are in the land of Egypt, listen to this message from the \divine{Lord}! \v{25}This is what the \divine{Lord} of the Heavenly Armies, the God of Israel, says: `You and your wives have spoken with your mouths and acted with your hands: ``We will certainly carry through\fnote{\fbackref{44:25} Lit. \fbib{cause to stand}} on the vows that we vowed to offer sacrifices to the Queen of Heaven and pour out liquid offerings to her!'' Go ahead, carry through on your vows, and diligently do what you vowed!' \v{26}But\fnote{\fbackref{44:26} Or \fbib{Therefore}} listen to this message from the \divine{Lord}, all you people of\fnote{\fbackref{44:26} The Heb. lacks \fbib{you people of}} Judah who are living in the land of Egypt. `Look, I've sworn by my great name', says the \divine{Lord}, `my name will no longer be invoked by the mouth of any person in the entire land of Egypt, as he says, ``As surely as the Lord \divine{God}\fnote{\fbackref{44:26} MT word usually translated \fbib{\divine{Lord}}} lives{\ldots}''\fnote{\fbackref{44:26} I.e. using the \divine{Lord}'s name to confirm an oath}

\v{27}```Look, I'm watching over them to bring disaster rather than good. Every person of Judah in the land of Egypt will be brought to an end by the sword and by famine until they're completely gone. \v{28}The ones who escape the sword will return from the land of Egypt to the land of Judah, few in number. Then all the remnant of Judah who have come into the land of Egypt to settle will know whose message will stand, mine or theirs. \v{29}This will be a sign to you,' declares the \divine{Lord}, `that I'll punish you in this place so that you may know that my words concerning disaster against you will surely stand.'

\v{30}This is what the \divine{Lord} says: ``Look, I'm going to give Pharaoh Hophra, king of Egypt, into the hands of his enemies and into the hands of those seeking his life, just as I gave Zedekiah king of Judah into the hands of Nebuchadnezzar king of Babylon, his enemy who was seeking his life.''
\labelchapt{45}
\passage{God's Message to Baruch}

\chapt{45}
\v{1}This is\fnote{\fbackref{45:1} The Heb. lacks \fbib{This is}} the message that Jeremiah the prophet spoke to Neriah's son Baruch, when in the fourth year of the reign of\fnote{\fbackref{45:1} The Heb. lacks \fbib{the reign of}} Josiah's son King Jehoiakim of Judah had, at Jeremiah's dictation, written these words in a scroll: \v{2}``This is what the \divine{Lord} God of Israel says to you, Baruch: \v{3}`You have said, ``How terrible for me, for the \divine{Lord} has added sorrow to my pain. I'm weary with my groaning, and I haven't found rest.''\,' \v{4}Say this to him: `This is what the \divine{Lord} says: ``Look! What I've built I'm about to tear down, and what I've planted I'm about to pull up---and this will involve the entire land.'' \v{5}Are you seeking great things for yourself? Don't seek them. Indeed, I'm about to bring disaster on all flesh,' declares the \divine{Lord}, `but your life will be spared\fnote{\fbackref{45:5} Lit. \fbib{life as a spoil of war}} wherever you go.'\,''
\labelchapt{46}
\passage{Prophecies against the Nations}

\chapt{46}
\v{1}This is\fnote{\fbackref{46:1} The Heb. lacks \fbib{This is}} the message from the \divine{Lord} that came to Jeremiah the prophet concerning the nations.
\passage{Prophecies against Egypt: Its Defeat at Carchemish}

\v{2}To Egypt: Concerning the army of King Pharaoh Neco of Egypt, which was encamped by the Euphrates River at Carchemish and which King Nebuchadnezzar of Babylon defeated in the fourth year of the reign of\fnote{\fbackref{46:2} The Heb. lacks \fbib{the reign of}} Josiah's son Jehoiakim, king of Judah.

\begin{poetry}
\poeml \v{3}``Prepare buckler and shield, \\
\poemll    and advance into the battle! \\
\poeml \v{4}Harness the horses! \\
\poemll    Riders, mount up! \\
\poeml Take your\fnote{\fbackref{46:4} The Heb. lacks \fbib{your}} positions with your\fnote{\fbackref{46:4} The Heb. lacks \fbib{your}} helmets! \\
\poemll    Polish lances, \\
\poemll    and put on armor! \\
\poeml \v{5}Why am I seeing this?\fnote{\fbackref{46:5} The Heb. lacks \fbib{this}} \\
\poeml They're terrified, \\
\poemll    they have turned back. \\
\poeml Their warriors are crushed, \\
\poemll    and they take flight. \\
\poeml They don't look back. \\
\poemll    Terror is on every side,'' \\
\poemlll       declares the \divine{Lord}. \\
\poeml \v{6}``The swift cannot flee, \\
\poemll    nor can the strong escape. \\
\poeml In the north, beside the Euphrates River, \\
\poemll    they stumble and fall. \\
\poeml \v{7}Who is this, rising like the Nile, \\
\poemll    like rivers whose waters surge? \\
\poeml \v{8}Egypt is rising like the Nile, \\
\poemll    like rivers whose waters surge. \\
\poeml He says, `I'll rise and cover the land.\fnote{\fbackref{46:8} Or \fbib{earth}} \\
\poemll    I'll destroy the city and its inhabitants.' \\
\poeml \v{9}Horses, get up! \\
\poeml Chariots, drive furiously! \\
\poeml Let the warriors go forward, \\
\poemll    Ethiopia and Put, who carry shields, \\
\poemlll       and the Lydians who handle and bend the bow. \\
\poeml \v{10}That day belongs to the \divine{Lord} of the Heavenly Armies. \\
\poemll    It is a day of vengeance to take vengeance on his foes. \\
\poeml The sword will devour and be satisfied, \\
\poemll    and will drink its fill of their blood. \\
\poeml For the Lord \divine{God} of the Heavenly Armies \\
\poemll    will hold a sacrifice in the land of the north, \\
\poemlll       by the Euphrates river. \\
\poeml \v{11}Go up to Gilead and get balm,\fnote{\fbackref{46:11} \fbib{Balm} was a spice with medicinal uses.} \\
\poemll    virgin daughter of Egypt! \\
\poeml In vain you multiply remedies, \\
\poemll    but there is no healing for you. \\
\poeml \v{12}The nations have heard of your disgrace, \\
\poemll    and your cry of distress fills the earth. \\
\poeml Indeed, one warrior stumbles over another, \\
\poemll    and both of them fall down together.''
\end{poetry}
\passage{Nebuchadnezzar's Conquest of Egypt}

\v{13}This is the message that the \divine{Lord} spoke to Jeremiah the prophet about the coming of King Nebuchadnezzar of Babylon to conquer\fnote{\fbackref{46:13} Lit. \fbib{strike down}} the land of Egypt.

\begin{poetry}
\poeml \v{14}``Announce in Egypt, proclaim in Migdol. \\
\poemll    Proclaim also in Memphis and Tahpanhes. \\
\poeml Say, `Take your positions and be ready, \\
\poemll    for the sword will devour all around you.' \\
\poeml \v{15}Why are your warriors prostrate? \\
\poemll    They don't stand\fnote{\fbackref{46:15} Or \fbib{Why does Apis flee and your bull not stand?}} because the \divine{Lord} has brought them down. \\
\poeml \v{16}They repeatedly stumble and fall. \\
\poemll    They say to each other, `Get up! \\
\poeml Let's go back to our people \\
\poemll    and to the land of our birth, \\
\poemlll       away from the oppressor's sword.' \\
\poeml \v{17}There they'll cry out, \\
\poeml `Pharaoh, king of Egypt, is just a big noise. \\
\poemll    He has let the appointed time pass by.'\fnote{\fbackref{46:17} I.e. he has missed the opportunity} \\
\poeml \v{18}As certainly as I'm alive and living,'' declares the King, \\
\poemll    whose name is the \divine{Lord} of the Heavenly Armies, \\
\poeml ``Indeed, one will come like Tabor among the mountains \\
\poemll    and like Carmel by the sea. \\
\poeml \v{19}Prepare your baggage for exile, \\
\poemll    daughter living in Egypt, \\
\poeml for Memphis will become a desolate place. \\
\poemll    It will become a ruin without inhabitant. \\
\poeml \v{20}Egypt is a beautiful calf,\fnote{\fbackref{46:20} Or \fbib{heifer}} \\
\poemll    but a horsefly from the north is surely coming. \\
\poeml \v{21}Even the mercenary troops in her ranks \\
\poemll    are like a fattened calf. \\
\poeml They too will turn around, \\
\poemll    and will flee together. \\
\poeml They won't stand, \\
\poemll    for the day of their disaster is coming on them, \\
\poemlll       the time of their punishment. \\
\poeml \v{22}Her cry will be like that of a fleeing serpent \\
\poemll    when they come in strength. \\
\poemlll       They're coming to her with axes like woodcutters. \\
\poeml \v{23}They'll cut down her forest, though it's impenetrable,'' \\
\poemll    declares the \divine{Lord}, \\
\poeml ``for they're more numerous than locusts, \\
\poemll    and there are too many of them to count. \\
\poeml \v{24}The daughter of Egypt will be put to shame, \\
\poemll    she will be given into the hands of the people from the north.''
\end{poetry}

\v{25}The \divine{Lord} of the Heavenly Armies, the God of Israel says, ``Look, I'm going to punish Amon of Thebes, Pharaoh, Egypt, its gods and its kings, Pharaoh, and those who trust in him. \v{26}I'll give them to those who are seeking their lives and to King Nebuchadnezzar of Babylon and his officers.\fnote{\fbackref{46:26} Or \fbib{servants}} Then afterwards, Egypt will be inhabited as in times past,'' declares the \divine{Lord}.
\passage{Israel will be Delivered}

\begin{poetry}
\poeml \v{27}``As for you, my servant Jacob, don't be afraid, \\
\poemll    and Israel, don't be dismayed. \\
\poeml For I'll deliver you from a distant place, \\
\poemll    and your descendants from the land of their captivity. \\
\poeml Jacob will return. \\
\poemll    He will be undisturbed and secure, \\
\poemlll       and no one will cause him to fear. \\
\poeml \v{28}``As for you, my servant Jacob, don't be afraid, \\
\poemll    and Israel, don't be dismayed,'' \\
\poemlll       declares the \divine{Lord}, ``for I am with you. \\
\poeml Indeed, I'll make an end of all the nations \\
\poemll    where I scattered you; \\
\poemll    but I won't make an end of you! \\
\poeml I'll discipline you justly, \\
\poemll    but I'll certainly not leave you unpunished.''
\end{poetry}
\labelchapt{47}
\passage{A Prophecy against the Philistines}

\chapt{47}
\v{1}This is\fnote{\fbackref{47:1} The Heb. lacks \fbib{This is}} the message from the \divine{Lord} that came to Jeremiah the prophet concerning the Philistines, before Pharaoh conquered Gaza. \v{2}This is what the \divine{Lord} says:

\begin{poetry}
\poeml ``Look, waters are rising from the north, \\
\poemll    and they'll become an overflowing river. \\
\poeml They'll overflow the land and all that fills it\fnote{\fbackref{47:2} Lit. \fbib{its fullness}}--- \\
\poemll    the city and those that live in it. \\
\poeml People will cry out, \\
\poemll    and all those living in the land will wail. \\
\poeml \v{3}At the sound of the galloping hooves of his horses,\fnote{\fbackref{47:3} Lit. \fbib{his strong ones}} \\
\poemll    at the rumbling of his chariots, \\
\poemlll       the clatter of his wheels, \\
\poeml fathers won't turn back for their\fnote{\fbackref{47:3} The Heb. lacks \fbib{their}} children \\
\poemll    because their hands are weak, \\
\poeml \v{4}for the day is coming to destroy all the Philistines, \\
\poemll    to cut off from Tyre and Sidon \\
\poemlll       every helper who remains. \\
\poeml For the \divine{Lord} is destroying the Philistines, \\
\poemll    the remnant of the coastlands of Caphtor.\fnote{\fbackref{47:4} I.e. Crete and the Aegean islands from which the Philistines came} \\
\poeml \v{5}Baldness\fnote{\fbackref{47:5} I.e. the head was shaved as a rite of mourning} is coming to Gaza. \\
\poemll    Ashkelon is silenced. \\
\poeml Remnant of their valley, \\
\poemll    how long will you gash yourself?\fnote{\fbackref{47:5} I.e. people made cuts on their bodies as an act of mourning} \\
\poeml \v{6}Ah, sword of the \divine{Lord}, \\
\poemll    how long before you are quiet? \\
\poeml Put yourself into your scabbard, \\
\poemll    be at rest, be silent! \\
\poeml \v{7}How can it be quiet, \\
\poemll    when the \divine{Lord} has ordered disaster \\
\poeml to come to Ashkelon and the seashore? \\
\poemll    That's where he has assigned it.''
\end{poetry}
\labelchapt{48}
\passage{A Prophecy against Moab}

\chapt{48}
\v{1}To Moab: This is what the \divine{Lord} of the Heavenly Armies, the God of Israel, says:

\begin{poetry}
\poeml ``How terrible for Nebo, for it's laid waste. \\
\poemll    Kiriathaim is put to shame, it's captured. \\
\poemlll       The fortress is put to shame, it's shattered. \\
\poeml \v{2}The pride of Moab is no more. \\
\poemll    In Heshbon they plotted evil against her: \\
\poeml `Come and let's eliminate her as a nation.' \\
\poemll    Madmen\fnote{\fbackref{48:2} \fbib{Madmen} was a town in Moab. The name sounds like the Heb. word \fbib{silenced}} will also be silenced, \\
\poemlll       and the sword will pursue you. \\
\poeml \v{3}The sound of crying will come from Horonaim, \\
\poemll    devastation and great destruction. \\
\poeml \v{4}Moab will be destroyed; \\
\poemll    her little ones will cry out. \\
\poeml \v{5}Indeed, at the ascent of Luhith \\
\poemll    people will go up with bitter weeping. \\
\poeml At the descent of Horonaim \\
\poemll    the anguished cries over the destruction will be heard. \\
\poeml \v{6}Flee, save your lives, \\
\poemll    and you will be like a wild donkey\fnote{\fbackref{48:6} So LXX; MT reads \fbib{Aroer}} in the desert. \\
\poeml \v{7}But, because you trust in your deeds and your riches, \\
\poemll    you will also be captured. \\
\poeml Chemosh\fnote{\fbackref{48:7} \fbib{Chemosh} was the chief Moabite deity.} will go out into exile, \\
\poemll    along with his priests and officials. \\
\poeml \v{8}A destroyer will come to every town \\
\poemll    and no town will escape. \\
\poeml The valley will be ruined and the plateau destroyed.'' \\
\poemll    This\fnote{\fbackref{48:8} The Heb. lacks \fbib{this}.} is what the \divine{Lord} has said! \\
\poeml \v{9}``Put salt\fnote{\fbackref{48:9} Or \fbib{take wing}} on Moab\fnote{\fbackref{48:9} I.e. as a sign of destruction} \\
\poemll    for she will surely fall. \\
\poeml Her towns will become desolate places,\fnote{\fbackref{48:9} Lit. \fbib{a desolation}} \\
\poemll    without any inhabitants in them. \\
\poeml \v{10}Cursed is the one who is slack \\
\poemll    in doing the \divine{Lord}'s work. \\
\poeml Cursed is the one who holds back his sword \\
\poemll    from shedding\fnote{\fbackref{48:10} The Heb. lacks \fbib{shedding}} blood. \\
\poeml \v{11}Moab has been at ease from his youth. \\
\poemll    He has been undisturbed like wine\fnote{\fbackref{48:11} The Heb. lacks \fbib{like wine}} on its dregs \\
\poemlll       and not poured from vessel to vessel. \\
\poeml He has not gone into exile. \\
\poemll    Therefore, his flavor has remained, \\
\poemlll       and his aroma has not changed.
\end{poetry}

\v{12}``Therefore, look, days are coming,'' declares the \divine{Lord}, ``when I'll send those who tip over vessels\fnote{\fbackref{48:12} The Heb. lacks \fbib{vessels}} to him, and they'll tip him over. They'll empty his vessels and shatter his jars. \v{13}Moab will be ashamed because of Chemosh just as the house of Israel was ashamed because of Bethel, their confidence.

\begin{poetry}
\poeml \v{14}``How can you say, `We're strong warriors, \\
\poemll    and soldiers ready\fnote{\fbackref{48:14} The Heb. lacks \fbib{ready}} for battle'? \\
\poeml \v{15}Moab will be destroyed, \\
\poemll    and the enemy\fnote{\fbackref{48:15} Lit. \fbib{he}} will come up against her cities. \\
\poeml Her finest young men will go down to slaughter,'' \\
\poemll    declares the King, \\
\poemlll       whose name is the \divine{Lord} of the Heavenly Armies. \\
\poeml \v{16}``Moab's disaster is near at hand, \\
\poemll    and his calamity is coming very quickly. \\
\poeml \v{17}Mourn for him, all who live around him, \\
\poemll    and all who know his name. \\
\poeml Say, `Oh how the mighty rod is broken, \\
\poemll    the glorious staff.' \\
\poeml \v{18}``Come down from glory, and sit on parched ground, \\
\poemll    O woman who lives in Dibon, \\
\poeml for the destroyer of Moab will come up \\
\poemll    against you to destroy you. \\
\poemlll       He will destroy your strongholds. \\
\poeml \v{19}Stand by the road and keep watch, \\
\poemll    O woman who lives in Aroer. \\
\poeml Ask the man who flees and the woman who escapes. \\
\poemll    Say, `What happened'? \\
\poeml \v{20}Moab will be put to shame, \\
\poemll    for it will be destroyed. \\
\poeml Wail and cry out. \\
\poemll    Announce by the Arnon that Moab is destroyed. \\
\poeml \v{21}Judgment has come to the plateau:\fnote{\fbackref{48:21} Moab was located on a plateau overlooking the Jordan River.} \\
\poemll    to Holon and Jahzah, \\
\poeml and against Mephaath, \\
\poemll    \v{22}Dibon, Nebo, and Beth-diblathaim, \\
\poeml \v{23}against Kiriathaim, Beth-gamul, and Beth-meon, \\
\poeml \v{24}against Kerioth, Bozrah, \\
\poeml and all the towns in the land of Moab, \\
\poemll    both far and near. \\
\poeml \v{25}The strength\fnote{\fbackref{48:25} Lit. \fbib{horn}} of Moab is cut off, and his arm is broken,'' \\
\poemll    declares the \divine{Lord}. \\
\poeml \v{26}``Make him drunk for he has exalted himself \\
\poemll    against the \divine{Lord}. \\
\poeml Moab will wallow in his vomit, \\
\poemll    and he will be the object of mocking. \\
\poeml \v{27}Wasn't Israel an object of mocking for you? \\
\poemll    Wasn't he treated like a thief,\fnote{\fbackref{48:27} Lit. \fbib{found among thieves}} \\
\poeml so that whenever you spoke about him \\
\poemll    you shook your head in contempt?\fnote{\fbackref{48:27} The Heb. lacks \fbib{in contempt}} \\
\poeml \v{28}Abandon the cities, and live on the cliffs,\fnote{\fbackref{48:28} Or \fbib{among the crags}; i.e. on the face of a cliff} \\
\poemll    you inhabitants of Moab. \\
\poeml Be like a dove that builds a nest \\
\poemll    by the mouth of a cave. \\
\poeml \v{29}We have heard about Moab's pride--- \\
\poemll    he's very proud--- \\
\poeml his haughtiness, his arrogance, \\
\poemll    his insolence, and his conceit.\fnote{\fbackref{48:29} Lit. \fbib{the elevation of his heart}} \\
\poeml \v{30}I know his gall,'' \\
\poemll    declares the \divine{Lord}, \\
\poeml ``and it's futile; \\
\poemll    the boasting that they do is futile. \\
\poeml \v{31}Therefore, I'll wail for Moab, \\
\poemll    and for the whole of Moab I'll cry out, \\
\poemll    for the men of Kir-heres I'll moan. \\
\poeml \v{32}More than the weeping for Jazer,\fnote{\fbackref{48:32} \fbib{Jazer} was a town in Moab.} \\
\poemll    I'll weep for you, vine of Sibmah.\fnote{\fbackref{48:32} \fbib{Sibmah} was a town in Moab noted for its vineyards.} \\
\poeml Your branches spread out to the sea, \\
\poemll    and reached as far as the Sea of Jazer.\fnote{\fbackref{48:32} I.e. the Dead Sea} \\
\poeml On your summer fruit and grapes\fnote{\fbackref{48:32} Lit. \fbib{your vintage}} \\
\poemll    the destroyer will fall. \\
\poeml \v{33}Gladness and rejoicing will be taken away \\
\poemll    from the fruitful land.\fnote{\fbackref{48:33} Or \fbib{from Carmel}} \\
\poeml From the land of Moab I'll cause the wine \\
\poemll    in the wine presses to stop flowing.\fnote{\fbackref{48:33} The Heb. lacks \fbib{flowing}} \\
\poeml The workers won't tread\fnote{\fbackref{48:33} Lit. \fbib{He won't tread}} the grapes with a loud shout.\fnote{\fbackref{48:33} Lit. \fbib{with a shout, shout}} \\
\poemll    There will be no shout!
\end{poetry}

\v{34}``From the cry of Heshbon, to Elealeh, to Jahaz they have lifted up their voice. From Zoar to Horonaim and to Eglath-shelishiyah, even the waters of Nimrim will become a desolate place. \v{35}In Moab,'' declares the \divine{Lord}, ``I'll put an end to the one who offers a burnt offering on the high place and to the one who burns incense to his gods. \v{36}Therefore my heart wails for Moab like flutes\fnote{\fbackref{48:36} I.e. flutes were played as a part of mourning for the dead} and my heart wails for the men of Kir-heres like flutes. Therefore they'll lose the abundance they produced. \v{37}Indeed every head will be bald\fnote{\fbackref{48:37} I.e. heads were shaved as a sign of mourning} and every beard cut short.\fnote{\fbackref{48:37} I.e. beards were cut short as a sign of mourning} There will be gashes on all the hands\fnote{\fbackref{48:37} I.e. people cut themselves as a sign of mourning} and sackcloth on the loins. \v{38}On all the housetops of Moab and in the streets there will be nothing but mourning, for I'll break Moab like a vessel that no one wants,'' declares the \divine{Lord}. \v{39}``How it will be shattered! How they'll wail! How Moab will turn his back in shame! Moab will be an object of ridicule and terror to all those around him.''

\begin{poetry}
\poeml \v{40}For this is what the \divine{Lord} says: \\
\poeml ``Look, like an eagle one will fly swiftly \\
\poemll    and spread his wings against Moab. \\
\poeml \v{41}The towns\fnote{\fbackref{48:41} Or \fbib{Kerioth}} will be captured \\
\poemll    and the strongholds seized. \\
\poeml On that day the hearts of the warriors of Moab \\
\poemll    will be like the heart of a woman in labor. \\
\poeml \v{42}Moab will be destroyed as a nation\fnote{\fbackref{48:42} Lit. \fbib{from being a nation}} \\
\poemll    because he exalted himself against the \divine{Lord}. \\
\poeml \v{43}Terror, pit, and trap will be used\fnote{\fbackref{48:43} The Heb. lacks \fbib{used}} against you \\
\poemll    who live in Moab,'' \\
\poemlll       declares the \divine{Lord}. \\
\poeml \v{44}``The one who flees from the terror \\
\poemll    will fall into a pit. \\
\poeml And the one who comes up out of the pit \\
\poemll    will be caught in a trap. \\
\poeml For I'll bring upon her, that is upon Moab, \\
\poemll    the time of her\fnote{\fbackref{48:44} Lit. \fbib{their}} punishment,'' \\
\poemlll       declares the \divine{Lord}. \\
\poeml \v{45}``The fugitives will stand without strength \\
\poemll    in the shadow of Heshbon, \\
\poeml for fire will go out from Heshbon \\
\poemll    and a flame from the middle of Sihon. \\
\poeml It will devour the forehead of Moab \\
\poemll    and the heads of the rebellious people.\fnote{\fbackref{48:45} Lit. \fbib{sons of tumult}} \\
\poeml \v{46}How terrible for you, Moab! \\
\poemll    The people of Chemosh\fnote{\fbackref{48:46} \fbib{Chemosh} was the chief deity of Moab.} will perish. \\
\poeml Indeed, your sons will be taken into captivity, \\
\poemll    and your daughters as well.\fnote{\fbackref{48:46} Lit. \fbib{daughters into captivity}} \\
\poeml \v{47}But I'll restore the fortunes of Moab in the latter days,'' \\
\poemll    declares the \divine{Lord}. \\
\poeml This concludes the judgment on Moab.
\end{poetry}
\labelchapt{49}
\passage{Prophecies against Ammon}

\chapt{49}
\v{1}To the people of Ammon:

This is what the \divine{Lord} says:

\begin{poetry}
\poeml ``Does Israel have no sons? \\
\poemll    Does he have no heir? \\
\poeml Why then has Milcom\fnote{\fbackref{49:1} \fbib{Milcom} or Molech was the chief deity of the Ammonites.} taken possession of Gad, \\
\poemll    and his people settled in its towns? \\
\poeml \v{2}Therefore, look, the time is\fnote{\fbackref{49:2} Lit. \fbib{the days are}} coming,'' \\
\poemll    declares the \divine{Lord}, \\
\poeml ``when I'll cause a battle cry to be heard in \\
\poemll    Rabbah\fnote{\fbackref{49:2} \fbib{Rabbah} was the capital city of Ammon.} of the Ammonites. \\
\poeml It will become a desolate mound, \\
\poemll    and its towns will be burned with fire. \\
\poeml Israel will take possession of those who possessed him,'' \\
\poemll    says the \divine{Lord}. \\
\poeml \v{3}``Wail, Heshbon, because Ai is destroyed. \\
\poemll    Cry out, daughters of Rabbah, \\
\poemlll       put on sackcloth and lament. \\
\poeml Run back and forth inside the walls, \\
\poemll    for Milcom is going into exile \\
\poemlll       along with his priests and his princes. \\
\poeml \v{4}Why do you boast in your valleys? \\
\poeml Your valley is flowing away, faithless daughter, \\
\poemll    who trusted in her treasures, \\
\poemlll       saying, `Who will come against me?' \\
\poeml \v{5}Look, I'm bringing terror on you from all around you,'' \\
\poemll    declares the Lord GOD of the Heavenly Armies. \\
\poeml ``You will be driven out, fleeing recklessly,\fnote{\fbackref{49:5} Lit. \fbib{each one before him}} \\
\poemll    and there will be no one to gather the fugitives. \\
\poeml \v{6}But afterwards I'll restore the fortunes of \\
\poemll    the people of Ammon,''\fnote{\fbackref{49:6} Lit. \fbib{the sons of Ammon}} \\
\poemlll       declares the \divine{Lord}.
\end{poetry}
\passage{Prophecies against Edom}

\v{7}To Edom:

\begin{poetry}
\poeml This is what the \divine{Lord} of the Heavenly Armies says: \\
\poemll    ``Is there no longer wisdom in Teman? \\
\poeml Has counsel perished among the prudent? \\
\poemll    Is their wisdom gone? \\
\poeml \v{8}Flee, turn around! \\
\poemll    Go to a remote place to stay, \\
\poeml residents of Dedan! \\
\poemll    For I'll bring Esau's\fnote{\fbackref{49:8} I.e. the Edomites were descendants of Esau} disaster on him \\
\poemll    at the time when I punish him. \\
\poeml \v{9}If the grape harvesters came to you, \\
\poemll    would they not leave gleanings? \\
\poeml If thieves came at night, they would destroy \\
\poemll    only until they had enough. \\
\poeml \v{10}But I'll strip Esau bare. \\
\poemll    I'll uncover his hiding places so he cannot conceal himself. \\
\poeml His offspring, his relatives, \\
\poemll    and his neighbors will be destroyed, \\
\poemlll       and he will no longer exist. \\
\poeml \v{11}Leave your orphans. I'll keep them alive. \\
\poemll    Let your widows trust in me.''
\end{poetry}

\v{12}For this is what the \divine{Lord} says: ``Look, those who don't deserve\fnote{\fbackref{49:12} Lit. \fbib{it was not their judgment}} to drink the cup will surely drink it, and will you actually go unpunished? You won't go unpunished! You will certainly drink it!\fnote{\fbackref{49:12} The Heb. lacks \fbib{it}} \v{13}Indeed, I've sworn by myself,'' declares the \divine{Lord}, ``that Bozrah will become an object of horror and scorn, a waste, and an object of ridicule. All her towns will become perpetual ruins.''

\begin{poetry}
\poeml \v{14}I've heard a message from the \divine{Lord}, \\
\poemll    and a messenger has been sent \\
\poemlll       among the nations: \\
\poeml ``Gather together and come up against her, \\
\poemll    and rise up to fight. \\
\poeml \v{15}Indeed, I'll make you the least of the nations, \\
\poemll    despised among men. \\
\poeml \v{16}The terror you cause and the pride of your heart have deceived you. \\
\poeml You who live in hidden places in the rocks, \\
\poemll    who hold on to the heights of the hill, \\
\poeml although you make your nest high like the eagle, \\
\poemll    I'll bring you down from there,'' \\
\poemlll       declares the \divine{Lord}.
\end{poetry}

\v{17}``Edom will become an object of horror. Everyone who passes by her will be horrified and will scoff\fnote{\fbackref{49:17} Lit. \fbib{hiss}; i.e. hissing was an expression of contempt} because of all her wounds. \v{18}Just like the overthrow of Sodom and Gomorrah and their\fnote{\fbackref{49:18} Lit. \fbib{her}} neighbors,'' says the \divine{Lord}, ``no one will live there. No human being will reside in it. \v{19}Look, like a lion comes up from the thicket of the Jordan to a pasture that grows year round,\fnote{\fbackref{49:19} Lit. \fbib{a perpetual pasture}} so I'll drive them\fnote{\fbackref{49:19} So LXX; Heb. \fbib{him}; i.e. the Edomites} away from her in an instant, and I'll appoint whomever is chosen over her. Indeed, who is like me? Who gives me counsel? Who is the shepherd who will stand against me?'' \v{20}Therefore, hear the plan that the \divine{Lord} has made against Edom, and the strategy that he devised against the inhabitants of Teman. Surely he will drag the little ones of the flock away. Surely their pasture will be desolate because of them. \v{21}The earth will quake at the sound of their fall. A cry---it's her voice---is heard at the Reed\fnote{\fbackref{49:21} So MT; LXX lacks \fbib{Reed}} Sea. \v{22}Look, he will rise up and fly swiftly like an eagle. He will spread his wings against Bozrah, and on that day the hearts of the warriors of Edom will be like the heart of a woman in labor.
\passage{Prophecies against Damascus}

\v{23}To Damascus:

\begin{poetry}
\poeml ``Hamath and Arpad will be humiliated. \\
\poemll    Their courage melts because they have heard bad news. \\
\poemlll       There is anxiety like\fnote{\fbackref{49:23} Lit. \fbib{by}} the sea that cannot be calmed. \\
\poeml \v{24}Damascus will become weak. \\
\poemll    She will turn to flee, but panic will seize her. \\
\poeml Distress and anguish will take hold of her \\
\poemll    like that of\fnote{\fbackref{49:24} The Heb. lacks \fbib{that of}} a woman giving birth. \\
\poeml \v{25}Why\fnote{\fbackref{49:25} Lit. \fbib{How}} is the famous city,\fnote{\fbackref{49:25} Lit. \fbib{city of praise}} the joyful town, \\
\poemll    not abandoned? \\
\poeml \v{26}Therefore her young men will fall in her streets, \\
\poemll    and all her soldiers will be silenced on that day,'' \\
\poemlll       declares the \divine{Lord} of the Heavenly Armies. \\
\poeml \v{27}``I'll kindle a fire in the wall of Damascus, \\
\poemll    and it will devour the strongholds of Ben-hadad.''
\end{poetry}
\passage{Prophecies against Kedar and Hazor}

\v{28}To Kedar and the kingdoms of Hazor that King Nebuchadnezzar of Babylon destroyed:

\begin{poetry}
\poeml This is what the \divine{Lord} says: \\
\poeml ``Arise, go against Kedar! \\
\poemll    Plunder the people of the east! \\
\poeml \v{29}Take their tents and their flocks, \\
\poemll    their tent curtains and all their goods. \\
\poeml Take their camels away from them. \\
\poemll    Cry out against them, `Terror is all around!' \\
\poeml \v{30}Flee! Run away quickly! \\
\poemll    Go to a remote place to stay, residents of Hazor,'' \\
\poemlll       declares the \divine{Lord}. \\
\poeml ``For King Nebuchadnezzar of Babylon has formed a plan \\
\poemll    and devised a strategy against them. \\
\poeml \v{31}``Arise, go up against a nation at ease, living securely,'' \\
\poemll    declares the \divine{Lord}, \\
\poemlll       ``without gates or bars, living alone. \\
\poeml \v{32}Their camels will become booty, \\
\poemll    their many herds will become spoil. \\
\poeml I'll scatter to the winds \\
\poemll    those who shave the corners of their beards,\fnote{\fbackref{49:32} Lit. \fbib{cut off of the side}} \\
\poeml and I'll bring disaster on them from every side,'' \\
\poemll    declares the \divine{Lord}. \\
\poeml \v{33}``Hazor will become a dwelling place for jackals, \\
\poemll    a perpetual wasteland. \\
\poeml No one will live there; \\
\poemll    no human being will reside in it.''
\end{poetry}
\passage{Prophecies against Elam}

\v{34}This is what came as a message from the \divine{Lord} to Jeremiah the prophet about Elam at the beginning of the reign of King Zedekiah of Judah:

\begin{poetry}
\poeml \v{35}This is what the \divine{Lord} of the Heavenly Armies says: \\
\poeml ``Look, I'm going to break the bow of Elam, \\
\poemll    the finest of their troops. \\
\poeml \v{36}I'll bring the four winds against Elam \\
\poemll    from the four corners of the heavens, \\
\poeml and I'll scatter them to all these winds. \\
\poemll    There will be no nation to which the exiles \\
\poemlll       from Elam won't go. \\
\poeml \v{37}I'll terrify Elam before their enemies \\
\poemll    and before those who seek to kill them. \\
\poeml I'll bring on them disaster and become fiercely angry at them,'' \\
\poemll    declares the \divine{Lord}. \\
\poeml ``I'll send the sword after them, \\
\poemll    until I've made an end of them. \\
\poeml \v{38}I'll put my throne in Elam, \\
\poemll    and destroy the king and the officials there,'' \\
\poemlll       declares the \divine{Lord}. \\
\poeml \v{39}``But in the latter days I'll restore \\
\poemll    the fortunes of Elam,'' \\
\poemlll       declares the \divine{Lord}.
\end{poetry}
\labelchapt{50}
\passage{Prophecies against Babylon}

\chapt{50}
\v{1}This is\fnote{\fbackref{50:1} The Heb. lacks \fbib{This is}} the message that the \divine{Lord} spoke through the prophet Jeremiah about Babylon, the land of the Chaldeans.

\begin{poetry}
\poeml \v{2}``Declare and proclaim among the nations. \\
\poemll    Lift up a banner and proclaim. \\
\poeml Don't conceal anything.\fnote{\fbackref{50:2} The Heb. lacks \fbib{anything}} \\
\poemll    Say, `Babylon will be captured. \\
\poeml Bel\fnote{\fbackref{50:2} \fbib{Bel} was another name for \fbib{Marduk}, the sun god of Babylon} will be disgraced, \\
\poemll    and Marduk will be destroyed. \\
\poeml Her idols will be disgraced, \\
\poemll    and her filthy images will be destroyed.' \\
\poeml \v{3}For a nation from the north will go up against her. \\
\poemll    It will make her land into an object of horror, \\
\poemlll       and no one will live in it. \\
\poeml Both people and animals will wander off, \\
\poemll    and they'll leave. \\
\poeml \v{4}In those days, and at that time,'' \\
\poemll    declares the \divine{Lord}, \\
\poeml ``the people of Israel will come together \\
\poemll    with the people of Judah. \\
\poeml They'll be weeping as they travel along, \\
\poemll    and they'll be seeking the \divine{Lord} their God. \\
\poeml \v{5}They'll ask the way to Zion, \\
\poemll    turning their faces in that direction. \\
\poeml They'll come\fnote{\fbackref{50:5} So with LXX; MT reads \fbib{Come!}} and join themselves to the \divine{Lord} \\
\poemll    in an everlasting covenant that won't be forgotten. \\
\poeml \v{6}My people have become lost sheep. \\
\poemll    Their shepherds have led them astray, \\
\poemlll       turning them toward the mountains. \\
\poeml They go from mountain to hill. \\
\poemll    They have forgotten their resting place. \\
\poeml \v{7}All who find them devour them, \\
\poemll    but their enemies say, `We're not guilty, \\
\poeml because they have sinned against \\
\poemll    the \divine{Lord}, the habitation of righteousness, \\
\poemlll       the \divine{Lord}, the hope of their ancestors.' \\
\poeml \v{8}Move away from the middle of Babylon, \\
\poemll    and go out of the land of the Chaldeans. \\
\poemlll       Be like male goats at the head\fnote{\fbackref{50:8} Lit. \fbib{in front of}} of the flock. \\
\poeml \v{9}Indeed, I'm going to stir up \\
\poemll    and bring against Babylon \\
\poeml a great company of nations \\
\poemll    from the land of the north. \\
\poeml They'll deploy for battle against her, \\
\poemll    and from there she will be captured. \\
\poeml Their arrows will be like a skilled warrior; \\
\poemll    they won't miss their targets.\fnote{\fbackref{50:9} Lit. \fbib{won't return empty-handed}} \\
\poeml \v{10}The Chaldeans will become plunder, \\
\poemll    and all who plunder them will get more than enough,'' \\
\poemlll       declares the \divine{Lord}. \\
\poeml \v{11}``Though you rejoice, though you exult, \\
\poemll    you plunderers of my inheritance, \\
\poeml though you skip around like a heifer in the grass\fnote{\fbackref{50:11} So LXX; MT reads \fbib{like a threshing heifer}} \\
\poemll    and neigh like stallions, \\
\poeml \v{12}your mother will be greatly devastated, \\
\poemll    she who gave birth to you will be ashamed. \\
\poeml She will become the least of the nations, \\
\poemll    a wilderness, a dry land, and a desert. \\
\poeml \v{13}Because of the anger of the \divine{Lord} \\
\poemll    she won't be inhabited, \\
\poemll    but will be utterly devastated. \\
\poeml Everyone who passes by Babylon will be horrified \\
\poemll    and will scoff\fnote{\fbackref{50:13} Lit. \fbib{hiss}; i.e. hissing was an expression of contempt} because of all her wounds. \\
\poeml \v{14}Deploy the troops all around Babylon. \\
\poemll    All who bend the bow, shoot at her \\
\poeml and spare no arrows, \\
\poemll    for she has sinned against the \divine{Lord}. \\
\poeml \v{15}Raise a battle cry against her on every side. \\
\poemll    She has surrendered,\fnote{\fbackref{50:15} Lit. \fbib{she has given her hand}} her pillars have fallen, \\
\poemlll       her walls are thrown down. \\
\poeml For this is the vengeance of the \divine{Lord}. \\
\poemll    Take vengeance on her; \\
\poemlll       as she has done, do to her. \\
\poeml \v{16}Eliminate from Babylon the one who plants seeds \\
\poemll    and the one who uses the sickle at harvest time. \\
\poeml Because of the oppressor's sword, let each one turn \\
\poemll    toward his own people and flee to his own land.''
\end{poetry}
\passage{Hope for Israel}

\v{17}``Israel is a scattered flock, driven out by lions. The first to devour him was the king of Assyria, and then afterward\fnote{\fbackref{50:17} The Heb. lacks \fbib{afterward}} King Nebuchadnezzar of Babylon gnawed\fnote{\fbackref{50:17} The Heb. lacks \fbib{gnawed}} his bones. \v{18}Therefore this is what the \divine{Lord} of the Heavenly Armies, the God of Israel, says: `Look, I'm about to judge the king of Babylon and his land, just as I've judged the king of Assyria. \v{19}I'll bring Israel back to his pasture. He will graze on Carmel, on Bashan, on Mt. Ephraim, and on Gilead---his hunger will be satisfied. \v{20}In those days and at that time,' declares the \divine{Lord}, `the iniquity of Israel will be searched for, but there will be none; and the sin of Judah, but none will be found, because I'll pardon those I leave as a remnant.'\,''
\passage{God's Judgment on Babylon}

\begin{poetry}
\poeml \v{21}``Go up against the land of Merathaim\fnote{\fbackref{50:21} \fbib{Merathaim} was an area in southern Mesopotamia; the Heb. word means \fbib{double rebellion}} \\
\poemll    and the inhabitants of Pekod.\fnote{\fbackref{50:21} \fbib{Pekod} was a region in southern Mesopotamia; the Heb. word means \fbib{punishment}} \\
\poeml Kill them with swords, and completely destroy them,'' \\
\poemll    declares the \divine{Lord}, \\
\poemlll       ``and do everything that I've commanded you. \\
\poeml \v{22}The noise of battle is in the land, \\
\poemll    and great destruction. \\
\poeml \v{23}How the hammer of all the earth is cut off and broken! \\
\poemll    How Babylon has become a horror among the nations! \\
\poeml \v{24}I'll set a trap for you, \\
\poemll    and you will be caught, Babylon, \\
\poemlll       but you don't realize it. \\
\poeml You will be found and also seized, \\
\poemll    because you challenged the \divine{Lord}! \\
\poeml \v{25}``The \divine{Lord} will open his armory, \\
\poemll    and bring out the weapons of his anger. \\
\poeml Indeed, a work of the Lord GOD\fnote{\fbackref{50:25} Heb. \fbib{Yahweh,} usually translated \fbib{\divine{Lord}}} of the Heavenly Armies \\
\poemll    will be in the land of the Chaldeans. \\
\poeml \v{26}Come to her from afar.\fnote{\fbackref{50:26} Lit. \fbib{from the end}} \\
\poemll    Open up her barns. \\
\poeml Pile her up like heaps of grain, \\
\poemll    and completely destroy her. \\
\poemlll       Don't leave any survivors. \\
\poeml \v{27}Put all her bulls to the sword, \\
\poemll    let them go down to the slaughter. \\
\poeml How terrible for them because their day has come, \\
\poemll    the time of their judgment. \\
\poeml \v{28}``The sound of fugitives and refugees \\
\poemll    will come from the land of Babylon \\
\poeml to declare in Zion the vengeance of the \divine{Lord} our God, \\
\poemll    vengeance for his Temple. \\
\poeml \v{29}``Summon many to Babylon, \\
\poemll    all those who bend the bow. \\
\poeml Camp all around her, \\
\poemll    let no one escape. \\
\poeml Repay her according to her deeds. \\
\poemll    Do to her just as she has done. \\
\poeml For she has behaved arrogantly against the Lord, \\
\poemll    against the Holy One of Israel. \\
\poeml \v{30}Therefore, her warriors will fall in her streets, \\
\poemll    and all her soldiers will be silenced on that day,'' \\
\poemlll       declares the \divine{Lord}. \\
\poeml \v{31}``Look, I'm against you, arrogant one,'' \\
\poemll    declares the \divine{Lord} God of the Heavenly Armies. \\
\poeml ``Indeed your day is coming, \\
\poemll    the time of your judgment. \\
\poeml \v{32}The arrogant one will stumble and fall, \\
\poemll    and there will be no one to lift him up. \\
\poeml I'll set fire to his cities, \\
\poemll    and it will devour everything around him.'' \\
\poeml \v{33}This is what the \divine{Lord} of the Heavenly Armies says: \\
\poeml ``The people of\fnote{\fbackref{50:33} Lit. \fbib{sons of}} Israel are oppressed, \\
\poemll    along with the people of\fnote{\fbackref{50:33} Lit. \fbib{sons of}} Judah. \\
\poeml All their captors have held on to them \\
\poemll    and refused to let them go. \\
\poeml \v{34}Their Redeemer\fnote{\fbackref{50:34} I.e. the one who pleads their case in a court of law} is strong, \\
\poemll    the \divine{Lord} of the Heavenly Armies is his name. \\
\poeml He will vigorously plead their case \\
\poemll    in order to bring rest to the earth, \\
\poemlll       but turmoil to the inhabitants of Babylon. \\
\poeml \v{35}A sword against the Chaldeans,'' \\
\poemll    declares the \divine{Lord}, \\
\poeml ``and against the inhabitants of Babylon, \\
\poemll    against her officials and her wise men. \\
\poeml \v{36}A sword against the diviners.\fnote{\fbackref{50:36} Lit. \fbib{empty talkers}; a pun on the Babylonian word for these priests} \\
\poemll    They'll be made fools. \\
\poeml A sword against her warriors. \\
\poemll    They'll be shattered. \\
\poeml \v{37}A sword against her horses, against her chariots,\fnote{\fbackref{50:37} Lit. \fbib{against his horses, against his chariots}} \\
\poemll    and against all the foreign troops\fnote{\fbackref{50:37} Lit. \fbib{mixed peoples}} in her midst. \\
\poeml They'll become women. \\
\poemll    A sword against her treasures. \\
\poemll    They'll be plundered. \\
\poeml \v{38}A drought against her waters. \\
\poemll    They'll dry up. \\
\poeml For it's a land of idols, \\
\poemll    and they go mad over their terrifying images. \\
\poeml \v{39}Therefore the desert creatures \\
\poemll    along with hyenas will live there. \\
\poeml They'll live in it with ostriches, \\
\poemll    but people won't live in it again. \\
\poemlll       They won't inhabit it from generation to generation. \\
\poeml \v{40}Just as when God overthrew Sodom, \\
\poemll    Gomorrah, and their neighbors,'' \\
\poemlll       declares the \divine{Lord}, \\
\poeml ``so also no one will live there. \\
\poemll    No human being will reside in it. \\
\poeml \v{41}``Look, people are coming from the north. \\
\poemll    A great nation and many kings will be stirred up \\
\poemlll       from the ends of the earth. \\
\poeml \v{42}They grab bow and spear. \\
\poemll    They're cruel and show no mercy. \\
\poeml Their sound roars like the sea, \\
\poemll    as they ride on horses \\
\poeml deployed like men ready for battle \\
\poemll    against you, daughter of Babylon. \\
\poeml \v{43}The king of Babylon has heard the news about them, \\
\poemll    and his hands hang limp. \\
\poeml Distress has seized him, \\
\poemll    like a woman in labor.
\end{poetry}

\v{44}``Look, like a lion comes up from the thicket of the Jordan to a pasture that grows year round,\fnote{\fbackref{50:44} Lit. \fbib{a perpetual pasture}} so I'll drive them away from her in an instant, and I'll appoint whomever is chosen over her. Indeed, who is like me? Who gives me counsel? Who is the shepherd who will stand against me?'' \v{45}Therefore, hear the plan that the \divine{Lord} has made against Babylon, and the strategy that he devised against the land of the Chaldeans. Surely they'll drag the little ones of the flock away. Surely their pasture will be desolate because of them. \v{46}At the shout that Babylon has been seized, the earth will be shaken, and the cry will be heard among the nations.
\labelchapt{51}
\passage{Judgment against Babylon}

\chapt{51}
\v{1}This is what the \divine{Lord} says:

\begin{poetry}
\poeml ``Look, I'm going to stir up a destroying wind \\
\poemll    against Babylon and the inhabitants of Leb-kamai.\fnote{\fbackref{51:1} I.e. a cryptogram for Chaldea} \\
\poeml \v{2}I'll send foreigners to Babylon, \\
\poemll    and they'll winnow her, \\
\poemlll       and devastate\fnote{\fbackref{51:2} Lit. \fbib{empty out}} her land. \\
\poeml They'll come against her from every side \\
\poemll    on the day of her\fnote{\fbackref{51:2} The Heb. lacks \fbib{her}} disaster. \\
\poeml \v{3}Don't let the archer\fnote{\fbackref{51:3} Lit. \fbib{one who bends the bow}} bend the bow; \\
\poemll    don't let him rise up in his armor. \\
\poeml Don't spare her young men. \\
\poemll    Completely destroy her entire army. \\
\poeml \v{4}The slain will fall in the land of Chaldea, \\
\poemll    pierced through in her streets. \\
\poeml \v{5}Indeed, Israel and Judah haven't been \\
\poemll    abandoned\fnote{\fbackref{51:5} Lit. \fbib{widowed}} by their\fnote{\fbackref{51:5} Lit. \fbib{his}} God, \\
\poeml by the \divine{Lord} of the Heavenly Armies, \\
\poemll    although their land is full of guilt \\
\poemlll       against the Holy One of Israel.'' \\
\poeml \v{6}Flee from Babylon,\fnote{\fbackref{51:6} Lit. \fbib{from the midst of Babylon}} \\
\poemll    and each of you, escape with your life! \\
\poeml Don't be destroyed\fnote{\fbackref{51:6} Or \fbib{silent}} because of her guilt, \\
\poemll    for it's time for the \divine{Lord}'s vengeance. \\
\poemlll       He is paying back what is due to her. \\
\poeml \v{7}Babylon was a golden cup in the \divine{Lord}'s hand, \\
\poemll    making the whole earth drunk. \\
\poeml The nations drank her wine, \\
\poemll    therefore the nations have gone mad. \\
\poeml \v{8}Suddenly, Babylon fell down and was shattered. \\
\poemll    Wail for her! \\
\poeml Bring balm for her wound, \\
\poemll    perhaps she will be healed. \\
\poeml \v{9}We tried to heal Babylon, \\
\poemll    but she wouldn't be healed. \\
\poeml Leave her, and let each of us go to his own country. \\
\poemll    For her judgment has reached to the heavens, \\
\poemlll       and is lifted up to the sky. \\
\poeml \v{10}The \divine{Lord} will vindicate us. \\
\poemll    Come! Let us declare the work of the \divine{Lord} our God in Zion. \\
\poeml \v{11}Sharpen the arrows, fill the quivers! \\
\poeml The \divine{Lord} has stirred up the spirit \\
\poemll    of the kings of the Medes--- \\
\poemlll       he has decided to destroy Babylon. \\
\poeml Indeed, it's the \divine{Lord}'s vengeance, \\
\poemll    vengeance for his Temple. \\
\poeml \v{12}Lift up the battle standard\fnote{\fbackref{51:12} I.e. Give the signal to attack} against Babylon's walls. \\
\poemll    Strengthen the guard; \\
\poemlll       post watchmen.\fnote{\fbackref{51:12} Or \fbib{guards}} \\
\poeml Set men in position for an ambush. \\
\poemll    For the \divine{Lord} will both plan and carry out what he has \\
\poemlll       declared against the inhabitants of Babylon. \\
\poeml \v{13}You who live beside many waters, \\
\poemll    rich in treasures, \\
\poeml your end has come, \\
\poemll    your life thread is cut.\fnote{\fbackref{51:13} Or \fbib{the measure of your unjust gain}} \\
\poeml \v{14}The \divine{Lord} of the Heavenly Armies \\
\poemll    has sworn by himself: \\
\poeml ``I'll surely fill you with soldiers\fnote{\fbackref{51:14} Lit. \fbib{men}} like a swarm of locusts, \\
\poemll    and they'll sing songs of victory over you.''
\passage{Praise to the God of Jacob}
\poeml \v{15}He made the earth by his power. \\
\poemll    He established the world by his wisdom, \\
\poemlll       and by his understanding he spread out the heavens. \\
\poeml \v{16}When his voice sounds, there is thunder from \\
\poemll    the waters of heaven, \\
\poeml and he makes clouds rise up \\
\poemll    from the ends of the earth. \\
\poeml He makes lightning for the rain \\
\poemll    and brings wind out of his storehouses. \\
\poeml \v{17}Everyone is stupid\fnote{\fbackref{51:17} I.e. like a beast} and without knowledge. \\
\poemll    Every goldsmith is put to shame by his own idols, \\
\poeml for his images are false,\fnote{\fbackref{51:17} Lit. \fbib{deception}} \\
\poemll    and there is no life in them. \\
\poeml \v{18}They're worthless, a work of mockery, \\
\poemll    and when the time of punishment comes,\fnote{\fbackref{51:18} Lit. \fbib{at the time of their punishment}} \\
\poemlll       they'll perish. \\
\poeml \v{19}The Portion of Jacob\fnote{\fbackref{51:19} I.e. \fbib{Portion of Jacob} is a name for the \fbib{\divine{Lord}}} is not like these. \\
\poemll    He made everything, \\
\poeml including the tribe of his inheritance. \\
\poemll    The \divine{Lord} of the Heavenly Armies is his name.
\passage{The \divine{Lord}'s Instrument of Judgment}
\poeml \v{20}``You are my war-club and \\
\poemll    weapons of war. \\
\poeml I'll smash nations with you \\
\poemll    and destroy kingdoms with you. \\
\poeml \v{21}I'll smash the horse and its rider with you. \\
\poemll    I'll smash the chariot and its rider with you. \\
\poeml \v{22}I'll smash man and woman with you. \\
\poemll    I'll smash old man and young boy with you. \\
\poemlll       I'll smash young man and young woman\fnote{\fbackref{51:22} Or \fbib{virgin}} with you. \\
\poeml \v{23}I'll smash the shepherd and his flock with you. \\
\poemll    I'll smash the farmer and his team of oxen with you. \\
\poemlll       I'll smash governors and officials with you.
\end{poetry}

\v{24}``Before your eyes I'll repay Babylon and all the inhabitants of Chaldea for all the evil that they did in Zion,'' declares the \divine{Lord}.

\begin{poetry}
\poeml \v{25}``Look, I'm against you, destroying mountain, \\
\poemll    who destroys the whole earth,'' \\
\poemlll       declares the \divine{Lord}. \\
\poeml ``I'll stretch out my hand against you \\
\poemll    and roll you down from the crags. \\
\poemlll       And I'll make you a burned-out mountain. \\
\poeml \v{26}They won't get a cornerstone \\
\poemll    or a foundation stone from you, \\
\poeml because you will be a wasteland forever,'' \\
\poemll    declares the \divine{Lord}. \\
\poeml \v{27}Lift up a battle standard in the land. \\
\poemll    Blow a trumpet among the nations. \\
\poeml Consecrate the nations against her. \\
\poemll    Summon the kingdoms of Ararat, Minni, \\
\poemlll       and Ashkenaz against her. \\
\poeml Appoint a commander against her, \\
\poemll    bring up horses like bristling locusts. \\
\poeml \v{28}Consecrate the nations against her, \\
\poemll    the kings of the Medes, their governors, their prefects, \\
\poemlll       and every land under their domination. \\
\poeml \v{29}The land quakes and writhes \\
\poemll    because the \divine{Lord}'s purposes \\
\poeml against Babylon stand firm, \\
\poemll    to make the land of Babylon a waste without inhabitants. \\
\poeml \v{30}The warriors of Babylon have stopped fighting. \\
\poemll    They stay in their strongholds; \\
\poeml their strength is dried up; \\
\poemll    they have become like women. \\
\poeml Her buildings are set on fire; \\
\poemll    the bars of her gates are broken. \\
\poeml \v{31}One runner runs to meet another runner,\fnote{\fbackref{51:31} Lit. \fbib{to meet a runner}} \\
\poemll    and one messenger to meet another messenger,\fnote{\fbackref{51:31} Lit. \fbib{to meet a messenger}} \\
\poeml to tell the king of Babylon that his city has been seized \\
\poemll    from one end to the other.\fnote{\fbackref{51:31} Lit. \fbib{from the end}} \\
\poeml \v{32}The fords have been captured, \\
\poemll    and the marshes burned with fire. \\
\poemlll       The soldiers are terrified. \\
\poeml \v{33}For this is what the \divine{Lord} of the Heavenly Armies, \\
\poemll    the God of Israel, says: \\
\poeml ``The daughter of Babylon is like a threshing floor \\
\poemll    at the time when it's pounded down.\fnote{\fbackref{51:33} I.e. threshing floors were pounded and smoothed in preparation for an upcoming harvest} \\
\poeml In just a little while, the time of her harvest will come.''
\passage{Judah's Complaint against Babylon}
\poeml \v{34}``King Nebuchadnezzar of Babylon has devoured me \\
\poemll    and crushed me. \\
\poeml He set me down \\
\poemll    like an empty vessel. \\
\poeml He swallowed me like a monster, \\
\poemll    and filled his belly with my delicacies. \\
\poemlll       Then he washed me away. \\
\poeml \v{35}May the violence done to me \\
\poemll    and my flesh be on Babylon,'' \\
\poemlll       says the inhabitant of Zion. \\
\poeml ``May my blood be on the inhabitants of Chaldea,'' \\
\poemll    says Jerusalem. \\
\poeml \v{36}Therefore this is what the \divine{Lord} says: \\
\poeml ``Look, I'm going to argue your case \\
\poemll    and take vengeance for you. \\
\poeml I'll dry up her sea \\
\poemll    and make her fountain dry.\fnote{\fbackref{51:36} I.e. dry up the source of Babylon's waters} \\
\poeml \v{37}Babylon will become a heap of ruins, \\
\poemll    a refuge for jackals, \\
\poeml a desolate place \\
\poemll    and an object of scorn.\fnote{\fbackref{51:37} Lit. \fbib{hissing}; i.e. as a sign of mocking and contempt} \\
\poeml \v{38}They'll roar together like young lions; \\
\poemll    they'll growl like lion cubs. \\
\poeml \v{39}When they're excited\fnote{\fbackref{51:39} Lit. \fbib{hot}} I'll serve them their banquet, \\
\poemll    and make them drunk until they're merry. \\
\poeml They'll sleep forever and won't wake up,'' \\
\poemll    declares the \divine{Lord}. \\
\poeml \v{40}``I'll bring them down like lambs for the slaughter, \\
\poemll    like rams with male goats. \\
\poeml \v{41}``How Sheshak\fnote{\fbackref{51:41} \fbib{Sheshak} is a cryptogram for Babylon.} will be captured, \\
\poemll    and the prince of all the earth seized! \\
\poeml How Babylon will become an object of horror \\
\poemll    among the nations! \\
\poeml \v{42}The sea will come up against Babylon, \\
\poemll    and she will be covered by wave upon wave.\fnote{\fbackref{51:42} Lit. \fbib{its many waves}} \\
\poeml \v{43}Her cities will become an object of horror, \\
\poemll    a dry land and a desert, \\
\poeml a land in which no one lives, \\
\poemll    and through which no human being passes. \\
\poeml \v{44}I'll punish Bel\fnote{\fbackref{51:44} \fbib{Bel} was another name for \fbib{Marduk}, the sun god of Babylon.} in Babylon, \\
\poemll    and I'll make what he has swallowed \\
\poemlll       come out of his mouth. \\
\poeml The nations will no longer stream to him. \\
\poemll    Even the wall of Babylon will fall. \\
\poeml \v{45}``Come out of her, my people, \\
\poemll    flee for your lives from the \divine{Lord}'s anger! \\
\poeml \v{46}Do this\fnote{\fbackref{51:46} Lit. \fbib{And}} now, so your heart does not grow faint, \\
\poemll    and so you don't become frightened \\
\poemlll       because of the rumors\fnote{\fbackref{51:46} Lit. \fbib{rumor}} that are heard in the land--- \\
\poeml a rumor comes one year\fnote{\fbackref{51:46} Lit. \fbib{in a year}} and then after it \\
\poemll    another rumor\fnote{\fbackref{51:46} Lit. \fbib{a rumor}} comes the next year\fnote{\fbackref{51:46} Lit. \fbib{in a year}} \\
\poeml about violence in the land \\
\poemll    and one ruler against another ruler.\fnote{\fbackref{51:46} Lit. \fbib{a ruler against a ruler}} \\
\poeml \v{47}Therefore, look, days are coming \\
\poemll    when I'll punish the idols of Babylon. \\
\poeml Her entire land will be put to shame, \\
\poemll    and all her slain will fall in her midst. \\
\poeml \v{48}Then the heavens and the earth \\
\poemll    and all that are in them \\
\poemlll       will shout for joy about Babylon \\
\poeml because the destroyers will come \\
\poemll    out of the north against her,'' \\
\poemlll       declares the \divine{Lord}. \\
\poeml \v{49}``So Babylon will fall \\
\poemll    because of the slain of Israel, \\
\poeml even as the slain of all the earth \\
\poemll    have fallen because of Babylon. \\
\poeml \v{50}Go, you who escaped the sword! \\
\poemll    Don't stand around! \\
\poeml Remember the \divine{Lord} from far away, \\
\poemll    and let Jerusalem come to your mind. \\
\poeml \v{51}We have been put to shame \\
\poemll    because we have heard insults. \\
\poeml Disgrace has covered our faces because foreigners have \\
\poemll    come into the Holy Places of the \divine{Lord}'s house. \\
\poeml \v{52}``Therefore, look, days are coming,'' \\
\poemll    declares the \divine{Lord}, \\
\poeml ``when I'll punish her idols, \\
\poemll    and throughout her land the wounded will groan. \\
\poeml \v{53}Though Babylon should reach up to the heavens \\
\poemll    and fortify her high fortresses, \\
\poeml from me destroyers will come to her,'' \\
\poemll    declares the \divine{Lord}. \\
\poeml \v{54}``The sound of a cry is coming from Babylon, \\
\poemll    great destruction from the land of the Chaldeans. \\
\poeml \v{55}For the \divine{Lord} is destroying Babylon, \\
\poemll    and he will make the loud sounds from her disappear.\fnote{\fbackref{51:55} Lit. \fbib{perish}} \\
\poeml Their waves will roar like many waters, \\
\poemll    the noise of their voices will sound forth. \\
\poeml \v{56}Indeed, the destroyer is coming against her, \\
\poemll    against Babylon. \\
\poeml Her warriors are captured, \\
\poemll    and her bows are broken. \\
\poeml For the \divine{Lord} is a God of recompense, \\
\poemll    and he will repay in full. \\
\poeml \v{57}I'll make their leaders, their wise men, \\
\poemll    their governors, their deputies, \\
\poeml and their warriors drunk so that they sleep forever \\
\poemll    and don't wake up,'' \\
\poeml declares the King \\
\poemll    whose name is the \divine{Lord} of the Heavenly Armies. \\
\poeml \v{58}This is what the \divine{Lord} of the Heavenly Armies says: \\
\poeml ``The broad wall of Babylon will be completely leveled, \\
\poemll    and its high gate set on fire. \\
\poeml and so the peoples toil for nothing, \\
\poemll    and the nations weary themselves only for fire.''
\end{poetry}
\passage{Jeremiah's Symbolic Message against Babylon}

\v{59}This is\fnote{\fbackref{51:59} The Heb. lacks \fbib{This is}} the message that Jeremiah the prophet delivered\fnote{\fbackref{51:59} Lit. \fbib{commanded}} to Neriah's son Seraiah, the grandson of Mahseiah, when he went with King Zedekiah of Judah to Babylon in the fourth year of his reign. Seraiah was the quartermaster. \v{60}Jeremiah wrote on a single scroll all the disasters that would come on Babylon, all these things that were written about Babylon. \v{61}Jeremiah told Seraiah, ``When you come to Babylon, see that you read all these words, \v{62}and say, `\divine{Lord}, you have declared about this place that you would destroy it so that there wouldn't be an inhabitant in it, neither human nor animal, because it will be a wasteland forever.' \v{63}When you finish reading this scroll, tie a rock around it and throw it into the middle of the Euphrates. \v{64}Then say, `Babylon will sink like this and won't rise from the disaster that I'm bringing on her. Her people\fnote{\fbackref{51:64} Lit. \fbib{They}} will be exhausted.'\,''

\begin{poetry}
\poeml This concludes the writings of Jeremiah.
\end{poetry}
\labelchapt{52}
\passage{The Fall of Jerusalem}

\chapt{52}
\v{1}Zedekiah was 21 years old when he began to rule, and he ruled for 11 years in Jerusalem. His mother's name was Hamutal, the daughter of Jeremiah of Libnah. \v{2}Zedekiah\fnote{\fbackref{52:2} Lit. \fbib{He}} had done evil in the \divine{Lord}'s sight, just as Jehoiakim had done. \v{3}Because Jerusalem and Judah had angered the Lord, he cast them out of his presence. Zedekiah rebelled against the king of Babylon, \v{4}and in the ninth year of his reign, in the tenth month, on the tenth day, King Nebuchadnezzar of Babylon came against Jerusalem with all his army. He encamped near it and set up siege works all around it. \v{5}The city was under siege until the eleventh year of the reign of\fnote{\fbackref{52:5} The Heb. lacks \fbib{the reign of}} King Zedekiah. \v{6}By the ninth day of the fourth month the famine became so severe that there was no food for the people of the land. \v{7}The wall of\fnote{\fbackref{52:7} The Heb. lacks \fbib{The wall of}} the city was broken through, and all the soldiers fled, leaving the city at night through the gate between the two walls next to the king's garden, even though the Chaldeans were all around the city. They went in the direction of the Arabah.\fnote{\fbackref{52:7} I.e. the Jordan Valley}

\v{8}The Chaldean army went after the king, overtook Zedekiah in the plains of Jericho, and all his troops were scattered from him. \v{9}They captured the king and brought him to the king of Babylon at Riblah in the land of Hamath, where the king of Babylon\fnote{\fbackref{52:8} Lit. \fbib{he}} passed judgment on him. \v{10}The king of Babylon killed Zedekiah's sons before his eyes, and he also killed all the Judean officials\fnote{\fbackref{52:9} Or \fbib{princes}} at Riblah. \v{11}He blinded Zedekiah and bound him in bronze shackles. Then the king of Babylon took him to Babylon and put him in prison until he died.
\passage{The Destruction of the Temple}

\v{12}In the fifth month, on the tenth day of the month---it was the nineteenth year of the reign of\fnote{\fbackref{52:12} The Heb. lacks \fbib{the reign of}} King Nebuchadnezzar, king of Babylon---Nebuzaradan the captain of the guard who served\fnote{\fbackref{52:12} Lit. \fbib{who stood before}} the king of Babylon, entered Jerusalem. \v{13}He burned the \divine{Lord}'s Temple, the king's house, and all the houses in Jerusalem. He also burned every public building\fnote{\fbackref{52:13} Or \fbib{He burned every large house}} with fire. \v{14}All the Chaldean troops who were with the captain of the guard tore down all the walls around Jerusalem. \v{15}Nebuzaradan the captain of the guard carried into exile some of the poorest of the people, the rest of the people left in the city, the deserters who had defected to the king of Babylon, and the rest of the craftsmen. \v{16}But Nebuzaradan the captain of the guard left some of the poorest people of the land to be vinedressers and farmers.\fnote{\fbackref{52:16} Lit. \fbib{tillers}}

\v{17}The Chaldeans broke in pieces the bronze pillars that were in the \divine{Lord}'s Temple and the stands and the bronze sea that were in the \divine{Lord}'s Temple, and they carried all the\fnote{\fbackref{52:17} Lit. \fbib{their bronze}} bronze to Babylon. \v{18}They took away the pots, the shovels, the snuffers, the basins, the pans, and all the bronze utensils that were used in the temple service. \v{19}The captain of the guard took away the bowls, the fire pans, the basins, the pots, the lamp stands, the pans, and the bowls for libations, both those made of gold and those made of silver. \v{20}There was too much bronze to weigh in the two pillars, the one sea, the twelve bronze oxen that were under the sea,\fnote{\fbackref{52:20} The Heb. lacks \fbib{the sea}} and the stands which King Solomon had made for the \divine{Lord}'s Temple. \v{21}Each of the pillars was twelve cubits\fnote{\fbackref{52:21} I.e. about eighteen feet; a cubit was about eighteen inches} high and its circumference twelve cubits.\fnote{\fbackref{52:21} Lit. \fbib{a line of twelve cubits would surround it};i.e. about eighteen feet; a cubit was about eighteen inches} It was hollow and about a handbreadth\fnote{\fbackref{52:21} Lit. \fbib{four fingers}} thick. \v{22}On each pillar\fnote{\fbackref{52:22} Lit. \fbib{on it}} was a capital of bronze, and the height of each capital was five cubits.\fnote{\fbackref{52:22} I.e. about seven and a half feet; a cubit was about eighteen inches} Latticework and pomegranates, all of bronze, were all around the capital. And the second pillar was like this, including the pomegranates. \v{23}There were 96 pomegranates open to view.\fnote{\fbackref{52:23} Or \fbib{evenly spread}} In all, there were 100 pomegranates all around the latticework.
\passage{Executions and Deportations to Babylon}

\v{24}The captain of the guard arrested Seraiah the chief priest, Zephaniah the next ranking priest,\fnote{\fbackref{52:24} Lit. \fbib{the number two priest}} and the three guards of the gate.\fnote{\fbackref{52:24} Lit. \fbib{of the threshold}; i.e. high Temple officials} \v{25}From the city he arrested one of the officers who had been in charge of the troops, seven men from the king's personal advisors who were found in the city, the secretary of the commander of the army who mustered the people of the land, and 60 men of the people of the land who were found inside the city. \v{26}Nebuzaradan the captain of the guard arrested them and brought them to the king of Babylon at Riblah. \v{27}The king of Babylon struck them down and killed them at Riblah in the land of Hamath. So Judah went into exile from the land.

\v{28}These are the people Nebuchadnezzar took into exile: in the seventh year, 3,023 Judeans; \v{29}in Nebuchadnezzar's eighteenth year, 832 people from Jerusalem; \v{30}in Nebuchadnezzar's twenty-third year, Nebuzaradan the captain of the guard took 745 people from Judah into exile. All the people taken into exile\fnote{\fbackref{52:30} The Heb. lacks \fbib{taken into exile}} numbered 4,600.
\passage{Jehoiachin Released from Prison}

\v{31}In the first year of his reign, King Evil-merodach of Babylon, showed favor to King Jehoiachin of Judah by releasing him from prison on the twenty-fifth day of the twelfth month in the thirty-seventh year of the exile of King Jehoiachin of Judah. \v{32}He spoke kindly to him and gave him a seat above the seats of the other\fnote{\fbackref{52:32} The Heb. lacks \fbib{other}} kings who were in Babylon with him. \v{33}Jehoiachin\fnote{\fbackref{52:33} Lit. \fbib{He}} changed his prison clothes and regularly dined with the king\fnote{\fbackref{52:33} Lit. \fbib{ate food before him}} as long as he lived. \v{34}As for his living expenses, a regular allowance was given him daily by the king of Babylon as long as he lived,\fnote{\fbackref{52:34} Lit. \fbib{all the days of his life}} until the day of his death.
