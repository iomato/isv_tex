\bookheader{Job}
\labelbook{Job}

\bookpretitle{The Book of}
\booktitle{Job}

\labelchapt{1}
\passage{Job's Faithfulness}

\chapt{1}
\v{1}There once was a man in the land of Uz\fnote{\fbackref{1:1} I.e. a city east of Israel in Arabia; the name means \fbib{Wooded}} named Job. The man was blameless as well as upright. He feared God and kept away from evil. \v{2}Seven sons and three daughters had been born to him. \v{3}His livestock included 7,000 sheep, 3,000 camels, 500 teams\fnote{\fbackref{1:3} Or \fbib{pairs}} of oxen, 500 female donkeys, and many servants. Indeed, the man's stature greatly exceeded that of many people who lived in the East. \v{4}His sons used to travel to each other's houses in turn on a regular schedule and hold festivals, inviting their three sisters to celebrate\fnote{\fbackref{1:4} Lit. \fbib{to eat and drink}} with them.

\v{5}When their time of feasting had concluded, Job would rise early in the morning to send for them\fnote{\fbackref{1:5} The Heb. lacks \fbib{for them}} and consecrate them to God.\fnote{\fbackref{1:5} The Heb. lacks \fbib{to God}} He would offer a burnt offering for each one,\fnote{\fbackref{1:5} Lit. \fbib{offering according to their number}} because Job thought, ``Perhaps my children sinned by cursing God in their hearts.'' Job did this time and again.\fnote{\fbackref{1:5} Lit. \fbib{all the days}}
\passage{Satan's First Attack on Job}

\v{6}One day, divine beings\fnote{\fbackref{1:6} Lit. \fbib{day, sons of God}} presented themselves to the \divine{Lord}, and Satan\fnote{\fbackref{1:6} The Heb. name \fbib{Satan} means \fbib{The Opponent} or \fbib{The Accuser}; and so throughout the book} accompanied them. \v{7}The \divine{Lord} asked Satan, ``Where have you come from?''

In response, Satan answered the \divine{Lord}, ``From wandering all over the earth and walking back and forth throughout it.''

\v{8}Then the \divine{Lord} asked Satan, ``Have you considered\fnote{\fbackref{1:8} Lit. \fbib{you set your heart over}} my servant Job? There is no one like him on earth. The man is blameless as well as upright. He fears God and keeps away from evil.''

\v{9}But in response, Satan asked the \divine{Lord}, ``Does Job fear God for nothing? \v{10}Haven't you surrounded him with a fence on all sides, around his house, and around all that he owns? You have blessed everything he puts his hands on and you have increased his livestock in the land. \v{11}However, stretch out your hand and strike everything he owns, and he will curse you to your face.''

\v{12}Then the \divine{Lord} told Satan, ``Very well then, everything he owns is under your control,\fnote{\fbackref{1:12} Lit. \fbib{hand}} only you may not extend your hand against him.'' So Satan left the \divine{Lord}'s presence.

\v{13}Some time later, when his children\fnote{\fbackref{1:13} Lit. \fbib{his sons and daughters}} were celebrating\fnote{\fbackref{1:13} Lit. \fbib{were eating and drinking wine}} in their oldest\fnote{\fbackref{1:13} Lit. \fbib{their firstborn}} brother's house, \v{14}a messenger approached Job and said, ``The oxen were plowing and the female donkeys were grazing nearby \v{15}when the Sabeans attacked, captured the servants, and killed them with swords. I alone escaped to tell you!''

\v{16}While this messenger\fnote{\fbackref{1:16} The Heb. lacks \fbib{messenger}} was still speaking, another\fnote{\fbackref{1:16} Lit. \fbib{this}} came and announced, ``A lightning storm struck\fnote{\fbackref{1:16} Lit. \fbib{Fire of God fell from heaven}} and incinerated the flock and the servants while they were eating. I alone escaped to tell you!''

\v{17}While this messenger\fnote{\fbackref{1:17} The Heb. lacks \fbib{messenger}} was still speaking, another\fnote{\fbackref{1:17} Lit. \fbib{this}} came and announced, ``The Chaldeans formed three companies, raided the camels, captured the servants, and killed them with swords. Only I alone escaped to tell you.''

\v{18}While this messenger\fnote{\fbackref{1:18} The Heb. lacks \fbib{messenger}} was still speaking, another\fnote{\fbackref{1:18} Lit. \fbib{this}} came and announced, ``Your children were celebrating\fnote{\fbackref{1:18} Lit. \fbib{eating and drinking wine}} in their oldest\fnote{\fbackref{1:18} Lit. \fbib{their firstborn}} brother's house \v{19}when a strong wind came straight out of the wilderness and struck the four corners of the house. It collapsed on the young people, and they died. I alone escaped to tell you!''
\passage{Job Blesses God Despite the Catastrophe}

\v{20}Then Job stood up, tore his robe, shaved his head, fell to the ground, bowed very low, \v{21}and exclaimed:

\begin{poetry}
\poeml ``I left my mother's womb naked, \\
\poemll    and I will return to God naked. \\
\poeml The \divine{Lord} has given, \\
\poemll    and the \divine{Lord} has taken. \\
\poemlll       May the name of the \divine{Lord} be blessed.''
\end{poetry}

\v{22}Job neither sinned nor charged God with wrongdoing in all of this.
\labelchapt{2}
\passage{Satan's Second Attack on Job}

\chapt{2}
\v{1}Some time later, divine beings again\fnote{\fbackref{2:1} Lit. \fbib{later, the sons of God}} presented themselves to the \divine{Lord}, and Satan accompanied them to present himself to the \divine{Lord}. \v{2}The \divine{Lord} asked Satan, ``Where have you come from?''

In response, Satan told the \divine{Lord}, ``From wandering all over the earth and walking back and forth throughout it.''

\v{3}The \divine{Lord} asked Satan, ``Have you considered\fnote{\fbackref{2:3} Lit. \fbib{you set your heart over}} my servant Job? There is no one like him on earth. The man is blameless as well as upright. He fears God and keeps away from evil. He remains firm in his integrity, even though you have been urging me to overwhelm him without cause.''

\v{4}Satan answered the \divine{Lord}, ``Skin for skin! The man will give up everything that he owns in exchange for his health.\fnote{\fbackref{2:4} Lit. \fbib{his soul}} \v{5}However, stretch out your hand\fnote{\fbackref{2:5} Or \fbib{send judgment}} and strike his bones and flesh, and he'll curse you to your face, won't he?''

\v{6}Then the \divine{Lord} told Satan, ``Very well then, he is under your control.\fnote{\fbackref{2:6} Lit. \fbib{hand}} Just preserve his life.''\fnote{\fbackref{2:6} Lit. \fbib{his soul}}

\v{7}So Satan left the \divine{Lord}'s presence and struck Job with terrible boils from the sole of his feet to the top of his head. \v{8}Job\fnote{\fbackref{2:8} Lit. \fbib{He}} took a broken piece of pottery to scrape himself while sitting among the ashes.
\passage{Job Refuses to Curse God}

\v{9}Then his wife told him, ``Do you remain firm in your integrity? Curse God and die!''

\v{10}But he replied to her, ``You're talking like foolish women do. Are we to accept\fnote{\fbackref{2:10} Or \fbib{receive}} what is good from God but not tragedy?''

Throughout all of this, Job did not sin by what he said.\fnote{\fbackref{2:10} Lit. \fbib{by his lips}}
\passage{Job's Friends Visit}

\v{11}When Job's three friends heard all these tragedies that happened to him, they each traveled from their home towns\fnote{\fbackref{2:11} Lit. \fbib{from his place}} to visit him. Eliphaz came from Teman,\fnote{\fbackref{2:11} Lit. \fbib{Eliphaz the Temanite}; i.e. from Teman in Edom, and so throughout the book} Bildad came from Shuah,\fnote{\fbackref{2:11} Lit. \fbib{Bildad the Shuhite}; i.e. from Shuah, and so throughout the book} and Zophar came from Naamath.\fnote{\fbackref{2:11} Lit. \fbib{Zophar the Naamathite}; i.e. from Naamath in Arabia, and so throughout the book} They met together and went to console and comfort him. \v{12}Observing him from a distance, at first they didn't even recognize him, so they raised their voices and burst into tears. They each ripped their robes, threw ashes into the air on their heads, \v{13}and sat with Job\fnote{\fbackref{2:13} Lit. \fbib{him}} on the ground for a full week\fnote{\fbackref{2:13} Lit. \fbib{for seven days and seven nights}} without saying a word, since they could see the great extent of his anguish.
\labelchapt{3}
\passage{Job Laments the Day He was Born}

\chapt{3}
\v{1}After this, Job spoke up solemnly, cursing\fnote{\fbackref{3:1} Or \fbib{Job opened his mouth and cursed}} the day he was born.\fnote{\fbackref{3:1} The Heb. lacks \fbib{he was born}} \v{2}This is what Job said:

\begin{poetry}
\poeml \v{3}``Let the day when I was born be annihilated, \\
\poemll    along with the night when it was announced, \\
\poemlll       `It's a boy!'\fnote{\fbackref{3:3} Lit. \fbib{A man has been conceived.}} \\
\poeml \v{4}Let that day be dark; \\
\poemll    let God above not care about it; \\
\poemlll       let no light shine over it. \\
\poeml \v{5}Let darkness and deep gloom reclaim it; \\
\poemll    let clouds settle down on it; \\
\poemlll       let blackness in mid-day terrify it. \\
\poeml \v{6}Let darkness carry that night away; \\
\poemll    let it not take its place joyfully among the days of the year; \\
\poemlll       let it not be entered into the calendar.\fnote{\fbackref{3:6} Lit. \fbib{entered among the numbering of months}} \\
\poeml \v{7}``Yes, let that night be barren; \\
\poemll    let it not appear with its joyful shout. \\
\poeml \v{8}Let whoever curses days curse it--- \\
\poemll    those who are ready to awaken monsters.\fnote{\fbackref{3:8} Lit. \fbib{Leviathan}; i.e. an ancient sea creature} \\
\poeml \v{9}Let the stars of its evening twilight be dark; \\
\poemll    let it hope for light but let there be none; \\
\poemlll       let it not see the breaking rays\fnote{\fbackref{3:9} Lit. \fbib{the eyelashes}} of the dawn. \\
\poeml \v{10}``Because that night\fnote{\fbackref{3:10} Lit. \fbib{It}} refused to shut the doors of my mother's\fnote{\fbackref{3:10} The Heb. lacks \fbib{mother's}} womb; \\
\poemll    it failed to keep me from seeing this trouble. \\
\poeml \v{11}Why didn't I die while I was still in the womb, \\
\poemll    or die while I was being born? \\
\poeml \v{12}Why was there a lap\fnote{\fbackref{3:12} Lit. \fbib{Why were there knees}} to hold me, \\
\poemll    and why were there breasts to nurse me? \\
\poeml \v{13}``If I had died,\fnote{\fbackref{3:13} Lit. \fbib{For}} I would be lying down by now, \\
\poemll    undisturbed, asleep, and at rest, \\
\poeml \v{14}along with kings and counselors of the earth, \\
\poemll    who used to build for themselves what are now only\fnote{\fbackref{3:14} The Heb. lacks \fbib{only}} ruins, \\
\poeml \v{15}or princes who amassed\fnote{\fbackref{3:15} The Heb. lacks \fbib{who amassed}} gold for themselves, \\
\poemll    and who kept filling their houses with silver. \\
\poeml \v{16}``Or why was I not buried\fnote{\fbackref{3:16} Lit. \fbib{hidden}} like a stillborn child,\fnote{\fbackref{3:16} Or \fbib{miscarriage}} \\
\poemll    like babies\fnote{\fbackref{3:16} Lit. \fbib{children}} who never saw the light? \\
\poeml \v{17}In that place, the wicked stop causing trouble, \\
\poemll    and there, those whose strength is exhausted are at rest. \\
\poeml \v{18}In that place, those who once were prisoners will be at ease together; \\
\poemll    they won't hear the voice of oppressors. \\
\poeml \v{19}The unimportant and the important are both there, \\
\poemll    and the servant is free from his master. \\
\poeml \v{20}``Why does God\fnote{\fbackref{3:20} Lit. \fbib{he}} give light to the sufferer \\
\poemll    or life to the bitter person: \\
\poeml \v{21}To those who are longing for death--- \\
\poemll    even though it does not come? \\
\poeml To those who search for it \\
\poemll    more than for hidden treasure? \\
\poeml \v{22}To those who are happy beyond measure\fnote{\fbackref{3:22} Lit. \fbib{happy with great rejoicing}} \\
\poemll    when they reach their graves? \\
\poeml \v{23}To the formerly successful\fnote{\fbackref{3:23} Lit. \fbib{the valiant}} man who lost his way in life, \\
\poemll    and God fenced him in? \\
\poeml \v{24}``As far as I'm concerned, my food comes to me in the form of sighs, \\
\poemll    and my cries of anguish pour out like water. \\
\poeml \v{25}For the dreaded thing that I feared has happened to me, \\
\poemll    what caused me to worry has engulfed\fnote{\fbackref{3:25} Lit. \fbib{come}} me. \\
\poeml \v{26}I will not be at ease; \\
\poemll    I will not be quiet; \\
\poeml I will not rest; \\
\poemll    because trouble has arrived.''
\end{poetry}
\labelchapt{4}
\passage{Eliphaz: the Innocent Don't Suffer}

\chapt{4}
\v{1}In reply, Eliphaz from Teman answered:

\begin{poetry}
\poeml \v{2}``Will you get offended if somebody tries to talk to you? \\
\poemll    Who can keep from speaking at a time like this?\fnote{\fbackref{4:3} The Heb. lacks \fbib{at a time like this}} \\
\poeml \v{3}Look! You've admonished many people,\fnote{\fbackref{4:3} The Heb. lacks \fbib{people}} \\
\poemll    and you've strengthened feeble hands. \\
\poeml \v{4}A word from you has supported those who have stumbled, \\
\poemll    and has strengthened faltering knees. \\
\poeml \v{5}``But now it's your turn, \\
\poemll    and you're the one who is worn out!\fnote{\fbackref{4:5} Or \fbib{impatient}} \\
\poeml Now it's striking you, \\
\poemll    and you're dismayed! \\
\poeml \v{6}``Your fear of God has been your confidence, hasn't it? \\
\poemll    The integrity of your life has been your hope, hasn't it? \\
\poeml \v{7}Now please think: \\
\poemll    Who has ever perished when they're innocent? \\
\poemlll       Where have the upright been destroyed? \\
\poeml \v{8}It's been my experience that those who plow the soil of\fnote{\fbackref{4:8} The Heb. lacks \fbib{the soil of}} iniquity \\
\poemll    and those who sow the seed of\fnote{\fbackref{4:8} The Heb. lacks \fbib{the seed of}} trouble will reap their harvest!\fnote{\fbackref{4:8} The Heb. lacks \fbib{their harvest}} \\
\poeml \v{9}They perish by the breath of God; \\
\poemll    they are consumed by the storm that is\fnote{\fbackref{4:9} Or \fbib{the breath of}} his anger.\fnote{\fbackref{4:9} Or \fbib{anger}} \\
\poeml \v{10}``The lioness may roar, \\
\poemll    and the lion cub may growl; \\
\poemlll       but even the ivory teeth of the full grown lion are broken. \\
\poeml \v{11}Full grown lions die when they cannot find prey; \\
\poemll    that's when the lion cubs are scattered. \\
\poeml \v{12}``A message was confided\fnote{\fbackref{4:12} Or \fbib{was delivered in secret}} to me; \\
\poemll    my ear caught a whisper of it. \\
\poeml \v{13}Disquieting thoughts from dreams at night; \\
\poemll    when deep sleep falls on everyone.\fnote{\fbackref{4:13} Lit. \fbib{men}} \\
\poeml \v{14}A fear fell upon me, along with trembling \\
\poemll    that caused all my bones to shake in terror.\fnote{\fbackref{4:14} Or \fbib{dread}} \\
\poeml \v{15}A spirit glided past me \\
\poemll    and made the hair on my skin\fnote{\fbackref{4:15} Lit. \fbib{flesh}} to bristle. \\
\poeml \v{16}It remained standing, \\
\poemll    but I couldn't recognize its appearance. \\
\poeml A form appeared before my eyes; \\
\poemll    At first there was\fnote{\fbackref{4:16} The Heb. lacks \fbib{At first there was}} silence, and then this voice: \\
\poeml \v{17}`Can a mortal person\fnote{\fbackref{4:17} Lit. \fbib{a man}} be more righteous than God? \\
\poemll    Or can the purity of the valiant exceed that of his maker?'\fnote{\fbackref{4:17} The quotation possibly continues through v 21.} \\
\poeml \v{18}``Indeed, since he doesn't trust his servants,\fnote{\fbackref{4:18} Cf. Job 15:15} \\
\poemll    since he charges\fnote{\fbackref{4:18} Or \fbib{lay upon}} his angels with error, \\
\poeml \v{19}how much less confidence\fnote{\fbackref{4:19} The Heb. lacks \fbib{confidence}} does he have \\
\poemll    in those who dwell in houses of clay; \\
\poeml who were formed from a foundation in dust \\
\poemll    and can perish\fnote{\fbackref{4:19} Lit. \fbib{can crush them}} like a moth? \\
\poeml \v{20}They are defeated between morning and evening; \\
\poemll    they perish forever---and no one notices! \\
\poeml \v{21}Their wealth\fnote{\fbackref{4:21} Or \fbib{remnants}, \fbib{left over}} perishes with them, doesn't it? \\
\poemll    They die, and do so without having wisdom, don't they?''
\end{poetry}
\labelchapt{5}
\passage{Eliphaz: God Blesses those who Seek Him}

\chapt{5}
\v{1}``Cry out, won't you!

\begin{poetry}
\poemll    Is there anyone who will answer you? \\
\poemlll       To whom will you turn among the holy ones? \\
\poeml \v{2}For wrath will slay a fool; \\
\poemll    zealous anger will kill the na\"{i}ve. \\
\poeml \v{3}``I myself saw a fool becoming established, \\
\poemll    but I suddenly cursed where he lived.\fnote{\fbackref{5:3} Or \fbib{dwelling place}} \\
\poeml \v{4}His children are far from deliverance; \\
\poemll    they'll be maltreated before they leave home,\fnote{\fbackref{5:4} Lit. \fbib{be crushed in the gate}} \\
\poemlll       with no one to rescue them. \\
\poeml \v{5}Then the hungry will devour his harvest, \\
\poemll    snatching it even from the midst of thorns, \\
\poemlll       while the thirsty covet their wealth. \\
\poeml \v{6}For wickedness doesn't crop up from dust, \\
\poemll    nor does trouble sprout out of the ground; \\
\poeml \v{7}But mankind is born headed for trouble, \\
\poemll    just as sparks soar skyward.''
\passage{God Can be Trusted in Adversity}
\poeml \v{8}``Now as for me, I would seek God if I were you;\fnote{\fbackref{5:8} The Heb. lacks \fbib{if I were you}} \\
\poemll    I would commit my case to God. \\
\poeml \v{9}He is always doing great things that cannot be explained, \\
\poemll    countless awesome deeds. \\
\poeml \v{10}He sends rain on the surface of the earth, \\
\poemll    and waters the surface of the open country. \\
\poeml \v{11}He sets the lowly on high, \\
\poemll    and lifts those who mourn to safety.\fnote{\fbackref{5:11} Or \fbib{deliverance}} \\
\poeml \v{12}He frustrates the plans of the crafty; \\
\poemll    so that what they work for never succeeds. \\
\poeml \v{13}He captures the wise in their own craftiness, \\
\poemll    bringing a quick end to their cunning advice. \\
\poeml \v{14}They meet with darkness in broad daylight; \\
\poemll    at noonday they grope around as if it were night. \\
\poeml \v{15}So he delivers from the sword of their mouth--- \\
\poemll    the poor from the power\fnote{\fbackref{5:15} Lit. \fbib{hand}} of the mighty. \\
\poeml \v{16}Therefore there is hope for those who are poor, \\
\poemll    and iniquity shuts its mouth. \\
\poeml \v{17}``Indeed, how blessed is the person whom God reproves! \\
\poemll    So never disrespect the discipline of the Almighty, \\
\poeml \v{18}because though he wounds, but then applies bandages; \\
\poemll    though he strikes, his hands still heal. \\
\poeml \v{19}``He will deliver you through six calamities; \\
\poemll    and calamity won't touch you throughout the seventh. \\
\poeml \v{20}He will deliver you from death by famine; \\
\poemll    in war from the power\fnote{\fbackref{5:20} Lit. \fbib{mouth}} of the sword. \\
\poeml \v{21}You'll be protected from the accusing\fnote{\fbackref{5:21} Lit. \fbib{lash of the}} tongue; \\
\poemll    you need not fear destruction when it heads your way. \\
\poeml \v{22}You'll laugh at destruction and famine; \\
\poemll    and you need not fear the beasts of the earth. \\
\poeml \v{23}For you'll have a pact\fnote{\fbackref{5:23} Or \fbib{be in league}} with the stones in the field; \\
\poemll    and the beasts of the field will be at peace with you. \\
\poeml \v{24}You'll know that your home\fnote{\fbackref{5:24} Lit. \fbib{tent}} is secure; \\
\poemll    when you search your possessions, and nothing will be missing. \\
\poeml \v{25}You'll know that you'll have many children; \\
\poemll    and that your offspring will be like the grass of the earth. \\
\poeml \v{26}You'll go to your grave at a ripe old age; \\
\poemll    like a stack of grain that's harvested at just the right time. \\
\poeml \v{27}``Look! We have thought all this through, \\
\poemll    and what we've said is true;\fnote{\fbackref{5:27} Lit. \fbib{and thus it is so}} \\
\poemlll       So please listen and learn for your own good!''
\end{poetry}
\labelchapt{6}
\passage{Job's Suffering is Grave}

\chapt{6}
\v{1}In rebuttal, Job replied:

\begin{poetry}
\poeml \v{2}``If only my grief could be weighed; \\
\poemll    or my calamity piled together on a balance scale! \\
\poeml \v{3}It would weigh more than the sand on the seashore!\fnote{\fbackref{6:3} Lit. \fbib{sea}} \\
\poemll    Here's why I've talked so rashly: \\
\poeml \v{4}``The arrows of the Almighty have pierced me; \\
\poemll    my spirit absorbs\fnote{\fbackref{6:4} Lit. \fbib{drinks}} their poison;\fnote{\fbackref{6:4} Lit. \fbib{heat}} \\
\poemlll       God's terrors have been arranged just for me! \\
\poeml \v{5}``Will the wild donkey bray from hunger\fnote{\fbackref{6:5} The Heb. lacks \fbib{from hunger}} if fresh grass is beside him? \\
\poemll    Will the ox low from distress\fnote{\fbackref{6:5} The Heb. lacks \fbib{from distress}} if it is near its feed? \\
\poeml \v{6}Tasteless food isn't eaten without salt, is it? \\
\poemll    Is there any taste in an egg white? \\
\poeml \v{7}I cannot bring myself to touch them;\fnote{\fbackref{6:7} The Heb. lacks \fbib{them}} \\
\poemll    food like this makes me sick.''
\passage{Job Desires Death}
\poeml \v{8}``Who will grant my wish?\fnote{\fbackref{6:8} Or \fbib{Oh, that I might have my request;}} \\
\poemll    I wish God would grant what I'm hoping for: \\
\poeml \v{9}that God would just be willing\fnote{\fbackref{6:9} Lit. \fbib{pleased}} to crush me; \\
\poemll    that he would let loose\fnote{\fbackref{6:9} Lit. \fbib{loose his hand}} and eliminate me! \\
\poeml \v{10}At least I could still take comfort \\
\poemll    and rejoice in unceasing anguish, \\
\poemlll       for I didn't conceal what the Holy One has to say. \\
\poeml \v{11}``Do I have the strength to wait? \\
\poemll    And why\fnote{\fbackref{6:11} Lit. \fbib{And to what end}} should I be patient? \\
\poeml \v{12}Am I as strong as a rock? \\
\poemll    Am I some kind of iron man?\fnote{\fbackref{6:12} Lit. \fbib{Is my flesh bronze?}} \\
\poeml \v{13}There is no help within me, is there? \\
\poemll    My resources have been driven away from me, haven't they?
\passage{Job Accuses His Friends of Treachery}
\poeml \v{14}The friend shows gracious love for his friend, \\
\poemll    even if he has forsaken the fear of the Almighty. \\
\poeml \v{15}But my brothers have acted treacherously like a cascading river, \\
\poemll    like torrential rivers that overflow. \\
\poeml \v{16}Filled with waters made cold\fnote{\fbackref{6:16} Lit. \fbib{dark}} by ice, \\
\poemll    they are where the snow goes to hide. \\
\poeml \v{17}But then the snow melts, and they disappear; \\
\poemll    when warmed, they evaporate from their stream beds.\fnote{\fbackref{6:17} Lit. \fbib{their place}} \\
\poeml \v{18}Travelers divert\fnote{\fbackref{6:18} Lit. \fbib{twist}} in their route; \\
\poemll    they go into a wasteland and die. \\
\poeml \v{19}Travelers from Tema search intently; \\
\poemll    caravans from Sheba hope to find them. \\
\poeml \v{20}For all their expectations, they are doomed to disappointment; \\
\poemll    even though they have come and searched this far. \\
\poeml \v{21}``And now you're all just like them, aren't you?\fnote{\fbackref{6:21} Lit. \fbib{it}} \\
\poemll    You see my terror and are terrified. \\
\poeml \v{22}When did I ever ask you for anything, \\
\poemll    say `Offer a bribe for me from your wealth?' \\
\poeml \v{23}or say `Deliver me from my enemy's control,'\fnote{\fbackref{6:23} Lit. \fbib{hand}} \\
\poemll    or `Redeem me from the domination\fnote{\fbackref{6:23} Lit. \fbib{hand}} of ruthless people'?''
\passage{Job Requests Mercy from His Friends}
\poeml \v{24}``Instruct me, and I'll remain silent. \\
\poemll    Help me understand where I've gone astray. \\
\poeml \v{25}The truth\fnote{\fbackref{6:25} Lit. \fbib{Upright words}} can be painful, \\
\poemll    but what has your argument proven? \\
\poeml \v{26}Did you intend your words to reprove, \\
\poemll    even though the speech of a desperate person is just wind? \\
\poeml \v{27}Indeed, you would gamble to buy an orphan; \\
\poemll    and barter to buy your friend! \\
\poeml \v{28}Now be willing to face me, \\
\poemll    and I won't lie to your face. \\
\poeml \v{29}Repent! Let there be no injustice; \\
\poemll    Change your ways!\fnote{\fbackref{6:29} The Heb. lacks \fbib{your ways}} My vindication\fnote{\fbackref{6:29} Or \fbib{righteousness}} is at stake. \\
\poeml \v{30}Have I said anything that's unjust? \\
\poemll    I can discern\fnote{\fbackref{6:30} Lit. \fbib{taste}} evil, can't I?''
\end{poetry}
\labelchapt{7}
\passage{Job Acknowledges the Brevity of Life}

\chapt{7}
\v{1}``Men have harsh servitude on earth, do they not?

\begin{poetry}
\poemll    His days are like those of a hired laborer, are they not?\fnote{\fbackref{7:1} Or \fbib{hireling}} \\
\poeml \v{2}I'm like a servant who longs for the shade, \\
\poemll    like a hired laborer who is looking for his wages. \\
\poeml \v{3}Truly I've been allotted months of emptiness; \\
\poemll    nights of trouble have been appointed for me. \\
\poeml \v{4}``When I lie down I ask, \\
\poemll    `When will I wake up?' \\
\poeml But the night continues \\
\poemll    and I keep tossing and turning until dawn.\fnote{\fbackref{7:4} Or \fbib{twilight}} \\
\poeml \v{5}My skin\fnote{\fbackref{7:5} Or \fbib{flesh}} is covered with worms and clods of dirt; \\
\poemll    my skin becomes rough and then breaks out afresh. \\
\poeml \v{6}My days pass as swiftly as a hand-loom; \\
\poemll    they come to their conclusion without hope. \\
\poeml \v{7}Remember that my life is a breath; \\
\poemll    my eyes won't go back to seeing good things.\fnote{\fbackref{7:7} The Heb. lacks \fbib{things}} \\
\poeml \v{8}The eyes of the one who sees me won't see me anymore; \\
\poemll    your eyes will look\fnote{\fbackref{7:8} The Heb. lacks \fbib{will look}} for me \\
\poemlll       but I won't be around!\fnote{\fbackref{7:8} The Heb. lacks \fbib{around}} \\
\poeml \v{9}As a cloud fades away and vanishes, \\
\poemll    the one who descends to the afterlife\fnote{\fbackref{7:9} Lit. \fbib{Sheol}; i.e. the dwelling place of the dead} doesn't return.\fnote{\fbackref{7:9} Lit. \fbib{doesn't come back up}} \\
\poeml \v{10}He doesn't return again to his house, \\
\poemll    and his place won't recognize him anymore.''
\passage{Job Intends to Complain}
\poeml \v{11}``In addition, I won't keep my opinion\fnote{\fbackref{7:11} Lit. \fbib{mouth}} to myself; \\
\poemll    I'll speak from my distressed spirit; \\
\poemlll       I'll complain with my bitter soul. \\
\poeml \v{12}Am I the sea, or a sea monster, \\
\poemll    that you keep watching me? \\
\poeml \v{13}For I've said, `My bed will comfort me; \\
\poemll    my couch will ease my burdens\fnote{\fbackref{7:13} Or \fbib{carry}} while I complain.' \\
\poeml \v{14}But then you scared me with dreams; \\
\poemll    you terrified me with visions. \\
\poeml \v{15}I would rather die by strangulation \\
\poemll    than continue living.\fnote{\fbackref{7:15} Lit. \fbib{my bones}} \\
\poeml \v{16}I hate the thought of living forever! \\
\poemll    Leave me alone, because my days are pointless.''
\passage{Job Acknowledges Humankind's Insignificance}
\poeml \v{17}``What is a human being, that you make so much of him; \\
\poemll    that you set your affections on him, \\
\poeml \v{18}visit him every morning, \\
\poemll    and test him continually? \\
\poeml \v{19}Why won't you look away from me? \\
\poemll    Why don't you leave me alone so I can swallow my saliva? \\
\poeml \v{20}So what if I sin? What have I done against you, \\
\poemll    you observer of humankind? \\
\poeml Why have you made me your target? \\
\poemll    Why burden yourself with me? \\
\poeml \v{21}Why haven't you pardoned my transgression \\
\poemll    and taken away my iniquity? \\
\poeml Now I'm about to lie down in the dust. \\
\poemll    You will seek me diligently, \\
\poemlll       but I won't be around!''\fnote{\fbackref{7:21} The Heb. lacks \fbib{around}}
\end{poetry}
\labelchapt{8}
\passage{Bildad: God Rewards the Godly}

\chapt{8}
\v{1}Then in response, Bildad from Shuah said:

\begin{poetry}
\poeml \v{2}``How long will you keep talking like this? \\
\poemll    How long will you keep talking like a wind storm? \\
\poeml \v{3}Does God pervert justice? \\
\poemll    Does the Almighty pervert what's right? \\
\poeml \v{4}``If your children sin against him, \\
\poemll    he'll make them a prisoner\fnote{\fbackref{8:4} Lit. \fbib{he'll place them into the hand}} of their sins. \\
\poeml \v{5}If you seek God, \\
\poemll    if you ask the Almighty for mercy, \\
\poeml \v{6}if you are clean and upright, \\
\poemll    surely then, he'll act on your behalf \\
\poemlll       and restore your rightful\fnote{\fbackref{8:6} Lit. \fbib{and bring peace to your righteous}} place. \\
\poeml \v{7}Your beginning may be small, \\
\poemll    but later years\fnote{\fbackref{8:7} Lit. \fbib{days}} will be very great. \\
\poeml \v{8}``Inquire of the previous generation, won't you please? \\
\poemll    Consider what their forefathers searched out. \\
\poeml \v{9}Because we are of yesterday and we know nothing, \\
\poemll    for our time on earth is only a shadow. \\
\poeml \v{10}Won't they instruct you, and tell you, \\
\poemll    and bring out words from the heart? \\
\poeml \v{11}``Can papyrus grow where there's no marsh? \\
\poemll    Can reeds flourish without water? \\
\poeml \v{12}While they are still green \\
\poemll    and not yet ready to be harvested, \\
\poemlll       they wither before any plant. \\
\poeml \v{13}Such are the paths of everyone who forgets God--- \\
\poemll    the hope of the godless will be destroyed: \\
\poeml \v{14}his confidence is shattered; \\
\poemll    his trust is in a spider's web. \\
\poeml \v{15}He leans against his house, \\
\poemll    but it won't stand; \\
\poeml he grabs hold of it firmly, \\
\poemll    but it doesn't last. \\
\poeml \v{16}He is a fresh sapling planted in the sunlight, \\
\poemll    spreading out its branches over its garden. \\
\poeml \v{17}Its roots weave around a pile of stones, \\
\poemll    seeking to entrench itself among the rocks. \\
\poeml \v{18}If he is uprooted\fnote{\fbackref{8:18} Lit. \fbib{is swallowed up}} from his place,
\end{poetry}

then it will deny him:

\begin{poetry}
\poemlll       `I never saw you.' \\
\poeml \v{19}``Indeed, this is the benefit of God's\fnote{\fbackref{8:19} Lit. \fbib{his}} way: \\
\poemll    from the soil other plants\fnote{\fbackref{8:19} The Heb. lacks \fbib{plants}} will sprout. \\
\poeml \v{20}Surely God won't reject those who are blameless \\
\poemll    or hold hands with those who practice evil. \\
\poeml \v{21}He will soon fill your mouth with laughter, \\
\poemll    and your lips will shout with joy. \\
\poeml \v{22}Those who hate you will be clothed with shame, \\
\poemll    and the tent of the wicked will no longer exist.
\end{poetry}
\labelchapt{9}
\passage{Job Cannot Argue with His Creator}

\chapt{9}
\v{1}This was Job's response:

\begin{poetry}
\poeml \v{2}``Indeed, I'm fully aware that this is so, \\
\poemll    but how can a person become right with God? \\
\poeml \v{3}If one were to seek to argue with him, \\
\poemll    he won't be able to answer him even once in a thousand times. \\
\poeml \v{4}He is wise in heart and strong in will--- \\
\poemll    who can be stubborn against him and succeed? \\
\poeml \v{5}``He removes mountains without their knowledge, \\
\poemll    overthrowing them in his anger. \\
\poeml \v{6}He shakes the earth from its orbit,\fnote{\fbackref{9:6} Lit. \fbib{place}} \\
\poemll    so that its foundations shudder. \\
\poeml \v{7}He commands the sun so that it doesn't shine\fnote{\fbackref{9:7} Lit. \fbib{rise}} \\
\poemll    and seals up the stars. \\
\poeml \v{8}He alone spreads out the heavens, \\
\poemll    he walks on the waves\fnote{\fbackref{9:8} Lit. \fbib{high places}} of the sea. \\
\poeml \v{9}He created Bear, Orion, the Pleiades, \\
\poemll    and the southern constellations. \\
\poeml \v{10}He does great things that cannot be explained, \\
\poemll    and awesome deeds that cannot be counted. \\
\poeml \v{11}``If he were to pass near me, I wouldn't notice; \\
\poemll    if he moves by, I wouldn't perceive him. \\
\poeml \v{12}Indeed, if he snatches someone\fnote{\fbackref{9:12} The Heb. lacks \fbib{someone}} away, \\
\poemll    who could restrain him? \\
\poemlll       Who can say to him, `What are you doing?' \\
\poeml \v{13}``God doesn't restrain his anger. \\
\poemll    Rahab's assistants are humiliated under him. \\
\poeml \v{14}So how am I to answer him, \\
\poemll    choosing what I am to say to him? \\
\poeml \v{15}Even if I'm in the right, \\
\poemll    I cannot answer him. \\
\poemlll       I can only appeal for mercy. \\
\poeml \v{16}``Were I to be summoned, and he were to answer me, \\
\poemll    I wouldn't even believe \\
\poemlll       that he was listening to what I have to say.\fnote{\fbackref{9:16} Lit. \fbib{to my voice}} \\
\poeml \v{17}For he crushes me with a storm, \\
\poemll    and keeps on wounding me for no reason. \\
\poeml \v{18}He won't let me catch my breath; \\
\poemll    instead, he fills me with bitterness. \\
\poeml \v{19}``Is this a contest of strength? He is obviously stronger! \\
\poemll    Is this a matter of justice? Who can sue him? \\
\poeml \v{20}Though I'm in the right, my own mouth will condemn me; \\
\poemll    though I'm blameless, he'll pronounce me as guilty. \\
\poeml \v{21}``I'm blameless; \\
\poemll    I don't know myself; \\
\poemlll       I despise my life. \\
\poeml \v{22}I say it's all the same--- \\
\poemll    he destroys both the blameless and the guilty.\fnote{\fbackref{9:22} Or \fbib{wicked}} \\
\poeml \v{23}If a calamity\fnote{\fbackref{9:23} Or \fbib{scourge}} causes sudden death, \\
\poemll    he'll mock at the despair of the innocent. \\
\poeml \v{24}A land is given into the hands of a wicked person;\fnote{\fbackref{9:24} Lit. \fbib{man}} \\
\poemll    he covers the faces of its judges. \\
\poemlll       If it is not God,\fnote{\fbackref{9:24} Lit. \fbib{he}} then who is it?''
\passage{Job Argues that God Won't Acquit Him}
\poeml \v{25}``My days pass faster than a runner; \\
\poemll    but they pass quickly without seeing anything good. \\
\poeml \v{26}They pass by like a ship made of reeds, \\
\poemll    like an eagle swooping down on its prey. \\
\poeml \v{27}If I were to say, `Let me forget my complaint,' \\
\poemll    change\fnote{\fbackref{9:27} Lit. \fbib{forsake}} the expression on\fnote{\fbackref{9:27} The Heb. lacks \fbib{the expression on}} my face, and look cheerful, \\
\poeml \v{28}then I still dread all of my suffering; \\
\poemll    I know you still won't acquit me. \\
\poeml \v{29}I will be condemned, \\
\poemll    so why should I wear myself out with this futility? \\
\poeml \v{30}``If I wash myself with water from snow, \\
\poemll    and cleanse my hands with lye, \\
\poeml \v{31}you'll still drop me into the Pit,\fnote{\fbackref{9:31} I.e. the place of punishment in the afterlife} \\
\poemll    and my own clothes will despise me. \\
\poeml \v{32}He's not a man like me, \\
\poemll    so that I can answer him, \\
\poemlll       or that we can enter into litigation\fnote{\fbackref{9:32} Lit. \fbib{controversy}} with one another. \\
\poeml \v{33}There is not yet a mediator between us, \\
\poemll    who would set his hand on the two of us, \\
\poeml \v{34}removing his rod from me, \\
\poemll    and not letting terror of him overwhelm me. \\
\poeml \v{35}Otherwise, I would speak without being terrified of him, \\
\poemll    because I'm not like that inside myself.''
\end{poetry}
\labelchapt{10}
\passage{Job Asks God to Acquit Him}

\begin{poetry}
\poeml \chapt{10}
\v{1}``I am disgusted with living, \\
\poemll    so I'm going to talk about my complaint freely. \\
\poemlll       I'll speak out from the bitterness of my soul. \\
\poeml \v{2}I'll say to God, \\
\poemll    `Don't condemn me! \\
\poemlll       Let me know why you are fighting me. \\
\poeml \v{3}Does it delight you to oppress \\
\poemll    or despise what you have made, \\
\poemlll       while you smile at the plans of the wicked?\fnote{\fbackref{10:3} Lit. \fbib{you cause the plans of the wicked to shine}} \\
\poeml \v{4}Do you have eyes made of flesh? \\
\poemll    Can you look at things as humans do? \\
\poeml \v{5}Can you live only as long as a human being? \\
\poemll    Or live the years\fnote{\fbackref{10:5} Lit. \fbib{days}} of a mortal man? \\
\poeml \v{6}```For you seek out my iniquity \\
\poemll    and search for my sin. \\
\poeml \v{7}Although you know that I'm not guilty, \\
\poemll    there's no one to deliver me from you!\fnote{\fbackref{10:7} Lit. \fbib{from your hand}} \\
\poeml \v{8}Your hands formed and fashioned me, \\
\poemll    but then you have destroyed me all at once on all sides. \\
\poeml \v{9}```Please remember that you've made me like clay \\
\poemll    and you'll return me to dust. \\
\poeml \v{10}Didn't you pour me out like milk \\
\poemll    and let me congeal like cheese? \\
\poeml \v{11}You covered me with skin and flesh, \\
\poemll    weaving me together with bones and sinews. \\
\poeml \v{12}You gave life and gracious love to me; \\
\poemll    your providential care has preserved my spirit. \\
\poeml \v{13}But you've hidden these things in your heart--- \\
\poemll    I know this was your purpose:\fnote{\fbackref{10:13} Or \fbib{was in you}} \\
\poeml \v{14}If I sin, you watch me \\
\poemll    and won't acquit me for my iniquity. \\
\poeml \v{15}```Woe to me if I'm guilty! \\
\poemll    If I'm innocent, I cannot lift my head, \\
\poeml because I am filled with disgrace. \\
\poemll    Look at my affliction! \\
\poeml \v{16}But if I do lift up my head, \\
\poemll    you will hunt me like a lion! \\
\poemlll       You will perform miracles in order to fight against me. \\
\poeml \v{17}```You have brought new witnesses against me, \\
\poemll    you're even more angry with me--- \\
\poemlll       you've brought fresh troops to attack me! \\
\poeml \v{18}So why did you bring me out from the womb? \\
\poemll    I wish I had died, before anyone had seen me, \\
\poeml \v{19}as if I had never existed; \\
\poemll    carried from the womb to the grave. \\
\poeml \v{20}My days are so few, aren't they? \\
\poemll    So leave me alone, then, \\
\poemlll       so I can smile a little \\
\poeml \v{21}before I go, never to return, \\
\poemll    leaving for the land of deep darkness and shadow. \\
\poeml \v{22}It's a gloomy land, like deepest darkness; \\
\poemll    where there's no order, \\
\poemlll       and where even\fnote{\fbackref{10:22} The Heb. lacks \fbib{even}} the brightness is like darkness.'\,''
\end{poetry}
\labelchapt{11}
\passage{Zophar Accuses Job}

\chapt{11}
\v{1}Zophar from Naamath had this to say:

\begin{poetry}
\poeml \v{2}``Shouldn't a multitude of words be answered, \\
\poemll    or a person who talks too much\fnote{\fbackref{11:2} Or \fbib{a talker}} be vindicated? \\
\poeml \v{3}Will your irrational babble silence people, \\
\poemll    and when you mock them, \\
\poemlll       will you escape without being shamed?\fnote{\fbackref{11:3} MT has \fbib{without being humiliated}} \\
\poeml \v{4}You've said, `My teaching is flawless; \\
\poemll    I'm clean\fnote{\fbackref{11:4} Or \fbib{pure}} in God's\fnote{\fbackref{11:4} Lit. \fbib{your}} sight.' \\
\poeml \v{5}``But what if God were to speak? \\
\poemll    What if he were\fnote{\fbackref{11:5} The Heb. lacks \fbib{What if he were}} to talk\fnote{\fbackref{11:5} Lit. \fbib{open his lips against}} with you, \\
\poeml \v{6}and disclose his wise secrets? \\
\poeml After all, there's so much more\fnote{\fbackref{11:6} Lit. \fbib{double}} to understanding. \\
\poemll    So be aware that God will exact from you \\
\poemlll       less than your sin deserves.''
\passage{God's Wisdom is Unfathomable}
\poeml \v{7}``Can you search through God's complex things? \\
\poemll    Can you uncover the limits of the Almighty? \\
\poeml \v{8}These things are higher than the heavens, \\
\poemll    so what can you do? \\
\poeml They are deeper than Sheol,\fnote{\fbackref{11:8} I.e. the place where the dead are in the afterlife} \\
\poemll    so what can you know? \\
\poeml \v{9}They are longer than the earth's circumference,\fnote{\fbackref{11:9} Lit. \fbib{measure}} \\
\poemll    and broader than the ocean. \\
\poeml \v{10}``If he bypasses, or imprisons, or convenes a court,\fnote{\fbackref{11:10} The Heb. lacks \fbib{a court}} \\
\poemll    who can stop\fnote{\fbackref{11:10} Or \fbib{repel}} him? \\
\poeml \v{11}For he knows mankind's\fnote{\fbackref{11:11} Lit. \fbib{men}} deceitfulness; \\
\poemll    when he sees iniquity, won't he himself consider it? \\
\poeml \v{12}An empty-headed person will gain understanding \\
\poemll    when a wild donkey is born a human being!''
\passage{Zophar Counsels Job to Repent}
\poeml \v{13}``Now for you, if you will prepare your heart, \\
\poemll    spread out your hands to him. \\
\poeml \v{14}If you have any iniquity, throw it far away. \\
\poemll    Don't let evil\fnote{\fbackref{11:14} Or \fbib{wrong}} live in your residence.\fnote{\fbackref{11:14} Lit. \fbib{tents}} \\
\poeml \v{15}Then your confidence will be flawless, \\
\poemll    and your security will keep you from terror. \\
\poeml \v{16}You'll forget your suffering; \\
\poemll    you'll remember it like water that has evaporated.\fnote{\fbackref{11:16} Or \fbib{has flowed past}} \\
\poeml \v{17}Your life will be brighter than noonday. \\
\poemll    Even its darkness will be like dawn. \\
\poeml \v{18}You'll be secure, because there is hope; \\
\poemll    you'll see that you're at rest and safe. \\
\poeml \v{19}When you sleep, there'll be nothing to fear; \\
\poemll    and many will court your favor.\fnote{\fbackref{11:19} Lit. \fbib{face}} \\
\poeml \v{20}But what the wicked look for will fail; \\
\poemll    their way of escape will be taken away\fnote{\fbackref{11:20} Lit. \fbib{destroyed}} from them; \\
\poemlll       their only hope is to take their final breath.''\fnote{\fbackref{11:20} Lit. \fbib{is to breathe out their soul}}
\end{poetry}
\labelchapt{12}
\passage{Job Responds to Zophar}

\chapt{12}
\v{1}In response Job replied:

\begin{poetry}
\poeml \v{2}``Truly, you are the people \\
\poemll    and wisdom will die with you! \\
\poeml \v{3}Like you, I also have understanding.\fnote{\fbackref{12:3} Lit. \fbib{my heart is like yours}} \\
\poemll    I'm not inferior to you; \\
\poemlll       who doesn't know things\fnote{\fbackref{12:3} The Heb. lacks \fbib{things}} like this?''
\passage{Job Has Become a Laughingstock}
\poeml \v{4}``I'm a laughingstock to my friends, \\
\poemll    someone\fnote{\fbackref{12:4} The Heb. lacks \fbib{someone}} who called on God. \\
\poeml But then he answered this upright and blameless man, \\
\poemll    and I have become\fnote{\fbackref{12:4} The Heb. lacks \fbib{have become}} a laughingstock. \\
\poeml \v{5}The carefree are thinking, `I have contempt for misfortune,'
\end{poetry}

Those who are about to stumble deserve it.

\begin{poetry}
\poeml \v{6}The tents of robbers are at rest, \\
\poemll    and those who provoke God are secure, \\
\poemlll       that is, those who carry their god in their pocket.\fnote{\fbackref{12:6} Lit. \fbib{hand}}
\passage{Wisdom Can Be Found in God's Creation}
\poeml \v{7}``Ask the wild animals, and they'll teach you; \\
\poemll    the birds of the sky will tell you. \\
\poeml \v{8}Or ask the green plants of the earth and they'll teach you; \\
\poemll    let the fish in the sea tell you. \\
\poeml \v{9}Who among all of these doesn't know \\
\poemll    that the \divine{Lord}'s hand made them,\fnote{\fbackref{12:9} Lit. \fbib{this}} \\
\poeml \v{10}and that the life of every living thing\fnote{\fbackref{12:10} Lit. \fbib{all the living}} rests in his control, \\
\poemll    along with the breath of every living human being? \\
\poeml \v{11}The ear scrutinizes speech \\
\poemll    just as the palate tastes food.''
\passage{God is the All-Wise and All-Powerful Creator}
\poeml \v{12}``Wisdom may be found in the company of the aged. \\
\poemll    Understanding comes\fnote{\fbackref{12:12} The Heb. lacks \fbib{comes}} with longevity. \\
\poeml \v{13}With God\fnote{\fbackref{12:13} Lit. \fbib{him}} is wisdom and strength; \\
\poemll    counsel and understanding belongs to him. \\
\poeml \v{14}When he tears down, nobody rebuilds; \\
\poemll    when\fnote{\fbackref{12:14} Lit. \fbib{man}} he incarcerates, nobody escapes. \\
\poeml \v{15}When he withholds water, rivers\fnote{\fbackref{12:15} Lit. \fbib{they}} dry up; \\
\poemll    when he lets them loose, they'll flood\fnote{\fbackref{12:15} Lit. \fbib{overthrow}} the land. \\
\poeml \v{16}``With God\fnote{\fbackref{12:16} Lit. \fbib{him}} are strength and sound wisdom; \\
\poemll    both the deceived and those who deceive are responsible to him.\fnote{\fbackref{12:16} Or \fbib{are his}} \\
\poeml \v{17}He leads counselors away naked; \\
\poemll    he turns judges into fools. \\
\poeml \v{18}He strips away the authority of kings to punish \\
\poemll    and puts them in prison clothes instead. \\
\poeml \v{19}He leads away the priests naked \\
\poemll    and overthrows the ruling class.\fnote{\fbackref{12:19} Lit. \fbib{strong ones}} \\
\poeml \v{20}He keeps reliable advisors from speaking,\fnote{\fbackref{12:20} Lit. \fbib{deprives the lips of advisors}} \\
\poemll    and removes discernment from elders. \\
\poeml \v{21}He pours contempt on nobles \\
\poemll    and embarrasses\fnote{\fbackref{12:21} Lit. \fbib{and loosens the belt of}} the mighty. \\
\poeml \v{22}He uncovers the hidden dimensions from darkness, \\
\poemll    bringing what is in deep shadow to light. \\
\poeml \v{23}He makes nations great, and then destroys them; \\
\poemll    he enlarges nations, but then sends them away to captivity.\fnote{\fbackref{12:23} The Heb. lacks \fbib{to captivity}} \\
\poeml \v{24}He withdraws understanding\fnote{\fbackref{12:24} Lit. \fbib{heart}} from national leaders of the world,\fnote{\fbackref{12:24} Or \fbib{land}} \\
\poemll    causing them to wander through uncharted\fnote{\fbackref{12:24} Or \fbib{trackless}} wilderness. \\
\poeml \v{25}They grope in the dark without light; \\
\poemll    he causes them to stagger around like a drunkard.''
\end{poetry}
\labelchapt{13}
\passage{Job Begins to Argues His Case}

\begin{poetry}
\poeml \chapt{13}
\v{1}``Look, I've seen everything; \\
\poemll    I've listened carefully and I understand. \\
\poeml \v{2}What you know, I know, too; \\
\poemll    I'm not inferior to you. \\
\poeml \v{3}But I want to talk to the Almighty; \\
\poemll    and I'm determined to argue my case\fnote{\fbackref{13:3} The Heb. lacks \fbib{my case}} before God.''
\passage{Job Accuses His Friends}
\poeml \v{4}``But as for you, you whitewash with lies; \\
\poemll    all of you are worthless quacks.\fnote{\fbackref{13:4} Lit. \fbib{physicians}} \\
\poeml \v{5}I wish you'd all just shut up. \\
\poemll    Then at least you would appear to be wise. \\
\poeml \v{6}``Now listen to my dispute! \\
\poemll    Pay attention to my arguments.\fnote{\fbackref{13:6} Lit. \fbib{arguments of my lips}} \\
\poeml \v{7}Why do you speak falsely on God's behalf \\
\poemll    and speak deceitfully\fnote{\fbackref{13:7} Or \fbib{treachery}} about him? \\
\poeml \v{8}Will you show partiality to him?\fnote{\fbackref{13:8} Lit. \fbib{lift up his face}} \\
\poemll    Will you plead God's case? \\
\poeml \v{9}Will things go well for you under his cross-examination? \\
\poemll    Can you lie to him, as you would to a human being?\fnote{\fbackref{13:9} Lit. \fbib{mankind}} \\
\poeml \v{10}``He will be sure to rebuke you, \\
\poemll    if you show partiality\fnote{\fbackref{13:10} Lit. \fbib{you lift up the face}} in secret. \\
\poeml \v{11}His splendor will petrify you with terror, \\
\poemll    paralyzing you with fear, won't it? \\
\poeml \v{12}Your maxims are just worthless proverbs; \\
\poemll    your defensive arguments are made of clay.''
\passage{Job Resolves to Present His Case}
\poeml \v{13}``Don't talk to me! It's my turn to speak; \\
\poemll    then whatever happens, happens. \\
\poeml \v{14}Why shouldn't I bite my flesh \\
\poemll    or take my life in my hands? \\
\poeml \v{15}Even though he kills me, \\
\poemll    I'll continue to hope in him. \\
\poemlll       At least I'll be able to argue my case\fnote{\fbackref{13:15} Or \fbib{way}} to his face! \\
\poeml \v{16}I have this as my salvation: \\
\poemll    the godless person won't be appearing before him. \\
\poeml \v{17}Pay attention\fnote{\fbackref{13:17} Lit. \fbib{listen}, \fbib{to listen}} to what I have to say \\
\poemll    and listen to my testimony with your own ears.''
\passage{Job Presents His Conditions}
\poeml \v{18}``Look, now! I've prepared my case for court.\fnote{\fbackref{13:18} Or \fbib{judgment}} \\
\poemll    I know that I'm going to win.\fnote{\fbackref{13:18} Lit. \fbib{I'm in the right}} \\
\poeml \v{19}Who can oppose me? \\
\poemll    If they do, I'll be silent and die. \\
\poeml \v{20}Grant me only two things as you deal with me; \\
\poemll    then I won't hide from you.\fnote{\fbackref{13:20} Lit. \fbib{from your face}} \\
\poeml \v{21}Withdraw your hand far from me \\
\poemll    and keep me from being petrified with terror. \\
\poeml \v{22}Then call and I'll answer, \\
\poemll    or let me speak and then you reply to me!''
\passage{Job Presents Himself for Cross-Examination}
\poeml \v{23}``How many of my iniquities and sins have you counted? \\
\poemll    Show me my transgression and sin. \\
\poeml \v{24}Why do you hide your face \\
\poemll    and regard me as your enemy? \\
\poeml \v{25}Are you a god who would make a leaf tremble \\
\poemll    or who would prosecute a dry straw? \\
\poeml \v{26}You've accused me of bitter things; \\
\poemll    you've caused me to reap\fnote{\fbackref{13:26} Lit. \fbib{inherit}} the sins of my youth. \\
\poeml \v{27}You've locked my feet in stocks; \\
\poemll    you watch all my steps; \\
\poemlll       You've limited where I can walk.\fnote{\fbackref{13:27} Lit. \fbib{limited the soles of my feet}} \\
\poeml \v{28}So I'm a man who wears out like something rotten, \\
\poemll    like a garment that has become moth-eaten.''
\end{poetry}
\labelchapt{14}
\passage{Human Beings Live and Die}

\chapt{14}
\v{1}Human beings born by women

\begin{poetry}
\poemll    are short-lived\fnote{\fbackref{14:1} Lit. \fbib{is of short days}} and full of trouble. \\
\poeml \v{2}He springs up\fnote{\fbackref{14:2} Lit. \fbib{goes out}} like a flower and then withers.\fnote{\fbackref{14:2} Lit. \fbib{is cut off}} \\
\poemll    Like a shadow, he disappears\fnote{\fbackref{14:2} Lit. \fbib{flees}} and doesn't last. \\
\poeml \v{3}Indeed, have you opened your eyes on one like this--- \\
\poemll    to bring me into a legal fight with you? \\
\poeml \v{4}Who can produce a clean thing from an unclean thing? \\
\poemll    No one! \\
\poeml \v{5}Since his days have been determined, \\
\poemll    the number of his months is known to you. \\
\poeml You've set his limit \\
\poemll    and he cannot exceed it. \\
\poeml \v{6}Look away from him and leave him alone, \\
\poemll    so he can enjoy his time, like a hired worker.''
\passage{Death is Certain}
\poeml \v{7}``There is hope for the tree, if it is cut down, \\
\poemll    that it will sprout again, \\
\poemlll       and that its shoots won't stop growing. \\
\poeml \v{8}Even if its roots have grown ancient in the earth, \\
\poemll    and its stump begins to rot\fnote{\fbackref{14:8} Lit. \fbib{die}} in the ground, \\
\poeml \v{9}the presence\fnote{\fbackref{14:9} Lit. \fbib{scent}} of water will make it to bud \\
\poemll    so that it sprouts new branches like a young plant. \\
\poeml \v{10}``But when a person\fnote{\fbackref{14:10} Lit. \fbib{man}} dies and wastes away, \\
\poemll    when a person\fnote{\fbackref{14:10} Lit. \fbib{the valiant man}} breathes his last, where will he be? \\
\poeml \v{11}As water disappears from the sea, \\
\poemll    or water evaporates from a river, \\
\poeml \v{12}so also a person\fnote{\fbackref{14:12} Lit. \fbib{man}} lies down and does not get up; \\
\poemll    they won't awaken until the heavens are no more, \\
\poemlll       nor will they arise from their sleep.''
\passage{There is Life after Death}
\poeml \v{13}``Won't you keep me safe in the afterlife?\fnote{\fbackref{14:13} Lit. \fbib{in Sheol}; i.e. the realm of the dead} \\
\poemll    Conceal me until your anger subsides. \\
\poeml Set an appointment for me, \\
\poemll    then remember me. \\
\poeml \v{14}If a human being\fnote{\fbackref{14:14} Lit. \fbib{strong man}} dies, will he live again? \\
\poemll    I will endure the entire time of my assigned service, \\
\poemlll       until I am changed.\fnote{\fbackref{14:14} Lit. \fbib{until my change comes}; i.e. change in bodily state at the resurrection; cf. 1 Cor 15:51} \\
\poeml \v{15}You'll call and I'll answer you; \\
\poemll    you'll long for your creatures that your hands have made.\fnote{\fbackref{14:15} Lit. \fbib{for the work of your hands}} \\
\poeml \v{16}Then you'll certainly count every step I took, \\
\poemll    but you won't keep an inventory of my sin. \\
\poeml \v{17}My transgressions would be sealed up in a bag; \\
\poemll    you would cover over my sins. \\
\poeml \v{18}``Mountains fall and crumble; \\
\poemll    rocks are dislodged from their places. \\
\poeml \v{19}Water wears away stones; \\
\poemll    floods wash away topsoil from the land--- \\
\poemlll       but you destroy the hope of human beings just like that! \\
\poeml \v{20}You overpower him once and for all, and then he departs; \\
\poemll    you change his appearance and then send him away. \\
\poeml \v{21}``If his children are honored, he doesn't know it; \\
\poemll    if they become insignificant, he never perceives it. \\
\poeml \v{22}He feels only his own pain,\fnote{\fbackref{14:22} Lit. \fbib{flesh}} \\
\poemll    and grieves only for himself.''
\end{poetry}
\labelchapt{15}
\passage{Eliphaz Speaks Again}

\chapt{15}
\v{1}Then Eliphaz from Teman responded:

\begin{poetry}
\poeml \v{2}``Should a wise person respond with knowledge based on wind? \\
\poemll    Should he fill his stomach with a wind storm from the east? \\
\poeml \v{3}Should he engage in unprofitable argument, \\
\poemll    or give a speech that benefits no one? \\
\poeml \v{4}Yet you dispense with fear of God \\
\poemll    and hinder meditations before God. \\
\poeml \v{5}Because your sin dictates your speech,\fnote{\fbackref{15:5} Lit. \fbib{mouth}} \\
\poemll    you have chosen the language\fnote{\fbackref{15:5} Lit. \fbib{tongue}} of the crafty. \\
\poeml \v{6}Your own mouth is condemning you, not I; \\
\poemll    your own lips will testify against you.''
\passage{Eliphaz Claims that Job is Guilty}
\poeml \v{7}``Were you the first person\fnote{\fbackref{15:7} Lit. \fbib{man}}to be born? \\
\poemll    Were you brought forth before the hills were made? \\
\poeml \v{8}Have you listened in on God's secret council? \\
\poemll    Have you limited wisdom only to yourself? \\
\poeml \v{9}What do you know that we don't know, \\
\poemll    or that you understand and that isn't clear to us? \\
\poeml \v{10}``We have both the gray-haired and the aged with us, \\
\poemll    and they are far older\fnote{\fbackref{15:10} Lit. \fbib{are older by many days}} than your father. \\
\poeml \v{11}Are God's encouragements inconsequential to you, \\
\poemll    even a word that has been spoken\fnote{\fbackref{15:11} The Heb. lacks \fbib{spoken}} gently to you? \\
\poeml \v{12}Why have your emotions\fnote{\fbackref{15:12} Lit. \fbib{heart}} carried you away? \\
\poemll    And why do your eyes flash \\
\poeml \v{13}that you turn your anger\fnote{\fbackref{15:13} Lit. \fbib{spirit}} against God \\
\poemll    and speak words like this? \\
\poeml \v{14}``What is mankind, that he can be blameless? \\
\poemll    Or does being born of a woman mean he'll be in the right? \\
\poeml \v{15}Look, if God\fnote{\fbackref{15:15} Lit. \fbib{he}} doesn't trust his holy ones,\fnote{\fbackref{15:15} Cf. Job 4:18} \\
\poemll    if even the heavens aren't pure as he looks at them, \\
\poeml \v{16}then how much less is one who is abhorred and corrupted, \\
\poemll    such as a man who drinks injustice like water?''
\passage{Eliphaz Describes the Plight of the Wicked}
\poeml \v{17}``I'll tell you what, listen to me! \\
\poemll    Let me relate what I've seen, \\
\poeml \v{18}which is what wise men have explained, \\
\poemll    who didn't withhold anything from their ancestors. \\
\poeml \v{19}To them alone was the land given, \\
\poemll    when no invader\fnote{\fbackref{15:19} Or \fbib{foreigner}} passed through their midst. \\
\poeml \v{20}``The wicked person writhes in pain throughout his life, \\
\poemll    a number of years has been reserved for the ruthless. \\
\poeml \v{21}Terrifying sounds ring\fnote{\fbackref{15:21} The Heb. lacks \fbib{ring}} in his ears; \\
\poemll    when times are prosperous, the Destroyer will attack\fnote{\fbackref{15:21} Or \fbib{come upon him}} him. \\
\poeml \v{22}He does not believe he will escape\fnote{\fbackref{15:22} Or \fbib{turn aside}} darkness; \\
\poemll    he is destined for the sword. \\
\poeml \v{23}He wanders around for food---where is it? \\
\poemll    He knows that a time of darkness is near.\fnote{\fbackref{15:23} Lit. \fbib{is at hand}} \\
\poeml \v{24}Distress and pressure terrify him; \\
\poemll    they overwhelm him, like a king poised for attack. \\
\poeml \v{25}``For he has raised his fist against God, \\
\poemll    defying the Almighty. \\
\poeml \v{26}He defiantly ran against him \\
\poemll    carrying his thick, reinforced shield. \\
\poeml \v{27}Though he covered his face with fat, \\
\poemll    and is grossly overweight at the waist, \\
\poeml \v{28}He will live in devastated towns, \\
\poemll    in abandoned houses \\
\poemlll       that are about to become heaps of rubble. \\
\poeml \v{29}``He won't become rich and his wealth won't last; \\
\poemll    he won't expand his holdings to cover the land. \\
\poeml \v{30}He won't escape darkness; \\
\poemll    a flame will wither his shoots; \\
\poemlll       and he'll depart by the breath of God's\fnote{\fbackref{15:30} Lit. \fbib{his}} mouth. \\
\poeml \v{31}Let him not trust in a worthless speech. \\
\poemll    He leads only himself astray, \\
\poemlll       for emptiness will be his reward. \\
\poeml \v{32}This will be accomplished before his time;\fnote{\fbackref{15:32} Lit. \fbib{day}} \\
\poemll    his branches won't grow luxuriant. \\
\poeml \v{33}``He is like a vine that drops its unripe grapes; \\
\poemll    like an olive tree that loses its blossoms. \\
\poeml \v{34}For the company of the godless is desolation, \\
\poemll    and fire consumes the tents of those who\fnote{\fbackref{15:34} The Heb. lacks \fbib{those who}} bribe. \\
\poeml \v{35}For they conceive mischief and give birth to iniquity; \\
\poemll    their womb is pregnant\fnote{\fbackref{15:35} Lit. \fbib{womb fashions}; i.e., as a womb fashions a child} with deception.''
\end{poetry}
\labelchapt{16}
\passage{Job Reasons with Eliphaz}

\chapt{16}
\v{1}In response, Job said:

\begin{poetry}
\poeml \v{2}``I've heard many things like this. \\
\poemll    What miserable comforters you all are! \\
\poeml \v{3}Will windy words like yours never end? \\
\poemll    What is upsetting you that you keep on arguing? \\
\poeml \v{4}``I could also talk like you \\
\poemll    if only you were in my place! \\
\poeml Then I would put together an argument\fnote{\fbackref{16:4} Lit. \fbib{together words}} against you. \\
\poemll    I would shake my head at you \\
\poeml \v{5}and encourage you with what I have to say;\fnote{\fbackref{16:5} Lit. \fbib{with my mouth}} \\
\poemll    my words of comfort would lessen your pain. \\
\poeml \v{6}``But if I speak, my pain isn't assuaged; \\
\poemll    if I refrain from speaking, what do I have to lose?''
\passage{Job Claims of God's Mistreatment}
\poeml \v{7}``God\fnote{\fbackref{16:7} Lit. \fbib{He}} has certainly worn me out; \\
\poemll    you devastated my entire world.\fnote{\fbackref{16:7} Lit. \fbib{community}} \\
\poeml \v{8}You've arrested me, making me testify against myself! \\
\poemll    My leanness rises up to attack me, accusing\fnote{\fbackref{16:8} Lit. \fbib{testifying}} me to my face. \\
\poeml \v{9}His anger tears me in his persistent resentment against me; \\
\poemll    he gnashes his teeth at me. \\
\poemlll       My adversary glares\fnote{\fbackref{16:9} Lit. \fbib{sharpens his eyes}} at me. \\
\poeml \v{10}People gaped at me with mouths wide open; \\
\poemll    they slap me in their scorn \\
\poemlll       and gather together against me. \\
\poeml \v{11}God has delivered me over to the ungodly, \\
\poemll    throwing me into the control of the wicked. \\
\poeml \v{12}``He tore me apart when I was at ease; \\
\poemll    grabbing me by my neck, he shook me to pieces--- \\
\poemlll       then he really made me his target. \\
\poeml \v{13}His archers surround me, \\
\poemll    slashing open my kidneys without pity; \\
\poemlll       he pours out my gall on the ground. \\
\poeml \v{14}Attack follows attack as he breaks through my defenses! \\
\poemll    He runs over me like a mighty warrior. \\
\poeml \v{15}``I've even sewn sackcloth directly to my skin; \\
\poemll    I've buried my strength\fnote{\fbackref{16:15} Lit. \fbib{horn}} in the dust. \\
\poeml \v{16}My face is red from my tears, \\
\poemll    and dark shadows encircle my eyelids, \\
\poeml \v{17}even though violence is not my intention, \\
\poemll    and my prayer is pure.''
\passage{Job Appeals to Witnesses}
\poeml \v{18}``Listen, earth! Don't cover my blood, \\
\poemll    for my outcry has no place to rest. \\
\poeml \v{19}Even now, behold! I have a witness in heaven, \\
\poemll    my Advocate is on high. \\
\poeml \v{20}My friends mock me, \\
\poemll    while my eyes overflow with tears to God, \\
\poeml \v{21}crying for him to arbitrate between this\fnote{\fbackref{16:21} The Heb. lacks \fbib{this}} man and God; \\
\poemll    as a human being does with his fellow neighbor. \\
\poeml \v{22}For when only a few years have elapsed, \\
\poemll    I'll start down a path from which I'll never return.''
\end{poetry}
\labelchapt{17}
\passage{Job Laments and Prepares for Death}

\begin{poetry}
\poeml \chapt{17}
\v{1}``My spirit is crushed, \\
\poemll    my days are over;\fnote{\fbackref{17:1} Lit. \fbib{extinguished}} \\
\poemlll       it's the grave for me! \\
\poeml \v{2}Mockers surround me; \\
\poemll    I cannot stop staring at their hostility all through the night. \\
\poeml \v{3}Offer, then, some collateral on my behalf. \\
\poemll    Is there anyone who will be my guarantor? \\
\poeml \v{4}``Because you're the one who closed their hearts to compassion;\fnote{\fbackref{17:4} Lit. \fbib{understanding}} \\
\poemll    therefore, you won't let them triumph. \\
\poeml \v{5}Now as for the one who testifies against his friends \\
\poemll    to take their property,\fnote{\fbackref{17:5} Or \fbib{inheritance}} \\
\poemlll       even the eyes of his children will fail. \\
\poeml \v{6}``He has made me a byword among people; \\
\poemll    I'm being spit on in the face. \\
\poeml \v{7}My eyes have grown weak from grief; \\
\poemll    and my whole body is as thin as a shadow. \\
\poeml \v{8}The upright are appalled over this, \\
\poemll    and the innocent person is troubled by the godless. \\
\poeml \v{9}But the righteous person will hold to his way, \\
\poemll    and those with clean hands will grow stronger and stronger.''
\passage{Job Prepares for Death}
\poeml \v{10}``Come here now, all of you, \\
\poemll    and I won't find a wise person among you. \\
\poeml \v{11}My days are passed; \\
\poemll    my plans have been shattered; \\
\poemlll       along with my heart's desires. \\
\poeml \v{12}They have transformed night into day--- \\
\poemll    `The light,' they say, `is about to become dark.' \\
\poeml \v{13}``If my hope were that my house is the afterlife\fnote{\fbackref{17:13} Lit. \fbib{Sheol}; i.e. the realm of the afterlife} itself, \\
\poemll    if I were to make my bed in darkness, \\
\poeml \v{14}if I call out to the Pit,\fnote{\fbackref{17:14} I.e. the realm of punishment in the afterlife} `You're my father!' \\
\poemll    or say to the worm,\fnote{\fbackref{17:14} I.e. an agent of punishment in the afterlife} `My mother!' or `My sister!' \\
\poeml \v{15}where would my hope be? \\
\poeml ``And speaking of my hope, who would notice it? \\
\poeml \v{16}Will it go down to the bars that lock the doors\fnote{\fbackref{17:16} The Heb. lacks \fbib{that lock the doors}} of the afterlife?\fnote{\fbackref{17:16} Lit. \fbib{Sheol}; i.e. the realm of the dead} \\
\poemlll       Will we descend together into the dust?''
\end{poetry}
\labelchapt{18}
\passage{Bildad Speaks Again}

\chapt{18}
\v{1}Bildad from Shuah replied, saying:

\begin{poetry}
\poeml \v{2}``When are you going to stop your word hunt? \\
\poemll    Think first, and then we can talk. \\
\poeml \v{3}Why do you think we're like dumb animals? \\
\poemll    Do you think we're stupid? \\
\poeml \v{4}You're tearing yourself to pieces in your anger. \\
\poemll    Will the land be abandoned because of you, \\
\poemlll       or the rock be moved from its place?''
\passage{The Wicked are Trapped}
\poeml \v{5}``Indeed, the light of the wicked is extinguished; \\
\poemll    the flame from his fire doesn't shine. \\
\poeml \v{6}Light in his tent is dark, \\
\poemll    and his lamp goes out above him. \\
\poeml \v{7}His strong steps are restricted, \\
\poemll    and his own advice trips him up. \\
\poeml \v{8}``For he has stumbled into a net with his own feet; \\
\poemll    he walked right into the network! \\
\poeml \v{9}The trap seizes him by the heel; \\
\poemll    a snare tightens its hold on him. \\
\poeml \v{10}A rope lies hidden in the dirt; \\
\poemll    a trap lies\fnote{\fbackref{18:10} The Heb. lacks \fbib{lies}} waiting for him where he is walking.''
\passage{The Wicked Perish without Descendants}
\poeml \v{11}``He is petrified by terror that surrounds him on all sides; \\
\poemll    they chase at his heels. \\
\poeml \v{12}He is starved for strength; \\
\poemll    and is ripe for a fall. \\
\poeml \v{13}Something gnaws on his skin; \\
\poemll    a deadly disease\fnote{\fbackref{18:13} Lit. \fbib{a firstborn of death}} consumes his limbs. \\
\poeml \v{14}Torn from the security of his home,\fnote{\fbackref{18:14} Lit. \fbib{tent}} \\
\poemll    he is brought before the king of terrors. \\
\poeml \v{15}``There's nothing in his tent that belongs to him; \\
\poemll    sulfur is scattered all over his dwelling place. \\
\poeml \v{16}His roots wither underneath, \\
\poemll    while his branches above are being cut off. \\
\poeml \v{17}No one remembers him anywhere in the land; \\
\poemll    no one names streets in his honor. \\
\poeml \v{18}He is driven away from light to darkness, \\
\poemll    made to wander the landscape. \\
\poeml \v{19}He has no children or descendants within his own people; \\
\poemll    and no survivors where he once lived. \\
\poeml \v{20}People\fnote{\fbackref{18:20} The Heb. lacks \fbib{people}} who live west of him are appalled at his fate;\fnote{\fbackref{18:20} Lit. \fbib{at his day}} \\
\poemll    those who live east of him are seized with terror. \\
\poeml \v{21}Indeed, the residences of the wicked are like this; \\
\poemll    and so are the homes of those who don't know God.''
\end{poetry}
\labelchapt{19}
\passage{Job Responds to Bildad}

\chapt{19}
\v{1}In response, Job said:

\begin{poetry}
\poeml \v{2}``How long do you intend to keep torturing me \\
\poemll    and trying to break me by what you're saying? \\
\poeml \v{3}Ten times you've tried to humiliate me! \\
\poemll    You're not ashamed to wrong me! \\
\poeml \v{4}Even if it's true that I've erred, \\
\poemll    my error only affects me. \\
\poeml \v{5}If you really intend to vaunt yourselves over me, \\
\poemll    and make my problems the basis of your case against me, \\
\poeml \v{6}then at least you must know that God has accused me of wrong, \\
\poemll    and trapped me with his net.''
\passage{Job Accuses God of Being Angry}
\poeml \v{7}``Although I cried out `Violence!' \\
\poemll    I received no answer; \\
\poeml I cried for help, \\
\poemll    but there was no justice. \\
\poeml \v{8}He blocked my path, \\
\poemll    so I cannot pass; \\
\poemlll       and he turned out the lights on my pathways. \\
\poeml \v{9}``He has stripped me of my honor; \\
\poemll    he has stolen the crown off my head! \\
\poeml \v{10}He is breaking me down on every side, \\
\poemll    and now it's too late for me;\fnote{\fbackref{19:10} Lit. \fbib{and I'm gone}} \\
\poemlll       he has uprooted my hopes like a tree. \\
\poeml \v{11}His anger burns against me; \\
\poemll    he regards me as his adversary. \\
\poeml \v{12}His troops march\fnote{\fbackref{19:12} Or \fbib{proceed}} in a column\fnote{\fbackref{19:12} Or \fbib{together}} against me, \\
\poemll    erecting their siege ramps against me; \\
\poemlll       they surround my tent.''
\passage{Job's Family and Friends Abandoned Him}
\poeml \v{13}``My brothers are alienated from me; \\
\poemll    my acquaintances are estranged; \\
\poeml \v{14}my relatives have failed me; \\
\poemll    and my friends\fnote{\fbackref{19:14} Lit. \fbib{and those who know me}} have abandoned me. \\
\poeml \v{15}Those who live in my house--- \\
\poemll    and my maidservants, too!--- \\
\poeml treat me like a stranger; \\
\poemll    they think I'm a foreigner. \\
\poeml \v{16}``I call to my servant, \\
\poemll    but he doesn't respond, \\
\poemlll       even though I beg to him earnestly.\fnote{\fbackref{19:16} Lit. \fbib{him with my mouth}} \\
\poeml \v{17}My wife says my breath stinks; \\
\poemll    even my children say I smell bad! \\
\poeml \v{18}Even little children hate me; \\
\poemll    when I get up, they mock me. \\
\poeml \v{19}My closest friends\fnote{\fbackref{19:19} Or \fbib{circle of familiar friends}} detest me; \\
\poemll    even the ones I love have turned against me. \\
\poeml \v{20}I'm a pile of skin and bones; \\
\poemll    I have barely escaped by the skin of my teeth.''
\passage{Job Pleads with His Friends}
\poeml \v{21}``Be gracious to me, be gracious to me, my friends, \\
\poemll    because God's hand has struck me. \\
\poeml \v{22}Why are you chasing me, as God has been doing? \\
\poemll    Aren't you satisfied that I'm sick?\fnote{\fbackref{19:22} Lit. \fbib{satisfied with my flesh}} \\
\poeml \v{23}If only my words were written down; \\
\poemll    if only they were inscribed in a book \\
\poeml \v{24}using an iron stylus with lead for ink! \\
\poemll    Then they'd be engraved in rock forever. \\
\poeml \v{25}``As for me, I know that my Vindicator\fnote{\fbackref{19:25} Or \fbib{Redeemer}} is alive; \\
\poemll    And he, the Last One,\fnote{\fbackref{19:25} Lit. \fbib{And the Last}} will take his stand on the soil.\fnote{\fbackref{19:25} Or \fbib{dust}} \\
\poeml \v{26}Even after my skin has been destroyed, \\
\poemll    clothed in my flesh I will see God, \\
\poeml \v{27}whom I will see for myself. \\
\poeml My own eyes will look at him--- \\
\poemll    there won't be anyone else for me!--- \\
\poemlll       He is the culmination of my innermost desire.''
\passage{Job Reminds His Friends of Judgment}
\poeml \v{28}``When you're thinking about asking yourselves, \\
\poemll    `How will we pursue him, \\
\poemlll       since the root of the problem is with him?'\fnote{\fbackref{19:28} Lit \fbib{me}} \\
\poeml \v{29}Make sure that you remain wary of God's sword, \\
\poemll    for God's wrath brings with it the sword of punishment, \\
\poemlll       by which you'll know there's a judgment.''
\end{poetry}
\labelchapt{20}
\passage{Zophar Speaks the Second Time}

\chapt{20}
\v{1}Then Zophar from Naamath replied:

\begin{poetry}
\poeml \v{2}``Therefore my anxious thoughts cause me to answer \\
\poemll    because I'm agitated within me. \\
\poeml \v{3}Whenever I hear an insulting rebuke, \\
\poemll    I respond from my spirit because I understand.''
\passage{Destruction Awaits the Wicked}
\poeml \v{4}``Haven't you known this from ancient times, \\
\poemll    since mankind was placed on the earth? \\
\poeml \v{5}The wicked triumph only briefly; \\
\poemll    the joy of the godless is momentary. \\
\poeml \v{6}Though he grow as tall as the sky, \\
\poemll    or though his head touches the clouds, \\
\poeml \v{7}he'll perish forever, like his own excrement; \\
\poemll    those who saw him will ask, `Where is he?' \\
\poeml \v{8}He'll vanish\fnote{\fbackref{20:8} Lit. \fbib{He'll fly away}} like a dream, and no one will find him; \\
\poemll    he will be chased away like a night vision.'' \\
\poeml \v{9}``An eye that gazes at him won't do so again; \\
\poemll    and his place won't even recognize him. \\
\poeml \v{10}His sons will make amends to the poor; \\
\poemll    their hands will return his wealth. \\
\poeml \v{11}Though his bones were full of youthful vigor; \\
\poemll    yet they will lie down with him in the dust. \\
\poeml \v{12}Though evil tastes sweet in his mouth, \\
\poemll    though he conceals it under his tongue, \\
\poeml \v{13}though he savors it and delays swallowing it \\
\poemll    so he can taste it again and again in his mouth,\fnote{\fbackref{20:13} Lit. \fbib{can hold it in the middle of his palate}} \\
\poeml \v{14}his food will turn rancid in his stomach--- \\
\poemll    it will become a cobra's poison inside him. \\
\poeml \v{15}``Though he swallows wealth, he will vomit it; \\
\poemll    God will dislodge it from his stomach. \\
\poeml \v{16}He will suck the poison of cobras; \\
\poemll    the fangs\fnote{\fbackref{20:16} Or \fbib{tongue}} of a viper will slay him. \\
\poeml \v{17}He won't look at the rivers--- \\
\poemll    the torrents of honey and curd.\fnote{\fbackref{20:17} Or \fbib{butter}} \\
\poeml \v{18}``He will restore what he has attained from his work \\
\poemll    and won't consume it; \\
\poemlll       he won't enjoy the profits from his business transactions, \\
\poeml \v{19}because he has crushed and abandoned the poor; \\
\poemll    he has seized a house that he didn't build. \\
\poeml \v{20}``Since his appetite won't quit;\fnote{\fbackref{20:20} Lit. \fbib{his belly knew no contentment}} \\
\poemll    he won't let anything escape his lust.\fnote{\fbackref{20:20} Lit. \fbib{delight}} \\
\poeml \v{21}Because nothing was left for him to devour, \\
\poemll    therefore his prosperity won't last. \\
\poeml \v{22}Even though he is satiated and self-sufficient, he suffers--- \\
\poemll    everyone in any sort of trouble will attack him. \\
\poeml \v{23}``It will come about that, \\
\poemll    when he has filled himself to the full, \\
\poeml God\fnote{\fbackref{20:23} Lit. \fbib{he}} will vent his burning anger on him; \\
\poemll    he will pour it out on him and on his body. \\
\poeml \v{24}Though he dodges an iron weapon, \\
\poemll    a bronze arrow will pierce him. \\
\poeml \v{25}It will impale him and come out through his back; \\
\poemll    the point will glisten as it protrudes through his gall bladder, \\
\poemlll       and he will be terrified. \\
\poeml \v{26}``Total darkness has been reserved for his treasures; \\
\poemll    a fire that has no need to be kindled will devour him \\
\poemlll       and consume whatever remains in his possession.\fnote{\fbackref{20:26} Lit. \fbib{tent}} \\
\poeml \v{27}Heaven will reveal his iniquity, \\
\poemll    while the earth will rise up against him. \\
\poeml \v{28}A flood will wash away his house; \\
\poemll    dragging it away when God becomes angry. \\
\poeml \v{29}This is what the wicked person inherits from God; \\
\poemll    it is the inheritance that God appoints for him.''
\end{poetry}
\labelchapt{21}
\passage{Job Reasons with Zophar}

\chapt{21}
\v{1}In response, Job said:

\begin{poetry}
\poeml \v{2}``Listen carefully to my words; \\
\poemll    let this encourage all of you. \\
\poeml \v{3}Bear with me and let me speak! \\
\poemll    Then, after I've spoken, you'll be free to mock me. \\
\poeml \v{4}After all, isn't my complaint against a human being? \\
\poemll    If so, why shouldn't I be impatient? \\
\poeml \v{5}Look at me, be appalled, \\
\poemll    and then shut up! \\
\poeml \v{6}When I think about this,\fnote{\fbackref{21:6} The Heb. lacks \fbib{of this}} I'm petrified with terror \\
\poemll    and my body shudders uncontrollably.''
\passage{The Wicked Prospers}
\poeml \v{7}``Why do the wicked live to reach old age \\
\poemll    and increase in power and wealth, too? \\
\poeml \v{8}Their children grow up while they're alive, \\
\poemll    and they live to see their grandchildren. \\
\poeml \v{9}Their houses are safe from fear, \\
\poemll    and God's chastisement\fnote{\fbackref{21:9} Lit. \fbib{rod}} never visits them. \\
\poeml \v{10}Their bull breeds without fail, \\
\poemll    and their cows calve without miscarriages. \\
\poeml \v{11}They release their children to play like sheep; \\
\poemll    their young ones\fnote{\fbackref{21:11} Or \fbib{children}} dance about, \\
\poeml \v{12}singing\fnote{\fbackref{21:12} Lit. \fbib{they take up}} with tambourines and lyres \\
\poemll    as they rejoice to the sound of flutes. \\
\poeml \v{13}They grow old\fnote{\fbackref{21:13} Lit. \fbib{wear out their days}} in prosperity, \\
\poemll    as they descend peacefully into the afterlife.\fnote{\fbackref{21:13} Lit. \fbib{Sheol}; i.e. the abode of the dead} \\
\poeml \v{14}``They say to God, `Turn away from us! \\
\poemll    We have no desire to know your ways. \\
\poeml \v{15}Who is the Almighty, that we should serve him? \\
\poemll    Where's the profit in talking to him?' \\
\poeml \v{16}Behold! Their prosperity isn't in their control! \\
\poemll    The counsel of the wicked will remain far from me.''
\passage{God will Punish the Wicked}
\poeml \v{17}``How often do the wicked have their lights put out? \\
\poemll    Does calamity ever fall on them? \\
\poemlll       Will God\fnote{\fbackref{21:17} Lit. \fbib{he}} in his anger ever apportion their destruction? \\
\poeml \v{18}May they become like a straw, \\
\poemll    blown away before the wind; \\
\poemlll       like a chaff that's swept off by a storm. \\
\poeml \v{19}God stores up their iniquity to repay their children; \\
\poemll    making them\fnote{\fbackref{21:19} Lit. \fbib{him}} repay so that they may be aware. \\
\poeml \v{20}Their own eyes will see their destruction; \\
\poemll    and they'll drink the wrath of the Almighty. \\
\poeml \v{21}What will they care for their household after them, \\
\poemll    when the number of his months comes to an end?''
\passage{Death Levels Everyone}
\poeml \v{22}``Can God learn anything? \\
\poemll    After all, he will judge even the exalted ones. \\
\poeml \v{23}Such persons will die in their full vigor, \\
\poemll    completely prosperous and secure. \\
\poeml \v{24}His buckets are filled with milk, \\
\poemll    his bone marrow is healthy.\fnote{\fbackref{21:24} Lit. \fbib{moist}} \\
\poeml \v{25}Others die with a bitter soul, \\
\poemll    never having tasted the good life.\fnote{\fbackref{21:25} The Heb. lacks \fbib{life}} \\
\poeml \v{26}They both lie down in the dust; \\
\poemll    and worms\fnote{\fbackref{21:26} Lit. \fbib{and a worm}} cover them.''
\passage{Job Suspects His Friends of Treachery}
\poeml \v{27}``Look! I know your thoughts, \\
\poemll    your plans\fnote{\fbackref{21:27} Or \fbib{purposes}} are going to harm me. \\
\poeml \v{28}You ask, `Where is the noble person's house?' \\
\poemll    and `Where are the tents where the wicked live?' \\
\poeml \v{29}Haven't you asked travelers on the highway? \\
\poemll    Don't you accept their word \\
\poeml \v{30}that the wicked person is spared from times of calamity, \\
\poemll    that he is rescued on the day of wrath? \\
\poeml \v{31}Who will expose his conduct to his face? \\
\poemll    Who will repay him for what he has done \\
\poeml \v{32}when he is carried away to the cemetery \\
\poemll    and guardians are placed to watch his tomb? \\
\poeml \v{33}The runoff from the streams will seem sweet to him; \\
\poemll    everyone will follow after him; \\
\poemlll       countless crows march ahead of him. \\
\poeml \v{34}How then, can you console me so worthlessly? \\
\poemll    What is left of your answers is treachery.''
\end{poetry}
\labelchapt{22}
\passage{Eliphaz Speaks a Third Time}

\chapt{22}
\v{1}Then in response, Eliphaz from Teman said:

\begin{poetry}
\poeml \v{2}``Can a human being be useful to God, \\
\poemll    since he, who is wise, is sufficient to himself? \\
\poeml \v{3}Will it please the Almighty if you are innocent, \\
\poemll    or does he profit if your life is\fnote{\fbackref{22:3} Lit. \fbib{your ways are}} blameless? \\
\poeml \v{4}Will he acquit you just because you fear him, \\
\poemll    and render a verdict on your behalf? \\
\poeml \v{5}Your wickedness is great, isn't it? \\
\poemll    There's no limit to your iniquity, is there? \\
\poeml \v{6}``After all, you've taken collateral from your relatives for no reason; \\
\poemll    you stripped the naked of their clothing.\fnote{\fbackref{22:6} I.e. in exchange for a short-term loan} \\
\poeml \v{7}You've neglected to give water to the weary,\fnote{\fbackref{22:7} MT has \fbib{cause the weary to drink}} \\
\poemll    and you've withheld food from the hungry. \\
\poeml \v{8}The land belongs to the powerful, \\
\poemll    and the privileged\fnote{\fbackref{22:8} Lit. \fbib{who lifts the face}} thrive in it. \\
\poeml \v{9}You sent away widows empty-handed, \\
\poemll    and broke the outstretched arms of orphans. \\
\poeml \v{10}That's why disaster surrounds you, \\
\poemll    terror suddenly overwhelms you, \\
\poeml \v{11}you see nothing but darkness, \\
\poemll    and a flood of troubles\fnote{\fbackref{22:11} The Heb. lacks \fbib{of troubles}} drowns you.''
\passage{Eliphaz Acknowledges God but Issues an Imprecatory Prayer}
\poeml \v{12}``Isn't God in heaven above? \\
\poemll    Consider how far away the stars are, \\
\poemlll       and how lofty they are! \\
\poeml \v{13}You've asked, `What does God know? \\
\poemll    Can he sort through pitch black darkness?'\fnote{\fbackref{22:13} Or \fbib{deep darkness}} \\
\poeml \v{14}Thick clouds cover him so he can't see \\
\poemll    as he walks back and forth at heaven's horizon. \\
\poeml \v{15}``Will you keep walking on the traditional path \\
\poemll    that sinners\fnote{\fbackref{22:15} MT has \fbib{men of iniquity}} have tread, \\
\poeml \v{16}who were snatched away before their time; \\
\poemll    when their foundation was swept away by a river? \\
\poeml \v{17}They told God, `Get away from us!' \\
\poemll    and `What will the Almighty do to them?' \\
\poeml \v{18}``Though God\fnote{\fbackref{22:18} Lit. \fbib{he}} fills their houses with good things, \\
\poemll    the counsel of the wicked will remain far from me. \\
\poeml \v{19}The righteous will see this and rejoice; \\
\poemll    the innocent will insult him, saying,\fnote{\fbackref{22:19} The Heb. lacks \fbib{saying}} \\
\poeml \v{20}`Our enemies are sure to be destroyed, \\
\poemll    and fire will burn up what's left of their riches.''
\passage{Eliphaz Challenges Job to Repent}
\poeml \v{21}``Get to know God, and you'll be at peace with him, \\
\poemll    and then prosperity will come to you. \\
\poeml \v{22}Accept what he has to teach you, \\
\poemll    and treasure his words in your heart. \\
\poeml \v{23}``If you return to the Almighty you'll be restored, \\
\poemll    as you remove iniquity from your household.\fnote{\fbackref{22:23} Lit. \fbib{tent}} \\
\poeml \v{24}Bury your gold nuggets in the dust, \\
\poemll    and your source of gold\fnote{\fbackref{22:24} Lit. \fbib{Ophir}; i.e., an ancient source fine gold; cf. 1Chr 29:4} among the stones in a streambed, \\
\poeml \v{25}and then the Almighty will be your gold \\
\poemll    and your refined silver. \\
\poeml \v{26}``Then you'll take delight in the Almighty; \\
\poemll    and will turn your face toward God. \\
\poeml \v{27}You'll entreat him and he'll listen to you \\
\poemll    as you fulfill your vows. \\
\poeml \v{28}When you make a decision on something, \\
\poemll    it will be established for you, \\
\poemlll       and light will brighten\fnote{\fbackref{22:28} Or \fbib{enlighten}} your way. \\
\poeml \v{29}``For when they're humbled, you may respond;\fnote{\fbackref{22:29} Lit. \fbib{say back}} \\
\poemll    `It's their pride!' \\
\poemlll       but he delivers the humble. \\
\poeml \v{30}He'll even deliver the guilty, \\
\poemll    who will be delivered through your innocence.''\fnote{\fbackref{22:30} Lit. \fbib{through the cleanness of your hands}}
\end{poetry}
\labelchapt{23}
\passage{Job Responds to Eliphaz}

\chapt{23}
\v{1}Job's response was to say:

\begin{poetry}
\poeml \v{2}``I'm still complaining bitterly today; \\
\poemll    my hand is heavy because of groaning. \\
\poeml \v{3}If only I knew where to find him, \\
\poemll    I would visit him where he has taken his seat. \\
\poeml \v{4}I would lay out my case before him; \\
\poemll    and fill my mouth with arguments. \\
\poeml \v{5}I know how he would answer me; \\
\poemll    I understand what he'll tell me. \\
\poeml \v{6}``Would he use his great power to fight me? \\
\poemll    No, he'll pay attention to me. \\
\poeml \v{7}In that place, the upright can reason with him; \\
\poemll    and I'll be acquitted once and for all by my judge.''
\passage{Job Justifies His Innocence}
\poeml \v{8}``Look! If I go east,\fnote{\fbackref{23:8} Or \fbib{forward}} he isn't there! \\
\poemll    If I go to the west,\fnote{\fbackref{23:8} Or \fbib{back}} I don't perceive him. \\
\poeml \v{9}If he's working in the north,\fnote{\fbackref{23:9} Or \fbib{left}} I can't observe him;\fnote{\fbackref{23:9} The Heb. lacks \fbib{him}} \\
\poemll    If he turns south,\fnote{\fbackref{23:9} Or \fbib{right}} I can't see him.\fnote{\fbackref{23:9} The Heb. lacks \fbib{him}} \\
\poeml \v{10}Because he knows the road on which I travel, \\
\poemll    when he had tested me, I'll come out like gold. \\
\poeml \v{11}My feet stay where his footsteps lead; \\
\poemll    I kept on his pathway and haven't turned aside. \\
\poeml \v{12}I haven't wandered away from the commands that he has spoken;\fnote{\fbackref{23:12} Lit. \fbib{commands of his lips}} \\
\poemll    I've treasured what he has said\fnote{\fbackref{23:12} Lit. \fbib{treasured the words of his mouth}} more than my own meals.''
\passage{Job Stands Petrified Before God}
\poeml \v{13}``But he is One---who can change him? \\
\poemll    He does whatever he wants to do. \\
\poeml \v{14}He'll complete what he has planned for me; \\
\poemll    he has many things in mind for me! \\
\poeml \v{15}That's why I'm terrified at his presence! \\
\poemll    When I think about it, I'm afraid of him. \\
\poeml \v{16}God has caused me to faint;\fnote{\fbackref{23:16} Or \fbib{tender hearted}} \\
\poemll    the Almighty makes me terrified! \\
\poeml \v{17}Nevertheless, I haven't been silenced because of the darkness, \\
\poemll    even when thick darkness obscures my vision.''\fnote{\fbackref{23:17} Lit. \fbib{face}}
\end{poetry}
\labelchapt{24}
\passage{Job Describes Social Injustice}

\begin{poetry}
\poeml \chapt{24}
\v{1}Why doesn't the Almighty reserve time for judgment? \\
\poemll    and why don't those who know him perceive his days? \\
\poeml \v{2}They move boundary stones,\fnote{\fbackref{24:2} Or \fbib{borders}} \\
\poemll    steal flocks, and pasture them.\fnote{\fbackref{24:2} The Heb. lacks \fbib{them}} \\
\poeml \v{3}They drive away the orphan's donkey; \\
\poemll    they take the ox of the widow as security for a loan;\fnote{\fbackref{24:3} The Heb. lacks \fbib{for a loan}} \\
\poeml \v{4}They push the needy off the road, \\
\poemll    and force the poor of the land into hiding. \\
\poeml \v{5}``Look! Like wild donkeys in the wilderness, \\
\poemll    they work diligently as they seek wild game in the desert, \\
\poemlll       food for them and their young ones. \\
\poeml \v{6}They reap fodder in the field \\
\poemll    and glean in the vineyard of the wicked. \\
\poeml \v{7}They spend the night naked, without clothing, \\
\poemll    with no covering against the cold. \\
\poeml \v{8}They are wet from mountain rains; \\
\poemll    without shelter, they cling to a rock. \\
\poeml \v{9}``The fatherless are torn from the breast; \\
\poemll    the poor are taken away as security for a loan.\fnote{\fbackref{24:9} The Heb. lacks \fbib{for a loan}} \\
\poeml \v{10}They wander around naked, without clothes; \\
\poemll    hungry, though they carry sheaves of grain.\fnote{\fbackref{24:10} The Heb. lacks \fbib{grain}} \\
\poeml \v{11}They press oil between the olive groves owned by the wicked; \\
\poemll    they suffer from thirst, even while treading the winepress. \\
\poeml \v{12}From the city, dying men groan aloud, \\
\poemll    and the wounded cries out for help, \\
\poemlll       but God charges no one with wrong. \\
\poeml \v{13}``Then there are those who rebel against the light; \\
\poemll    they are not acquainted with its ways; \\
\poemlll       and they don't stay on its course.\fnote{\fbackref{24:13} Or \fbib{path}} \\
\poeml \v{14}The murderer rises at dawn to kill the poor and needy; \\
\poemll    at night, he is like a thief. \\
\poeml \v{15}The adulterer watches for twilight,\fnote{\fbackref{24:15} Lit. \fbib{twilight}} \\
\poemll    saying, `No eye is watching me' \\
\poemlll       while he veils his face. \\
\poeml \v{16}They break into houses in the dark; \\
\poemll    during the day they remained sealed in. \\
\poemlll       They don't know daylight. \\
\poeml \v{17}As a group, deep darkness is their morning time; \\
\poemll    fear that lives in darkness is their friend.''
\passage{Social Injustice will Be Punished}
\poeml \v{18}``They remain only a short time on the water's surface; \\
\poemll    their inheritance will be cursed in the land; \\
\poemlll       no one will work in their vineyards. \\
\poeml \v{19}As drought and heat evaporate melting snow, \\
\poemll    that's what Sheol\fnote{\fbackref{24:19} I.e. the realm of the afterlife} does with sinners. \\
\poeml \v{20}The womb will forget them. \\
\poemll    Maggots will find them to be a delicacy! \\
\poeml They won't be remembered anymore, \\
\poemll    their iniquity will be cut to pieces like firewood.\fnote{\fbackref{24:20} Lit. \fbib{like a tree}} \\
\poeml \v{21}``They prey on the barren woman, \\
\poemll    and do no favors for widows. \\
\poeml \v{22}God\fnote{\fbackref{24:22} Lit. \fbib{He}} prolongs the life of the strong by his power, \\
\poemll    but they get up in the morning\fnote{\fbackref{24:22} The Heb. lacks \fbib{in the morning}} without purpose in life. \\
\poeml \v{23}He gives them security and financial support, \\
\poemll    but he watches everything they do. \\
\poeml \v{24}They're exalted momentarily, but then they are gone; \\
\poemll    they are humbled,\fnote{\fbackref{24:24} Or \fbib{brought low}} just like all the others. \\
\poemlll       They are cut down like heads of corn. \\
\poeml \v{25}If this weren't so, who can prove that I'm a liar \\
\poemll    by showing that there's nothing to what I'm saying?''
\end{poetry}
\labelchapt{25}
\passage{Bildad Speaks a Third Time}

\chapt{25}
\v{1}Bildad from Shuah responded and said:

\begin{poetry}
\poeml \v{2}``Dominion and fear belong to God;\fnote{\fbackref{25:2} Lit. \fbib{him}} \\
\poemll    who fashions peace in his high heaven. \\
\poeml \v{3}Is there any limit to his armies? \\
\poemll    On whom does his light not shine?\fnote{\fbackref{25:3} Lit. \fbib{rise}} \\
\poeml \v{4}How can a human being\fnote{\fbackref{25:4} Lit. \fbib{man}} become right with God? \\
\poemll    How can a human being\fnote{\fbackref{25:4} Lit. \fbib{can one born of a woman}} be pure? \\
\poeml \v{5}Behold, even the moon isn't bright, \\
\poemll    and the stars aren't pure in his eyes. \\
\poeml \v{6}How much less is man, who is only a maggot, \\
\poemll    or a man's children, who are only worms!''
\end{poetry}
\labelchapt{26}
\passage{Job Reasons with Bildad}

\chapt{26}
\v{1}In reply, Job responded:

\begin{poetry}
\poeml \v{2}``What a help you are to the weak! \\
\poemll    How powerfully you deliver those without strength! \\
\poeml \v{3}What counsel you provide to the fool! \\
\poemll    What insight you provide so abundantly! \\
\poeml \v{4}Who helped you say all of this? \\
\poemll    Who inspired you?''
\passage{Job Acknowledges God's Power}
\poeml \v{5}``The ghosts of the dead\fnote{\fbackref{26:5} Lit. \fbib{Rephaim}; i.e., souls of the dead} writhe under the waters \\
\poemll    along with those who live there with them. \\
\poeml \v{6}Sheol\fnote{\fbackref{26:6} I.e. the realm of the afterlife} is naked before God\fnote{\fbackref{26:6} Lit. \fbib{him}} \\
\poemll    and Abaddon\fnote{\fbackref{26:6} I.e. the realm of punishment in the afterlife} has no clothes. \\
\poeml \v{7}He spreads out the north over empty space, \\
\poemll    suspending the earth over nothing. \\
\poeml \v{8}``He restricts the waters within clouds \\
\poemll    and the clouds don't burst open under them. \\
\poeml \v{9}He has enclosed the face of the full moon \\
\poemll    and spread his clouds over it. \\
\poeml \v{10}He has delimited a boundary\fnote{\fbackref{26:10} Lit. \fbib{statute}} over the surface of the oceans \\
\poemll    as a limit between light and darkness. \\
\poeml \v{11}The pillars of the heavens tremble \\
\poemll    and are astounded at his rebuke. \\
\poeml \v{12}By his power he disturbs the sea; \\
\poemll    and with his skill he shatters the sea monster.\fnote{\fbackref{26:12} Lit. \fbib{shattered Rahab}} \\
\poeml \v{13}he clears the skies with his wind; \\
\poemll    his hands have pierced the fleeing serpent. \\
\poeml \v{14}Indeed, these are the fringes of his ways, \\
\poemll    and how faint is the whisper we've heard of it! \\
\poemlll       But who can comprehend the thunder of his might?''
\end{poetry}
\labelchapt{27}
\passage{Job Asserts His Innocence}

\chapt{27}
\v{1}Job continued with his discussion and said:

\begin{poetry}
\poeml \v{2}``The living God has withheld justice from me; \\
\poemll    the Almighty has made my life\fnote{\fbackref{27:2} Or \fbib{soul}} bitter. \\
\poeml \v{3}As long as I can breathe; \\
\poemll    as long as God's breath is in my nostrils, \\
\poeml \v{4}I won't speak lies \\
\poemll    nor will I utter deceit. \\
\poeml \v{5}Far be it from me to admit that you're right! \\
\poemll    I intend to maintain my integrity\fnote{\fbackref{27:5} Cf. Job 2:9} even if it kills me! \\
\poeml \v{6}I'll retain my righteousness and not compromise it; \\
\poemll    my conscience won't rebuke me at any time. \\
\poeml \v{7}``May my enemy be like the wicked; \\
\poemll    my adversary like the unjust.\fnote{\fbackref{27:7} Or \fbib{unrighteous one}} \\
\poeml \v{8}For where is the hope of the godless when he is eliminated; \\
\poemll    when God takes away his life? \\
\poeml \v{9}Will God hear his cry \\
\poemll    when distress overtakes him? \\
\poeml \v{10}Will he take delight in the Almighty? \\
\poemll    Will he call on God at all times?''
\passage{On the Demise of the Wicked}
\poeml \v{11}``I'll teach you about the power\fnote{\fbackref{27:11} Lit. \fbib{hand}} of God, \\
\poemll    that which is with the Almighty I won't conceal. \\
\poeml \v{12}Look! All of you have been watching, \\
\poemll    so why have you become so completely worthless? \\
\poeml \v{13}``This is what a wicked person\fnote{\fbackref{27:13} Lit. \fbib{man}} inherits from God, \\
\poemll    and what the ruthless will receive from the Almighty: \\
\poeml \v{14}If he has many children, \\
\poemll    their destiny is to die by the sword, \\
\poemll    and his descendants won't have enough food. \\
\poeml \v{15}Those who do survive him disease will bury, \\
\poemll    and his widow won't even weep. \\
\poeml \v{16}``Though he hoards silver\fnote{\fbackref{27:16} Or \fbib{money}} like dust, \\
\poemll    and stores away garments like clay, \\
\poeml \v{17}whatever he stores up, the righteous will wear, \\
\poemll    and the innocent will inherit that silver! \\
\poeml \v{18}``He has built his house like a moth's cocoon,\fnote{\fbackref{27:18} The Heb. lacks \fbib{cocoon}} \\
\poemll    like a temporary\fnote{\fbackref{27:18} The Heb. lacks \fbib{temporary}} sunshade that a watchman makes. \\
\poeml \v{19}He will go to bed wealthy, \\
\poemll    but won't be doing that anymore! \\
\poemlll       When he opens his eyes, it will be gone! \\
\poeml \v{20}Terror will overtake him like a flood,\fnote{\fbackref{27:20} Lit. \fbib{water}} \\
\poemll    at night, a tornado will sweep him away. \\
\poeml \v{21}He'll be swept up by a storm\fnote{\fbackref{27:21} Lit. \fbib{east}} wind and carried away; \\
\poemll    he'll be whirled away from his place. \\
\poeml \v{22}It will toss him around without pity. \\
\poemll    He'll try to break free\fnote{\fbackref{27:22} Lit. \fbib{grip to flee}, \fbib{he will flee}} from its grip,\fnote{\fbackref{27:22} Lit. \fbib{hand}} \\
\poeml \v{23}but it will clap its hands over him, \\
\poemll    hissing at him as it lunges toward him.''\fnote{\fbackref{27:23} Lit. \fbib{him from its place}}
\end{poetry}
\labelchapt{28}
\passage{Priceless Wisdom is Sourced in God}

\begin{poetry}
\poeml \chapt{28}
\v{1}``Surely there are mines for silver \\
\poemll    and places where gold is refined. \\
\poeml \v{2}Iron is taken from the ground;\fnote{\fbackref{28:2} Or \fbib{dry earth}} \\
\poemll    and copper is smelted from ore. \\
\poeml \v{3}Mankind limits the darkness \\
\poemll    as they search the deepest depths \\
\poemlll       for ore\fnote{\fbackref{28:3} Lit. \fbib{for darkest stone}} in unfathomable darkness. \\
\poeml \v{4}He sinks his shaft far from human habitations, \\
\poemll    in a place\fnote{\fbackref{28:4} The Heb. lacks \fbib{in a place}} forgotten by explorers; \\
\poeml they hang on harnesses \\
\poemll    as they swing back and forth. \\
\poeml \v{5}``While the ground produces food, \\
\poemll    underneath it is torn up and burning hot,\fnote{\fbackref{28:5} Lit. \fbib{is turned up as by fire}} \\
\poeml \v{6}where stones are sapphire \\
\poemll    and gold dust can be found, \\
\poeml \v{7}a place where birds of prey never fly, \\
\poemll    and the eyes of the falcon have never seen. \\
\poeml \v{8}The proud beasts haven't walked there; \\
\poemll    lions have never passed over it. \\
\poeml \v{9}``Using a flint, he thrusts his hand, \\
\poemll    overturning mountains by the roots. \\
\poeml \v{10}He cuts a channel through the rocks, \\
\poemll    while his eyes search for anything of value. \\
\poeml \v{11}He dams up flowing rivers, \\
\poemll    bringing hidden things to light.''
\passage{Wisdom is of Greater Value than Precious Stones}
\poeml \v{12}``Where can wisdom be found? \\
\poemll    Where is understanding's home? \\
\poeml \v{13}Mankind doesn't appreciate their value; \\
\poemll    and you won't find it anywhere on earth.\fnote{\fbackref{28:13} Lit. \fbib{it in the land of the living}} \\
\poeml \v{14}The deepest ocean says, `It's not within me.' \\
\poemll    and the sea says, `You'll never find it with me.' \\
\poeml \v{15}You can't buy it with gold, \\
\poemll    and its value cannot be calculated in silver. \\
\poeml \v{16}It cannot be compared to gold from Ophir,\fnote{\fbackref{28:16} I.e. an ancient source of fine gold; cf. 1Chr 29:4} \\
\poemll    with precious onyx, or with sapphire. \\
\poeml \v{17}It cannot be compared to gold and fine glass\fnote{\fbackref{28:17} The Heb. lacks \fbib{fine glass}} crystal, \\
\poemll    nor can it be exchanged for gold-plated weaponry.\fnote{\fbackref{28:17} Or \fbib{for refined implements made of gold}} \\
\poeml \v{18}Don't even bother to mention coral and crystal--- \\
\poemll    wisdom is more valuable than a bag of rubies.\fnote{\fbackref{28:18} Or \fbib{pearls}} \\
\poeml \v{19}It can neither be compared with the topaz of Ethiopia \\
\poemll    nor valued in comparison to pure gold.''
\passage{Wisdom is from God}
\poeml \v{20}``From where, then, does wisdom originate? \\
\poemll    Where does understanding live?\fnote{\fbackref{28:20} Lit. \fbib{Where is its place?}} \\
\poeml \v{21}It has been concealed from the sight of every living creature \\
\poemll    and hidden even from the birds in the skies. \\
\poeml \v{22}Abaddon\fnote{\fbackref{28:22} I.e. the realm of eternal judgment in the afterlife} and death said, \\
\poemll    `We did hear a rumor about it.' \\
\poeml \v{23}God understands how to get there; \\
\poemll    he knows where they live.\fnote{\fbackref{28:23} Lit. \fbib{knows its place}} \\
\poeml \v{24}For he looks as far as the ends of the earth \\
\poemll    and sees everything under the sky.\fnote{\fbackref{28:24} Or \fbib{under heaven}} \\
\poeml \v{25}``He imparted weight to the wind; \\
\poemll    he regulated water by his measurement. \\
\poeml \v{26}He set in place ordinances for the rain; \\
\poemll    and determined the pathway for thunder that accompanies lightning.\fnote{\fbackref{28:26} \fbib{The sound of a thunderbolt}} \\
\poeml \v{27}Then he looked at wisdom, \\
\poemll    and fixed it in place; \\
\poeml he established it, \\
\poemll    and also examined it. \\
\poeml \v{28}He has commanded mankind: \\
\poemll    `To fear the Lord---that is wisdom; \\
\poemlll       to move away from evil---that is understanding.'\,''
\end{poetry}
\labelchapt{29}
\passage{Job Wishes for the Old Days}

\chapt{29}
\v{1}Then Job continued with his discourse:

\begin{poetry}
\poeml \v{2}``I wish I could go back to how things were a few months ago; \\
\poemll    when God used to watch over me, \\
\poeml \v{3}when his lamp used to shine over my head, \\
\poemll    so I could walk through the dark, \\
\poeml \v{4}like when I was in my prime \\
\poemll    and God trusted me with his secrets!\fnote{\fbackref{29:4} Lit. \fbib{God's counsel was over my tent}} \\
\poeml \v{5}``The Almighty was still with me back then, \\
\poemll    and my children were still around me. \\
\poeml \v{6}I was successful wherever I went,\fnote{\fbackref{29:6} Lit. \fbib{When my feet were bathed in cream;}} \\
\poemll    and even the rocks poured out streams of olive oil for me.''
\passage{Job Remembers His Respected Position}
\poeml \v{7}``Whenever I went out to the city gate, \\
\poemll    a seat had been reserved for me in the plaza.\fnote{\fbackref{29:7} Lit. \fbib{square}; i.e. he served as a ruling elder in his home city} \\
\poeml \v{8}The young men would see me and withdraw, \\
\poemll    and the aged would rise and stand. \\
\poeml \v{9}Nobles would refrain from speaking, \\
\poemll    covering their mouths with their hands. \\
\poeml \v{10}The voices of the commanders-in-chief\fnote{\fbackref{29:10} Lit. \fbib{Nagidim}; i.e. senior officers entrusted with dual roles of operational oversight and administrative authority} were hushed, \\
\poemll    and their tongues would cling to the roofs of their mouths.''
\passage{Job Remembers His Acts of Kindness}
\poeml \v{11}``When people heard me speak, they blessed me; \\
\poemll    when people saw me, they approved me, \\
\poeml \v{12}because I delivered the poor who were crying for help, \\
\poemll    along with orphans who had no one to help them. \\
\poeml \v{13}Those who were about to die blessed me, \\
\poemll    and I made widows sing for joy. \\
\poeml \v{14}I put on righteousness like clothing; \\
\poemll    my just decisions were like a robe and a turban. \\
\poeml \v{15}I served as eyes for the blind \\
\poemll    and feet for the lame. \\
\poeml \v{16}I was a father to the needy; \\
\poemll    I diligently inquired into the case of those I didn't know. \\
\poeml \v{17}I broke the fangs of the wicked, \\
\poemll    and made him drop the prey.''
\passage{Job Remembers His Previous Condition}
\poeml \v{18}``I used to say: `I will die in my home.\fnote{\fbackref{29:18} Lit. \fbib{nest}} \\
\poemll    I'm going to live as many days \\
\poemlll       as there are grains of sand on the shore.\fnote{\fbackref{29:18} The Heb. lacks \fbib{on the shore}} \\
\poeml \v{19}My roots have spread out and have found water, \\
\poemll    and dew settles at night on my branches. \\
\poeml \v{20}My glory renews for me \\
\poemll    and my bow is as good as new in my hand.' \\
\poeml \v{21}``They listened and waited for me, \\
\poemll    as they remained in silence for my counsel. \\
\poeml \v{22}After I spoke, they had nothing to say, \\
\poemll    when what I said hit them. \\
\poeml \v{23}They waited for me as one waits for rain, \\
\poemll    as one opens his mouth to drink in a spring rain shower. \\
\poeml \v{24}I smiled at them when they had no confidence, \\
\poemll    and no one could discourage me. \\
\poeml \v{25}I set an example of the way to live,\fnote{\fbackref{29:25} Lit. \fbib{I chose their way}} as a leader would; \\
\poemll    I lived like a king among his army; \\
\poemlll       like one who comforts mourners.''
\end{poetry}
\labelchapt{30}
\passage{Job Describes His Current Status in Life}

\chapt{30}
\v{1}``But now they mock me;

\begin{poetry}
\poemll    men who are far younger than I, \\
\poeml whose fathers I would have hated \\
\poemll    to entrust with my own sheep dogs. \\
\poeml \v{2}Furthermore, what could I have gained \\
\poemll    from men whose strength is gone? \\
\poeml \v{3}Unproductive due to poverty\fnote{\fbackref{30:3} Or \fbib{want}} and hunger, \\
\poemll    they could only scratch in parched soil, \\
\poemlll       devastated and desolated. \\
\poeml \v{4}``They would pluck off herbs from salt marshes to eat; \\
\poemll    and roots of the broom shrub\fnote{\fbackref{30:4} I.e. a desert bush native to Israel whose bitter roots could be harvested by the destitute and eaten when food was scarce} for food. \\
\poeml \v{5}Driven away from human company, \\
\poemll    they were shouted at as though they were thieves. \\
\poeml \v{6}They lived in the most dangerous of ravines, \\
\poemll    in holes in the ground, and among rocks. \\
\poeml \v{7}They bray like donkeys\fnote{\fbackref{30:7} The Heb. lacks \fbib{like donkeys}} among the bushes \\
\poemll    and huddle together under the desert weeds. \\
\poeml \v{8}Sons of fools and of uncertain reputation,\fnote{\fbackref{30:8} The Or \fbib{and without a name}} \\
\poemll    they have been driven from the land by scourging.''
\passage{Job Presents the Actions of the Mockers}
\poeml \v{9}``Now, I've become the object of their mocking melodies;\fnote{\fbackref{30:9} Lit. \fbib{their neginnoth}} \\
\poemll    I'm nothing but a fool's proverb to them! \\
\poeml \v{10}They abhor me---they keep their distance from me; \\
\poemll    but they don't refrain from spitting at the sight of me. \\
\poeml \v{11}But God\fnote{\fbackref{30:11} Lit. \fbib{he}} has loosened his cord and afflicted me; \\
\poemll    so they've cast off all restraints in my presence. \\
\poeml \v{12}``A wretched crowd ambushes me to my right; \\
\poemll    they trip my feet; \\
\poemlll       they build up their path of calamity for me. \\
\poeml \v{13}They tear up my pathways; \\
\poemll    they profit from my destruction, \\
\poemlll       and they need no help to do this! \\
\poeml \v{14}They come like those who breach through a wall; \\
\poemll    as everything crashes around me they'll roll on and on! \\
\poeml \v{15}My greatest fears have overcome me; \\
\poemll    my honor is assaulted as though by a wind storm; \\
\poemlll       my prosperity evaporates like a morning cloud.''
\passage{Job Accuses God of Mistreating Him}
\poeml \v{16}``Now, my soul pours itself out; \\
\poemll    the time of my affliction has taken control of me. \\
\poeml \v{17}The night racks my bones; \\
\poemll    and the pain that gnaws on me will not rest. \\
\poeml \v{18}My clothes are disheveled by his forceful treatment of me;\fnote{\fbackref{30:18} The Heb. lacks \fbib{of me}} \\
\poemll    he restricts my movement like the collar of my cloak. \\
\poeml \v{19}``He tossed me into the mire; \\
\poemll    I've become like dust and ashes. \\
\poeml \v{20}I cry for help to you, \\
\poemll    but you won't answer me; \\
\poeml I stand still, \\
\poemll    but you only look at me. \\
\poeml \v{21}You changed toward me, and now you're cruel to me; \\
\poemll    with your mighty hand you are persecuting me; \\
\poeml \v{22}you carried me off in a wind storm, \\
\poemll    making me ride on it \\
\poemlll       while you toss me about as the storm roars around me. \\
\poeml \v{23}I know that you're about to kill me, \\
\poemll    so I'm about to go to the house that's appointed for all the living.''
\passage{Job Lists His Hopes Despite His Deplorable Condition}
\poeml \v{24}``Surely he won't stretch his hand against the needy, will he, \\
\poemll    especially if they cry to him in their calamity? \\
\poeml \v{25}Haven't I wept for the one who is going through hard times? \\
\poemll    Haven't I grieved for the needy? \\
\poeml \v{26}I have hoped for good, but evil came instead; \\
\poemll    I have hoped for light, but darkness came. \\
\poeml \v{27}I'm boiling mad inside, and I won't remain silent; \\
\poemll    the time for my affliction to confront me has arrived. \\
\poeml \v{28}``In growing darkness, I walked without sunlight; \\
\poemll    I stood in the congregation to cry for help. \\
\poeml \v{29}I've become a brother to jackals, \\
\poemll    and a friend to ostriches. \\
\poeml \v{30}My skin turns black all over me; \\
\poemll    and my bones seem burned from the heat. \\
\poeml \v{31}But my harp is in mourning; \\
\poemll    my flute plays only songs for those who are weeping.''
\end{poetry}
\labelchapt{31}
\passage{Job Asserts His Moral Innocence}

\begin{poetry}
\poeml \chapt{31}
\v{1}``I made a covenant with my eyes; \\
\poemll    how, then, can I focus my attention on a virgin? \\
\poeml \v{2}What would I have\fnote{\fbackref{31:2} The Heb. lacks \fbib{would one have}} from God above, \\
\poemll    what heritage from the Almighty on high, \\
\poeml \v{3}if not calamity that is due the unjust, \\
\poemll    and misfortune that is due those who practice iniquity? \\
\poeml \v{4}He watches my life, \\
\poemll    observing every one of my actions,\fnote{\fbackref{31:4} Lit. \fbib{steps}} does he not?''
\passage{No Lies and Deception}
\poeml \v{5}``If I've lived my life in the company of vanity, \\
\poemll    or run quickly to embrace deception, \\
\poeml \v{6}let my righteousness be weighed in honest scales, \\
\poemll    and God will make known my integrity. \\
\poeml \v{7}If I have stepped away from the way, \\
\poemll    or if my heart covets whatever my eyes see, \\
\poemlll       or if some other blemish clings to my hands, \\
\poeml \v{8}what I've planted, let another eat \\
\poemll    or let my crops be uprooted.''
\passage{No Adultery}
\poeml \v{9}``If my heart has been seduced by a woman \\
\poemll    and I've laid in wait at my friend's door, \\
\poeml \v{10}then let my wife cook\fnote{\fbackref{31:10} Lit. \fbib{grind}} for another person \\
\poemll    and may someone else sleep with her, \\
\poeml \v{11}because something as lascivious as that \\
\poemll    is an iniquity that should be judged. \\
\poeml \v{12}The fires of Abaddon\fnote{\fbackref{31:12} Or \fbib{Destruction}; i.e. the realm of eternal punishment in the afterlife} will burn,\fnote{\fbackref{31:12} Lit. \fbib{consume}} \\
\poemll    disrupting every part of my eternal reward.''\fnote{\fbackref{31:12} Lit. \fbib{my harvest}}
\passage{No Abuse of Servants}
\poeml \v{13}``If I've refused to help my male and female servants \\
\poemll    when they complain against me, \\
\poeml \v{14}what will I do when God stands up to act? \\
\poemll    When he asks the questions, how will I answer him? \\
\poeml \v{15}The one who made me in the womb made them,\fnote{\fbackref{31:15} Lit. \fbib{him}} too, didn't he? \\
\poemll    Didn't the same one prepare each of us in the womb?''
\passage{No Injustice on the Poor}
\poeml \v{16}``If I refused to grant the desire of the poor \\
\poemll    or exhausted the eyes of the widow, \\
\poeml \v{17}if I ate my meals by myself \\
\poemll    without feeding orphans, \\
\poeml \v{18}(even a poor man had grown up with me as if I were his father, \\
\poemll    and even though I had guided the widow\fnote{\fbackref{31:18} Lit. \fbib{her}} \\
\poemlll       from the time I was born), \\
\poeml \v{19}if I've observed someone who is about to die for lack of clothes \\
\poemll    or if I have no clothing to give to the poor, \\
\poeml \v{20}if he hadn't thanked me from the bottom of his heart,\fnote{\fbackref{31:20} Lit. \fbib{hadn't blessed me from his loins}} \\
\poemll    if he had not been warmed by wool from my sheep, \\
\poeml \v{21}if I've raised my hand against an orphan \\
\poemll    when I thought I would against him in court,\fnote{\fbackref{31:21} Lit. \fbib{when I saw help for me at the gate,}} \\
\poeml \v{22}then let my arm\fnote{\fbackref{31:22} Lit. \fbib{side}} fall from its socket; \\
\poemll    and may my arm be torn off at the shoulder. \\
\poeml \v{23}For I'm terrified of what calamity God may have in store for me; \\
\poemll    and I cannot endure his grandeur.''
\passage{No Trust in Wealth and Heavenly Bodies}
\poeml \v{24}``If I've put my confidence in gold, \\
\poemll    if I've told gold, `You're my security,' \\
\poeml \v{25}if I've found joy in great wealth that I own, \\
\poemll    if I've earned a lot with my own hands, \\
\poeml \v{26}if I look at the sun\fnote{\fbackref{31:26} Lit. \fbib{light}} when it shines \\
\poemll    or the moon as it rises in steady splendor, \\
\poeml \v{27}so that in the depths of my deceived heart \\
\poemll    I worshipped them with my mouth and hands, \\
\poeml \v{28}this is also a sin that deserves to be judged, \\
\poemll    since I would have tried to deceive\fnote{\fbackref{31:28} Or \fbib{have denied}} God above.''
\passage{No Rejoicing over the Plight of Adversary}
\poeml \v{29}``Have I rejoiced in the destruction of those who hate me, \\
\poemll    or have I been happy that evil caught up with him? \\
\poeml \v{30}No, I haven't allowed my mouth to sin \\
\poemll    by asking for his life\fnote{\fbackref{31:30} Lit. \fbib{soul}} with a curse. \\
\poeml \v{31}People in my household have said, \\
\poemll    `We cannot find anyone who has not been satisfied with his meat,' haven't they? \\
\poeml \v{32}No stranger ever spent the night in the street, \\
\poemll    because I opened my doors to travelers.''
\passage{No Secret Sins}
\poeml \v{33}``Have I covered my transgression like other people, \\
\poemll    to conceal iniquity within myself?\fnote{\fbackref{31:33} Or \fbib{bosom}} \\
\poeml \v{34}Have I feared large crowds? \\
\poemll    Has my family's contempt ever terrified me \\
\poemlll       so that I remained silent and wouldn't go outside?''
\passage{Request for A Hearing}
\poeml \v{35}``Who will grant me a hearing? \\
\poemll    Here's my signature\fnote{\fbackref{31:35} Lit. \fbib{seal}}---let the Almighty answer! \\
\poeml Since my adversary indicted me, \\
\poeml \v{36}I'll wear it on my shoulder, \\
\poemlll       or tie it on my head for a crown! \\
\poeml \v{37}I'll give an account for every step I've taken; \\
\poemll    I'll approach him confidently like a Commander-in-Chief.''\fnote{\fbackref{31:37} Lit. \fbib{Nagid}; i.e. a senior officer entrusted with dual roles of operational oversight and administrative authority}
\passage{No Abuse of the Land}
\poeml \v{38}``If my land were to cry out against me \\
\poemll    or if all its furrows wept as one, \\
\poeml \v{39}If I've consumed its produce\fnote{\fbackref{31:39} Lit. \fbib{strength}} without paying for it \\
\poemll    and snuffed out the life of its owners; \\
\poeml \v{40}may thorns spring up instead of wheat, \\
\poemll    and obnoxious weeds instead of barley.''
\end{poetry}

With this, Job's discourse with his friends\fnote{\fbackref{31:40} The Heb. lacks \fbib{with his friends}} is completed.
\labelchapt{32}
\passage{Elihu Addresses Job and His Friends}

\chapt{32}
\v{1}These three men stopped responding to Job, because he was claiming to be righteous, in his own opinion.\fnote{\fbackref{32:1} Lit. \fbib{eyes}} \v{2}But then Barachel's son Elihu from Buz, one of Ram's descendants, got really angry. He was furious with Job because he had been declaring himself righteous instead of vindicating God. \v{3}Furthermore, he was furious with his three friends because they had not answered Job, but instead had condemned him. \v{4}Elihu waited to have a word with Job, since the others were older than he, \v{5}but when he saw that there had been no response\fnote{\fbackref{32:5} Lit. \fbib{mouth}} from those three, he got even more angry. \v{6}Barachel's son Elihu from Buz responded and said:

\begin{poetry}
\poeml I'm younger than you are. \\
\poemll    Because you're older,\fnote{\fbackref{32:6} Lit. \fbib{aged}} I was terrified \\
\poemlll       to tell you what I know. \\
\poeml \v{7}I thought, experience\fnote{\fbackref{32:7} Lit. \fbib{days}} should speak; \\
\poemll    abundance of years teaches wisdom. \\
\poeml \v{8}However, a spirit exists in mankind, \\
\poemll    and the Almighty's breath gives him insight.
\passage{There's No Fool Like an Old Fool}
\poeml \v{9}``The aged aren't always wise, \\
\poemll    nor do the elderly always understand justice. \\
\poeml \v{10}Therefore I'm saying, `Listen to me!' \\
\poemll    Then I'll declare what I know. \\
\poeml \v{11}``Look! I have waited to hear your speech, \\
\poemll    so I listened to your insights \\
\poemlll       while you searched for the right words to say.\fnote{\fbackref{32:11} The Heb. lacks \fbib{to say}} \\
\poeml \v{12}Indeed, I paid close attention to you all, \\
\poemll    but none of you were able to refute\fnote{\fbackref{32:12} Or \fbib{rebuke}} Job \\
\poemlll       or answer his arguments convincingly. \\
\poeml \v{13}``So that you cannot claim, `We have found wisdom!' \\
\poemll    let God do the rebuking, not man; \\
\poeml \v{14}let him not direct a rebuke toward me. \\
\poemll    I won't be responding to him with your arguments. \\
\poeml \v{15}``Job's friends\fnote{\fbackref{32:15} Lit. \fbib{They}} won't reason with him anymore; \\
\poemll    discouraged, words escape them. \\
\poeml \v{16}Shall I continue to wait, since they're no longer talking? \\
\poemll    After all, they're only standing there; \\
\poemlll       they're no longer responding. \\
\poeml \v{17}``I will contribute my arguments\fnote{\fbackref{32:17} Lit. \fbib{portion}} as an answer; \\
\poemll    I'll declare what I know, \\
\poeml \v{18}because I'm filled with things to say, \\
\poemll    and my spirit within me compels me to speak.\fnote{\fbackref{32:18} The Heb. lacks \fbib{to speak}} \\
\poeml \v{19}My insides feel like unvented wine, \\
\poemll    like it's about to burst like a new wineskin. \\
\poeml \v{20}``Let me speak! I need relief! \\
\poemll    Let me open my lips and respond. \\
\poeml \v{21}I won't discriminate against anyone, \\
\poemll    and I won't flatter any person, \\
\poeml \v{22}since I don't know the first thing about how to flatter; \\
\poemll    and the one who made me would sweep me away \\
\poemlll       as if I were nothing.''
\end{poetry}
\labelchapt{33}
\passage{Elihu Begins His Discourse}

\begin{poetry}
\poeml \chapt{33}
\v{1}``Now please listen to what I have to say, Job. \\
\poemll    Listen to every word! \\
\poeml \v{2}Look! I've begun to speak,\fnote{\fbackref{33:2} Lit. \fbib{I've opened my mouth}} \\
\poemll    and I'm fashioning my words.\fnote{\fbackref{33:2} Lit. \fbib{and my tongue speaks in my mouth}} \\
\poeml \v{3}I speak from the innocence\fnote{\fbackref{33:3} Or \fbib{integrity}} of my heart; \\
\poemll    and my lips will utter what I sincerely know. \\
\poeml \v{4}``The spirit of God fashioned me; \\
\poemll    and the breath of the Almighty gives me life. \\
\poeml \v{5}Answer me, if you can! \\
\poemll    Present your case! Take your stand in my presence! \\
\poeml \v{6}Look! As far as God is concerned,\fnote{\fbackref{33:6} Lit. \fbib{Look! Before God}} I'm just like you are--- \\
\poemll    I, too, have been pinched off from a piece of clay. \\
\poeml \v{7}Don't be afraid of me; \\
\poemll    I'll go easy\fnote{\fbackref{33:7} Lit. \fbib{my hand won't be heavy}} on you.''
\passage{Elihu Reviews Job's Claim for Innocence}
\poeml \v{8}``You spoke clearly so I could hear; \\
\poemll    I've heard what you've said: \\
\poeml \v{9}`I'm pure. I'm without sin; \\
\poemll    I'm innocent. I'm harboring no iniquity inside of me. \\
\poeml \v{10}Nevertheless, God\fnote{\fbackref{33:10} Lit. \fbib{he}} has found a pretext to attack me; \\
\poemll    he considers me his enemy. \\
\poeml \v{11}He has bound my feet in shackles, \\
\poemll    and keeps watching everything I do.'\,''\fnote{\fbackref{33:11} Lit. \fbib{watching all my paths}}
\passage{God Responds to Humanity's Need}
\poeml \v{12}``You aren't right about this; \\
\poemll    My response is that God is greater than human beings. \\
\poeml \v{13}Why are you arguing with him? \\
\poemll    He doesn't have to give explanations for what he does to you! \\
\poeml \v{14}``God speaks time and time again\fnote{\fbackref{33:14} Lit. \fbib{speaks once and twice}}--- \\
\poemll    but nobody notices--- \\
\poeml \v{15}in a dream or night vision, \\
\poemll    when a deep sleep falls on mankind \\
\poemlll       while they sleep on their beds. \\
\poeml \v{16}That's when he opens the ear of mankind, \\
\poemll    authenticating his messages\fnote{\fbackref{33:16} Lit. \fbib{mankind, sealing his instruction}} to them, \\
\poeml \v{17}turning a person from his actions, \\
\poemll    keeping him\fnote{\fbackref{33:17} Lit. \fbib{man}} from pride, \\
\poeml \v{18}sparing his soul from the Pit\fnote{\fbackref{33:18} I.e. the realm of punishment in the afterlife} \\
\poemll    and his life from violent death.\fnote{\fbackref{33:18} Lit. \fbib{from death by the sword}} \\
\poeml \v{19}``He is being reproved by painful bed rest, \\
\poemll    with continual aching in his bones. \\
\poeml \v{20}He cannot stand his food, \\
\poemll    and he\fnote{\fbackref{33:20} Lit. \fbib{his soul}} has no desire for appetizing food. \\
\poeml \v{21}His flesh wastes away; \\
\poemll    his bones, which once couldn't be seen, are visible. \\
\poeml \v{22}His soul is getting close to the Pit;\fnote{\fbackref{33:22} I.e. the realm of punishment in the afterlife} \\
\poemll    his life is approaching its executioner.''
\passage{God Delivers through His Ransom}
\poeml \v{23}``If there's a messenger\fnote{\fbackref{33:23} Or \fbib{an angel}} appointed to mediate for Job\fnote{\fbackref{33:23} Lit. \fbib{him}} \\
\poemll    ---one out of a thousand--- \\
\poemlll       to represent the man's integrity on his behalf, \\
\poeml \v{24}to show favor to him and to plead, \\
\poemll    `Deliver him from having to go down to the Pit\fnote{\fbackref{33:24} I.e. the realm of punishment in the afterlife}--- \\
\poemlll       I know where his ransom is!' \\
\poeml \v{25}Let his flesh be rejuvenated\fnote{\fbackref{33:25} Lit. \fbib{grew fresh}} as he was in his youth! \\
\poemll    Let him recover the strength of his youth. \\
\poeml \v{26}Let him pray to God \\
\poemll    and he will accept him; \\
\poemlll       he will appear before him with joyful shouts!''
\passage{The Song of the Ransomed}
\poeml \v{27}``He'll sing to mankind with these words: \\
\poeml `I've sinned. I have twisted what is right. \\
\poemll    Yet he has not repaid me like I deserve.\fnote{\fbackref{33:27} The Heb. lacks \fbib{like I deserve}} \\
\poeml \v{28}He has redeemed my soul from going down to the Pit;\fnote{\fbackref{33:28} I.e. the realm of punishment in the afterlife} \\
\poemll    my life will see the light.' \\
\poeml \v{29}Indeed God does all these things \\
\poemll    again and again\fnote{\fbackref{33:29} Lit. \fbib{things twice, three times}} with a person \\
\poeml \v{30}to bring back his soul from the Pit;\fnote{\fbackref{33:30} I.e. the realm of punishment in the afterlife} \\
\poemll    to light him with the light of life.''
\passage{Elihu Invites Job to Respond}
\poeml \v{31}``Job, pay attention! Listen to me! \\
\poemll    Be silent and let me speak. \\
\poeml \v{32}If you have anything to say, answer me; \\
\poemll    speak up, because I'd be happy to vindicate you. \\
\poeml \v{33}But if you have nothing to say, then at least listen to me! \\
\poemll    Be quiet and learn some wisdom from me.''
\end{poetry}
\labelchapt{34}
\passage{Elihu Continues Speaking}

\chapt{34}
\v{1}Elihu continued speaking, and said:

\begin{poetry}
\poeml \v{2}``Listen to what I have to say, you wise men! \\
\poemll    Pay attention to me, you educated people! \\
\poeml \v{3}Since the ear tests words \\
\poemll    like a palate tastes food, \\
\poeml \v{4}let's choose what's right for us. \\
\poemll    Let's consider among ourselves what is good.''
\passage{Elihu Reviews Job's Complaint against God's Injustice}
\poeml \v{5}Now this is Job's claim: \\
\poeml `Even though I'm innocent, \\
\poemll    God has stopped treating me righteously. \\
\poeml \v{6}Have I lied concerning the justice that I deserve?\fnote{\fbackref{34:6} Lit. \fbib{concerning my justice}} \\
\poemll    My wound\fnote{\fbackref{34:6} Or \fbib{cut}} is incurable, \\
\poemlll       though transgression cannot be attributed to me.' \\
\poeml \v{7}``What man is like Job, \\
\poemll    who drinks mockery like water, \\
\poeml \v{8}traffics in those who practice evil, \\
\poemll    and walks with wicked people? \\
\poeml \v{9}Because he says, `There's no profit \\
\poemll    for a man to find joy with God.'\,''\fnote{\fbackref{34:9} Cf. Mal 3:14}
\passage{God is Just}
\poeml \v{10}``Therefore you men of understanding,\fnote{\fbackref{34:10} Lit. \fbib{heart}} listen to me! \\
\poemll    Far be it for God to practice wickedness, \\
\poemlll       or the Almighty to do what is wrong, \\
\poeml \v{11}because he repays a person for his behavior; \\
\poemll    and according to a person's\fnote{\fbackref{34:11} Lit. \fbib{man}} conduct, \\
\poemlll       he lets it happen to\fnote{\fbackref{34:11} Lit. \fbib{it find}} him. \\
\poeml \v{12}Truly, God doesn't practice wickedness, \\
\poemll    and the Almighty doesn't pervert justice. \\
\poeml \v{13}Who entrusted the earth to him? \\
\poemll    Who made him responsible for the entire inhabited world? \\
\poeml \v{14}If he were to decide to do so, \\
\poemll    that is, to take back to himself\fnote{\fbackref{34:14} The Heb. lacks \fbib{to himself}} his spirit and breath of life,\fnote{\fbackref{34:14} The Heb. lacks \fbib{of life}} \\
\poeml \v{15}every living thing would die all at once,\fnote{\fbackref{34:15} Lit. \fbib{die together}} \\
\poemll    and mankind would return to dust.''
\passage{God's Rule is Just}
\poeml \v{16}If you have\fnote{\fbackref{34:16} The Heb. lacks \fbib{you have}} understanding, listen to this! \\
\poemll    Pay attention to what I have to say: \\
\poeml \v{17}Can one who hates justice really govern? \\
\poemll    And if God\fnote{\fbackref{34:17} Lit. \fbib{he}} is righteous and mighty, can you condemn him?\fnote{\fbackref{34:17} The Heb. lacks \fbib{him}} \\
\poeml \v{18}Can one say to a king, `You're vile!' \\
\poemll    or to nobles, `You're wicked!'? \\
\poeml \v{19}Who isn't partial to\fnote{\fbackref{34:19} Lit. \fbib{Who doesn't lift the faces of}} princes? \\
\poemll    Who doesn't give preference to the nobles over the poor? \\
\poemlll       Nevertheless, all of them are his handiwork. \\
\poeml \v{20}``They die suddenly, in the middle of the night; \\
\poemll    people suffer seizures and pass away; \\
\poeml even valiant men can be taken away--- \\
\poemll    and not by human hands. \\
\poeml \v{21}Yes, Job,\fnote{\fbackref{34:21} The Heb. lacks \fbib{Job}} his eyes constantly watch the behavior of human beings; \\
\poemll    he carefully observes their every step. \\
\poeml \v{22}There's no such thing as darkness to him--- \\
\poemll    not even deep darkness--- \\
\poemlll       that can conceal those who practice evil. \\
\poeml \v{23}He won't examine mankind further, \\
\poemll    that they would go before God to judgment. \\
\poeml \v{24}He shatters valiant men without a need to investigate, \\
\poemll    and he raises others in their place. \\
\poeml \v{25}Thus he acknowledges their behavior, and overcomes them; \\
\poemll    when night time comes, they are crushed. \\
\poeml \v{26}``He strikes\fnote{\fbackref{34:26} Or \fbib{slaps}} the wicked among them \\
\poemll    in a place where they can be seen \\
\poeml \v{27}because they've abandoned their pursuit of him \\
\poemll    and had no respect for any of his ways. \\
\poeml \v{28}As a result, the cries of the poor have reached him \\
\poemll    and he has heard the cry of the afflicted. \\
\poeml \v{29}``If he remains silent, who will condemn him? \\
\poemll    If he conceals his face, who can see him? \\
\poemlll       He watches over both nation and individual alike, \\
\poeml \v{30}to keep the godless man from reigning \\
\poemll    or laying a snare for the people.''
\passage{Elihu's Challenge to Job}
\poeml \v{31}``Has anyone ever really told God, \\
\poemll    `I've endured,\fnote{\fbackref{34:31} Or \fbib{carry}} and I won't act corruptly anymore. \\
\poeml \v{32}What I don't see, instruct me! \\
\poemll    If I've done anything evil, I won't repeat it!' \\
\poeml \v{33}``Should you not be paid back, \\
\poemll    since you have rejected him? \\
\poeml You do the choosing! I won't! \\
\poemll    Tell us what you know!
\passage{Elihu's Verdict: Job is not Wise}
\poeml \v{34}``Men of understanding, speak to me! \\
\poemll    Are any of you men wise? Then listen to me! \\
\poeml \v{35}Job has been speaking from his own ignorance, \\
\poemll    and what he has to say lacks insight! \\
\poeml \v{36}Oh, how Job needs to be given a full court trial, \\
\poemll    as a rebuke to those who practice evil, \\
\poeml \v{37}because he has been adding rebellion to his sin; \\
\poemll    he claps his hands among us,\fnote{\fbackref{34:37} I.e. as a gesture of disrespect} \\
\poemlll       and keeps on ranting against God.''
\end{poetry}
\labelchapt{35}
\passage{Elihu Speaks Again}

\chapt{35}
\v{1}In response, Elihu said:

\begin{poetry}
\poeml \v{2}``Are you saying that it's just for you to claim, \\
\poemll    `I'm more righteous than God?' \\
\poeml \v{3}After all, you've asked what your benefit will be: \\
\poemll    `What will I profit from refraining from sin?' \\
\poeml \v{4}I'm going to respond to that statement, \\
\poemll    and to your friends with you.''
\passage{God's Justice Remains Unsullied}
\poeml \v{5}``Observe the heavens! Take a look around! \\
\poemll    Look! The clouds are higher than you, aren't they? \\
\poeml \v{6}If you sin, what will that do to harm him? \\
\poemll    If you add transgression to transgression \\
\poemlll       what will it do to him? \\
\poeml \v{7}If you are righteous, what will you add to him? \\
\poemll    What can God receive from your efforts?\fnote{\fbackref{35:7} Lit. \fbib{hand}} \\
\poeml \v{8}Your wickedness affects only\fnote{\fbackref{35:8} The Heb. lacks \fbib{only}} yourself; \\
\poemll    and your righteousness, only human beings.\fnote{\fbackref{35:8} Lit. \fbib{only a son of man}} \\
\poeml \v{9}``They cry out because they have many oppressors; \\
\poemll    they cry for help because the powerful are abusing them.\fnote{\fbackref{35:9} Lit. \fbib{because of the arm of the powerful}} \\
\poeml \v{10}He never asks, `Where is God, my Creator, \\
\poemll    who gives me songs in the night, \\
\poeml \v{11}who teaches us more than the earth's wild animals, \\
\poemll    and makes us wiser than the birds of the sky?' \\
\poeml \v{12}``They cry out there, but he doesn't answer \\
\poemll    because of the arrogance of those who practice evil. \\
\poeml \v{13}Theirs is a useless plea--- \\
\poemll    God won't listen; \\
\poemlll       the Almighty won't pay any attention. \\
\poeml \v{14}Even though you complain that you can't perceive him, \\
\poemll    your case is already pending for judgment in his presence \\
\poemlll       so keep on placing your hope in him. \\
\poeml \v{15}``So now, if he doesn't inflict punishment in his anger, \\
\poemll    then he doesn't keep track of your many transgressions. \\
\poeml \v{16}When he began speaking, he communicated only worthlessness; \\
\poemll    he added words upon words without knowing anything.''
\end{poetry}
\labelchapt{36}
\passage{Elihu Concludes His Arguments}

\chapt{36}
\v{1}Elihu responded again and said:

\begin{poetry}
\poeml \v{2}``Be patient with me a moment longer, \\
\poemll    and I'll show you that there's more to say on God's behalf. \\
\poeml \v{3}I'll take what I know to its logical conclusion\fnote{\fbackref{36:3} Lit. \fbib{I'll bring my knowledge from a long ways away}} \\
\poemll    and ascribe righteousness to my Creator, \\
\poeml \v{4}because what I have to say isn't deceptive, \\
\poemll    and the one who has perfect knowledge is with you.''
\passage{God Disciplines}
\poeml \v{5}``Indeed God is mighty and he doesn't show disrespect; \\
\poemll    he is mighty and strong of heart. \\
\poeml \v{6}He doesn't let the wicked live; \\
\poemll    he grants justice to the afflicted. \\
\poeml \v{7}He won't stop looking at righteous people; \\
\poemll    he seats them on thrones with kings forever, \\
\poemlll       and they are exalted. \\
\poeml \v{8}``If they're bound in chains, \\
\poemll    caught in ropes of affliction, \\
\poeml \v{9}he'll reveal their actions to them, \\
\poemll    when their transgressions have become excessive. \\
\poeml \v{10}He opens their ears and instructs them, \\
\poemll    commanding them to repent from evil. \\
\poeml \v{11}If they listen and serve him,\fnote{\fbackref{36:11} The Heb. lacks \fbib{him}} \\
\poemll    they'll finish\fnote{\fbackref{36:11} Or \fbib{finish}} their lives in prosperity \\
\poemlll       and their years will be pleasant. \\
\poeml \v{12}``But if they won't listen, \\
\poemll    they'll perish\fnote{\fbackref{36:12} Lit. \fbib{by the sword they'll pass through}} by the sword \\
\poemlll       and die in their ignorance. \\
\poeml \v{13}The godless at heart cherish\fnote{\fbackref{36:13} Lit. \fbib{lay up}} anger; \\
\poemll    they won't cry out for help when God\fnote{\fbackref{36:13} Lit. \fbib{he}} afflicts\fnote{\fbackref{36:13} Lit. \fbib{binds}} them. \\
\poeml \v{14}They\fnote{\fbackref{36:14} Lit. \fbib{Their souls}} die in their youth; \\
\poemll    and their life will end\fnote{\fbackref{36:14} The Heb. lacks \fbib{will end}} among temple prostitutes. \\
\poeml \v{15}He'll deliver the afflicted through their afflictions \\
\poemll    and open their ears when they are oppressed.''
\passage{God is an All-Powerful and Just Teacher}
\poeml \v{16}``Indeed, he drew you away from the brink of distress \\
\poemll    to a spacious place without constraints, \\
\poemlll       filling your festive\fnote{\fbackref{36:16} Lit. \fbib{restful}} table with bountiful\fnote{\fbackref{36:16} Lit. \fbib{fat}} food. \\
\poeml \v{17}But now you are occupied with the case of the wicked; \\
\poemll    but justice and judgment will be served. \\
\poeml \v{18}So that no one entices you with riches, \\
\poemll    don't let a large ransom turn you astray. \\
\poeml \v{19}``Will your wealth sustain you when you're in distress, \\
\poemll    despite your most powerful efforts?\fnote{\fbackref{36:19} Or \fbib{your force of strength}} \\
\poeml \v{20}Don't long for night, \\
\poemll    when people vanish\fnote{\fbackref{36:20} Lit. \fbib{go up}} in their place. \\
\poeml \v{21}Be careful! Don't turn to evil, \\
\poemll    because of this you will be tried by more than affliction. \\
\poeml \v{22}``Indeed, God is exalted in his power. \\
\poemll    Who is like him as a teacher? \\
\poeml \v{23}Who ordained his path for him, \\
\poemll    and who has asked him, `You are wrong, aren't you?' \\
\poeml \v{24}Remember to magnify his awesome activities, \\
\poemll    about which mortal man has sung. \\
\poeml \v{25}All of mankind sees him; \\
\poemll    human beings observe him from afar off.''
\passage{God Controls the Weather}
\poeml \v{26}``God is truly awesome, beyond what we know; \\
\poemll    the number of his years is unknowable.\fnote{\fbackref{36:26} Or \fbib{unreachable}} \\
\poeml \v{27}He draws up drops of water, \\
\poemll    distilling it to rain and mist.\fnote{\fbackref{36:27} Or \fbib{distilling rain into mist}} \\
\poeml \v{28}When the clouds pour down;\fnote{\fbackref{36:28} Or \fbib{drops}} \\
\poemll    they drop their rain on all of humanity. \\
\poeml \v{29}``Furthermore, can anyone understand cloud patterns, \\
\poemll    or the thundering in his pavilion? \\
\poeml \v{30}He scatters his lightning above it, \\
\poemll    and covers the bottom\fnote{\fbackref{36:30} Or \fbib{root}} of the sea. \\
\poeml \v{31}He uses them to judge some people \\
\poemll    and give food to many. \\
\poeml \v{32}His hands are covered with lightning \\
\poemll    that he commands to strike his designated target. \\
\poeml \v{33}His thunder\fnote{\fbackref{36:33} Lit. \fbib{shout}} declares his presence; \\
\poemll    and tells the animals what is coming.''
\end{poetry}
\labelchapt{37}
\passage{Elihu Concludes His Argument}

\begin{poetry}
\poeml \chapt{37}
\v{1}``Now I'll conclude with this: \\
\poemll    my heart is trembling violently; \\
\poemlll       it feels like it's about to leap from my body! \\
\poeml \v{2}Listen carefully to his thundering voice; \\
\poemll    to the sound that rumbles from his mouth. \\
\poeml \v{3}He releases his lightning throughout the sky, \\
\poemll    to the ends\fnote{\fbackref{37:3} Lit. \fbib{wingtips}} of the earth. \\
\poeml \v{4}His thunder roars after it; \\
\poemll    his majestic voice will thunder; \\
\poeml and no one can trace them\fnote{\fbackref{37:4} Lit. \fbib{follow at the heel of their feet}} \\
\poemll    once his voice has been heard. \\
\poeml \v{5}``God thunders with his wondrous voice; \\
\poemll    he does awesome works that we don't comprehend. \\
\poeml \v{6}For he says to the snow, `Fall to the earth.' \\
\poemll    He tells the rain, `Pour down,' \\
\poemlll       then it rains profusely. \\
\poeml \v{7}``He puts a limit to the skill\fnote{\fbackref{37:7} Lit. \fbib{a seal on the hand}} of every person; \\
\poemll    to delineate all people from what they do. \\
\poeml \v{8}``Then a beast enters its lair \\
\poemll    and remains in its den. \\
\poeml \v{9}``From the south,\fnote{\fbackref{37:9} Or \fbib{From a storeroom}} a whirlwind proceeds, \\
\poemll    out of the icy north winds. \\
\poeml \v{10}From the breath of God ice is produced, \\
\poemll    and a wide body of water is frozen. \\
\poeml \v{11}He also loads the clouds with moisture, \\
\poemll    scattering his lightning with the clouds. \\
\poeml \v{12}It whirls about in circles at his direction \\
\poemll    to accomplish all that he commands \\
\poemlll       throughout the surface of the entire world, \\
\poeml \v{13}whether for discipline on his land \\
\poemll    or to demonstrate his gracious love, \\
\poemlll       he causes it to be realized.''
\passage{Elihu Challenges Job to Pay Attention}
\poeml \v{14}``Pay attention to this, Job! \\
\poemll    Stand still, \\
\poemlll       and consider the wondrous attributes of God. \\
\poeml \v{15}Do you know how God ordains them, \\
\poemll    and makes his lightning to flash throughout his clouds? \\
\poeml \v{16}Do you understand his wondrous work of balancing the clouds, \\
\poemll    the one\fnote{\fbackref{37:16} The Heb. lacks \fbib{the one}} whose knowledge is perfect, \\
\poeml \v{17}you whose garments are hot, \\
\poemll    even though the land is cooled by a south wind? \\
\poeml \v{18}Can you spread out the skies like he does; \\
\poemll    can you cast them as one might a mirror? \\
\poeml \v{19}Tell us! What are we to say to him? \\
\poemll    Can we prepare our case to face him \\
\poemlll       when our faces are in darkness? \\
\poeml \v{20}Has it been relayed to God\fnote{\fbackref{37:20} Lit. \fbib{him}} that I want to talk? \\
\poemll    Can a person\fnote{\fbackref{37:20} Lit. \fbib{man}} speak when he is confused?''
\passage{God is Revered}
\poeml \v{21}``So then, the sun\fnote{\fbackref{37:21} Or \fbib{light}} is too bright to gaze at, is it not? \\
\poemll    The sky is swept clean by the wind that blows,\fnote{\fbackref{37:21} Lit. \fbib{that passes through}} is it not? \\
\poeml \v{22}From the north he brings gold; \\
\poemll    around God is awesome splendor. \\
\poeml \v{23}We cannot find the Almighty--- \\
\poemll    he is majestic in power and justice, \\
\poeml and overflowing with righteousness; \\
\poemll    he never oppresses. \\
\poeml \v{24}Therefore humanity fears him, \\
\poemll    which none of the wise\fnote{\fbackref{37:24} Lit. \fbib{wise of heart}} can quite comprehend.''
\end{poetry}
\labelchapt{38}
\passage{The \divine{Lord} Speaks to Job}

\chapt{38}
\v{1}The \divine{Lord} responded to Job from the whirlwind and said:

\begin{poetry}
\poeml \v{2}``Who is this who keeps darkening my counsel \\
\poemll    without knowing what he's talking about? \\
\poeml \v{3}Stand up\fnote{\fbackref{38:3} Lit. \fbib{Gird up your loins}} like a man! \\
\poemll    I'll ask you some questions, \\
\poemlll       and you give me some answers!''
\passage{On the Natural World}
\poeml \v{4}``Where were you when I laid the foundation of my earth? \\
\poemll    Tell me,\fnote{\fbackref{38:4} Or \fbib{declare}} since you're so informed! \\
\poeml \v{5}Who set its measurement? Am I to assume you know? \\
\poemll    Who stretched a boundary line over it? \\
\poeml \v{6}On what were its bases set? \\
\poemll    Who laid its corner stone \\
\poeml \v{7}while the morning stars sang together \\
\poemll    and all the divine beings\fnote{\fbackref{38:7} Lit. \fbib{sons of God}} shouted joyfully? \\
\poeml \v{8}``Who\fnote{\fbackref{38:8} Lit. \fbib{and he}} enclosed the sea with limits\fnote{\fbackref{38:8} Lit. \fbib{doors}} \\
\poemll    when it gushed out of the womb, \\
\poeml \v{9}when I made clouds to be its clothes \\
\poemll    and thick darkness its swaddling blanket, \\
\poeml \v{10}when I proscribed a boundary for it, \\
\poemll    set in place bars and doors for it; \\
\poeml \v{11}and said, `You may come only this far and no more. \\
\poemll    Your majestic waves will stop here.'? \\
\poeml \v{12}``Have you ever commanded the morning at any time during your life?\fnote{\fbackref{38:12} Lit. \fbib{morning in your days}} \\
\poemll    Do you know where the dawn lives, \\
\poeml \v{13}where it seizes the edge of the earth \\
\poemll    and shakes the wicked out of it? \\
\poeml \v{14}Like clay is molded by a signet ring, \\
\poemll    the earth's hills and valleys\fnote{\fbackref{38:14} The Heb. lacks \fbib{the earth's hills and valleys}} then stand out \\
\poemlll       like the colors of a garment. \\
\poeml \v{15}Then from the wicked their light is withheld \\
\poemll    and their upraised arm is broken. \\
\poeml \v{16}``Have you been to the source of the sea \\
\poemll    and walked about in the recesses of the deepest ocean? \\
\poeml \v{17}Have the gates of death been revealed to you? \\
\poemll    Have you seen the gates of the deepest darkness? \\
\poeml \v{18}Do you understand the breadth of the earth? \\
\poemll    Tell me, since you know it all! \\
\poeml \v{19}``Where is the road to where the light lives? \\
\poemll    Or where does the darkness live? \\
\poeml \v{20}Can you take it to its homeland, \\
\poemll    since you know the path to his house? \\
\poeml \v{21}You should know! After all, you had been born back then, \\
\poemll    so the number of your days is great! \\
\poeml \v{22}``Have you entered the storehouses of the snow \\
\poemll    or seen where the hail is stored, \\
\poeml \v{23}which I've reserved for the tribulation to come, \\
\poemll    for the day of battle and war? \\
\poeml \v{24}Where is the lightning diffused \\
\poemll    or the east wind scattered around the earth? \\
\poeml \v{25}``Who cuts canals for storm floods, \\
\poemll    and paths for the lightning and thunder, \\
\poeml \v{26}to bring rain upon a land without inhabitants, \\
\poemll    a desert in which no human beings live, \\
\poeml \v{27}to satisfy a desolate and devastated desert, \\
\poemll    causing it to sprout vegetation? \\
\poeml \v{28}``Does the rain have a father? \\
\poemll    Who fathered the dew? \\
\poeml \v{29}Whose womb brings forth the ice? \\
\poemll    Who gives birth to frost out of an empty\fnote{\fbackref{38:29} The Heb. lacks \fbib{an empty}} sky, \\
\poeml \v{30}when water solidifies\fnote{\fbackref{38:30} Or \fbib{harden}} like stone \\
\poemll    and the surface of the deepest sea freezes?
\passage{On the Heavens}
\poeml \v{31}``Can you bind the chains of Pleiades \\
\poemll    or loosen the cords of Orion? \\
\poeml \v{32}Can you bring out constellations in their season? \\
\poemll    Can you guide the Bear with her cubs? \\
\poeml \v{33}Do you know the laws of the heavens? \\
\poemll    Can you regulate their authority over the earth? \\
\poeml \v{34}``Can you call out to the clouds, \\
\poemll    so that abundant water drenches you? \\
\poeml \v{35}Can you command the lightning, \\
\poemll    so that it goes forth and calls to you, `Look at us!'\fnote{\fbackref{38:35} Lit. \fbib{Here we are}} \\
\poeml \v{36}``Who sets wisdom within you, \\
\poemll    or imbues your mind with understanding? \\
\poeml \v{37}Who has the wisdom to be able to count the clouds, \\
\poemll    or to empty\fnote{\fbackref{38:37} Lit. \fbib{cause to rest}, \fbib{lie down}} the water jars of heaven, \\
\poeml \v{38}when dust dries into a mass \\
\poemll    and then breaks apart into clods?
\passage{On the Animal World}
\poeml \v{39}``Can you hunt prey for the lioness \\
\poemll    to satisfy young lions \\
\poeml \v{40}when they crouch in their dens \\
\poemll    and lie in ambush in their lairs? \\
\poeml \v{41}Who prepares food for the raven, \\
\poemll    when its offspring cry out to God \\
\poemlll       as they wander for lack of food?''
\end{poetry}
\labelchapt{39}
\passage{On the Birth of Young}

\begin{poetry}
\poeml \chapt{39}
\v{1}``Do you know when the mountain goat gives birth? \\
\poemll    Do you watch the doe as it calves its young? \\
\poeml \v{2}Can you count the months of their gestation? \\
\poemll    Do you know the time when they give birth, \\
\poeml \v{3}when they crouch down\fnote{\fbackref{39:3} Or \fbib{bow down}} to give birth\fnote{\fbackref{39:3} Lit. \fbib{cleave open}} to their offspring, \\
\poemll    and let go\fnote{\fbackref{39:3} Lit. \fbib{send}} of their birth pangs? \\
\poeml \v{4}Their young are strong; \\
\poemll    they grow up in the open field; \\
\poeml then they go off \\
\poemll    and don't return to them.''
\passage{On Wild Animals}
\poeml \v{5}``Who sets the wild donkey free? \\
\poemll    Who loosens the bonds of the wild donkey \\
\poeml \v{6}to whom I've given the Arabah\fnote{\fbackref{39:6} I.e. the desert wilderness of southern Israel} for a home; \\
\poemll    the salt plain for his dwelling place? \\
\poeml \v{7}He despises city noises;\fnote{\fbackref{39:7} Or \fbib{sound}} \\
\poemll    he ignores the shouts\fnote{\fbackref{39:7} Or \fbib{noise}} of the driver. \\
\poeml \v{8}He ranges the mountains that are his pasture \\
\poemll    to search for anything green. \\
\poeml \v{9}Is the wild ox willing to serve you? \\
\poemll    Will he sleep at night near your feeding trough? \\
\poeml \v{10}Can you bind the ox to plow a furrow with a rope? \\
\poemll    Will he harrow after you in the valley? \\
\poeml \v{11}Will you trust him because of his great strength \\
\poemll    and entrust your labor to him? \\
\poeml \v{12}Will you trust him that he'll bring in your grain, \\
\poemll    and gather it to your threshing floor?''
\passage{On the Ostrich}
\poeml \v{13}``The wings of the ostrich flap joyously, \\
\poemll    but aren't its pinions and feathers like the stork? \\
\poeml \v{14}She abandons her eggs on the ground \\
\poemll    and lets them be warmed in the sand, \\
\poeml \v{15}but she forgets that a foot might crush them \\
\poemll    or any wild animal might trample them. \\
\poeml \v{16}She mistreats her young as though they're not hers, \\
\poemll    and she has no fear that her labor may be in vain, \\
\poeml \v{17}because God didn't grant her wisdom \\
\poemll    and never gave her understanding. \\
\poeml \v{18}And yet when she gets ready to run, \\
\poemll    she laughs at the horse and its rider.''
\passage{On the Horse}
\poeml \v{19}Do you instill the horse with strength? \\
\poemll    Do you clothe its neck with a mane? \\
\poeml \v{20}Can you make him leap like the locust, \\
\poemll    and make the splendor of his snorting terrifying? \\
\poeml \v{21}He paws the ground\fnote{\fbackref{39:21} The Heb. lacks \fbib{the ground}} in the valley \\
\poemll    and rejoices in his strength; \\
\poemlll       he goes out to face weapons. \\
\poeml \v{22}He scoffs at fear \\
\poemll    and is never scared; \\
\poemlll       he never retreats from a sword. \\
\poeml \v{23}A quiver of arrows rattles against his side, \\
\poemll    along with a flashing spear and a lance. \\
\poeml \v{24}Leaping in his excitement, he takes in\fnote{\fbackref{39:24} Lit. \fbib{swallows}} the ground; \\
\poemll    he cannot stand still when the trumpets sound! \\
\poeml \v{25}When the trumpet blasts he'll neigh, `Aha! Aha!' \\
\poemll    From a distance he can sense war, \\
\poemlll       the war cry of generals,\fnote{\fbackref{39:25} Or \fbib{officers}} and their shouting.''
\passage{On Raptors}
\poeml \v{26}``Is it by your understanding that the hawk flies, \\
\poemll    spreading its wings toward the south? \\
\poeml \v{27}Does the eagle soar high at your command\fnote{\fbackref{39:27} Lit. \fbib{mouth}} \\
\poemll    and build its nest on the highest crags? \\
\poeml \v{28}He dwells on the crags where he makes his home, \\
\poemll    there on the rocky crag is his stronghold. \\
\poeml \v{29}From there he searches for prey, \\
\poemll    and his eyes recognize it from a distance. \\
\poeml \v{30}His young ones feast\fnote{\fbackref{39:30} Lit. \fbib{suck up}} on blood; \\
\poemll    he'll be found wherever there's a carcass.''\fnote{\fbackref{39:30} Or \fbib{slain}}
\end{poetry}
\labelchapt{40}
\passage{The \divine{Lord} Challenges Job Again}

\chapt{40}
\v{1}The \divine{Lord} continued his response to Job by saying:

\begin{poetry}
\poeml \v{2}``Should the one who is fighting the Almighty find fault with him?\fnote{\fbackref{40:2} The Heb. lacks \fbib{him}} \\
\poemll    Let God's accuser answer.''
\end{poetry}
\passage{Job Acknowledges His Limitations}

\v{3}Then Job replied to the \divine{Lord}. He said:

\begin{poetry}
\poeml \v{4}``I must look insignificant to you! \\
\poemll    How can I answer you? \\
\poemlll       I'm speechless.\fnote{\fbackref{40:4} Lit. \fbib{I put my hand over my mouth}} \\
\poeml \v{5}I spoke once, \\
\poemll    but I can't answer; \\
\poeml I tried\fnote{\fbackref{40:5} The Heb. lacks \fbib{tried}} a second time, \\
\poemll    but I won't do so anymore.''
\end{poetry}
\passage{The \divine{Lord} Continues to Interrogate Job}

\v{6}The \divine{Lord} answered Job from the wind storm and told him:

\begin{poetry}
\poeml \v{7}``Stand up\fnote{\fbackref{40:7} Lit. \fbib{Gird up your loins}} like a man! \\
\poemll    I'll ask you some questions, \\
\poemlll       and you give me some answers! \\
\poeml \v{8}Indeed would you annul my justice and condemn me, \\
\poemll    just so you can claim that you're righteous? \\
\poeml \v{9}Do you have strength\fnote{\fbackref{40:9} Lit. \fbib{have an arm}} like God? \\
\poemll    Can you create thunder with a sound\fnote{\fbackref{40:9} Lit. \fbib{voice}} like he can?''
\passage{Can You Save Yourself?}
\poeml \v{10}``When you have adorned yourself with exalted majesty, \\
\poemll    clothed yourself with splendor and dignity,\fnote{\fbackref{40:10} Lit. \fbib{splendor} and \fbib{majesty}} \\
\poeml \v{11}dispensed the fury of your anger, \\
\poemll    made sure\fnote{\fbackref{40:11} Lit. \fbib{see}} that you have humbled every proud person, \\
\poeml \v{12}stared down and subdued every proud person, \\
\poemll    trampled the wicked right where they are, \\
\poeml \v{13}buried\fnote{\fbackref{40:13} MT has \fbib{hide}} them in the dust together, \\
\poemll    and sent them bound to that secret place,\fnote{\fbackref{40:13} I.e. the afterlife} \\
\poeml \v{14}then I will applaud you myself! \\
\poemll    I'll admit that you can deliver yourself by your own efforts!''
\passage{On Behemoth}
\poeml \v{15}``Please observe\fnote{\fbackref{40:15} Lit. \fbib{look}} Behemoth,\fnote{\fbackref{40:15} I.e. an ancient, gigantic herbivore, living in Job's time but now apparently extinct} which I made along with you. \\
\poemll    He eats grass like an ox. \\
\poeml \v{16}Now take a look at the strength that he has in his loins, \\
\poemll    and in the muscles of his abdomen. \\
\poeml \v{17}His tail protrudes stiffly, like cedar;\fnote{\fbackref{40:17} I.e. a genus of coniferous evergreen in the family \fbib{Pinaceae}} \\
\poemll    the sinews of his thigh interlink for strength. \\
\poeml \v{18}His bones are conduits\fnote{\fbackref{40:18} Or \fbib{channels}} of bronze;\fnote{\fbackref{40:18} Or \fbib{copper}} \\
\poemll    his strong bones are like bars of iron. \\
\poeml \v{19}He is the grandest\fnote{\fbackref{40:19} Or \fbib{first}} of God's undertakings,\fnote{\fbackref{40:19} Lit. \fbib{ways}} \\
\poemll    yet his creator is approaching him with his sword.\fnote{\fbackref{40:19} I.e. God was about to make Behemoth extinct} \\
\poeml \v{20}Mountains produce food for him, \\
\poemll    where all the wild animals frolic. \\
\poeml \v{21}He lies under the lotus trees, \\
\poemll    hiding under reeds and marshes.\fnote{\fbackref{40:21} Lit. reed and marsh} \\
\poeml \v{22}The lotus trees cover him with their shade, \\
\poemll    and willows that line the wadis\fnote{\fbackref{40:22} I.e. seasonal streams that channel water during rain seasons but are dry at other times} surround him. \\
\poeml \v{23}What you see as a raging river doesn't alarm him; \\
\poemll    he is confident when the Jordan overflows. \\
\poeml \v{24}Are your eyes looking to capture him, \\
\poemll    or to pierce his snout with a bridle?''
\end{poetry}
\labelchapt{41}
\passage{On Leviathan}

\begin{poetry}
\poemll    \chapt{41}
\v{1}\fnote{\fbackref{41:1} This v. is 40:25 in MT, v2 is 40:26 in MT, and so through v8.}``Can you draw Leviathan\fnote{\fbackref{41:1} I.e. an ancient, gigantic sea creature, living in Job's time but now apparently extinct} out of the water\fnote{\fbackref{41:1} The Heb. lacks \fbib{of the water}} with a hook, \\
\poemll    or tie down\fnote{\fbackref{41:1} Lit. \fbib{or sink}} his tongue with a rope? \\
\poeml \v{2}Can you attach a bridle\fnote{\fbackref{41:2} Lit. \fbib{rope}} to his snout, \\
\poemll    or pierce his jaw with a hook? \\
\poeml \v{3}Will he make many supplications to you, \\
\poemll    or will he beg you for mercy?\fnote{\fbackref{41:3} Lit. \fbib{you with gentle words}} \\
\poeml \v{4}Will he try to make a deal with you, \\
\poemll    so that you may take him in servitude forever? \\
\poeml \v{5}``Will you play with him like a pet bird? \\
\poemll    Will you put a leash on him for your little girls? \\
\poeml \v{6}Will your business be able to buy him, \\
\poemll    Will you divide him among your merchant friends? \\
\poeml \v{7}Will you fill his flesh with harpoons, \\
\poemll    or his head with lances? \\
\poeml \v{8}Lay your hand on him, \\
\poemll    and you'll remember the struggle. \\
\poemlll       You'll never do that again! \\
\poeml \v{9}``Look! Anyone's hope to capture him\fnote{\fbackref{41:9} The Heb. lacks \fbib{to capture him}} will prove itself false; \\
\poemll    anyone would be terrified\fnote{\fbackref{41:9} Or \fbib{subdued}} just by looking at him. \\
\poeml \v{10}No one is fierce enough to dare to arouse him.
\end{poetry}

\begin{poetry}
\poeml ``Who, then, can stand in my presence and face me? \\
\poeml \v{11}Who can take me to court and be reconciled to me? \\
\poemlll       All of heaven is mine. \\
\poeml \v{12}``I won't be silent concerning his limbs, \\
\poemll    his mighty strength, and orderly frame. \\
\poeml \v{13}Who can strip off his outer armor?\fnote{\fbackref{41:13} Lit. \fbib{clothing}} \\
\poemll    Who can approach him with a bridle? \\
\poeml \v{14}Who dares to open his mouth,\fnote{\fbackref{41:14} Lit. \fbib{door of his face}} \\
\poemll    since it is ringed with his terrible teeth! \\
\poeml \v{15}His protective scales are his pride, \\
\poemll    they lie sealed tightly together. \\
\poeml \v{16}Each one is so close to the other \\
\poemll    that not even air comes in between them. \\
\poeml \v{17}Each is attached to the other,\fnote{\fbackref{41:17} Lit. \fbib{with his brother}} \\
\poemll    grasping each other so they cannot be separated. \\
\poeml \v{18}``His snorting releases flashes of light; \\
\poemll    his eyes are like the rays\fnote{\fbackref{41:18} Lit. \fbib{eyelids}} of the dawn. \\
\poeml \v{19}Flames blaze from his mouth; \\
\poemll    streams of sparking fire fly out. \\
\poeml \v{20}Smoke billows from his nostrils; \\
\poemll    like a boiling pot or burning reeds. \\
\poeml \v{21}His breath can ignite coal; \\
\poemll    and flames proceed from his mouth. \\
\poeml \v{22}``His neck is so powerful \\
\poemll    that all who meet him are terrified. \\
\poeml \v{23}There is no flaw in his body's armor; \\
\poemll    it is firmly fixed on him and unbreachable. \\
\poeml \v{24}His heart is as strong as stone, \\
\poemll    it is as hard as a lower millstone. \\
\poeml \v{25}When he rears up, the mighty are terrified; \\
\poemll    they are bewildered as he thrashes about. \\
\poeml \v{26}``Thrusting at him with a sword won't be effective, \\
\poemll    nor will spears, darts, or javelins. \\
\poeml \v{27}He regards iron like straw, \\
\poemll    and hardened bronze like a dead tree. \\
\poeml \v{28}Arrows won't make him flee; \\
\poemll    stones from a sling are only pebbles to him. \\
\poeml \v{29}Clubs are like twigs;\fnote{\fbackref{41:29} Lit. \fbib{stubble}} \\
\poemll    he laughs at the whoosh of the javelin. \\
\poeml \v{30}``Beneath him he is armored as with sharp potsherds; \\
\poemll    he tears through muddy ground \\
\poemlll       like a threshing sledge through grain.\fnote{\fbackref{41:30} The Heb. lacks \fbib{through grain}} \\
\poeml \v{31}He causes the deep to boil like water in\fnote{\fbackref{41:31} The Heb. lacks \fbib{water in}} a pot, \\
\poemll    and churns the sea like one stirs ointment. \\
\poeml \v{32}The sea is luminescent behind him; \\
\poemll    his wake turns the sea white, like those with gray hair. \\
\poeml \v{33}``There's nothing like him on earth; \\
\poemll    he was created without the ability to fear. \\
\poeml \v{34}He looks down on everything that is high; \\
\poemll    he rules over every kind\fnote{\fbackref{41:34} Lit. \fbib{son}} of pride.''
\end{poetry}
\labelchapt{42}
\passage{Job Repents and is Restored}

\chapt{42}
\v{1}Job replied to the \divine{Lord}:

\begin{poetry}
\poeml \v{2}``I know\fnote{\fbackref{42:2} Or \fbib{You know that I know}} that you can do anything \\
\poemll    and nothing that you plan is impossible. \\
\poeml \v{3}You asked,\fnote{\fbackref{42:3} The Heb. lacks \fbib{You asked}} `Who is this that darkens counsel without knowledge?' \\
\poemll    Well now, I have talked about what I don't understand--- \\
\poemlll       awesome things beyond me that I don't know. \\
\poeml \v{4}Listen now, and I will speak for myself; \\
\poemll    I'll interrogate you and then inform me. \\
\poeml \v{5}I've heard you with my ears; \\
\poemll    and now I've seen you with my eyes. \\
\poeml \v{6}As a result, I despise myself and repent \\
\poemll    in dust and ashes.''
\end{poetry}
\passage{Job's Friends are Restored}

\v{7}After these words had been spoken by the \divine{Lord} to Job, the \divine{Lord} spoke to Eliphaz from Teman: ``My anger is burning against you along with your two friends, since you haven't spoken correctly about me, as did my servant Job. \v{8}So take seven bulls and seven rams and bring them to my servant Job. And bring a whole burnt offering for yourselves and my servant Job will pray for you. I'll encourage him\fnote{\fbackref{42:8} Lit. \fbib{I'll lift his face}} by not responding as your disgraceful folly deserves, since you didn't speak about me correctly as did my servant Job. \v{9}So Eliphaz from Teman, Bildad from Shuah, and Zophar from Naamath did precisely as the \divine{Lord} had spoken to them, because the \divine{Lord} showed favor to\fnote{\fbackref{42:9} Lit. \fbib{lift his face}} Job.
\passage{Job's Prosperity Returns}

\v{10}The \divine{Lord} restored Job's prosperity after he prayed for his friends. The \divine{Lord} doubled everything that Job had once possessed. \v{11}Then all his brothers and sisters and all those who knew him before arrived. They ate food with him in his house, mourned for him, and consoled him for all the trouble that the \divine{Lord} had brought and placed on him. Some\fnote{\fbackref{42:11} Lit. \fbib{A man}} gave him gold bullion\fnote{\fbackref{42:11} Lit. \fbib{him one kesitah}; a unit of gold weight, the value of which is unknown today} and some brought\fnote{\fbackref{42:11} Lit. \fbib{and a man}} gold earrings.

\v{12}The \divine{Lord} blessed Job during the latter part of his life\fnote{\fbackref{42:12} The Heb. lacks \fbib{part of his life}} more than the former, since he owned 14,000 sheep, 6,000 camels, 1,000 teams of oxen\fnote{\fbackref{42:12} Or \fbib{1,000 pairs of cattle}} and 1,000 female donkeys. \v{13}He also had seven sons and three daughters. \v{14}He named the first daughter Jemima,\fnote{\fbackref{42:14} The name means \fbib{day by day}} the second Keziah,\fnote{\fbackref{42:14} The name means \fbib{cinnamon}} and the name of the third was Keren-happuch.\fnote{\fbackref{42:14} The name means \fbib{power of antimony}; i.e. an element valued for medicinal uses} \v{15}No one could find more beautiful women in the whole land than Job's daughters. Their father gave them their inheritance along with their brothers. \v{16}Job lived 140 years after this, and saw his children and grandchildren to the fourth generation. \v{17}Then Job died at an old age, having lived a full life.\fnote{\fbackref{42:17} Lit. \fbib{died old and full of days}}
