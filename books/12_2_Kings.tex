\bookheader{2 Kings}
\labelbook{2King}

\bookpretitle{The Book of}
\booktitle{Second Kings}

\labelchapt{1}
\passage{Elijah Rebukes King Ahaziah}

\chapt{1}
\v{1}Moab rebelled against Israel\fnote{\fbackref{1:1} Cf. 2Sam 8:2} after Ahab died. \v{2}Meanwhile, Ahaziah had fallen through the lattice in his upper room in Samaria and lay injured. He sent messengers to Ekron with these orders: ``Go and consult with Ekron's god Baal-zebub to find out\fnote{\fbackref{1:2} The Heb. lacks \fbib{and find out}} if I'm going to recover from this injury.''\fnote{\fbackref{1:2} Lit. \fbib{sickness}}

\v{3}But the angel of the \divine{Lord} spoke to Elijah the foreigner,\fnote{\fbackref{1:3} Lit. \fbib{Tishbite}; or \fbib{sojourner}} ``Get up and go meet the messengers from the king of Samaria. Ask them `Is it because there is no God in Israel that you're going to consult with Ekron's god Baal-zebub? \v{4}Now therefore this is what the \divine{Lord} says: ``You won't be getting up from that bed of yours on which you're lying. You will most certainly die!''\,'\,'' So Elijah got up and\fnote{\fbackref{1:4} The Heb. lacks \fbib{got up and}} went.

\v{5}The messengers returned to the king and he asked them, ``What's this? You've come back?''

\v{6}They replied, ``We met a man who told us, `Go back to the king who sent you and ask him, ``Is it because there is no God in Israel that you're going to consult with Baal-zebub, the god of Ekron? Therefore you won't be getting up from that bed on which you're lying. You will most certainly die!''\,'\,''

\v{7}He told them, ``Describe the man who met you and told you these things.''

\v{8}They answered, ``The man was a hairy fellow. He wore a leather sash around his waist.''

The king\fnote{\fbackref{1:8} Lit. \fbib{He}} responded, ``It's Elijah, that foreigner!''\fnote{\fbackref{1:8} Lit. \fbib{Elijah the Tishbite}; or \fbib{Elijah, the sojourner}}
\passage{Fire from Heaven Destroys the King's Henchmen}

\v{9}So the king sent out 50 men, along with their leader.\fnote{\fbackref{1:9} Lit. \fbib{a captain of fifty}; and so through v. 14; modern equivalent to a Second Lieutenant commanding a platoon of four twelve-member squads} The leader\fnote{\fbackref{1:9} Lit. \fbib{He}} approached Elijah, who was sitting at the top of a hill. He ordered Elijah,\fnote{\fbackref{1:9} Lit. \fbib{him}} ``Hey, man of God! The king orders you to come down!''

\v{10}Elijah responded to the leader who was in charge of the 50 soldiers, ``So I'm a man of God, am I? If so, may fire\fnote{\fbackref{1:10} MT word \fbib{fire} sounds like MT word \fbib{man}} fall from heaven and devour you and your 50 soldiers{\ldots}''\fnote{\fbackref{1:10} The Heb. lacks \fbib{soldiers}} Just then, fire fell from heaven and devoured that leader and his 50 soldiers.\fnote{\fbackref{1:10} The Heb. lacks \fbib{soldiers}}

\v{11}Later the king tried again---he sent another company of 50 soldiers, along with their leader, who ordered Elijah, ``Hey, man of God! This is what the king orders: `Come down!'\,''

\v{12}Elijah responded to the leader and to his entire company,\fnote{\fbackref{1:12} Lit. \fbib{to them}} ``So I'm a man of God, am I? If so, may fire\fnote{\fbackref{1:12} MT word \fbib{fire} sounds like MT word \fbib{man}} fall from heaven and devour you and your 50 soldiers{\ldots}''\fnote{\fbackref{1:12} The Heb. lacks \fbib{soldiers}} Just then, fire fell from heaven and devoured him and his 50 soldiers.\fnote{\fbackref{1:12} The Heb. lacks \fbib{soldiers}}

\v{13}Then the king tried yet again! The king sent a third company of 50 soldiers along with their leader. The third leader went up the hill,\fnote{\fbackref{1:13} The Heb. lacks \fbib{the hill}} approached Elijah,\fnote{\fbackref{1:13} The Heb. lacks \fbib{Elijah}} fell on his knees in front of him, and begged him,\fnote{\fbackref{1:13} Lit. \fbib{Elijah}} ``Hey, man of God, please treat\fnote{\fbackref{1:13} Lit. \fbib{see}} my life and the lives of these servants of yours as precious! \v{14}Look how fire fell from heaven and devoured the two other companies of 50 soldiers, along with their captains, but now please treat me as if my life were precious!''

\v{15}The angel of the \divine{Lord} told Elijah, ``Go down the hill with that man. Don't be afraid of him!'' So Elijah\fnote{\fbackref{1:15} Lit. \fbib{he}} got up and went down with him to meet the king.

\v{16}Then Elijah spoke to the king, ``This is what the \divine{Lord} says: `Since you sent messengers to consult with Baal-zebub, the god of Ekron---is it because there is no God in Israel with whom to consult regarding his word?---therefore you're not getting up from the bed on which you're lying. You certainly will die!'\,'' \v{17}And die he did, just as the \divine{Lord} had said and just as Elijah had spoken!

After this, Jehoram ascended to the throne during the second year of the reign of Jehoshaphat's son Jehoram from Judah. He took the place of Ahaziah, who had no son. \v{18}The rest of Ahaziah's activities are recorded in the Book of the Chronicles of the Kings of Israel,\fnote{\fbackref{1:18} An ancient chronicle of Israel, apparently now lost; and so throughout the book} are they not?
\labelchapt{2}
\passage{Elijah is Taken to Heaven}

\chapt{2}
\v{1}As the time drew near when the \divine{Lord} was about to take Elijah to heaven in a wind storm, Elijah and Elisha were on their way from Gilgal. \v{2}Elijah instructed Elisha, ``Remain here on this side, please, because the \divine{Lord} is sending me as far as Bethel.''

But Elisha replied, ``As the \divine{Lord} lives, I'm not going to leave you while you're still alive!'' So they both went on to Bethel.

\v{3}When the Guild of Prophets\fnote{\fbackref{2:3} Lit. \fbib{The children of the prophets}; and so throughout the book; i.e. a group of disciples within Israel's prophetic community whose precise association with Elijah and Elisha is never specified} who lived in Bethel came out to greet Elisha, they asked him, ``You are aware, aren't you, that later today the \divine{Lord} is going to remove your master from being your mentor?''\fnote{\fbackref{2:3} Lit. \fbib{your head over you}}

``Of course I'm aware of it,'' he said. ``Calm down.''

\v{4}Elijah also spoke to him, ``Elisha, remain here on this side, please, because the \divine{Lord} is sending me to Jericho.''

But Elisha responded, ``As the \divine{Lord} lives, and while you're still alive, I'm not going to leave you!'' So they went to Jericho.

\v{5}The Guild of Prophets who lived in Jericho approached Elisha and asked him, ``You are aware, aren't you, that later today the \divine{Lord} is going to remove your master from being your mentor?''\fnote{\fbackref{2:5} Lit. \fbib{your head over you}}

``Of course I'm aware of it,'' he said. ``Calm down.''

\v{6}Elijah also spoke to him, ``Elisha, remain here on this side, please, because the \divine{Lord} is sending me to the Jordan River.''\fnote{\fbackref{2:6} The Heb. lacks \fbib{River}}

But Elisha responded, ``As the \divine{Lord} lives, and while you're still alive, I'm not going to leave you!'' So they went on their way,\fnote{\fbackref{2:6} The Heb. lacks \fbib{their way}} \v{7}accompanied by 50 men from the Guild of Prophets, who stood at a short distance from them while they were both standing by the Jordan. \v{8}Elijah took off his ornamented cloak, wrapped it up in a roll, struck the water, and all of a sudden the water divided into two parts! One side of the river stood still opposite the other until the two of them crossed over on dry ground.

\v{9}When they had crossed the Jordan River,\fnote{\fbackref{2:9} The Heb. lacks \fbib{the Jordan River}} Elijah invited Elisha, ``Ask me what you want me to do for you before I'm taken away from you.''

So Elisha asked, ``Please, may there be a double portion of\fnote{\fbackref{2:9} Or \fbib{double desire for}; Lit. \fbib{double mouth for}} your spirit upon me!''

\v{10}``That's a hard thing to ask for,'' Elijah answered, ``but if you see me while I'm being taken from you, it will happen for you. But if you don't see me, it won't happen.''

\v{11}As they continued on, talking as they went, suddenly chariots blazing with fire and pulled by fiery horses appeared, separated the two of them, and Elijah ascended in a wind storm to heaven! \v{12}As Elisha continued to watch, he cried out, ``My father! My father! The chariots of Israel and its cavalry!''\fnote{\fbackref{2:12} The Heb. word \fbib{cavalry} can refer to \fbib{horses}, to \fbib{horsemen}, or to both} Then he did not see Elijah anymore.

After this, Elisha\fnote{\fbackref{2:12} Lit. \fbib{he}} gripped his clothes that he was wearing, tore them apart into two pieces, \v{13}picked up Elijah's ornamented cloak that had fallen from him, and went back to stand on the bank of the Jordan River.\fnote{\fbackref{2:13} The Heb. lacks \fbib{River}} \v{14}Elisha\fnote{\fbackref{2:14} Lit. \fbib{He}} took hold of Elijah's ornamental cloak that had been left behind,\fnote{\fbackref{2:14} Lit. \fbib{had fallen from him}} struck the water, and cried out: ``Where is the \divine{Lord} God of Elijah?'' All of a sudden, after he had struck the water, the water divided into two parts! One side of the river stood opposite the other, and Elisha crossed over.
\passage{Elisha is Recognized as Elijah's Successor}

\v{15}As soon as the Guild of Prophets who lived adjacent to Jericho saw Elisha,\fnote{\fbackref{2:15} Lit. \fbib{him}} they began to announce, ``The spirit\fnote{\fbackref{2:15} Or \fbib{Spirit}} of Elijah is at rest on Elisha!'' So they came out to meet him and they greeted him by bowing low to the ground in front of him.

\v{16}Then they asked Elisha,\fnote{\fbackref{2:16} Lit. \fbib{him}} ``Look! We have 50 valiant men here with your servant! Please let them go out and search for your master Elijah.\fnote{\fbackref{2:16} The Heb. lacks \fbib{Elijah}} Perhaps the Spirit of the \divine{Lord} has taken him up on a mountain or into a valley.''

Elisha responded, ``Don't bother searching.''

\v{17}But they persisted until he was frustrated, so he said, ``Send them out!'' So they sent out the 50 men, and they looked around for three days but did not find Elijah.\fnote{\fbackref{2:17} Lit. \fbib{him}} \v{18}By the time they returned, Elisha\fnote{\fbackref{2:18} Lit. \fbib{he}} was living in Jericho. Then Elisha asked them, ``Didn't I tell you not to go?''
\passage{Elisha Cures the Waters of Jericho}

\v{19}The men who lived in the city addressed Elisha. ``Look now,'' they said, ``our\fnote{\fbackref{2:19} Lit. \fbib{The}} city's location is good, as you\fnote{\fbackref{2:19} Lit. \fbib{as my Lord}} have been observing, but the water springs\fnote{\fbackref{2:19} Lit. \fbib{the waters}} here are bad and the land isn't sustaining crops.''

\v{20}Elisha ordered them, ``Bring me a new bowl and put some salt in it.'' So they brought him what he had requested.\fnote{\fbackref{2:20} The Heb. lacks \fbib{what he had requested}}

\v{21}Elisha went out to the springs, threw the salt into them, and declared, ``This is what the \divine{Lord} says: `I have purified these waters. Neither death nor barrenness is to flow from them anymore.'\,'' \v{22}As a result, the water springs\fnote{\fbackref{2:22} Lit. \fbib{the waters}} remain pure to this day, just as\fnote{\fbackref{2:22} Lit. \fbib{as the word of}} Elisha had declared.
\passage{Elisha Rebukes Some Mockers}

\v{23}Later, Elisha\fnote{\fbackref{2:23} Lit. \fbib{he}} left there to go up to Bethel, and as he was traveling along the road, some insignificant\fnote{\fbackref{2:23} Or \fbib{small}; i.e. in significance, not stature or age; the individuals were adults} young men came from the city and started mocking him. They told him, ``Get on up,\fnote{\fbackref{2:23} The taunt may be an allusion to Elijah's experience described in v. 11.} baldy! Get on up, baldy!'' \v{24}He looked behind him, took note of the young men, and cursed them in the name of the \divine{Lord}. Suddenly two female bears emerged from the woods and mauled 42 of the young men. \v{25}After this, he left from there to go to Mt. Carmel, and from there he went back to Samaria.
\labelchapt{3}
\passage{Jehoram Becomes King}

\chapt{3}
\v{1}Ahab's son Jehoram ascended to the throne of Israel at Samaria during the eighteenth year of the reign of Judah's king Jehoshaphat. He reigned for twelve years, \v{2}practicing evil in the \divine{Lord}'s presence,\fnote{\fbackref{3:2} Lit. \fbib{sight}} only not to the extent that his mother and father had done\fnote{\fbackref{3:2} The Heb. lacks \fbib{had done}}---he forced abolition of the sacred pillar to Baal\fnote{\fbackref{3:2} I.e. the main Canaanite male deity, and so throughout the book} that his father had crafted. \v{3}Even so,\fnote{\fbackref{3:3} Lit. \fbib{Only}} he kept on committing the sins that Nebat's son Jeroboam had done, which ensnared Israel in sin---he never abandoned them.
\passage{Moab Rebels against Israel}

\v{4}Meanwhile, Moab's king Mesha was a sheep breeder. He used to pay 100,000 lambs and the wool from 100,000 rams to the king of Israel as tribute. \v{5}After Ahab died, the king of Moab rebelled against the king of Israel. \v{6}So king Jehoram left Samaria at that time\fnote{\fbackref{3:6} Lit. \fbib{in those days}} and mustered the entire army of\fnote{\fbackref{3:6} The Heb. lacks \fbib{army of}} Israel. \v{7}As he was going out, he sent this message\fnote{\fbackref{3:7} The Heb. lacks \fbib{this message}} to King Jehoshaphat of Judah: ``The king of Moab has rebelled against me. Will you go with me to fight Moab?''

``I'm coming,'' Jehoshaphat\fnote{\fbackref{3:7} Lit. \fbib{He}} replied. ``I'm like you! My army will act like your army and my cavalry like your cavalry,'' Then Jehoshaphat\fnote{\fbackref{3:7} Lit. \fbib{He}} added: \v{8}``What road do we take?''

Jehoram\fnote{\fbackref{3:8} Lit. \fbib{He}} answered, ``We'll go along the Edom desert road.''

\v{9}So the king of Israel, the king of Judah, and the king of Edom made a complete circuit on the road for seven days, but there was no water for the army or for the livestock that accompanied\fnote{\fbackref{3:9} Or \fbib{followed}} them.

\v{10}Then the king of Israel remarked, ``Oh no! The \divine{Lord} has summoned us three kings so he can hand us over to Moab, hasn't he?''
\passage{The Kings Seek Elisha's Counsel}

\v{11}Jehoshaphat asked, ``Isn't there a prophet who belongs to the \divine{Lord} and through whom we can ask the \divine{Lord} a question?''

One of the king of Israel's attendants replied, ``Shaphat's son Elisha lives here. He used to be Elijah's personal attendant.''\fnote{\fbackref{3:11} \fbib{to pour water on Elijah's hands}}

\v{12}Jehoshaphat answered, ``He receives messages from\fnote{\fbackref{3:12} Lit. \fbib{He has the word of}} the \divine{Lord}.'' So the king of Israel, Jehoshaphat, and the king of Edom went to visit Elisha.\fnote{\fbackref{3:12} Lit. \fbib{him}}

\v{13}Elisha asked the king of Israel, ``What do I have in common\fnote{\fbackref{3:13} The Heb. lacks \fbib{in common}} with you? Go visit your parents' prophets.''\fnote{\fbackref{3:13} Lit. \fbib{the prophets of your father and the prophets of your mother}}

The king of Israel replied, ``No! The \divine{Lord} has summoned these three kings so he can hand them over to Moab!''

\v{14}But Elisha responded, ``As the \divine{Lord} of the Heavenly Armies lives, in whose presence I stand, I would never pay attention to you or even look in your direction were it not for my continuous respect for the presence of King Jehoshaphat of Judah. \v{15}Now bring me a musician.''

As the musician played, the hand of the \divine{Lord} rested on Elisha, \v{16}so he said, ``This is what the \divine{Lord} says: `Fill this valley with trench after trench!' \v{17}This is what the \divine{Lord} says: `Though you won't see wind or storm, nevertheless that river\fnote{\fbackref{3:17} Or \fbib{seasonal stream}} will overflow with water so that you, your cattle, and your livestock may drink.' \v{18}And this is the easy part for the \divine{Lord}\fnote{\fbackref{3:18} Lit. \fbib{part in the \divine{Lord}'s eyes}}---he's also going to hand the Moabites over to you! \v{19}Then you are to attack every fortified city and every significant city. Cut down every significant tree, fill in all of the water springs, and ruin every prime piece of land with stones.''
\passage{War with Moab}

\v{20}The very next day, about the time of the morning offering, water suddenly appeared, coming from the direction of Edom, and the land overflowed with water! \v{21}Meanwhile, all the Moabites heard that the kings had come up to attack them, so everyone old enough to wear battle armor was mustered to stand guard at the border. \v{22}As the Moabites arose early that morning, the sun cast its rays on the water, and to the Moabites, the water across from them appeared to be red like blood. \v{23}So they concluded,\fnote{\fbackref{3:23} Lit. \fbib{said}} ``This must be blood! The kings must have had one mighty big fight and each man killed the other! So let's go get the battle spoil, Moab!''

\v{24}But when the Moabites arrived at the Israeli encampment, the Israelis got up and attacked them. The Moabites ran away from the Israelis,\fnote{\fbackref{3:24} Lit. \fbib{of them}} who followed them into the land as they continued their pursuit against Moab. \v{25}They destroyed their cities, and all of them threw stones onto every piece of farm\fnote{\fbackref{3:25} Or \fbib{good}} land, ruining the fields.\fnote{\fbackref{3:25} I.e. for future cultivation} Then they filled in all the water wells\fnote{\fbackref{3:25} Or \fbib{springs}} and chopped down all of the useful\fnote{\fbackref{3:25} Or \fbib{good}} trees. Stone walls remained surrounding Kir-hareseth only, until the archers surrounded and attacked that city. \v{26}When the king of Moab realized that the battle was going strongly against him, he took 700 expert swordsmen to attempt to break through to the king of Edom, but was unable to do so. \v{27}So he took his firstborn son, whom he intended to reign after him, and offered him up as a burnt offering on the wall. There subsequently came great anger against Israel, so they abandoned the attack and returned to their homeland.
\labelchapt{4}
\passage{The Miracle of the Oil Vessels}
\passageinfo{(1 Kings 17:14-16)}

\chapt{4}
\v{1}Now there happened to be a certain woman who had been the wife of a member of the Guild of Prophets. She cried out to Elisha, ``My husband who served you has died, and you know that your servant feared the \divine{Lord}. But a creditor has come to take away my children into indentured servitude!''

\v{2}Elisha responded, ``What shall I do for you? Tell me what you have in your house.''

She replied, ``Your servant has nothing in the entire house except for a flask of oil.''

\v{3}He told her, ``Go out to all of your neighbors in the surrounding streets and borrow lots of pots from them. Don't get just a few empty vessels, either. \v{4}Then go in and shut the door behind you, taking only your children, and pour oil\fnote{\fbackref{4:4} The Heb. lacks \fbib{oil}} into all of the pots. As each one is filled, set it aside.''

\v{5}So she left Elisha,\fnote{\fbackref{4:5} Lit. \fbib{him}} shut the door behind her and her children, and while they kept on bringing vessels to her, she kept on pouring oil.\fnote{\fbackref{4:5} The Heb. lacks \fbib{oil}} \v{6}When the last of\fnote{\fbackref{4:6} The Heb. lacks \fbib{last of}} the vessels had been filled, she told her son, ``Bring me another pot!''

But he replied, ``There isn't even one pot left.'' Then the oil stopped flowing. \v{7}After this, she went and told the man of God what had happened.\fnote{\fbackref{4:7} The Heb. lacks \fbib{what had happened}} So he said, ``Go sell the oil, pay your debt, and you and your children will be able to live on the proceeds.''
\passage{The Hospitality of a Woman from Shunem}

\v{8}Some time later, Elisha went to Shunem,\fnote{\fbackref{4:8} I.e. a town in the territory belonging to Issachar} where he met a prominent and wealthy\fnote{\fbackref{4:8} Lit. \fbib{strong}} woman who persuaded him to have a meal with her. As a result, whenever he was in the area, he stopped by to eat with her. \v{9}So she had a talk with her husband. ``Look here! I've learned that this is a holy and godly man\fnote{\fbackref{4:9} Or \fbib{a holy man of God}} who comes by here on a regular basis. \v{10}Now then, let's build a small upper room and put a bed in it for him there, along with a table, a chair, and a lamp stand. That way, when he comes to visit, he can rest\fnote{\fbackref{4:10} Or \fbib{can turn in}} there.''

\v{11}One day, Elisha\fnote{\fbackref{4:11} Lit. \fbib{he}} came by to visit and stopped in to rest in the upper chamber. \v{12}He told his attendant\fnote{\fbackref{4:12} Lit. \fbib{his young man}; and so throughout the chapter} Gehazi, ``Call this Shunammite.'' So when he had summoned her, she stood in front of him.

\v{13}Elisha\fnote{\fbackref{4:13} Lit. \fbib{He}} told him, ``Ask her, `Look how you've gone to all this trouble to care for us! What can I do for you? Do you wish to be mentioned to the king or to the head of the army?'\,''

She replied, ``I'm at home\fnote{\fbackref{4:13} So LXX; the Heb. lacks \fbib{at home}} living among my own people.''

\v{14}He responded, ``What, then, is to be done on her behalf?''

Gehazi answered, ``Well, she has no son and her husband is growing old.''

\v{15}``Call her,'' Elisha\fnote{\fbackref{4:15} Lit. \fbib{he}} ordered. After he called her, she came and stood in the doorway, \v{16}and he told her, ``About this time next year you will be embracing a son.''

``No, sir! Please, as a godly man,\fnote{\fbackref{4:16} Or \fbib{a man of God}} don't mislead your servant!'' \v{17}But the woman did conceive and did bear a son at that very same time the next year, just as Elisha had told her.
\passage{Elisha Raises the Shunammite's Son}
\passageinfo{(1 Kings 17:17-24)}

\v{18}After the child had grown up a bit, one day he went out to visit his father, who was with the harvesters. \v{19}He told his father, ``My head! My head!''

So his father ordered his servant, ``Carry him over to his mother!'' \v{20}So the servant carried him over to his mother, where he rested on her lap until mid-day,\fnote{\fbackref{4:20} Or \fbib{noon}} and then he died. \v{21}The woman went upstairs, laid him on the bed belonging to the man of God, and shut the door, leaving him behind as she left.

\v{22}Then she called to her husband and asked him, ``Please send me one of the servants, along with one of the donkeys, so I can ride quickly to see that godly man.\fnote{\fbackref{4:22} Or \fbib{that man of God}} I'll be right back.''

\v{23}He asked her, ``What's the point of visiting him today? It's not a New Moon, and it isn't the Sabbath!''

But she kept saying, ``Things will go well.''\fnote{\fbackref{4:23} Lit. \fbib{Peace}; i.e. a general statement of good will; and so through v. 26}

\v{24}So she saddled a donkey and told her servant, ``Forward, driver! Don't slow down on my account, unless I tell you!'' \v{25}So out she went and eventually she arrived at Mount Carmel to visit the man of God.

When the man of God noticed her from a distance, he told his attendant Gehazi, ``Look! There's the woman from Shunem! \v{26}Please run out quickly and greet her. Ask her, `Are things going well with you? Are things going well with your husband? Are things going well with your child?'\,''

She answered Gehazi,\fnote{\fbackref{4:26} The Heb. lacks \fbib{Gehazi}} ``Things are going well.''

\v{27}As she came near the man of God on the mountain, she grabbed his feet. When Gehazi intervened to push her away, the man of God said, ``Leave her alone! She is deeply troubled! The \divine{Lord} has concealed the thing from me, and hasn't informed me.''

\v{28}Then she asked, ``Did I ask my lord for a son? Didn't I beg you, `Don't mislead me?'\,''

\v{29}At this he told Gehazi, ``Get ready to run!\fnote{\fbackref{4:29} Lit. \fbib{Tie up your garments}; i.e. to secure one's robes with a belt in preparation for running} Take my staff in your hand, and get on the road. Don't greet anyone you meet. If anyone greets you, don't respond. Just go lay my staff on the youngster's face.''

\v{30}At this, the youngster's mother replied, ``As long as you and the \divine{Lord} live, I'm not leaving you!'' So he got up and followed her.

\v{31}Meanwhile, Gehazi went on ahead of them and placed the staff on the youngster's face, but when there was no sound or reaction, he returned, met Elisha,\fnote{\fbackref{4:31} Lit. \fbib{him}} and told him, ``The youngster has shown no sign of awakening.''

\v{32}When Elisha entered the house, there was the youngster, dead and laid out on Elisha's\fnote{\fbackref{4:32} Lit. \fbib{his}} bed! \v{33}So he entered, shut the door behind them both, and prayed to the \divine{Lord}. \v{34}Then he approached the child and lay down with his mouth near the child's, with his eyes near those of the child, and taking the child's hands in his. As Elisha\fnote{\fbackref{4:34} Lit. \fbib{he}} stretched himself on the child, the child's flesh began to grow warm. \v{35}Then he went downstairs, walked around back and forth inside the house once, went back up to his upper chamber,\fnote{\fbackref{4:35} The Heb. lacks \fbib{up to his upper chamber}} and stretched himself over the child again. The young man sneezed seven times and then opened his eyes.

\v{36}He called out to Gehazi, ``Go get the Shunammite woman!'' So he called her. When she came in to see Elisha,\fnote{\fbackref{4:36} Lit. \fbib{him}} he told her, ``Take back your son!'' \v{37}Then she approached him, fell at his feet, bowing low to the ground, took back her son, and went out.
\passage{Poisoned Stew is Purified}

\v{38}Elisha returned to Gilgal during a time of famine in the land. While the Guild of Prophets were having a meal\fnote{\fbackref{4:38} Lit. \fbib{were sitting}} with him, he instructed his attendant, ``Put a large pot on the fire and boil some stew for the Guild of Prophets.'' \v{39}Somebody went out into the fields to grab some herbs, found a wild vine, and gathered a lap full of wild gourds, which he came and sliced up into the stew pot, but nobody else knew.

\v{40}When they served the men, they began to eat the stew. But they cried out, ``That pot of stew is deadly, you man of God!'' So they couldn't eat the stew.

\v{41}But he replied, ``Bring me some flour.'' He tossed it into the pot and said, ``Serve the people so they can eat.'' Then there was nothing harmful in the pot.
\passage{Feeding of the Crowd}
\passageinfo{(Matthew 14:13-31; 15:32-39)}

\v{42}Later on, a man arrived from Baal-shalishah, bringing the man of God some bread as a first fruit offering. He had 20 loaves of barley and ripe ears of corn in his sack. So Elisha\fnote{\fbackref{4:42} Lit \fbib{he}} said, ``Give them to the people so they can eat.''

\v{43}Elisha's attendant asked, ``What? Will this serve 100 men?''

But he replied, ``Distribute it to the people so they can eat, because this is what the \divine{Lord} says: `They will eat and have a surplus!'\,'' \v{44}So he served them, and they ate and had some left over, just as the \divine{Lord} had indicated.
\labelchapt{5}
\passage{The Healing of Naaman}

\chapt{5}
\v{1}Naaman, the commander of the army of the king of Aram,\fnote{\fbackref{5:1} I.e., ancient Assyria, and so throughout the book} was a great man in the opinion\fnote{\fbackref{5:1} Lit. \fbib{eyes}} of his master. He was highly favored, because by him the \divine{Lord} had given victory to Aram. Though he was a mighty and valiant man, he was suffering from leprosy. \v{2}On one of their raids to the territory of Israel, Aram had taken captive a young girl when she was an infant,\fnote{\fbackref{5:2} Or \fbib{young little girl}; cf. v. 14; i.e., a young girl of small size} who had eventually become an attendant to\fnote{\fbackref{5:2} Lit. \fbib{girl, and she was in the presence of}} Naaman's wife. \v{3}She mentioned to her mistress, ``If only my master were to visit the prophet who is in Samaria! He would cure him of his leprosy.''

\v{4}Later, Naaman\fnote{\fbackref{5:4} Lit. \fbib{he}} went to inform his master and told him something like this: ``Thus and so spoke the young woman from the territory of Israel.''

\v{5}The king of Aram replied, ``Go now, and I'll send a letter to the king of Israel.'' So he left and took with him ten talents\fnote{\fbackref{5:5} I.e. about 750 pounds; a talent weighed about 75 pounds} of silver and 6,000 units\fnote{\fbackref{5:5} The unit of weight is unspecified.} of gold, along with ten sets\fnote{\fbackref{5:5} So MT; LXX reads \fbib{changes}} of clothing. \v{6}He also brought the letter to the king of Israel, which read as follows: ``{\ldots}and now as this letter finds its way to you, look! I've sent my servant Naaman to you so you may heal him of his leprosy.''

\v{7}When the king of Israel read the letter, he ripped his clothes and cried out, ``Am I God? Can I kill and give life? Is this man sending me a request\fnote{\fbackref{5:7} The Heb. lacks \fbib{a request}} to heal a man's leprosy? Let's think about this---he's looking for a reason to start a fight\fnote{\fbackref{5:7} The Heb. lacks \fbib{to start a fight}} with me!''

\v{8}When Elisha the man of God heard that the king of Israel had torn his clothes, he sent a message\fnote{\fbackref{5:8} The Heb. lacks \fbib{a message}} to the king and asked, ``Why did you tear your clothes? Please, let the man come visit me and he will learn that there is a prophet in Israel!''

\v{9}So Naaman arrived with his horses and chariots and stood in front of the door to Elisha's house. \v{10}Elisha sent a messenger out to him, who told him, ``Go bathe in the Jordan River\fnote{\fbackref{5:10} The Heb. lacks \fbib{River}} seven times. Your flesh will be restored for you. Now stay clean!''

\v{11}But Naaman flew into a rage and left, telling himself, ``Look! I thought `He's surely going to come out to me, stand still, call out in the name of the \divine{Lord} his God, wave his hand over the infection,\fnote{\fbackref{5:11} Lit. \fbib{place}} and cure the leprosy!' \v{12}Aren't the Abana and Pharpar rivers in Damascus better than all of the water in Israel? Couldn't I just bathe in them and become clean?'' So he turned away and left, filled with anger.

\v{13}But then his servants approached him and spoke with him. They said, ``My father, had the prophet only asked of you something great, you would have done it, wouldn't you? Yet he told you, `Bathe, and be clean{\ldots}!'\,'' \v{14}So he went down and plunged himself into the Jordan River\fnote{\fbackref{5:14} The Heb. lacks \fbib{River}} seven times, just as the man of God had said, and his flesh rejuvenated like the flesh of a newborn child. And he was clean.
\passage{Gehazi's Greed is Punished}

\v{15}Naaman\fnote{\fbackref{5:15} Lit. \fbib{He}} went back to the man of God, along with his entire entourage, and stood before him. ``Please look!'' he said. ``I know that there is no God in all the earth, except in Israel! So please, take a present from your servant.''

\v{16}But Elisha\fnote{\fbackref{5:16} Lit. \fbib{he}} replied, ``As the \divine{Lord} lives, before whom I stand, I will not receive anything from you.'' Though Naaman\fnote{\fbackref{5:16} Lit. \fbib{he}} urged him to take it, Elisha\fnote{\fbackref{5:16} Lit. \fbib{he}} declined.

\v{17}So Naaman asked, ``No? Then please let your servant load two mules with dirt from Israel,\fnote{\fbackref{5:17} The Heb. lacks \fbib{from Israel}} because your servant will no longer offer any burnt offering or sacrifice to any other god but the \divine{Lord}. \v{18}In this one area may the \divine{Lord} pardon your servant: Whenever my master enters the temple of Rimmon to worship there, he will lean on my hand while I bow down in the temple of Rimmon. So may the \divine{Lord} pardon your servant in this one area.''

\v{19}``Go in peace,'' he said. So Naaman\fnote{\fbackref{5:19} Lit. \fbib{he}} left.

After Naaman had gone only a short distance, \v{20}Gehazi, the attendant to Elisha, the man of God, told himself, ``Look how my master has spared this Aramean, Naaman! He declined to take from him what he brought. As the \divine{Lord} lives, I'm going to run after him and get something from him.'' \v{21}So Gehazi ran after Naaman.

When Naaman noticed someone running after him, he came down from his chariot, greeted him and asked, ``Is everything all right?''\fnote{\fbackref{5:21} Lit. \fbib{Peace}; i.e. a general statement of good will; and so through v. 26}

\v{22}Gehazi said, ``Everything's all right. My master sent me to tell you, `Just now two men from the Guild of Prophets have arrived from the hill country of Ephraim. Please give them each a talent\fnote{\fbackref{5:22} I.e. about 75 pounds; a talent weighed about 75 pounds} of silver bullion and two sets\fnote{\fbackref{5:22} So MT; LXX reads \fbib{changes}} of clothes.'\,''

\v{23}But Naaman said, ``Please accept my invitation to take two talents\fnote{\fbackref{5:23} The Heb. is dual; i.e. about 150 pounds; a talent weighed about 75 pounds} of silver.'' He urged him, binding two talents\fnote{\fbackref{5:23} The Heb. is dual; i.e. about 150 pounds; a talent weighed about 75 pounds} of silver in two bags, along with two sets of clothes. He placed them in the care of two of his young men, and they went on ahead of Gehazi.\fnote{\fbackref{5:23} Lit. \fbib{him}} \v{24}When he arrived at the stronghold, Gehazi\fnote{\fbackref{5:24} Lit. \fbib{he}} took the bags from their custody and hid them away in the house. Then he sent the men away and they left.

\v{25}Later he went to address\fnote{\fbackref{5:25} Or \fbib{to stand before}} his master. Elisha asked him, ``Where did you go, Gehazi?''

``Your servant went nowhere in particular,'' he said.

\v{26}But Elisha\fnote{\fbackref{5:26} Lit. \fbib{he}} responded, ``Didn't my heart break\fnote{\fbackref{5:26} Lit. \fbib{go}} as the man was turning from his chariot to greet you? Is now the time to receive money? To receive clothes? And olive groves, vineyards, sheep, oxen, servants, or female attendants? \v{27}Naaman's leprosy will plague you and your descendants forever!'' As he left Elisha's presence, he was infected with leprosy that looked like white snow.
\labelchapt{6}
\passage{The Miracle of the Ax Head}

\chapt{6}
\v{1}One day the Guild of Prophets told Elisha, ``Notice how the place where we are living is too small for us. \v{2}Let's go to the Jordan River,\fnote{\fbackref{6:2} The Heb. lacks \fbib{River}} fashion some rafters,\fnote{\fbackref{6:2} Lit. \fbib{take a beam}} and build a place for us so we can live there.''

So he said, ``Go!''

\v{3}Someone asked, ``Would you be willing to come with your servants?''

``I'm willing,'' he replied. \v{4}So he accompanied them, and when they came to the Jordan River,\fnote{\fbackref{6:4} The Heb. lacks \fbib{River}} they cut down some trees.

\v{5}It happened that as one of them was felling a beam, his axe head fell into the water. He cried out, ``Oh no! Master! The axe was on loan to me!''

\v{6}The man of God asked, ``Where did it fall?'' When he was shown the place, he cut off a branch, tossed it there, and made the iron axe head float. \v{7}Then Elisha said, ``Pick it up!'' So the young man reached out and picked it\fnote{\fbackref{6:7} The Heb. lacks \fbib{it}} up.
\passage{The Arameans Attack}

\v{8}Eventually the king of Aram went to war against Israel, taking counsel with his advisors and concluding, ``In such and such a place I'll build my encampment.''

\v{9}So the man of God sent a message\fnote{\fbackref{6:9} The Heb. lacks \fbib{a message}} to the king of Israel, warning him, ``Keep an eye on that area, because the Arameans are going to be there!'' \v{10}The king of Israel confirmed the matter\fnote{\fbackref{6:10} Lit. \fbib{Israel sent}} about which the man of God had warned him. Having been forewarned, he was able to protect himself there on more than one or two occasions.

\v{11}The king of Aram flew into a rage over this, so he called in his advisors and asked them, ``Will you please tell me which of us has joined the king of Israel?''

\v{12}``No, your majesty,'' one of his servants said. ``Elisha the prophet, who lives in Israel, tells the king of Israel what you talk about in your bedroom!''

\v{13}So the king\fnote{\fbackref{6:13} Lit. \fbib{So he}} ordered, ``Go and discover where he is, so I may send men\fnote{\fbackref{6:13} The Heb. lacks \fbib{men}} to take him into custody.''

Later somebody told him, ``Look! He's in Dothan!''

\v{14}So the king of Aram\fnote{\fbackref{6:14} Lit. \fbib{So he}} sent out horses, chariots, and an elite force, and they arrived during the night and surrounded the city. \v{15}Meanwhile, the attendant to the man of God got up early in the morning and went outside, and there were the elite forces, surrounding the city, accompanied by horses and chariots! So Elisha's attendant cried out to him, ``Oh no! Master! What will we do!?''

\v{16}Elisha\fnote{\fbackref{6:16} Lit. \fbib{He}} replied, ``Stop being afraid, because there are more with us than with them!'' \v{17}Then Elisha prayed, asking the \divine{Lord}, ``Please make him able to really see!'' And so when the \divine{Lord} enabled the young man to see, he looked, and there was the mountain, filled with horses and fiery chariots surrounding Elisha!

\v{18}When the army approached him, Elisha spoke to the \divine{Lord}, asking him, ``\divine{Lord}, I'm asking you please to afflict this group of people with blindness!'' So he afflicted them with blindness, just as Elisha had asked.

\v{19}Then Elisha told the army, ``This isn't the way, and this isn't the city! Follow me, and I'll bring you to the man you're seeking. Then he led them to Samaria. \v{20}When they arrived in Samaria, Elisha asked the \divine{Lord}, ``Enable them to see again.'' So the \divine{Lord} did so, and there they were---right in the middle of Samaria!

\v{21}When the king of Israel saw Elisha, he asked him, ``Shall I execute them, my father?''

\v{22}But he replied, ``No! You're not to kill them! Would you execute those whom you've taken captive at the point of a sword or with your bow? Give them food and water so they can eat and drink. Then send them back to their master!'' \v{23}So he prepared a large festival for them, and when they had finished eating and drinking, he sent them back to their master, and marauding gangs of Arameans never came into the territory of Israel again.
\passage{Ben-hadad Attacks Samaria}

\v{24}Some time later, King Ben-hadad from Aram mustered his army, invaded the land,\fnote{\fbackref{6:24} The Heb. lacks \fbib{the land}} and attacked Samaria \v{25}until there was a great famine throughout Samaria. The siege lasted until a donkey's head cost\fnote{\fbackref{6:25} The Heb. lacks \fbib{sold}} 80 silver coins\fnote{\fbackref{6:25} The exact weight or denomination of silver coin is unspecified.} and one quarter of a unit\fnote{\fbackref{6:25} Lit. \fbib{a kab}; a unit of dry weight in volume about 1.5 quarts} of dove's dung cost\fnote{\fbackref{6:25} The Heb. lacks \fbib{cost}} five silver coins.\fnote{\fbackref{6:25} The exact weight or denomination of silver coin is unspecified.}

\v{26}While the king of Israel was walking along the city\fnote{\fbackref{6:26} The Heb. lacks \fbib{city}} wall, a woman cried out to him. ``Help me, your majesty!''\fnote{\fbackref{6:26} Lit \fbib{me, my lord the king}} she said.

\v{27}He replied, ``No! Since the \divine{Lord} won't give you victory, how will I be able to deliver you? From the threshing floor? From the wine press?'' \v{28}Then the king asked her, ``What's bothering\fnote{\fbackref{6:28} The Heb. lacks \fbib{bothering}} you?''

She said, ``This woman told me, `Give up your son, and we'll eat him today, and we'll eat your son tomorrow.'\,'' \v{29}So we boiled my son and ate him. The next day, I told her, `Give me your son so we can eat him!' But she has hidden her son!''

\v{30}When the king heard what the woman said, he ripped his garments as he continued walking along the city\fnote{\fbackref{6:30} The Heb. lacks \fbib{city}} wall. As the people watched, all of a sudden they noticed he was wearing sackcloth underneath his clothes, inside next to his flesh! \v{31}He said, ``May God do to me---and more also!---if the head of Shaphat's son Elisha remains on his shoulders\fnote{\fbackref{6:31} Lit. \fbib{on him}} today!''

\v{32}Meanwhile, Elisha was sitting in his house, along with the elders, when the king\fnote{\fbackref{6:32} Lit. \fbib{when he}} sent a man to kill him,\fnote{\fbackref{6:32} The Heb. lacks \fbib{to kill him}} but before the messenger arrived, Elisha\fnote{\fbackref{6:32} Lit. \fbib{he}} told the elders, ``Are you watching how this descendant of murderers has ordered my head be cut off? Look, when the messenger arrives, shut the door and hold it to shut them out! Don't you hear the sound of his master's feet right behind him?''

\v{33}While he was still talking with them, the messenger arrived to see him and delivered the king's message to Elisha,\fnote{\fbackref{6:33} Lit. \fbib{and told him}} ``Look! This evil has come from the \divine{Lord}! Why should I wait for the \divine{Lord} anymore?''
\labelchapt{7}
\passage{Elisha Predicts Deliverance the Next Day}

\chapt{7}
\v{1}So Elisha responded, ``Listen to this message from the \divine{Lord}! `This is what the \divine{Lord} says: ``At about this time tomorrow, in Samaria's city gate, a seah\fnote{\fbackref{7:1} I.e. a dry measure of grain equal to about 2 gallons in volume.} of finely ground flour will sell for a shekel, and two seahs of barley for a shekel.''\,'\,''

\v{2}But the royal attendant on whom the king depended responded to the man of God: ``Look here! Even if the \divine{Lord} were to open a window in the sky, how could this happen?''

He replied, ``No, you look! You'll see it with your eyes, but you won't eat any of it!''
\passage{The Arameans Flee}

\v{3}Now there happened to be four lepers who were at that very moment at the entrance to the city gate. As they were talking with one another, they said, ``Why are we sitting here waiting to die? \v{4}If we tell ourselves, `Let's remain in the city,' we'll die there since there's famine in the city. But if we sit here, we'll die, too. So let's go over\fnote{\fbackref{7:4} Lit. \fbib{let's fall}} to the Arameans! If they spare our lives, we'll live, and if they kill us{\ldots}we're dying anyway!''\fnote{\fbackref{7:4} The Heb. lacks \fbib{anyway}}

\v{5}So they got up at dusk and went out to the Aramean encampment. But when they arrived at the outskirts of the Aramean encampment, there was no one there! \v{6}The \divine{Lord} had made the Aramean army hear the sounds of chariots, horses, and a large army, so they told one another, ``Look! The king of Israel has hired the kings of the Hittites and the Egyptians to come attack us!'' \v{7}So the Arameans\fnote{\fbackref{7:7} Lit. \fbib{So they}} got up and ran away in the gathering darkness. They left behind their tents, horses, and donkeys just as they were---and fled for their lives!

\v{8}When the lepers arrived at the outskirts of the encampment, they entered one tent and ate and drank. Then they carried off from there some silver, gold, and clothes, and went out and hid them. After this, they returned, entered another tent, raided it, and went and hid all of that,\fnote{\fbackref{7:8} The Heb. lacks \fbib{all of that}} too! \v{9}But then they told each other, ``We're not doing the right thing. This is a day of good news, but if we keep quiet until morning, we're sure to be punished! So let's leave and go tell the king's household!'' \v{10}So they left, called out to the city gatekeepers, and reported to them: ``We went out to the Aramean encampment, and there was nobody there! Not even the sound of men---only horses and donkeys tied up, and tents left just as they were!''

\v{11}The gatekeepers announced the report to the king's attendants, \v{12}so the king got up in the middle of the night and ordered his servants: ``Let me explain what the Arameans have done to us. They know that we're hungry, so they've left their encampment to conceal themselves in the surrounding fields. They're telling themselves, `When they come out of the city, we'll capture them alive and enter the city!'\,''

\v{13}One of his attendants suggested, ``Please, let's take five of the remaining horses, since those who remain here will end up like the rest of Israel, which has already died, and we'll send them out to look.'' \v{14}So they took two chariots and horses, and the king sent them out after the Aramean army with the orders, ``Go and look!''
\passage{The Prophecy is Fulfilled}

\v{15}They went out in the direction of the Jordan River,\fnote{\fbackref{7:15} The Heb. lacks \fbib{River}} and the entire roadway was strewn with clothes and equipment that the Arameans had abandoned in their haste to leave!\fnote{\fbackref{7:15} The Heb. lacks \fbib{to leave}} So the messengers returned and reported to the king. \v{16}At this, the people went out and plundered the camp of the Arameans. At that time, a seah\fnote{\fbackref{7:16} I.e. a dry measure of grain equal to about 2 gallons in volume.} of finely ground flour was sold for a shekel, and two seahs of barley for a shekel, in accordance with the \divine{Lord}'s message.

\v{17}Meanwhile, the king appointed the same royal attendant on whom he depended\fnote{\fbackref{7:17} Cf. v. 2} to take control of the city gate, but the people trampled him to death in the gate, just as the man of God had told the king when the king came down to him. \v{18}It happened just as the man of God had spoken to the king:

\begin{poetry}
\poeml ``At about this time tomorrow, in Samaria's city gate, a seah\fnote{\fbackref{7:18} I.e. a dry measure of grain equal to about 2 gallons in volume.} of finely ground flour will sell for a shekel, and two seahs of barley for a shekel.''
\end{poetry}

\v{19}But the royal attendant on whom the king depended responded to the man of God: ``Look here! Even if the \divine{Lord} were to make a window in the sky, how could this happen?''

He replied, ``No, you look! You'll see it with your eyes, but you won't eat any of it!''\fnote{\fbackref{7:19} Cf. v. 1-2}

\v{20}And so it happened to him, because the people trampled him in the city gate and he died.
\labelchapt{8}
\passage{The Shunammite's Land is Restored}

\chapt{8}
\v{1}Meanwhile, Elisha urged the woman whose son he had restored to life, ``You must get up and leave with your household to go live wherever you can, because the \divine{Lord} has called for a famine, and it's going to come over the land for seven years.'' \v{2}So the woman followed the instructions given to her by the man of God, and she went to the territory of the Philistines to live for seven years with her household. \v{3}At the end of the seven years, the woman returned from the territory of the Philistines and went to the king in order to file an appeal regarding her house and her grain field.

\v{4}The king was talking with Gehazi, the attendant of the man of God. He had asked Gehazi, ``Please tell me about all of the great things that Elisha has done.'' \v{5}Just as he was telling the king about Elisha's having restored the dead to life, the woman whose son had been restored arrived and appealed to the king for her house and her land!

Gehazi told the king, ``Your majesty, this is the woman! And here's her son, whom Elisha restored to life!''

\v{6}The king consulted with the woman, who related the story. So the king appointed a court official to represent her and ordered him: ``Restore to her everything that belonged to her, including all of the produce that her fields yielded from the day she left the land until now.''
\passage{The Murder of King Ben-hadad of Aram}

\v{7}Later on, Elisha traveled to Damascus. King Ben-hadad of Aram was ill, but someone informed him, ``The man of God has come here!''

\v{8}So the king told Hazael, ``Take a gift with you and go meet the man of God. Inquire of the \divine{Lord} through him and ask, `Will I recover from this sickness?'\,''

\v{9}So Hazael went out to meet with him and took a gift with him---40 camel loads filled with samples of everything good in Damascus. He approached the man of God\fnote{\fbackref{8:9} Lit. \fbib{approached him}} and said, ``Your son King Ben-hadad from Aram has sent me to you to ask you, `Will I recover from this sickness?'\,''

\v{10}But Elisha told him, ``Go tell him, `You will certainly recover,' but the \divine{Lord} has shown me that he will certainly die.'' \v{11}Then Elisha\fnote{\fbackref{8:11} Lit. \fbib{he}} looked steadily at Hazael\fnote{\fbackref{8:11} The Heb. lacks \fbib{at Hazael}} until Hazael grew ashamed, and then the man of God began to cry.

\v{12}``Why are you crying, sir?'' Hazael asked.

``Because I know the evil that you're about to bring on the Israelis,'' he replied. ``You'll burn down their fortified cities, execute their young men with swords, dash to pieces their little ones, and you'll tear open their pregnant women!''

\v{13}But Hazael responded, ``What? Who am I, your servant, that I should do such a horrible thing?''

But Elisha answered, ``The \divine{Lord} has shown me that you will be king over Aram.''

\v{14}So he left Elisha and returned to his master, who asked him, ``What did Elisha tell you?''

He replied, ``He told me that you would certainly get better.''

\v{15}But the very next day, Hazael\fnote{\fbackref{8:15} Lit. \fbib{he}} grabbed a thick covering, soaked it in water, and spread it over the king's\fnote{\fbackref{8:15} Lit. \fbib{over his}} face, and he suffocated.\fnote{\fbackref{8:15} Lit. \fbib{died}} Then Hazael succeeded Ben-hadad\fnote{\fbackref{8:15} Lit. \fbib{succeeded him}} as king.
\passage{Jehoram Comes to the Throne of Judah}

\v{16}Sometime during the fifth year of the reign of Ahab's son Joram, king of Israel (while Jehoshaphat was still ruling as king of Judah), Jehoshaphat's son Jehoram ascended to the throne of Judah. \v{17}He was 32 years old when he became king, and he reigned in Jerusalem for eight years. \v{18}He lived his life like the kings of Israel did, following the example of Ahab's household when he married Ahab's daughter and practiced what was evil in the \divine{Lord}'s presence.\fnote{\fbackref{8:18} Lit. \fbib{sight}} \v{19}But the \divine{Lord} remained unwilling to destroy Judah for the sake of his servant David, since he had promised to keep\fnote{\fbackref{8:19} Lit. \fbib{give}} David's lamp burning brightly through his descendants every day.

\v{20}During Jehoram's lifetime, Edom rebelled from Judah's hegemony and appointed a king to rule over themselves. \v{21}Then Joram crossed over to Zair, along with all of his chariots. At night he attacked the Edomites who had surrounded him and the commanders of his chariots, but the army\fnote{\fbackref{8:21} Lit. \fbib{people}} ran away to their tents. \v{22}Edom remains in rebellion against Judah to this day, and Libnah revolted at the same time. \v{23}The rest of the official\fnote{\fbackref{8:23} The Heb. lacks \fbib{official}} acts of Joram, along with everything else that he did, are recorded in the Book of the Chronicles of the Kings of Judah,\fnote{\fbackref{8:23} An ancient chronicle of Israel, apparently now lost; and so throughout the book} are they not?
\passage{Ahaziah Succeeds Jehoram}

\v{24}After Joram was laid to rest with his ancestors in the City of David, his son Ahaziah replaced him as king. \v{25}Joram's son Ahaziah began to reign as king of Judah during the twelfth year of the reign of Ahab's son Joram, king of Israel. \v{26}Ahaziah was 22 years old when he became king, and he reigned in Jerusalem for one year.

His mother was named Athaliah. She was the granddaughter of Omri, king of Israel. \v{27}Ahaziah lived his life following the example of Ahab's household, practicing what the \divine{Lord} considered to be evil, just like the household of Ahab, because he was a son-in-law to Ahab's household. \v{28}He joined Ahab's son Joram in an attack on King Hazael of Aram at Ramoth-gilead, and that's where the Arameans wounded Joram. \v{29}Then King Joram retreated to Jezreel to recover from the wounds that the Arameans had inflicted on him at Ramah during the battle against King Hazael of Aram. Jehoram's son Ahaziah, king of Judah, went to visit Ahab's son Joram in Jezreel because Joram was sick.\fnote{\fbackref{8:29} I.e. during Joram's recovery from his battle wounds}
\labelchapt{9}
\passage{Jehu Anointed King of Israel}

\chapt{9}
\v{1}Elisha called one of the members of the\fnote{\fbackref{9:1} The Heb. lacks \fbib{members of the}} Guild of Prophets and told him, ``Get ready to run,\fnote{\fbackref{9:1} Lit. \fbib{Tie up your garments}} take this flask of oil in your hand, and go to Ramoth-gilead. \v{2}As soon as you get there, go find Jehoshaphat's son Jehu, the grandson of Nimshi. When you do,\fnote{\fbackref{9:2} The Heb. lacks \fbib{When you do}} go in, tell him to get up and go apart with you away from his brothers. Lead him into a private chamber, \v{3}take the flask of oil, and pour it out on his head. Then tell him, `This is what the \divine{Lord} says: I'm anointing you king over Israel.' Then open the door and leave. Don't linger there!''

\v{4}So the young man, who was an attendant to the prophet, went to Ramoth-gilead. \v{5}When he arrived, the army commanders were seated, so he said, ``I have a message for you, captain!''

Jehu asked, ``For which one of us?''

``For you, captain!'' he answered.

\v{6}So Jehu\fnote{\fbackref{9:6} Lit. \fbib{he}} got up and went inside the house, and the young man\fnote{\fbackref{9:6} Lit. \fbib{and he}} told him, ``This is what the \divine{Lord}, the God of Israel says: `I have anointed you king over the people of the \divine{Lord}---that is, over Israel. \v{7}You are to attack the household of your master Ahab, so I may avenge the blood of my servants the prophets, as well as the blood of all of the servants of the \divine{Lord} that has been spilled\fnote{\fbackref{9:7} The Heb. lacks \fbib{that has been spilled}} at Jezebel's orders.\fnote{\fbackref{9:7} Lit. \fbib{hand}} \v{8}The entire household of Ahab will die, and I will cut off from Ahab every male person in Israel, whether imprisoned or surviving.\fnote{\fbackref{9:8} Or \fbib{whether in servitude or left behind}} \v{9}I will make the household of Ahab like the household of Nebat's son Jeroboam and the household of Ahijah's son Baasha. \v{10}Furthermore, the dogs will eat Jezebel in the territory of Jezreel. There will be no burial for her.'\,'' Then he opened the door and left.

\v{11}As Jehu was coming out to his master's attendants, one of them asked him, ``Is everything all right? Why did this maniac visit you?''

``You know the man and how he speculates,'' Jehu replied.

\v{12}``That's a lie!'' they said. ``Tell us what's going on!''

``He said `This and that' to me,'' he responded. ```This is what the \divine{Lord} says: ``I have anointed you king over Israel.''\,'\,''

\v{13}At this, each man quickly grabbed his own garment, placed it under him at the top of the stairs,\fnote{\fbackref{9:13} Or \fbib{him on the uncovered ascent}; i.e. on the roof of the building} sounded a trumpet, and announced, ``Jehu is king!''
\passage{Joram (Also Known as Jehoram) is Assassinated}
\passageinfo{(2 Chronicles 22:7-9)}

\v{14}Meanwhile, Jehoshaphat's son Jehu, the grandson of Nimshi, had been conspiring against Joram while Joram and all the army of\fnote{\fbackref{9:14} The Heb. lacks \fbib{the army of}} Israel had been defending Ramoth-gilead against King Hazael from Aram. \v{15}King Jehoram had returned to Jezreel to recover from wounds he had sustained from the Arameans when he had fought against King Hazael from Aram. So Jehu concluded, ``Since this is what you've decided,\fnote{\fbackref{9:15} Lit. \fbib{is your soul}} then let no one get away, leave the city, and go report to Jezreel!'' \v{16}Then Jehu rode by chariot to Jezreel, since Joram was recovering\fnote{\fbackref{9:16} Lit. \fbib{lying}} there. King Ahaziah from Judah had come to visit Joram.

\v{17}While the watchman was standing guard in the tower at Jezreel, he watched Jehu's entourage arrive. So he called out, ``I see a group arriving.''

Joram ordered, ``Take a horseman, send him out to meet them, and have him ask, `Have you come in peace?'\,''\fnote{\fbackref{9:17} Lit. \fbib{The peace}; i.e. a general inquiry of welfare}

\v{18}So a horseman went out, greeted Jehu and said, ``This is what the king said: `Have you come in peace?'\,''

But Jehu responded, ``What do you have to do with peace? Fall in behind me.''

The watchman reported, ``The messenger arrived there, but he hasn't returned.''

\v{19}Then Joram sent out a second horseman, who went out to them and said, ``This is what the king said: `Have you come in peace?'\,''

Jehu responded, ``What do you have to do with peace? Fall in behind me.''

\v{20}The watchman reported to Joram, ``He arrived there, but he hasn't returned. Also, he drives like Nimshi's son Jehu drives---irrationally!''

\v{21}Joram replied, ``Let's begin our attack!'' As soon as his chariot was prepared, both King Joram of Israel and King Ahaziah of Judah went out, each in his own chariot, to fight against Jehu. They met together in the property that had belonged to Naboth the Jezreelite.\fnote{\fbackref{9:21} Cf. 1 King 21:1-19}

\v{22}As soon as Joram noticed Jehu, he cried out, ``Peace, Jehu?''

Jehu\fnote{\fbackref{9:22} Lit. \fbib{He}} replied, ``What peace, given\fnote{\fbackref{9:22} The Heb. lacks \fbib{given}} your mother Jezebel's prostitution and all of\fnote{\fbackref{9:22} Lit. \fbib{and many}} her witchcraft?''\fnote{\fbackref{9:22} Or \fbib{sorcery}; i.e. wielding power through demonic spirits}

\v{23}Joram reined his horse\fnote{\fbackref{9:23} Lit. \fbib{hands}} around to flee and cried out to Ahaziah, ``Ahaziah! Treachery!'' \v{24}But Jehu drew his bow with all of his strength, shooting Joram between his shoulder blades.\fnote{\fbackref{9:24} Lit. \fbib{his arms}} The arrow pierced his heart, and he collapsed in his chariot.

\v{25}After this, Jehu called out to Bidkar, his third in command, ``Pick up Joram's body and throw it in the field, the property that belonged to Naboth the Jezreelite, because you and I remember how when we were riding together in pursuit of his father Ahab, that the \divine{Lord} pronounced this oracle\fnote{\fbackref{9:25} Lit. \fbib{burden}; a prophetic message of solemn import} against him:

\begin{poetry}
\poeml \v{26}`This is what the \divine{Lord} says, ``I have certainly observed the blood of Naboth and his sons, and I will repay you on this property,'' declares the \divine{Lord}.' \\
\poeml ``Therefore take the body and throw it in the field, just as the \divine{Lord} said.''
\end{poetry}
\passage{King Ahaziah is Also Killed}
\passageinfo{(2 Chronicles 22:7-9)}

\v{27}As soon as King Ahaziah of Judah observed this, he attempted to flee by the garden house road, but Jehu pursued him. At the ascent toward Gur which is near Ibleam, he ordered, ``Shoot him in the chariot, too!''

Ahaziah fled to Megiddo, where he died. \v{28}Ahaziah's servants transported the king's body\fnote{\fbackref{9:28} Lit. \fbib{transported him}} by chariot to Jerusalem and buried it in his own sepulcher near his ancestors in the City of David. \v{29}Ahaziah had begun to reign over Judah in the eleventh year of the reign of\fnote{\fbackref{9:29} The Heb. lacks \fbib{the reign of}} Ahab's son Joram.
\passage{Jezebel is Executed}

\v{30}As soon as Jehu arrived at Jezreel, Jezebel adorned her eyes, arranged her hair, and peered out a window. \v{31}When Jehu had entered through the gate, she asked, ``Was Zimri, who murdered his master,\fnote{\fbackref{9:31} Cf. 1King 16:9-10} received well?''

\v{32}Jehu\fnote{\fbackref{9:32} Lit. \fbib{He}} looked up toward the window and called out, ``Who is on my side? Who?'' When two or three eunuchs looked out at him, \v{33}he ordered, ``Throw her down!''

So they did, and her blood splashed against the wall and on the horses, while Jehu trampled her underfoot. \v{34}Later on, after he had come in to eat and drink, he ordered, ``Go and see to this cursed woman, and bury her, because she was a king's daughter.'' \v{35}But when they went out to bury her, they found nothing left of her except her skull, her feet, and the palms of her hands. \v{36}So they returned and reported to Jehu,\fnote{\fbackref{9:36} Lit. \fbib{him}} and he responded, ``This fulfills\fnote{\fbackref{9:36} The Heb. lacks \fbib{fulfills}} this message from the \divine{Lord} that he spoke through his servant Elijah the foreigner,\fnote{\fbackref{9:36} Lit. \fbib{Tishbite}; or \fbib{sojourner}} who said:

\begin{poetry}
\poeml `Dogs will eat Jezebel's flesh on the property of Jezreel, \v{37}and her corpse will lie like dung on the surface of the field on the property in Jezreel, but no one will say, ``This is Jezebel.''\,'\,''
\end{poetry}
\labelchapt{10}
\passage{Ahab's Dynasty is Ended}

\chapt{10}
\v{1}Meanwhile, Ahab had 70 sons who lived in Samaria. So Jehu wrote letters and sent them to Samaria---to the rulers of Jezreel, the elders, and the guardians of Ahab's children.\fnote{\fbackref{10:1} The Heb. lacks \fbib{children}} He told them, \v{2}``As soon as you receive this letter (since your master's children are with you, you have chariots and horses there with you, and you are protected by a walled city and weaponry), \v{3}select the best and most qualified of your master's sons, set him in place on his father's throne, and fight for your master's dynasty!''

\v{4}But they were too terrified, and so they told one another,\fnote{\fbackref{10:4} The Heb. lacks \fbib{to one another}} ``Look! Two previous kings couldn't stand up to Jehu, so how can we?'' \v{5}So the household overseer and the city supervisor, along with the elders and the children's guardians, sent word\fnote{\fbackref{10:5} The Heb. lacks \fbib{word}} to Jehu, telling him, ``We will serve you and do everything you ask. We won't set up a king, so do what you want to do.''

\v{6}But Jehu wrote them another letter: ``If you're loyal to me, and if you intend to obey my commands,\fnote{\fbackref{10:6} Lit. \fbib{voice}} then bring the heads of your master's sons and meet me in Jezreel about this time tomorrow.''

Now the king's sons, totaling 70 men, were living with the leading men of the city, who were their guardians. \v{7}When the letter from Jehu\fnote{\fbackref{10:7} The Heb. lacks \fbib{from Jehu}} arrived, the city leaders arrested the king's sons, slaughtered all 70 of them, put their heads in baskets, and sent them to Jehu\fnote{\fbackref{10:7} Lit. \fbib{him}} at Jezreel.

\v{8}When the messenger arrived to report to the king, he said, ``They have brought the heads of the king's sons.''

Jehu\fnote{\fbackref{10:8} Lit. \fbib{He}} replied, ``Put them in two piles at the entrance of the city gate until morning.'' \v{9}The next morning, Jehu went out, stood still, and announced to all the people: ``Are you righteous? I conspired against my master and killed him, but who slaughtered all of these? \v{10}Keep this in mind---not a single statement by the \divine{Lord} will fail to come about that he spoke concerning Ahab's dynasty, because the \divine{Lord} has accomplished what he predicted by his servant Elijah.''

\v{11}So Jehu executed all those who remained from Ahab's dynasty in Jezreel, including all of Ahab's men, his friends, and his priests, until there remained not even one survivor. \v{12}Then Jehu got up, left the city,\fnote{\fbackref{10:12} The Heb. lacks \fbib{the city}} and went to Samaria. When he arrived at the shearing house\fnote{\fbackref{10:12} Or \fbib{Beth-eked}; Lit. \fbib{the house of binding}; i.e. binding sheep to shear them} that was located on the way, \v{13}Jehu met up with the relatives of king Ahaziah of Judah. He asked them, ``Who are you?''

They answered, ``We're Ahaziah's relatives, and we've come down to greet the king's sons and the sons of the queen mother.''

\v{14}Jehu ordered, ``Take them alive!'' So Jehu's soldiers captured them and executed all 42 of them near the pit at the shearing house.\fnote{\fbackref{10:14} Or \fbib{Beth-eked}; Lit. \fbib{the house of binding}; i.e. binding sheep to shear them} He left none of them alive.

\v{15}After he left there, he encountered Rechab's son Jehonadab. After he greeted him, Jehu\fnote{\fbackref{10:15} Lit. \fbib{he}} asked him, ``Is your heart right, as my heart is with yours?''

``It is,'' Jehonadab answered.

``If it is,'' Jehu replied,\fnote{\fbackref{10:15} The Heb. lacks \fbib{Jehu replied}} ``Put out your hand.'' So Jehonadab stuck out his hand, and Jehu took him up to stand in his chariot. \v{16}He told him, ``Come with me and see my enthusiasm for the \divine{Lord}!'' So Jehu\fnote{\fbackref{10:16} Lit. \fbib{he}} had Jehonadab\fnote{\fbackref{10:16} Lit. \fbib{him}} ride in his chariot.

\v{17}When Jehu\fnote{\fbackref{10:17} Lit. \fbib{he}} arrived in Samaria, he executed everyone who remained of Ahab's household in Samaria, until he had utterly destroyed Ahab in accordance with the message from the \divine{Lord} that he spoke to Elijah.
\passage{Jehu Executes the Prophets of Baal}

\v{18}Then Jehu assembled all the people and announced to them, ``Ahab served Baal a little, but Jehu will serve him a lot! \v{19}Therefore summon all of Baal's prophets to me, including all his worshipers and all his priests. Don't leave even one out, because I've prepared a great sacrifice for Baal. Whoever doesn't show up doesn't live!'' But Jehu did this deceptively, intending to destroy Baal's worshippers. \v{20}Jehu ordered, ``Set aside a solemn assembly for Baal!''

And so they proclaimed it. \v{21}Jehu sent the proclamation\fnote{\fbackref{10:21} The Heb. lacks \fbib{the proclamation}} throughout Israel, and all the Baal worshipers came. There wasn't a single man left who failed to come. When they entered Baal's temple, it was filled from one end to the other. \v{22}Then Jehu\fnote{\fbackref{10:22} Lit. \fbib{he}} ordered the one in charge of the wardrobe, ``Bring out garments for all of the worshipers of Baal.'' So he brought out garments for them.

\v{23}Jehu and Rechab's son Jehonadab entered Baal's temple, and Jehu told the Baal worshipers, ``Look around and be sure that no servant of the \divine{Lord} is here among you, but only worshipers of Baal.'' \v{24}Then they went in to offer sacrifices and burnt offerings. Meanwhile, Jehu had stationed 80 men outside, ordering them, ``If any of these men whom I've brought into your control escape, the one who allows it will forfeit his life.''

\v{25}As soon as he had completed the burnt offering, Jehu ordered the guards and the officers, ``Go in and execute them. Don't let even one man escape.'' So they executed them with swords, and the guards and the officers threw the bodies out and proceeded into the inner room of Baal's temple, \v{26}from which they brought out the sacred pillars and burned them. \v{27}They also cut down the pillar to Baal, tore apart Baal's temple, and turned it into a latrine---and it remains that way today. \v{28}That's how Jehu eradicated Baal from Israel. \v{29}Even so, Jehu never abandoned the sins of Nebat's son Jeroboam, who caused Israel to sin, regarding the golden calves that were at Bethel and Dan.
\passage{Israel Begins to Reduce in Size}

\v{30}Nevertheless, the \divine{Lord} told Jehu, ``Because you have done well in carrying out what I saw as the right thing to do by completing everything I had in mind regarding Ahab's dynasty, your sons will sit on the throne of Israel to the fourth generation.'' \v{31}But Jehu did not remain careful to walk in the instruction\fnote{\fbackref{10:31} Or \fbib{Law}} of the \divine{Lord} God of Israel with all his heart. He never abandoned the sins of Jeroboam that had caused Israel to sin. \v{32}In those days, the \divine{Lord} began to reduce Israel in size: Hazael defeated them throughout the territory of Israel, \v{33}from the Jordan River\fnote{\fbackref{10:33} The Heb. lacks \fbib{River}} eastward, all the territory of Gilead, the descendants of Gad, the descendants of Reuben, and the descendants of Manasseh, from Aroer by the Valley of the Arnon, including Gilead and Bashan.
\passage{Jehoahaz Succeeds Jehu}

\v{34}Now as to the rest of Jehu's activities, including his valiant deeds, they are recorded in the Book of the Chronicles of the Kings of Israel, are they not? \v{35}Then Jehu died, as did\fnote{\fbackref{10:35} Lit. \fbib{Jehu slept with}} his ancestors, and they buried him in Samaria. His son Jehoahaz reigned in his place. \v{36}Jehu reigned over Israel in Samaria for 28 years.
\labelchapt{11}
\passage{Athaliah Reigns as Queen of Judah}
\passageinfo{(2 Chronicles 22:10-23:11)}

\chapt{11}
\v{1}As soon as Ahaziah's mother Athaliah learned that her son had died, she seized the throne\fnote{\fbackref{11:1} Lit. \fbib{she arose}} and executed the entire royal bloodline.\fnote{\fbackref{11:1} Lit. \fbib{seed}} \v{2}But King Joram's daughter Jehosheba, who was Ahaziah's sister, rescued\fnote{\fbackref{11:2} Lit. \fbib{took}} Ahaziah's son Joash from the group of the king's sons who were being executed and hid him and his nurse in her bedroom, concealing him from Athaliah so he was not put to death. \v{3}So Joash remained hidden with her in the \divine{Lord}'s Temple for six years while Athaliah reigned over the land.

\v{4}But during the seventh year of her reign,\fnote{\fbackref{11:4} The Heb. lacks \fbib{of her reign}} Jehoiada went out and called together the rulers of hundreds, the captains, and the guards, and assembled them together inside the \divine{Lord}'s Temple. He made a covenant with them, making them take an oath in the \divine{Lord}'s Temple, and then he revealed the king's son to them. \v{5}He ordered them:

\begin{poetry}
\poeml ``Here's what we'll do: A third of you will enter here on this coming\fnote{\fbackref{11:5} The Heb. lacks \fbib{this coming}} Sabbath dressed\fnote{\fbackref{11:5} The Heb. lacks \fbib{dressed}} as guardians of the watch for the king's palace, \v{6}with a third of you at the Sur gate, and a third at the gate behind the guards. Keep watch over the palace\fnote{\fbackref{11:6} Or \fbib{Temple}; Lit. \fbib{house}} and defend it. \v{7}Two\fnote{\fbackref{11:7} Lit. \fbib{Two hands}} of you who enter here on this coming\fnote{\fbackref{11:7} The Heb. lacks \fbib{this coming}} Sabbath are to stand watch at the \divine{Lord}'s Temple, \v{8}guarding the king and surrounding him with weapons in hand. Whoever comes within range is to be killed. Stay with the king wherever he goes, coming or going.''
\end{poetry}

\v{9}So the captains of hundreds did just as Jehoiada the priest ordered. Each one of them assembled his men who were to enter on the Sabbath, along with those who were to leave on the Sabbath, and approached Jehoiada the priest.

\v{10}The priest issued king David's personal spears and shields that had been stored in the \divine{Lord}'s Temple to the captains of hundreds. \v{11}So the guards stood assembled, every soldier with weapons in hand, surrounding the king from the right side corner of the Temple to the left side corner, including around the altar and the Temple.

\v{12}Then he brought out the king's son, put the royal crown on him, presented him with the Testimony,\fnote{\fbackref{11:12} I.e. the tablets that were stored in the ark; cf. Ex 25:16, 31:18} and installed him as king. They anointed him, applauded, and said, ``May the king live!''

\v{13}When Athaliah heard all of the commotion coming from those who were guarding the people, she approached the people who were in the \divine{Lord}'s Temple. \v{14}She looked around---and there was the king, standing near a column, as was the royal custom! He was accompanied by the commanding officers, along with trumpeters who stood beside the king. All the people of the land sounded trumpets in their excitement.

But Athaliah tore her clothes and bellowed, ``It's a plot! A conspiracy!''
\passage{Athaliah is Executed}
\passageinfo{(2 Chronicles 23:12-15)}

\v{15}Jehoiada the priest commanded the captains in charge of\fnote{\fbackref{11:15} Lit. \fbib{captains of hundred over}} the army, ``Take her out the back way\fnote{\fbackref{11:15} Lit. \fbib{out by the ranks}; i.e. using a utility entrance (cf. vs. 16)} and execute anybody who follows her,'' since the priest had also issued this order: ``Let's not put her to death in the \divine{Lord}'s Temple.'' \v{16}So they arrested Athaliah, took her out through the same entrance used by the horses for entering the king's palace, and executed her.
\passage{A Covenant is Made}
\passageinfo{(2 Chronicles 23:16-21)}

\v{17}Then Jehoiada entered into a covenant with the \divine{Lord}, the king, and the people, that they would live as the \divine{Lord}'s people, and also entered into a covenant with the king and the people. \v{18}Then all of the people of the land entered Baal's temple, tore it down, and broke his altars and his images to pieces, killing Mattan the priest of Baal right in front of the altars. Furthermore, Jehoiada\fnote{\fbackref{11:18} The Heb. lacks \fbib{Jehoiada}} the priest appointed officers to guard the \divine{Lord}'s Temple, \v{19}and brought the commanders of hundreds, the Carites, the guards, and all the people of the land, taking the king out of the \divine{Lord}'s Temple, marching through the guard's gate to the king's palace, where Joash\fnote{\fbackref{11:19} Lit. \fbib{he}} took his seat on the throne of the kings. \v{20}After this, everyone throughout the land rejoiced and the city was at peace, because they had executed Athaliah at the king's palace.
\labelchapt{12}
\passage{Jehoash (Joash) Reigns over Judah}

\v{21}\fnote{\fbackref{11:21} This vs. is 12:1 in MT} Jehoash began to reign as king when he was seven years old,\chapt{12}
\v{1}\fnote{\fbackref{12:1} This vs. is 12:2 in MT, and so throughout the chapter} ascending to the throne in the seventh year of the reign of\fnote{\fbackref{12:1} The Heb. lacks \fbib{the reign of}} Jehu and then reigning for 40 years in Jerusalem. His mother's name was Zibiah from Beer-sheba. \v{2}Jehoash did what the \divine{Lord} considered to be right during the entire time when Jehoiada the priest was instructing him, \v{3}except that the high places were not demolished, so the people continued to sacrifice and burn incense on the high places.
\passage{Jehoash Institutes Temple Repairs}

\v{4}Jehoash spoke to the priests about all of the proceeds\fnote{\fbackref{12:4} Lit. \fbib{silver}; i.e., money from conversion of gifts into cash} of the consecrated gifts that were being brought into the \divine{Lord}'s Temple, cash from every man who was traveling through the area,\fnote{\fbackref{12:4} The Heb. lacks \fbib{the area}} cash obtained by personal assessment,\fnote{\fbackref{12:4} Lit. \fbib{cash from souls to their appointment}} and all the cash that came through voluntary gifts\fnote{\fbackref{12:4} Lit. \fbib{through the heart of a man}} into the \divine{Lord}'s Temple:

\begin{poetry}
\poeml \v{5}``Let the priests get support for themselves from their own donors, and let them repair the Temple wherever a leak in need of repair is discovered.''
\end{poetry}

\v{6}But 23 years into the reign of king Jehoash, the priests still had not repaired the leaks in the Temple. \v{7}So king Jehoash called for Jehoiada the priest, along with other\fnote{\fbackref{12:7} Lit. \fbib{the}} priests, and asked them, ``Why haven't you fixed the leaks in the Temple? Stop receiving donations from your acquaintances for repairing the leaks in the Temple.''

\v{8}So the priests agreed to receive no more cash from the people, but they didn't repair the leaks in the Temple, either. \v{9}So Jehoiada the priest grabbed a chest, bored an opening in its lid, and placed it next to the altar, on the right side as one enters the \divine{Lord}'s Temple. The priests who tended the entryway put all the money that was brought into the \divine{Lord}'s Temple into the chest.\fnote{\fbackref{12:9} Lit. \fbib{into it}} \v{10}As a result, whenever they noticed that there was a lot of money in the chest, the king's secretary and the high priest went forward, put the money in bags, counted the money that had been given over to the \divine{Lord}'s Temple, \v{11}and disbursed the cash directly into the hands of those who did the work and who were in charge of the oversight of the \divine{Lord}'s Temple. They paid it to the carpenters and builders who worked on the \divine{Lord}'s Temple, \v{12}to masons and stonecutters, and for procurement of timber and quarried stone for making repairs to the \divine{Lord}'s Temple, and for all outlays needed for repairs of the Temple.

\v{13}But no provision was included for the \divine{Lord}'s Temple from the money that was brought into the \divine{Lord}'s Temple for silver basins, snuffers, bowls, trumpets, or any vessels made of gold or silver, \v{14}because that money had been allocated to the workmen who were repairing the \divine{Lord}'s Temple. \v{15}Furthermore, they required no accounting from the men into whose hand they had paid the money to do the work, because the workers acted in good faith. \v{16}The money from the guilt offerings and\fnote{\fbackref{12:16} Lit. \fbib{and the money}} from the sin offerings was not brought into the \divine{Lord}'s Temple, because it was allocated to the priests.
\passage{Hazael Attacks Israel}

\v{17}Later, King Hazael of Aram invaded and attacked Gath, captured it, and then set out to approach Jerusalem. \v{18}So King Jehoash of Judah took all of the sacred things that his ancestors Jehoshaphat, Jehoram, and Ahaziah, kings of Judah, had dedicated, along with his own dedicated things, and all the gold that could be located within the treasure vaults of the \divine{Lord}'s Temple and in the king's palace, and paid off King Hazael of Aram. Then Hazael\fnote{\fbackref{12:18} Lit. \fbib{he}} left Jerusalem.
\passage{Amaziah Succeeds Jehoash (Joash)}
\passageinfo{(2 Chronicles 24:23-27)}

\v{19}Now the rest of the Joash's activities---everything he did---are written in the Book of the Chronicles of the Kings of Judah, are they not? \v{20}His servants rose up in rebellion, formed a conspiracy, and assassinated Joash in the palace at the terrace ramparts\fnote{\fbackref{12:20} Lit. \fbib{the Millo}, fortified areas of ancient Jerusalem with terraces and retaining walls} while he was on his way down to Silla. \v{21}Shimeath's son Jozacar and Shomer's son Jehozabad, his servants, attacked him and he died. They buried him alongside his ancestors in the City of David, and his son Amaziah became king to replace him.
\labelchapt{13}
\passage{Jehoahaz Becomes King of Israel}

\chapt{13}
\v{1}During the twenty-third year of the reign of\fnote{\fbackref{13:1} The Heb. lacks \fbib{the reign of}} Ahaziah's son Joash, king of Judah, Jehu's son Jehoahaz began his seventeen year reign in Samaria over Israel.\fnote{\fbackref{13:1} I.e. over the northern kingdom} \v{2}He did what the \divine{Lord} considered to be evil, after the pattern of Nebat's son Jeroboam. By doing so, he caused Israel to sin, and he never changed course from it. \v{3}As a result, the \divine{Lord}'s wrath flared up against Israel, so he handed them over to domination by king Hazael of Aram and later into constant domination\fnote{\fbackref{13:3} Lit. \fbib{into domination all their days}} by Hazael's son Ben-hadad. \v{4}But Jehoahaz sought the \divine{Lord},\fnote{\fbackref{13:4} Lit. \fbib{the \divine{Lord}'s face}} and the \divine{Lord} paid attention to him, because the \divine{Lord}\fnote{\fbackref{13:4} Lit. \fbib{because he}} had been watching the oppression that Israel was enduring from the king of Aram.\fnote{\fbackref{13:4} The Heb. lacks \fbib{that Israel was enduring from the king of Aram.}}
\passage{God Delivers Israel}

\v{5}The \divine{Lord} provided Israel with a deliverer, so they escaped the Aramean oppression while the descendants of Israel lived in tents as they had formerly. \v{6}Nevertheless, they did not change course away from the sins of Jeroboam's household, by which he caused Israel to sin, but continued on that same course, with Asherah poles\fnote{\fbackref{13:6} I.e. cultic pillars used in pagan worship, and so throughout the book} remaining in place in Samaria. \v{7}For the Aramean king\fnote{\fbackref{13:7} Lit. \fbib{For he}} had left only 50 cavalry, ten chariots, and 10,000 soldiers out of the army belonging to Jehoahaz, because the king of Aram had destroyed the others,\fnote{\fbackref{13:7} Lit. \fbib{destroyed them}} making them like chaff left over after threshing.

\v{8}Now the rest of the activities of Jehoahaz, including everything he did and his grandeur, are recorded in the Book of the Chronicles of the Kings of Israel, are they not? \v{9}So Jehoahaz died, as did\fnote{\fbackref{13:9} Lit. \fbib{Jehoahaz slept with}} his ancestors, and he was buried in Samaria while his son Joash replaced him as king.
\passage{Jehoash Reigns in Samaria}

\v{10}During the thirty-seventh year of the reign of\fnote{\fbackref{13:10} The Heb. lacks \fbib{the reign of}} king Joash\fnote{\fbackref{3:10} \fbib{Joash} and \fbib{Jehoash} are alternate spellings of the same name} of Judah, Jehoahaz's son Jehoash began a sixteen year reign as king over Israel in Samaria. \v{11}He practiced what the \divine{Lord} considered to be evil, not changing course from all of the sins practiced by Nebat's son Jeroboam by which he caused Israel to sin. Instead, he continued on that same course. \v{12}The rest of Joash's activities, including everything he did and the vehemence with which he fought against king Amaziah of Judah are recorded in the Book of the Chronicles of the Kings of Israel, are they not? \v{13}So Joash died, as did\fnote{\fbackref{13:13} Lit. \fbib{Joash slept with}} his ancestors, and Jeroboam assumed his throne after Joash was buried in Samaria with the kings of Israel.
\passage{Elisha Predicts Partial Victory for Joash}

\v{14}When Elisha fell ill with the sickness from which he was about to die, king Joash of Israel came down to see\fnote{\fbackref{13:14} The Heb. lacks \fbib{see}} him, wept in his presence, and told him, ``My father, Israel's chariots and horsemen!''

\v{15}Elisha told him, ``Pick up a bow and some arrows.'' So he picked up a bow and some arrows.

\v{16}Then Elisha told Israel's king, ``Draw the bow!'' As he did so, Elisha laid his hands on top of the king's hands \v{17}and ordered him, ``Open a window that faces east.'' So he did so.

Elisha ordered him, ``Shoot!'' So he shot.

Then Elisha said, ``This is the \divine{Lord}'s arrow of victory---the victory arrow against Aram, because you will defeat the Arameans at Aphek until you will have utterly finished them off.''

\v{18}After this Elisha said, ``Pick up the arrows.'' So the king picked them up.

Then Elisha told the king of Israel, ``Strike the ground!'' So he struck it three times and then stood still.

\v{19}At this, the man of God became angry at him and told him, ``You should have struck five or six times! Then you would have attacked Aram until you would have destroyed it! But as it is now, you'll defeat Aram only three times!''
\passage{The Death of Elisha}

\v{20}Later, Elisha died and was buried. Now at that time, various Moabite marauders had been invading the land each spring. \v{21}One day while some Israelis\fnote{\fbackref{13:21} Lit. \fbib{As they}} were burying a man, they saw some marauders, so they threw the man into Elisha's grave. But when the man fell against Elisha's remains,\fnote{\fbackref{13:21} Lit. \fbib{bones}} he revived and rose to his feet.
\passage{Elisha's Prophecy of Partial Victory is Fulfilled}

\v{22}Meanwhile, king Hazael of Aram had been oppressing Israel throughout the reign of Jehoahaz, \v{23}but the \divine{Lord} showed grace to them, displayed his compassion toward them, and turned to them due to his covenant with Abraham, Isaac, and Jacob. He would not destroy them or evict them from his presence up until that time. \v{24}After king Hazael of Aram died, his son Ben-hadad replaced him as king. \v{25}At that time, Jehoahaz's son Jehoash recaptured from Hazael's son Ben-hadad the cities that Hazael\fnote{\fbackref{13:25} Lit. \fbib{he}} had captured through warfare from the control of Jehoahaz, Jehoash's father. Joash\fnote{\fbackref{13:25} \fbib{Joash} and \fbib{Jehoash} are alternate spellings of the same name} defeated and recovered cities of Israel from Ben-hadad\fnote{\fbackref{13:25} Lit. \fbib{him}} three times.
\labelchapt{14}
\passage{Amaziah Becomes King of Judah}

\chapt{14}
\v{1}Amaziah, son of Judah's king Joash, became king during the second year of the reign of\fnote{\fbackref{14:1} The Heb. lacks \fbib{the reign of}} Joash, son of king Joahaz of Israel, \v{2}at the age of 25. He reigned 29 years in Jerusalem. His mother's name was Jehoaddin; she was\fnote{\fbackref{14:2} The Heb. lacks \fbib{she was}} from Jerusalem.

\v{3}He practiced what the \divine{Lord} considered to be right, but not like his ancestor David did. He acted as his father Joash had done, \v{4}except that the high places were not abolished. The people continued to offer sacrifices and to burn incense on the high places. \v{5}Later on, as soon as he was in firm control of his kingdom, he executed the servants who had murdered his father the king, \v{6}but he did not execute the children of the murderers, in keeping with what is written in the Book of the Law of Moses, as the \divine{Lord} had commanded: ``Fathers must not be put to death because of their children's sin; nor are children to die because of their fathers' sin, for each person is to be put to death for his own sin.''\fnote{\fbackref{14:6} Cf. Deut 24:16}
\passage{The Edomites are Defeated}
\passageinfo{(2 Chronicles 25:5-16)}

\v{7}Joash executed 10,000 Edomites in the Salt Valley and captured Sela in battle, renaming it Joktheel, which remains its name to this day. \v{8}Later, Amaziah sent couriers to Jehoahaz's son Jehoash, grandson of king Jehu of Israel, challenging him, ``Come on! Let's fight face to face!''

\v{9}But king Jehoash of Israel sent this message to king Amaziah of Judah: ``The thorn bush in Lebanon sent this message to the cedar\fnote{\fbackref{14:9} I.e. a genus of coniferous evergreen in the family \fbib{Pinaceae}; and so throughout the book} of Lebanon: `Give your daughter to my son in marriage.' But just then a wild beast from Lebanon wandered by and trampled down the thorn bush. \v{10}You just defeated Edom and you're\fnote{\fbackref{14:10} Lit. \fbib{and your heart is}} arrogant. Bask in your victory and stay home. Why incite trouble so that you---yes, you!---fall, along with Judah with you?''

\v{11}But Amaziah refused to listen. So Israel's king Jehoash and Judah's king Amaziah faced each other at Beth-shemesh, which is part of Judah. \v{12}Judah was defeated by Israel, and everybody fled to their own tents. \v{13}Then king Jehoash of Israel captured Judah's king Amaziah, the son of Jehoash and grandson of Ahaziah, at Beth-shemesh. He went to Jerusalem and demolished 400 cubits\fnote{\fbackref{14:13} I.e. about 600 feet a cubit was about eighteen inches} of the wall of Jerusalem from the Ephraim Gate to the Corner Gate. \v{14}He confiscated all the gold and silver, all the instruments he could find in the \divine{Lord}'s Temple and in the palace treasuries. He also captured some hostages and then returned to Samaria.
\passage{Jeroboam Succeeds Israel's King Jehoash}

\v{15}The rest of Jehoash's activities that he undertook, including his valor in fighting king Amaziah of Judah, are recorded in the Book of the Chronicles of the Kings of Israel, are they not? \v{16}Jehoash died, as had\fnote{\fbackref{14:16} Lit. \fbib{Jehoash slept with}} his ancestors, and he was buried in Samaria alongside the kings of Israel. His son Jeroboam reigned in his place.
\passage{The Death of Judah's King Amaziah}
\passageinfo{(2 Chronicles 25:25-28)}

\v{17}Joash's son, king Amaziah of Judah, lived for fifteen years after Jehoahaz' son, king Jehoash of Israel, died. \v{18}The rest of Amaziah's activities are recorded in the Book of the Chronicles of the Kings of Judah, are they not? \v{19}A conspiracy arose against him in Jerusalem, and he ran off to Lachish, but he was pursued to Lachish and killed there. \v{20}His body was brought back on horses and he was buried at Jerusalem alongside his ancestors in the City of David.
\passage{Azariah's Reign over Judah}

\v{21}All the people of Judah took Azariah, who was sixteen years old, and installed him as king to take the place of his father Amaziah. \v{22}He rebuilt Elath and restored it to Judah. Later on the king died, as did\fnote{\fbackref{14:22} Lit. \fbib{king slept with}} his ancestors.
\passage{Jeroboam's Reign over Israel}

\v{23}In the fifteenth year of the reign of\fnote{\fbackref{14:23} The Heb. lacks \fbib{the reign of}} Amaziah son of Joash, king of Judah, Jeroboam son of Joash, king of Israel, began a 41 year reign in Samaria. \v{24}He did what the \divine{Lord} considered to be evil by not abandoning all the sins of Nebat's son Jeroboam, who made Israel sin. \v{25}He rebuilt Israel's coastline from the entrance of Hamath as far as the Sea of the Arabah,\fnote{\fbackref{14:25} I.e. the Dead Sea; cf. Deut 3:17} in accordance with the message from the \divine{Lord} God of Israel that he spoke through his servant Jonah the prophet, Amittai's son, who was from Gath-hepher. \v{26}For the \divine{Lord} observed Israel's bitter misery, and there was no one left, neither slave nor free, and there was no deliverer for Israel. \v{27}The \divine{Lord} had never said that he would erase the name of Israel from under heaven. Instead, he delivered them by Joash's son Jeroboam. \v{28}The rest of Jeroboam's actions---everything he did, including his powerful fighting and how on behalf of Israel he restored Damascus and Hamath to Judah---are recorded in the Book of the Chronicles of the Kings of Israel, are they not?
\passage{Zechariah's Reign over Israel}

\v{29}Jeroboam died, as had\fnote{\fbackref{14:29} Lit. \fbib{Jeroboam slept with}} his ancestors the kings of Israel, and his son Zechariah became king in his place.
\labelchapt{15}
\passage{Azariah Becomes King of Judah}

\chapt{15}
\v{1}Amaziah's son Azariah began reigning during the twenty-seventh year of the reign of\fnote{\fbackref{15:1} The Heb. lacks \fbib{the reign of}} Jeroboam, king of Israel. \v{2}He was sixteen years old when he began to reign, and he reigned 52 years in Jerusalem. His mother's name was Jecoliah; she was\fnote{\fbackref{15:2} The Heb. lacks \fbib{she was}} from Jerusalem. \v{3}He did what the \divine{Lord} considered to be right, just as his father Amaziah had done in everything, \v{4}except that the high places were never removed, and the people kept on sacrificing and burning incense on the high places.

\v{5}The \divine{Lord} struck the king so that he was afflicted with leprosy until the day he died. He lived in a separate house while his son Jotham managed the household and ruled\fnote{\fbackref{15:5} Lit. \fbib{judged}} the people who lived in the land. \v{6}Now the rest of Azariah's activities, including everything he did, are recorded in the Book of the Chronicles of the Kings of Judah, are they not? \v{7}Later, Azariah died, as had\fnote{\fbackref{15:7} Lit. \fbib{Azariah slept with}} his ancestors, and they buried him with his ancestors in the City of David. His son Jotham then reigned in his place.
\passage{Zachariah's Reign over Israel}

\v{8}During the thirty-eighth year of the reign of\fnote{\fbackref{15:8} The Heb. lacks \fbib{the reign of}} Azariah, king of Judah, Jeroboam's son Zachariah began a six-month reign in Samaria. \v{9}He did what the \divine{Lord} considered to be evil, just as his ancestors had done. He never abandoned the sins of Nebat's son Jeroboam, who caused Israel to sin. \v{10}So Jabesh's son Shallum conspired against him and attacked him in full view of the people, killed him, and reigned in his place. \v{11}The rest of Zachariah's activities are recorded in the Book of the Chronicles of the Kings of Israel.
\passage{Shallum's Reign over Israel}

\v{12}This is what the \divine{Lord} told Jehu: ``Your children will sit on Israel's throne for the next four generations.''\fnote{\fbackref{15:12} The Heb. lacks \fbib{generations}} And that is what happened:\fnote{\fbackref{15:12} Lit. \fbib{And so it was}} \v{13}Jabesh's son Shallum began his reign in the thirty-ninth year of the reign of Uzziah,\fnote{\fbackref{15:13} \fbib{Uzziah} is an alternate spelling of \fbib{Azariah} (cf. 2King 15:1)} king of Judah. He reigned a full month\fnote{\fbackref{15:13} Lit. \fbib{a lunar of days}; i.e. one complete lunar month (through all four phases of the moon)} in Samaria, \v{14}then Gadi's son Menahem approached Samaria from Tirzah and attacked Jabesh's son Shallum, executed him, and reigned in his place. \v{15}The rest of Shallum's activities, including the conspiracy that he carried out, are recorded in the Book of the Chronicles of the Kings of Israel, are they not?
\passage{Menahem's Reign over Israel}

\v{16}At another time, Menahem attacked Tiphsah and all of its inhabitants, including its coastlands from Tirzah, because they would not open the city gate for him. After defeating them, he ripped open all of their pregnant women. \v{17}In the thirty-ninth year of the reign of\fnote{\fbackref{15:17} The Heb. lacks \fbib{the reign of}} Azariah, king of Judah, Gadi's son Menahem began a ten-year reign over Israel from Samaria. \v{18}He did what the \divine{Lord} considered to be evil by never abandoning the sins of Nebat's son Jeroboam, who caused Israel to sin, as long as he lived.

\v{19}Later on, King Pul of Aram attacked the land, and Menahem paid Pul 1,000 silver talents\fnote{\fbackref{15:19} I.e. about 75,000 pounds; a talent weighed about 75 pounds} so Pul\fnote{\fbackref{15:19} Lit. \fbib{he}} would join forces with Menahem\fnote{\fbackref{15:19} Lit. \fbib{him}} to secure his hold on the kingdom. \v{20}Menahem exacted the money from all of Israel's powerful and wealthy men, 50 shekels\fnote{\fbackref{15:20} I.e. about 20 ounces; a shekel weighed about 0.4 ounce} from each, to pay the king of Aram. As a result, the king of Aram retreated and did not remain there in the land. \v{21}The rest of Menahem's activities, including everything that he did, are recorded in the Book of the Chronicles of the Kings of Israel, are they not? \v{22}Then Menahem died, as did\fnote{\fbackref{15:22} Lit. \fbib{Menahem slept with}} his ancestors, and his son Pekahiah reigned in his place.
\passage{Pekahiah's Reign over Israel}

\v{23}Menahem's son Pekahiah became king over Israel for two years during the fiftieth year of the reign of\fnote{\fbackref{15:23} The Heb. lacks \fbib{the reign of}} King Azariah of Judah. \v{24}He did what the \divine{Lord} considered to be evil. Just as Nebat's son Jeroboam had led Israel into sin, so also Pekahiah did not stop doing the same thing. \v{25}Then Remaliah's son Pekah, Pekahiah's\fnote{\fbackref{15:25} Lit. \fbib{his}} officer, conspired against him with Argob and Arieh. Accompanied by 50 Gileadite men, Pekah attacked Pekahiah inside the palace of the king's compound\fnote{\fbackref{15:25} Lit. \fbib{house}} in Samaria, executed him, and reigned as king in his place. \v{26}The rest of Pekahiah's activities, including everything he did, are written in the Book of the Chronicles of the Kings of Israel.
\passage{Pekah's Reign over Israel}

\v{27}Remaliah's son Pekah began a 20-year reign as Israel's king during the fifty-second year of King Azariah of Judah. \v{28}He did what the \divine{Lord} considered to be evil by never abandoning the sins of Nebat's son Jeroboam, by which he caused Israel to sin. \v{29}During the lifetime of King Pekah of Israel, King Tiglath-pileser of Assyria attacked. He captured the cities of Ijon, Abel Beth Maacah, Janoah, Kedesh, and Hazor. He also captured Gilead, Galilee, and the entire territory of Naphtali, and carried its people off to Assyria. \v{30}So during the twentieth year of the reign of\fnote{\fbackref{15:30} The Heb. lacks \fbib{the reign of}} Uzziah's son Jotham, Elah's son Hoshea conspired against Remaliah's son Pekah, attacked him, executed him, and became king in his place. \v{31}The rest of Pekah's activities, including everything that he accomplished, are written in the Book of the Chronicles of the Kings of Israel.
\passage{Jotham's Reign over Judah}

\v{32}Uzziah's son Jotham became king over Judah during the second year of the reign of\fnote{\fbackref{15:32} The Heb. lacks \fbib{the reign of}} Remaliah's son Pekah, king of Israel. \v{33}He was 25 years old when he became king. He reigned sixteen years in Jerusalem. Zadok's daughter Jerusha was his mother. \v{34}He did what the \divine{Lord} considered to be right, following everything his father Uzziah had done, \v{35}except the high places were not torn down, and the people still sacrificed and burned incense on the high places. But he rebuilt the upper gate of the \divine{Lord}'s Temple. \v{36}The rest of Jotham's activities, including everything that he accomplished, are recorded in the Book of the Chronicles of the Kings of Judah, are they not?

\v{37}Right about that time, the \divine{Lord} began to send King Rezin of Aram and Remaliah's son Pekah against Judah. \v{38}Meanwhile, Jotham died, as did\fnote{\fbackref{15:38} Lit. \fbib{Jotham slept with}} his ancestors, and was buried with them\fnote{\fbackref{15:38} Lit. \fbib{with his ancestors}} in the City of David, his ancestor. Then Jotham's son Ahaz reigned in his place.
\labelchapt{16}
\passage{Ahaz Becomes King of Judah}

\chapt{16}
\v{1}During the seventeenth year of the reign of\fnote{\fbackref{16:1} The Heb. lacks \fbib{the reign of}} Remaliah's son Pekah, Jotham's son Ahaz became king of Judah. \v{2}Ahaz was 20 years old when he became king, and he ruled in Jerusalem for sixteen years. He did not practice what the \divine{Lord} considered to be right, as had his ancestor David. \v{3}Instead, he behaved like the kings of Israel did by making his son pass through fire, the very same abomination that the heathen practiced, whom the \divine{Lord} evicted from the land right in front of the Israelis. \v{4}Furthermore, Ahaz\fnote{\fbackref{16:4} Lit. \fbib{he}} sacrificed and burned incense on the high places, on top of hills, and under every green tree.
\passage{Ahaz Seeks Help from Assyria}
\passageinfo{(2 Chronicles 28:16-21; Isaiah 7:1-16)}

\v{5}Later, King Rezin of Aram and Remaliah's son Pekah, king of Israel, approached Jerusalem to attack it. They besieged Ahaz but could not conquer him. \v{6}But at that time, King Rezin of Aram recovered Elath for Aram, completely removing the Judeans from Elath. Then the Arameans returned to Elath and have remained there to this day. \v{7}So Ahaz sent envoys to Tiglath-pileser, king of Assyria, to tell him, ``I am your servant and son. Save me from the king of Aram and the king of Israel, who are attacking me.'' \v{8}Then Ahaz took the silver and gold that was in the \divine{Lord}'s Temple and in the palace treasuries and sent them as a gift to the king of Assyria, \v{9}so the king of Assyria listened to Ahaz. He attacked Damascus, captured it, sent its people away into exile to Kir, and executed Rezin.
\passage{King Ahaz Constructs a Pagan Altar}
\passageinfo{(2 Chronicles 28:22-25)}

\v{10}King Ahaz traveled to Damascus and met with King Tiglath-pileser of Assyria, where he observed the altar at Damascus. So King Ahaz sent a set of construction patterns of this altar to Uriah the priest. \v{11}Uriah the priest built an altar, following the plans that King Ahaz had sent him from Damascus and finishing the altar before King Ahaz returned from Damascus. \v{12}When the king returned from Damascus, as soon as he saw the altar, he\fnote{\fbackref{16:12} Lit. \fbib{altar, the king}} approached it and offered sacrifices on it. \v{13}He presented a burnt offering, a meat offering, poured out a drink offering, and sprinkled the blood of a peace offering on his altar. \v{14}Then he took the bronze altar that stood in the \divine{Lord}'s presence from in front of the Temple, moved it to the north side of his altar, \v{15}and issued these orders to Uriah the priest:

\begin{poetry}
\poeml ``Burn the morning burnt offering, the evening grain offering, the king's burnt offering and grain offering, the whole burnt offering, the grain offering, and the drink offering on behalf of all the people of the land on the large altar. And sprinkle all the blood from the burnt offering and from the sacrifice. But I will use the bronze altar to ask God questions.''
\end{poetry}

\v{16}So Uriah the priest did precisely what King Ahaz ordered. \v{17}Later, King Ahaz ordered the side panels removed from the bases, along with the washing bowls that had stood on top of the bases. He also removed the large bowl that was called the Sea from on top of the bronze bulls that supported it, and put it on a stone base. \v{18}Then Ahaz removed the covered walkway for use on the Sabbath that they had built in the Temple. Because of the king of Assyria, he also removed the outside entrance from the \divine{Lord}'s Temple that had been built exclusively\fnote{\fbackref{16:18} The Heb. lacks \fbib{that had been built exclusively}} for the king.

\v{19}Now the rest of Ahaz's activities are recorded in the Book of the Chronicles of the Kings of Judah, are they not? \v{20}Later, Ahaz died, as did\fnote{\fbackref{16:20} Lit. \fbib{Ahaz slept with}} his ancestors, and was buried alongside his ancestors in the City of David. His son Hezekiah reigned in his place.
\labelchapt{17}
\passage{Israel Falls to Assyria during Hoshea's Reign}

\chapt{17}
\v{1}During the twelfth year of the reign of\fnote{\fbackref{17:1} The Heb. lacks \fbib{the reign of}} King Ahaz of Judah, Elah's son Hoshea became king over Israel for nine years in Samaria. \v{2}He practiced what the \divine{Lord} considered to be evil,\fnote{\fbackref{17:2} Lit. \fbib{sight}} though not like the kings of Israel who had preceded him. \v{3}King Shalmaneser of Assyria attacked him, and Hoshea became his servant and paid tribute to him. \v{4}But the king of Assyria uncovered a conspiracy involving Hoshea, who had sent envoys to King So of Egypt and stopped offering tribute to the king of Assyria, as he had done annually. As a result, the king of Assyria placed him under arrest and sent him to prison. \v{5}After this, the king of Assyria invaded the entire land, approached Samaria, and began a three year siege. \v{6}As a result, during the ninth year of the reign of\fnote{\fbackref{17:6} The Heb. lacks \fbib{the reign of}} Hoshea, the king of Assyria captured Samaria and took the Israelis off to Assyria, placing them in Halah, along the Habor River in Gozan, and in cities ruled by the Medes.
\passage{The Idolatry of the Northern Kingdom}

\v{7}This happened because the Israelis had sinned against the \divine{Lord} their God, who had brought them up from the land of Egypt and from the domination\fnote{\fbackref{17:7} Lit. \fbib{hand}} of Pharaoh, king of Egypt, because\fnote{\fbackref{17:7} The Heb. lacks \fbib{because}} they were fearing other gods, \v{8}and because they were following\fnote{\fbackref{17:8} Lit. \fbib{were walking in}} the rules of the nations whom the \divine{Lord} had expelled before the Israelis and that the kings of Israel had practiced.

\v{9}The Israelis practiced secret things that were not right, offending the \divine{Lord} their God. In addition, they built high places for use by all their towns, watchtowers, and fortified cities. \v{10}They set up pillars and Asherim on every high hill and in the shade of every green tree, \v{11}where they made offerings on all the high places, as did the nations whom the \divine{Lord} had expelled before them. They also practiced other\fnote{\fbackref{17:11} The Heb. lacks \fbib{other}} wickedness, provoking the \divine{Lord} to become angry, \v{12}and they served idols, a practice that the \divine{Lord} had warned them, ``You are not to do this.''

\v{13}Nevertheless, the \divine{Lord} had warned both Israel and Judah by means\fnote{\fbackref{17:13} Lit. \fbib{by the hand}} of every prophet and seer: ``Turn away from your evil practices\fnote{\fbackref{17:13} Lit. \fbib{ways}} and keep my commandments and statutes according to the entire Law that I gave your ancestors and that I sent to you through my servants, the prophets.'' \v{14}But they would not listen. Instead, they were stubborn,\fnote{\fbackref{17:14} Lit. \fbib{they hardened their necks}} just like their ancestors had been, who did not believe in the \divine{Lord} their God. \v{15}They rejected the \divine{Lord}'s\fnote{\fbackref{17:15} Lit. \fbib{rejected his}} statutes, the covenant that he had made with their ancestors, and his warnings that he gave them. They pursued meaninglessness---and became meaningless themselves---as they followed the lifestyles of the nations that surrounded them, a practice that the \divine{Lord} had warned them not to do.

\v{16}They abandoned all of the commands given by\fnote{\fbackref{17:16} The Heb. lacks \fbib{given by}} the \divine{Lord} their God, crafted for themselves cast images of two calves, constructed an Asherah, worshipped all of the stars in heaven, and served Baal. \v{17}They passed their sons and daughters through fire, practiced divination, cast spells, and sold themselves to practice what the \divine{Lord} considered to be evil, thereby\fnote{\fbackref{17:17} The Heb. lacks \fbib{thereby}} provoking him. \v{18}As a result, the \divine{Lord} was angry with Israel and removed them from his presence. No one was left except for the tribe of Judah.

\v{19}But Judah, too, did not keep the commands of the \divine{Lord} their God. Instead, they lived the lifestyle\fnote{\fbackref{17:19} Lit. \fbib{customs}} that Israel had chosen, \v{20}so the \divine{Lord} rejected all of the descendants\fnote{\fbackref{17:20} Lit. \fbib{seed}} of Israel, afflicted them, and handed them over to the control of plunderers until he had thrown them away from his presence.\fnote{\fbackref{17:20} Lit. \fbib{face}} \v{21}He ripped them away from the heritage of David, even as the people\fnote{\fbackref{17:21} Lit. \fbib{David, and they}} appointed Nebat's son Jeroboam to be king. Jeroboam drove Israel away from following the \divine{Lord} and made them commit great sin.

\v{22}The Israelis practiced\fnote{\fbackref{17:22} Lit. \fbib{Israelis walked in}} all the sins that Jeroboam had practiced, and never wavered from them \v{23}until the \divine{Lord} removed Israel from his presence,\fnote{\fbackref{17:23} Lit. \fbib{sight}} just as he had warned through\fnote{\fbackref{17:23} Lit. \fbib{spoken by the hand of}} all of his prophets who served him. So Israel was carried off into exile from their own land into Assyria, where they remain to this day.\fnote{\fbackref{17:23} I.e. c. during the late Babylonian captivity or slightly after that time}
\passage{Assyria Supplants the Northern Kingdom}

\v{24}Because the king of Assyria brought captives\fnote{\fbackref{17:24} The Heb. lacks \fbib{captives}} from Babylon, Cuthah, Avva, Hamath, and Sephar-vaim and settled them in the cities of Samaria to replace the Israelis, the settlers\fnote{\fbackref{17:24} The Heb. lacks \fbib{the settlers}} possessed Samaria and lived in its cities. \v{25}When they first began to live there, the settlers\fnote{\fbackref{17:25} The Heb. lacks \fbib{the settlers}} did not fear the \divine{Lord}, so he sent lions among them, and they killed a few of them. \v{26}As a result, they reported to the king of Assyria, ``Because the nations whom you exiled to live in the cities of Samaria don't know the law\fnote{\fbackref{17:26} Or \fbib{justice}} of the god of the land, he has sent lions among them. Look how the lions\fnote{\fbackref{17:26} Lit. \fbib{how they}} are killing them, because they don't know the law of the god of the land!''

\v{27}So the king of Assyria issued this order: ``Take one of the priests whom you carried away and let him go back and live there. Let him teach them the law of the god of the land.'' \v{28}So one of the priests whom they had carried away from Samaria went to live in Bethel to teach them how they ought to fear the \divine{Lord}.
\passage{Assyrian Settlers Create Lasting Corruption}

\v{29}Nevertheless, each nation continued to craft their own gods and install them in the temples on the high places that the people of Samaria had constructed---every nation in their own cities where they continued to live. \v{30}Settlers\fnote{\fbackref{17:30} Lit. \fbib{Men}} from Babylon built Succoth-benoth, settlers\fnote{\fbackref{17:30} Lit. \fbib{men}} from Cuth built Nergal, settlers\fnote{\fbackref{17:30} Lit. \fbib{men}} from Hamath built Ashima, \v{31}and settlers\fnote{\fbackref{17:31} Lit. \fbib{men}} from Avva built Nibhaz and Tartak. The residents of Sephar-vaim burned their children in fire to Adrammelech and Anammelech, the gods of Sephar-vaim.

\v{32}Because they feared the \divine{Lord}, they also appointed from among themselves priests for the high places who acted on their behalf in the temples on the high places. \v{33}While they continued to fear the \divine{Lord}, they served their own gods, following the custom of the nations whom they had carried away from there. \v{34}To this very day, they still follow the former customs: they don't fear the \divine{Lord} and they don't live in accordance with the statutes, ordinances, laws, or commandments that the \divine{Lord} had given to the descendants of Jacob---whom he renamed Israel--- \v{35}and with whom the \divine{Lord} had made a covenant when he gave these\fnote{\fbackref{17:35} The Heb. lacks \fbib{these}} orders to them:

\begin{poetry}
\poeml ``You are not to fear other gods, bow down to them, serve them, or sacrifice to them. \v{36}Instead, it is to be the \divine{Lord}, who brought you up from the land of Egypt, showing great power and public demonstrations of might,\fnote{\fbackref{17:36} Lit. \fbib{and with an outstretched arm}} whom you are to fear, worship, and to whom you are to offer sacrifice. \v{37}Furthermore, you are to be careful to observe forever the statutes, ordinances, law, and the commandment that he wrote for you. And you are not to fear other gods. \v{38}You are not to forget the covenant that I've made with you, and you are not to fear other gods. \v{39}But you are to fear the \divine{Lord}, and he will deliver you from the control\fnote{\fbackref{17:39} Lit. \fbib{hand}} of all your enemies.''
\end{poetry}

\v{40}But they wouldn't listen. Instead, they did what they had been doing before. \v{41}These nations feared the \divine{Lord} and also served their carved images. Their descendants did the same thing, as did their grandchildren. Just as their ancestors had done, they also do the same thing to this day.
\labelchapt{18}
\passage{Hezekiah Becomes King of Judah}
\passageinfo{(2 Chronicles 29:1-2)}

\chapt{18}
\v{1}Now it happened that during the third year of the reign of\fnote{\fbackref{18:1} The Heb. lacks \fbib{the reign of}} Elah's son Hoshea, king of Israel, that Ahaz' son Hezekiah became king. \v{2}He was 25 years old when he became king, and he reigned in Jerusalem for 29 years. His mother was Zechariah's daughter Abi. \v{3}He did what the \divine{Lord} considered to be right, according to everything that his ancestor David had done.
\passage{Hezekiah's Reforms}
\passageinfo{(2 Chronicles 29:3; 31:1)}

\v{4}He removed the high places, demolished the sacred pillars, and tore down the Asherah poles. He also demolished the bronze serpent that Moses had crafted, because the Israelis had been burning incense to it right up until that time. Hezekiah\fnote{\fbackref{18:4} Lit. \fbib{He}} called it a piece of brass.\fnote{\fbackref{18:4} Lit. \fbib{Nehushtan}; so MT; LXX reads \fbib{Neeshthan}} \v{5}He trusted the \divine{Lord} God of Israel, and after him there were none like him among all the kings of Judah, \v{6}because he depended on the \divine{Lord}, not abandoning pursuit of him, and keeping the \divine{Lord}'s commands that he had commanded Moses. \v{7}So the \divine{Lord} was with him, and Hezekiah prospered wherever he went, even when he rebelled against the king of Assyria, refusing to serve him. \v{8}He attacked the Philistines, invading Gaza and its borders from watchtower to fortified garrison.
\passage{Shalmaneser Attacks Samaria}

\v{9}In the fourth year of King Hezekiah's reign (that is, during the seventh year of Elah's son Hoshea's reign as king of Israel), King Shalmaneser from Assyria invaded Samaria and besieged it. \v{10}Three years later, they captured Samaria during the sixth year of Hezekiah's reign,\fnote{\fbackref{18:10} The Heb. lacks \fbib{reign}} which was the ninth year of Hoshea's reign as king of Israel. \v{11}After this, the king of Assyria carried Israel off into exile in Assyria, settling them in Halah, on the Habor River in Gozan, and in cities controlled by the Medes, \v{12}because they would not obey the voice of the \divine{Lord} their God. Instead, they transgressed his covenant, including everything that Moses, the servant of the \divine{Lord}, had commanded, by neither listening nor putting what he had commanded\fnote{\fbackref{18:12} The Heb. lacks \fbib{what he had commanded}} into practice.

\v{13}During the fourteenth year of the reign of\fnote{\fbackref{18:13} The Heb. lacks \fbib{the reign of}} King Hezekiah, King Sennacherib of Assyria approached all of the walled cities of Judah and seized them. \v{14}So Hezekiah sent this message to the king of Assyria at Lachish: ``I have offended you. Withdraw from me, and I'll accept whatever tribute you impose.'' So the king of Assyria required Hezekiah to pay him 300 talents\fnote{\fbackref{18:14} I.e. about 11,500 pounds; a talent weighed about 75 pounds} of silver and 30 talents\fnote{\fbackref{18:14} I.e. about 1,150 pounds; a talent weighed about 75 pounds} of gold. \v{15}Hezekiah gave him all the silver that could be removed from the \divine{Lord}'s Temple and from the treasuries in the king's palace. \v{16}At that time, Hezekiah removed the doors to the \divine{Lord}'s Temple and the doorposts that he had overlaid with gold,\fnote{\fbackref{18:16} The Heb. lacks \fbib{with gold}} and gave the gold\fnote{\fbackref{18:16} Lit. \fbib{gave it}} to the king of Assyria.
\passage{Assyria's King Taunts Hezekiah}
\passageinfo{(2 Chronicles 29:9-19)}

\v{17}Sometime later, the king of Assyria sent Tartan, Rab-saris, and Rab-shakeh from Lachish to King Hezekiah in Jerusalem, accompanied with a large army. \v{18}When they called for the king, Hilkiah's son Eliakim, who managed the household, Shebnah the scribe, and Asaph's son Joah the recorder went out to them. \v{19}Rab-shakeh told them, ``Tell Hezekiah right now, `This is what the great king, the king of Assyria says:

\begin{poetry}
\poeml ```Why are you so confident? \v{20}You're saying---but they're only empty words---`I have enough\fnote{\fbackref{18:20} The Heb. lacks \fbib{I have enough}} advice and resources to conduct warfare!' \\
\poeml ```Now who are you relying on, that you have rebelled against me? \v{21}Look, you're trusting on Egypt to lean on like a staff, but it's a crushed reed, and if you lean on it, it will collapse and pierce your hand. Pharaoh, king of Egypt, is just like that to everyone who relies on him! \\
\poeml \v{22}```Of course, you might tell me, ``We rely on the \divine{Lord} our God!'' But isn't it he whose high places and whose altars Hezekiah has demolished, all the while telling Jerusalem, ``You're to worship in front of this altar in Jerusalem?''\,' \\
\poeml \v{23}```Come now, and make a deal with my master, the king of Assyria, and I'll give you 2,000 horses, if you can furnish them with riders. \v{24}How can you refuse even one official from the least of my master's servants and rely on Egypt for chariots and horsemen? \v{25}``Now then, haven't I come up---apart from the \divine{Lord}---to attack and destroy this place? The \divine{Lord} told me, `Go up against this land and destroy it!'\,''\,'\,''
\end{poetry}

\v{26}At this, Hilkiah's son Eliakim, Shebnah, and Joah asked Rab-shakeh, ``Please speak to your servants in Aramaic, because we understand it, but don't speak the language of Judah to us within the hearing of the people who are on the wall.''

\v{27}But Rab-shakeh spoke to them, ``Has my master sent me to talk about this just to your master and to you, and not also to the men who are sitting on the wall, who will soon be eating their own feces and drinking their own urine\fnote{\fbackref{18:27} An alternate MT reading is \fbib{own water at their feet}}---along with you?'' \v{28}Then Rab-shakeh stood up and cried out loud, ``Listen to what the great king, the king of Assyria has to say. \v{29}This is what the king says:

\begin{poetry}
\poeml `Don't let Hezekiah deceive you, because he will prove to be unable to deliver you from my control.\fnote{\fbackref{18:29} Lit. \fbib{hand}} \v{30}And don't let Hezekiah make you trust in the \divine{Lord} by telling you, ``The \divine{Lord} will certainly deliver us and this city will not be handed over to the king of Assyria.'' \v{31}Don't listen to Hezekiah, because this is what the king of Assyria says: ``Make peace with me and come out to me! Each of you will eat from his own vine. Each will eat from his own fig tree. And each of you will drink water from his own cistern \v{32}until I come and take you away to a land like your own land, one overflowing with grain and new wine, a land filled with bread and vineyards, with olive trees and honey, so you may live and not die.'' \\
\poeml `But don't listen to Hezekiah when he misleads you by saying, ``The \divine{Lord} will deliver us!'' \v{33}Has any of the gods of the nations delivered his land from control by\fnote{\fbackref{18:33} Lit. \fbib{from the hand of}} the king of Assyria? \v{34}Where are the gods of Hamath and Arpad? Where are the gods of Sephar-vaim, of Hena, and Ivvah? Have they delivered Samaria from my control?\fnote{\fbackref{18:34} Lit. \fbib{hand}} \v{35}Who among all the gods of these lands has delivered their land from my control\fnote{\fbackref{18:35} Lit. \fbib{hand}}, so that the \divine{Lord} should deliver Jerusalem from me?'\,''\fnote{\fbackref{18:35} Lit. \fbib{from my hand}}
\end{poetry}

\v{36}But the people remained silent and did not answer with even so much as a word, because the king's order was, ``Don't answer him.''

\v{37}But Hilkiah's son Eliakim, who managed the household, Shebna the scribe, and Asaph's son Joah the recorder came back to Hezekiah with their clothes torn\fnote{\fbackref{18:37} I.e. as a visible response to the pending calamity} and told him what Rab-shakeh had said.
\labelchapt{19}
\passage{Isaiah Encourages Hezekiah}

\chapt{19}
\v{1}When King Hezekiah heard Eliakim's report,\fnote{\fbackref{19:1} The Heb. lacks \fbib{Eliakim's report}} he tore his clothes, put on a sackcloth covering, entered the \divine{Lord}'s Temple, \v{2}and sent Eliakim the household supervisor, Shebna the scribe, and the elders of the priests---all of them covered in sackcloth---to Amoz's son, the prophet Isaiah. \v{3}They announced to him:

\begin{poetry}
\poeml ``This is what Hezekiah says: `Today is a day of trouble, rebuke, and blasphemy,\fnote{\fbackref{19:3} Or \fbib{contempt}} because children are about to be born, but there is no strength to bring them to birth. \v{4}Perhaps the \divine{Lord} your God will take note of everything that Rab-shakeh has said, whom his master the king of Assyria sent to taunt the living God, and then he will rebuke the words that the \divine{Lord} your God has heard. Therefore offer a prayer for the survivors who remain.'\,''
\end{poetry}

\v{5}That is how the King Hezekiah's servants approached Isaiah.

\v{6}In reply, Isaiah responded to them, ``Here's how you're to report to your master:

\begin{poetry}
\poeml `This is what the \divine{Lord} says: ``Never be afraid of the words that you have heard by which the servants of the king of Assyria have blasphemed me. \v{7}Look! I'm going to cause an attitude\fnote{\fbackref{19:7} Or \fbib{to bring a spirit} } to grow within him so that he'll hear a rumor and return to his own territory, where I'll make him die by the sword in his own land!''\,'\,''
\end{poetry}
\passage{Sennacherib Defies God}
\passageinfo{(2 Chronicles 29:17-19)}

\v{8}So Rab-shakeh returned and found the king of Assyria at war with Libnah, because Rab-shakeh had heard that the king had left Lachish. \v{9}When he heard that it was being said about King Tirhakah of Ethiopia,\fnote{\fbackref{19:9} Lit. \fbib{Cush}} ``Look! He has come out to attack you!'' he again sent messengers to Hezekiah.

The messengers were told, \v{10}``This is what you are to say to King Hezekiah of Judah: `Don't let your God in whom you trust deceive you by telling you\fnote{\fbackref{19:10} The Heb. lacks \fbib{you}} ``Jerusalem won't be turned over to the control\fnote{\fbackref{19:10} Lit. \fbib{hand}} of Assyria's king.'' \v{11}`Look! you've heard what the kings of Assyria have done to all the lands---they completely destroyed them! Will you be spared? \v{12}Did the gods of those nations whom my ancestors destroyed deliver them, including Gozan, Haran, Rezeph, and Eden's descendants in Telassar? \v{13}Where is the king of Hamath, the king of Arpad, the king of the city of Sephar-vaim, the king of Hena, or the king of Ivvah?'\,''
\passage{Hezekiah's Prayer for Help}

\v{14}Hezekiah took the messages from the couriers, read them, went up to the \divine{Lord}'s Temple, and laid them out in the presence of the \divine{Lord}. \v{15}Then Hezekiah prayed in the presence of the \divine{Lord}, ``\divine{Lord} God of Israel! You live between the cherubim! You alone are the God of all the kingdoms of the earth. You have fashioned the heavens and the earth. \v{16}Turn\fnote{\fbackref{19:16} Or \fbib{Bow down}} your ear, \divine{Lord}, and listen! Open your eyes, \divine{Lord}, and observe! Listen to the message sent by Sennacherib to insult the living God! \v{17}Truly, \divine{Lord}, the kings of Assyria have devastated nations and their territories, \v{18}throwing their gods into the fire, since they weren't gods but rather were the product of men's handiwork---wood and stone. And so they destroyed them. \v{19}Now, \divine{Lord} our God, I'm praying that you will deliver us from his control, so that all the kingdoms of the earth may know that you alone, \divine{Lord}, are God!''
\passage{God's Answer through Isaiah the Prophet}

\v{20}Then Amoz's son Isaiah sent word to Hezekiah, ``This is what the \divine{Lord}, the God of Israel says: `Because you have prayed to me about King Sennacherib of Assyria, I have listened.'\,''

\v{21}``This is what the \divine{Lord} has spoken against him:

\begin{poetry}
\poeml `She despises and mocks you, \\
\poemll    this virgin daughter of Zion! \\
\poeml Behind your back she shakes her head, \\
\poemll    this daughter of Jerusalem! \\
\poeml \v{22}Who are you reproaching and blaspheming? \\
\poemll    Against whom have you raised your voice? \\
\poeml And against whom\fnote{\fbackref{19:22} The Heb. lacks \fbib{against whom}} have you lifted up your eyes in arrogance? \\
\poemll    Against the Holy One of Israel! \\
\poeml \v{23}By your messengers you have insulted the \divine{Lord}. \\
\poemll    You have claimed, \\
\poeml ``With my many chariots \\
\poemll    I ascended the heights of the mountains, \\
\poemlll       including the remotest regions of Lebanon; \\
\poeml I cut down its tall cedars \\
\poemll    and the best of its cypress trees. \\
\poeml I entered its most remote lodging place \\
\poemll    and its most fruitful\fnote{\fbackref{19:23} Or \fbib{its densest}} forest. \\
\poeml \v{24}I myself dug for and drank foreign water. \\
\poemll    With the sole of my foot I dried up all the streams of Egypt!'' \\
\poeml \v{25}`Didn't you hear? \\
\poemll    I determined it years ago! \\
\poeml I planned this from ancient times, \\
\poemll    and now I've brought it to pass, \\
\poeml to turn fortified cities \\
\poemll    into piles of ruins \\
\poeml \v{26}while their inhabitants, lacking strength, \\
\poemll    stand dismayed and confused. \\
\poeml They were like vegetation out in the fields, \\
\poemll    and like green herbs--- \\
\poeml just as grass that grows on a housetop \\
\poemll    dries out before it can grow. \\
\poeml \v{27}`But when you sit down, \\
\poemll    when you go out, \\
\poeml and when you come in, \\
\poemll    I'm aware of it! \\
\poeml \v{28}Because of your rage against me, \\
\poemll    your complacency has reached my ears. \\
\poeml I'll put my hook into your nostrils \\
\poemll    and my bit into your mouth. \\
\poeml Then I'll turn you back on the road \\
\poemll    by which you came.'
\end{poetry}

\v{29}``This will serve as a sign for you: you'll eat this year from what grows by itself, in the second year what grows from that, and in the third year you'll sow, reap, plant vineyards, and enjoy\fnote{\fbackref{19:29} Lit. \fbib{eat}} their fruit. \v{30}Those who survive from Judah's household will again put down deep roots and bear fruit extensively,\fnote{\fbackref{19:30} Or \fbib{upwards}} \v{31}because a remnant will go out from Jerusalem, and survivors from Mount Zion. The zeal of the \divine{Lord}\fnote{\fbackref{19:31} So MT; LXX and a MT variant read \fbib{\divine{Lord} of the Heavenly Armies}} will bring this about.''

\v{32}``Therefore this is what the \divine{Lord} says concerning the king of Assyria: `Not only will he not approach this city or shoot an arrow in its direction, he won't approach it with so much as a shield, nor will he throw up a siege ramp against it. \v{33}He'll return on the same route by which he came---he won't come to this city,' declares the \divine{Lord}. \v{34}`I will defend this city and preserve it for my own reasons, and because of my servant David.'\,''
\passage{God Destroys the Assyrian Army}
\passageinfo{(2 Chronicles 32:20-21)}

\v{35}That very night, the angel of the \divine{Lord} went out to the camp of the Assyrian army and killed 185,000 men. Early the next morning, when the army of Israel\fnote{\fbackref{19:35} Lit. \fbib{when they}} arose, all 185,000 soldiers\fnote{\fbackref{19:35} The Heb. lacks \fbib{185,000 soldiers}} were dead. \v{36}As a result, King Sennacherib of Assyria left and returned to Nineveh where he lived. \v{37}Later on, as he was worshiping in the temple of his god Nisroch, Adrammelech\fnote{\fbackref{19:37} So MT; LXX and a MT variant read \fbib{his sons Adrammelech}} and Sharezer killed him with a sword and fled into the territory of Ararat. Then Sennacherib's\fnote{\fbackref{19:37} Lit. \fbib{his}} son Esarhaddon became king in his place.
\labelchapt{20}
\passage{Hezekiah's Sickness and Recovery}
\passageinfo{(2 Chronicles 32:24-26)}

\chapt{20}
\v{1}During this time, Hezekiah became sick with a fatal illness, so Isaiah the prophet, the son of Amoz, approached him and told him, ``This is what the \divine{Lord} says: `Put your household in order, because you are dying. You will not survive.'\,''

\v{2}So Hezekiah turned his face to the wall and prayed to the \divine{Lord}. \v{3}``Remember me, \divine{Lord},'' he said, ``how I have walked in your presence with integrity, with an undivided heart, and I have accomplished what is good in your sight.'' And Hezekiah wept deeply.

\v{4}Before Isaiah had left the middle court, this message from the \divine{Lord} came to him. \v{5}``Return to Hezekiah,'' he said, ``and tell the Commander-in-Chief\fnote{\fbackref{20:5} Lit. \fbib{Nagid}; i.e. a senior officer entrusted with dual roles of operational oversight and administrative authority} of my people: `This is what the \divine{Lord}, the God of your ancestor David, says: ``I've heard your prayer and I've observed your tears. Look! I'm healing you. Three days from now, you'll go visit the \divine{Lord}'s Temple. \v{6}Furthermore, I'll add fifteen years to your life. I'll deliver you and this city from domination by\fnote{\fbackref{20:6} Lit. \fbib{from the hand of}} the king of Assyria, and I'll defend this city for my own sake and for the sake of my servant David.''\,'\,''

\v{7}Isaiah said, ``Take a fig cake.'' So some attendants\fnote{\fbackref{20:7} Lit. \fbib{So they}} took it, laid it on Hezekiah's\fnote{\fbackref{20:7} Lit. \fbib{the}} boil, and he recovered.

\v{8}Now Hezekiah had asked Isaiah, ``What is to be the sign that the \divine{Lord} is healing me and that I'll be going up to the \divine{Lord}'s Temple three days from now?''

\v{9}So Isaiah replied, ``This will be your sign from the \divine{Lord} that the \divine{Lord} will do what he has promised. Shall the shadow go forward ten steps or go back ten steps?''

\v{10}Hezekiah answered, ``It's an easy thing for a shadow to lengthen ten steps. So let the shadow go backward ten steps.''

\v{11}So Isaiah cried out to the \divine{Lord}, who brought the shadow back ten steps after it had gone down the stairway of Ahaz.
\passage{Hezekiah Shows His Treasure to the Babylonian Envoys}

\v{12}Some time later, Berodach-baladan,\fnote{\fbackref{20:12} So MT; LXX and a MT variant read \fbib{Marodach-baladan}} the son of King Baladan of Babylon, sent letters and a gift to Hezekiah, because he had heard that Hezekiah had been ill. \v{13}Hezekiah listened to the entourage\fnote{\fbackref{20:13} Lit. \fbib{to them}} and showed them his entire treasury, including the silver, gold, and spices, the precious oil, his armory, and everything that was inventoried in his treasuries. There was nothing in his household or in his holdings that Hezekiah did not show them.

\v{14}Then Isaiah the prophet came to King Hezekiah and asked him, ``What did these men have to say, and where did they come from?''

Hezekiah replied, ``They came from a country far away---from Babylon.''

\v{15}He asked, ``What did they see in your household?''

Hezekiah answered, ``They have seen everything. In my household there is nothing in my treasuries that I haven't shown them.''

\v{16}Then Isaiah replied to Hezekiah, ``Listen to this message from the \divine{Lord}: \v{17}`Watch out! The days are coming when everything that's in your house---everything that your ancestors have saved up right to this day---will be carried off to Babylon. Nothing will be left,' declares the \divine{Lord}. \v{18}`Some of your descendants---your very own seed, whom you will father---will be carried away to become officials\fnote{\fbackref{20:18} Or \fbib{court officials}; the position may have mandated castration as a condition of service} in the palace of the king of Babylon.'\,''

\v{19}At this, Hezekiah replied to Isaiah, ``What you've spoken from the \divine{Lord} is good,'' because he had been thinking, ``Why not, as long as there's peace and security\fnote{\fbackref{20:19} Lit. \fbib{truth}} in my lifetime{\ldots}?''

\v{20}Now the rest of Hezekiah's actions, as well as his glorious deeds, including how he constructed the pool and the conduit to bring water into the city, are recorded in the Book of the Chronicles of the Kings of Judah, are they not? \v{21}Hezekiah died, as did\fnote{\fbackref{20:21} Lit. \fbib{Hezekiah slept with}} his ancestors, and his son Manasseh became king in his place.
\labelchapt{21}
\passage{Manasseh Succeeds Hezekiah}

\chapt{21}
\v{1}Manasseh began to reign at the age of twelve, and he reigned for 55 years in Jerusalem. His mother was named Hephzibah. \v{2}He did what the \divine{Lord} considered to be evil, following the despicable practices of the nations whom the \divine{Lord} had expelled in full view of the people of Israel. \v{3}He rebuilt the high places that his father Hezekiah had destroyed. He erected altars for Baal, crafted an Asherah, just as King Ahab of Israel had done, and worshipped and served the stars of heaven. \v{4}He also built altars in the \divine{Lord}'s Temple, about which the \divine{Lord} had said, ``In Jerusalem I will place my Name.'' \v{5}He built two altars to every star in the heavens in the two courts of the \divine{Lord}'s Temple. \v{6}He made his son into a burnt offering, practiced witchcraft, used divination, and consorted with mediums and spirit-channelers.\fnote{\fbackref{21:6} Or \fbib{wizards}} He practiced many things that the \divine{Lord} considered to be evil and provoked him.

\v{7}He also erected the carved image of Asherah that he had made inside the Temple about which the \divine{Lord} had spoken to David and to his son Solomon, ``I will put my Name forever in this Temple and in Jerusalem, which I have chosen from all of the tribes of Israel. \v{8}And I will not make Israel's feet to wander anymore from the land that I have given to their ancestors, if they will only be careful to do everything that I have commanded them according to the entire Law that my servant Moses commanded them.'' \v{9}But they would not listen. Manasseh led them astray to practice more evil than the nations whom the \divine{Lord} had destroyed in the presence of the Israelis.
\passage{The \divine{Lord} Rebukes Manasseh's Idolatry}

\v{10}So the \divine{Lord} announced through his prophets, \v{11}``Because King Manasseh of Judah has committed these despicable things, acting more sinfully than did all of the Amorites who preceded him, including making Judah sin with its idols, \v{12}therefore this is what the \divine{Lord} God of Israel says: `Look! I'm going to bring such a\fnote{\fbackref{21:12} The Heb. lacks \fbib{such a}} disaster to Jerusalem and Judah that both ears of those who hear about it will ring. \v{13}I'll stretch out over Jerusalem the measuring line that is Samaria and the plumb line that is Ahab's dynasty. Then I'll wipe Jerusalem like one wipes a dish, wiping it and turning it upside down! \v{14}I will abandon the survivors of my heritage and hand them over to their enemies. They will become war booty and spoil to all of their enemies, \v{15}because they have done what I consider to be evil and they have provoked me from the day their ancestors left Egypt right up to this day!'\,''

\v{16}In addition to this, Manasseh shed lots of innocent blood---until he had filled Jerusalem from one end to another---besides his sin by which he caused Judah to sin by practicing what the \divine{Lord} considered to be evil. \v{17}The rest of Manasseh's deeds, including everything that he accomplished and the sin that he practiced, are recorded in the Book of the Chronicles of the Kings of Judah, are they not? \v{18}Manasseh died, as did\fnote{\fbackref{21:18} Lit. \fbib{Manasseh slept with}} his ancestors, and he was buried in the garden at his home in the Garden of Uzza. His son Amon became king in his place.
\passage{Amon Reigns in Judah}

\v{19}Amon began to reign at the age of 22, and ruled for two years in Jerusalem. His mother was named Meshullemeth, the daughter of Haruz of Jotbah. \v{20}He practiced what the \divine{Lord} considered to be evil, just as his father Manasseh had done, \v{21}because he completely adopted his father's lifestyle, serving the same idols his father had served and worshipped. \v{22}As a result, he abandoned the \divine{Lord} God of his ancestors and did not walk in the \divine{Lord}'s way. \v{23}Later on, Amon's staff conspired against him and killed the king inside his own home. \v{24}But afterward, the people of the land executed everyone who had conspired against King Amon, and the people of the land installed his son Josiah to be king in his place.

\v{25}Now the rest of Amon's activities that he undertook are recorded in the Book of the Chronicles of the Kings of Judah, are they not? \v{26}He was buried in his own grave in the Garden of Uzza, and his son Josiah became king in his place.
\labelchapt{22}
\passage{Josiah Succeeds Amon}

\chapt{22}
\v{1}Josiah was an eight year old child when he began to reign, and he reigned for 31 years in Jerusalem. His mother was named Jedidah, the daughter of Adaiah of Bozkath. \v{2}He practiced what the \divine{Lord} considered to be right, living the way his ancestor David had lived, turning neither to the right nor to the left.

\v{3}Eighteen years after King Josiah had begun to reign, the king sent Azaliah's son Shaphan, grandson of Meshullam the scribe, to the \divine{Lord}'s Temple. He told him, \v{4}``Go to the high priest Hilkiah, so he can count the money that has been brought into the \divine{Lord}'s Temple by the doorkeepers who have been gathering it from the people. \v{5}Have them deliver it to the workmen who are supervising the \divine{Lord}'s Temple, so that they may pay it over to the workmen who serve in the \divine{Lord}'s Temple to repair its damages, \v{6}including paying\fnote{\fbackref{22:6} The Heb. lacks \fbib{paying}} the carpenters, builders, and masons, as well as buying timber and pre-carved stone to repair the Temple. \v{7}But you won't need to force them to be accountable for money already paid to them, since they're faithful.''
\passage{Hilkiah Discovers an Ancient Archive}

\v{8}Later on, Hilkiah the high priest informed Shaphan the scribe, ``I've discovered the Book of the Law in the \divine{Lord}'s Temple.'' Hilkiah gave the book to Shaphan, and he began to read it.

\v{9}Shaphan the scribe reported to King Josiah, brought up the matter to him, and told him, ``Your servants have distributed the money that was found in the Temple by giving it to the workmen who supervise the \divine{Lord}'s Temple.'' \v{10}Then Shaphan the scribe informed the king, ``Hilkiah the priest has given me a book.'' Then Shaphan read from it in the king's presence.

\v{11}When the king heard what was written in the Book of the Law, he tore his clothes \v{12}and issued these orders to Hilkiah the priest, Shaphan's son Ahikam, Micaiah's son Achbor, Shaphan the scribe, and the king's servant Asaiah: \v{13}``Go ask the \divine{Lord} for me, for the people, and for all of Judah about what's written in this book that has been discovered, because the \divine{Lord}'s anger is burning against us, since our ancestors have not listened to the words written in this book and have not lived according to everything that is written concerning us.''
\passage{Huldah Predicts Disaster}

\v{14}So Hilkiah the priest, Ahikam, Achbor, Shaphan, and Asaiah went to the prophet Huldah, the wife of Tikvah's son Shallum, the grandson of Harhas and supervisor of the royal wardrobe, who lived in the Second Quarter in Jerusalem. They spoke with her, \v{15}and she told them, ``This is what the \divine{Lord} God of Israel says: `Tell the man who sent you to me: \v{16}``This is what the Lord says: `Look! I'm bringing disaster on this place and on its inhabitants---everything written in the book that the king of Judah has read---\v{17}because they have abandoned me, burned incense to other gods, and they have provoked me to anger with everything that they've done. Therefore my anger is kindled against this place and it won't be quenched!'\,'' \v{18}Nevertheless, tell the king of Judah who sent you to ask the \divine{Lord} about this,\fnote{\fbackref{22:18} The Heb. lacks \fbib{about this}} ``This is what the \divine{Lord} God of Israel says: `Now about what you've heard, \v{19}because your heart was sensitive, and you humbled yourself in the \divine{Lord}'s presence when you heard what I had to say against this place and against its inhabitants---that they would become a desolation and a curse---and you have torn your clothes and cried out before me, be assured that I have truly heard you,' declares the \divine{Lord}. \v{20}`Therefore, look! I will gather you to your ancestors, and you will be placed in your grave in peace. Your eyes will never see all the evil that I will bring on this place.'\,''\,'\,''
\labelchapt{23}
\passage{Josiah's Covenant}

\chapt{23}
\v{1}At this, the king sent for and gathered together all the elders of Judah and Jerusalem. \v{2}The king went up to the \divine{Lord}'s Temple, accompanied by all the men of Judah, everyone who lived in Jerusalem, the priests, the prophets, and everyone---including those who were unimportant and those who were important---and he read to them everything written in the Book of the Covenant that had been discovered in the \divine{Lord}'s Temple. \v{3}The king stood beside a pillar and made a covenant in the presence of the \divine{Lord}: to follow after the \divine{Lord}, to keep his commandments, his testimonies, and his statutes with all of his heart and soul, and to carry out what was written in the covenant contained in the book. All the people consented to enter into the covenant.
\passage{Josiah Abolishes Idolatry}

\v{4}The king ordered Hilkiah the high priest, the priests of the secondary order, and the doorkeepers to take out of the \divine{Lord}'s Temple all of the implements that had been crafted for Baal, for Asherah, and for every star in the heavens. Then he burned them outside Jerusalem in the fields of the Kidron and carried the ashes to Bethel. \v{5}The king unseated the idolatrous priests whom the kings of Judah had appointed to burn incense in the high places throughout the cities of Judah and in the environs surrounding Jerusalem, including those who had been burning incense to Baal, to the sun, to the moon, to the constellations, and to every star in the heavens. \v{6}He brought the Asherah from the \divine{Lord}'s Temple to the Kidron Brook outside Jerusalem, burned it at the Kidron brook, pulverized the ashes\fnote{\fbackref{23:6} The Heb. lacks \fbib{the ashes}} to dust, and scattered it\fnote{\fbackref{23:6} The Heb. lacks \fbib{it}} over the graves of the common people.

\v{7}He also demolished the temples of the cultic male prostitutes that had been operating\fnote{\fbackref{23:7} The Heb. lacks \fbib{operating}} in the \divine{Lord}'s Temple, where the women had been doing weaving for the Asherah. \v{8}Then he gathered together all the priests from the cities of Judah and defiled the high places from Geba to Beer-sheba, where the priests had burned incense. He also demolished the high places of the gates that had been erected to the left as one enters the city gate---that is, near the entrance operated by Joshua, the governor of the city. \v{9}Nevertheless, the priests of the high places did not approach the \divine{Lord}'s altar in Jerusalem, but instead they ate unleavened bread given to them by their\fnote{\fbackref{23:9} Or \fbib{bread among}} relatives.

\v{10}He also defiled Topheth, which is located in the Ben-hinnom Valley,\fnote{\fbackref{23:10} So MT; LXX and MT variant read \fbib{the valley of the descendants of Hinnom}} so that no one would force his son or daughter to pass through the fire in dedication to Molech. \v{11}He abolished the horses that the kings of Judah had dedicated to the sun at the entrance to the \divine{Lord}'s Temple, near the offices of Nathan-melech, the official, that were in the precincts. He also set fire to the chariots of the sun.

\v{12}The king demolished the rooftop altars on top of Ahaz's upper chamber that the kings of Judah had erected, as well as the altars that Manasseh had made in the two courts of the \divine{Lord}'s Temple. He pulverized them where they stood and cast their dust into the Kidron Brook. \v{13}The king defiled the high places which faced\fnote{\fbackref{23:13} So LXX.} Jerusalem on the south\fnote{\fbackref{23:13} Lit. \fbib{right}; i.e. the side on the right when facing east} side of Corruption Mountain, which King Solomon of Israel had constructed for Ashtoreth, the Sidonian abomination, for Chemosh, the Moabite abomination, and for Milcom, the Ammonite abomination. \v{14}He broke the pillars to pieces, cut down the Asherim, and filled their locations with human bones.

\v{15}Furthermore, he even broke down the altar that had been at Bethel as well as the high place constructed by Nebat's son Jeroboam, who had caused Israel to sin. He demolished its stones, pulverized them to dust, and burned the Asherah. \v{16}As Josiah turned around, he observed the graves located there on the mountain, so he sent for and recovered the bones from the graves and burned them on the altar to defile it, in keeping with the message from the \divine{Lord} that the godly man had proclaimed when he was declaring these things. \v{17}He asked, ``What is this monument that I'm looking at?''

The men who lived in that city answered him, ``It's the grave of that godly man who came from Judah and predicted these things that you've done against the altar at Bethel!''

\v{18}Josiah\fnote{\fbackref{23:18} Lit. \fbib{He}} replied, ``Leave him alone. No one is to disturb his bones.'' So they preserved his bones undisturbed, along with the bones of the prophet who had come from Samaria. \v{19}Josiah also removed all of the temples on the high places that had been in the cities of Samaria and that the kings of Israel had erected, thereby provoking the \divine{Lord}.\fnote{\fbackref{23:19} So LXX. The Heb. lacks \fbib{the \divine{Lord}}} He treated Samaria\fnote{\fbackref{23:19} Lit. \fbib{them}} just as he had Bethel. \v{20}After he had slaughtered all the priests who served at the high places and burned their bones on those high places, he returned to Jerusalem.
\passage{Josiah Reinstates the Passover}

\v{21}After this, the king commanded all of the people, ``Celebrate the Passover to the \divine{Lord} your God, just as it's prescribed in this Book of the Covenant.'' \v{22}From the days of the judges who ruled in Israel, no Passover had been celebrated like this, not even in all the reigns of the kings of Israel and the kings of Judah. \v{23}In the eighteenth year of the reign of\fnote{\fbackref{23:23} The Heb. lacks \fbib{the reign of}} King Josiah, this Passover was observed in Jerusalem to honor the \divine{Lord}. \v{24}Furthermore, Josiah removed the mediums, the necromancers, the household gods,\fnote{\fbackref{23:24} Lit. \fbib{the teraphim}} the idols, and every despicable thing that could be seen in the territory of Judah and in Jerusalem, so that he might confirm the words of the Law that had been written in the book that Hilkiah the priest had discovered in the \divine{Lord}'s Temple. \v{25}There had been no king like him before him, who turned to the \divine{Lord} with all his heart, with all his soul, and with all his strength, in obeying everything in the Law of Moses. No king arose like Josiah after him.

\v{26}Even so, the \divine{Lord} did not turn away from his fierce and great anger that burned against Judah because of everything with which Manasseh had provoked him. \v{27}The \divine{Lord} said, ``I'm going to remove Judah from my sight as well, just as I've removed Israel. I will abandon Jerusalem, this city that I've chosen, as well as the Temple, about which I've spoken, `My Name shall remain there.'\,''
\passage{Pharaoh Neco Kills Josiah}

\v{28}Now the rest of Josiah's actions, including everything that he did, are recorded in the Book of the Chronicles of the Kings of Judah, are they not? \v{29}During his reign, Pharaoh Neco, king of Egypt, marched out toward the Euphrates River to meet the king of Assyria. King Josiah went out to engage him in battle, but Pharaoh Neco\fnote{\fbackref{23:29} Lit. \fbib{but he}} killed him at Megiddo as soon as he saw him. \v{30}Josiah's servants drove his corpse in a chariot from Megiddo to Jerusalem and buried him in a tomb made for him.
\passage{Jehoahaz is Anointed King}

The people of the land took Josiah's son Jehoahaz, anointed him, and installed him as king in his father's place. \v{31}Jehoahaz was 23 years old when he became king. He reigned three months in Jerusalem. His mother's name was Hamutal. She was the daughter of Jeremiah of Libnah. \v{32}He practiced what the \divine{Lord} considered to be evil, just as all of his ancestors had done. \v{33}Pharaoh Neco placed him in custody at Riblah, in the land of Hamath, so that he would not reign in Jerusalem, and imposed a tribute of 100 talents\fnote{\fbackref{23:33} I.e. about 7,500 pounds; a talent weighed about 75 pounds} of silver and a talent\fnote{\fbackref{23:33} I.e. about 75 pounds; a talent weighed about 75 pounds} of gold.
\passage{Jehoiakim is Made King by Pharaoh Neco}

\v{34}Pharaoh Neco installed Josiah's son Eliakim as king to replace his father Josiah and changed his name to Jehoiakim. He transported Jehoahaz off to Egypt, where he died. \v{35}As a result, Jehoiakim paid the silver and gold tribute\fnote{\fbackref{23:35} The Heb. lacks \fbib{tribute}} to Pharaoh, but he passed on the costs to the inhabitants of the land in taxes, in keeping with Pharaoh's orders. He exacted the silver and gold from the people who lived in the land, from each according to his assessment, in order to pay it to Pharaoh Neco. \v{36}Jehoiakim was 25 years old when he became king, and he reigned for eleven years in Jerusalem. His mother was named Zebidah. She was the daughter of Pedaiah of Rumah. \v{37}Eliakim practiced what the \divine{Lord} considered to be evil, just as his ancestors had done.
\labelchapt{24}
\passage{Jehoiakim Serves Nebuchadnezzar}

\chapt{24}
\v{1}During his lifetime, King Nebuchadnezzar of Babylon attacked Jehoiakim, who became his vassal for three years, after which he turned against Nebuchadnezzar\fnote{\fbackref{24:1} Lit. \fbib{him}} and rebelled. \v{2}The \divine{Lord} sent raiding parties from the Chaldeans, Arameans, Moabites, and Ammonites against Jehoiakim. He sent them against Judah to destroy it, in keeping with the message from the \divine{Lord} that he had spoken through his servants, the prophets. \v{3}It was truly by the command of the \divine{Lord} against Judah that it came, in order to remove them from his sight, because of every sin that Manasseh had committed, \v{4}as well as for the innocent blood that he had shed. He had filled Jerusalem with innocent blood, and the \divine{Lord} would not forgive them.\fnote{\fbackref{24:4} The Heb. lacks \fbib{them}} \v{5}Now the rest of Jehoiakim's actions, and everything that he undertook, are recorded in the Book of the Chronicles of the Kings of Judah, are they not? \v{6}Jehoiakim died, as did\fnote{\fbackref{24:6} Lit. \fbib{Jehoiakim slept with}} his ancestors, and his son Jehoiachin became king in his place. \v{7}The king of Egypt did not leave his territory again, because the king of Babylon had taken everything that belonged to the king of Egypt from the Wadi\fnote{\fbackref{24:7} I.e. a seasonal stream or river that channels water during rain seasons but is dry at other times} of Egypt to the Euphrates River.
\passage{Jehoiachin Becomes King}

\v{8}Jehoiachin became king at the age of eighteen years, and he reigned for three months in Jerusalem. His mother was named Hausa. She was the daughter of Elzaphan of Jerusalem. \v{9}He practiced what the \divine{Lord} considered to be evil, just as his ancestors had done. \v{10}At that time, the servants of King Nebuchadnezzar of Babylon attacked Jerusalem and the city was placed under siege. \v{11}King Nebuchadnezzar of Babylon came up against the city, along with his servants, who besieged it. \v{12}King Jehoiachin of Judah surrendered to the king of Babylon (as did his mother, his servants, his princes, and his officers) during the eighth year of his reign.
\passage{Jerusalem's Citizens are Sent into Exile}

\v{13}Nebuchadnezzar\fnote{\fbackref{24:13} Lit. \fbib{He}} carried off from there all of the treasures of the \divine{Lord}'s Temple, along with the treasures in the king's palace. He cut into pieces all the gold vessels in the \divine{Lord}'s Temple that King Solomon of Israel had made, just as the \divine{Lord} had said would happen.\fnote{\fbackref{24:13} The Heb. lacks \fbib{would happen}} \v{14}Then Nebuchadnezzar sent away into exile all of Jerusalem---all the captains, all the valiant soldiers, 10,000 captives, and all of the craftsmen and ironworkers. Nobody remained except the poorest people of the land. \v{15}He sent Jehoiachin into exile to Babylon, along with the king's mother, the king's wives, his officials, and the leading men of the land. He took them into exile from Jerusalem to Babylon. \v{16}All 7,000 of the most valiant soldiers and 1,000 of the craftsmen and ironworkers---all physically fit and trained for battle---were brought by the king of Babylon into exile in Babylon.
\passage{Zedekiah is Installed as King}

\v{17}The king of Babylon installed Jehoiachin's\fnote{\fbackref{24:17} Lit. \fbib{installed his}} uncle Mattaniah as king in his place and then changed his name to Zedekiah. \v{18}Zedekiah was 21 years old when he became king. He reigned for eleven years in Jerusalem. His mother was named Hamutal. She was the daughter of Jeremiah of Libnah. \v{19}Zedekiah practiced what the \divine{Lord} considered to be evil, just as Jehoiakim had done, \v{20}because through the \divine{Lord}'s anger these things happened\fnote{\fbackref{24:20} The Heb. lacks \fbib{these things}} to Jerusalem and Judah until he threw them from his presence.
\labelchapt{25}
\passage{Nebuchadnezzar Captures Jerusalem}

Zedekiah then rebelled against the king of Babylon,\chapt{25}
\v{1}so on the tenth day of the tenth month of the ninth year of Zedekiah's\fnote{\fbackref{25:1} Lit. \fbib{his}; but cf. 25:3, which suggests it refers to Zedekiah} reign, King Nebuchadnezzar of Babylon and his entire army approached Jerusalem, attacked it, encamped against it, and built a siege wall that surrounded the city. \v{2}The city remained under siege until the eleventh year of the reign of\fnote{\fbackref{25:2} The Heb. lacks \fbib{the reign of}} King Zedekiah. \v{3}By the ninth day of the fourth\fnote{\fbackref{25:3} The Heb. lacks \fbib{fourth}; but cf. Jer. 52:6} month, the resulting\fnote{\fbackref{25:3} The Heb. lacks \fbib{resulting}} famine had become so severe in the city that no food remained for the people who lived in the land. \v{4}The city was breached, and the entire army left during the night through the gate that stood between the two walls beside the royal garden, even though the Chaldeans had surrounded the city. They escaped through the Arabah, \v{5}but the Chaldean army pursued the king and overtook him in the Jericho plains, where his entire army was scattered. \v{6}The Chaldeans captured the king and brought him to Riblah, where the king of Babylon determined his sentence. \v{7}They executed Zedekiah's sons in his presence, blinded Zedekiah, bound him with bronze chains, and transported him to Babylon.
\passage{Jerusalem is Burned and the Temple Demolished}

\v{8}On the seventh\fnote{\fbackref{25:8} Cf. Jer 52:12, which reads \fbib{tenth}} day of the fifth month, which was during the nineteenth year of King Nebuchadnezzar's reign as king of Babylon, captain of the guard Nebuzaradan, a servant of the king of Babylon, arrived in Jerusalem \v{9}and set fire to the \divine{Lord}'s Temple, the royal palace, and all the houses of Jerusalem. He even incinerated the lavish\fnote{\fbackref{25:9} Lit. \fbib{great}} homes. \v{10}The Chaldean army that accompanied the captain of the guard demolished the walls that surrounded Jerusalem. \v{11}Nebuzaradan, the captain of the guard, carried the survivors of the people who remained in the city, those who had deserted to the king of Babylon, and the rest of the multitude into exile. \v{12}However, the captain of the guard left some of the poor people of the land to work as vinedressers and farmers.

\v{13}The Chaldeans also broke into pieces and carried back to Babylon the bronze pillars that stood in the \divine{Lord}'s Temple, along with the stands and the bronze sea\fnote{\fbackref{25:13} Cf. 1King 7:23-26; 2Chr 4:2-4} that used to be in the \divine{Lord}'s Temple. \v{14}They also confiscated\fnote{\fbackref{25:14} Or \fbib{took away}} the pots, shovels, snuffers, spoons, and the rest of the bronze vessels that were used in ministry. \v{15}The captain of the guard also confiscated\fnote{\fbackref{25:15} Or \fbib{took away}} the fire pans, basins, and whatever had been crafted of pure gold and pure silver. \v{16}The bronze contained in the two pillars, the one sea, and the stands that Solomon had crafted for the \divine{Lord}'s Temple could not be inventoried for weight. \v{17}The height of one of the pillars was eighteen cubits,\fnote{\fbackref{25:17} I.e. about 24 feet; a cubit was about eighteen inches long} and the capital on top of it was three cubits\fnote{\fbackref{25:17} I.e. about 4 and a half feet; a cubit was about eighteen inches long} high.\fnote{\fbackref{25:17} The Heb. lacks \fbib{high}} A latticework carved in the form of pomegranates encircled the capital, crafted completely out of brass. The second pillar was identical to the first.\fnote{\fbackref{25:17} Lit. \fbib{to these with latticework}}
\passage{Judah's Leaders are Executed}

\v{18}The captain of the guard arrested Seraiah the chief priest, Zephaniah the second priest, three temple officials,\fnote{\fbackref{25:18} Lit. \fbib{three threshold keepers}} \v{19}one overseer from the city who supervised the soldiers, five of the king's advisors who had been discovered in the city, the scribe who served the army captain who mustered the army of the land, and 60 men of the land who were discovered in the city. \v{20}Nebuzaradan, the captain of the guard, took them to the king of Babylon at Riblah, \v{21}where the king of Babylon executed them in the land of Hamath. And so Judah was transported into exile from the land.
\passage{Gedaliah is Appointed Governor}

\v{22}Now as for the people who remained in the land of Judah whom King Nebuchadnezzar of Babylon had left behind, he appointed Ahikam's son Gedaliah, the grandson of Shaphan, to rule. \v{23}When all the captains of the armies, along with their men, heard that the king of Babylon had appointed Gedaliah, these men visited Gedaliah at Mizpah: Nethaniah's son Ishmael, Kareah's son Johanan, Tanhumeth the Netophathite's son Seraiah, and Jaazaniah, who was descended from the Maacathites. \v{24}Gedaliah made this promise to them and to their men: ``Don't be afraid of the servants of the Chaldeans. Live in the land and serve the king of Babylon, and things will go well with you.'' \v{25}Nevertheless, seven months later, Nethaniah's son Ishmael, the grandson of Elishama from the royal family, came with ten men and attacked Gedaliah. As a result, he died along with the Jews and Chaldeans who were with him at Mizpah. \v{26}Then all the people, including those who were insignificant and those who were important, fled with the captains of the armed forces to Egypt, because they were afraid of the Chaldeans.
\passage{Jehoiachin Leaves Prison}

\v{27}Later on, after King Jehoiachin of Judah had been in exile for 37 years, on the twenty-seventh day of the twelfth month, during the first year of his reign, King Evil-merodach of Babylon released King Jehoiachin of Judah from prison. \v{28}He spoke kindly to him and elevated his position\fnote{\fbackref{25:28} Lit. \fbib{throne}} above the thrones of the kings with him in Babylon. \v{29}Jehoiachin changed out of his prison clothes and had regular meals in the king's presence every day for the rest of his life, \v{30}and a regular stipend was provided to him by the king in accordance with his needs for as long as he lived.
