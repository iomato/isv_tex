\bookheader{Psalms}
\labelbook{Ps}

\bookpretitle{The Book of}
\booktitle{Psalms}

\booksection{BOOK I (Psalms 1-41)}
\labelpsalm{1}
\passage{The Righteous and the Wicked\passagenote{Note: (1) Verse numbers may be different from MT because the titles of many psalms in MT are part of the first verse. (2) The phrase \fbib{A song of}, which appears in many psalm titles, may also be translated \fbib{A song by, A song for,} or \fbib{A song to}. (3) Psalm title terminologies ``a psalm of David,'' ``a song of David,'' etc., may connote---but do not necessarily connote---authorship by Israel's King David. They are rendered herein as ``a Davidic psalm,'' ``a Davidic song,'' etc. (4) The traditionally unpronounced literary term \fbib{Selah}, which may indicate that the oral reader or cantor is to pause briefly after reading the line in which the term appears, is rendered herein as \fbib{Interlude}.}}

\begin{poetry}
\poeml \v{1}How blessed is the person, \\
\poemll    who does not take\fnote{Lit. \fbib{not walk by}} the advice of the wicked, \\
\poeml who does not stand on the path with sinners, \\
\poemll    and who does not sit in the seat of mockers. \\
\poeml \v{2}But he delights in the \divine{Lord}'s instruction,\fnote{Or \fbib{Law}} \\
\poemll    and meditates in his instruction\fnote{Or \fbib{Law}} day and night. \\
\poeml \v{3}He will be like a tree planted by streams of water, \\
\poemll    yielding its fruit in its season, \\
\poemlll       and whose leaf does not wither. \\
\poeml He will prosper in everything he does. \\
\poeml \v{4}But this is not the case with the wicked. \\
\poemll    They are like chaff that the wind blows away. \\
\poeml \v{5}Therefore the wicked will not escape\fnote{Lit. \fbib{stand in the}} judgment, \\
\poemll    nor will sinners have a place\fnote{The Heb. lacks \fbib{have a place in}} in the assembly of the righteous. \\
\poeml \v{6}For the \divine{Lord} knows the way of the righteous, \\
\poemll    but the way of the wicked will be destroyed.
\end{poetry}
\labelpsalm{2}
\passage{The Nations and God's Anointed}

\begin{poetry}
\poeml \v{1}Why are the nations in an uproar, \\
\poemll    and their people involved in a vain plot? \\
\poeml \v{2}As the kings of the earth take their stand \\
\poemll    and the rulers conspire together against the \divine{Lord} and his anointed one, they say,\fnote{The Heb. lacks \fbib{say}} \\
\poeml \v{3}``Let us tear off their shackles from us,\fnote{The Heb. lacks \fbib{from us}} \\
\poemll    and cast off their chains.'' \\
\poeml \v{4}He who sits in the heavens laughs; \\
\poemll    the Lord scoffs at them. \\
\poeml \v{5}In his anger he rebukes them, \\
\poemll    and in his wrath he terrifies them: \\
\poeml \v{6}``I have set my king on Zion, \\
\poemll    my holy mountain.''
\passage{The Anointed King Speaks}
\poeml \v{7}Let me announce the decree of the \divine{Lord} \\
\poemll    that he told me: \\
\poeml ``You are my son, \\
\poemll    today I have become your father. \\
\poeml \v{8}Ask of me, and I will give you \\
\poemll    the nations as your inheritance, \\
\poemlll       the ends of the earth as your possession. \\
\poeml \v{9}You will break them with an iron rod, \\
\poemll    you will shatter them like pottery.'' \\
\poeml \v{10}Therefore, kings, act wisely! \\
\poemll    Earthly rulers, be warned! \\
\poeml \v{11}Serve the \divine{Lord} with fear, \\
\poemll    and rejoice with trembling. \\
\poeml \v{12}Kiss\fnote{Or \fbib{Worship}} the son before he becomes\fnote{Or \fbib{son lest he become}} angry, \\
\poemll    and you die where you stand.\fnote{Lit. \fbib{you perish in the way}} \\
\poeml Indeed, his wrath can flare up quickly. \\
\poeml How blessed are those who take refuge in him.
\end{poetry}
\labelpsalm{3}
\psalminfo{A Davidic Psalm, when he fled from his son Absalom.}
\passage{God Delivers His Servants}

\begin{poetry}
\poeml \v{1}\divine{Lord}, I have so many persecutors! \\
\poemll    Many are rising up against me! \\
\poeml \v{2}Many are saying about me, \\
\poemll    ``God will never deliver him!''
\end{poetry}
\interlude{Interlude}

\begin{poetry}
\poeml \v{3}But you, \divine{Lord}, are a shield around me, \\
\poemll    my glory, and the one who lifts up my head. \\
\poeml \v{4}I cry aloud\fnote{Lit. \fbib{with my voice}} to the \divine{Lord}, \\
\poemll    and he answers me from his holy mountain.
\end{poetry}
\interlude{Interlude}

\begin{poetry}
\poeml \v{5}I lie down and sleep, \\
\poemll    I wake up, because the \divine{Lord} sustains me. \\
\poeml \v{6}I will not fear multitudes of\fnote{Or \fbib{ten thousand}} people, \\
\poemll    who set themselves against me on every side. \\
\poeml \v{7}Arise, \divine{Lord}! \\
\poemll    Deliver me, my God! \\
\poeml For you strike the jaw of all my enemies, \\
\poemll    and you break the teeth of the wicked. \\
\poeml \v{8}Deliverance comes from the \divine{Lord}! \\
\poemll    May your blessing be on your people.
\end{poetry}
\interlude{Interlude}
\labelpsalm{4}
\psalminfo{To the Director: With stringed instruments. A Davidic Psalm}
\passage{Trust God under Adversity}

\begin{poetry}
\poeml \v{1}When I call, answer me, \\
\poemll    my righteous God!\fnote{Or \fbib{God of my righteousness}} \\
\poeml When I was in distress, you set me free. \\
\poemll    Be gracious to me and hear my prayer. \\
\poeml \v{2}You people, \\
\poemll    how long will you malign my reputation? \\
\poeml How long will you love what is vain\fnote{I.e. what has no substance} \\
\poemll    and what is false?
\end{poetry}
\interlude{Interlude}

\begin{poetry}
\poeml \v{3}But understand this:\fnote{The Heb. lacks \fbib{this}} \\
\poemll    the \divine{Lord} has set apart the godly for himself! \\
\poemlll       The \divine{Lord} will hear me when I cry out to him! \\
\poeml \v{4}Be angry, yet do not sin.\fnote{Cf. Eph 4:26} \\
\poemll    Think about this\fnote{The Heb. lacks \fbib{this}} when upon your beds, \\
\poemlll       and be silent.
\end{poetry}
\interlude{Interlude}

\begin{poetry}
\poeml \v{5}Offer sacrifices that are righteous, \\
\poemll    and put your confidence in the \divine{Lord}. \\
\poeml \v{6}Many are asking, ``Who will help us to see better days?''\fnote{Lit. \fbib{cause us to see good}} \\
\poemll    \divine{Lord}, may the light of your favor\fnote{Lit. \fbib{face}} shine upon us. \\
\poeml \v{7}You have given me more joy in my heart than at harvest times, \\
\poemll    when grain and wine abound. \\
\poeml \v{8}I will lie down and sleep in peace, \\
\poemll    for you alone, \divine{Lord}, enable me to live securely.
\end{poetry}
\labelpsalm{5}
\psalminfo{To the Director: For flutes. A Davidic Psalm}
\passage{A Prayer for God's Help}

\begin{poetry}
\poeml \v{1}\divine{Lord}, listen to my words, \\
\poemll    consider my groaning. \\
\poeml \v{2}Pay attention to my cry for help,\fnote{Lit. \fbib{the sound of my cry for help}} \\
\poemll    my king and my God, \\
\poemlll       for unto you will I pray. \\
\poeml \v{3}\divine{Lord}, in the morning you will hear my voice; \\
\poemll    in the morning I will pray\fnote{Lit. \fbib{arrange my prayer}} to you, \\
\poemll    and I will watch for your answer.\fnote{The Heb. lacks \fbib{for your answer}} \\
\poeml \v{4}Indeed, you aren't a God who delights in wickedness; \\
\poemll    evil will never dwell with you. \\
\poeml \v{5}Boastful ones will not stand before you; \\
\poemll    you hate all those who practice wickedness. \\
\poeml \v{6}You will destroy those who speak lies. \\
\poemll    The \divine{Lord} abhors the person of bloodshed and deceit. \\
\poeml \v{7}But I, because of the abundance of your gracious love, \\
\poemll    may come into your house. \\
\poemlll       In awe of you, I will worship in your holy Temple. \\
\poeml \v{8}\divine{Lord}, lead me in your righteousness because of my enemies. \\
\poemll    Make your path straight before me. \\
\poeml \v{9}But as for the wicked,\fnote{The Heb. lacks \fbib{as for the wicked}} \\
\poemll    they do not speak truth at all. \\
\poemlll       Inside them there is only wickedness. \\
\poeml Their throat is an open grave, \\
\poemll    on their tongue is deceitful flattery. \\
\poeml \v{10}Declare them guilty, God! \\
\poemll    Let them fall by their own schemes. \\
\poeml Drive them away because of their many transgressions, \\
\poemll    for they have rebelled against you. \\
\poeml \v{11}Let all those who take refuge in you rejoice! \\
\poemll    Let them shout for joy forever, \\
\poeml and may you protect them. \\
\poemll    Let those who love your name exult in you. \\
\poeml \v{12}Indeed, you will bless the righteous one, \divine{Lord}, \\
\poemll    like a large shield, you will surround him with favor.
\end{poetry}
\labelpsalm{6}
\psalminfo{To the Director: With stringed instruments. On an eight-stringed harp.\fnote{T Or \fbib{On a lower octave}} A Davidic Psalm}
\passage{A Prayer in Times of Trouble}

\begin{poetry}
\poeml \v{1}\divine{Lord}, in your anger, do not rebuke me, \\
\poemll    in your wrath, do not discipline me. \\
\poeml \v{2}Be gracious to me, \divine{Lord}, \\
\poemll    because I am fading away. \\
\poeml Heal me, \\
\poemll    because my body\fnote{Or \fbib{bones}} is distressed. \\
\poeml \v{3}And my soul\fnote{Or \fbib{And I am}} is deeply distressed. \\
\poemll    But you, \divine{Lord}, how long do I wait?\fnote{The Heb. lacks \fbib{do I wait}} \\
\poeml \v{4}Return, \divine{Lord}, \\
\poemll    save my life! \\
\poemlll       Deliver me, because of your gracious love. \\
\poeml \v{5}In death, there is no memory of you. \\
\poemll    Who will give you thanks where the dead are?\fnote{Lit. \fbib{thanks in Sheol}; a reference to the realm of the dead} \\
\poeml \v{6}I am weary from my groaning. \\
\poemll    Every night my couch is drenched with tears, \\
\poemlll       my bed is soaked through. \\
\poeml \v{7}My eyesight has faded because of grief, \\
\poemll    it has dimmed because of all my enemies. \\
\poeml \v{8}Get away from me, all of you who practice evil, \\
\poemll    for the \divine{Lord} has heard the sound of my weeping. \\
\poeml \v{9}The \divine{Lord} has heard my plea; \\
\poemll    the \divine{Lord} receives my prayer. \\
\poeml \v{10}As for all my enemies, they will be put to shame; \\
\poemll    they will be greatly frightened \\
\poemlll       and suddenly turn away ashamed.
\end{poetry}
\labelpsalm{7}
\psalminfo{A Davidic psalm,\fnote{T Heb. \fbib{Shiggaion}} which he sang to the \divine{Lord}, because of the words of Cush the descendant of Benjamin.}
\passage{A Prayer for Vindication}

\begin{poetry}
\poeml \v{1}\divine{Lord}, my God, \\
\poemll    I seek refuge in you. \\
\poeml Deliver me from those who persecute me! \\
\poemll    Rescue me! \\
\poeml \v{2}Otherwise, they will rip me to shreds like a lion, \\
\poemll    tearing me\fnote{The Heb. lacks \fbib{me}} apart with no one to rescue me.\fnote{The Heb. lacks \fbib{me}} \\
\poeml \v{3}\divine{Lord}, my God, if I have done this thing, \\
\poemll    if there is injustice on my hands, \\
\poeml \v{4}if I have rewarded those who did me good with evil, \\
\poemll    if I have plundered my enemy without justification, \\
\poeml \v{5}then, let my enemy pursue me, \\
\poemll    let him overtake me, \\
\poemlll       and let him trample my life to the ground.
\end{poetry}
\interlude{Interlude}

\begin{poetry}
\poeml Let him put my honor into the dust. \\
\poeml \v{6}Get up, \divine{Lord}, in your anger! \\
\poemll    Rise up, because of the fury of my enemies; \\
\poeml Arouse yourself for me; \\
\poemll    you have ordained justice. \\
\poeml \v{7}Let the assembly of the peoples gather around you, \\
\poemll    and you will sit\fnote{Lit. \fbib{return}} high above them. \\
\poeml \v{8}For the \divine{Lord} will judge the peoples. \\
\poemll    Judge me according to my righteousness, \divine{Lord}, \\
\poemlll       and according to my integrity, Exalted One. \\
\poeml \v{9}Let the evil of the wicked come to an end, \\
\poemll    but establish the righteous. \\
\poeml For you are the righteous God \\
\poemll    who discerns the inner thoughts.\fnote{Lit. \fbib{hearts and innards}} \\
\poeml \v{10}God is my shield,\fnote{Lit. \fbib{My shield is on God}} \\
\poemll    the one who delivers the upright in heart. \\
\poeml \v{11}God is a righteous judge, \\
\poemll    a God who is angry with sinners\fnote{The Heb. lacks \fbib{sinners}} every day. \\
\poeml \v{12}If the ungodly one\fnote{Lit. \fbib{If he}} doesn't repent, \\
\poemll    God will sharpen his sword; \\
\poemlll       he will string his bow and prepare it. \\
\poeml \v{13}He prepares weapons of death for himself, \\
\poemll    he makes his arrows into fiery shafts. \\
\poeml \v{14}But the wicked one\fnote{Lit. \fbib{But he}} travails with evil, \\
\poemll    he conceives malice and gives birth to lies. \\
\poeml \v{15}He digs a pit, even excavates it; \\
\poemll    then he fell into the hole that he had made. \\
\poeml \v{16}The trouble\fnote{Lit. \fbib{His trouble}} he planned will return on his own head, \\
\poemll    and his violence will descend on his skull. \\
\poeml \v{17}But as for me, \\
\poemll    I will praise the \divine{Lord} for his righteousness, \\
\poemlll       and I will sing to the name of the \divine{Lord} Most High.
\end{poetry}
\labelpsalm{8}
\psalminfo{To the Director: On a stringed instrument.\fnote{T Or \fbib{according to a Gittite melody}} A Davidic Psalm.}
\passage{Divine Glory and Human Dignity}

\begin{poetry}
\poeml \v{1}\divine{Lord}, our Lord, \\
\poemll    how excellent is your name in all the earth! \\
\poemlll       You set your glory above the heavens! \\
\poeml \v{2}Out of the mouths of infants and nursing babies \\
\poemll    you have established strength\fnote{LXX reads \fbib{praise}} \\
\poemlll       on account of your adversaries, \\
\poeml in order to silence the enemy and vengeful foe. \\
\poeml \v{3}When I look at the heavens, \\
\poemll    the work of your fingers, \\
\poemlll       the moon and the stars that you established--- \\
\poeml \v{4}what is man that you take notice of him, \\
\poemll    or the son of man\fnote{A title of Messiah (cf. Dan 7:13-14) or a Heb. synonym for a human being (cf. Dan 8:17)} that you pay attention to him? \\
\poeml \v{5}You made him a little less than divine,\fnote{Or \fbib{God}; or \fbib{gods}; or \fbib{than heavenly beings}} \\
\poemll    but you crowned him with glory and honor. \\
\poeml \v{6}You gave him dominion over the work of your hands, \\
\poemll    you put all things under his feet: \\
\poeml \v{7}Sheep and cattle---all of them, \\
\poemll    wild creatures of the field, \\
\poeml \v{8}birds in the sky, \\
\poemll    fish in the sea--- \\
\poemlll       whatever moves through the currents of the oceans. \\
\poeml \v{9}\divine{Lord}, our Lord, \\
\poemll    how excellent is your name in all the earth!
\end{poetry}
\labelpsalm{9}
\psalminfo{To the Director: Accompanied by female voices.\fnote{T Or \fbib{according to the tune `Death of a Son'}} A Davidic Psalm.}
\passage{A Cry for God's Justice}

\begin{poetry}
\poeml \v{1}\fnote{Psalms 9 \& 10 constitute a single psalm in the LXX.}I will give thanks to the \divine{Lord} with all my heart, \\
\poemll    I will declare all your wonderful deeds. \\
\poeml \v{2}I will be glad and exult in you; \\
\poemll    I will sing praises to your name, Most High! \\
\poeml \v{3}When my enemies turn back, \\
\poemll    they will stumble and perish before you. \\
\poeml \v{4}For you have brought about justice for me and my cause; \\
\poemll    you sit on the throne judging righteously. \\
\poeml \v{5}You rebuked the nations, \\
\poemll    you destroyed the wicked, \\
\poemlll       you wiped out their name forever and ever. \\
\poeml \v{6}The enemy has perished, \\
\poemll    reduced to ruins forever. \\
\poeml You uprooted their cities, \\
\poemll    the very memory of them vanished. \\
\poeml \v{7}But the \divine{Lord} sits on his throne\fnote{The Heb. lacks \fbib{on his throne}} forever; \\
\poemll    his throne is established for judgment. \\
\poeml \v{8}He will judge the world righteously \\
\poemll    and make just decisions for the people. \\
\poeml \v{9}The \divine{Lord} is a refuge for the oppressed, \\
\poemll    a refuge in times of distress. \\
\poeml \v{10}Those who know your name will trust you, \\
\poemll    for you have not forsaken those who seek you, \divine{Lord}. \\
\poeml \v{11}Sing praises to the \divine{Lord} who dwells in Zion; \\
\poemll    declare his mighty deeds among the peoples. \\
\poeml \v{12}As an avenger of blood, he remembers them; \\
\poemll    he has not forgotten the cry of the afflicted. \\
\poeml \v{13}Be gracious to me, \divine{Lord}, \\
\poemll    take note of my affliction, \\
\poemlll       because of those who hate me. \\
\poeml You snatch me away from the gates of death, \\
\poeml \v{14}so I may declare everything for which you should be praised\fnote{Lit. \fbib{declare all your praise}} \\
\poeml in the gates of the daughter of Zion,\fnote{I.e. Jerusalem} \\
\poemll    so I will rejoice in your deliverance. \\
\poeml \v{15}The nations have sunk down into the pit they made, \\
\poemll    their feet are ensnared in the trap\fnote{Lit. \fbib{net}} they set. \\
\poeml \v{16}The \divine{Lord} has made himself known, \\
\poemll    executing judgment. \\
\poeml The wicked are ensnared \\
\poemll    by what their hands have made.
\end{poetry}
\interlude{Interlude\fnote{Heb. \fbib{Higgaion Selah}}}

\begin{poetry}
\poeml \v{17}The wicked will turn back to where the dead are\fnote{Lit. \fbib{to Sheol}; a reference to the realm of the dead}--- \\
\poemll    all the nations that have forgotten God. \\
\poeml \v{18}For he will not always overlook the plight of the poor, \\
\poemll    nor will the hope of the afflicted perish forever. \\
\poeml \v{19}Rise up, \divine{Lord}, \\
\poemll    do not let man prevail! \\
\poemlll       The nations will be judged in your presence. \\
\poeml \v{20}Make them afraid, \divine{Lord}, \\
\poemll    Let the nations know that they are only human.\fnote{Or \fbib{men}}
\end{poetry}
\interlude{Interlude}
\labelpsalm{10}
\passage{A Prayer for Judging the Wicked}

\begin{poetry}
\poeml \v{1}\fnote{Psalms 9 \& 10 constitute a single psalm in the LXX.}Why do you stand far away, \divine{Lord}? \\
\poemll    Why do you hide in times of distress? \\
\poeml \v{2}The wicked one arrogantly pursues the afflicted,\fnote{Or \fbib{the poor}} \\
\poemll    who are trapped in the schemes he devises. \\
\poeml \v{3}For the wicked one boasts about his own desire; \\
\poemll    he blesses the greedy \\
\poemlll       and despises the \divine{Lord}. \\
\poeml \v{4}With haughty arrogance, the wicked thinks, \\
\poemll    ``God will not seek justice.''\fnote{The Heb. lacks \fbib{justice}} \\
\poemlll       He always presumes ``There is no God.'' \\
\poeml \v{5}Their ways always seem prosperous. \\
\poeml Your judgments are on high, \\
\poemlll       far away from them. \\
\poeml They scoff at all their enemies. \\
\poeml \v{6}They say to themselves, \\
\poemll    ``We will not be moved throughout all time, \\
\poemlll       and we will not experience adversity.'' \\
\poeml \v{7}Their mouth is full of curses, lies, and oppression, \\
\poemll    their tongues\fnote{Lit. \fbib{under his tongue}} spread trouble and iniquity. \\
\poeml \v{8}They wait\fnote{Lit. \fbib{sit}} in ambush in the villages, \\
\poemll    they kill the innocent in secret. \\
\poeml \v{9}Their eyes secretly watch the helpless, \\
\poemll    lying in wait like a lion in his den. \\
\poeml They lie in wait to catch the afflicted. \\
\poemll    They catch the afflicted when they pull him into their net. \\
\poeml \v{10}The victim\fnote{Lit. \fbib{He}} is crushed, \\
\poemll    and he sinks down; \\
\poemlll       the helpless fall by their might. \\
\poeml \v{11}The wicked say to themselves, \\
\poemll    ``God has forgotten, \\
\poeml he has hidden his face, \\
\poemll    he will never see it.'' \\
\poeml \v{12}Rise up, \divine{Lord}! \\
\poemll    Raise your hand, God. \\
\poemlll       Don't forget the afflicted! \\
\poeml \v{13}Why do the wicked despise God \\
\poemll    and say to themselves, ``God\fnote{Lit. \fbib{He}} will not seek justice.''?\fnote{The Heb. lacks \fbib{justice}} \\
\poeml \v{14}But you do see! \\
\poemll    You take note of trouble and grief \\
\poemlll       in order to take the matter into your own hand. \\
\poeml The helpless one commits himself\fnote{The Heb. lacks \fbib{himself}} to you; \\
\poemll    you have been the orphan's helper. \\
\poeml \v{15}Break the arm of the wicked and evil man; \\
\poemll    so that when you seek out his wickedness \\
\poemlll       you will find it no more. \\
\poeml \v{16}The \divine{Lord} is king forever and ever; \\
\poemll    nations will perish from his land. \\
\poeml \v{17}\divine{Lord}, you heard the desire of the afflicted; \\
\poemll    you will strengthen them,\fnote{Lit. \fbib{strengthen their heart}} \\
\poemlll       you will listen carefully, \\
\poeml \v{18}to do justice for the orphan\fnote{Or \fbib{fatherless}} and the oppressed, \\
\poemll    so that men of the earth may cause terror no more.
\end{poetry}
\labelpsalm{11}
\psalminfo{To the Director: A Davidic Song.\fnote{T The Heb. lacks \fbib{A song}}}
\passage{Confident Trust in God}

\begin{poetry}
\poeml \v{1}I take refuge in the \divine{Lord}. \\
\poemll    So how can you say to me, \\
\poemlll       ``Flee like a bird to the mountains.''? \\
\poeml \v{2}Look, the wicked have bent their bow \\
\poemll    and placed their arrow\fnote{So MT DSS 5/6HevPs; DSS 4QCatena A LXX read \fbib{arrows}} on the string,\fnote{So MT; LXX reads \fbib{arrows for the quiver}} \\
\poemlll       to shoot from the darkness\fnote{So MT DSS; LXX reads \fbib{shoot on a moonless night}} at the upright in heart. \\
\poeml \v{3}When the foundations are destroyed, \\
\poemll    what can the righteous do? \\
\poeml \v{4}The \divine{Lord} is in his holy Temple; \\
\poemll    the \divine{Lord}'s throne is in the heavens. \\
\poeml His eyes see, \\
\poemll    his glance\fnote{Lit. \fbib{eyelids}} examines humanity.\fnote{Lit. \fbib{examines the children of men}} \\
\poeml \v{5}The \divine{Lord} examines the righteous, \\
\poemll    but the wicked and those who love violence, he hates. \\
\poeml \v{6}He rains on the wicked burning coals and sulfur; \\
\poemll    a scorching wind is their destiny.\fnote{Lit. \fbib{the portion of their cup}} \\
\poeml \v{7}Indeed, the \divine{Lord} is righteous; \\
\poemll    he loves righteousness; \\
\poemlll       the upright will see him face-to-face.
\end{poetry}
\labelpsalm{12}
\psalminfo{To the Director: On an eight stringed harp.\fnote{T Or \fbib{on a lower octave}} A Davidic Psalm.}
\passage{Human and Divine Words Contrasted}

\begin{poetry}
\poeml \v{1}Help, \divine{Lord}, for godly people no longer exist; \\
\poemll    trustworthy people have disappeared from humanity.\fnote{Lit. \fbib{from among the children of men}} \\
\poeml \v{2}Everyone speaks lies to his neighbor; \\
\poemll    they speak with flattering lips and hidden motives.\fnote{Lit. \fbib{with slippery lips and a double heart}} \\
\poeml \v{3}The \divine{Lord} will cut off all slippery lips, \\
\poemll    and the tongue that boasts great things, \\
\poeml \v{4}those who say, \\
\poemll    ``By our tongues we will prevail; \\
\poemlll       our lips belong to us. \\
\poemll    Who is master\fnote{Or \fbib{lord}} over us?'' \\
\poeml \v{5}``Because the poor are being oppressed, \\
\poemll    because the needy are sighing, \\
\poemll    I will now arise,'' says the \divine{Lord}, \\
\poemlll       ``I will establish in safety those who yearn for it.'' \\
\poeml \v{6}The words of the \divine{Lord} are pure, \\
\poemll    like silver refined in an earthen furnace, \\
\poemlll       purified seven times over. \\
\poeml \v{7}You, \divine{Lord}, will keep them\fnote{So MT DSS 5/6HevPs 11QPs\textsuperscript{c}; LXX reads \fbib{us}} safe, \\
\poemll    you will guard them\fnote{So MT DSS 5/6HevPs 11QPs\textsuperscript{c}; LXX reads \fbib{us}} from this generation forever. \\
\poeml \v{8}The wicked, however,\fnote{The Heb. lacks \fbib{however}} keep walking around, \\
\poemll    exalting the vileness of human beings.\fnote{Lit. \fbib{of children of men}}
\end{poetry}
\labelpsalm{13}
\psalminfo{To the Director: A Davidic Psalm.}
\passage{A Prayer for Deliverance}

\begin{poetry}
\poeml \v{1}How long? \divine{Lord}, will you forget me forever?\fnote{Or \fbib{How long, \divine{Lord}, will you forget me? Forever?}} \\
\poemll    How long will you hide your face from me? \\
\poeml \v{2}How long must I struggle in my soul at night \\
\poemll    and have sorrow in my heart during the day? \\
\poemlll       How long will my enemy rise up against me? \\
\poeml \v{3}Look at me! \\
\poemll    Answer me, \divine{Lord}, my God! \\
\poeml Give light to my eyes! \\
\poemll    Otherwise, I will sleep in death; \\
\poeml \v{4}Otherwise, my enemy will say, \\
\poemll    ``I have overcome him;'' \\
\poeml Otherwise, my persecutor will rejoice \\
\poemll    when I am shaken. \\
\poeml \v{5}As for me, I have trusted in your gracious love, \\
\poemll    my heart will rejoice in your deliverance. \\
\poeml \v{6}I will sing to the \divine{Lord}, \\
\poemll    for he has dealt bountifully with me.
\end{poetry}
\labelpsalm{14}
\psalminfo{To the Director: A Davidic Psalm.}
\passage{The Fool and God's Response}

\begin{poetry}
\poeml \v{1}Fools say to themselves, ``There is no God.'' \\
\poemll    They are corrupt and commit evil deeds; \\
\poemlll       not one of them practices what is good. \\
\poeml \v{2}The \divine{Lord} looks down from the heavens upon humanity\fnote{Lit. \fbib{upon the sons of Adam}} \\
\poemll    to see if anyone shows discernment as he searches for God. \\
\poeml \v{3}All have turned away, \\
\poemll    together they have become corrupt; \\
\poemlll       no one practices what is good, not even one. \\
\poeml \v{4}Will those who do evil ever learn? \\
\poemll    They devour my people like they devour bread, \\
\poemlll       and never call on the \divine{Lord}. \\
\poeml \v{5}There they are seized with terror, \\
\poemll    because God is with those who are\fnote{Lit. \fbib{with the generation of the}} righteous. \\
\poeml \v{6}You would frustrate the plans of the oppressed,\fnote{Or \fbib{the poor}} \\
\poemll    but the \divine{Lord} is their refuge. \\
\poeml \v{7}May Israel's deliverance come from Zion! \\
\poemll    When the \divine{Lord} restores the fortunes of his people, \\
\poemlll       Jacob will rejoice, and Israel will be glad.\fnote{Cf. Ps 53:1-6}
\end{poetry}
\labelpsalm{15}
\psalminfo{A Davidic Psalm.}
\passage{Welcomed into God's Presence}

\begin{poetry}
\poeml \v{1}\divine{Lord}, who may stay in your tent? \\
\poemll    Who may dwell on your holy mountain? \\
\poeml \v{2}The one who lives with integrity, \\
\poemll    who does righteous deeds, \\
\poemlll       and who speaks truth to himself. \\
\poeml \v{3}The one who does not slander with his tongue, \\
\poemll    who does no evil to his neighbor, \\
\poemlll       and who does not destroy his friend's reputation. \\
\poeml \v{4}The one who despises those who are utterly wicked, \\
\poemll    but who honors the one who fears the \divine{Lord}, \\
\poeml who keeps his word even when it hurts and does not change, \\
\poeml \v{5}who does not loan his money with interest, \\
\poemlll       and who does not take a bribe against those who are innocent. \\
\poeml The one who does these things will stand firm\fnote{Lit. \fbib{won't be shaken}} forever.
\end{poetry}
\labelpsalm{16}
\psalminfo{A special Davidic Psalm.\fnote{T Heb. \fbib{miktam}}}
\passage{Trust in the Face of Death}

\begin{poetry}
\poeml \v{1}Keep me safe, God, \\
\poemll    for I take refuge in you. \\
\poeml \v{2}I told the \divine{Lord}, \\
\poemll    ``You are my master,\fnote{Heb. \fbib{Adonai}} \\
\poemlll       I have nothing good apart from you.'' \\
\poeml \v{3}As for the saints that are in the land, \\
\poemll    they are noble, and all my delight is in them. \\
\poeml \v{4}Those who hurry after another god\fnote{The Heb. lacks \fbib{god}} will have many sorrows; \\
\poemll    I will not present\fnote{Lit. \fbib{pour out}} their drink offerings of blood, \\
\poemlll       nor will my lips speak\fnote{Lit. \fbib{lift up on my lips}} their names. \\
\poeml \v{5}The \divine{Lord} is my inheritance and my cup; \\
\poemll    you support my lot. \\
\poeml \v{6}The boundary lines have fallen in pleasant places for me; \\
\poemll    truly, I have a beautiful heritage. \\
\poeml \v{7}I will bless the \divine{Lord} who has counseled me; \\
\poemll    indeed, my conscience instructs\fnote{Lit. \fbib{thoughts instruct}} me during the night. \\
\poeml \v{8}I have set the \divine{Lord} before me continuously; \\
\poemll    because he stands at my right hand, I will stand firm.\fnote{Lit. \fbib{not be shaken}} \\
\poeml \v{9}Therefore, my heart is glad, \\
\poemll    my whole being\fnote{Lit. \fbib{glory}} rejoices, \\
\poemlll       and my body will dwell securely. \\
\poeml \v{10}For you will not leave my soul in Sheol,\fnote{I.e. the realm of the dead} \\
\poemll    you will not allow your holy one to experience corruption.\fnote{Lit. \fbib{to see the Pit}; i.e. the realm of punishment in the afterlife} \\
\poeml \v{11}You cause me to know the path of life; \\
\poemll    in your presence is joyful abundance, \\
\poemlll       at your right hand there are pleasures forever.
\end{poetry}
\labelpsalm{17}
\psalminfo{A Davidic Prayer.}
\passage{A Cry for Justice}

\begin{poetry}
\poeml \v{1}\divine{Lord}, hear my just plea! \\
\poemll    Pay attention to my cry! \\
\poeml Listen to my prayer, \\
\poemll    since it does not come from lying lips. \\
\poeml \v{2}Justice for me will come from your presence; \\
\poemll    your eyes see what is right. \\
\poeml \v{3}When you probe my heart, \\
\poemll    and examine me at night; \\
\poeml when you refine me, \\
\poemll    you will find nothing wrong,\fnote{The Heb. lacks \fbib{wrong}} \\
\poemlll       for I have determined that I will not transgress with my mouth. \\
\poeml \v{4}As for the ways of mankind, \\
\poemll    I have, according to the words of your lips, \\
\poemlll       avoided the ways of the violent. \\
\poeml \v{5}Because my steps have held fast to your paths, \\
\poemll    my footsteps have not faltered. \\
\poeml \v{6}I call upon you, for you will answer me, God. \\
\poemll    Listen closely to me \\
\poemlll       and hear my prayer. \\
\poeml \v{7}Show forth your gracious love, \\
\poemll    save those who take refuge in you \\
\poemlll       from those who rebel against your sovereign power.\fnote{Lit. \fbib{against your right hand}} \\
\poeml \v{8}Protect me as the most precious part of the eye;\fnote{Lit. \fbib{as the pupil of the daughter of the eye}} \\
\poemll    hide me under the shadow of your wings \\
\poeml \v{9}from the wicked\fnote{Lit. \fbib{face of the wicked}} who have afflicted me, \\
\poemll    from my enemies who have surrounded me. \\
\poeml \v{10}They are imprisoned by their own prosperity,\fnote{Lit. \fbib{fat}} \\
\poemll    they have boasted proudly with their mouth. \\
\poeml \v{11}Now they have encircled our paths\fnote{So MT; DSS 11QPs\textsuperscript{c} LXX read \fbib{have expelled me}} \\
\poemll    and are determined\fnote{Lit. \fbib{and have set their eyes}} to cast us down to the ground. \\
\poeml \v{12}Like a lion they desire to rip us to pieces, \\
\poemll    like a young lion waiting in ambush. \\
\poeml \v{13}Arise, \divine{Lord}, \\
\poemll    confront them, \\
\poemlll       bring them to their knees! \\
\poeml Deliver me from the wicked by your sword--- \\
\poeml \v{14}from men, \divine{Lord}, by your hand--- \\
\poeml from men who belong to this world, \\
\poemll    whose reward is only\fnote{The Heb. lacks \fbib{only}} in this\fnote{So MT; DSS 11QPs\textsuperscript{c} LXX read \fbib{their}} life. \\
\poeml But as for your treasured ones, \\
\poemll    may their stomachs be full, \\
\poeml may their children have an abundance, \\
\poemll    and may they leave wealth to their offspring. \\
\poeml \v{15}But as for me, justified, I will behold your face; \\
\poemll    when I awake, your presence\fnote{Lit. \fbib{form, likeness}} will satisfy me.
\end{poetry}
\labelpsalm{18}
\psalminfo{To the Director: By the servant of the \divine{Lord}, David, who spoke the words of this song to the \divine{Lord} on the day when the \divine{Lord} delivered him from the hands of all his enemies and from the hand of Saul.}
\passage{Gratitude for Victory}

\begin{poetry}
\poeml \v{1}He said: \\
\poemll    ``I love you, Lord, my strength. \\
\poeml \v{2}The \divine{Lord} is my rock, my fortress, my deliverer, my God, \\
\poemll    my stronghold\fnote{Lit. \fbib{rock}} in whom I take refuge, my shield, the glory\fnote{Lit. \fbib{horn}} \\
\poemlll       of my salvation, and my high tower.'' \\
\poeml \v{3}I cried out to the \divine{Lord}, who is worthy to be praised, \\
\poemll    and I was delivered from my enemies. \\
\poeml \v{4}The cords of death entangled me; \\
\poemll    the rivers of Belial\fnote{I.e. the forces of death and destruction} made me afraid. \\
\poeml \v{5}The cords of Sheol\fnote{I.e. the realm of the dead} surrounded me; \\
\poemll    the snares of death confronted me. \\
\poeml \v{6}In my distress I cried to the \divine{Lord}; \\
\poemll    to my God I cried for help. \\
\poeml From his Temple he heard my voice; \\
\poemll    my cry reached his ears. \\
\poeml \v{7}The world shook and trembled; \\
\poemll    the foundations of the mountains quaked, \\
\poemlll       they shook because he was angry. \\
\poeml \v{8}In his anger smoke poured out of his nostrils, \\
\poemll    and consuming fire from his mouth; \\
\poemlll       coals were lit from it. \\
\poeml \v{9}He bent the sky and descended, \\
\poemll    and darkness was under his feet. \\
\poeml \v{10}He rode upon a cherub and flew; \\
\poemll    he soared upon the wings of the wind. \\
\poeml \v{11}He made darkness his hiding place, \\
\poemll    his canopy surrounding him was dark waters and thick clouds. \\
\poeml \v{12}The brightness before him scattered the thick clouds, \\
\poemll    with hail stones and flashes of fire. \\
\poeml \v{13}Then the \divine{Lord} thundered in\fnote{So MT DSS 4QPs\textsuperscript{c}; LXX Targ Vg (cf. 2 Sam 22:14) reads \fbib{from}} the heavens, \\
\poemll    and the Most High sounded aloud, \\
\poemlll       calling for hail stones and flashes of fire.\fnote{So MT DSS 4QPs\textsuperscript{c}; LXX lacks \fbib{calling for hail stones and flashes of fire}} \\
\poeml \v{14}He shot his arrows and scattered them; \\
\poemll    with many lightning bolts he frightened them. \\
\poeml \v{15}Then the channels of the sea could be seen, \\
\poemll    and the foundations of the earth were uncovered \\
\poeml because of your rebuke, \divine{Lord}, \\
\poemll    because of the blast from the breath of your nostrils. \\
\poeml \v{16}He reached down and took me; \\
\poemll    he drew me from many waters. \\
\poeml \v{17}He delivered me from my strong enemies, \\
\poemll    from those who hated me because \\
\poemlll       they were stronger than I. \\
\poeml \v{18}They confronted me in the day of my calamity, \\
\poemll    but the \divine{Lord} was my support. \\
\poeml \v{19}He brought me out to a spacious place; \\
\poemll    he delivered me, for in me he takes delight.
\passage{God's Reward to the Righteous}
\poeml \v{20}The \divine{Lord} will reward me because I am righteous; \\
\poemll    because my hands are clean he will restore me; \\
\poeml \v{21}because I have kept the ways of the \divine{Lord}, \\
\poemll    and I have not wickedly departed from my God; \\
\poeml \v{22}because all his judgments were always before me, \\
\poemll    and I did not cast off his statutes. \\
\poeml \v{23}I was upright\fnote{Or \fbib{blameless}} before him, \\
\poemll    and I kept myself from iniquity. \\
\poeml \v{24}So the \divine{Lord} restored me according to my righteousness, \\
\poemll    because my hands were clean in his sight. \\
\poeml \v{25}To the holy, you show your gracious love, \\
\poemll    to the upright, you show yourself upright; \\
\poeml \v{26}to the pure, you show yourself pure, \\
\poemll    and to the morally corrupt, you appear to be perverse. \\
\poeml \v{27}Indeed, you deliver the oppressed,\fnote{Or \fbib{poor}} \\
\poemll    but you bring down those who exalt themselves \\
\poemlll       in their own eyes. \\
\poeml \v{28}For you, \divine{Lord}, make my lamp shine; \\
\poemll    my God enlightens my darkness. \\
\poeml \v{29}With your help\fnote{The Heb. lacks \fbib{help}} I will run through an army, \\
\poemll    with help from\fnote{The Heb. lacks \fbib{help from}} my God I leap over walls. \\
\poeml \v{30}As for God, his way is upright;\fnote{Or \fbib{blameless}} \\
\poemll    the word of God is pure; \\
\poemlll       he is a shield to all those who take refuge in him.
\passage{The Acts of God for the Righteous}
\poeml \v{31}For who is God but the \divine{Lord}, \\
\poemll    and who is a Rock other than our God?--- \\
\poeml \v{32}the God who clothes me with strength, \\
\poemll    and who makes my way upright;\fnote{Or \fbib{blameless}} \\
\poeml \v{33}who makes my feet swift as the deer; \\
\poemll    who makes me stand on high places; \\
\poeml \v{34}who teaches my hands to make war, \\
\poemll    and my arms to bend a bronze bow. \\
\poeml \v{35}You have given to me the shield of your deliverance, \\
\poemll    and your right hand holds me up; \\
\poemlll       your gentleness made me great. \\
\poeml \v{36}You make a broad place for my steps, \\
\poemll    so my feet\fnote{Lit. \fbib{ankle}} won't slip. \\
\poeml \v{37}I pursued my enemies and overtook them; \\
\poemll    I did not turn around until they were utterly defeated. \\
\poeml \v{38}I struck them down, \\
\poemll    so they are not able to rise up; \\
\poemlll       they fell under my feet. \\
\poeml \v{39}You clothed me with strength for war; \\
\poemll    you will subdue under me those who rise up against me. \\
\poeml \v{40}You have made my enemies turn their back to me, \\
\poemll    and I will destroy those who hate me. \\
\poeml \v{41}They cried out for deliverance, \\
\poemll    but there was no one to deliver; \\
\poeml they cried out\fnote{The Heb. lacks \fbib{they cried out}} to the \divine{Lord}, \\
\poemll    but he did not answer them. \\
\poeml \v{42}I ground them like wind-swept dust; \\
\poemll    I emptied them out\fnote{So MT DSS 5/6HevPs; LXX reads \fbib{I will grind them down}} like dirt in the street. \\
\poeml \v{43}You rescued me from conflict with the people; \\
\poemll    you made me head of the nations. \\
\poemlll       People who did not know me will serve me. \\
\poeml \v{44}When they hear of me,\fnote{Lit. \fbib{At the hearing of the ear}} they will obey me; \\
\poemll    foreigners will submit to me. \\
\poeml \v{45}Foreigners will wilt away; \\
\poemll    they will come trembling out of their stronghold. \\
\poeml \v{46}The \divine{Lord} lives! \\
\poemll    Blessed be my Rock! \\
\poemlll       May the God of my deliverance be exalted! \\
\poeml \v{47}He is the God who executes vengeance on my behalf; \\
\poemll    who destroys peoples under me; \\
\poeml \v{48}who delivers me from my enemies. \\
\poemll    Truly you will exalt me above those who oppose me; \\
\poemlll       you will deliver me from the violent person. \\
\poeml \v{49}Therefore, I will give thanks to you among the nations, \divine{Lord}; \\
\poemll    I will sing praises to your name. \\
\poeml \v{50}He is the one who gives victories to his king; \\
\poemll    who shows gracious love to his anointed, \\
\poemlll       to David and his seed forever.
\end{poetry}
\labelpsalm{19}
\psalminfo{To the Director: A Davidic Psalm.}
\passage{God's Revelation in the Heavens}

\begin{poetry}
\poeml \v{1}The heavens are declaring the glory of God, \\
\poemll    and their expanse shows the work of his hands. \\
\poeml \v{2}Day after day they pour forth speech, \\
\poemll    night after night they reveal knowledge. \\
\poeml \v{3}There is no speech nor are there words--- \\
\poemll    their voice is not heard--- \\
\poeml \v{4}yet their message\fnote{Or \fbib{sound}; so LXX; MT DSS 11QPs\textsuperscript{c} Syr read \fbib{line}} goes out into all the world, \\
\poemll    and their words to the ends of the earth. \\
\poeml He has set up a tent for the sun in the heavens,\fnote{Lit. \fbib{them}} \\
\poeml \v{5}which is like a bridegroom coming out of his chamber, \\
\poemlll       or like a champion who rejoices at the beginning of a race. \\
\poeml \v{6}Its circuit is from one end of the sky to the other, \\
\poemll    and nothing is hidden from its heat.
\passage{God's Revelation in the Law}
\poeml \v{7}The Law of the \divine{Lord} is perfect, \\
\poemll    restoring life. \\
\poeml The testimony of the \divine{Lord} is steadfast, \\
\poemll    making foolish people wise. \\
\poeml \v{8}The precepts of the \divine{Lord} are upright, \\
\poemll    making the heart rejoice. \\
\poeml The commandment of the \divine{Lord} is pure, \\
\poemll    giving light to the eyes. \\
\poeml \v{9}The fear of the \divine{Lord} is clean, \\
\poemll    standing forever. \\
\poeml The judgments of the \divine{Lord} are true; \\
\poemll    they are altogether righteous. \\
\poeml \v{10}They are more desirable than gold, \\
\poemll    even much fine gold. \\
\poeml They are sweeter than honey, \\
\poemll    even the drippings from a honeycomb. \\
\poeml \v{11}Moreover your servant is warned by them; \\
\poemll    and there is great reward in keeping them. \\
\poeml \v{12}Who can detect his own\fnote{The Heb. lacks \fbib{his own}} mistake? \\
\poemll    Cleanse me from hidden sin. \\
\poeml \v{13}Preserve your servant from arrogant people;\fnote{Or \fbib{from presumptuous sins}} \\
\poemll    do not let them rule over me. \\
\poeml Then I will be upright\fnote{Or \fbib{perfect}, or \fbib{blameless}} \\
\poemll    and acquitted of great wickedness. \\
\poeml \v{14}May the words of my mouth and the meditations of my heart \\
\poemll    be acceptable in your sight, \divine{Lord}, my Rock and my Redeemer.
\end{poetry}
\labelpsalm{20}
\psalminfo{To the Director: A Davidic Psalm.}
\passage{A Prayer for Victory}

\begin{poetry}
\poeml \v{1}May the \divine{Lord} answer you in the day of distress; \\
\poemll    may the name of the God of Jacob\fnote{I.e. Israel} protect you. \\
\poeml \v{2}May he send you help from the sanctuary, \\
\poemll    and may he sustain you from Zion. \\
\poeml \v{3}May he remember all your gifts, \\
\poemll    and may he accept your burnt offerings.
\end{poetry}
\interlude{Interlude}

\begin{poetry}
\poeml \v{4}May he give you what your heart desires, \\
\poemll    and may he fulfill all your plans. \\
\poeml \v{5}May we shout for joy at your deliverance \\
\poemll    and unfurl our banners in the name of our God. \\
\poemlll       May the \divine{Lord} fulfill all your petitions. \\
\poeml \v{6}Now I know that the \divine{Lord} has delivered his anointed; \\
\poemll    he has answered him from his sanctuary \\
\poemlll       with the strength of his right hand of deliverance. \\
\poeml \v{7}Some boast\fnote{The Heb. lacks \fbib{Some boast}} in chariots, \\
\poemll    others in horses; \\
\poemlll       but we will boast in\fnote{Or \fbib{remember}} the name of the \divine{Lord} our God. \\
\poeml \v{8}While they bowed down and fell, \\
\poemll    we arose and stood upright. \\
\poeml \v{9}Deliver us, \divine{Lord}! \\
\poemll    Answer us, our King,\fnote{I.e. God} on the day we cry out!
\end{poetry}
\labelpsalm{21}
\psalminfo{To the Director: A Davidic Psalm.}
\passage{Praise for the \divine{Lord}'s Deliverance}

\begin{poetry}
\poeml \v{1}The king rejoices in your strength, \divine{Lord}. \\
\poemll    How greatly he rejoices in your deliverance. \\
\poeml \v{2}You have granted him the desire of his heart, \\
\poemll    and have not withheld what his lips requested.
\end{poetry}
\interlude{Interlude}

\begin{poetry}
\poeml \v{3}You go before him with wonderful blessings, \\
\poemll    and put a crown of fine gold on his head. \\
\poeml \v{4}He asked life from you, and you gave it to him--- \\
\poemll    a long life for ever and ever. \\
\poeml \v{5}His glory is great because of your deliverance, \\
\poemll    you have given him honor and majesty. \\
\poeml \v{6}Indeed, you have given him eternal blessings; \\
\poemll    you will make him glad with the joy of your presence. \\
\poeml \v{7}The king trusts in the \divine{Lord}; \\
\poemll    because of the gracious love of the Most High, \\
\poemlll       he will stand firm.\fnote{Lit. \fbib{will not be shaken}} \\
\poeml \v{8}Your hand will find all your enemies, \\
\poemll    your right hand will find those who hate you. \\
\poeml \v{9}When you appear, \\
\poemll    you will set them ablaze like a fire furnace. \\
\poeml In his wrath, the \divine{Lord} will consume them, \\
\poemll    and the fire will devour them. \\
\poeml \v{10}You will destroy their descendants\fnote{Lit. \fbib{his fruit}} from the earth, \\
\poemll    even their offspring from the ranks\fnote{Lit. \fbib{children}} of mankind. \\
\poeml \v{11}Though they plot evil against you and devise schemes, \\
\poemll    they will not succeed. \\
\poeml \v{12}Indeed, you will make them retreat,\fnote{Lit. \fbib{will turn the shoulder}} \\
\poemll    when you aim your bow\fnote{Lit. \fbib{when your bow string is ready}} at their faces. \\
\poeml \v{13}Rise up, \divine{Lord}, because you are strong; \\
\poemll    we will sing and praise your power.
\end{poetry}
\labelpsalm{22}
\psalminfo{To the Director: To the tune of\fnote{T Lit. \fbib{According to}} ``Doe of the Dawn''.\\ A Davidic Psalm.}
\passage{God Delivers His Suffering Servant}

\begin{poetry}
\poeml \v{1}My God! My God! \\
\poemll    Why have you abandoned me? \\
\poeml Why are you so far from delivering me--- \\
\poemll    from my groaning words? \\
\poeml \v{2}My God, I cry out to you throughout the day, \\
\poemll    but you do not answer; \\
\poeml and throughout the night, \\
\poemll    but I have no rest.\fnote{Lit. \fbib{but there is no silence for me}} \\
\poeml \v{3}You are holy, \\
\poemll    enthroned on the praises of Israel. \\
\poeml \v{4}Our ancestors trusted in you; \\
\poemll    they trusted and you delivered them. \\
\poeml \v{5}They cried out to you and escaped; \\
\poemll    they trusted in you and were not put to shame. \\
\poeml \v{6}But as for me, \\
\poemll    I am only a worm and not a man, \\
\poemlll       scorned by mankind and despised by people. \\
\poeml \v{7}Everyone who sees me mocks me; \\
\poemll    they gape at me with open mouths \\
\poemlll       and shake their heads at me. \\
\poeml \v{8}They say,\fnote{The Heb. lacks \fbib{They say}} ``Commit yourself to the \divine{Lord}; \\
\poemll    perhaps the \divine{Lord}\fnote{Lit. \fbib{he}} will deliver him, \\
\poeml perhaps he will cause him to escape, \\
\poemll    since he delights in him.'' \\
\poeml \v{9}Yet, you are the one who took me from the womb, \\
\poemll    and kept me safe on my mother's breasts. \\
\poeml \v{10}I was dependent on you from birth; \\
\poemll    from my mother's womb you have been my God. \\
\poeml \v{11}Do not be so distant from me, \\
\poemll    for trouble is at hand; \\
\poemlll       indeed, there is no deliverer. \\
\poeml \v{12}Many bulls have surrounded me; \\
\poemll    the vicious bulls of Bashan have encircled me. \\
\poeml \v{13}Their mouths are opened wide toward me, \\
\poemll    like roaring and attacking lions. \\
\poeml \v{14}I am poured out like water; \\
\poemll    all my bones are out of joint. \\
\poemlll       My heart is like wax, melting within me. \\
\poeml \v{15}My strength is dried up like broken pottery; \\
\poemll    my tongue sticks to the roof of my mouth,\fnote{Lit. \fbib{to my jaws}} \\
\poemlll       and you have brought me down to the dust of death. \\
\poeml \v{16}For dogs have surrounded me; \\
\poemll    a gang of those who practice of evil has encircled me. \\
\poemlll       They gouged\fnote{So LXX Syr DSS 5/6 HevPS XHev/Se4; MT reads \fbib{Like a lion}} my hands and my\fnote{So MT; LXX lacks \fbib{my}} feet. \\
\poeml \v{17}I can count all my bones. \\
\poemll    They look at me; \\
\poemlll       they stare at me. \\
\poeml \v{18}They divide my clothing among themselves; \\
\poemll    they cast lots for my clothing! \\
\poeml \v{19}But as for you, \divine{Lord}, do not be far away from me; \\
\poemll    My Strength, come quickly to help me. \\
\poeml \v{20}Deliver me from the sword; \\
\poemll    my precious life from the power of the dog. \\
\poeml \v{21}Deliver me from the mouth of the lion, \\
\poemll    from the horns of the wild oxen. \\
\poeml You have answered me. \\
\poeml \v{22}I will declare your name to my brothers; \\
\poemll    in the midst of the congregation, I will praise you, saying,\fnote{The Heb. lacks \fbib{saying}} \\
\poeml \v{23}``All who fear the \divine{Lord}, praise him! \\
\poemll    All the seed of Jacob, glorify him! \\
\poeml All the seed of Israel, fear him! \\
\poeml \v{24}For he does not despise nor detest the afflicted person; \\
\poeml he does not hide his face from him, \\
\poemll    but he hears him when he cries out to him.'' \\
\poeml \v{25}My praise in the great congregation is because of you; \\
\poemll    I will pay my vows before those who fear you.\fnote{Lit. \fbib{him}} \\
\poeml \v{26}The afflicted will eat and be satisfied; \\
\poemll    those who seek the \divine{Lord} will praise him, \\
\poemlll       ``May you\fnote{Lit. \fbib{your heart}} live forever!'' \\
\poeml \v{27}All the ends of the earth will remember and turn to the \divine{Lord}; \\
\poemll    all the families of the nations will bow in submission to the \divine{Lord}. \\
\poeml \v{28}Indeed, the kingdom belongs to the \divine{Lord}; \\
\poemll    he rules over the nations. \\
\poeml \v{29}All the prosperous people will eat and bow down in submission. \\
\poemll    All those who are about to go down to the grave\fnote{Lit. \fbib{dust}} \\
\poemlll       will bow down in submission, \\
\poemll    along with the one who can no longer keep himself alive. \\
\poeml \v{30}Our\fnote{The Heb. lacks \fbib{our}} descendants will serve him, \\
\poemll    and that generation will be told about the Lord. \\
\poeml \v{31}They will come and declare his righteousness \\
\poemll    to a people yet to be born; \\
\poemlll       indeed, he has accomplished it!
\end{poetry}
\labelpsalm{23}
\psalminfo{A Davidic Psalm.}
\passage{The \divine{Lord} Shepherds His People}

\begin{poetry}
\poeml \v{1}The \divine{Lord} is the one who is shepherding me; \\
\poemll    I lack nothing. \\
\poeml \v{2}He causes me to lie down in pastures of green grass; \\
\poemll    he guides me beside quiet waters. \\
\poeml \v{3}He revives my life; \\
\poemll    he leads me in pathways that are righteous \\
\poemlll       for the sake of his name.\fnote{I.e. his reputation} \\
\poeml \v{4}Even when I walk through a valley of deep darkness,\fnote{Or \fbib{valley of the shadow of death}} \\
\poemll    I will not be afraid \\
\poemlll       because you are with me. \\
\poeml Your rod and your staff---they comfort me. \\
\poeml \v{5}You prepare a table before me, \\
\poemll    even in the presence of my enemies. \\
\poeml You anoint my head with oil; \\
\poemll    my cup overflows. \\
\poeml \v{6}Truly, goodness and gracious love will pursue me \\
\poemll    all the days of my life, \\
\poemlll       and I will remain in\fnote{MT DSS 5/6HevPs read \fbib{will return to}; LXX reads \fbib{and my residing will be}} the \divine{Lord}'s Temple forever.\fnote{Lit. \fbib{for the length of days}}
\end{poetry}
\labelpsalm{24}
\psalminfo{A Davidic Psalm.}
\passage{A Song for the King of Glory}

\begin{poetry}
\poeml \v{1}The earth and everything in it exists for the \divine{Lord}--- \\
\poemll    the world and those who live in it. \\
\poeml \v{2}Indeed, he founded it upon the seas, \\
\poemll    he established it upon deep waters.\fnote{Lit. \fbib{rivers}; i.e. the subterranean waters} \\
\poeml \v{3}Who may ascend the mountain of the \divine{Lord}?\fnote{I.e. the temple mount} \\
\poemll    Who may stand in his Holy Place? \\
\poeml \v{4}The one who has innocent hands and a pure heart; \\
\poemll    the person who does not delight in what is false \\
\poemlll       and does not swear an oath deceitfully. \\
\poeml \v{5}This person\fnote{Lit. \fbib{he}} will receive blessing from the \divine{Lord} \\
\poemll    and righteousness from the God of his salvation. \\
\poeml \v{6}This is the generation that seeks him. \\
\poemll    Those who seek your face \\
\poemlll       are the true seed of\fnote{The Heb. lacks \fbib{the true seed of}} Jacob.
\end{poetry}
\interlude{Interlude}

\begin{poetry}
\poeml \v{7}Lift up your heads,\fnote{I.e. \fbib{Open}} gates! \\
\poemll    Be lifted up, ancient doors, \\
\poemlll       so the King of Glory may come in. \\
\poeml \v{8}Who is the King of Glory? \\
\poemll    The \divine{Lord} strong and mighty, \\
\poemlll       the \divine{Lord}, mighty in battle. \\
\poeml \v{9}Lift up your heads,\fnote{I.e. \fbib{Open}} gates! \\
\poemll    Be lifted up, ancient doors, \\
\poemlll       so the King of Glory may come in. \\
\poeml \v{10}Who is he, this King of Glory? \\
\poemll    The \divine{Lord} of the heavenly armies--- \\
\poemlll       He is the King of Glory.
\end{poetry}
\interlude{Interlude}
\labelpsalm{25}
\psalminfo{Davidic\fnote{T This acrostic psalm begins each verse with a consecutive letter of the Hebrew alphabet}}
\passage{A Prayer for Help and Forgiveness}

\begin{poetry}
\poeml \v{1}I will lift up my soul to you, \divine{Lord}. \\
\poeml \v{2}I trust in you, my God, \\
\poemll    do not let me be ashamed; \\
\poemlll       do not let my enemies triumph over me. \\
\poeml \v{3}Indeed, no one who waits on you will be ashamed, \\
\poemll    but those who offend for no reason will be put to shame. \\
\poeml \v{4}Cause me to understand your ways, \divine{Lord}; \\
\poemll    teach me your paths. \\
\poeml \v{5}Guide me in your truth and teach me; \\
\poemll    for you are the God who delivers me. \\
\poemlll       All day long I have waited for you. \\
\poeml \v{6}Remember, \divine{Lord}, your tender mercies and your gracious love; \\
\poemll    indeed, they are eternal! \\
\poeml \v{7}Do not remember my youthful sins and transgressions; \\
\poemll    but remember me in light of your gracious love, \\
\poemlll       in light of your goodness, \divine{Lord}. \\
\poeml \v{8}The \divine{Lord} is good and just; \\
\poemll    therefore he will teach sinners concerning the way. \\
\poeml \v{9}He will guide the humble\fnote{Or \fbib{afflicted}} to justice; \\
\poemll    he will teach the humble\fnote{Or \fbib{afflicted}} his way. \\
\poeml \v{10}All the paths of the \divine{Lord} lead to gracious love and truth \\
\poemll    for those who keep his covenant and his decrees.\fnote{Or \fbib{testimonies}} \\
\poeml \v{11}For the sake of your name,\fnote{I.e. reputation} \divine{Lord}, \\
\poemll    forgive my sin, for it is great. \\
\poeml \v{12}Who is the man who fears the \divine{Lord}? \\
\poemll    God\fnote{Lit. \fbib{He}} will teach him the path he should choose. \\
\poeml \v{13}He\fnote{Lit. \fbib{His soul}} will experience good things; \\
\poemll    his descendants will inherit the earth. \\
\poeml \v{14}The intimate counsel of the \divine{Lord} is for those who fear him \\
\poemll    so they may know his covenant. \\
\poeml \v{15}My eyes look to the \divine{Lord} continuously, \\
\poemll    because he's the one who releases my feet from the trap.\fnote{Lit. \fbib{net}} \\
\poeml \v{16}Turn toward me and have mercy on me, \\
\poemll    for I am lonely and oppressed. \\
\poeml \v{17}The troubles of my heart have increased; \\
\poemll    bring me out of my distress! \\
\poeml \v{18}Look upon my distress and affliction; \\
\poemll    forgive all my sins. \\
\poeml \v{19}Look how many enemies I have gained! \\
\poemll    They hate me with a vicious hatred. \\
\poeml \v{20}Preserve my life and deliver me; \\
\poemll    do not let me be ashamed, \\
\poemlll       because I take refuge in you. \\
\poeml \v{21}Integrity and justice will preserve me, \\
\poemll    because I wait on you. \\
\poeml \v{22}Redeem Israel, God, from all its troubles.
\end{poetry}
\labelpsalm{26}
\psalminfo{Davidic}
\passage{A Man of Integrity Pleads for Justice}

\begin{poetry}
\poeml \v{1}Vindicate me, \divine{Lord}, \\
\poemll    because I have walked in integrity; \\
\poemlll       I have trusted in the \divine{Lord} without wavering. \\
\poeml \v{2}Examine me, \divine{Lord}, and inspect me! \\
\poemll    Test my heart and mind.\fnote{Lit. \fbib{kidneys}; i.e. the center of emotions} \\
\poeml \v{3}For your gracious love precedes me, \\
\poemll    and I continuously walk according to your truth. \\
\poeml \v{4}I do not sit with those committed to what is false, \\
\poemll    nor do I travel with hypocrites. \\
\poeml \v{5}I hate the company of those who practice evil, \\
\poemll    nor do I sit with the wicked. \\
\poeml \v{6}I wash my hands innocently. \\
\poemll    I go around your altar, \divine{Lord}, \\
\poeml \v{7}so I may praise you loudly with thanksgiving \\
\poemll    and declare all your wondrous acts. \\
\poeml \v{8}\divine{Lord}, I love the dwelling place that is your house, \\
\poemll    the place where your glory resides. \\
\poeml \v{9}Do not group me\fnote{Lit. \fbib{my soul}} with sinners, \\
\poemll    nor include me\fnote{The Heb. lacks \fbib{include me}} with men who shed blood. \\
\poeml \v{10}Their hands are filled with wicked schemes, \\
\poemll    and their right hands with bribes. \\
\poeml \v{11}But as for me, \\
\poemll    I walk in my integrity. \\
\poemlll       Redeem me and be gracious to me! \\
\poeml \v{12}My feet stand on level ground; \\
\poemll    among the worshiping congregations \\
\poemlll       I will bless the \divine{Lord}.
\end{poetry}
\labelpsalm{27}
\psalminfo{Davidic}
\passage{Confidence in the \divine{Lord}}

\begin{poetry}
\poeml \v{1}The \divine{Lord} is my light and my salvation--- \\
\poemll    whom will I fear? \\
\poeml The \divine{Lord} is the strength of my life; \\
\poemll    of whom will I be afraid? \\
\poeml \v{2}When those who practice evil, my enemies, and my oppressors \\
\poemll    come near me to devour my flesh, \\
\poemlll       they stumble and fall. \\
\poeml \v{3}If an army encamps against me, \\
\poemll    my heart will not fear. \\
\poeml If a war is launched against me, \\
\poemll    I will even trust in that situation. \\
\poeml \v{4}I have asked one thing from the \divine{Lord}; \\
\poemll    it is what I really seek: \\
\poeml that I may remain in the \divine{Lord}'s Temple \\
\poemll    all the days of my life, \\
\poeml to gaze on the beauty of the \divine{Lord}; \\
\poemll    and to inquire in his Temple. \\
\poeml \v{5}For he will conceal me in his shelter on the day of evil; \\
\poemll    He will hide me in a secluded chamber within his tent; \\
\poemlll       He will place me on a high rock. \\
\poeml \v{6}Now my head will be lifted up above my enemies, \\
\poemll    even those who surround me. \\
\poeml I will sacrifice in his tent with shouts of joy; \\
\poemll    I will sing and make melodies to the \divine{Lord}. \\
\poeml \v{7}Hear my voice, \divine{Lord}, when I cry out! \\
\poemll    Be gracious to me and answer me. \\
\poeml \v{8}My mind recalls your word,\fnote{The Heb. lacks \fbib{your word}} \\
\poemll    ``Seek my face,'' \\
\poemlll       so your face, \divine{Lord}, I will seek. \\
\poeml \v{9}Do not hide your face from me; \\
\poemll    do not turn away in anger from your servant. \\
\poeml You have been my help, \\
\poemll    therefore do not abandon or forsake me, \\
\poemlll       God of my salvation. \\
\poeml \v{10}Though my father and my mother abandoned me, \\
\poemll    the \divine{Lord} gathers me up. \\
\poeml \v{11}Teach me your way, \divine{Lord}, \\
\poemll    and lead me on a level path because of my enemies. \\
\poeml \v{12}Do not hand me over to the desires of my enemies; \\
\poemll    for false witnesses have risen up against me; \\
\poemlll       even the one who breathes out violence. \\
\poeml \v{13}I believe that I will see the \divine{Lord}'s goodness \\
\poemll    in the land of the living. \\
\poeml \v{14}Wait on the \divine{Lord}. \\
\poemll    Be courageous, and he will strengthen your heart. \\
\poemlll       Wait on the \divine{Lord}!
\end{poetry}
\labelpsalm{28}
\psalminfo{Davidic}
\passage{A Prayer for Help}

\begin{poetry}
\poeml \v{1}To you, \divine{Lord}, I cry out! \\
\poemll    My Rock, do not refuse to answer me.\fnote{Lit. \fbib{do not be silent to me}} \\
\poeml If you remain silent, \\
\poemll    I will become like those who descend into the Pit.\fnote{I.e. the place of punishment in the afterlife} \\
\poeml \v{2}Hear the sound of my supplications when I cry to you for help, \\
\poemll    as I lift up my hands toward your most holy sanctuary. \\
\poeml \v{3}Do not drag me away with the wicked, \\
\poemll    with those who practice iniquity, \\
\poeml who speak peace to their neighbors \\
\poemll    while harboring evil in their hearts. \\
\poeml \v{4}Reward them according to their deeds; \\
\poemll    according to the evil of their actions. \\
\poeml Reward them based on what they do;\fnote{Lit. \fbib{them according to work of their hands}} \\
\poemll    give them what they deserve. \\
\poeml \v{5}Because they do not understand the deeds of the \divine{Lord} \\
\poemll    or the work of his hands, \\
\poemlll       He will tear them down and never build them up. \\
\poeml \v{6}Blessed be the \divine{Lord}! \\
\poemll    For he has heard the sound of my supplications. \\
\poeml \v{7}The \divine{Lord} is my strength and my shield; \\
\poemll    my heart trusts in him, \\
\poemlll       and I received help. \\
\poeml My heart rejoices, \\
\poemll    and I give thanks to him with my song. \\
\poeml \v{8}The \divine{Lord} is the strength of his people;\fnote{Lit. \fbib{of them}} \\
\poemll    he is a refuge of deliverance for his anointed. \\
\poeml \v{9}Deliver your people \\
\poemll    and bless your inheritance! \\
\poeml Shepherd them \\
\poemll    and lift them up forever!
\end{poetry}
\labelpsalm{29}
\psalminfo{A Davidic Psalm.}
\passage{Praise to the Majestic \divine{Lord}}

\begin{poetry}
\poeml \v{1}Ascribe to the \divine{Lord}, you heavenly beings; \\
\poemll    ascribe to the \divine{Lord} glory and strength. \\
\poeml \v{2}Ascribe to the \divine{Lord} the glory due his name; \\
\poemll    worship the \divine{Lord} wearing holy attire. \\
\poeml \v{3}The voice of the \divine{Lord} was heard\fnote{The Heb. lacks \fbib{heard}} above the waters; \\
\poemll    the God of glory thundered; \\
\poemlll       the \divine{Lord} was heard\fnote{The Heb. lacks \fbib{heard}} over many waters. \\
\poeml \v{4}The voice of the \divine{Lord} is powerful; \\
\poemll    the voice of the \divine{Lord} is majestic. \\
\poeml \v{5}The voice of the \divine{Lord} snaps the cedars;\fnote{I.e. a genus of coniferous evergreen in the family \fbib{Pinaceae}; and so throughout the book} \\
\poemll    the \divine{Lord} snaps the cedars of Lebanon. \\
\poeml \v{6}He makes them stagger like a calf, \\
\poemll    even Lebanon and Sirion\fnote{I.e. Mount Hermon; cf. Deut 3:9} like a young wild ox. \\
\poeml \v{7}The voice of the \divine{Lord} shoots out flashes of fire. \\
\poeml \v{8}The voice of the \divine{Lord} shakes the wilderness; \\
\poemlll       the voice of the \divine{Lord} shakes\fnote{The Heb. lacks \fbib{shakes}} the wilderness of Kadesh. \\
\poeml \v{9}The voice of the \divine{Lord} causes deer to give birth, \\
\poemll    and strips the forest bare. \\
\poemlll       In his Temple all of them shout, ``Glory!'' \\
\poeml \v{10}The \divine{Lord} sat enthroned over the flood, \\
\poemll    and the \divine{Lord} sits as king forever. \\
\poeml \v{11}The \divine{Lord} will give strength to his people; \\
\poemll    the \divine{Lord} will bless his people with peace.
\end{poetry}
\labelpsalm{30}
\psalminfo{A Davidic Psalm for the dedication of the Temple.}
\passage{Thanksgiving for Deliverance}

\begin{poetry}
\poeml \v{1}I exalt you, \divine{Lord}, \\
\poemll    for you have lifted me up, \\
\poemlll       and my enemies could not gloat over me. \\
\poeml \v{2}\divine{Lord}, my God! \\
\poemll    I cried out to you for help \\
\poemlll       and you healed me. \\
\poeml \v{3}\divine{Lord}, you brought me from death;\fnote{Lit. \fbib{Sheol}, a reference to the realm of the dead} \\
\poemll    you kept me alive so that I did not descend into the Pit.\fnote{I.e. the place of punishment in the afterlife} \\
\poeml \v{4}You, his godly ones, \\
\poemll    sing to the \divine{Lord}, \\
\poemlll       give thanks at the mention of his holiness. \\
\poeml \v{5}For his wrath is only momentary; \\
\poemll    yet his favor is for a lifetime. \\
\poeml Weeping may lodge for the night, \\
\poemll    but shouts of joy will come in the morning. \\
\poeml \v{6}As for me, \\
\poemll    I said in my prosperity, \\
\poemlll       ``I will never be moved.'' \\
\poeml \v{7}By your favor, \divine{Lord}, \\
\poemll    you established me as a strong mountain; \\
\poeml Then you hid your face, \\
\poemll    and I was dismayed. \\
\poeml \v{8}I cried out to you, \divine{Lord}, \\
\poemll    and I make supplication to the Lord: \\
\poeml \v{9}``What profit is there in my death\fnote{Lit. \fbib{my blood}} if I go down to the Pit?\fnote{I.e. the place of punishment in the afterlife} \\
\poemll    Can dust worship you? \\
\poemlll       Can it proclaim your faithfulness?'' \\
\poeml \v{10}Hear me, \divine{Lord}, \\
\poemll    and have mercy on me! \\
\poemlll       \divine{Lord}, help me! \\
\poeml \v{11}You have turned my mourning into dancing; \\
\poemll    you took off my sackcloth \\
\poemlll       and clothed me with a garment of joy, \\
\poeml \v{12}so that I may sing praise to you \\
\poemll    and not remain silent. \\
\poeml \divine{Lord}, my God, \\
\poemll    I will give you thanks forever!
\end{poetry}
\labelpsalm{31}
\psalminfo{To the Director: A Davidic Psalm.}
\passage{Prayer and Thanksgiving}

\begin{poetry}
\poeml \v{1}In you, \divine{Lord}, I have taken refuge. \\
\poemll    Let me never be ashamed. \\
\poemlll       Because you are righteous, deliver me! \\
\poeml \v{2}Listen to me, \\
\poemll    and deliver me quickly. \\
\poeml Become a rock of safety for me, \\
\poemll    a fortified citadel to deliver me; \\
\poeml \v{3}For you are my rock and my fortress; \\
\poemll    for the sake of your name guide me and lead me. \\
\poeml \v{4}Rescue me from the net that they concealed to trap me; \\
\poemll    for you are my strength. \\
\poeml \v{5}Into your hands I commit my spirit; \\
\poemll    for you have redeemed me, \\
\poemlll       \divine{Lord} God of truth. \\
\poeml \v{6}I despise those who trust vain idols; \\
\poemll    but I have trusted in the \divine{Lord}. \\
\poeml \v{7}I will rejoice and be glad in your gracious love, \\
\poemll    for you see my affliction \\
\poemlll       and take note that my soul is distressed. \\
\poeml \v{8}You have not delivered me into the hand of the enemy, \\
\poemll    but you have set my feet in a sturdy\fnote{Lit. \fbib{broad}} place. \\
\poeml \v{9}Be gracious to me, \divine{Lord}, \\
\poemll    for I am in distress. \\
\poeml My eyes have been consumed by my grief \\
\poemll    along with my soul and my body. \\
\poeml \v{10}My life is consumed by sorrow, \\
\poemll    my years with groaning. \\
\poeml My strength has faltered because of my iniquity;\fnote{So MT DSS 5/6HevPs; LXX reads \fbib{strength grew weak in poverty}} \\
\poemll    my bones have been consumed. \\
\poeml \v{11}I have become an object of reproach to all my enemies, \\
\poemll    especially to my neighbors. \\
\poeml I have become an object of fear to my friends, \\
\poemll    and whoever sees me outside runs away from me. \\
\poeml \v{12}Like a dead man, I am forgotten in their thoughts\fnote{Lit. \fbib{hearts}}--- \\
\poemll    like broken pottery. \\
\poeml \v{13}I have heard the slander of many; \\
\poemll    it is like terror all around me, \\
\poemlll       as they conspire together and plot to take my life. \\
\poeml \v{14}But I trust in you, \divine{Lord}. \\
\poemll    I say, ``You are my God.'' \\
\poeml \v{15}My times are in your hands. \\
\poemll    Deliver me from the hands of my enemies \\
\poemlll       and from those who pursue me. \\
\poeml \v{16}May your face shine on your servant; \\
\poemll    in your gracious love, deliver me. \\
\poeml \v{17}Let me not be ashamed, \divine{Lord}, \\
\poemll    for I have called upon you. \\
\poeml Let the wicked be put to shame, \\
\poemll    let them be silent in the next life.\fnote{Lit. \fbib{in Sheol}; i.e. the realm of the dead} \\
\poeml \v{18}Let the lying lips be made still, \\
\poemll    especially those who speak arrogantly \\
\poemlll       against the righteous with pride and contempt. \\
\poeml \v{19}How great is your goodness \\
\poemll    that you have reserved for those who fear you, \\
\poeml that you have set in place for those who take refuge in you, \\
\poemll    in the presence of the children of men. \\
\poeml \v{20}You will hide them in the secret place of your presence, \\
\poemll    away from the conspiracies of men. \\
\poeml You will hide them in your tent, \\
\poemll    away from their contentious tongues. \\
\poeml \v{21}Blessed be the \divine{Lord}! \\
\poemll    In a marvelous way he demonstrated his gracious love to me, \\
\poemlll       when I was in a city under siege. \\
\poeml \v{22}When I said in my panic, \\
\poemll    ``I have been cut off in your sight,'' \\
\poeml then you surely heard the voice of my prayer \\
\poemll    in my plea to you for help. \\
\poeml \v{23}Love the \divine{Lord}, all his godly ones! \\
\poemll    The \divine{Lord} preserves the faithful \\
\poemlll       and repays those who act with proud motives. \\
\poeml \v{24}Be strong, \\
\poemll    and let your heart be courageous, \\
\poemlll       all you who put your hope in the \divine{Lord}.
\end{poetry}
\labelpsalm{32}
\psalminfo{A Davidic instruction.\fnote{T Lit. \fbib{maskil}}}
\passage{The Blessings of Forgiveness}

\begin{poetry}
\poeml \v{1}How blessed is the one whose transgression is forgiven, \\
\poemll    whose sin is covered. \\
\poeml \v{2}How blessed is the person against whom the \divine{Lord} does not charge iniquity, \\
\poemll    and in whose spirit there is no deceit. \\
\poeml \v{3}When I kept silent about my sin,\fnote{The Heb. lacks \fbib{about my sin}} \\
\poemll    my body\fnote{Lit. \fbib{bones}} wasted away \\
\poemlll       by my groaning all day long. \\
\poeml \v{4}For your hand was heavy upon me day and night; \\
\poemll    my strength was exhausted \\
\poemlll       as in a summer drought.
\end{poetry}
\interlude{Interlude}

\begin{poetry}
\poeml \v{5}My sin I acknowledged to you; \\
\poemll    my iniquity I did not hide. \\
\poeml I said, ``I will confess my transgressions to the \divine{Lord}.'' \\
\poemll    And you forgave the guilt of my sin!
\end{poetry}
\interlude{Interlude}

\begin{poetry}
\poeml \v{6}Therefore every godly person should pray to you at such a time.\fnote{Lit. \fbib{at a time of finding}} \\
\poemll    Surely a flood of great waters will not reach him. \\
\poeml \v{7}You are my hiding place; \\
\poemll    you will deliver me from trouble \\
\poemlll       and surround me with shouts of deliverance.
\end{poetry}
\interlude{Interlude}

\begin{poetry}
\poeml \v{8}I will instruct you and teach you \\
\poemll    concerning the path you should walk; \\
\poemlll       I will direct you with my eye. \\
\poeml \v{9}Don't be like a horse or mule, \\
\poemll    without understanding. \\
\poeml They are held in check by a bit and bridle in their mouths; \\
\poemll    otherwise they will not remain near you. \\
\poeml \v{10}The wicked have many sorrows, \\
\poemll    but gracious love surrounds those who trust in the \divine{Lord}. \\
\poeml \v{11}Righteous ones, be glad in the \divine{Lord} and rejoice! \\
\poemll    Shout for joy, all of you who are upright in heart!
\end{poetry}
\labelpsalm{33}
\passage{Praise to the Creator and Deliverer}

\begin{poetry}
\poeml \v{1}Rejoice in the \divine{Lord}, righteous ones; \\
\poemll    for the praise of the upright is beautiful. \\
\poeml \v{2}With the lyre, give thanks to the \divine{Lord}; \\
\poemll    with the ten stringed harp, play music to him; \\
\poeml \v{3}with a new song, sing to him; \\
\poemll    with shouts of joy, play skillfully. \\
\poeml \v{4}For the word of the \divine{Lord} is upright; \\
\poemll    and all his works are done in faithfulness. \\
\poeml \v{5}He loves righteousness and justice; \\
\poemll    the world is filled with the gracious love of the \divine{Lord}. \\
\poeml \v{6}By the word of the \divine{Lord} the heavens were made; \\
\poemll    all the heavenly bodies\fnote{Lit. \fbib{all their host}} by the breath of his mouth. \\
\poeml \v{7}He gathered the oceans into a single place; \\
\poemll    he put the deep water into storehouses. \\
\poeml \v{8}Let all the world fear the \divine{Lord}; \\
\poemll    let all the inhabitants of the world stand in awe of him; \\
\poeml \v{9}because he spoke and it came to be, \\
\poemll    because he commanded, it stood firm. \\
\poeml \v{10}The \divine{Lord} makes void the counsel of nations; \\
\poemll    he frustrates the plans of peoples. \\
\poeml \v{11}But the \divine{Lord}'s counsel stands firm forever, \\
\poemll    the plans in his mind for all generations. \\
\poeml \v{12}How blessed is the nation whose God is the \divine{Lord}, \\
\poemll    the people he has chosen as his own inheritance. \\
\poeml \v{13}When the \divine{Lord} looks down from heaven, \\
\poemll    he observes every human being. \\
\poeml \v{14}From his dwelling place, \\
\poemll    he looks down on all the inhabitants of the earth. \\
\poeml \v{15}He formed the hearts of them all; \\
\poemll    he understands everything they do. \\
\poeml \v{16}A king is not saved by a large army; \\
\poemll    a mighty soldier is not delivered by his great strength. \\
\poeml \v{17}It is vain to trust in a horse for deliverance, \\
\poemll    even with its great strength, it cannot deliver. \\
\poeml \v{18}Indeed, the \divine{Lord} watches those who fear him; \\
\poemll    those who trust in his gracious love \\
\poeml \v{19}to deliver them from death; \\
\poemll    to keep them alive in times of famine. \\
\poeml \v{20}We wait on the \divine{Lord}; \\
\poemll    he is our help and our shield. \\
\poeml \v{21}Indeed, our heart will rejoice in him, \\
\poemll    because we have placed our trust in his holy name. \\
\poeml \v{22}\divine{Lord}, may your gracious love be upon us, \\
\poemll    even as we hope in you.
\end{poetry}
\labelpsalm{34}
\psalminfo{By David, when he pretended to be insane before Abimelech, who drove him away. So David\fnote{T Lit. \fbib{he}} left.}
\passage{Learning about God's Deliverance}

\begin{poetry}
\poeml \v{1}\fnote{This Psalm is an acrostic poem.}I will bless the \divine{Lord} at all times; \\
\poemll    his praise will be in my mouth continuously. \\
\poeml \v{2}My soul will glorify the \divine{Lord}; \\
\poemll    the humble will hear about it and rejoice. \\
\poeml \v{3}Magnify the \divine{Lord} with me! \\
\poeml Let us lift up his name together! \\
\poeml \v{4}I sought the \divine{Lord} and he answered me; \\
\poemll    he delivered me from all of my fears. \\
\poeml \v{5}Look to him and be radiant; \\
\poemll    and you\fnote{Lit. \fbib{their faces}} will not be ashamed. \\
\poeml \v{6}This poor man cried out, and the \divine{Lord} heard \\
\poemll    and delivered him from all of his distress. \\
\poeml \v{7}The angel of the \divine{Lord} surrounds those who fear him, \\
\poemll    and he delivers them. \\
\poeml \v{8}Taste and see that the \divine{Lord} is good! \\
\poemll    How blessed is the person who trusts in him! \\
\poeml \v{9}Fear the \divine{Lord}, you holy ones of his; \\
\poemll    for those who fear him lack nothing. \\
\poeml \v{10}Young lions lack and go hungry, \\
\poemll    but those who seek the \divine{Lord} will never lack any good thing. \\
\poeml \v{11}Come, children, listen to me, \\
\poemll    and I will teach you the fear of the \divine{Lord}. \\
\poeml \v{12}Who among you\fnote{Lit. \fbib{Who is the person who}} desires life, \\
\poemll    and wants long life in order to see good? \\
\poeml \v{13}Then keep your tongue from doing evil \\
\poemll    and your lips from spreading lies. \\
\poeml \v{14}Avoid evil and do good! \\
\poemll    Seek peace and pursue it! \\
\poeml \v{15}The\fnote{Lit. \fbib{The eyes of the}} \divine{Lord} looks on the righteous, \\
\poemll    and he listens to their cries. \\
\poeml \v{16}The face of the \divine{Lord} is set against those who do evil, \\
\poemll    and he will remove people's recollection of them from the earth. \\
\poeml \v{17}The \divine{Lord} hears those who cry out, \\
\poemll    and he delivers them from all their distress. \\
\poeml \v{18}The \divine{Lord} is close to the brokenhearted, \\
\poemll    and he delivers those whose spirit has been crushed. \\
\poeml \v{19}A righteous person will have many troubles, \\
\poemll    but the \divine{Lord} will deliver him from them all. \\
\poeml \v{20}God\fnote{Lit. \fbib{He}} protects all his bones; \\
\poemll    not one of them will be broken. \\
\poeml \v{21}Evil will kill the wicked; \\
\poemll    those who hate the righteous will be held guilty. \\
\poeml \v{22}The \divine{Lord} redeems the lives of his servants; \\
\poemll    and none of those who trust in him will be held guilty.
\end{poetry}
\labelpsalm{35}
\psalminfo{Davidic}
\passage{A Prayer for Deliverance}

\begin{poetry}
\poeml \v{1}Argue my case,\fnote{The Heb. lacks \fbib{my case}} \divine{Lord}, \\
\poemll    against those who argue against me. \\
\poemlll       Fight against those who fight against me. \\
\poeml \v{2}Take up the buckler\fnote{I.e. a small shield} and the shield, \\
\poemll    and rise up to help me. \\
\poeml \v{3}Take out the spear and the ax to confront the one who pursues me; \\
\poemll    say to me, ``I am your deliverer!'' \\
\poeml \v{4}Let those who seek my life be ashamed and disgraced; \\
\poemll    let those who plot evil against me be driven back and confounded. \\
\poeml \v{5}Make them like the chaff before the wind, \\
\poemll    as the messenger of the \divine{Lord} pushes them aside. \\
\poeml \v{6}May their path be dark and slippery, \\
\poemll    as the messenger of the \divine{Lord} tracks them down. \\
\poeml \v{7}Without justification they laid a snare for me; \\
\poemll    without justification they dug a pit to trap me. \\
\poeml \v{8}Let destruction come upon them\fnote{Lit. \fbib{him}} unawares, \\
\poemll    and let the net that he hid catch him; \\
\poemlll       let him fall into destruction. \\
\poeml \v{9}My soul will rejoice in the \divine{Lord} \\
\poemll    and be glad in his deliverance. \\
\poeml \v{10}All my bones will say, \\
\poemll    ``\divine{Lord}, who is like you? \\
\poeml Who delivers the weak from the one who is stronger than he, \\
\poemll    and the weak and the needy from the one who wants to rob him?'' \\
\poeml \v{11}False witnesses stepped forward \\
\poemll    and questioned me concerning things \\
\poemlll       about which I knew nothing. \\
\poeml \v{12}They paid me back evil for good; \\
\poemll    my soul mourns. \\
\poeml \v{13}But when they were sick, \\
\poemll    I wore sackcloth, humbled myself with fasting, \\
\poemlll       and prayed from my heart repeatedly for them.\fnote{The Heb. lacks \fbib{for them}} \\
\poeml \v{14}I paced about as for my friend or my brother, \\
\poemll    and fell down mourning as one weeps for one's mother. \\
\poeml \v{15}But when I stumbled, \\
\poemll    they rejoiced and gathered together. \\
\poeml They gathered together against me--- \\
\poemll    attackers whom I did not know. \\
\poemlll       They tore me apart and would not stop. \\
\poeml \v{16}Malicious mockers\fnote{So LXX; DSS 4QPs\textsuperscript{a} read \fbib{They mocked me viciously}; MT reads \fbib{Mockers of cake}}--- \\
\poemll    they gnashed\fnote{So DSS 4QPs\textsuperscript{a} LXX; MT reads \fbib{gnashing}} their teeth against me. \\
\poeml \v{17}Lord, how long will you just watch? \\
\poemll    Rescue me from their destruction, \\
\poemlll       my precious life from these young lions. \\
\poeml \v{18}Then I will give you thanks in front of the great congregation; \\
\poemll    in the midst of the mighty throng I will praise you. \\
\poeml \v{19}Do not let my deceitful enemies gloat over me, \\
\poemll    nor let those who hate me without justification mock me with their eyes. \\
\poeml \v{20}For they do not speak peace; \\
\poemll    they devise clever lies against the peaceful people of the land. \\
\poeml \v{21}They open their mouth wide against me, \\
\poemll    claiming, ``Yes! Yes! We saw him do\fnote{The Heb. lacks \fbib{him do}} it with our own eyes!'' \\
\poeml \v{22}You see this, \divine{Lord}, \\
\poemll    so do not be silent. \\
\poemlll       Lord, do not be far from me! \\
\poeml \v{23}Wake up! Arouse yourself to vindicate me \\
\poemll    and argue my case, my God and my Lord. \\
\poeml \v{24}Judge me according to your righteousness, \divine{Lord} my God! \\
\poemll    But do not let them gloat over me. \\
\poeml \v{25}Don't let them say in their hearts, \\
\poemll    ``Yes! We got what we wanted.'' \\
\poeml Don't let them say, \\
\poemll    ``We have swallowed him up.'' \\
\poeml \v{26}Instead, let those who gloat over the evil directed against me \\
\poemll    be ashamed and confounded together; \\
\poeml Let those who exalt themselves over me \\
\poemll    be clothed with shame and dishonor. \\
\poeml \v{27}Let those who delight in my vindication \\
\poemll    shout for joy and rejoice! \\
\poeml Let them continuously say, \\
\poemll    ``Magnify the \divine{Lord}, who delights in giving peace to\fnote{So MT; DSS 4QPsa LXX read \fbib{\divine{Lord}, you who delight in the welfare of}} his servant.'' \\
\poeml \v{28}My tongue will declare your righteousness \\
\poemll    and praise you all day long.
\end{poetry}
\labelpsalm{36}
\psalminfo{To the Director: By the servant of the \divine{Lord}, David.}
\passage{An Oracle from the \divine{Lord}}

\begin{poetry}
\poeml \v{1}An oracle that came to me\fnote{So MT DSS 4QPs\textsuperscript{a}; lit. \fbib{oracle in the midst of my heart}; Syr Origen read \fbib{of his heart}} about the transgressions of the wicked: \\
\poeml There is no fear of God before his eyes. \\
\poeml \v{2}He flatters himself\fnote{Lit. \fbib{himself in his own eyes}} too much\fnote{The Heb. lacks \fbib{too much}} to discover his transgression and hate it. \\
\poeml \v{3}The words from his mouth are vain and deceptive. \\
\poemll    He has abandoned behaving wisely and doing good. \\
\poeml \v{4}He devises iniquity on his bed \\
\poemll    and is determined to follow a path that is not good. \\
\poemlll       He does not resist evil.
\passage{Praise to the \divine{Lord}}
\poeml \v{5}Your gracious love, \divine{Lord}, reaches to the heavens; \\
\poemll    your truth\fnote{Or \fbib{faithfulness}} extends to the skies.\fnote{Or \fbib{clouds}} \\
\poeml \v{6}Your righteousness is like the mountains of God; \\
\poemll    your justice is like the great depths of the sea.\fnote{The Heb. lacks \fbib{of the sea}} \\
\poemlll       You deliver both\fnote{The Heb. lacks \fbib{both}} people and animals, \divine{Lord}. \\
\poeml \v{7}How precious is your gracious love, God! \\
\poemll    The children of men take refuge in the shadow of your wings. \\
\poeml \v{8}They are refreshed from the abundance of your house; \\
\poemll    You cause them to drink from the river of your pleasures. \\
\poeml \v{9}For with you is a fountain of life, \\
\poemll    and in your light we will see light. \\
\poeml \v{10}Send forth your gracious love to those who know you, \\
\poemll    and your righteousness to those who are upright in heart. \\
\poeml \v{11}Do not let the foot of the proud crush me; \\
\poemll    and do not let the hand of the wicked dissuade me. \\
\poeml \v{12}There, those who do evil have fallen; \\
\poemll    They have been thrown down, \\
\poemlll       and they cannot get up.
\end{poetry}
\labelpsalm{37}
\psalminfo{Davidic\fnote{T This acrostic psalm begins each verse with a consecutive letter of the Hebrew alphabet}}
\passage{Patiently Trust in God}

\begin{poetry}
\poeml \v{1}Don't be angry because of those who do evil, \\
\poemll    do not be jealous because of those who commit iniquity. \\
\poeml \v{2}Indeed, they soon will wither like grass, \\
\poemll    and like green herbs they will fade away. \\
\poeml \v{3}Trust in the \divine{Lord} and do good. \\
\poemll    Dwell in the land and feed on faithfulness. \\
\poeml \v{4}Delight yourself in the \divine{Lord}, \\
\poemll    and he will give you the desires of your heart. \\
\poeml \v{5}Commit your way to the \divine{Lord}; \\
\poemll    Trust him, and he will act. \\
\poeml \v{6}He will bring forth your righteousness as a light, \\
\poemll    and your justice as the noonday sun.\fnote{The Heb. lacks \fbib{sun}} \\
\poeml \v{7}Be silent in the \divine{Lord}'s presence \\
\poemll    and wait patiently for him. \\
\poeml Don't be angry because of the one whose way prospers \\
\poemll    or the one who implements evil schemes. \\
\poeml \v{8}Calm your anger and abandon wrath. \\
\poemll    Don't be angry--- \\
\poemlll       it only leads to evil. \\
\poeml \v{9}Those who do evil will perish. \\
\poemll    But those who wait\fnote{I.e. \fbib{trust}} on the \divine{Lord} will inherit the land. \\
\poeml \v{10}Yet a little while longer, \\
\poemll    and the wicked will be no more. \\
\poeml You will search for his place, \\
\poemll    but he will not be there. \\
\poeml \v{11}The humble will inherit the land; \\
\poemll    they will take in abundant peace. \\
\poeml \v{12}The wicked person plots against the righteous, \\
\poemll    and grinds his teeth at him. \\
\poeml \v{13}But the Lord laughs at him \\
\poemll    because he sees that his day is coming! \\
\poeml \v{14}The wicked take out a sword and bend the bow, \\
\poemll    to bring down the humble and the poor \\
\poemlll       to slay those who are righteous in conduct. \\
\poeml \v{15}But their sword will pierce their own heart, \\
\poemll    and their bows will be broken! \\
\poeml \v{16}Better is the little that the righteous have \\
\poemll    than the abundance of many wicked people. \\
\poeml \v{17}For the arms of the wicked will be broken, \\
\poemll    but the \divine{Lord} upholds the righteous. \\
\poeml \v{18}The \divine{Lord} knows the day of the blameless, \\
\poemll    and their inheritance will last forever. \\
\poeml \v{19}They will not experience shame in times of trouble; \\
\poemll    in times of famine they will have plenty. \\
\poeml \v{20}Indeed, the wicked will perish. \\
\poemll    The \divine{Lord}'s enemies will be consumed like flowers\fnote{Lit. \fbib{like glorious things}} in the fields. \\
\poemlll       They will vanish like\fnote{So LXX DSS 4QpPs\textsuperscript{a}; MT reads \fbib{in}} smoke. \\
\poeml \v{21}The wicked borrow but never pay back; \\
\poemll    but the righteous are generous and give. \\
\poeml \v{22}For those blessed by God\fnote{Lit. \fbib{him}} will inherit the land, \\
\poemll    but those cursed by him will be cut off. \\
\poeml \v{23}A man's steps are established by the \divine{Lord}, \\
\poemll    and the \divine{Lord}\fnote{Lit. \fbib{he}} delights in his way. \\
\poeml \v{24}Though he stumbles, \\
\poemll    he will not fall down flat, \\
\poemlll       for the \divine{Lord} will hold up his hand. \\
\poeml \v{25}I once was young and now I am old, \\
\poemll    but I have not seen a righteous person forsaken \\
\poemlll       or his descendants begging for bread. \\
\poeml \v{26}Every day he is generous, lending freely, \\
\poemll    and his descendants are blessed. \\
\poeml \v{27}Depart from evil, and do good, \\
\poemll    and you will live in the land\fnote{The Heb. lacks \fbib{in the land}} forever. \\
\poeml \v{28}Indeed, the \divine{Lord} loves justice, \\
\poemll    and he will not abandon his godly ones. \\
\poeml They are kept safe forever, \\
\poemll    but the lawless will be chased away,\fnote{So LXX DSS 4QpPs\textsuperscript{a}; the Heb. lacks this line} \\
\poemlll       and the descendants of the wicked will be cut off. \\
\poeml \v{29}The righteous will inherit the land, \\
\poemll    and they will dwell in it forever. \\
\poeml \v{30}The mouth of the righteous one produces wisdom; \\
\poemll    his tongue speaks justice. \\
\poeml \v{31}The instruction\fnote{Or \fbib{law}} of his God is in his heart; \\
\poemll    his steps will not slip. \\
\poeml \v{32}The wicked stalks the righteous person, seeking to kill him, \\
\poeml \v{33}but the \divine{Lord} will not let him fall into his hands. \\
\poemlll       He will not be condemned when he is put on trial. \\
\poeml \v{34}Wait on the \divine{Lord}, \\
\poemll    Keep faithful to his way, \\
\poemlll       and he will exalt you to possess the land. \\
\poeml You will see the wicked cut off. \\
\poeml \v{35}I once observed a wicked and oppressive person, \\
\poemll    flourishing like a green tree in native soil. \\
\poeml \v{36}But then he\fnote{So MT; LXX 4QpPs\textsuperscript{a} read \fbib{I}} passed away;\fnote{So MT; LXX reads \fbib{I passed by}; Syr Hieronymus DSS 4QpPs\textsuperscript{a} read \fbib{I passed by in front of him}} \\
\poemll    in fact, he simply was not there. \\
\poeml When I looked for him, \\
\poemll    he could not be found. \\
\poeml \v{37}Observe the blameless! \\
\poemll    Take note of the upright! \\
\poemlll       Indeed, the future of that man is peace. \\
\poeml \v{38}Sinners will be destroyed together; \\
\poemll    the future of the wicked will be cut off. \\
\poeml \v{39}But deliverance for the righteous one comes from the \divine{Lord}; \\
\poemll    he is their strength in times of distress. \\
\poeml \v{40}The \divine{Lord} helps and delivers them; \\
\poemll    he will deliver them from the wicked, \\
\poemlll       and he will save them because they have sought refuge in him.
\end{poetry}
\labelpsalm{38}
\psalminfo{A Davidic Psalm: As a Reminder.}
\passage{The Outcast Cries Out}

\begin{poetry}
\poeml \v{1}\divine{Lord}! Do not rebuke me in your anger; \\
\poemll    do not correct me in your wrath, \\
\poeml \v{2}because your arrows have sunk deep into me, \\
\poemll    and your hand has come down hard on me. \\
\poeml \v{3}My body is unhealthy due to your anger, \\
\poemll    and my bones have no rest due to my sin. \\
\poeml \v{4}My iniquities loom over my head; \\
\poemll    like a cumbersome burden, they are too heavy for me. \\
\poeml \v{5}My wounds have putrefied and festered \\
\poemll    because of my foolishness. \\
\poeml \v{6}I am bent over and walk about greatly bowed down; \\
\poemll    all day long I go around mourning. \\
\poeml \v{7}My insides\fnote{Lit. \fbib{loins}} are burning \\
\poemll    and my body is unhealthy. \\
\poeml \v{8}I am weak and utterly crushed; \\
\poemll    I cry out in distress because of my heart's anguish. \\
\poeml \v{9}Lord, all my longings are before you, \\
\poemll    and my groaning is not hidden from you. \\
\poeml \v{10}My heart pounds, \\
\poemll    my strength fails me, \\
\poemlll       even the gleam in my eye is gone. \\
\poeml \v{11}As for my friends and my neighbors, \\
\poemll    they stand aloof from my distress; \\
\poemlll       even my close relatives stand at a distance. \\
\poeml \v{12}Those who seek my life lay snares for me; \\
\poemll    those who seek to do me harm brag all day long about their wicked planning. \\
\poeml \v{13}I am like the deaf, who cannot hear, \\
\poemll    and like the mute, who cannot open his mouth. \\
\poeml \v{14}Indeed, I have become like a man who hears nothing, \\
\poemll    and in whose mouth there is no rebuke. \\
\poeml \v{15}Because I have placed my hope in you, \divine{Lord}, \\
\poemll    you will answer, Lord, my God. \\
\poeml \v{16}For I said, ``Do not let them gloat over me, \\
\poemll    as they congratulate themselves when my foot slips.'' \\
\poeml \v{17}Indeed, I am being set up for a fall, \\
\poemll    and I am continuously reminded of my pain. \\
\poeml \v{18}I confess my iniquity, \\
\poemll    and my sin troubles me. \\
\poeml \v{19}But my enemies are alive and well;\fnote{So MT LXX; DSS 4QPs\textsuperscript{a} lack this line} \\
\poemll    those who hate me\fnote{So MT LXX; DSS 4QPs\textsuperscript{a} read \fbib{Those who are my enemies}} for no reason are numerous.\fnote{DSS 4QPs\textsuperscript{a} read \fbib{numerous, and many are those who hate me by deceiving me}; cf. Ps 35:19; 69:5} \\
\poeml \v{20}They\fnote{So LXX DSS 4QPs\textsuperscript{a}; MT reads \fbib{And they}} reward my good with evil, \\
\poemll    opposing me because I seek to do good.\fnote{So MT; DSS 4QPs\textsuperscript{a} read \fbib{evil plunder me instead of a good thing}} \\
\poeml \v{21}Don't forsake me, \divine{Lord}. \\
\poemll    My God, do not be so distant from me. \\
\poeml \v{22}Come quickly and help me, \\
\poemll    Lord, my deliverer.
\end{poetry}
\labelpsalm{39}
\psalminfo{To the Director: To Jeduthun. A Davidic Psalm.}
\passage{A Prayer about Life's Priorities}

\begin{poetry}
\poeml \v{1}I told myself, ``I will keep watch over my tongue to keep from sinning. \\
\poemll    I will muzzle my mouth when the wicked are around.'' \\
\poeml \v{2}I was as silent as a mute person; \\
\poemll    I said nothing, not even something good, \\
\poemlll       and my distress deepened. \\
\poeml \v{3}My heart within me became incensed;\fnote{Lit. \fbib{hot}} \\
\poemll    as I thought about it, the fire burned. \\
\poeml Then I\fnote{Lit. \fbib{Then my mouth}} spoke out: \\
\poeml \v{4}``\divine{Lord}, let me know how my life ends,\fnote{Lit. \fbib{my end}} \\
\poemll    and the standard by which you will measure\fnote{Lit. \fbib{the measure of}} my days, whatever it is! \\
\poemlll       Then I will know how transient my life is. \\
\poeml \v{5}Look, you have made my life span fit in your hand; \\
\poemll    It is nothing compared to yours. \\
\poemlll       Surely every person at their best is a puff of wind.
\end{poetry}
\interlude{Interlude}

\begin{poetry}
\poeml \v{6}In fact, people walk around as shadows. \\
\poemll    Surely, they busy themselves for nothing, \\
\poemlll       heaping up possessions but not knowing who will get them. \\
\poeml \v{7}How long, \divine{Lord}, will I wait expectantly? \\
\poemll    I have placed my hope in you. \\
\poeml \v{8}Deliver me from all my transgressions, \\
\poemll    and do not let fools scorn me.'' \\
\poeml \v{9}I remain silent; \\
\poemll    I do not open my mouth, \\
\poemlll       for you are the one who acted. \\
\poeml \v{10}Stop scourging me, \\
\poemll    since I have been crushed by your heavy hand. \\
\poeml \v{11}You rebuke by chastening a man with the consequence of iniquities; \\
\poemll    you destroy what is attractive to him, as one would treat a moth. \\
\poemlll       Indeed, every person is a puff of wind.
\end{poetry}
\interlude{Interlude}

\begin{poetry}
\poeml \v{12}Hear my prayer, \divine{Lord}, \\
\poemll    pay attention to my cry, \\
\poemlll       and do not ignore my tears. \\
\poeml I am an alien in your presence, \\
\poemll    a stranger just like my ancestors were. \\
\poeml \v{13}Stop looking at me with chastisement,\fnote{The Heb. lacks \fbib{with chastisement}} so I can smile again, \\
\poemll    before I depart and am no more.
\end{poetry}
\labelpsalm{40}
\psalminfo{To the Director: A Davidic Psalm.}
\passage{Prayer for Help and Praise to God}

\begin{poetry}
\poeml \v{1}I waited expectantly\fnote{Or \fbib{eagerly}} for the \divine{Lord}, \\
\poemll    and he took notice of me \\
\poemlll       and heard my cry. \\
\poeml \v{2}He plucked me out of a pit of confusion,\fnote{Or \fbib{destruction}} \\
\poemll    even out of the quicksand; \\
\poeml he placed my feet on a rock \\
\poemll    and established my steps. \\
\poeml \v{3}He put a new song in my mouth, \\
\poemll    praise to our God! \\
\poeml Many will watch and be in awe, \\
\poemll    and they will place their trust in the \divine{Lord}. \\
\poeml \v{4}How blessed is that strong person \\
\poemll    who places his trust in the \divine{Lord}, \\
\poemll    and who has not acknowledged the proud \\
\poemlll       nor resorted to lies. \\
\poeml \v{5}\divine{Lord}, my God, \\
\poemll    You have done great things: \\
\poemlll       marvelous works and your thoughts toward us. \\
\poeml There is no one who compares to you! \\
\poemll    I will try to recite your actions,\fnote{Lit. \fbib{recite them}} \\
\poemlll       even though there are too many to number. \\
\poeml \v{6}You take no delight in sacrifices and offerings--- \\
\poemll    you have prepared my ears to listen---\fnote{The Heb. lacks \fbib{to listen}} \\
\poemlll       you require no burnt offerings or sacrifices for sin. \\
\poeml \v{7}Then I said, ``Here I am! I have come! \\
\poemll    In the scroll of the book it is written about me. \\
\poeml \v{8}I delight to do your will, my God. \\
\poemll    Your Law is part of my inner being.'' \\
\poeml \v{9}In the great congregation I have proclaimed the righteous good news. \\
\poemll    Behold, I did not seal my lips, \divine{Lord}, as you know. \\
\poeml \v{10}I have not ignored\fnote{Lit. \fbib{not covered over}} your righteousness in my heart; \\
\poemll    instead, I have proclaimed your faithfulness and deliverance. \\
\poeml I have not concealed your gracious love and truthfulness \\
\poemll    from the great congregation. \\
\poeml \v{11}\divine{Lord}, do not withhold your mercy\fnote{Lit. \fbib{mercies}} from me, \\
\poemll    for your gracious love and truthfulness will keep me safe continuously. \\
\poeml \v{12}Innumerable evils have surrounded me; \\
\poemll    my iniquities have overtaken me so that I cannot see. \\
\poeml They are more in number than the hair on my head, \\
\poemll    and my courage\fnote{Lit. \fbib{heart}} has forsaken me. \\
\poeml \v{13}Be pleased, \divine{Lord}, to deliver me; \\
\poemll    \divine{Lord}, hurry up and help me! \\
\poeml \v{14}May those who seek to destroy my life be ashamed and confounded; \\
\poemll    let them be driven backwards and humiliated, \\
\poemlll       particularly those who wish me evil. \\
\poeml \v{15}Let shame be the reward for those who say to me, ``Aha! Aha!'' \\
\poeml \v{16}Let all who seek you shout for joy and be glad in you. \\
\poeml May those who love your deliverance say, \\
\poemll    ``The \divine{Lord} be magnified!'' continuously. \\
\poeml \v{17}But I am poor and needy; \\
\poemll    may the Lord think about me. \\
\poeml You are my help and deliverer. \\
\poemll    My God, do not tarry too long!
\end{poetry}
\labelpsalm{41}
\psalminfo{To the Director: A Davidic Psalm.}
\passage{When Things Go Wrong}

\begin{poetry}
\poeml \v{1}Blessed is the one who is considerate of the destitute;\fnote{Or \fbib{poor}} \\
\poemll    the \divine{Lord} will deliver him when the times are evil. \\
\poeml \v{2}The \divine{Lord} will protect him and keep him alive; \\
\poemll    he will be blessed in the land; \\
\poemlll       and he will not be handed over to the desires of his enemies. \\
\poeml \v{3}The \divine{Lord} will uphold him even on his sickbed; \\
\poemll    you will transform his bed of illness into health. \\
\poeml \v{4}As for me, I said, \\
\poemll    ``\divine{Lord}, be gracious to me! \\
\poemlll       Heal me, for I have sinned against you!'' \\
\poeml \v{5}As for my enemies, with malice they said, \\
\poemll    ``When will he die and memory of\fnote{The Heb. lacks \fbib{memory of}} his name perish?'' \\
\poeml \v{6}The one who comes to visit me speaks lies; \\
\poemll    in his heart he thinks slanderous things about me \\
\poemlll       and goes around spreading them. \\
\poeml \v{7}As for all who hate me, \\
\poemll    they whisper together against me; \\
\poemlll       they desire to do me harm. \\
\poeml \v{8}They say, ``Wickedness is entrenched in him. \\
\poemll    Once he is brought low, \\
\poemlll       he will not rise again.'' \\
\poeml \v{9}As for my best friend, \\
\poemll    the one in whom I trusted, \\
\poeml the one who ate my bread, \\
\poemll    even he has insulted\fnote{Lit. \fbib{has lifted up his heel against}} me! \\
\poeml \v{10}But you, \divine{Lord}, be gracious to me and raise me up \\
\poemll    so that I may pay them back! \\
\poeml \v{11}In this way I will know that you are pleased with me, \\
\poemll    and that my enemies will not shout in triumph over me. \\
\poeml \v{12}As for me, you will maintain my just cause, \\
\poemll    and you will cause me to stand in your presence forever. \\
\poeml \v{13}Blessed be the \divine{Lord} God of Israel, \\
\poemll    from eternity to eternity. \\
\poemlll       Amen and amen!
\end{poetry}
\booksection{BOOK II (Psalms 42-72)}
\labelpsalm{42}
\psalminfo{To the Director: An instruction\fnote{T Lit. \fbib{maskil}} of the Sons of Korah.}
\passage{Hope in God When Times of Trouble Come}

\begin{poetry}
\poeml \v{1}As an antelope pants for streams of water, \\
\poemll    so my soul pants for you, God. \\
\poeml \v{2}My soul thirsts for God, for the living God. \\
\poemll    When may I come and appear in God's presence? \\
\poeml \v{3}My tears have been my food day and night, \\
\poemll    while people\fnote{The Heb. lacks \fbib{people}} keep asking me all day long, \\
\poemlll       ``Where is your God?'' \\
\poeml \v{4}These things I will recall as I pour out my troubles\fnote{Lit. \fbib{soul}} within me: \\
\poemll    I used to go with the crowd in a procession to the house of God, \\
\poemlll       accompanied with shouts of joy and thanksgiving. \\
\poeml \v{5}Why are you in despair, my soul? \\
\poemll    Why are you disturbed within me? \\
\poeml Hope in God, \\
\poemll    for once again I will praise him, \\
\poemlll       since his presence saves me. \\
\poeml \v{6}My God, my soul feels depressed\fnote{Lit. \fbib{soul is bowed down}} within me; \\
\poemll    therefore I will remember you from the land of Jordan, \\
\poeml from the heights of Hermon, \\
\poemll    even from the foothills.\fnote{Or \fbib{from Mount Mizar}} \\
\poeml \v{7}Deep waters call out to what is deeper still;\fnote{Lit. \fbib{Deep calls to deep}} \\
\poemll    at the roar of your waterfalls \\
\poemlll       all your breakers and your waves swirled over me. \\
\poeml \v{8}By day the \divine{Lord} will command his gracious love, \\
\poemll    and by night his song is with me--- \\
\poemlll       a prayer to the God of my life. \\
\poeml \v{9}I will ask God, my Rock, ``Why have you forsaken me? \\
\poemll    Why do I go around mourning under the enemy's oppression?'' \\
\poeml \v{10}Like the shattering of my bones are the taunts of my oppressors, \\
\poemll    saying to me all day long, \\
\poemlll       ``Where is your God?'' \\
\poeml \v{11}Why are you in despair, my soul? \\
\poemll    Why are you disturbed within me? \\
\poeml Hope in God, \\
\poemll    for once again I will praise him, \\
\poeml since his presence saves me \\
\poemll    and he is my God.
\end{poetry}
\labelpsalm{43}
\passage{God is my Hope during Times of Trouble}

\begin{poetry}
\poeml \v{1}\fnote{Some Heb. MSS constitute Psalms 42 and 43 as a single psalm.}You be my judge,\fnote{Lit. \fbib{Judge me}} God, \\
\poemll    and plead my case against an unholy nation; \\
\poemlll       rescue me from the deceitful and unjust man. \\
\poeml \v{2}Since you are the God who strengthens me, \\
\poemll    why have you forsaken me? \\
\poeml Why do I go around mourning under the enemy's oppression?'' \\
\poeml \v{3}Send forth your light and your truth \\
\poemll    so they may guide me. \\
\poeml Let them bring me to your holy mountain and to your dwelling places.\fnote{Or \fbib{tents}} \\
\poeml \v{4}Then I will approach the altar of God, \\
\poemll    even to God in whom my joy finds its source.\fnote{Lit. \fbib{God who is the gladness of my joy}} \\
\poeml Then I will praise you with the lyre, \\
\poemll    God, my God, \\
\poeml \v{5}Why are you in despair, my soul? \\
\poemll    Why are you disturbed within me? \\
\poeml Hope in God, \\
\poemll    because I will praise him once again, \\
\poeml since his presence saves me \\
\poemll    and he is my God.
\end{poetry}
\labelpsalm{44}
\psalminfo{To the Director: An instruction\fnote{T Lit. \fbib{maskil}} of the Sons of Korah.}
\passage{A Prayer in Times of Defeat}

\begin{poetry}
\poeml \v{1}God, we heard it with our ears; \\
\poemll    our ancestors told us about what you did in their day--- \\
\poemlll       a long time ago. \\
\poeml \v{2}With your hand you expelled the nations \\
\poemll    and established our ancestors.\fnote{Lit. \fbib{them}} \\
\poeml You afflicted nations \\
\poemll    and cast them out. \\
\poeml \v{3}It was not with their sword that they inherited the land, \\
\poemll    nor did their own arm deliver them. \\
\poeml But it was by your power,\fnote{Lit. \fbib{right hand}} your strength, \\
\poemll    and by the light of your face; \\
\poemlll       because you were pleased with them. \\
\poeml \v{4}You are my king, God, \\
\poemll    command\fnote{So MT DSS 11QPs\textsuperscript{c}; LXX reads \fbib{truly my king and my God, who commands}} victories\fnote{Lit. \fbib{deliverances}} for Jacob. \\
\poeml \v{5}Through you we will knock down our oppressors; \\
\poemll    through your name we will tread down those who rise up against us. \\
\poeml \v{6}For I place no confidence in my bow, \\
\poemll    nor will my sword deliver me. \\
\poeml \v{7}For you delivered us from our oppressors \\
\poemll    and put to shame those who hate us. \\
\poeml \v{8}We will praise God all day long; \\
\poemll    and to your name we will give thanks forever.
\end{poetry}
\interlude{Interlude}

\begin{poetry}
\poeml \v{9}However, you cast us off and made us ashamed! \\
\poemll    You did not even march with our armies! \\
\poeml \v{10}You made us retreat from our oppressors. \\
\poemll    Our enemies ransacked us. \\
\poeml \v{11}You handed us over to be slaughtered like sheep \\
\poemll    and you scattered us among the nations. \\
\poeml \v{12}You sold out your people for nothing, \\
\poemll    and made no profit at that price. \\
\poeml \v{13}You made us a laughing stock to our neighbors, \\
\poemll    a source of mockery and derision to those around us. \\
\poeml \v{14}You made us an object lesson among the nations; \\
\poemll    people shake their heads at us.\fnote{The Heb. lacks \fbib{at us}} \\
\poeml \v{15}My dishonor tortures\fnote{Lit. \fbib{dishonor remains before}} me continuously;\fnote{Lit. \fbib{all the day}} \\
\poemll    the shame on my face overwhelms\fnote{Lit. \fbib{covers}} me \\
\poeml \v{16}because of the voice of the one who mocks and reviles, \\
\poemll    because of the enemy and the avenger. \\
\poeml \v{17}All this came upon us, \\
\poemll    yet we did not forsake you, \\
\poemlll       and we have not dealt falsely with your covenant; \\
\poeml \v{18}Our hearts have not turned away; \\
\poemll    our steps have not swerved from your path. \\
\poeml \v{19}Nevertheless, you crushed us in the lair of jackals, \\
\poemll    and covered us in deep darkness.\fnote{Or \fbib{in the shadow of death}} \\
\poeml \v{20}If we had forgotten the name of our God \\
\poemll    or lifted our hands to a foreign god, \\
\poeml \v{21}wouldn't God find out \\
\poemll    since he knows the secrets of the heart? \\
\poeml \v{22}For your sake we are being killed all day long. \\
\poemll    We are thought of as sheep to be slaughtered. \\
\poeml \v{23}Wake up! Why are you asleep, Lord? \\
\poemll    Get up! Don't cast us off forever! \\
\poeml \v{24}Why are you hiding your face? \\
\poemll    Why are you ignoring our affliction and oppression? \\
\poeml \v{25}For we\fnote{Lit. \fbib{our souls}} have collapsed in the dust; \\
\poemll    our bodies cling to the ground. \\
\poeml \v{26}Arise! Deliver us! \\
\poemll    Redeem us according to your gracious love!
\end{poetry}
\labelpsalm{45}
\psalminfo{To the Director: An instruction\fnote{T Lit. \fbib{maskil}} by the Sons of Korah. A love song to the tune of\fnote{T The Heb. lacks \fbib{the tune of}} ``Lilies''.}
\passage{A Royal Wedding Song}

\begin{poetry}
\poeml \v{1}My heart is overflowing with good news; \\
\poemll    I speak what I have composed to the king; \\
\poemlll       my tongue is like the pen of an articulate scribe. \\
\poeml \v{2}You are the most handsome of Adam's descendants; \\
\poemll    grace has anointed your lips; \\
\poemlll       therefore God has blessed you forever. \\
\poeml \v{3}Strap your sword to your side, \\
\poemll    mighty warrior, along with your honor and majesty. \\
\poeml \v{4}In your majesty ride forth for the cause of truth, humility, and righteousness; \\
\poemll    and your strong right hand will teach you awesome things. \\
\poeml \v{5}Your arrows are sharpened \\
\poemll    to penetrate the hearts of the king's enemies. \\
\poemlll       People will fall under you. \\
\poeml \v{6}Your throne, God, exists forever and ever, \\
\poemll    and the scepter of your kingdom is a righteous scepter. \\
\poeml \v{7}You love justice and hate wickedness. \\
\poemll    That is why God, even your God, has anointed you \\
\poemlll       rather than your companions with the oil of gladness. \\
\poeml \v{8}All your clothes are scented with\fnote{The Heb. lacks \fbib{are scented with}} myrrh, aloes, and cassia. \\
\poemll    From ivory palaces stringed instruments have made you glad. \\
\poeml \v{9}The king's daughters are among your honorable women; \\
\poemll    the queen, dressed in gold from Ophir, has taken her place at your right hand.'' \\
\poeml \v{10}Listen, daughter! Consider and pay attention. \\
\poemll    Forget your people and your father's house, \\
\poeml \v{11}and the king will greatly desire your beauty. \\
\poemll    Because he is your lord, you should bow in respect before him. \\
\poeml \v{12}The daughter\fnote{I.e. The people} of Tyre will come with\fnote{The Heb. lacks \fbib{will come with}} a wedding gift; \\
\poemll    wealthy people will entreat your favor. \\
\poeml \v{13}In her chamber,\fnote{The Heb. lacks \fbib{her chamber}} the king's daughter is glorious; \\
\poemll    her clothing is embroidered with gold thread. \\
\poeml \v{14}In embroidered garments \\
\poemll    she is presented to the king. \\
\poeml Her virgin companions who follow her train \\
\poemll    will be presented to you. \\
\poeml \v{15}Filled with joy and gladness, they are presented \\
\poemll    when they enter the king's palace. \\
\poeml \v{16}Your sons will take the place of your ancestors, \\
\poemll    and you will set them up as princes in all the earth. \\
\poeml \v{17}From generation to generation, \\
\poemll    I will cause your name to be remembered. \\
\poemlll       Therefore people will thank you forever and ever.
\end{poetry}
\labelpsalm{46}
\psalminfo{To the Director: A song by the Sons of Korah, to the tune of\fnote{T Lit. \fbib{according to}} ``The Maidens''.}
\passage{God is the Refuge of His People}

\begin{poetry}
\poeml \v{1}God is our refuge and strength, \\
\poemll    a great help in times of distress. \\
\poeml \v{2}Therefore we will not be frightened \\
\poemll    when the earth roars, \\
\poeml when the mountains shake in the depths of the seas, \\
\poeml \v{3}when its waters roar and rage, \\
\poemlll       when the mountains tremble despite their pride.\fnote{Or \fbib{tumult}}
\end{poetry}
\interlude{Interlude}

\begin{poetry}
\poeml \v{4}Look! There is a river \\
\poemll    whose streams make the city of God rejoice, \\
\poemlll       even the Holy Place of the Most High. \\
\poeml \v{5}Since God is in her midst, \\
\poemll    she will not be shaken. \\
\poeml God will help her \\
\poemll    at the break of dawn. \\
\poeml \v{6}The nations roared; \\
\poemll    the kingdoms were shaken. \\
\poeml His voice boomed; \\
\poemll    the earth melts. \\
\poeml \v{7}The \divine{Lord} of the heavenly armies is with us; \\
\poemll    our refuge is the God of Jacob.
\end{poetry}
\interlude{Interlude}

\begin{poetry}
\poeml \v{8}Come, observe the mighty works of the \divine{Lord}, \\
\poemll    who causes desolation in the earth. \\
\poeml \v{9}He causes wars to cease all over\fnote{Lit. \fbib{cease to the end of}} the earth, \\
\poemll    he causes the bow to break, the spear to snap, \\
\poemlll       the chariots to ignite and burn. \\
\poeml \v{10}Be in awe and know that I am God. \\
\poemll    I will be exalted among the nations. \\
\poemlll       I will be exalted throughout the earth. \\
\poeml \v{11}The \divine{Lord} of the heavenly armies is with us; \\
\poemll    the God of Jacob is our refuge.
\end{poetry}
\interlude{Interlude}
\labelpsalm{47}
\psalminfo{To the Director: A song by the Sons of Korah.}
\passage{The Ruler over the Nations}

\begin{poetry}
\poeml \v{1}Clap your hands, all you peoples! \\
\poemll    Shout to God with a loud cry of joy! \\
\poeml \v{2}For the \divine{Lord}, the Most High, is to be feared, \\
\poemll    a great king over all the earth. \\
\poeml \v{3}He subdued peoples under us, \\
\poemll    and nations under our feet. \\
\poeml \v{4}He chose our inheritance for us, \\
\poemll    even the pride of Jacob whom he loved.
\end{poetry}
\interlude{Interlude}

\begin{poetry}
\poeml \v{5}God has ascended on high with a shout, \\
\poemll    the \divine{Lord} has ascended\fnote{The Heb. lacks \fbib{has ascended}} with the blast of a trumpet. \\
\poeml \v{6}Sing songs to God! \\
\poemll    Sing songs! \\
\poeml Sing songs to our King! \\
\poemll    Sing songs! \\
\poeml \v{7}Indeed, God is king over all the earth; \\
\poemll    sing a song of praise. \\
\poeml \v{8}God is king over the nations; \\
\poemll    God is seated on his holy throne. \\
\poeml \v{9}The nobles among the nations \\
\poemll    have joined the people of the God of Abraham. \\
\poeml For the shields of the earth belong to God; \\
\poemll    he is greatly exalted.
\end{poetry}
\labelpsalm{48}
\psalminfo{A song: Lyrics\fnote{T Or \fbib{A song: A song}} by the Sons of Korah.}
\passage{Zion, City of God}

\begin{poetry}
\poeml \v{1}Great is the \divine{Lord}! \\
\poemll    For he is to be praised greatly, \\
\poeml even in the city of our God, \\
\poemll    his holy mountain. \\
\poeml \v{2}Beautifully situated, \\
\poemll    the joy of all the earth, \\
\poeml Mount Zion towards the north,\fnote{Or \fbib{on the northern side}} \\
\poemll    the city of the great King. \\
\poeml \v{3}Within her citadels \\
\poemll    God is known as a place of refuge. \\
\poeml \v{4}Behold, when the kings assembled together, \\
\poemll    when they traveled together, \\
\poeml \v{5}they looked and were awestruck; \\
\poemll    they became afraid and ran away. \\
\poeml \v{6}Trembling seized them there, \\
\poemll    pains like those of a woman in labor, \\
\poeml \v{7}as when an east wind destroyed the ships of Tarshish. \\
\poeml \v{8}Just as we have heard, \\
\poemll    so have we seen; \\
\poeml in the city of the \divine{Lord} of the heavenly armies--- \\
\poemll    even in the city of our God--- \\
\poemlll       God will establish her forever.
\end{poetry}
\interlude{Interlude}

\begin{poetry}
\poeml \v{9}God, we have meditated on your gracious love \\
\poemll    in the midst of your Temple. \\
\poeml \v{10}God, according to your name, \\
\poemll    so is your praise to the ends of the earth. \\
\poemlll       Your right hand is filled with righteousness. \\
\poeml \v{11}Mount Zion will be glad; \\
\poemll    the towns\fnote{Lit. \fbib{daughters}} of Judah will rejoice because of your judgments. \\
\poeml \v{12}March around Zion; \\
\poemll    encircle her; \\
\poemlll       count her towers. \\
\poeml \v{13}Take note of her ramparts; \\
\poemll    investigate her citadels; \\
\poemlll       that you may speak about them to the next generation. \\
\poeml \v{14}For this God is our God forever and ever. \\
\poemll    He will guide us until death.
\end{poetry}
\labelpsalm{49}
\psalminfo{To the Director: A song by the Sons of Korah.}
\passage{The Destiny of the Wicked and the Upright}

\begin{poetry}
\poeml \v{1}Listen to this, all you people! \\
\poemll    Pay attention, all you who live in the world, \\
\poeml \v{2}both average people and those of means,\fnote{Lit. \fbib{both sons of Adam and sons of men}} \\
\poemll    the rich and the poor together. \\
\poeml \v{3}My mouth will speak wisely, \\
\poemll    and I will understand what I think about. \\
\poeml \v{4}I will focus my attention on\fnote{Lit. \fbib{will incline my ear to}} a proverb; \\
\poemll    I will use the harp to expound my riddle. \\
\poeml \v{5}Why should I be afraid when evil days come my way, \\
\poemll    when the wickedness of those who deceive me surrounds me--- \\
\poeml \v{6}those who put confidence in their wealth \\
\poemll    and boast about their great riches? \\
\poeml \v{7}No man can redeem the life of another,\fnote{Lit. \fbib{of a brother}} \\
\poemll    nor can he give to God a sufficient payment for him--- \\
\poeml \v{8}for it would cost too much to redeem his life, \\
\poemll    and the payments would go on forever--- \\
\poeml \v{9}that he should go on living \\
\poemll    and not see corruption. \\
\poeml \v{10}Indeed, he will see wise people die; \\
\poemll    the stupid and the senseless will meet their doom \\
\poemlll       and leave their wealth to others. \\
\poeml \v{11}Their inner thoughts are on\fnote{So MT DSS 4QPs\textsuperscript{c} 4QPs\textsuperscript{j}; LXX reads \fbib{Their graves are}} their homes forever; \\
\poemll    their dwellings from generation to generation. \\
\poemlll       They even name their lands after themselves. \\
\poeml \v{12}But humanity cannot last, despite its conceit;\fnote{So MT; DSS 4QPs\textsuperscript{c} Syr LXX read \fbib{Humans, held in honor, had no understanding;}} \\
\poemll    it will pass away just like the animals.\fnote{So MT; LXX reads \fbib{they resembled senseless animals, and became like them}; DSS 4QPs\textsuperscript{c} read \fbib{they are like animals that perish}} \\
\poeml \v{13}This is the fate of those who are foolish \\
\poemll    and of those who correct their words after they speak.
\end{poetry}
\interlude{Interlude}

\begin{poetry}
\poeml \v{14}Like sheep, they are destined for the realm of the dead,\fnote{Lit. \fbib{for Sheol}; i.e. the realm of the dead} \\
\poemll    with death as their shepherd. \\
\poeml The upright will have dominion over them in the morning; \\
\poemll    their strength will be consumed in the afterlife,\fnote{Lit. \fbib{in Sheol}; i.e. the realm of the dead} \\
\poemlll       so that they have no home. \\
\poeml \v{15}God will truly redeem me from the power\fnote{Lit. \fbib{hand}} of Sheol.\fnote{I.e. the realm of the dead} \\
\poemll    He will surely receive me!
\end{poetry}
\interlude{Interlude}

\begin{poetry}
\poeml \v{16}Don't be afraid when someone gets rich, \\
\poemll    when the glory of his household increases. \\
\poeml \v{17}When he dies, he will not be able to take it all with him\fnote{The Heb. lacks \fbib{with him}}--- \\
\poemll    his possessions\fnote{Or \fbib{glory}} will not follow him to the grave,\fnote{The Heb. lacks \fbib{to the grave}} \\
\poeml \v{18}although he considers himself blessed while he's alive. \\
\poeml Though people praise you for doing well, \\
\poeml \v{19}you will end up like your\fnote{Lit. \fbib{his}} ancestors' generation, \\
\poemlll       never again to see the light of day! \\
\poeml \v{20}Humanity, despite its conceit, does not understand \\
\poemll    that it will perish, just like the animals.
\end{poetry}
\labelpsalm{50}
\psalminfo{A song of Asaph.}
\passage{The Acceptable Sacrifice}

\begin{poetry}
\poeml \v{1}God, the \divine{Lord},\fnote{Or \fbib{The mighty God}} has spoken. \\
\poemll    He has summoned the earth \\
\poemlll       from the rising of the sun to its setting place. \\
\poeml \v{2}From Zion, the perfection of beauty, \\
\poemll    God has shined forth. \\
\poeml \v{3}Our God has appeared and he has not been silent; \\
\poemll    a devouring fire blazed before him, \\
\poemlll       and a mighty storm swirled around him. \\
\poeml \v{4}He summoned the heavens above \\
\poemll    and the earth below,\fnote{The Heb. lacks \fbib{below}} \\
\poemlll       to sit in judgment on his people. \\
\poeml \v{5}``Assemble before me, my saints, \\
\poemll    who have entered into my covenant by sacrifice.'' \\
\poeml \v{6}The heavens revealed his justice, \\
\poemll    for God is himself the judge.
\end{poetry}
\interlude{Interlude}

\begin{poetry}
\poeml \v{7}``Listen, my people, \\
\poemll    for I am making a pronouncement: \\
\poemlll       Israel, I, God, your God, am testifying against you. \\
\poeml \v{8}I do not rebuke you because of your sacrifices; \\
\poemll    indeed, your burnt offerings are continuously before me. \\
\poeml \v{9}I will no longer accept a sacrificial\fnote{The Heb. lacks \fbib{sacrificial}} bull from your household; \\
\poemll    nor goats from your pens. \\
\poeml \v{10}Indeed, every animal of the forest is mine, \\
\poemll    even the cattle on a thousand hills. \\
\poeml \v{11}I know all the birds in the mountains; \\
\poemll    indeed, everything that moves in the field is mine. \\
\poeml \v{12}``If I were hungry, I would not tell you; \\
\poemll    for the world is mine along with everything in it. \\
\poeml \v{13}Why should I eat the flesh of oxen \\
\poemll    or drink the blood of goats? \\
\poeml \v{14}Offer to God a thanksgiving praise; \\
\poemll    pay your vows to the Most High. \\
\poeml \v{15}Call on me in the day of distress; \\
\poemll    I will deliver you, and you will glorify me.'' \\
\poeml \v{16}As for the wicked, God says, \\
\poeml ``How dare you recite my statutes \\
\poemll    or speak about my covenant with your lips! \\
\poeml \v{17}You hate instruction \\
\poemll    and toss my words behind you. \\
\poeml \v{18}When you see a thief, you befriend him, \\
\poemll    and you keep company with adulterers. \\
\poeml \v{19}You give your mouth free reign for evil, \\
\poemll    and your tongue devises deceit. \\
\poeml \v{20}You sit and speak against your brother; \\
\poemll    you slander your own mother's son. \\
\poeml \v{21}These things you did, and I kept silent, \\
\poemll    because you assumed that I was like you. \\
\poeml But now I am going to rebuke you, \\
\poemll    and I will set forth my case before your very own eyes.'' \\
\poeml \v{22}Consider this, you who have forgotten God--- \\
\poemll    Otherwise, I will tear you in pieces \\
\poemlll       and there will be no deliverer: \\
\poeml \v{23}Whoever offers a sacrifice of thanksgiving glorifies me, \\
\poemll    and I will reveal the salvation of God \\
\poemlll       to whomever continues in my way.''\fnote{Lit. \fbib{sets a way}}
\end{poetry}
\labelpsalm{51}
\psalminfo{To the Director: A Davidic Psalm. When the prophet Nathan came to him, after he had gone in to Bathsheba.}
\passage{A Prayer for Cleansing and Pardon}

\begin{poetry}
\poeml \v{1}Have mercy, God, according to your gracious love, \\
\poemll    according to your unlimited compassion, \\
\poemlll       erase my transgressions. \\
\poeml \v{2}Wash me from my iniquity, \\
\poemll    cleanse me from my sin. \\
\poeml \v{3}For I acknowledge my transgression; \\
\poemll    my sin remains continuously before me. \\
\poeml \v{4}Against you, you only, have I sinned, \\
\poemll    and done what was evil in your sight. \\
\poeml As a result, you are just in your pronouncement \\
\poemll    and clear in your judgment. \\
\poeml \v{5}Indeed, in iniquity I was brought forth; \\
\poemll    in sin my mother conceived me. \\
\poeml \v{6}Indeed, you are pleased with truth in the inner person, \\
\poemll    and you will teach me wisdom in my\fnote{The Heb. lacks \fbib{my}} innermost parts. \\
\poeml \v{7}Purge me with hyssop, \\
\poemll    and I will be clean. \\
\poeml Wash me, \\
\poemll    and I will be whiter than snow. \\
\poeml \v{8}Let me know\fnote{Lit. \fbib{hear}} joy and gladness; \\
\poemll    let the bones that you have broken rejoice. \\
\poeml \v{9}Hide your countenance from my sins \\
\poemll    and erase the record of my iniquities. \\
\poeml \v{10}God, create a pure heart in me, \\
\poemll    and renew a right attitude within me. \\
\poeml \v{11}Do not cast me from your presence; \\
\poemll    do not take your Holy Spirit from me. \\
\poeml \v{12}Restore to me the joy of your salvation, \\
\poemll    and let a willing attitude control me. \\
\poeml \v{13}Then I will teach transgressors about your ways, \\
\poemll    and sinners will turn to you. \\
\poeml \v{14}Deliver me from the guilt of shedding blood,\fnote{Lit. \fbib{from bloods}} \\
\poemll    God, God of my salvation. \\
\poemlll       Then my tongue will sing about your righteousness. \\
\poeml \v{15}Lord, open my lips, \\
\poemll    and my mouth will declare your praise. \\
\poeml \v{16}Indeed, you do not delight in sacrifices, \\
\poemll    or I would give them, \\
\poemlll       nor do you desire burnt offerings. \\
\poeml \v{17}True sacrifice to God\fnote{Lit. \fbib{The sacrifice of God}} is a broken spirit. \\
\poemll    A broken and chastened heart, God, \\
\poemlll       you will not despise. \\
\poeml \v{18}Show favor to Zion in your good pleasure; \\
\poemll    and rebuild the walls of Jerusalem. \\
\poeml \v{19}Then you will be pleased with right sacrifices, \\
\poemll    with burnt offerings, and with whole burnt offerings. \\
\poemlll       Then they will offer bulls on your altar.
\end{poetry}
\labelpsalm{52}
\psalminfo{To the Director: A Davidic instruction\fnote{T Lit. \fbib{maskil}} about Doeg, the Edomite, when he went to Saul and told him, ``David went to the house of Abimelech.''}
\passage{A Rebuke to the Deceitful}

\begin{poetry}
\poeml \v{1}Why do you make evil \\
\poemll    the foundation of your boasting, mighty one?\fnote{Or \fbib{O warrior}} \\
\poemlll       God's gracious love never ceases.\fnote{Lit. \fbib{love is all the day}}
\end{poetry}

\begin{poetry}
\poeml \v{2}Your tongue, like a sharp razor, devises wicked things \\
\poemll    and crafts treachery. \\
\poeml \v{3}You love evil rather than good, \\
\poemll    falsehood rather than speaking uprightly.
\end{poetry}
\interlude{Interlude}

\begin{poetry}
\poeml \v{4}You love all words that destroy, you deceitful tongue! \\
\poeml \v{5}But God will tear you down forever; \\
\poemll    he will take you away, \\
\poemlll       even snatching you out of your tent! \\
\poeml He will uproot you from the land of the living.
\end{poetry}
\interlude{Interlude}

\begin{poetry}
\poeml \v{6}The righteous will fear when they see this, \\
\poemll    but then they will laugh at him, saying, \\
\poeml \v{7}``Look, here is a young man who refused to make God his strength; \\
\poemll    instead, he trusted in his great wealth \\
\poemlll       and made his wickedness his strength. \\
\poeml \v{8}But I am like a green olive tree in the house of God; \\
\poemll    I trust in the gracious love of God forever and ever. \\
\poeml \v{9}Therefore I will praise you forever \\
\poemll    because of what you did; \\
\poeml I will proclaim that your name is good \\
\poemll    in the midst of your faithful ones.
\end{poetry}
\labelpsalm{53}
\psalminfo{To the Director: Upon machalath.\fnote{T A Heb. musical term} A Davidic instruction.\fnote{T Lit. \fbib{maskil}}}
\passage{The Fool and God's Response}

\begin{poetry}
\poeml \v{1}Fools say to themselves ``There is no God.'' \\
\poemll    They are corrupt and commit iniquity; \\
\poemlll       not one of them practices what is good. \\
\poeml \v{2}God looks down from the heavens upon humanity\fnote{Lit. \fbib{upon the sons of Adam}} \\
\poemll    to see if anyone shows discernment as he searches for God. \\
\poeml \v{3}All of them\fnote{So MT; DSS 4QPs\textsuperscript{a} lack \fbib{of them}} have fallen away; \\
\poemll    together they have become corrupt; \\
\poemlll       no one does what is good, not even one. \\
\poeml \v{4}Will those who do evil ever learn? \\
\poemll    They devour my people like they devour bread, \\
\poemlll       and never call on God. \\
\poeml \v{5}There the Israelis\fnote{Lit. \fbib{they}} were seized with terror, \\
\poemll    when there was nothing to fear. \\
\poeml For God scattered the bones of those who laid siege against you\fnote{So MT DSS 4QPs\textsuperscript{a}; LXX reads \fbib{of men pleasers}}--- \\
\poemll    you put them to shame,\fnote{So MT DSS 4QPs\textsuperscript{a}; LXX reads \fbib{they were put to shame}} \\
\poemlll       for God rejected them. \\
\poeml \v{6}Would that Israel's deliverance come out of Zion! \\
\poemll    When God restores the fortunes of his people, \\
\poemlll       Jacob will rejoice and Israel will be glad.\fnote{Cf. Ps 14:1-7}
\end{poetry}
\labelpsalm{54}
\psalminfo{To the Director: With stringed instruments. A Davidic instruction,\fnote{T Lit. \fbib{maskil}} when the Ziphites came and told Saul, ``David is hiding among us, is he not?''}
\passage{A Prayer in Times of Trouble}

\begin{poetry}
\poeml \v{1}God, by your name deliver me, \\
\poemll    and by your power vindicate me. \\
\poeml \v{2}God, listen to my prayer, \\
\poemll    and pay attention to the words of my mouth. \\
\poeml \v{3}For the arrogant have arisen against me; \\
\poemll    oppressors have sought to take my life. \\
\poeml They do not keep God in mind!\fnote{Lit. \fbib{before them}}
\end{poetry}
\interlude{Interlude}

\begin{poetry}
\poeml \v{4}Look, God is my helper; \\
\poemll    the Lord is with those who are guarding my life. \\
\poeml \v{5}He will turn the evil upon those who lie in wait for me. \\
\poemll    Cut them off with your truth. \\
\poeml \v{6}With a free will offering I will sacrifice to you; \\
\poemll    I will give thanks to your name, \divine{Lord}, \\
\poemlll       because it is good, \\
\poeml \v{7}for he has delivered me from every trouble, \\
\poemll    and my eyes have seen the end of\fnote{The Heb. lacks \fbib{the end of}} my enemies.
\end{poetry}
\labelpsalm{55}
\psalminfo{To the Director: With stringed instruments. A Davidic instruction.\fnote{T Lit. \fbib{maskil}}}
\passage{Betrayal by a Friend}

\begin{poetry}
\poeml \v{1}Pay attention to my prayer, God, \\
\poemll    and do not hide yourself from my appeal. \\
\poeml \v{2}Pay attention to me and answer me. \\
\poemll    I moan and groan in my thoughts, \\
\poeml \v{3}because of the voice of the enemy, \\
\poeml and because of the oppression of the wicked. \\
\poeml They bring down evil upon me, \\
\poemll    and in anger they hate me. \\
\poeml \v{4}My heart is trembling within me, \\
\poemll    and the terrors of death have assaulted me. \\
\poeml \v{5}Fear and trembling have overwhelmed me, \\
\poemll    and horror has covered me. \\
\poeml \v{6}I said, ``O, who will give me the wings of a dove, \\
\poemll    so that I could fly away and live somewhere else? \\
\poeml \v{7}Look, I want to flee far away; \\
\poemll    I want to settle down in the wilderness.
\end{poetry}
\interlude{Interlude}

\begin{poetry}
\poeml \v{8}I want to deliver myself quickly \\
\poemll    from this windstorm and tempest.'' \\
\poeml \v{9}Confound them, Lord, \\
\poemll    and confuse their speech, \\
\poemlll       because I have seen violence and strife in the city. \\
\poeml \v{10}Day and night they prowl around its walls; \\
\poemll    evil and iniquity are within it. \\
\poeml \v{11}Wickedness is at the center of it; \\
\poemll    fraud and lies never leave its streets. \\
\poeml \v{12}For it is not an enemy who insults me--- \\
\poemll    I could have handled that--- \\
\poeml nor is it someone who hates me and who now arises against me--- \\
\poemll    I could have hidden myself from him--- \\
\poeml \v{13}but it is you--- \\
\poemll    a man whom I treated as my equal--- \\
\poeml my personal confidant, \\
\poemll    my close friend! \\
\poeml \v{14}We had good fellowship together; \\
\poemll    and we even walked together in the house of God! \\
\poeml \v{15}Let death seize them! \\
\poemll    May they be plunged alive into the afterlife,\fnote{Lit. \fbib{into Sheol}; a reference to the realm of the dead} \\
\poeml for wicked things are in their homes \\
\poemll    and among them. \\
\poeml \v{16}I call upon God, \\
\poemll    and the \divine{Lord} will deliver me. \\
\poeml \v{17}Morning, noon, and night, \\
\poemll    I mulled over these things \\
\poeml and cried out in my distress, \\
\poemll    and he heard my voice. \\
\poeml \v{18}He calmly ransomed my soul from the war waged against me, \\
\poemll    for there was a vast crowd who stood against me. \\
\poeml \v{19}God, who is enthroned from long ago, \\
\poemll    will hear me and humble them.
\end{poetry}
\interlude{Interlude}

\begin{poetry}
\poeml Because they do not repent, \\
\poemll    they do not fear God. \\
\poeml \v{20}Each of my friends\fnote{Lit. \fbib{Each one}} raises his hand against his allies; \\
\poemll    each of my friends\fnote{Lit. \fbib{Each one}} breaks his word.\fnote{Lit. \fbib{covenant}} \\
\poeml \v{21}His mouth is as smooth as butter, \\
\poemll    while war is in his heart. \\
\poeml His words were as smooth as olive oil, \\
\poemll    while his sword is drawn. \\
\poeml \v{22}Cast on the \divine{Lord} whatever he sends your way, \\
\poemll    and he will sustain you. \\
\poemlll       He will never allow the righteous to be shaken. \\
\poeml \v{23}But you, God, bring them down to the Pit of corruption;\fnote{I.e. the place of punishment in the afterlife} \\
\poemll    bloodthirsty and deceitful people will not live out half their days. \\
\poemlll       But I put my full confidence in you.
\end{poetry}
\labelpsalm{56}
\psalminfo{To the Director: A special Davidic psalm\fnote{T Heb. \fbib{miktam}} to the tune of\fnote{T Lit. \fbib{David according to}} ``A Silent Dove Far Away,'' when the Philistines seized him in Gath.}
\passage{A Prayer about Trust in God}

\begin{poetry}
\poeml \v{1}Have mercy on me, God, \\
\poemll    because men have harassed me. \\
\poemlll       Those who oppress me have fought against me all day long. \\
\poeml \v{2}Those who watch me all day have harassed me, \\
\poemll    for there are many who fight against me out of conceit. \\
\poeml \v{3}On days when I am afraid, \\
\poemll    I put my trust in you. \\
\poeml \v{4}In God, whose word I praise, \\
\poemll    in God I put my trust. \\
\poemlll       I will not fear what mortal man\fnote{Lit. \fbib{what flesh}} can do to me. \\
\poeml \v{5}All day long people\fnote{Lit. \fbib{they}} distort what I say; \\
\poemll    all their schemes against me are for evil purposes. \\
\poeml \v{6}They gather together \\
\poemll    and hide in ambush. \\
\poeml They watch my every step \\
\poemll    as they lie in wait for my life. \\
\poeml \v{7}Cast them away because of their wickedness. \\
\poemll    In wrath, God, cast down these\fnote{The Heb. lacks \fbib{these}} people! \\
\poeml \v{8}You have kept count of my wanderings. \\
\poemll    Put my tears in your bottle--- \\
\poemlll       have not you recorded them in your book? \\
\poeml \v{9}My enemies will retreat when I call on you.\fnote{The Heb. lacks \fbib{on you}} \\
\poemll    This has been my experience, \\
\poemlll       because God is with me. \\
\poeml \v{10}In God, whose word I praise, \\
\poemll    in the \divine{Lord}, whose word I praise, \\
\poeml \v{11}in God I will put my trust. \\
\poemll    I will not fear what mortal man can do to me. \\
\poeml \v{12}God, I have taken vows before you;\fnote{Lit. \fbib{your vows are upon me}} \\
\poemll    therefore I will offer thanksgiving sacrifices to you. \\
\poeml \v{13}For you have delivered me\fnote{Or \fbib{my soul}} from death \\
\poemll    and my feet from stumbling, \\
\poemlll       so that I may walk before God in the light of the living!
\end{poetry}
\labelpsalm{57}
\psalminfo{To the Director: A special Davidic psalm\fnote{T Heb. \fbib{miktam}} to the tune of\fnote{T Lit. \fbib{David according to}} ``Do Not Destroy,'' when he fled from Saul into a cave.}
\passage{A Prayer for Deliverance}

\begin{poetry}
\poeml \v{1}Have mercy on me, God, have mercy, \\
\poemll    for in you I\fnote{Or \fbib{my soul}} have placed my trust. \\
\poeml Even in the shadow of your wings \\
\poemll    will I find my refuge until this calamity passes. \\
\poeml \v{2}I call upon the God Most High; \\
\poemll    to the God who completes what he began\fnote{The Heb. lacks \fbib{what he began}} in me. \\
\poeml \v{3}He will send help from heaven to deliver me \\
\poemll    from those who harass and despise me.
\end{poetry}
\interlude{Interlude}

\begin{poetry}
\poemlll       God will send his gracious love and truth. \\
\poeml \v{4}I am\fnote{Or \fbib{My soul is}} surrounded by lions. \\
\poemll    I lie down with those who burn with fire--- \\
\poeml that is, with people whose teeth are like spears and arrows--- \\
\poemll    whose tongues are like sharp swords. \\
\poeml \v{5}Be exalted above the heavens, God! \\
\poemll    May your glory cover the earth! \\
\poeml \v{6}They have set a snare for my feet, \\
\poemll    which makes me\fnote{Lit. \fbib{my soul}} depressed. \\
\poeml They dug a pit in front of me, \\
\poemll    but they are the ones who fell into it!
\end{poetry}
\interlude{Interlude}

\begin{poetry}
\poeml \v{7}My heart is committed, God, \\
\poemll    my heart is committed, \\
\poemlll       so I will sing and play music. \\
\poeml \v{8}Wake up, my soul,\fnote{Lit. \fbib{glory}} \\
\poemll    wake up, lyre and harp! \\
\poemlll       I will awaken at dawn. \\
\poeml \v{9}I will exalt you among the peoples, Lord. \\
\poemll    I will play music among the nations. \\
\poeml \v{10}For your gracious love is great, \\
\poemll    extending even to the heavens, \\
\poemlll       and your truth even to the skies. \\
\poeml \v{11}Be exalted above the heavens, God! \\
\poemll    May your glory cover the earth!
\end{poetry}
\labelpsalm{58}
\psalminfo{To the Director: A special Davidic psalm\fnote{T Heb. \fbib{miktam}} to the tune of\fnote{T Lit. \fbib{David according to}} ``Do Not Destroy''.}
\passage{A Prayer for Justice}

\begin{poetry}
\poeml \v{1}How is it that by remaining silent you can speak righteously? \\
\poemll    How can you judge people fairly? \\
\poeml \v{2}As a matter of fact, in your heart you plan iniquities! \\
\poemll    In the land your hands are violent! \\
\poeml \v{3}The wicked go astray from the womb; \\
\poemll    they go astray, telling lies even from birth. \\
\poeml \v{4}Their venom is like a poisonous snake; \\
\poemll    even like a deaf serpent that shuts its ears, \\
\poeml \v{5}refusing to hear the voice of the snake charmer, \\
\poemll    the cunning enchanter. \\
\poeml \v{6}God, shatter their teeth in their mouths; \\
\poemll    \divine{Lord}, break the fangs of the young lions! \\
\poeml \v{7}May they flow away like rain water that runs off, \\
\poemll    may they become like someone who shoots broken arrows. \\
\poeml \v{8}May they be like a snail that dries up as it crawls; \\
\poemll    like a woman's stillborn baby, who never saw the sun. \\
\poeml \v{9}Before your clay pots are placed on a fire of burning\fnote{The Heb. lacks \fbib{a fire of burning}} thorns--- \\
\poemll    whether green or ablaze--- \\
\poemlll       wrath will sweep them away like a storm. \\
\poeml \v{10}The righteous person will rejoice when he sees your\fnote{The Heb. lacks \fbib{your}} vengeance; \\
\poemll    when he washes his feet in the blood of the wicked. \\
\poeml \v{11}A person will say, \\
\poemll    ``Certainly, the righteous are rewarded; \\
\poemlll       certainly there is a God who judges the earth.''
\end{poetry}
\labelpsalm{59}
\psalminfo{To the Director: A special Davidic psalm\fnote{T Heb. \fbib{miktam}} to the tune of\fnote{T Lit. \fbib{David according to}} ``Do Not Destroy,'' when Saul sent men to watch the house in order to kill him.}
\passage{A Prayer for Deliverance and Justice}

\begin{poetry}
\poeml \v{1}Save me from my enemies, my God! \\
\poemll    Keep me safe from those who rise up against me. \\
\poeml \v{2}Save me from those who practice evil; \\
\poemll    deliver me from bloodthirsty men. \\
\poeml \v{3}Look, they lie in ambush for my life; \\
\poemll    these violent men gather together against me, \\
\poemlll       but not because of any transgression or sin of mine, \divine{Lord}. \\
\poeml \v{4}Without any fault on my part, \\
\poemll    they rush together and prepare themselves. \\
\poeml Get up! \\
\poemll    Come help me! \\
\poemlll       Pay attention! \\
\poeml \v{5}You, \divine{Lord} God of the Heavenly Armies, God of Israel, \\
\poemll    stir yourself up to punish all the nations. \\
\poemlll       Show no mercy to those wicked transgressors.
\end{poetry}
\interlude{Interlude}

\begin{poetry}
\poeml \v{6}At night they return like howling dogs; \\
\poemll    they prowl around the city. \\
\poeml \v{7}Look what pours out of their mouths! \\
\poemll    They use their lips like swords, \\
\poemlll       saying\fnote{The Heb. lacks \fbib{saying}} ``Who will hear us?'' \\
\poeml \v{8}But you, \divine{Lord}, will laugh at them; \\
\poemll    you will mock all the nations. \\
\poeml \v{9}My Strength, I will watch for you, \\
\poemll    for God is my fortress. \\
\poeml \v{10}My God of Gracious Love will meet me; \\
\poemll    God will enable me to see what happens\fnote{The Heb. lacks \fbib{what happens}} to my enemies. \\
\poeml \v{11}Don't kill them! \\
\poemll    Otherwise, my people may forget. \\
\poeml By your power make them stumble around; \\
\poemll    bring them down low, \\
\poemlll       Lord, our Shield. \\
\poeml \v{12}The sin of their mouth is the word on their lips. \\
\poemll    They will be caught in their own conceit; \\
\poemlll       for they speak curses and lies. \\
\poeml \v{13}Go ahead and destroy them in anger! \\
\poemll    Wipe them out, \\
\poeml and they will know to the ends of the earth \\
\poemll    that God rules over Jacob.\fnote{Or \fbib{know that God rules over Jacob to the ends of the earth}}
\end{poetry}
\interlude{Interlude}

\begin{poetry}
\poeml \v{14}At night they return like howling dogs; \\
\poemll    they prowl around the city. \\
\poeml \v{15}They scavenge for food. \\
\poemll    If they find nothing, \\
\poemlll       they become hungry and growl. \\
\poeml \v{16}But I will sing of your power \\
\poemll    and in the morning I will shout for joy about your gracious love. \\
\poeml For you have been a fortress for me; \\
\poemll    and a refuge when I am distressed.\fnote{Lit. \fbib{refuge in the day of my distress}} \\
\poeml \v{17}My Strength, I will sing praises to you, \\
\poemll    for you, God of Gracious Love, are my fortress.
\end{poetry}
\labelpsalm{60}
\psalminfo{To the Director: A special Davidic psalm to the tune of\fnote{T Lit. \fbib{David according to}} ``Lily of The Covenant,'' for teaching about his battle with Aram-naharaim and Aram-zobah, when Joab returned and attacked 12,000 Edomites in the Salt Valley.\fnote{T I.e. Dead Sea region}}
\passage{A Prayer for God's Help against Adversaries}

\begin{poetry}
\poeml \v{1}God, you have cast us off; \\
\poemll    you have breached our defenses \\
\poeml and you have become enraged. \\
\poemll    Return to us! \\
\poeml \v{2}You made the earth quake; \\
\poemll    you broke it open. \\
\poeml Repair its fractures, \\
\poemll    because it has shifted. \\
\poeml \v{3}You made your people go through hard times; \\
\poemll    you had us drink wine that makes us stagger. \\
\poeml \v{4}But you have given a banner to those who fear you, \\
\poemll    so they may display it in honor of truth.\fnote{Or \fbib{display it because of the archer}}
\end{poetry}
\interlude{Interlude}

\begin{poetry}
\poeml \v{5}So your loved ones may be delivered, \\
\poemll    save us by your power\fnote{Lit. \fbib{right hand}} \\
\poemlll       and answer us quickly! \\
\poeml \v{6}Then God spoke in his holiness, \\
\poeml ``I will rejoice--- \\
\poemll    I will divide Shechem; \\
\poemlll       I will portion out the Succoth Valley. \\
\poeml \v{7}Gilead belongs to me, \\
\poemll    and Manasseh is mine. \\
\poeml Ephraim is my helmet, \\
\poemll    and Judah my scepter. \\
\poeml \v{8}Moab is my wash basin; \\
\poemll    over Edom I will throw my shoes; \\
\poemlll       over Philistia I will celebrate my triumph.'' \\
\poeml \v{9}Who will lead me to the fortified city? \\
\poemll    Who will lead me to Edom? \\
\poeml \v{10}Aren't you the one, God, who has cast us off? \\
\poemll    Didn't you refuse, God, to accompany our armies? \\
\poeml \v{11}Help us in our distress, \\
\poemll    for human help is worthless. \\
\poeml \v{12}Through God we will fight\fnote{Lit. \fbib{will do}} valiantly; \\
\poemll    and it is he who will crush our enemies.\fnote{vv.5-12 is the same as Psalm 108:6-13.}
\end{poetry}
\labelpsalm{61}
\psalminfo{To the Director: A composition\fnote{T The Heb. lacks \fbib{A composition}} by David for stringed instruments.}
\passage{A Prayer for God's Protection}

\begin{poetry}
\poeml \v{1}God, hear my cry; \\
\poemll    pay attention to my prayer. \\
\poeml \v{2}From the end of the earth I will cry to you \\
\poemll    whenever my heart is overwhelmed. \\
\poemlll       Place me on the rock that's too high for me. \\
\poeml \v{3}For you have been a refuge for me, \\
\poemll    a tower of strength before the enemy. \\
\poeml \v{4}Let me make my home in your tent forever; \\
\poemll    let me hide under the shelter of your wings.
\end{poetry}
\interlude{Interlude}

\begin{poetry}
\poeml \v{5}For you, God, have heard my promises; \\
\poemll    you have assigned to me\fnote{The Heb. lacks \fbib{to me}} the heritage of those who fear your name. \\
\poeml \v{6}Add day after day to the king's life; \\
\poemll    may his years continue\fnote{The Heb. lacks \fbib{continue}} for many generations. \\
\poeml \v{7}May he be enthroned before God forever; \\
\poemll    Appoint your\fnote{The Heb. lacks \fbib{your}} gracious love and truth to guard him. \\
\poeml \v{8}So I will sing songs to your name forever; \\
\poemll    I will fulfill my promises day by day.
\end{poetry}
\labelpsalm{62}
\psalminfo{To the Director: According to Jeduthun's style. A Davidic Psalm.}
\passage{A Psalm of Trust in God}

\begin{poetry}
\poeml \v{1}My soul rests quietly only when it looks\fnote{The Heb. lacks \fbib{when it looks}} to God; \\
\poemll    from him comes my deliverance. \\
\poeml \v{2}He alone is my rock, my deliverance, and my high tower; \\
\poemll    nothing will shake me. \\
\poeml \v{3}How long will you rage against someone? \\
\poemll    Would you attack him \\
\poemlll       as if he were a leaning wall or a tottering fence? \\
\poeml \v{4}They plan to cast him down from his exalted position. \\
\poemll    They delight in lies; \\
\poeml their mouth utters blessings, \\
\poemll    while their heart is cursing.
\end{poetry}
\interlude{Interlude}

\begin{poetry}
\poeml \v{5}My soul, be quiet before God, \\
\poemll    for from him comes my hope. \\
\poeml \v{6}He alone is my rock, my deliverance, and my high tower; \\
\poemll    nothing will shake me. \\
\poeml \v{7}I rely on God who is my deliverance and my glory; \\
\poemll    he is my strong rock, \\
\poemlll       and my refuge is in God. \\
\poeml \v{8}People, in every situation put your trust in God;\fnote{Lit. \fbib{in him}} \\
\poemll    pour out your heart before him; \\
\poemlll       for God is a refuge for us.
\end{poetry}
\interlude{Interlude}

\begin{poetry}
\poeml \v{9}Human beings\fnote{Lit. \fbib{sons of Adam}} are a mere vapor, \\
\poemll    while people in high positions\fnote{Lit. \fbib{sons of man}} are not what they appear. \\
\poemll    When they are placed on the scales, they weigh nothing; \\
\poemlll       even when weighed together, they are less than nothing. \\
\poeml \v{10}Don't trust in oppression \\
\poemll    or put false hope in stealing; \\
\poeml if you become wealthy, \\
\poemll    do not set your heart on it. \\
\poeml \v{11}God spoke once, \\
\poemll    but I heard it twice, \\
\poemlll       ``Power belongs to God.'' \\
\poeml \v{12}Also to you, Lord, belongs gracious love, \\
\poemll    because you reward each person according to what he does.
\end{poetry}
\labelpsalm{63}
\psalminfo{A Davidic Psalm, while he was in the Judean wilderness.}
\passage{Joyful Trust in God}

\begin{poetry}
\poeml \v{1}God, you are my God! \\
\poemll    I will fervently seek you. \\
\poeml My soul thirsts for you; \\
\poemll    my flesh longs for you in a dry, weary, and parched land. \\
\poeml \v{2}So I have looked for you in the sanctuary, \\
\poemll    to behold your power and glory. \\
\poeml \v{3}Because your gracious love is better than life itself, \\
\poemll    my lips will praise you. \\
\poeml \v{4}So I will bless you as long as I live; \\
\poemll    I will lift up my hands in your name. \\
\poeml \v{5}Just as I am satisfied with the choicest of foods,\fnote{Lit. \fbib{with marrow and fatness}} \\
\poemll    so my lips will praise you joyfully. \\
\poeml \v{6}When I think of you in bed, \\
\poemll    I will meditate on you in the night watches. \\
\poeml \v{7}For you have been my strength, \\
\poemll    and in the shadow of your wings I will shout for joy. \\
\poeml \v{8}My soul clings to you, \\
\poemll    even as your right hand supports me. \\
\poeml \v{9}But as for those who seek to destroy me, \\
\poemll    they will go down to the depths of the earth; \\
\poeml \v{10}May they be given over to the power of\fnote{The Heb. lacks \fbib{to the power of}} the sword; \\
\poemll    may they become carrion for jackals. \\
\poeml \v{11}But as for the king, \\
\poemll    he will rejoice in God. \\
\poeml Indeed, everyone who swears by God\fnote{Lit. \fbib{him}} will exult, \\
\poemll    because the mouths of liars will be silenced.
\end{poetry}
\labelpsalm{64}
\psalminfo{To the Director: A Davidic Psalm.}
\passage{A Prayer for Protection}

\begin{poetry}
\poeml \v{1}Hear, God, as I express my concern; \\
\poemll    protect me\fnote{Lit. \fbib{my life}} from fear of the enemy. \\
\poeml \v{2}Hide me from the secret plots of the wicked, \\
\poemll    from the mob of those who practice evil, \\
\poeml \v{3}who sharpen their tongues like swords, \\
\poemll    and aim their bitter words like arrows, \\
\poeml \v{4}shooting at the innocent from concealment. \\
\poeml Suddenly they shoot, fearing nothing. \\
\poeml \v{5}They concoct an evil scheme for themselves; \\
\poeml they enumerate their hidden snares; \\
\poemll    they say, ``Who will see them?''\fnote{Lit. \fbib{see him}; or \fbib{see it}} \\
\poeml \v{6}They devise wicked schemes, saying, \\
\poemll    ``We have completed our plans, \\
\poemlll       hiding them deep in our hearts.'' \\
\poeml \v{7}But God shot an arrow at them, \\
\poemll    and they were wounded immediately. \\
\poeml \v{8}They tripped over their own tongues, \\
\poemll    and everyone who was watching ran away. \\
\poeml \v{9}Everyone was gripped with fear \\
\poemll    and acknowledged God's deeds, \\
\poemlll       understanding what he had done. \\
\poeml \v{10}The righteous rejoiced in the \divine{Lord}, \\
\poemll    because they had fled to him for refuge. \\
\poemlll       Let all the upright in heart exult.
\end{poetry}
\labelpsalm{65}
\psalminfo{To the Director: A song. Lyrics\fnote{T Lit. \fbib{A song. A song}} by David.}
\passage{A Song of Praise to God}

\begin{poetry}
\poeml \v{1}In Zion, God, praise silently awaits you, \\
\poemll    and vows will be paid to you. \\
\poeml \v{2}Since you hear prayer, \\
\poemll    everybody will come to you. \\
\poeml \v{3}My acts of iniquity---they overwhelm me! \\
\poemll    Our transgressions---you blot them out! \\
\poeml \v{4}How blessed is the one you choose, \\
\poemll    the one you cause to live in your courts. \\
\poeml We will be satisfied with the goodness of your house, \\
\poemll    yes, even with the holiness of your Temple. \\
\poeml \v{5}With awesome deeds of justice\fnote{Or \fbib{righteousness}} \\
\poemll    you will answer us, God our Deliverer; \\
\poeml you are\fnote{The Heb. lacks \fbib{you are}} the confidence for everyone at the ends of the earth, \\
\poemll    even for those far away overseas. \\
\poeml \v{6}The One who established the mountains by his strength \\
\poemll    is clothed with omnipotence. \\
\poeml \v{7}He calmed the roar of seas, \\
\poemll    the roaring of the waves, \\
\poemlll       and the turmoil of the peoples. \\
\poeml \v{8}Those living at the furthest ends of the earth\fnote{The Heb. lacks \fbib{of the earth}} are seized by fear because of your miraculous deeds. \\
\poeml You make the going forth of the morning and the evening shout for joy. \\
\poeml \v{9}You take care of the earth, \\
\poemll    you water it, \\
\poemlll       and you enrich it greatly with the river of God that overflows with water. \\
\poeml You provide grain for them, \\
\poemll    for you have ordained it this way. \\
\poeml \v{10}You fill the furrows of the field with water \\
\poemll    so that their ridges overflow. \\
\poeml You soften them with rain showers; \\
\poemll    their sprouts you have blessed. \\
\poeml \v{11}You crown the year with your goodness; \\
\poemll    your footsteps drop prosperity behind them. \\
\poeml \v{12}The wilderness pastures drip with dew,\fnote{The Heb. lacks \fbib{with dew}} \\
\poemll    and the hills wrap themselves with joy. \\
\poeml \v{13}The meadows are clothed with flocks of sheep, \\
\poemll    and the valleys are covered with grain. \\
\poeml They shout for joy; \\
\poemll    yes, they burst out in song!
\end{poetry}
\labelpsalm{66}
\psalminfo{To the Director: A song. A Psalm.}
\passage{A Song of Praise}

\begin{poetry}
\poeml \v{1}Shout praise to God all the earth! \\
\poeml \v{2}Sing praise about the glory of his name.\fnote{I.e. \fbib{reputation}; and so throughout the Psalms} \\
\poemll    Make his praise glorious. \\
\poeml \v{3}Say to God: ``How awesome are your works! \\
\poemll    Because of your great strength \\
\poemlll       your enemies cringe before you.'' \\
\poeml \v{4}The whole earth worships you. \\
\poemll    They sing praise to you. \\
\poemlll       They sing praise to your name.
\end{poetry}
\interlude{Interlude}

\begin{poetry}
\poeml \v{5}Come and see the awesome works of God \\
\poemll    on behalf of human beings: \\
\poeml \v{6}He turned the sea into dry land. \\
\poemll    Israel\fnote{Lit. \fbib{He}} crossed the river on foot; \\
\poemlll       let us rejoice in him. \\
\poeml \v{7}He rules by his power forever, \\
\poemll    his eyes watching over the nations. \\
\poemlll       Do not let the rebellious exalt themselves.
\end{poetry}
\interlude{Interlude}

\begin{poetry}
\poeml \v{8}Bless our God, people, \\
\poemll    and let the sound of his praise be heard. \\
\poeml \v{9}He gives us life \\
\poemll    and does not permit our feet to slip. \\
\poeml \v{10}For you, God, tested us, \\
\poemll    to purify us like fine silver. \\
\poeml \v{11}You have led us into a trap\fnote{Or \fbib{net}} \\
\poemll    and set burdens on our backs. \\
\poeml \v{12}You caused men to ride over us.\fnote{Lit. \fbib{over our head}} \\
\poemll    You brought us through fire and water, \\
\poemlll       but you led us to abundance. \\
\poeml \v{13}I will come to your house with burnt offerings. \\
\poemll    I will fulfill my vows to you \\
\poeml \v{14}that my lips uttered and that my mouth spoke \\
\poemll    when I was in trouble. \\
\poeml \v{15}I will offer to you burnt offerings of fat, \\
\poemll    along with the smoke of the sacrifice of rams. \\
\poemlll       I will offer bulls along with goats.
\end{poetry}
\interlude{Interlude}

\begin{poetry}
\poeml \v{16}Come and listen, all of you who fear God, \\
\poemll    and I will tell you what he did for me. \\
\poeml \v{17}I called aloud to him \\
\poemll    and praised him with my tongue. \\
\poeml \v{18}Were I to cherish iniquity in my heart, \\
\poemll    the Lord would not listen to me. \\
\poeml \v{19}Surely God has heard, \\
\poemll    and he paid attention to my\fnote{Lit. \fbib{to the voice of my}} prayers. \\
\poeml \v{20}Blessed be God, who did not turn away my prayers \\
\poemll    nor his gracious love from me.
\end{poetry}
\labelpsalm{67}
\psalminfo{To the Director of music: Accompanied by stringed instruments. A Psalm. A song.}
\passage{A Call to Thanksgiving}

\begin{poetry}
\poeml \v{1}May God show us favor and bless us; \\
\poemll    may he truly show us his favor.\fnote{Lit. \fbib{he cause his face to shine on us}}
\end{poetry}
\interlude{Interlude}

\begin{poetry}
\poeml \v{2}Let your ways be known by all the nations of the earth, \\
\poemll    along with your deliverance. \\
\poeml \v{3}Let the people thank you, God. \\
\poemll    Let all the people thank you. \\
\poeml \v{4}Let the nations rejoice and sing for joy, \\
\poemll    because you judge people with fairness \\
\poemlll       and you govern the people of the earth.
\end{poetry}
\interlude{Interlude}

\begin{poetry}
\poeml \v{5}Let the people thank you, God; \\
\poemll    let all the people thank you. \\
\poeml \v{6}May the earth yield its produce. \\
\poemll    May God, our God, bless us. \\
\poeml \v{7}May God truly bless us \\
\poemll    so that all the peoples\fnote{Lit. \fbib{ends}} of the earth will fear him.
\end{poetry}
\labelpsalm{68}
\psalminfo{To the Director of music: A Psalm. A song.}
\passage{A Song of Praise to God}

\begin{poetry}
\poeml \v{1}God arises, \\
\poemll    and his enemies are scattered. \\
\poemlll       Those who hate him flee from his presence.\fnote{Lit. \fbib{face}} \\
\poeml \v{2}As smoke is driven away, so you drive them away. \\
\poemll    As wax melts in the presence of fire, \\
\poemlll       so the wicked die in the presence of God. \\
\poeml \v{3}But the righteous rejoice and exult before God; \\
\poemll    they are overwhelmed with joy. \\
\poeml \v{4}Sing to God! \\
\poemll    Sing praise to his name! \\
\poemlll       Exalt the one who rides on the clouds. \\
\poeml The \divine{Lord} is his name. \\
\poemll    Be jubilant in his presence. \\
\poeml \v{5}A father to orphans and an advocate for widows \\
\poemll    is God in his holy dwelling place. \\
\poeml \v{6}God causes the lonely to dwell in families.\fnote{Lit. \fbib{in a house}} \\
\poemll    He leads prisoners into prosperity, \\
\poemlll       but rebels live on parched land. \\
\poeml \v{7}God, when you led out your people, \\
\poemll    when you marched through the desert,
\end{poetry}
\interlude{Interlude}

\begin{poetry}
\poeml \v{8}the land quaked. \\
\poeml Indeed, the heavens poured down rain \\
\poemll    from the presence of God, \\
\poemlll       this God of Sinai, \\
\poemll    from the presence of God, \\
\poemlll       the God of Israel. \\
\poeml \v{9}God, you poured out abundant rain on your inheritance. \\
\poemll    When Israel\fnote{Lit. \fbib{it}} was weary, you sustained her. \\
\poeml \v{10}Your people live\fnote{Or \fbib{tribe lives}} there; \\
\poemll    you sustain the needy\fnote{Or \fbib{afflicted}} with your goodness, God. \\
\poeml \v{11}The Lord issues a command. \\
\poemll    Numerous are the women who announce the news: \\
\poeml \v{12}``Kings of armies retreat and flee, \\
\poemll    while the lady of the house divides the spoil. \\
\poeml \v{13}When you men lie down among the sheepfolds, \\
\poemll    you are like\fnote{The Heb. lacks \fbib{you are like}} the wings of the dove covered with silver, \\
\poemlll       with its feathers in glittering gold.'' \\
\poeml \v{14}When the Almighty scattered the kings there, \\
\poemll    there was snow on Mt. Zalmon. \\
\poeml \v{15}The mountain of God is as the mountain of Bashan; \\
\poemll    a mountain of many peaks is Mount Bashan. \\
\poeml \v{16}You mountains of many peaks, why do you watch with envy \\
\poemll    the mountain in which God has chosen to dwell? \\
\poemlll       Indeed, the \divine{Lord} will live there forever. \\
\poeml \v{17}God's chariots were many thousands. \\
\poemll    The Lord was there with them at Sinai in holiness. \\
\poeml \v{18}You ascended to the heights, \\
\poemll    you took captives. \\
\poeml You received gifts among mankind, \\
\poemll    even the rebellious, \\
\poemlll       so the \divine{Lord} God may live there.\fnote{The Heb. lacks \fbib{there}} \\
\poeml \v{19}Blessed be the Lord who daily carries us. \\
\poemll    God is our deliverer. \\
\poeml \v{20}God is for us the God of our deliverance. \\
\poemll    The Lord \divine{God} rescues us from death. \\
\poeml \v{21}God surely strikes the heads of his enemies, \\
\poemll    even the hairy heads of those who continue in their guilt. \\
\poeml \v{22}The Lord says, ``From Bashan I will bring them, \\
\poemll    I will bring them from the depths of the sea, \\
\poeml \v{23}that your feet may wade through blood. \\
\poeml The tongues of your dogs will have their portions \\
\poemll    from your enemies.'' \\
\poeml \v{24}They have observed your processions, God, \\
\poemll    the processions of my God, \\
\poemlll       my king, in the sanctuary. \\
\poeml \v{25}The singers are in front, \\
\poemll    the musicians follow, \\
\poemlll       strumming their stringed instruments \\
\poeml among the maidens who are playing their tambourines. \\
\poeml \v{26}Bless God in the great congregation, \\
\poemll    the \divine{Lord} who is the fountain of Israel. \\
\poeml \v{27}Little Benjamin is there, leading them, \\
\poemll    and the princes of Judah all together \\
\poemlll       with the princes of Zebulun and the princes of Naphtali. \\
\poeml \v{28}Summon the power of your God, \\
\poemll    the power, God, that you have shown us. \\
\poeml \v{29}Because of your Temple in Jerusalem, \\
\poemll    kings bring tribute to you. \\
\poeml \v{30}Rebuke the wildlife that lives among the reeds, \\
\poemll    the nations that congregate like bulls and cows, \\
\poeml humbling themselves with pieces of silver, \\
\poemll    for God\fnote{Lit. \fbib{he}} scatters the nations that delight in battle. \\
\poeml \v{31}Envoys will come from Egypt. \\
\poemll    Let the Ethiopians stretch out their hands to God. \\
\poeml \v{32}You kingdoms of the earth, sing to God! \\
\poemll    Sing praises to the Lord,
\end{poetry}
\interlude{Interlude}

\begin{poetry}
\poeml \v{33}to the one who rides the heavens, the ancient heavens. \\
\poemll    Behold! He thunders with a mighty voice. \\
\poeml \v{34}Ascribe power to God, whose glory is over Israel, \\
\poemll    whose power is in the skies. \\
\poeml \v{35}You are awesome, God, from your sanctuaries. \\
\poemll    The God of Israel is the one \\
\poemlll       who gives strength and power to the people. \\
\poeml Blessed be God!
\end{poetry}
\labelpsalm{69}
\psalminfo{To the Director: To the tune of\fnote{T Lit. \fbib{According to}} ``The Lilies''. Davidic.}
\passage{When God Seems Distant}

\begin{poetry}
\poeml \v{1}Deliver me, God, \\
\poemll    because the waters are up to my neck.\fnote{Lit. \fbib{soul}} \\
\poeml \v{2}I am sinking in deep mire, \\
\poemll    and there is no solid ground.\fnote{Or \fbib{foothold}} \\
\poeml I have come into deep water, \\
\poemll    and the flood overwhelms me. \\
\poeml \v{3}I am exhausted from calling for help. \\
\poemll    My throat is parched. \\
\poemlll       My eyes are strained from looking for God. \\
\poeml \v{4}Those who hate me without cause \\
\poemll    are more than the hairs of my head. \\
\poeml My persecutors are mighty, \\
\poemll    and they want to destroy me. \\
\poemlll       Must I be forced to return what I did not steal? \\
\poeml \v{5}God, you know my sins, \\
\poemll    and my guilt is not hidden from you. \\
\poeml \v{6}Do not let those who look up to you be ashamed \\
\poemll    because of me, \\
\poemlll       Lord God of the Heavenly Armies. \\
\poeml Let not those who seek you be humiliated \\
\poemll    because of me, \\
\poemlll       God of Israel. \\
\poeml \v{7}I am being mocked because of you. \\
\poemll    Dishonor overwhelms me. \\
\poeml \v{8}I am a stranger to my brothers, \\
\poemll    a foreigner to my mother's sons. \\
\poeml \v{9}Zeal for your house consumes me, \\
\poemll    and the mockeries of those who insult you fall on me. \\
\poeml \v{10}I weep and fast, \\
\poemll    and I am mocked for it. \\
\poeml \v{11}When I dressed in sackcloth, \\
\poemll    I became an object of gossip among them. \\
\poeml \v{12}The prominent people mock me, \\
\poemll    composing drinking songs.
\passage{Seeking God for Deliverance}
\poeml \v{13}As for me, \divine{Lord}, may my prayer to you come at a favorable time. \\
\poemll    God, in the abundance of your gracious love, \\
\poemlll       answer me with your sure deliverance. \\
\poeml \v{14}Rescue me from the mud \\
\poemll    and do not let me sink. \\
\poeml Rescue me from those who hate me, \\
\poemll    and from the deep waters. \\
\poeml \v{15}Let neither the floodwaters overwhelm me \\
\poemll    nor let the deep swallow me up, \\
\poemlll       nor the mouth of the well close over me. \\
\poeml \v{16}Answer me, \divine{Lord}, for your gracious love is good; \\
\poemll    Turn to me in keeping with your great compassion, \\
\poeml \v{17}and\fnote{So MT; DSS 4QPs\textsuperscript{a} lack \fbib{and}} do not ignore your servant, \\
\poemll    because I am in distress. \\
\poemlll       Hurry to answer me! \\
\poeml \v{18}Draw near and redeem me; \\
\poemll    ransom me because of my enemies. \\
\poeml \v{19}Truly you know my reproach, shame, and disgrace. \\
\poemll    All my enemies are known to\fnote{Lit. \fbib{are before}} you. \\
\poeml \v{20}Insults broke my heart. \\
\poemll    I despaired and looked for sympathy; \\
\poeml but there was none, \\
\poemll    for comforters, but I found none. \\
\poeml \v{21}They put poison in my food, \\
\poemll    in my thirst they forced me to drink vinegar. \\
\poeml \v{22}May their dining\fnote{The Heb. lacks \fbib{dining}} tables entrap them, \\
\poemll    and become a snare for their allies. \\
\poeml \v{23}May their eyes be blinded \\
\poemll    and may their bodies tremble continuously. \\
\poeml \v{24}May you pour out your fury on them. \\
\poemll    May your burning anger overtake them. \\
\poeml \v{25}May their camp become desolate \\
\poemll    and their tents remain unoccupied. \\
\poeml \v{26}For they persecute those whom you have struck, \\
\poemll    and they brag about the pain of those you have wounded. \\
\poeml \v{27}May you punish them for their crimes; \\
\poemll    may they receive no verdict of innocence\fnote{Lit. \fbib{no righteousness}} from you. \\
\poeml \v{28}May they be erased from the Book of Life, \\
\poemll    and their names not be written with the righteous. \\
\poeml \v{29}As for me, I am afflicted and hurting; \\
\poemll    may your deliverance, God, establish me on high. \\
\poeml \v{30}Let me praise the name of God with a song \\
\poemll    that I may magnify him with thanksgiving. \\
\poeml \v{31}That will please the \divine{Lord} \\
\poemll    more than oxen and bulls with horns and hooves. \\
\poeml \v{32}The afflicted will watch and rejoice. \\
\poemll    May you who seek God take courage. \\
\poeml \v{33}For the \divine{Lord} listens to the needy \\
\poemll    and doesn't despise those in bondage. \\
\poeml \v{34}Let the heavens and earth praise him, \\
\poemll    along with the sea and its swarming creatures.\fnote{The Heb. lacks \fbib{creatures}} \\
\poeml \v{35}For God will deliver Zion \\
\poemll    and will rebuild the cities of Judah \\
\poemlll       so they may live there and possess them. \\
\poeml \v{36}The descendants of his servants will inherit it, \\
\poemll    and those who cherish his name will live there.
\end{poetry}
\labelpsalm{70}
\psalminfo{To the Music director. Davidic. As a memorial.}
\passage{A Call for Help}

\begin{poetry}
\poeml \v{1}God, come to my rescue. \\
\poemll    \divine{Lord}, hurry to help me. \\
\poeml \v{2}May those who seek to kill me be publicly humiliated. \\
\poemll    May those who take pleasure in my harm \\
\poemlll       be turned back in humiliation. \\
\poeml \v{3}May those who say ``Aha! Aha!'' \\
\poemll    be turned back because of their shameful deeds.\fnote{The Heb. lacks \fbib{deeds}} \\
\poeml \v{4}Let those who seek you greatly rejoice in you. \\
\poemll    Let those who love your deliverance say, \\
\poemlll       ``May God be continuously exalted.'' \\
\poeml \v{5}As for me, I am poor and needy. \\
\poemll    God, come quickly to me. \\
\poeml You are my helper and my deliverer. \\
\poemll    \divine{Lord}, please do not delay.
\end{poetry}
\labelpsalm{71}
\passage{A Prayer for Deliverance}

\begin{poetry}
\poeml \v{1}In you, \divine{Lord}, I take refuge; \\
\poemll    let me never be humiliated. \\
\poeml \v{2}Rescue and deliver me,\fnote{So LXX DSS 4QPs\textsuperscript{a}; MT reads \fbib{In your righteousness you are delivering me and rescuing me}} because you are righteous. \\
\poemll    Turn your ear to me and save me. \\
\poeml \v{3}Be my sheltering refuge where I may go continuously; \\
\poemll    command my deliverance \\
\poemlll       for you are my rock and fortress. \\
\poeml \v{4}My God, deliver me from the power of the wicked \\
\poemll    and the grasp of ruthless practicers of evil. \\
\poeml \v{5}For you are my hope, Lord \divine{God}, \\
\poemll    my security since I was young. \\
\poeml \v{6}I depended on you since birth,\fnote{Lit. \fbib{you from the womb}} \\
\poemll    when you brought me\fnote{So MT; LXX reads \fbib{birth, it was you who sheltered me}; DSS 4QPs\textsuperscript{a} reads \fbib{birth, you are my protector}} from my mother's womb; \\
\poemlll       I praise you continuously. \\
\poeml \v{7}I have become an example to many \\
\poemll    that you are my strong refuge. \\
\poeml \v{8}My mouth is filled with your praise \\
\poemll    and your splendor daily. \\
\poeml \v{9}Don't throw me away when I am old; \\
\poemll    do not abandon me when my strength fails. \\
\poeml \v{10}For my enemies talk against me; \\
\poemll    those who seek to kill me plot together \\
\poeml \v{11}and say, ``God has abandoned him. \\
\poemll    Run after him and seize him, \\
\poemlll       because there's no deliverer.'' \\
\poeml \v{12}God, do not be distant from me. \\
\poemll    My God, come quickly to help me. \\
\poeml \v{13}Let my adversaries be ashamed and consumed;\fnote{So MT; LXX reads \fbib{and let them expire}; DSS 4QPs\textsuperscript{a} reads \fbib{and let them be consumed}} \\
\poemll    let those who seek my destruction \\
\poemlll       be covered with scorn and disgrace. \\
\poeml \v{14}As for me, I will hope continuously \\
\poemll    and will praise you more and more. \\
\poeml \v{15}I\fnote{Lit. \fbib{My mouth}} will declare your righteousness \\
\poemll    and your salvation every day, \\
\poeml though I do not fully understand \\
\poemll    what the outcome will be.\fnote{Lit. \fbib{understand the sum}} \\
\poeml \v{16}Lord \divine{God}, I will come in the power of\fnote{The Heb. lacks \fbib{the power of}} your mighty acts, \\
\poemll    remembering your righteousness---yours alone. \\
\poeml \v{17}God, you taught me from my youth, \\
\poemll    so I am still declaring your awesome deeds. \\
\poeml \v{18}Also, when I reach old age and have gray hair, \\
\poemll    God, do not forsake me, \\
\poeml until I have declared your power \\
\poemll    to this generation \\
\poemlll       and your might to the next one. \\
\poeml \v{19}Your many righteous deeds,\fnote{Lit. \fbib{righteous deeds as far as the height}} God, are great, \\
\poeml \v{20}God, who can compare to you, \\
\poeml who caused me to experience\fnote{Lit. \fbib{see}} troubles \\
\poeml that were numerous and disastrous? \\
\poeml You will return to revive me \\
\poemll    and lift me up from the depths of the earth. \\
\poeml \v{21}You will increase my honor \\
\poemll    and comfort me once again. \\
\poeml \v{22}I also will praise you with the harp; \\
\poemll    because of your faithfulness, my God, \\
\poeml I will praise you with the lyre--- \\
\poemll    Holy One of Israel. \\
\poeml \v{23}My lips will shout for joy when I sing praise to you, \\
\poemll    whose life you have redeemed. \\
\poeml \v{24}Moreover, my tongue will speak all day about your justice; \\
\poemll    for those who seek my destruction will be utterly humiliated.
\end{poetry}
\labelpsalm{72}
\psalminfo{Solomonic}
\passage{A Prayer for the King}

\begin{poetry}
\poeml \v{1}God, endow the king with ability to render\fnote{The Heb. lacks \fbib{to render}} your justice, \\
\poemll    and the king's son to render your right decisions. \\
\poeml \v{2}May he rule your people with right decisions \\
\poemll    and your oppressed ones with justice. \\
\poeml \v{3}May the mountains bring prosperity to the people \\
\poemll    and the hills bring righteousness. \\
\poeml \v{4}May he defend the afflicted of the people \\
\poemll    and deliver the children of the poor, \\
\poemlll       but crush the oppressor. \\
\poeml \v{5}May they fear you as long as the sun and moon shine\fnote{The Heb. lacks \fbib{shine}}--- \\
\poemll    from generation to generation. \\
\poeml \v{6}May he be like the rain that descends on mown grass, \\
\poemll    like showers sprinkling on the ground. \\
\poeml \v{7}The righteous will flourish at the proper time \\
\poemll    and peace will prevail until the moon is no more. \\
\poeml \v{8}May he rule from sea to sea, \\
\poemll    from the Euphrates River\fnote{The Heb. lacks \fbib{Euphrates}} to the ends of the earth. \\
\poemll    \v{9}May the nomads bow down before him, \\
\poemll    and his enemies lick the dust. \\
\poeml \v{10}May the kings of Tarshish and of distant shores bring gifts, \\
\poemll    and may the kings of Sheba and Seba offer tribute. \\
\poeml \v{11}May all kings bow down to him, \\
\poemll    and all nations serve him. \\
\poeml \v{12}For he will deliver the needy when they cry out for help, \\
\poemll    and the poor when there is no deliverer. \\
\poeml \v{13}He will have compassion on the poor and the needy, \\
\poemll    and he will save the lives of the needy. \\
\poeml \v{14}He will redeem them\fnote{Lit. \fbib{redeem their souls}} from oppression and violence, \\
\poemll    since their lives are\fnote{Lit. \fbib{their blood is}} precious in his sight.
\passage{Prayer for the King}
\poeml \v{15}May he live long and be given gold from Sheba, \\
\poemll    and may prayer be offered for him continuously, \\
\poemlll       and may he be blessed every day. \\
\poeml \v{16}May grain be abundant in the land \\
\poemll    all the way\fnote{The Heb. lacks \fbib{all the way}} to the mountain tops; \\
\poeml may its fruits flourish \\
\poemll    like the forests of Lebanon, \\
\poeml and may the cities sprout \\
\poemll    like the grass of the earth.
\passage{Praising the God of Israel}
\poeml \v{17}May his fame\fnote{Lit. \fbib{name}} be eternal--- \\
\poemll    as long as the sun--- \\
\poeml may his name endure, \\
\poemll    and may they be blessed through him, \\
\poemlll       and may all nations call him blessed. \\
\poeml \v{18}Blessed be the \divine{Lord} God, the God of Israel, \\
\poemll    who alone does awesome deeds. \\
\poeml \v{19}And blessed be his glorious name forever, \\
\poemll    and may the whole earth be filled with his glory. \\
\poemlll       Amen and amen! \\
\poeml \v{20}This ends the prayers of Jesse's son David.
\end{poetry}
\booksection{BOOK III (Psalms 73-89)}
\labelpsalm{73}
\psalminfo{A song of Asaph.}
\passage{A Plea for Deliverance}

\begin{poetry}
\poeml \v{1}God is indeed good to Israel, \\
\poemll    to those pure in heart. \\
\poeml \v{2}Now as for me, my feet nearly stumbled, \\
\poemll    as I almost lost my step. \\
\poeml \v{3}For I was envious of the proud \\
\poemll    when I observed the prosperity of the wicked. \\
\poeml \v{4}For there is no struggle at their deaths, \\
\poemll    and their bodies are healthy. \\
\poeml \v{5}They do not experience problems common to ordinary people; \\
\poemll    they aren't afflicted as others\fnote{Lit. \fbib{human beings}} are. \\
\poeml \v{6}Therefore pride is their necklace \\
\poemll    and violence covers them like a garment. \\
\poeml \v{7}Their eyes bulge from obesity \\
\poemll    and the imaginations of their mind cross the border into sin.\fnote{The Heb. lacks \fbib{into sin}} \\
\poeml \v{8}In their mockery they speak evil; \\
\poemll    from their arrogant position they speak oppression. \\
\poeml \v{9}They choose to speak\fnote{Lit. \fbib{They set their mouth}} against heaven; \\
\poemll    while they talk about things on earth. \\
\poeml \v{10}Therefore God's\fnote{Lit. \fbib{his}} people return there \\
\poemll    and drink it all in like water until they're satiated. \\
\poeml \v{11}Then they say, \\
\poemll    ``How can God know? \\
\poemlll       Does the Most High have knowledge?'' \\
\poeml \v{12}Just look at these wicked people! \\
\poemll    They're perpetually carefree \\
\poemlll       as they increase their wealth. \\
\poeml \v{13}I kept my heart pure for nothing \\
\poemll    and kept my hands clean from guilt. \\
\poeml \v{14}For I suffer all day long \\
\poemll    and I am punished every morning. \\
\poeml \v{15}If I say, ``I will talk like this,'' \\
\poemll    I would betray a generation of your children. \\
\poeml \v{16}When I tried to understand this, \\
\poemll    it was too difficult for me \\
\poeml \v{17}until I entered the sanctuaries of God. \\
\poemll    Then I understood their destiny. \\
\poeml \v{18}You have certainly set them in slippery places; \\
\poemll    you will make them fall to their ruin. \\
\poeml \v{19}How desolate they quickly become, \\
\poemll    completely destroyed by calamities. \\
\poeml \v{20}Like a dream when one awakens, Lord, \\
\poemll    you will despise their image when you arise. \\
\poeml \v{21}When I chose to be bitter \\
\poemll    I was emotionally pained. \\
\poeml \v{22}Then, I was too stupid \\
\poemll    and didn't realize I was acting like\fnote{The Heb. lacks \fbib{acting like}} a wild animal with you. \\
\poeml \v{23}But now I am always with you, \\
\poemll    for you keep holding my right hand. \\
\poeml \v{24}You will guide me with your wise advice, \\
\poemll    and later you will receive me with honor. \\
\poeml \v{25}Whom do I have in heaven but you? \\
\poemll    I desire nothing on this\fnote{The Heb. lacks \fbib{this}}earth. \\
\poeml \v{26}My body and mind may fail, \\
\poemll    but God is my strength\fnote{Lit. \fbib{is the rock of my heart}} and my portion forever. \\
\poeml \v{27}Those far from you will perish; \\
\poemll    you will destroy those who are unfaithful to you. \\
\poeml \v{28}As for me, how good for me it is that God is near! \\
\poemll    I have made the Lord \divine{God} my refuge \\
\poemlll       so I can tell about all your deeds.
\end{poetry}
\labelpsalm{74}
\psalminfo{An instruction\fnote{T Lit. \fbib{maskil}} of Asaph}
\passage{A Plea for Deliverance}

\begin{poetry}
\poeml \v{1}Why, God? Have you rejected us forever? \\
\poemll    Your anger is burning against the sheep of your pasture. \\
\poeml \v{2}Remember your community, \\
\poeml whom you purchased long ago, \\
\poeml the tribe whom you redeemed \\
\poemll    for your possession. \\
\poeml Remember\fnote{The Heb. lacks \fbib{Remember}} Mount Zion, \\
\poemll    where you live. \\
\poemlll       \v{3}Hurry! Look at the permanent ruins---
\end{poetry}

\begin{poetry}
\poemll    every calamity the enemy brought upon the Holy Place. \\
\poeml \v{4}Those who are opposing you roar \\
\poemll    where we were meeting with you; \\
\poemlll       they unfurl their war banners as signs. \\
\poeml \v{5}As one blazes a trail \\
\poemll    through a forest with an ax, \\
\poeml \v{6}now they're tearing down all its carved work \\
\poemll    with hatchets and hammers. \\
\poeml \v{7}They burned your sanctuary to the ground, \\
\poemll    desecrating your dwelling place. \\
\poeml \v{8}They say to themselves, \\
\poemll    ``We'll crush them completely;'' \\
\poemlll       They burned down all the meeting places of God in the land. \\
\poeml \v{9}We see no signs for us; \\
\poemll    there is no longer a prophet, \\
\poemlll       and no one among us knows the future.\fnote{Lit. \fbib{knows when}} \\
\poeml \v{10}God, how long will the adversary scorn \\
\poemll    while the enemy despises your name endlessly? \\
\poeml \v{11}Why do you not withdraw your hand--- \\
\poemll    your right hand---from your bosom \\
\poemlll       and destroy them?\fnote{The Heb. lacks \fbib{them}} \\
\poeml \v{12}But God is my king from ancient times, \\
\poemll    who brings acts of deliverance throughout the earth. \\
\poeml \v{13}You split the sea by your own power. \\
\poemll    You shattered the heads of sea monsters in the water. \\
\poeml \v{14}You crushed the heads of Leviathan. \\
\poemll    You set it as food for desert creatures.\fnote{Or \fbib{people}} \\
\poeml \v{15}You opened both the spring and the river; \\
\poemll    you dried up flowing rivers. \\
\poeml \v{16}Yours is the day, and yours is the night; \\
\poemll    you established the moon and the sun. \\
\poeml \v{17}You set all the boundaries of the earth; \\
\poemll    you made summer and winter. \\
\poeml \v{18}Remember this: The enemy scorns the \divine{Lord} \\
\poemll    and a foolish people despises your name. \\
\poeml \v{19}Don't hand over the life of your dove to beasts; \\
\poemll    do not continuously forget your afflicted ones. \\
\poeml \v{20}Pay attention to your covenant, \\
\poemll    for the dark regions of the earth are full of violence. \\
\poeml \v{21}Don't let the oppressed return in humiliation. \\
\poemll    The poor and needy will praise your name. \\
\poeml \v{22}Get up, God, and prosecute your case--- \\
\poemll    remember that you're being scorned \\
\poemlll       by fools all day long. \\
\poeml \v{23}Don't ignore the shout of those opposing you, \\
\poemll    The uproar of those who rebel against you continuously.
\end{poetry}
\labelpsalm{75}
\psalminfo{To the Director: To the tune of\fnote{T The Heb. lacks \fbib{the tune of}} ``Do not Destroy!'' \\ A psalm of Asaph. A song.}
\passage{Praise to God for Justice}

\begin{poetry}
\poeml \v{1}We praise you, God! \\
\poemll    We praise you\fnote{The Heb. lacks \fbib{you}}---your presence\fnote{Lit. \fbib{name}} draws near--- \\
\poemlll       as we declare your wonderful deeds. \\
\poeml \v{2}``At the time that I choose \\
\poemll    I will judge the righteous.\fnote{Or \fbib{judge righteously}} \\
\poeml \v{3}While the earth and all its inhabitants melt away, \\
\poemll    it is I who keep its pillars firm.''
\end{poetry}
\interlude{Interlude}

\begin{poetry}
\poeml \v{4}I will say to the proud, ``Don't brag,'' \\
\poemll    and to the wicked, \\
\poemlll       ``Don't vaunt your strength.\fnote{Lit. \fbib{Don't lift up your horn}} \\
\poeml \v{5}Don't use your strength to fight heaven\fnote{Lit. \fbib{Don't lift your horns to the height}} \\
\poemll    or speak from stubborn arrogance.''\fnote{Lit. \fbib{speak with a stiff neck}} \\
\poeml \v{6}For exaltation comes not from the east, \\
\poemll    the west, or the wilderness, \\
\poeml \v{7}since God is the judge. \\
\poemll    This one he will debase or that one he will exalt. \\
\poeml \v{8}For there is a cup in the hand of the \divine{Lord}, \\
\poemll    foaming with well-mixed wine \\
\poeml that he will pour out, leaving only the dregs, \\
\poemll    from which all the wicked of the earth will drink. \\
\poeml \v{9}But as for me, I will declare forever, \\
\poemll    singing praise to the God of Jacob. \\
\poeml \v{10}I will cut down the strength\fnote{Lit. \fbib{horn}} of the wicked, \\
\poemll    but the strength\fnote{Lit. \fbib{horn}} of the righteous will be lifted up.
\end{poetry}
\labelpsalm{76}
\psalminfo{To the Director: With stringed instruments. A psalm of Asaph. A song.}
\passage{The Awesome God}

\begin{poetry}
\poeml \v{1}God is known in Judah; \\
\poemll    in Israel his reputation is great. \\
\poeml \v{2}His abode is in Salem,\fnote{I.e. Jerusalem} \\
\poemll    his dwelling place in Zion. \\
\poeml \v{3}There he shattered sharp arrows, \\
\poemll    shields, swords, and weapons of\fnote{The Heb. lacks \fbib{weapons of}} war.
\end{poetry}
\interlude{Interlude}

\begin{poetry}
\poeml \v{4}You are enveloped by light; \\
\poemll    more majestic than mountains filled with game. \\
\poeml \v{5}Brave men were plundered \\
\poemll    while they slumbered in their sleep. \\
\poemlll       All the men of the army were immobilized. \\
\poeml \v{6}At the sound of your battle cry, God of Jacob, \\
\poemll    both horse and chariot rider fell into deep sleep. \\
\poeml \v{7}You are awesome! \\
\poemll    who can stand in your presence when you're angry? \\
\poeml \v{8}From heaven you declared judgment. \\
\poemll    The earth stands in awe and is quiet \\
\poeml \v{9}when God arose to execute justice \\
\poemll    and to deliver all the afflicted of the earth.
\end{poetry}
\interlude{Interlude}

\begin{poetry}
\poeml \v{10}Even human anger praises you; \\
\poemll    you will wear the survivors of your wrath as an ornament.\fnote{The Heb. lacks \fbib{as an ornament}} \\
\poeml \v{11}Let everyone who surrounds the \divine{Lord} your God \\
\poemll    make a vow and fulfill it to the Awesome One.\fnote{Or \fbib{to the one whom they fear}} \\
\poeml \v{12}He will humble the arrogant\fnote{Lit. \fbib{the spirit of}} commanders-in-chief,\fnote{Lit. \fbib{Nagidim}; i.e. senior officers entrusted with dual roles of operational oversight and administrative authority} \\
\poemll    instilling fear among the kings of the earth.
\end{poetry}
\labelpsalm{77}
\psalminfo{To the director: To Jeduthun. A psalm of Asaph.}
\passage{Remembering God in Times of Trouble}

\begin{poetry}
\poeml \v{1}I cry out to God! \\
\poemll    I cry out to God and he hears me. \\
\poeml \v{2}When I was in distress, I sought the Lord; \\
\poemll    my hands were raised at night \\
\poeml and they did not grow weary. \\
\poemlll       I refused to be comforted. \\
\poeml \v{3}I remember God, and I groan; \\
\poemll    I meditate, while my spirit grows faint.
\end{poetry}
\interlude{Interlude}

\begin{poetry}
\poeml \v{4}You kept my eyes open; \\
\poemll    I was troubled and couldn't speak. \\
\poeml \v{5}I thought of ancient times, \\
\poemll    considering years long past. \\
\poeml \v{6}During the night I remembered my song. \\
\poemll    I meditate in my heart, \\
\poemlll       and my spirit ponders. \\
\poeml \v{7}Will the Lord reject me\fnote{The Heb. lacks \fbib{me}} forever \\
\poemll    and not show favor again? \\
\poeml \v{8}Has his gracious love ceased forever? \\
\poemll    Will his promise be unfulfilled in future generations? \\
\poeml \v{9}Has God forgotten to be gracious? \\
\poemll    Has he in anger withheld his compassion?
\end{poetry}
\interlude{Interlude}

\begin{poetry}
\poeml \v{10}So I say: ``It causes me pain \\
\poemll    that the right hand of the Most High has changed.'' \\
\poeml \v{11}I will remember the \divine{Lord}'s deeds; \\
\poemll    indeed, I will remember your awesome deeds from long ago. \\
\poeml \v{12}As I meditate on all your works, \\
\poemll    I will consider your awesome deeds. \\
\poeml \v{13}God, your way is holy. \\
\poemll    What god is like our great God? \\
\poeml \v{14}God, you are the one performing awesome deeds. \\
\poemll    You reveal your might among the nations. \\
\poeml \v{15}You delivered\fnote{Or \fbib{redeemed}} your people--- \\
\poemll    the descendants of Jacob and Joseph--- \\
\poemlll       with your power.
\end{poetry}
\interlude{Interlude}

\begin{poetry}
\poeml \v{16}The waters saw you, God; \\
\poemll    the waters saw you and writhed. \\
\poemlll       Indeed, the depths of the sea quaked. \\
\poeml \v{17}The clouds poured rain; \\
\poemll    the skies rumbled. \\
\poemlll       Indeed, your lightning bolts flashed.\fnote{Lit. \fbib{your fierce arrows traveled}} \\
\poeml \v{18}Your thunderous sound was in a whirlwind; \\
\poemll    your lightning lights up the world; \\
\poemlll       the earth becomes agitated and quakes. \\
\poeml \v{19}Your way was through the sea, \\
\poemll    and your path through mighty waters, \\
\poemlll       but your footprints cannot be traced.\fnote{Lit. \fbib{steps are not recognized}} \\
\poeml \v{20}You have led your people like a flock \\
\poemll    by the hands of Moses and Aaron.
\end{poetry}
\labelpsalm{78}
\psalminfo{An instruction\fnote{T Lit. \fbib{maskil}} of Asaph}
\passage{Remembering God in Times of Trouble}

\begin{poetry}
\poeml \v{1}Listen, my people, to my instruction. \\
\poemll    Hear\fnote{Lit. \fbib{Stretch out your ear}} the words of my mouth. \\
\poeml \v{2}I will tell\fnote{Lit. \fbib{will open my mouth in}} a parable, \\
\poemll    speaking riddles from long ago--- \\
\poeml \v{3}things that we have heard and known \\
\poemll    and that our ancestors related to us. \\
\poeml \v{4}We will not withhold them from their descendants; \\
\poemll    we'll declare to the next generation the praises of the \divine{Lord}--- \\
\poemlll       his might and awesome deeds that he has performed. \\
\poeml \v{5}He established a decree in Jacob, \\
\poemll    and established the Law in Israel, \\
\poeml that he commanded our ancestors \\
\poemll    to reveal to their children \\
\poeml \v{6}in order that the next generation--- \\
\poemll    children yet to be born--- \\
\poeml will know them and \\
\poemll    in turn teach them to their children. \\
\poeml \v{7}Then they will put their trust in God \\
\poemll    and they will not forget his awesome deeds. \\
\poemlll       Instead, they will keep his commandments. \\
\poeml \v{8}They will not be like the rebellious generation of their ancestors, \\
\poemll    a rebellious generation, \\
\poeml whose heart was not steadfast, \\
\poemll    and whose spirits were unfaithful to God. \\
\poeml \v{9}The descendants of Ephraim were sharp shooters with the bow, \\
\poemll    but they retreated in the day of battle. \\
\poeml \v{10}They did not keep God's covenant, \\
\poemll    and refused to live by his Law. \\
\poeml \v{11}They have forgotten what he has done, \\
\poemll    his awesome deeds that they witnessed. \\
\poeml \v{12}He performed marvelous things \\
\poemll    in the presence of their ancestors \\
\poeml in the land of Egypt--- \\
\poemll    in the fields of Zoan. \\
\poeml \v{13}He divided the sea so that they were able to cross; \\
\poemll    he caused the water to stand in a single location. \\
\poeml \v{14}He led them with a cloud during the day, \\
\poemll    and during the night with light from the fire. \\
\poeml \v{15}He caused the rocks to split in the wilderness, \\
\poemll    and gave them water\fnote{Lit. \fbib{drink}} as from an abundant sea. \\
\poeml \v{16}He brought streams from rock, \\
\poemll    causing water to flow like a river. \\
\poeml \v{17}But time and again, they sinned against him, \\
\poemll    rebelling against the Most High in the desert. \\
\poeml \v{18}To test God was in their minds, \\
\poemll    when they demanded food to satisfy their cravings.\fnote{Lit. \fbib{food for their soul}} \\
\poeml \v{19}They spoke against God by asking, \\
\poemll    ``Is God able to prepare a feast\fnote{Or \fbib{table}} in the desert? \\
\poeml \v{20}It's true that\fnote{Lit. \fbib{Indeed,}} Moses\fnote{Lit. \fbib{he}} struck the rock so that water flowed forth \\
\poemll    and torrents of water gushed out, \\
\poeml but is he also able to give bread \\
\poemll    or to supply meat for his people?'' \\
\poeml \v{21}Therefore, when the \divine{Lord} heard this, he was angry, \\
\poemll    and fire broke out against Jacob. \\
\poeml Moreover, his anger flared against Israel, \\
\poeml \v{22}because they didn't believe in God \\
\poemlll       and didn't trust in his deliverance. \\
\poeml \v{23}Yet he commanded the skies above \\
\poemll    and the doors of the heavens to open, \\
\poeml \v{24}so that manna rained down on them for food \\
\poemll    and he sent them the grain of heaven. \\
\poeml \v{25}Mortal men\fnote{Lit. \fbib{A man}} ate the food of angels; \\
\poemll    he sent provision to them in abundance. \\
\poeml \v{26}He stirred up the east wind in the heavens \\
\poemll    and drove the south wind by his might. \\
\poeml \v{27}He caused meat to rain on them like dust \\
\poemll    and winged birds as the sand of the sea. \\
\poeml \v{28}He caused these to fall in the middle of the camp \\
\poemll    and all around their tents. \\
\poeml \v{29}So they ate and were very satisfied, \\
\poemll    because he granted their desire. \\
\poeml \v{30}However, before they had fulfilled their desire, \\
\poemll    while their food was still in their mouths, \\
\poeml \v{31}the anger of God flared against them, \\
\poemll    and he killed the strongest men \\
\poemlll       and humbled Israel's young men. \\
\poeml \v{32}In spite of all of this, they kept on sinning \\
\poemll    and didn't believe in his marvelous deeds. \\
\poeml \v{33}So he made their days end in futility, \\
\poemll    and their years with sudden terror. \\
\poeml \v{34}When he struck them, they sought him; \\
\poemll    they repented, and eagerly sought God. \\
\poeml \v{35}Then they remembered that God was their rock, \\
\poemll    and the Most High God was their deliverer. \\
\poeml \v{36}But they deceived him with their mouths; \\
\poemll    they lied to him with their tongues. \\
\poeml \v{37}For their hearts weren't committed to him, \\
\poemll    and they weren't faithful to his covenant. \\
\poeml \v{38}But he, being merciful, forgave their iniquity \\
\poemll    and didn't destroy them; \\
\poeml He restrained his anger \\
\poemll    and didn't vent all his wrath. \\
\poeml \v{39}For he remembered that they were only flesh, \\
\poemll    a passing wind that doesn't return. \\
\poeml \v{40}How they rebelled against him in the desert, \\
\poemll    grieving him in the wilderness! \\
\poeml \v{41}They tested God again and again, \\
\poemll    provoking the Holy One of Israel. \\
\poeml \v{42}They did not remember his power--- \\
\poemll    the day he delivered them from their adversary, \\
\poeml \v{43}when he set his signs in Egypt \\
\poemll    and his wonders in the plain of Zoan. \\
\poeml \v{44}He turned their rivers into blood \\
\poemll    and made their streams undrinkable. \\
\poeml \v{45}He sent swarms of insects to bite them \\
\poemll    and frogs to destroy them. \\
\poeml \v{46}He gave their crops to caterpillars \\
\poemll    and what they worked for to locusts. \\
\poeml \v{47}He destroyed their vines with hail \\
\poemll    and their sycamore\fnote{The sycamore fruit tree native to Israel bears figs} trees with frost. \\
\poeml \v{48}He delivered their beasts to hail \\
\poemll    and their livestock to lightning bolts. \\
\poeml \v{49}He inflicted his burning anger, \\
\poemll    wrath, indignation, and distress, \\
\poemlll       sending destroying angels among them. \\
\poeml \v{50}He blazed a path for his anger; \\
\poemll    he did not stop short from killing them, \\
\poemlll       but handed them over to pestilence. \\
\poeml \v{51}He struck every firstborn in Egypt, \\
\poemll    the first fruits of their manhood in the tents of Ham. \\
\poeml \v{52}Yet he led out his people like sheep, \\
\poemll    guiding them like a flock in the desert. \\
\poeml \v{53}He led them to safety so they would not fear. \\
\poemll    As for their enemies, the sea covered them. \\
\poeml \v{54}He brought the people\fnote{Lit. \fbib{brought them}} to the border of his holy mountain, \\
\poemll    which he acquired by his might. \\
\poeml \v{55}He drove out nations before them \\
\poemll    and allotted their tribal inheritance, \\
\poemlll       settling the tribes of Israel in their tents. \\
\poeml \v{56}But they tested the Most High God by rebelling against him, \\
\poemll    and they did not obey his statutes. \\
\poeml \v{57}They fell away and were as disloyal as their ancestors. \\
\poemll    They became unreliable, like a defective bow; \\
\poeml \v{58}they angered him with their high places \\
\poemll    and with their carved images they made him jealous. \\
\poeml \v{59}God heard and became furious, \\
\poemll    and he completely rejected Israel. \\
\poeml \v{60}He abandoned the tent at Shiloh, \\
\poemll    the tent that he established among mankind. \\
\poeml \v{61}Then he sent his might\fnote{I.e. the Ark of the Covenant} into captivity \\
\poemll    and his glory into the control of the adversary. \\
\poeml \v{62}He delivered his people over to the sword \\
\poemll    and was angry with his possession. \\
\poeml \v{63}The young men were consumed by fire, \\
\poemll    and the virgins had no marriage celebrations.\fnote{Lit. \fbib{virgins sang no wedding song}} \\
\poeml \v{64}The priests fell by the sword, \\
\poemll    yet their widows couldn't weep. \\
\poeml \v{65}The \divine{Lord} awoke as though from sleep, \\
\poemll    like a mighty warrior stimulated by wine. \\
\poeml \v{66}He beat back his adversaries, \\
\poemll    permanently disgracing them. \\
\poeml \v{67}He rejected the clan\fnote{Lit. \fbib{tent}} of Joseph; \\
\poemll    and the tribe of Ephraim he did not choose. \\
\poeml \v{68}But he chose the tribe of Judah, \\
\poemll    the mountain of Zion, which he loves. \\
\poeml \v{69}He built his sanctuary, high as the heavens, \\
\poemll    like the earth that he established forever. \\
\poeml \v{70}Then he chose his servant David, \\
\poemll    whom he took from the sheepfold. \\
\poeml \v{71}He brought him from birthing sheep \\
\poemll    to care for Jacob, his people, \\
\poemlll       Israel, his possession. \\
\poeml \v{72}David\fnote{Lit. \fbib{He}} shepherded them with a devoted heart, \\
\poemll    and led them with skillful hands.
\end{poetry}
\labelpsalm{79}
\psalminfo{A Psalm of Asaph}
\passage{A Prayer for Jerusalem}

\begin{poetry}
\poeml \v{1}God, nations have invaded your land\fnote{Lit. \fbib{your possession}; or \fbib{your inheritance}} \\
\poemll    to desecrate your holy Temple, \\
\poemlll       to destroy Jerusalem, \\
\poeml \v{2}to give the corpses of your servants \\
\poemll    as food for the birds of the skies \\
\poeml and the flesh of your godly ones \\
\poemll    to the beasts of the earth; \\
\poeml \v{3}to make their blood flow like water around Jerusalem, \\
\poemll    with no one being buried. \\
\poeml \v{4}We have become a reproach to our neighbors, \\
\poemll    a mockery and a derision to those around us. \\
\poeml \v{5}How long, \divine{Lord}, will you be angry? Forever? \\
\poemll    Will your jealousy burn like fire? \\
\poeml \v{6}Pour out your wrath upon the nations \\
\poemll    that do not acknowledge you, \\
\poeml and over the kingdoms \\
\poemll    that do not call on your name. \\
\poemll    \v{7}For they consumed Jacob, \\
\poemll    making his dwelling place desolate. \\
\poeml \v{8}Don't charge\fnote{Lit. \fbib{remember}} us for previous iniquity, \\
\poemll    but let your compassion come quickly to us, \\
\poemlll       for we have been brought very low. \\
\poeml \v{9}Help us, God, our deliverer, \\
\poemll    on account of your glorious name, \\
\poeml deliver us and forgive\fnote{Lit. \fbib{cover}} our sins \\
\poemll    on account of your name. \\
\poeml \v{10}Why should the nations say, ``Where is their God?'' \\
\poemll    Let vengeance for the blood of your servants be meted\fnote{Lit. \fbib{spilled}} out \\
\poemlll       before our eyes and among the nations. \\
\poeml \v{11}Let the cries of the prisoners reach you. \\
\poemll    With the strength of your power, \\
\poemlll       release those condemned to death.\fnote{Lit. \fbib{the sons of death}} \\
\poeml \v{12}Pay back our neighbors seven times\fnote{Or \fbib{seven-fold}} \\
\poemll    the reproach with which they reproached you, \divine{Lord}. \\
\poemll    \v{13}Then we, your people, the sheep of your pasture, \\
\poemll    will praise you always, from generation to generation. \\
\poemlll       We will declare your praise.
\end{poetry}
\labelpsalm{80}
\psalminfo{For the Director of Music: According to ``The Lilies''. A testimony of Asaph. A psalm.}
\passage{A Prayer for Jerusalem}

\begin{poetry}
\poeml \v{1}Shepherd of Israel, listen! \\
\poemll    The one who leads Joseph like a flock, \\
\poeml the one enthroned on the cherubim, \\
\poemll    display your glory.\fnote{The Heb. lacks \fbib{your glory}} \\
\poeml \v{2}Reveal\fnote{Or \fbib{rouse}, \fbib{stir up}} your power before Ephraim, Benjamin, and Manasseh, \\
\poemll    then come to our rescue. \\
\poeml \v{3}God, restore us, \\
\poemll    show your favor\fnote{Lit. \fbib{cause your face to shine}} and deliver us. \\
\poeml \v{4}\divine{Lord} God of the Heavenly Armies, \\
\poemll    when will your smoldering anger\fnote{Lit. \fbib{Until when will you burn in anger}} \\
\poemlll       toward your people's prayers cease?\fnote{The Heb. lacks \fbib{cease}} \\
\poeml \v{5}You fed them tears as their food, \\
\poemll    and caused them to drink a full measure of tears. \\
\poeml \v{6}You have set us at strife against our neighbors \\
\poemll    and our enemies deride us. \\
\poeml \v{7}God of the Heavenly Armies, restore us \\
\poemll    and show your favor,\fnote{Lit. \fbib{cause your face to shine}} \\
\poemlll       so we may be delivered. \\
\poeml \v{8}You uprooted a vine from Egypt, \\
\poemll    and drove out nations to transplant it. \\
\poeml \v{9}You cleared the ground\fnote{The Heb. lacks \fbib{ground}} so that its roots grew \\
\poemll    and filled the land. \\
\poeml \v{10}Mountains were covered by its shadows, \\
\poemll    and the mighty cedars by its branches. \\
\poeml \v{11}Its branches spread out to the Mediterranean\fnote{The Heb. lacks \fbib{Mediterranean}} Sea \\
\poemll    and its shoots to the Euphrates\fnote{The Heb. lacks \fbib{Euphrates}} River. \\
\poeml \v{12}Why did you break down its walls \\
\poemll    so that those who pass by pluck its fruits?\fnote{Heb. lacks \fbib{its fruits}} \\
\poeml \v{13}Wild boars of the forest gnaw at it, \\
\poemll    and creatures of the field feed on it. \\
\poeml \v{14}God of the Heavenly Armies, return! \\
\poemll    Look down from heaven and see. \\
\poemlll       Show care\fnote{Lit. \fbib{Visit}} toward this vine. \\
\poeml \v{15}The root\fnote{Or \fbib{stock}} that your right hand planted, \\
\poemll    the shoot\fnote{Lit. \fbib{son}} that you tended for yourself, \\
\poeml \v{16}was burned with fire, cut off, \\
\poemll    and destroyed on account of your rebuke. \\
\poeml \v{17}May you support the man at your right hand; \\
\poemll    the son of man whom you have raised for yourself. \\
\poeml \v{18}Then we will not turn away from you. \\
\poemll    Restore us, so we can call upon your name. \\
\poeml \v{19}God of hosts, restore to us the light of your favor.\fnote{Lit. \fbib{face}} \\
\poemll    Then we'll be delivered.
\end{poetry}
\labelpsalm{81}
\psalminfo{For the Director: On the Gittith. By Asaph.}
\passage{Celebrating and Remembering God}

\begin{poetry}
\poeml \v{1}Sing joyfully to God, our strength. \\
\poemll    Raise a shout to the God of Jacob. \\
\poeml \v{2}Sing a song and play the tambourine, \\
\poemll    the pleasant-sounding lyre along with the harp. \\
\poeml \v{3}Blow the ram's horn when there is a New Moon, \\
\poemll    when there is a full moon, \\
\poemlll       on our festival day, \\
\poeml \v{4}because it is a statute in Israel, \\
\poemll    an ordinance by the God of Jacob, \\
\poeml \v{5}a decree that he prescribed for Joseph \\
\poemll    when he went throughout the land of Egypt, \\
\poemlll       speaking a language I did not recognize.\fnote{Lit. \fbib{hear}} \\
\poeml \v{6}I removed the burden from your\fnote{Lit. \fbib{his}} shoulder; \\
\poemll    your\fnote{Lit. \fbib{his}} hands were freed of the burdensome basket.\fnote{Lit. \fbib{hands let pass through the basket}} \\
\poeml \v{7}In a time of need you called out and I delivered you; \\
\poemll    I answered you from the dark thundercloud; \\
\poemlll       I tested you at the waters of Meribah.
\end{poetry}
\interlude{Interlude}

\begin{poetry}
\poeml \v{8}Listen, My people and I will warn you. \\
\poemll    Israel, if only you would obey me! \\
\poeml \v{9}You must neither have a foreign god over you \\
\poemll    or worship a strange god. \\
\poeml \v{10}I am the \divine{Lord} your God, \\
\poemll    who brought you out of the land of Egypt, \\
\poemlll       open your mouth that I may fill it. \\
\poeml \v{11}Yet my people didn't obey my voice; \\
\poemll    Israel didn't submit to me. \\
\poeml \v{12}So I allowed them\fnote{Or \fbib{it} / \fbib{her}} to continue in their stubbornness, \\
\poemll    living by their own advice. \\
\poeml \v{13}If only my people would obey me, \\
\poemll    if only Israel would walk in my ways! \\
\poeml \v{14}Then I would quickly subdue their enemies. \\
\poemll    I would turn against their foes. \\
\poeml \v{15}Those who hate the \divine{Lord} will cringe before him; \\
\poemll    their punishment will be permanent. \\
\poeml \v{16}But I will feed Israel\fnote{Lit. \fbib{him}} with the finest wheat, \\
\poemll    satisfying you with honey from the rock.
\end{poetry}
\labelpsalm{82}
\psalminfo{A Psalm of Asaph}
\passage{Asking God for Justice}

\begin{poetry}
\poeml \v{1}God takes his stand in the divine assembly; \\
\poemll    among the divine\fnote{Or \fbib{angelic}} beings\fnote{Or \fbib{the gods}} he renders judgment: \\
\poeml \v{2}``How long will you judge partially \\
\poemll    by showing favor on the wicked?\fnote{Lit. \fbib{you lift the face}}
\end{poetry}
\interlude{Interlude}

\begin{poetry}
\poeml \v{3}``Defend the poor and the fatherless. \\
\poemll    Vindicate the afflicted and the poor. \\
\poeml \v{4}Rescue the poor and the needy, \\
\poemll    delivering them from the power of the wicked. \\
\poeml \v{5}They neither know nor understand; \\
\poemll    they walk about in the dark \\
\poemlll       while all the foundations of the earth are shaken. \\
\poeml \v{6}``Indeed I said, `You are gods, \\
\poemll    and all of you are sons of the Most High. \\
\poeml \v{7}However, as all human beings do, you will die, \\
\poemll    and like other rulers, you will fall.' \\
\poeml \v{8}Arise, God, to judge the earth, \\
\poemll    for all nations belong to you.
\end{poetry}
\labelpsalm{83}
\psalminfo{A song. A Psalm of Asaph}
\passage{A Plea for Judgment}

\begin{poetry}
\poeml \v{1}God, do not rest! \\
\poemll    Don't be silent! \\
\poemlll       Don't stay inactive, God! \\
\poeml \v{2}See! Your enemies rage; \\
\poemll    those who hate you issue threats.\fnote{Lit. \fbib{you lift their head}} \\
\poeml \v{3}They plot against\fnote{Lit. \fbib{they make shrewd secret counsel}} your people \\
\poemll    and conspire against your cherished ones. \\
\poeml \v{4}They say, ``Let us go and erase them as a nation \\
\poemll    so the name of Israel will not be remembered anymore.'' \\
\poeml \v{5}Indeed, they shrewdly planned together, \\
\poemll    forming an alliance against you--- \\
\poeml \v{6}the tents of Edom, the Ishmaelites, \\
\poemll    Moab, the Hagrites, \\
\poeml \v{7}Gebal, Ammon, Amalek, Philistia, \\
\poemll    and the inhabitants of Tyre. \\
\poeml \v{8}Even Assyria joined them \\
\poemll    to strengthen the descendants of Lot.
\end{poetry}
\interlude{Interlude}

\begin{poetry}
\poeml \v{9}Deal with them as you did to Midian,\fnote{Cf. Judg 7:1-24} \\
\poemll    Sisera, and Jabin at the Kishon Brook.\fnote{Cf. Judg 4:7, 15, 21-24} \\
\poeml \v{10}They were destroyed at En-dor \\
\poemll    and became as dung on the ground. \\
\poeml \v{11}Punish their nobles like Oreb and Zeeb,\fnote{Cf. Judg 7:25} \\
\poemll    and all their princes like Zebah and Zalmunna,\fnote{Cf. Judg 8:12, 21} \\
\poeml \v{12}who said, ``Let us possess the pastures of God.'' \\
\poeml \v{13}God, set them up like dried thistles, \\
\poemll    like straw before the wind. \\
\poeml \v{14}Like a fire burning a forest, \\
\poemll    and a flame setting mountains ablaze. \\
\poeml \v{15}Pursue them with your storm and \\
\poemll    terrify them with your whirlwind. \\
\poeml \v{16}Fill their faces with shame \\
\poemll    until they seek your name, God. \\
\poeml \v{17}Let them be humiliated and terrified permanently \\
\poemll    until they die in shame.\fnote{Lit. \fbib{they are abased and destroyed}} \\
\poeml \v{18}Then they will know that you alone--- \\
\poemll    whose name is \divine{Lord}--- \\
\poemlll       are the Most High over all the earth.
\end{poetry}
\labelpsalm{84}
\psalminfo{To the Director: On the Gittith. \\ A Psalm by the descendants of Korah.}
\passage{Longing for God}

\begin{poetry}
\poeml \v{1}How lovely are your dwelling places, \\
\poemll    \divine{Lord} of the Heavenly Armies. \\
\poeml \v{2}I desire and long \\
\poemll    for the Temple\fnote{The Heb. lacks \fbib{temple}} courts of the \divine{Lord}. \\
\poeml My heart and body\fnote{Lit. \fbib{flesh}} sing for joy \\
\poemll    to the living God.\fnote{Or \fbib{the God of life}} \\
\poeml \v{3}Even the sparrow found a house for herself \\
\poemll    and the swallow a nest \\
\poeml to lay\fnote{Or \fbib{to set up}} her young at your altar, \\
\poemll    \divine{Lord} of the Heavenly Armies, \\
\poemlll       my king and God. \\
\poeml \v{4}How happy are those who live in your Temple, \\
\poemll    for they can praise you continuously.
\end{poetry}
\interlude{Interlude}

\begin{poetry}
\poeml \v{5}How happy are those whose strength is in you, \\
\poemll    whose heart is on your path. \\
\poeml \v{6}They will pass through the Baca Valley \\
\poemll    where he will prepare a spring for them; \\
\poemlll       even the early rain will cover it with blessings. \\
\poeml \v{7}They will walk from strength to strength; \\
\poemll    each will appear before God in Zion. \\
\poeml \v{8}\divine{Lord} God of the Heavenly Armies, hear my prayer! \\
\poemll    Listen, God of Jacob!
\end{poetry}
\interlude{Interlude}

\begin{poetry}
\poeml \v{9}God, look at our shield, \\
\poemll    and show favor to your anointed, \\
\poeml \v{10}for a day in your Temple\fnote{The Heb. lacks \fbib{temple}} courts is better \\
\poemll    than a thousand elsewhere; \\
\poeml I would rather stand \\
\poemll    at the entrance of God's house \\
\poemlll       than live in the tent of wickedness. \\
\poeml \v{11}For the \divine{Lord} God is a sun and shield; \\
\poemll    the \divine{Lord} grants grace and favor; \\
\poeml the \divine{Lord} will not withhold any good thing \\
\poemll    from those who walk blamelessly. \\
\poeml \v{12}\divine{Lord} of Heavenly Armies, \\
\poemll    how happy are those who trust in you.
\end{poetry}
\labelpsalm{85}
\psalminfo{To the Director: A Psalm by the descendants of Korah.}
\passage{Restore Us, God}

\begin{poetry}
\poeml \v{1}\divine{Lord} you have favored your land \\
\poemll    and restored the fortunes of Jacob. \\
\poeml \v{2}You took away the iniquity of your people, \\
\poemll    forgiving all their sins
\end{poetry}
\interlude{Interlude}

\begin{poetry}
\poeml \v{3}You withdrew all your wrath \\
\poemll    and turned away from your burning anger. \\
\poeml \v{4}Restore us, God of our salvation, \\
\poemll    and stop being angry with us. \\
\poeml \v{5}Will you be angry with us forever? \\
\poemll    Will you prolong your anger from generation to generation? \\
\poeml \v{6}Will you restore our lives again \\
\poemll    so that your people may rejoice in you? \\
\poeml \v{7}\divine{Lord}, show your gracious love \\
\poemll    and deliver us. \\
\poeml \v{8}Let me listen to what God, the \divine{Lord}, says; \\
\poemll    for the \divine{Lord} will promise peace \\
\poeml to his people, to his holy ones; \\
\poemll    may they not return to foolishness. \\
\poeml \v{9}Surely, he will soon deliver those who fear him, \\
\poemll    for his glory will live in our land. \\
\poeml \v{10}Gracious love and truth meet; \\
\poemll    righteousness and peace kiss. \\
\poeml \v{11}Truth sprouts up from the ground, \\
\poemll    while righteousness looks down from the sky. \\
\poeml \v{12}The \divine{Lord} will also provide what is good, \\
\poemll    and our land will yield its produce. \\
\poeml \v{13}Righteousness will go before him \\
\poemll    to prepare a path for his steps.
\end{poetry}
\labelpsalm{86}
\psalminfo{A Davidic prayer}
\passage{Help Us, God}

\begin{poetry}
\poeml \v{1}\divine{Lord}, listen and answer me, \\
\poemll    for I am afflicted and needy. \\
\poeml \v{2}Protect me, for I am faithful;\fnote{Or \fbib{righteous}} \\
\poemll    My God, deliver your servant who trusts in you. \\
\poeml \v{3}Have mercy on me Lord, \\
\poemll    for I call on you all day long. \\
\poeml \v{4}Your servant rejoices, \\
\poemll    because, Lord, I set my hope on\fnote{Or \fbib{I lift my soul to}} you. \\
\poeml \v{5}Indeed you, Lord, are kind and forgiving, \\
\poemll    overflowing with gracious love to everyone who calls on you. \\
\poeml \v{6}Hear my prayer, \divine{Lord}; \\
\poemll    attend to my prayer of supplication. \\
\poeml \v{7}In my troubled times I will call on you, \\
\poemll    for you will answer me. \\
\poeml \v{8}No one can compare with you among the gods, Lord; \\
\poemll    No one can accomplish\fnote{The Heb. lacks \fbib{can accomplish}} your work. \\
\poeml \v{9}All the nations that you have established will come \\
\poemll    and worship you, my Lord. \\
\poemlll       They will honor your name. \\
\poeml \v{10}For you are great, \\
\poemll    and you are doing awesome things; \\
\poemlll       you alone are God. \\
\poeml \v{11}Teach me your ways, \divine{Lord}, \\
\poemll    that I may walk in your truth; \\
\poemlll       let me wholeheartedly\fnote{Lit. \fbib{my heart be undivided}} revere your name. \\
\poeml \v{12}I will praise you, Lord my God, with my whole being; \\
\poemll    and I will honor your name continuously. \\
\poeml \v{13}For great is your gracious love to me; \\
\poemll    you've delivered me from the depths of Sheol.\fnote{I.e. the realm of the dead} \\
\poeml \v{14}God, arrogant men rise up against me, \\
\poemll    while a company of ruthless individuals want to kill me. \\
\poemlll       They do not have regard for you.\fnote{Lit. \fbib{don't set you before them}} \\
\poeml \v{15}But you, Lord, are a compassionate God, \\
\poemll    merciful and patient,\fnote{Or \fbib{slow to anger}} \\
\poemlll       with unending gracious love and faithfulness. \\
\poeml \v{16}Return to me and have mercy on me; \\
\poemll    clothe your servant with your strength \\
\poemlll       and deliver the son of your maid servant. \\
\poeml \v{17}Show me a sign of your goodness, \\
\poemll    so that those who hate me will see it and be ashamed. \\
\poemlll       For you, \divine{Lord}, will help and comfort me.
\end{poetry}
\labelpsalm{87}
\psalminfo{A psalm by the descendants of Korah. A song.}
\passage{The Holy City for All People}

\begin{poetry}
\poeml \v{1}God's\fnote{Lit. \fbib{His}} foundation is in the holy mountains. \\
\poeml \v{2}The \divine{Lord} loves the gates of Zion \\
\poemll    more than the dwellings of Jacob. \\
\poeml \v{3}Glorious things are spoken about you, \\
\poemll    city of God.
\end{poetry}
\interlude{Interlude}

\begin{poetry}
\poeml \v{4}I will mention Rahab and Babylon \\
\poemll    among those who acknowledge me--- \\
\poeml including Philistia, Tyre, and Ethiopia\fnote{Lit. \fbib{Cush}}--- \\
\poemll    ``This one was born there,'' they say.\fnote{The Heb. lacks \fbib{they say}} \\
\poeml \v{5}Indeed, about Zion it will be said: \\
\poemll    ``More than one person\fnote{Lit. \fbib{a man and a man}} was born in it,'' and \\
\poemlll       ``The Most High himself did\fnote{Or \fbib{secured}} it.'' \\
\poeml \v{6}The \divine{Lord} will record, \\
\poemll    as he registers the peoples,\fnote{Lit. \fbib{record, in a registry of people}} \\
\poemlll       ``This one was born there.''
\end{poetry}
\interlude{Interlude}

\begin{poetry}
\poeml \v{7}Then singers, as they play their instruments,\fnote{Or \fbib{singers and flute players}} will declare, \\
\poemll    ``All my roots\fnote{Lit. \fbib{springs}} are in you.''
\end{poetry}
\labelpsalm{88}
\psalminfo{A song. A psalm by the descendants of Korah. According to Machalath Leannoth. An instruction\fnote{T Lit. \fbib{maskil}} by Heman the Ezrahite.}
\passage{A Cry for Help}

\begin{poetry}
\poeml \v{1}\divine{Lord}, God of my salvation, \\
\poemll    by day and by night I cry out before you. \\
\poeml \v{2}Let my prayer come before you; \\
\poemll    listen\fnote{Lit. \fbib{stretch your ears}} to my cry. \\
\poeml \v{3}For my life is filled with troubles \\
\poemll    as I approach Sheol.\fnote{I.e. the realm of the dead} \\
\poeml \v{4}I am considered as one of those descending into the Pit,\fnote{I.e. the place of punishment in the afterlife} \\
\poemll    like a mighty man without strength, \\
\poeml \v{5}released to remain\fnote{The Heb. lacks \fbib{to remain}} with the dead, \\
\poemll    lying in a grave like a corpse, \\
\poeml remembered no longer, \\
\poemll    and cut off from your power. \\
\poeml \v{6}You have assigned me to the lowest part of the Pit,\fnote{I.e. the place of punishment in the afterlife} \\
\poemll    to the darkest depths. \\
\poeml \v{7}Your anger lies heavily upon me; \\
\poemll    you pound\fnote{Lit. \fbib{oppress}} me with all your waves.
\end{poetry}
\interlude{Interlude}

\begin{poetry}
\poeml \v{8}You caused my acquaintances to shun me;\fnote{Lit. \fbib{to be distant}} \\
\poemll    you make me extremely abhorrent to them. \\
\poemlll       Restrained, I am unable to go out. \\
\poeml \v{9}My eyes languish on account of my affliction; \\
\poemll    all day long I call out to you, \divine{Lord}, \\
\poemlll       I spread out my hands to you. \\
\poeml \v{10}Can you perform wonders for the dead? \\
\poemll    Can departed spirits stand up to praise you?
\end{poetry}
\interlude{Interlude}

\begin{poetry}
\poeml \v{11}Can your gracious love be declared in the grave \\
\poemll    or your faithfulness in Abaddon?\fnote{I.e. the realm of destruction in the afterlife} \\
\poeml \v{12}Can your awesome deeds be known in darkness \\
\poemll    or your righteousness in the land of oblivion? \\
\poeml \v{13}As for me, I cry out to you \divine{Lord}, \\
\poemll    and in the morning my prayer greets you. \\
\poeml \v{14}Why, \divine{Lord}, have you rejected me? \\
\poemll    Why have you hidden your face from me? \\
\poeml \v{15}Since my youth I have been oppressed \\
\poemll    and in danger of death. \\
\poeml I bear your dread \\
\poemll    and am overwhelmed. \\
\poeml \v{16}Your burning anger overwhelms me; \\
\poemll    your terrors destroy me. \\
\poeml \v{17}Like waters, they engulf me all day long; \\
\poemll    they surround me on all sides. \\
\poeml \v{18}You caused my friend and neighbor to shun me;\fnote{Lit. \fbib{be distant from}} \\
\poemll    and my acquaintances are confused.\fnote{Lit. \fbib{are in darkness}}
\end{poetry}
\labelpsalm{89}
\psalminfo{An instruction\fnote{T Lit. \fbib{maskil}}. By Ethan, the Ezrahite}
\passage{God's Covenant with David}

\begin{poetry}
\poeml \v{1}I will sing forever about the gracious love of the \divine{Lord}; \\
\poemll    from generation to generation \\
\poemlll       I will declare your faithfulness with my mouth. \\
\poeml \v{2}I will declare that your gracious love was established forever; \\
\poemll    in the heavens itself, you have established your faithfulness. \\
\poeml \v{3}I have made a covenant with my chosen one; \\
\poemll    I have made a promise to David, my servant. \\
\poeml \v{4}``I will establish your dynasty forever, \\
\poemll    and I will lift up one who will build\fnote{Or \fbib{confirm}} your throne \\
\poemlll       from generation to generation.''
\end{poetry}
\interlude{Interlude}

\begin{poetry}
\poeml \v{5}Even the heavens praise your awesome deeds, \divine{Lord}, \\
\poemll    your faithfulness in the assembly of the holy ones. \\
\poeml \v{6}For who in the skies compares to the \divine{Lord}? \\
\poemll    Who is like the \divine{Lord} among the divine beings? \\
\poeml \v{7}God is feared in the council of the holy ones, \\
\poemll    revered by all those around him. \\
\poeml \v{8}\divine{Lord} God of the Heavenly Armies, \\
\poemll    who is as mighty as you, \divine{Lord}? \\
\poemlll       Your faithfulness surrounds you. \\
\poeml \v{9}You rule over the majestic\fnote{Lit. \fbib{roaring}} sea; \\
\poemll    when its waves surge, \\
\poemlll       you calm them. \\
\poeml \v{10}You crushed the proud one\fnote{Lit. \fbib{Rahab}} to death; \\
\poemll    with your powerful arm \\
\poemlll       you scattered your enemies. \\
\poeml \v{11}Heaven and the earth belong to you, \\
\poemll    the world and everything it contains--- \\
\poemlll       you established them. \\
\poeml \v{12}The north and south---you created them; \\
\poemll    Tabor and Hermon joyously praise your name. \\
\poeml \v{13}Your arm is strong; \\
\poemll    your hand is mighty; \\
\poemlll       indeed, your right hand is victorious.\fnote{Lit. \fbib{lifted up}} \\
\poeml \v{14}Righteousness and justice make up \\
\poemll    the foundation of your throne; \\
\poemlll       gracious love and truth meet before you. \\
\poeml \v{15}How happy are the people who can worship joyfully!\fnote{Lit. \fbib{who know the joyful shout}} \\
\poemll    \divine{Lord}, they walk in the light of your presence. \\
\poeml \v{16}In your name they rejoice all day long; \\
\poemll    they exult in your justice.\fnote{Or \fbib{righteousness}} \\
\poeml \v{17}For you are their strength's grandeur; \\
\poemll    by your favor you exalted our power.\fnote{Lit. \fbib{horn}} \\
\poeml \v{18}Indeed, our shield belongs to the \divine{Lord}, \\
\poemll    and our king to the Holy One of Israel.
\end{poetry}
\passage{God's Describes His Anointed}

\begin{poetry}
\poeml \v{19}You spoke to your faithful\fnote{Or \fbib{godly}; so MT LXX; DSS 4Q98\textsuperscript{g} reads \fbib{chosen}} ones through a vision:\fnote{So MT LXX; DSS 4Q98\textsuperscript{g} reads \fbib{vision; you said}}
\end{poetry}

\begin{poetry}
\poeml ``I will set a helper over\fnote{So MT LXX; DSS 4Q98\textsuperscript{g} reads \fbib{I have lent support to}} a warrior. \\
\poemll    I will raise up a chosen one from the people. \\
\poeml \v{20}I have found my servant David; \\
\poemll    I have anointed him with my sacred oil, \\
\poeml \v{21}with whom my power\fnote{Lit. \fbib{hand}} will be firmly established; \\
\poemll    for my arm will strengthen him. \\
\poeml \v{22}No enemy will deceive him; \\
\poemll    no wicked person\fnote{Lit. \fbib{no son of iniquity}} will afflict him. \\
\poemll    \v{23}I will crush his enemies before him \\
\poemll    and strike those who hate him. \\
\poemll    \v{24}My faithfulness and gracious love will be with him, \\
\poemll    and in my name his power\fnote{Lit. \fbib{horn}} will be exalted. \\
\poeml \v{25}I will place his hand\fnote{I.e. his authority} over the sea, \\
\poemll    and his right hand\fnote{I.e. his authority} over the rivers. \\
\poeml \v{26}He will announce to me \\
\poemll    `You are my father, \\
\poemlll       my God, and the rock of my salvation.' \\
\poeml \v{27}``Indeed, I myself made him the firstborn, \\
\poemll    the highest of the kings of the earth. \\
\poeml \v{28}I will show\fnote{Lit. \fbib{keep}} my gracious love toward him forever, \\
\poemll    since my covenant is securely established with him. \\
\poeml \v{29}I will establish his dynasty\fnote{Lit. \fbib{seed}} forever, \\
\poemll    and his throne as long as heaven endures.\fnote{Lit. \fbib{as the days of the heavens}} \\
\poeml \v{30}``But if his sons abandon my laws and \\
\poemll    do not follow my ordinances, \\
\poeml \v{31}if they profane my statutes; \\
\poemll    and do not keep my commands, \\
\poeml \v{32}then I will punish their disobedience with a rod \\
\poemll    and their iniquity with lashes. \\
\poeml \v{33}But I will not cut off\fnote{Lit. \fbib{break}} my gracious love from him, \\
\poemll    and I will not stop being faithful. \\
\poeml \v{34}I will not dishonor my covenant, \\
\poemll    because I will not change what I have spoken.\fnote{Lit. \fbib{what goes out of my lips}} \\
\poeml \v{35}I have sworn by my holiness once for all: \\
\poemll    I will not lie to David. \\
\poeml \v{36}His dynasty\fnote{Lit. \fbib{seed}} will last forever \\
\poemll    and his throne will be like the sun before me. \\
\poeml \v{37}It will be established forever like the moon, \\
\poemll    a faithful witness in the sky.''
\end{poetry}
\interlude{Interlude}
\passage{A Commitment to Persevere}

\begin{poetry}
\poeml \v{38}But you have spurned, rejected, \\
\poemll    and became angry with your anointed one. \\
\poeml \v{39}You have dishonored the covenant with your servant; \\
\poemll    you have defiled his crown on the ground. \\
\poeml \v{40}You have broken through all his\fnote{Or \fbib{its}} walls; \\
\poemll    you have laid his fortresses in ruin. \\
\poeml \v{41}All who pass by on their way plunder him; \\
\poemll    he has become a reproach to his neighbors. \\
\poeml \v{42}You have exalted the right hand of his adversaries; \\
\poemll    you have caused all of his enemies to rejoice. \\
\poeml \v{43}Moreover, you have turned back the edge of his sword \\
\poemll    and did not support him in battle. \\
\poeml \v{44}You have caused his splendor\fnote{Or \fbib{luster}} to cease \\
\poemll    and cast down his throne to the ground. \\
\poeml \v{45}You have caused the days of his youth to be cut short; \\
\poemll    you have covered him with shame.
\end{poetry}
\interlude{Interlude}

\begin{poetry}
\poeml \v{46}How long, \divine{Lord}, will you hide yourself? Forever? \\
\poemll    Will your anger continuously burn like fire? \\
\poeml \v{47}Remember how short my lifetime is! \\
\poemll    How powerless have you created all human beings!\fnote{Lit. \fbib{all sons of Adam}} \\
\poeml \v{48}What valiant man can live and not see death? \\
\poemll    Who can deliver himself\fnote{Lit. \fbib{deliver his soul}} from the power\fnote{Lit. \fbib{hand}} of Sheol.\fnote{I.e. the realm of the dead}
\end{poetry}
\interlude{Interlude}

\begin{poetry}
\poeml \v{49}Where is your gracious love of old, Lord, \\
\poemll    that in your faithfulness you promised to David? \\
\poeml \v{50}Remember, Lord, the reproach of your servant! \\
\poemll    I carry inside me all the insults of many people, \\
\poeml \v{51}when your enemies reproached you, \divine{Lord}, \\
\poemll    when they reproached the footsteps\fnote{Lit. \fbib{the hind part}} of your anointed. \\
\poeml \v{52}Blessed is the \divine{Lord} forever! \\
\poemll    Amen and amen!
\end{poetry}
\booksection{\divine{BOOK IV} (Psalms 90-106)}
\labelpsalm{90}
\psalminfo{A prayer by Moses, the godly man}
\passage{Life is Short}

\begin{poetry}
\poeml \v{1}Lord, you've been our refuge\fnote{Or \fbib{our dwelling place}} \\
\poemll    from generation to generation. \\
\poeml \v{2}Before the mountains were formed \\
\poemll    or the earth and the world were brought forth, \\
\poemlll       you are God from eternity to eternity. \\
\poeml \v{3}You return people to dust \\
\poemll    merely by\fnote{The Heb. lacks \fbib{merely by}} saying, ``Return, you mortals!'' \\
\poeml \v{4}One thousand years in your sight are but a single day \\
\poemll    that passes by, just like a night watch. \\
\poeml \v{5}You will sweep them away while they are asleep--- \\
\poemll    by morning they are like growing grass. \\
\poeml \v{6}In the morning it blossoms and is renewed, \\
\poemll    but toward evening, it fades and withers. \\
\poeml \v{7}Indeed, we are consumed\fnote{Lit. \fbib{finished}} by your anger \\
\poemll    and terrified by your wrath. \\
\poeml \v{8}You have set our iniquities before you, \\
\poemll    what we have concealed in the light of your presence. \\
\poeml \v{9}All our days pass\fnote{Or \fbib{turn}} away in your wrath; \\
\poemll    our years fade away\fnote{Lit. \fbib{are finished}} and end like a sigh. \\
\poeml \v{10}We live for 70 years, \\
\poemll    or 80 years if we're healthy,\fnote{Lit. \fbib{strong}} \\
\poeml yet even in the prime years\fnote{Lit. \fbib{the pride}} there are troubles and sorrow. \\
\poemll    They pass by quickly and we fly away. \\
\poeml \v{11}Who can know the intensity of your anger? \\
\poemll    Because our fear of you matches your wrath, \\
\poeml \v{12}teach us to keep account of our days \\
\poemll    so we may develop inner wisdom. \\
\poeml \v{13}Please return, \divine{Lord}! When will it be? \\
\poemll    Comfort your servants. \\
\poeml \v{14}Satisfy us in the morning with your gracious love \\
\poemll    so we may sing for joy \\
\poemlll       and rejoice every day. \\
\poeml \v{15}Cause us to rejoice throughout the time when you have afflicted us, \\
\poemll    the years when we have known\fnote{Lit. \fbib{seen}} trouble. \\
\poeml \v{16}May your awesome deeds be revealed to your servants, \\
\poemll    as well as your splendor to their children. \\
\poeml \v{17}May your favor be on us, Lord our God; \\
\poemll    make our endeavors successful; \\
\poemlll       yes, make our endeavors secure!
\end{poetry}
\labelpsalm{91}
\psalminfo{A Davidic Psalm\fnote{T So LXX; DSS 11QPs\textsuperscript{a} lacks \fbib{Psalm}; the Heb. lacks this line}}
\passage{God is My Refuge}

\begin{poetry}
\poeml \v{1}The one who lives in the shelter of the Most High, \\
\poemll    who rests in the shadow of the Almighty, \\
\poeml \v{2}will say to the \divine{Lord}, \\
\poemll    ``You are my refuge, my fortress, \\
\poemlll       and my God in whom I trust!'' \\
\poeml \v{3}He will surely deliver you from the hunter's snare \\
\poemll    and from the destructive plague. \\
\poeml \v{4}With his feathers he will cover you, \\
\poemll    under his wings you will find safety. \\
\poemlll       His truth is your shield and armor. \\
\poeml \v{5}You need not fear terror that stalks\fnote{The Heb. lacks \fbib{that stalks}} in the night, \\
\poemll    the arrow that flies in the day, \\
\poeml \v{6}plague that strikes in the darkness, \\
\poemll    or calamity that destroys at noon. \\
\poeml \v{7}If a thousand fall at your side \\
\poemll    or ten thousand at your right hand, \\
\poemlll       it will not overcome you. \\
\poeml \v{8}Only observe\fnote{Or \fbib{Only you will observe}} it with your eyes, \\
\poemll    and you will see how the wicked are paid back. \\
\poeml \v{9}``\divine{Lord}, you are my refuge!'' \\
\poeml Because you chose the Most High as your dwelling place, \\
\poeml \v{10}no evil will fall upon you, \\
\poemlll       and no affliction will approach your tent, \\
\poeml \v{11}for he will command his angels \\
\poemll    to protect you in all your ways. \\
\poeml \v{12}With their hands they will lift you up \\
\poemll    so you will not trip over a stone. \\
\poeml \v{13}You will stomp on lions and snakes; \\
\poemll    you will trample young lions and serpents.
\passage{The \divine{Lord} Speaks}
\poeml \v{14}Because he has focused his love on me, \\
\poemll    I will deliver him. \\
\poeml I will protect him\fnote{Or \fbib{will set him on high}} \\
\poemll    because he knows my name. \\
\poeml \v{15}When he calls out to me, \\
\poemll    I will answer him. \\
\poeml I will be with him in his\fnote{The Heb. lacks \fbib{his}} distress. \\
\poemll    I will deliver him, \\
\poemlll       and I will honor him. \\
\poeml \v{16}I will satisfy him with long life; \\
\poemll    I will show him my deliverance.
\end{poetry}
\labelpsalm{92}
\psalminfo{A Psalm. A song for the Sabbath Day}
\passage{Praise and Thanksgiving to God}

\begin{poetry}
\poeml \v{1}It is good to give thanks to the \divine{Lord} \\
\poemll    and to sing praise to your name, Most High; \\
\poeml \v{2}to proclaim your gracious love in the morning \\
\poemll    and your faithfulness at night, \\
\poeml \v{3}accompanied by a ten-stringed instrument and a lyre, \\
\poemll    and the contemplative sound of a harp. \\
\poeml \v{4}Because you made me glad \\
\poemll    with your awesome deeds, \divine{Lord}, \\
\poemlll       I will sing for joy at the works of your hands. \\
\poeml \v{5}How great are your works, \divine{Lord}! \\
\poemll    Your thoughts are unfathomable.\fnote{Lit. \fbib{very deep}} \\
\poeml \v{6}A stupid man doesn't know, \\
\poemll    and a fool can't comprehend this: \\
\poeml \v{7}Though the wicked sprout like grass; \\
\poemll    and all who practice iniquity flourish, \\
\poemlll       it is they who will be eternally destroyed. \\
\poeml \v{8}But you are exalted forever, \divine{Lord}. \\
\poeml \v{9}Look at your enemies, \divine{Lord}! \\
\poemll    Look at your enemies, for they are destroyed; \\
\poemlll       everyone who practices iniquity will be scattered.\fnote{Lit. \fbib{divided}; or \fbib{separated}} \\
\poeml \v{10}You've grown my strength\fnote{Lit. \fbib{horn}} like the horn of a wild ox; \\
\poemll    I was anointed with fresh oil. \\
\poeml \v{11}My eyes gloated over those who lie in wait for me;\fnote{The Heb. lacks \fbib{for me}} \\
\poemll    when those of evil intent attack me, my ears will hear. \\
\poeml \v{12}The righteous will flourish like palm trees; \\
\poemll    they will grow like a cedar in Lebanon. \\
\poeml \v{13}Planted in the \divine{Lord}'s Temple, \\
\poemll    they will flourish in the courtyard of our God. \\
\poeml \v{14}They will still bear fruit even in old age;\fnote{Lit. \fbib{Even with gray hair}} \\
\poemll    they will be luxuriant and green. \\
\poeml \v{15}They will proclaim: ``The \divine{Lord} is upright; \\
\poemll    my rock, in whom there is no injustice.''
\end{poetry}
\labelpsalm{93}
\passage{God Reigns}

\begin{poetry}
\poeml \v{1}The \divine{Lord} reigns! He is clothed in majesty; \\
\poemll    the \divine{Lord} is clothed, \\
\poemlll       and he is girded\fnote{So MT; DSS 11QPsa reads \fbib{is robed in power and girded himself}} with strength. \\
\poeml Indeed, the world is well established, \\
\poemll    and cannot be shaken. \\
\poeml \v{2}Your throne has been established since time immemorial; \\
\poemll    you are king from eternity. \\
\poeml \v{3}The rivers have flooded, \divine{Lord}; \\
\poemll    the rivers have spoken aloud, \\
\poemlll       the rivers have lifted up their crushing waves. \\
\poeml \v{4}More than the sound of surging waters--- \\
\poemll    the majestic waves of the sea--- \\
\poemlll       the \divine{Lord} on high is majestic. \\
\poeml \v{5}Your decrees are very trustworthy, \\
\poemll    and holiness always befits your house, \divine{Lord}.
\end{poetry}
\labelpsalm{94}
\passage{God Avenges His Own}

\begin{poetry}
\poeml \v{1}God of vengeance, \\
\poemll    \divine{Lord} God of vengeance, \\
\poemlll       display your splendor!\fnote{The Heb. lacks \fbib{your splendor}} \\
\poeml \v{2}Stand up, judge of the earth, \\
\poemll    and repay the proud. \\
\poeml \v{3}How long will the wicked, \divine{Lord}, \\
\poemll    how long will the wicked continue to triumph? \\
\poeml \v{4}When they speak, they spew arrogance. \\
\poemll    Everyone who practices iniquity brags about it.\fnote{The Heb. lacks \fbib{about it}} \\
\poeml \v{5}\divine{Lord}, they have crushed your people, \\
\poemll    afflicting your heritage. \\
\poeml \v{6}The wicked\fnote{Lit. \fbib{They}} kill widows and foreigners; \\
\poemll    they murder orphans. \\
\poeml \v{7}They say, ``The \divine{Lord} cannot see, \\
\poemll    and the God of Jacob will not notice.'' \\
\poeml \v{8}Pay attention, you dull ones among the crowds! \\
\poemll    You fools! Will you ever become wise? \\
\poeml \v{9}The one who formed\fnote{Lit. \fbib{planted}} the ear can hear, can he not? \\
\poemll    The one who made the eyes can see, can he not? \\
\poeml \v{10}The one who disciplines nations can rebuke them, can he not? \\
\poemll    The one who teaches mankind can discern, can he not? \\
\poeml \v{11}The \divine{Lord} knows the thoughts of human beings--- \\
\poemll    that they are futile. \\
\poeml \v{12}How blessed is the man whom you instruct, \divine{Lord}, \\
\poemll    whom you teach from your Law, \\
\poeml \v{13}keeping him calm when times are troubled \\
\poemll    until a pit has been dug for the wicked. \\
\poeml \v{14}For the \divine{Lord} will not forsake his people; \\
\poemll    he will not abandon his heritage. \\
\poeml \v{15}Righteousness will be restored with justice, \\
\poemll    and all the pure of heart will follow it. \\
\poeml \v{16}Who will rise up for me against the wicked? \\
\poemll    Who will stand for me against those who practice iniquity? \\
\poeml \v{17}If the \divine{Lord} had not been my helper, \\
\poemll    I would have quickly become silent. \\
\poeml \v{18}When I say that my foot is shaking, \\
\poemll    your gracious love, \divine{Lord}, will sustain me. \\
\poeml \v{19}When my anxious inner thoughts become overwhelming, \\
\poemll    your comfort encourages me. \\
\poeml \v{20}Will destructive national leaders,\fnote{Lit. \fbib{destructive throne}} \\
\poemll    who plan wicked things through misuse of the Law, \\
\poemlll       be allied with you? \\
\poeml \v{21}They gather together against the righteous, \\
\poemll    condemning the innocent to death. \\
\poeml \v{22}But the \divine{Lord} is my stronghold, \\
\poemll    and my God, the rock, is my refuge. \\
\poeml \v{23}He will repay them for their sin; \\
\poemll    he will annihilate them because of their evil. \\
\poemlll       The \divine{Lord} our God will annihilate them.
\end{poetry}
\labelpsalm{95}
\passage{Worship and Obedience}

\begin{poetry}
\poeml \v{1}Come! Let us sing joyfully to the \divine{Lord}! \\
\poemll    Let us shout for joy to the rock of our salvation. \\
\poeml \v{2}Let us come into his presence with thanksgiving; \\
\poemll    let us shout with songs of praise to him. \\
\poeml \v{3}For the \divine{Lord} is an awesome God; \\
\poemll    a great king above all divine beings.\fnote{Or \fbib{all gods}} \\
\poeml \v{4}He holds in his hand the lowest parts of the earth \\
\poemll    and the mountain peaks belong to him. \\
\poeml \v{5}The sea that he made belongs to him, \\
\poemll    along with the dry land that his hands formed. \\
\poeml \v{6}Come! Let us worship and bow down; \\
\poemll    let us kneel in the presence of the \divine{Lord}, who made us. \\
\poeml \v{7}For he is our God, and we are the people of his pasture \\
\poemll    and the flock in his care.\fnote{Lit. \fbib{flock of his hand}} \\
\poeml If only you would listen to his voice today, \\
\poeml \v{8}do not be stubborn like your ancestors were\fnote{Lit. \fbib{stubborn as at}} at Meribah, \\
\poeml as on that day at Massah, in the wilderness, \\
\poeml \v{9}where your ancestors tested me. \\
\poeml They tested me, \\
\poemll    even though they had seen my awesome deeds. \\
\poeml \v{10}For forty years I loathed that generation, so I said, \\
\poemll    ``They are a people whose hearts continuously err, \\
\poemlll       and they have not understood my ways.'' \\
\poeml \v{11}So in my anger I declared an oath: \\
\poemll    ``They are not to enter my place of rest.''
\end{poetry}
\labelpsalm{96}
\passage{Give Glory to the \divine{Lord}}

\begin{poetry}
\poeml \v{1}Sing a new song to the \divine{Lord}! \\
\poemll    Sing to the \divine{Lord}, all the earth! \\
\poeml \v{2}Sing to the \divine{Lord}! \\
\poemll    Bless his name! \\
\poemlll       Proclaim his deliverance every day! \\
\poeml \v{3}Declare his glory among the nations \\
\poemll    and his awesome deeds among all the peoples! \\
\poeml \v{4}For the \divine{Lord} is great, \\
\poemll    and greatly to be praised; \\
\poemlll       he is awesome above all gods. \\
\poeml \v{5}For all the gods of the peoples are worthless idols, \\
\poemll    but the \divine{Lord} made the heavens. \\
\poeml \v{6}Splendor and majesty are before him; \\
\poemll    might and beauty are in his sanctuary. \\
\poeml \v{7}Ascribe to the \divine{Lord}, you families of peoples, \\
\poemll    ascribe to the \divine{Lord} glory and strength! \\
\poeml \v{8}Ascribe to the \divine{Lord} the glory due his name, \\
\poemll    bring an offering and enter his courts! \\
\poeml \v{9}Worship the \divine{Lord} in holy splendor; \\
\poemll    tremble before him, all the earth. \\
\poeml \v{10}Declare among the nations, ``The \divine{Lord} reigns!'' \\
\poemll    Indeed, he established the world so that it will not falter. \\
\poemlll       He will judge people fairly. \\
\poeml \v{11}The heavens will be glad \\
\poemll    and the earth will rejoice; \\
\poemlll       even the sea and everything that fills it will roar.\fnote{Or \fbib{thunder}} \\
\poeml \v{12}The field and all that is in it will rejoice; \\
\poemll    then all the trees of the forest will sing for joy \\
\poeml \v{13}in the \divine{Lord}'s presence, \\
\poeml because he is coming; \\
\poemll    indeed, he will come to judge the earth. \\
\poeml He will judge the world fairly \\
\poemll    and its people reliably.
\end{poetry}
\labelpsalm{97}
\passage{The \divine{Lord} is King}

\begin{poetry}
\poeml \v{1}The \divine{Lord} reigns! \\
\poemll    Let the earth rejoice! \\
\poemlll       May many islands be glad! \\
\poeml \v{2}Thick clouds are all around him; \\
\poemll    righteousness and justice are his throne's foundation. \\
\poeml \v{3}Fire goes out from his presence \\
\poemll    to consume his enemies on every side. \\
\poeml \v{4}His lightning bolts light the world; \\
\poemll    the earth sees and shakes. \\
\poeml \v{5}Mountains melt like wax in the \divine{Lord}'s presence--- \\
\poemll    In the presence of the \divine{Lord} of all the earth. \\
\poeml \v{6}The heavens declare his righteousness \\
\poemll    so that all the nations see his glory. \\
\poeml \v{7}All who serve carved images--- \\
\poemll    and those who praise idols---will be humiliated. \\
\poemlll       Worship him, all you ``gods''! \\
\poeml \v{8}Zion hears and rejoices; \\
\poemll    the towns\fnote{Lit. \fbib{daughters}} of Judah rejoice \\
\poemlll       on account of your justice, \divine{Lord}. \\
\poeml \v{9}For you, \divine{Lord}, are the Most High above all the earth; \\
\poemll    you are exalted high above all divine beings.\fnote{Or \fbib{all gods}} \\
\poeml \v{10}Hate evil, you who love the \divine{Lord}! \\
\poemll    He guards the lives of those who love him,\fnote{Or \fbib{his saints}} \\
\poemlll       delivering them from domination by\fnote{Lit. \fbib{from the hand of}} the wicked. \\
\poeml \v{11}Light shines on the righteous; \\
\poemll    gladness on the morally upright.\fnote{Lit. \fbib{the upright of heart}} \\
\poeml \v{12}Rejoice in the \divine{Lord}, you righteous ones! \\
\poemll    Give thanks at the mention of his holiness!
\end{poetry}
\labelpsalm{98}
\psalminfo{A psalm}
\passage{Sing Praise to the King}

\begin{poetry}
\poeml \v{1}Sing to the \divine{Lord} a new song, \\
\poemll    for he has done awesome deeds! \\
\poeml His right hand and powerful\fnote{Lit. \fbib{holy}} arm\fnote{I.e. \fbib{the Messiah}} \\
\poemll    have brought him victory. \\
\poeml \v{2}The \divine{Lord} has made his deliverance known; \\
\poemll    he has disclosed his justice before the nations. \\
\poeml \v{3}He has remembered his gracious love; \\
\poemll    his faithfulness toward the house of Israel; \\
\poemlll       all the ends of the earth saw our God's deliverance. \\
\poeml \v{4}Make a joyful noise to the \divine{Lord}, all the earth! \\
\poemll    Break forth into joyful songs of praise! \\
\poeml \v{5}Sing praises to the \divine{Lord} with a lyre--- \\
\poemll    with a lyre and a melodious song! \\
\poeml \v{6}With trumpets and the sound of a ram's horn \\
\poemll    shout in the presence of the \divine{Lord}, the king! \\
\poeml \v{7}Let the sea and everything in it shout,\fnote{Lit. \fbib{thunder}} \\
\poemll    along with the world and its inhabitants; \\
\poeml \v{8}let the rivers clap their hands in unison; \\
\poemll    and let the mountains sing for joy \\
\poeml \v{9}in the \divine{Lord}'s presence, who comes to judge the earth; \\
\poeml He'll judge the world righteously; \\
\poemll    and its people fairly.
\end{poetry}
\labelpsalm{99}
\passage{The \divine{Lord} is Holy}

\begin{poetry}
\poeml \v{1}The \divine{Lord} reigns--- \\
\poemll    let people tremble; \\
\poeml he is seated above the cherubim--- \\
\poemll    let the earth quake. \\
\poeml \v{2}The \divine{Lord} is great in Zion \\
\poemll    and is exalted above all peoples. \\
\poeml \v{3}Let them praise your great and awesome name. \\
\poemll    He is holy! \\
\poeml \v{4}A mighty king who loves justice, \\
\poemll    you have established fairness. \\
\poeml You have exercised justice \\
\poemll    and righteousness over Jacob. \\
\poeml \v{5}Exalt the \divine{Lord} our God; \\
\poemll    worship and bow down at his footstool; \\
\poemlll       He is holy! \\
\poeml \v{6}Moses and Aaron were among his priests; \\
\poemll    Samuel also was among those who invoked his name. \\
\poeml When they called on the \divine{Lord}, \\
\poemll    he answered them. \\
\poeml \v{7}In a pillar of cloud he spoke to them; \\
\poemll    they obeyed his decrees \\
\poemlll       and the Law that he gave them. \\
\poeml \v{8}\divine{Lord} our God, you answered them; \\
\poemll    you were their God who forgave\fnote{Lit. \fbib{carried}} them, \\
\poemlll       but also avenged their evil deeds. \\
\poeml \v{9}Exalt the \divine{Lord} our God and worship at his holy mountain, \\
\poemll    for the \divine{Lord} our God is holy!
\end{poetry}
\labelpsalm{100}
\psalminfo{A psalm of thanksgiving}
\passage{Give Thanks to the \divine{Lord}}

\begin{poetry}
\poeml \v{1}Shout to the \divine{Lord} all the earth! \\
\poeml \v{2}Serve the \divine{Lord} with joy. \\
\poemlll       Come before him with a joyful shout! \\
\poeml \v{3}Acknowledge that the \divine{Lord} is God. \\
\poemll    He made us and we belong to him; \\
\poeml we are his people \\
\poemll    and the sheep of his pasture. \\
\poeml \v{4}Enter his gates with thanksgiving \\
\poemll    and his courts with praise. \\
\poeml Thank him and bless his name, \\
\poeml \v{5}for the \divine{Lord} is good \\
\poemlll       and his gracious love stands forever. \\
\poeml His faithfulness remains from generation to generation.
\end{poetry}
\labelpsalm{101}
\psalminfo{A Davidic Psalm}
\passage{Remembering God's Love}

\begin{poetry}
\poeml \v{1}I will sing about gracious love and justice; \\
\poemll    \divine{Lord}, I will sing praise to you. \\
\poeml \v{2}I will pay attention to living a life of integrity--- \\
\poemll    when will I attain it? \\
\poemlll       I will live with integrity of heart in my house. \\
\poeml \v{3}I will not even think about doing anything lawless; \\
\poemll    I hate to do evil deeds; \\
\poemlll       I will have none of it. \\
\poeml \v{4}I will not allow anyone with a perverted mind in my presence; \\
\poemll    I will not be involved with\fnote{Lit. \fbib{not know}} anything evil. \\
\poeml \v{5}I will destroy the one who secretly slanders a friend. \\
\poemll    I will not allow the proud and haughty to prevail. \\
\poeml \v{6}My eyes are looking at the faithful of the land, \\
\poemll    so they may live with me; \\
\poemlll       The one who lives a life of integrity will serve me. \\
\poeml \v{7}A deceitful person will not sit in my house; \\
\poemll    A liar will not remain in my presence. \\
\poeml \v{8}Every morning I will destroy all the wicked of the land, \\
\poemll    eliminating everyone who practices iniquity from the \divine{Lord}'s city.
\end{poetry}
\labelpsalm{102}
\psalminfo{A prayer by the afflicted man who is overwhelmed and talks about his troubles with the \divine{Lord}.}
\passage{A Prayer for Help}

\begin{poetry}
\poeml \v{1}\divine{Lord}, hear my prayer! \\
\poemll    May my cry for help come to you. \\
\poeml \v{2}Do not hide your face from me when I am in trouble. \\
\poemll    Listen to me. \\
\poeml When I call to out you, \\
\poemll    hurry to answer me! \\
\poeml \v{3}For my days are vanishing like smoke; \\
\poemll    my bones are charred as in a fireplace. \\
\poeml \v{4}Withered like grass, my heart is overwhelmed, \\
\poemll    and I have even forgotten to eat my food. \\
\poeml \v{5}Because of the sound of my sighing, \\
\poemll    my bones cling to my skin. \\
\poeml \v{6}I resemble a pelican in the wilderness \\
\poemll    or an owl in a desolate land. \\
\poeml \v{7}I lie awake, \\
\poemll    yet I am like a bird isolated on a rooftop. \\
\poeml \v{8}My enemies revile me all day long; \\
\poemll    those who ridicule me use my name to curse. \\
\poeml \v{9}I have eaten ashes as food \\
\poemll    and mixed my drink with tears \\
\poeml \v{10}because of your indignation and wrath, \\
\poemll    when you lifted and threw me away. \\
\poeml \v{11}My life is\fnote{Lit. \fbib{My days are}} like a declining shadow, \\
\poemll    and I am withering like a plant. \\
\poeml \v{12}But you, \divine{Lord}, are enthroned forever; \\
\poemll    You are remembered throughout all generations. \\
\poeml \v{13}You will arise to extend compassion on Zion, \\
\poemll    for it is time to show her favor--- \\
\poemlll       the appointed time has come. \\
\poeml \v{14}Your servants take pleasure in its stones \\
\poemll    and delight in its debris. \\
\poeml \v{15}Nations will fear the name of the \divine{Lord}, \\
\poemll    and all the kings of the earth, your splendor. \\
\poeml \v{16}When the \divine{Lord} rebuilds Zion, \\
\poemll    he will appear in his glory. \\
\poeml \v{17}He will turn to the prayer of the destitute, \\
\poemll    not despising their prayer. \\
\poeml \v{18}Write this for the next generation, \\
\poemll    that a people yet to be created will praise the \divine{Lord}. \\
\poeml \v{19}For when he looked down from his holy heights--- \\
\poemll    the \divine{Lord} looked over the earth from heaven--- \\
\poeml \v{20}to listen to the groans of prisoners, \\
\poemll    to set free those condemned to death, \\
\poeml \v{21}so they would declare the name of the \divine{Lord} in Zion \\
\poemll    and his praise in Jerusalem, \\
\poeml \v{22}when people and kingdoms gather together \\
\poemll    to serve the \divine{Lord}. \\
\poeml \v{23}He has weakened my\fnote{So MT Qere (oral reading) DSS 4QPsb; Symmachus, Syr, Targ, and Hieronymous; MT \fbib{Qetiv} (written) reads \fbib{his}} strength along the way.\fnote{Or \fbib{strength in mid-course}} \\
\poemll    He has cut short my days. \\
\poeml \v{24}I say, ``My God, whose years continue through all generations, \\
\poemll    do not take me in the middle of my life. \\
\poeml \v{25}You established the earth long ago; \\
\poemll    the heavens are the work\fnote{So MT DSS 4QPs\textsuperscript{b}; LXX Targ DSS 11QPs\textsuperscript{a} read \fbib{works}} of your hands. \\
\poeml \v{26}They will perish, \\
\poemll    but you will remain; \\
\poeml and they all will become worn out,\fnote{So MT DSS 4QPs\textsuperscript{b} 11QPs\textsuperscript{a}; LXX reads \fbib{will grow old}} like a garment. \\
\poemll    You\fnote{So MT; LXX DSS 4QPs\textsuperscript{b} 11QPs\textsuperscript{a} read \fbib{And you}} will change them like clothing, \\
\poemlll       and they will pass away. \\
\poeml \v{27}But you remain the same; \\
\poemll    your years never end. \\
\poeml \v{28}May the descendants of your servants live securely, \\
\poemll    and may their children be established in your presence.''
\end{poetry}
\labelpsalm{103}
\psalminfo{Davidic}
\passage{Praise God, who Forgives}

\begin{poetry}
\poeml \v{1}Bless the \divine{Lord}, my soul, \\
\poemll    and all that is within me, bless\fnote{The Heb. lacks \fbib{bless}} his holy name. \\
\poeml \v{2}Bless the \divine{Lord}, my soul, \\
\poemll    and never forget any of his benefits: \\
\poeml \v{3}He continues to forgive all your sins, \\
\poemll    he continues to heal all your diseases, \\
\poeml \v{4}he continues to redeem your life from the Pit,\fnote{I.e. the place of punishment in the afterlife} \\
\poemll    and he continuously surrounds you \\
\poemlll       with gracious love and compassion. \\
\poeml \v{5}He keeps satisfying you with good things, \\
\poemll    and he keeps renewing your youth like the eagle's. \\
\poeml \v{6}The \divine{Lord} continuously does what is right, \\
\poemll    executing justice for all who are being oppressed. \\
\poeml \v{7}He revealed his plans\fnote{Lit. \fbib{ways}} to Moses \\
\poemll    and his deeds to the people of Israel. \\
\poeml \v{8}The \divine{Lord} is compassionate and gracious, \\
\poemll    patient,\fnote{Lit. \fbib{slow of anger}} and abundantly rich in gracious love. \\
\poeml \v{9}He does not maintain a dispute\fnote{Or \fbib{not rebuke}} continuously \\
\poemll    or remain angry for all time. \\
\poeml \v{10}He neither deals with us according to our sins, \\
\poemll    nor repays us equivalent to our iniquity. \\
\poeml \v{11}As high as heaven rises above earth, \\
\poemll    so his gracious love strengthens\fnote{MT phrase \fbib{high as} sounds like MT verb \fbib{strengthens}} those who fear him. \\
\poeml \v{12}As distant as the east is from the west, \\
\poemll    that is how far he has removed our sins from us. \\
\poeml \v{13}As a father has compassion for his children, \\
\poemll    so the \divine{Lord} has compassion for those who fear him. \\
\poeml \v{14}For he knows how we were formed, \\
\poemll    aware that we were made from dust. \\
\poeml \v{15}A person's life is like grass--- \\
\poemll    it blossoms like wild flowers, \\
\poeml \v{16}but when the wind blows through it, \\
\poemll    it withers away and no one remembers where it was. \\
\poeml \v{17}Yet the \divine{Lord}'s gracious love remains \\
\poemll    throughout eternity for those who fear him \\
\poemlll       and his righteous acts extend to their children's children, \\
\poeml \v{18}to those who keep his covenant \\
\poemll    and to those who remember to observe his precepts. \\
\poeml \v{19}The \divine{Lord} has established his throne in heaven \\
\poemll    and his kingdom rules over all. \\
\poeml \v{20}Bless the \divine{Lord}, you angels who belong to him, \\
\poemll    you mighty warriors who carry out his commands, \\
\poemlll       who are obedient to the sound of his words.\fnote{So LXX 4QPs\textsuperscript{b}; MT LXX read \fbib{word}} \\
\poeml \v{21}Bless the \divine{Lord}, all his heavenly armies, \\
\poemll    his ministers who do his will. \\
\poeml \v{22}Bless the \divine{Lord}, all his creation,\fnote{Lit. \fbib{works}; or \fbib{deeds}} \\
\poemll    in all the places of his dominion. \\
\poeml Bless the \divine{Lord}, my soul.
\end{poetry}
\labelpsalm{104}
\psalminfo{Davidic\fnote{T So LXX DSS 4QPs\textsuperscript{a} 11QPs\textsuperscript{a}; MT DSS 4QPs\textsuperscript{d} lack this line}}
\passage{Praise God, who Creates}

\begin{poetry}
\poeml \v{1}Bless the \divine{Lord}, my soul; \\
\poemll    \divine{Lord}, my God, you are very great. \\
\poeml You are clothed in splendor and majesty; \\
\poeml \v{2}you are wrapped in light like a garment, \\
\poemlll       stretching out the sky like a curtain. \\
\poeml \v{3}He lays the beams of his roof loft on the water above,\fnote{The Heb. lacks \fbib{above}} \\
\poemll    making clouds his chariot, \\
\poemlll       walking on the wings of the wind. \\
\poeml \v{4}He makes the winds his messengers, \\
\poemll    blazing fires his servants. \\
\poeml \v{5}He established the earth on its foundations, \\
\poemll    so that it never falters. \\
\poeml \v{6}You covered the primeval ocean like a garment; \\
\poemll    the water stood above the mountains. \\
\poeml \v{7}They flee at your rebuke; \\
\poemll    they rush away at the sound of your thunders. \\
\poeml \v{8}Mountains rise up and valleys sink \\
\poemll    to the place you have ordained for them. \\
\poeml \v{9}You have set a boundary they cannot cross; \\
\poemll    they will never again cover the earth. \\
\poeml \v{10}He causes springs to gush forth into rivers \\
\poemll    that flow between the\fnote{So LXX DSS 4QPs\textsuperscript{d}; the Heb. lacks \fbib{the}} mountains. \\
\poeml \v{11}They give water\fnote{The Heb. lacks \fbib{water}} for animals of the field to drink; \\
\poemll    the wild donkeys quench their thirst. \\
\poeml \v{12}Birds of the sky live beside them \\
\poemll    and chirp a song\fnote{Lit. \fbib{and they give a voice}} among the foliage. \\
\poeml \v{13}He waters the mountains from his heavenly rooms; \\
\poemll    the earth is satisfied from the fruit of your work. \\
\poeml \v{14}He causes grass to sprout for the cattle \\
\poemll    and plants for people to cultivate, \\
\poemlll       to produce food from the land, \\
\poeml \v{15}like wine that makes the heart of people\fnote{Lit. \fbib{man}} happy, \\
\poemll    oil that makes the face glow, \\
\poemlll       and food\fnote{Or \fbib{bread}} that sustains people.\fnote{Lit. \fbib{heart of man}} \\
\poeml \v{16}The loftiest trees\fnote{Lit. \fbib{trees of the \divine{Lord}}} are satisfied, \\
\poemll    the cedars of Lebanon that he planted, \\
\poeml \v{17}the birds build their nests there, \\
\poemll    and the heron builds\fnote{The Heb. lacks \fbib{builds}} its nest among the evergreen. \\
\poeml \v{18}The high mountains are for wild goats; \\
\poemll    the cliffs are a refuge for the rock badger. \\
\poeml \v{19}He made the moon to mark time;\fnote{Lit. \fbib{for an appointed time}} \\
\poemll    the sun knows its setting time. \\
\poeml \v{20}You bring darkness and it becomes night; \\
\poemll    when every beast of the forest prowls. \\
\poeml \v{21}Young lions roar for prey, \\
\poemll    seeking their food from God. \\
\poeml \v{22}When the sun rises, they\fnote{So MT; LXX DSS 4QPs\textsuperscript{d} 11QPs\textsuperscript{a} read \fbib{and they}} gather \\
\poemll    and lie down in their dens. \\
\poeml \v{23}People go out to their work \\
\poemll    and labor until evening. \\
\poeml \v{24}How numerous are your works, \divine{Lord}! \\
\poemll    You have made them all wisely; \\
\poemlll       the earth is filled with your creations.\fnote{Lit. \fbib{acquisitions}} \\
\poeml \v{25}There is the deep and wide sea, \\
\poemll    teeming with numberless creatures, \\
\poemlll       living things small and great. \\
\poeml \v{26}There, the ships pass through; \\
\poemll    Leviathan, which you created, frolics in it. \\
\poeml \v{27}All of them look to you \\
\poemll    to provide them\fnote{So LXX DSS 11QPs\textsuperscript{a}; the Heb. lacks \fbib{them}} their food at the proper time. \\
\poeml \v{28}They receive what you give them; \\
\poemll    when you open your hand, \\
\poemlll       they are filled with good things. \\
\poeml \v{29}When you withdraw your favor,\fnote{Lit. \fbib{you conceal your face}} \\
\poemll    they are disappointed; \\
\poeml Take away their breath, \\
\poemll    and\fnote{So MT; LXX DSS 11QPs\textsuperscript{a} read \fbib{then}} they die\fnote{So MT DSS 11QPs\textsuperscript{a}; LXX reads \fbib{they will fail}} and return to dust. \\
\poeml \v{30}When you send your spirit,\fnote{Or \fbib{breath}} they are\fnote{So MT; LXX reads \fbib{they will be}; DSS 11QPs\textsuperscript{a} reads \fbib{then they are}} created, \\
\poemll    and you replenish the surface of the earth. \\
\poeml \v{31}May the glory of the \divine{Lord} last forever; \\
\poemll    may the \divine{Lord} rejoice in his works! \\
\poeml \v{32}He looks at the earth and it shakes; \\
\poemll    he touches the mountains and they smoke. \\
\poeml \v{33}I will sing to the \divine{Lord} with my whole being;\fnote{Lit. \fbib{with my life}} \\
\poemll    I will sing to my God continually! \\
\poeml \v{34}May my thoughts be pleasing to him; \\
\poemll    indeed, I will rejoice in the \divine{Lord}! \\
\poeml \v{35}May sinners disappear from the land \\
\poemll    and the wicked live no longer. \\
\poeml Bless the \divine{Lord}, my soul! Hallelujah!
\end{poetry}
\labelpsalm{105}
\passage{Thanksgiving for God's Deliverance}

\begin{poetry}
\poeml \v{1}Give thanks to the \divine{Lord}, \\
\poemll    call on his name, \\
\poemlll       and make his deeds known among the people. \\
\poeml \v{2}Sing to him! Praise him! \\
\poemll    Declare all his awesome deeds! \\
\poeml \v{3}Exult in his holy name; \\
\poemll    let all\fnote{Lit. \fbib{Let the heart of}} those who seek the \divine{Lord} rejoice! \\
\poeml \v{4}Seek the \divine{Lord} and his strength; \\
\poemll    seek his face continually. \\
\poeml \v{5}Remember his awesome deeds that he has done, \\
\poemll    his wonders and the judgments he declared. \\
\poeml \v{6}You descendants of Abraham, his servant, \\
\poemll    You children of Jacob, his chosen ones. \\
\poeml \v{7}He is the \divine{Lord} our God; \\
\poemll    his judgments extend to the entire earth. \\
\poeml \v{8}He remembers his eternal covenant--- \\
\poemll    every promise he made\fnote{Lit. \fbib{every word he commanded}} for a thousand generations, \\
\poeml \v{9}like the covenant he made\fnote{Lit. \fbib{like he cut}} with Abraham, \\
\poemll    and his promise to Isaac. \\
\poeml \v{10}He presented it to Jacob as a decree, \\
\poemll    to Israel as an everlasting covenant. \\
\poeml \v{11}He said: ``I will give Canaan to you \\
\poemll    as the allotted portion that is your inheritance.'' \\
\poeml \v{12}When the Hebrews\fnote{Lit. \fbib{When they}} were few in number---so very few--- \\
\poemll    and were sojourners in it, \\
\poeml \v{13}they wandered from nation to nation, \\
\poemll    from one kingdom to another.\fnote{Lit. \fbib{one kingdom to another nation}} \\
\poeml \v{14}He did not allow anyone to oppress them, \\
\poemll    or any kings to reprove them. \\
\poeml \v{15}``Don't touch my anointed \\
\poemll    or hurt my prophets!'' \\
\poeml \v{16}He declared a famine on the land; \\
\poemll    destroying the entire food supply.\fnote{Lit. \fbib{every staff of bread}} \\
\poeml \v{17}He sent a man before them--- \\
\poemll    Joseph, who had been sold as a slave. \\
\poeml \v{18}They bound his feet with fetters \\
\poemll    and placed an iron collar on his neck,\fnote{Lit. \fbib{soul}} \\
\poeml \v{19}until the time his prediction came true, \\
\poemll    as the word of the \divine{Lord} refined him. \\
\poeml \v{20}He sent a king who released him, \\
\poemll    a ruler of people who set him free. \\
\poeml \v{21}He made him the master over his household, \\
\poemll    the manager of all his possessions--- \\
\poeml \v{22}to discipline his rulers at will \\
\poemll    and make his elders wise. \\
\poeml \v{23}Then Israel came to Egypt; \\
\poemll    indeed, Jacob lived in the land of Ham.\fnote{I.e. Egypt} \\
\poeml \v{24}He caused his people to multiply greatly; \\
\poemll    and be more numerous than their enemies. \\
\poeml \v{25}He caused them\fnote{Lit. \fbib{He turned their hearts}} to hate his people \\
\poemll    and to deceive his servants. \\
\poeml \v{26}He sent his servant Moses, along with Aaron, \\
\poemll    whom he had chosen. \\
\poeml \v{27}They performed his signs among them, \\
\poemll    his wonders in the land of Ham.\fnote{I.e. Egypt} \\
\poeml \v{28}He sent darkness, and it became dark. \\
\poemll    Did they not rebel against\fnote{So MT DSS 11QPs\textsuperscript{a}; LXX reads \fbib{not embitter}} his words? \\
\poeml \v{29}He turned their water into blood, \\
\poemll    so that the fish died. \\
\poeml \v{30}Their land swarmed with frogs \\
\poemll    even to the chambers of their kings. \\
\poeml \v{31}He spoke, \\
\poemll    and a swarm of insects invaded their land.\fnote{Or \fbib{borders}} \\
\poeml \v{32}He sent hail instead of rain, \\
\poemll    and lightning throughout their land. \\
\poeml \v{33}It destroyed their vines and their figs, \\
\poemll    breaking trees throughout their country.\fnote{Or \fbib{borders}} \\
\poeml \v{34}Then he commanded the locust to come--- \\
\poemll    grasshoppers without number. \\
\poeml \v{35}They consumed every green plant in their land, \\
\poemll    and devoured the fruit of their soil. \\
\poeml \v{36}He struck down every firstborn in their land, \\
\poemll    the first fruits of all their progeny. \\
\poeml \v{37}Then he brought Israel\fnote{Lit. \fbib{them}} out with silver and gold, \\
\poemll    and no one among his tribes stumbled. \\
\poeml \v{38}The Egyptians rejoiced when they left, \\
\poemll    because fear of Israel\fnote{Lit. \fbib{them}} descended on them. \\
\poeml \v{39}He spread out a cloud for a cover, \\
\poemll    and fire for light at night. \\
\poeml \v{40}Israel\fnote{The Heb. lacks \fbib{Israel}} asked, and quail came; \\
\poemll    food from heaven satisfied them. \\
\poeml \v{41}He opened a rock, and water gushed out \\
\poemll    flowing like a river in the desert. \\
\poeml \v{42}Indeed, he remembered his sacred promise \\
\poemll    to his servant Abraham. \\
\poeml \v{43}He led his people out with gladness, \\
\poemll    his chosen ones with shouts of joy. \\
\poeml \v{44}He gave to them the land of nations; \\
\poemll    they inherited the labor of other\fnote{The Heb. lacks \fbib{other}} people \\
\poeml \v{45}so they might keep his statutes \\
\poemll    and observe his laws. \\
\poemlll       Hallelujah!
\end{poetry}
\labelpsalm{106}
\passage{The Unfaithfulness of God's People}

\begin{poetry}
\poeml \v{1}Hallelujah! \\
\poeml Give thanks to the \divine{Lord}, \\
\poemll    since he is good, \\
\poemlll       for his gracious love exists forever. \\
\poeml \v{2}Who can fully describe the mighty acts of the \divine{Lord} \\
\poemll    or proclaim all his praises? \\
\poeml \v{3}How happy are those who enforce justice, \\
\poemll    who live righteously all the time. \\
\poeml \v{4}Remember me, \divine{Lord}, \\
\poemll    when you show favor to your people. \\
\poeml Visit us with your deliverance, \\
\poeml \v{5}to witness the prosperity of your chosen ones, \\
\poeml to rejoice in your nation's joy, \\
\poemll    to glory in your inheritance. \\
\poeml \v{6}We have sinned, along with our ancestors; \\
\poemll    we have committed iniquity and wickedness. \\
\poeml \v{7}In Egypt, our ancestors neither comprehended your awesome deeds \\
\poemll    nor remembered your abundant gracious love. \\
\poemlll       Instead, they rebelled beside the sea, the Reed\fnote{So MT; LXX reads \fbib{Red}} Sea. \\
\poeml \v{8}He delivered for the sake of his name,\fnote{Or \fbib{reputation}} \\
\poemll    to make his power known. \\
\poeml \v{9}He shouted at the Reed\fnote{So MT; LXX reads \fbib{Red}} Sea and it dried up; \\
\poemll    and led them through the sea as though through a desert. \\
\poeml \v{10}He delivered them from the power of their foe; \\
\poemll    redeeming them from the power of their enemy. \\
\poeml \v{11}The water overwhelmed their enemies, \\
\poemll    so that not one of them survived.\fnote{Or \fbib{remained}} \\
\poeml \v{12}Then they believed his word \\
\poemll    and sung his praise. \\
\poeml \v{13}But they quickly forgot his deeds \\
\poemll    and did not wait for his counsel. \\
\poeml \v{14}They were overwhelmed with craving in the wilderness, \\
\poemll    so God tested them in the wasteland. \\
\poeml \v{15}God granted them their request, \\
\poemll    but sent leanness into their lives. \\
\poeml \v{16}They were envious of Moses in the camp, \\
\poemll    and of Aaron, the holy one of the \divine{Lord}. \\
\poeml \v{17}The earth opened and swallowed Dathan, \\
\poemll    closing over Abiram's clan. \\
\poeml \v{18}Then a fire burned among their company, \\
\poemll    a flame that set the wicked ablaze. \\
\poeml \v{19}They fashioned a calf at Horeb \\
\poemll    and worshipped a carved image. \\
\poeml \v{20}They exchanged their glory\fnote{I.e. their God} \\
\poemll    with the image of a grass-eating bull. \\
\poeml \v{21}They forgot God their Savior, \\
\poemll    who performed great things in Egypt--- \\
\poeml \v{22}awesome deeds in the land of Ham,\fnote{I.e. Egypt} \\
\poemll    astonishing deeds at the Reed\fnote{So MT; LXX reads \fbib{Red}} Sea. \\
\poeml \v{23}He would have destroyed them \\
\poemll    but for Moses, his chosen one, \\
\poeml who stood in the breach before him \\
\poemll    to avert\fnote{Or \fbib{turn back}} his destructive wrath. \\
\poeml \v{24}They rejected the desirable land, \\
\poemll    and they didn't trust his promise. \\
\poeml \v{25}They murmured in their tents, \\
\poemll    and didn't listen to the voice of the \divine{Lord}. \\
\poeml \v{26}So he swore an oath concerning them--- \\
\poemll    that he would cause them to die in the wilderness, \\
\poeml \v{27}to cause their children to perish among the nations \\
\poemll    and be scattered among many\fnote{The Heb. lacks \fbib{many}} lands. \\
\poeml \v{28}For they adopted the worship\fnote{Lit. \fbib{they attached themselves with Baal Peor}} of Baal Peor \\
\poemll    and ate sacrifices offered to the dead. \\
\poeml \v{29}They had provoked anger by their deeds, \\
\poemll    so that a plague broke out against them. \\
\poeml \v{30}But Phinehas intervened and prayed \\
\poemll    so that the plague was restrained. \\
\poeml \v{31}And it was credited to him as a righteous act, \\
\poemll    from generation to generation---to eternity. \\
\poeml \v{32}They provoked wrath at the waters of Meribah, \\
\poemll    and Moses suffered\fnote{Lit. \fbib{and it was evil for Moses}} on account of them. \\
\poeml \v{33}For they rebelled against him,\fnote{Lit. \fbib{against his spirit}} \\
\poemll    so that he spoke thoughtlessly with his lips. \\
\poeml \v{34}They never destroyed the people, \\
\poemll    as the \divine{Lord} had commanded them. \\
\poeml \v{35}Instead, they mingled among the nations \\
\poemll    and learned their ways.\fnote{Lit. \fbib{deeds}} \\
\poeml \v{36}They worshipped\fnote{Lit. \fbib{served}} their idols, \\
\poemll    and this became a trap for them. \\
\poeml \v{37}They sacrificed their sons and daughters to demons. \\
\poeml \v{38}They shed innocent blood--- \\
\poemll    the blood of their sons and daughters--- \\
\poeml whom they sacrificed to the idols of Canaan, \\
\poemll    thereby polluting the land with blood. \\
\poeml \v{39}Therefore, they became unclean because of what they did; \\
\poemll    they have acted like whores by their evil deeds. \\
\poeml \v{40}The \divine{Lord}'s anger burned against his people, \\
\poemll    so that he despised his own inheritance. \\
\poeml \v{41}He turned them over to domination by nations \\
\poemll    where those who hated them ruled over them. \\
\poeml \v{42}Their enemies oppressed them, \\
\poemll    so that they were humiliated by their power. \\
\poeml \v{43}He delivered them many times, \\
\poemll    but they demonstrated rebellion by their evil plans; \\
\poemlll       therefore they sunk deep in their sins. \\
\poeml \v{44}Yet when he saw their distress \\
\poemll    and heard their cries for help,\fnote{The Heb. lacks \fbib{help}} \\
\poeml \v{45}he remembered his covenant with them, \\
\poemll    and so relented \\
\poemlll       according to the greatness of his gracious love. \\
\poeml \v{46}He caused all their captors to show compassion toward them. \\
\poeml \v{47}Deliver us, \divine{Lord} our God, \\
\poemll    gather us from among the nations \\
\poeml so we may praise your holy name \\
\poemll    and rejoice in praising you. \\
\poeml \v{48}Blessed are you, \divine{Lord} God of Israel, \\
\poemll    from eternity to eternity; \\
\poeml Let all the people say, ``Amen!'' \\
\poemll    Hallelujah!
\end{poetry}
\booksection{\divine{BOOK V} (Psalms 107-150)}
\labelpsalm{107}
\passage{Gratitude for God's Deliverance}

\begin{poetry}
\poeml \v{1}Give thanks to the \divine{Lord}, for he is good! \\
\poemll    His gracious love exists forever. \\
\poeml \v{2}Let those who have been redeemed by the \divine{Lord} declare it--- \\
\poemll    those whom he redeemed \\
\poemlll       from the power\fnote{Lit. \fbib{hand}} of the enemy, \\
\poeml \v{3}those whom he gathered from other lands--- \\
\poemll    from the east, west, north, and south.\fnote{Lit. \fbib{and the sea}; i.e. the Reed Sea} \\
\poeml \v{4}They wandered in desolate wilderness; \\
\poemll    they found no road to a city where they could live. \\
\poeml \v{5}Hungry and thirsty, \\
\poemll    their spirits\fnote{Lit. \fbib{soul}} failed. \\
\poeml \v{6}Then they cried out to the \divine{Lord} in their trouble, \\
\poemll    and he delivered them from their distress. \\
\poeml \v{7}He led them in a straight way \\
\poemll    to find a city where they could live. \\
\poeml \v{8}Let them give thanks to the \divine{Lord} \\
\poemll    for his gracious love \\
\poemlll       and his awesome deeds for mankind. \\
\poeml \v{9}He has satisfied the one who thirsts, \\
\poemll    filling the hungry with what is good. \\
\poeml \v{10}Some sat in deepest darkness, \\
\poemll    shackled with cruel iron, \\
\poeml \v{11}because they had rebelled against the command of God, \\
\poemll    despising the advice of the Most High. \\
\poeml \v{12}He humbled them\fnote{Lit. \fbib{humbled their hearts}} through suffering, \\
\poemll    as they stumbled without a helper. \\
\poeml \v{13}Then they cried out to the \divine{Lord} in their trouble; \\
\poemll    he delivered them from their distress. \\
\poeml \v{14}And he\fnote{So LXX DSS 4QPs\textsuperscript{f}; MT reads \fbib{He}} brought them out from darkness and the shadow of death,\fnote{So LXX; MT reads \fbib{and gloom}} \\
\poemll    shattering their chains. \\
\poeml \v{15}Let them give\fnote{So MT LXX; DSS 4QPs\textsuperscript{f} reads \fbib{Give}} thanks to the \divine{Lord} for his gracious love, \\
\poemll    and for his awesome deeds to mankind. \\
\poeml \v{16}For he shattered bronze gates \\
\poemll    and cut through iron bars. \\
\poeml \v{17}Because of their rebellious ways, \\
\poemll    fools suffered for their iniquities. \\
\poeml \v{18}They\fnote{Lit. \fbib{their souls}} loathed all food, \\
\poemll    and even reached the gates of death. \\
\poeml \v{19}Yet when they cried out to the \divine{Lord} in their trouble, \\
\poemll    he delivered them from certain destruction. \\
\poeml \v{20}He issued his command\fnote{Lit. \fbib{word}} and healed them; \\
\poemll    he delivered them from their destruction. \\
\poeml \v{21}Let them give thanks to the \divine{Lord} for his gracious love, \\
\poemll    and for his awesome deeds for mankind. \\
\poeml \v{22}Let them offer sacrifices of thanksgiving \\
\poemll    and talk about his works with shouts of joy. \\
\poeml \v{23}Those who go down to the sea in ships, \\
\poemll    who work in the great waters, \\
\poeml \v{24}witnessed the works of the \divine{Lord}--- \\
\poemll    his awesome deeds in the ocean's depth. \\
\poeml \v{25}He spoke and stirred up a windstorm \\
\poemll    that made its waves surge. \\
\poeml \v{26}The people\fnote{Lit. \fbib{They}} ascended skyward and descended to the depths, \\
\poemll    their courage\fnote{Lit. \fbib{souls}} melting away in their peril. \\
\poeml \v{27}They reeled and staggered like a drunkard, \\
\poemll    as all their wisdom became useless. \\
\poeml \v{28}Yet when they cried out to the \divine{Lord} in their trouble, \\
\poemll    the \divine{Lord} brought them out of their distress. \\
\poeml \v{29}He calmed the storm \\
\poemll    and its waves\fnote{So MT LXX; DSS 4QPs\textsuperscript{f} reads \fbib{and the waves of the sea}; cf. Psa 107:25} quieted down. \\
\poeml \v{30}So they rejoiced that the waves\fnote{Lit. \fbib{they}} became quiet, \\
\poemll    and he led them to their desired haven. \\
\poeml \v{31}Let them give thanks to the \divine{Lord} for his gracious love \\
\poemll    and for his awesome deeds on behalf of mankind. \\
\poeml \v{32}Let them exalt him in the assembly of the people \\
\poemll    and praise him in the counsel of the elders. \\
\poeml \v{33}He turns rivers into a desert, \\
\poemll    springs of water into dry ground, \\
\poeml \v{34}and a fruitful land into a salty waste, \\
\poemll    due to the wickedness of its inhabitants. \\
\poeml \v{35}He turns a desert into a pool of water, \\
\poemll    dry land into springs of water. \\
\poeml \v{36}There he settled the hungry, \\
\poemll    where they built a city to live in. \\
\poeml \v{37}They sowed fields and planted vineyards \\
\poemll    that yielded a productive harvest. \\
\poeml \v{38}Then he blessed them, and they became numerous; \\
\poemll    he multiplied their cattle.\fnote{Or \fbib{he didn't let their cattle become few}} \\
\poeml \v{39}But they became few in number, and humiliated \\
\poemll    by continued oppression, agony, and sorrow. \\
\poeml \v{40}Having poured contempt on their nobles, \\
\poemll    causing them to err aimlessly in the way. \\
\poeml \v{41}Yet he lifted the needy from affliction \\
\poemll    and made them families like a flock. \\
\poeml \v{42}The upright see it and rejoice, \\
\poemll    but the mouth of an evil person is shut. \\
\poeml \v{43}Let whoever is wise observe these things, \\
\poemll    that they may comprehend the gracious love of the \divine{Lord}.
\end{poetry}
\labelpsalm{108}
\psalminfo{A song. A Davidic psalm.}
\passage{A Plea for Victory}

\begin{poetry}
\poeml \v{1}My heart is firm, God; \\
\poemll    I will sing and praise you with my whole being. \\
\poeml \v{2}Awake, harp and lyre! \\
\poemll    I will wake up at dawn. \\
\poeml \v{3}I will give thanks to you among the peoples, \divine{Lord}! \\
\poemll    I will sing praise to you among the nations. \\
\poeml \v{4}For your gracious love extends to the sky,\fnote{Or is \fbib{great above the heavens}} \\
\poemll    and your faithfulness reaches to the clouds. \\
\poeml \v{5}May you be exalted above the heavens, God, \\
\poemll    and your glory be over all the earth. \\
\poeml \v{6}In order that those you love may be rescued, \\
\poemll    deliver with your power\fnote{Lit. \fbib{right hand}} and answer me! \\
\poeml \v{7}God had promised in his sanctuary: \\
\poeml ``I will triumph and divide Shechem, \\
\poemll    then I will measure the valley of Succoth! \\
\poeml \v{8}Gilead and Manasseh belong to me, \\
\poemll    while Ephraim is my chief stronghold \\
\poemlll       and Judah is my scepter. \\
\poeml \v{9}Moab is my washbasin; \\
\poemll    I will fling my shoe on Edom \\
\poemlll       and shout over Philistia.'' \\
\poeml \v{10}Who will lead me to the fortified city? \\
\poemll    Who will lead me as far as Edom? \\
\poeml \v{11}God, you have rejected us, have you not, \\
\poemll    since you did not march out with our army, God? \\
\poeml \v{12}Give us help against the enemy, \\
\poemll    because human help is useless.\fnote{Or \fbib{vain}} \\
\poeml \v{13}I will find strength in God, \\
\poemll    for he will trample on our foes.
\end{poetry}
\labelpsalm{109}
\psalminfo{To the Director. A Davidic psalm.}
\passage{A Prayer against the Evil One}

\begin{poetry}
\poeml \v{1}God, whom I praise, \\
\poemll    do not be silent, \\
\poeml \v{2}for the mouths of wicked and deceitful people \\
\poemll    are opened against me; \\
\poemlll       they speak against me with lying tongues. \\
\poeml \v{3}They surround me with hate-filled words, \\
\poemll    attacking me for no reason. \\
\poeml \v{4}Instead of receiving\fnote{The Heb. lacks \fbib{receiving}} my love, they accuse me, \\
\poemll    though I continue in prayer. \\
\poeml \v{5}They devise evil against me instead of good, \\
\poemll    and hatred in place of my love. \\
\poeml \v{6}Appoint an evil person over him; \\
\poemll    may an accuser stand at his right side.\fnote{Lit. \fbib{hand}} \\
\poeml \v{7}When he is judged, may he be found guilty; \\
\poemll    may his prayer be regarded as sin. \\
\poeml \v{8}May his days be few; \\
\poemll    may another take over his position.\fnote{Or \fbib{office}} \\
\poeml \v{9}May his children become fatherless, \\
\poemll    and his wife a widow. \\
\poeml \v{10}May his children roam around begging, \\
\poemll    seeking food\fnote{The Heb. lacks \fbib{food}} while driven far\fnote{So LXX; the Heb. lacks \fbib{while driven far}} from their ruined homes. \\
\poeml \v{11}May creditors seize all his possessions, \\
\poemll    and may foreigners loot the property he has acquired.\fnote{Or \fbib{the result of his labor}} \\
\poeml \v{12}May no one extend gracious love to him, \\
\poemll    or show favor to his fatherless children. \\
\poeml \v{13}May his descendants\fnote{Lit. \fbib{May those after him}} be eliminated, \\
\poemll    and their memory\fnote{Or \fbib{their name}} be erased from the next generation. \\
\poeml \v{14}May his ancestors' guilt be remembered in the \divine{Lord}'s presence, \\
\poemll    and may his mother's guilt not be erased. \\
\poeml \v{15}May what\fnote{The Heb. lacks \fbib{what}} they have done\fnote{The Heb. lacks \fbib{have done}} be continually in the \divine{Lord}'s presence; \\
\poemll    and may their memory be excised from the earth. \\
\poeml \v{16}For he didn't think to extend gracious love; \\
\poemll    he harassed to death the poor, the needy, and the broken hearted.\fnote{Or \fbib{downhearted}} \\
\poeml \v{17}He loved to curse---may his curses\fnote{Lit. \fbib{may it}} return upon him! \\
\poemll    He took no delight in blessing others\fnote{The Heb. lacks \fbib{others}}--- \\
\poemlll       so may blessings\fnote{Lit. \fbib{it}} be far from him. \\
\poeml \v{18}He wore curses like a garment--- \\
\poemll    may they\fnote{Lit. \fbib{it}} enter his inner being like water \\
\poemlll       and his bones like oil. \\
\poeml \v{19}May those curses\fnote{Lit. \fbib{may it}} wrap around him like a garment, \\
\poemll    or like a belt that one always wears. \\
\poeml \v{20}May this be the way the \divine{Lord} repays my accuser, \\
\poemll    those who speak evil against me. \\
\poeml \v{21}Now you, \divine{Lord} my God, defend\fnote{Lit. \fbib{God, do to}} me for your name's sake; \\
\poemll    because your gracious love is good, deliver me! \\
\poeml \v{22}Indeed, I am poor and needy, \\
\poemll    and my heart is wounded within me. \\
\poeml \v{23}I am fading\fnote{Or \fbib{walking}} away like a shadow late in the day; \\
\poemll    I am shaken off like a locust. \\
\poeml \v{24}My knees give way\fnote{Or \fbib{knees stagger}} from fasting, \\
\poemll    and my skin is lean, deprived of oil. \\
\poeml \v{25}I have become an object of derision to them--- \\
\poemll    they shake their heads whenever they see me. \\
\poeml \v{26}Help me, \divine{Lord} my God! \\
\poemll    Deliver me in accord with your gracious love! \\
\poeml \v{27}Then they will realize that your hand is in this--- \\
\poemll    that you, \divine{Lord}, have accomplished it. \\
\poeml \v{28}They will curse, \\
\poemll    but you will bless. \\
\poeml When they attack,\fnote{Lit. \fbib{arise}} they will\fnote{So MT DSS 4QPs\textsuperscript{f} 11QPs\textsuperscript{a}; LXX reads \fbib{arise, let my opponents}} be humiliated, \\
\poemll    while your servant rejoices. \\
\poeml \v{29}May my accusers be clothed with shame \\
\poemll    and wrapped in their humiliation as with a robe. \\
\poeml \v{30}I will give many thanks to the \divine{Lord} with my mouth, \\
\poemll    praising him publicly, \\
\poeml \v{31}for he stands\fnote{So MT; LXX DSS 11QPs\textsuperscript{a} read \fbib{he has stood}} at the right hand of the needy one, \\
\poemll    to deliver him from his accusers.\fnote{Or \fbib{from those who condemn him}}
\end{poetry}
\labelpsalm{110}
\psalminfo{A Davidic psalm}
\passage{A Priestly Ruler}

\begin{poetry}
\poeml \v{1}A declaration from the \divine{Lord}\fnote{So MT; LXX reads \fbib{The \divine{Lord} said}} to my Lord:
\end{poetry}

\begin{poetry}
\poemll    ``Sit at my right hand \\
\poemlll       until I make your enemies your footstool.'' \\
\poeml \v{2}When the \divine{Lord} extends your mighty scepter from Zion, \\
\poemll    rule in the midst of your enemies. \\
\poeml \v{3}Your soldiers\fnote{Lit. \fbib{people}} are willing volunteers on your day of battle; \\
\poemll    in majestic holiness, from the womb, \\
\poemlll       from the dawn, the dew of your youth belongs to you.
\end{poetry}

\begin{poetry}
\poeml \v{4}The \divine{Lord} took an oath and will never recant: \\
\poemll    ``You are a priest forever, \\
\poemlll       after the manner of Melchizedek.'' \\
\poeml \v{5}The Lord is at your right hand; \\
\poemll    he will utterly destroy kings in the time of his wrath. \\
\poeml \v{6}He will execute judgment against the nations, \\
\poemll    filling graves\fnote{The Heb. lacks \fbib{graves}} with corpses. \\
\poemlll       He will utterly destroy leaders far and wide. \\
\poeml \v{7}He will drink from a stream on the way, \\
\poemll    then hold his head high.
\end{poetry}
\labelpsalm{111}
\passage{Praise for God's Amazing Deeds\fnote{T In this acrostic psalm each line begins with a consecutive letter of the Heb. alphabet.}}

\begin{poetry}
\poeml \v{1}Hallelujah!
\end{poetry}

\begin{poetry}
\poeml I will give thanks to the \divine{Lord} with all of my heart \\
\poemll    in the assembled congregation of the upright. \\
\poeml \v{2}Great are the acts of the \divine{Lord}; \\
\poemll    they are within reach of\fnote{Or \fbib{are sought by}} all who desire them. \\
\poeml \v{3}Splendid and glorious are his awesome deeds, \\
\poemll    and his righteousness endures forever. \\
\poeml \v{4}He is remembered for his awesome deeds; \\
\poemll    the \divine{Lord} is gracious and compassionate. \\
\poeml \v{5}He prepares food\fnote{Or \fbib{prey}} for those who fear him; \\
\poemll    he is ever mindful of his covenant. \\
\poeml \v{6}He revealed his mighty deeds to his people \\
\poemll    by giving them a country of their own.\fnote{Lit. \fbib{an inheritance of nations}} \\
\poeml \v{7}Whatever he does is\fnote{Lit. \fbib{The works of his hands are}} reliable and just, \\
\poemll    and all his precepts are trustworthy, \\
\poeml \v{8}sustained through all eternity, \\
\poemll    and fashioned in both truth and righteousness. \\
\poeml \v{9}He sent deliverance to his people; \\
\poemll    he ordained his covenant to last forever; \\
\poemlll       his name is holy and awesome. \\
\poeml \v{10}The fear of the \divine{Lord} is the beginning of wisdom; \\
\poemll    sound understanding belongs to those who practice it. \\
\poeml Praise of God\fnote{Lit. \fbib{him}} endures forever.
\end{poetry}
\labelpsalm{112}
\passage{The Gracious Person\fnote{T In this acrostic psalm each line begins with a consecutive letter of the Heb. alphabet.}}

\begin{poetry}
\poeml \v{1}Hallelujah!
\end{poetry}

\begin{poetry}
\poeml How happy is the person who fears the \divine{Lord}, \\
\poemll    who truly delights in his commandments. \\
\poeml \v{2}His descendants will be powerful in the land, \\
\poemll    a generation of the upright who will be blessed. \\
\poeml \v{3}Wealth and riches are in his house, \\
\poemll    and his righteousness endures forever. \\
\poeml \v{4}A light shines in the darkness for the upright, \\
\poemll    for the one who is gracious, compassionate, and just. \\
\poeml \v{5}It is good for the person who lends generously, \\
\poemll    conducting his affairs with fairness. \\
\poeml \v{6}He will never be shaken; \\
\poemll    the one who is just will always be remembered. \\
\poeml \v{7}He need not fear a bad report, \\
\poemll    for his heart is unshaken, since he trusts in the \divine{Lord}. \\
\poeml \v{8}His heart is steadfast, he will not fear. \\
\poemll    In the end he will look in triumph over his enemy. \\
\poeml \v{9}He gives generously to the poor; \\
\poemll    his righteousness endures forever; \\
\poemlll       his horn is exalted in honor. \\
\poeml \v{10}The wicked person sees this and flies into a rage; \\
\poemll    his teeth gnash and wear away. \\
\poeml The desire of the wicked will amount to nothing.
\end{poetry}
\labelpsalm{113}
\passage{Praise to the Loving God}

\begin{poetry}
\poeml \v{1}Hallelujah!
\end{poetry}

\begin{poetry}
\poeml Give praise, you servants of the \divine{Lord}. \\
\poemll    Praise the name of the \divine{Lord}! \\
\poeml \v{2}May the name of the \divine{Lord} be blessed \\
\poemll    from now to eternity. \\
\poeml \v{3}From rising\fnote{Lit. \fbib{from the eastern}} to setting\fnote{Lit. \fbib{to the western}} sun, \\
\poemll    may the name of the \divine{Lord} be praised. \\
\poeml \v{4}The \divine{Lord} is exalted high above all the nations; \\
\poemll    his glory beyond the heavens. \\
\poeml \v{5}Who is like the \divine{Lord} our God, \\
\poemll    enthroned on high, \\
\poeml \v{6}yet stooping low to observe \\
\poemll    the sky and the earth? \\
\poeml \v{7}He lifts the poor person from the dust, \\
\poemll    raising the needy from the trash pile \\
\poeml \v{8}and giving him a seat among nobles--- \\
\poemll    with the nobles of his people. \\
\poeml \v{9}He makes the barren woman among her household \\
\poemll    a happy mother of joyful children. \\
\poeml Hallelujah!
\end{poetry}
\labelpsalm{114}
\passage{Deliverance of Israel from Egypt}

\begin{poetry}
\poeml \v{1}When Israel came out of Egypt--- \\
\poemll    the household of Jacob from a people of foreign speech--- \\
\poeml \v{2}Judah became his sanctuary \\
\poemll    and Israel his place of dominion. \\
\poeml \v{3}The sea saw this\fnote{The Heb. lacks \fbib{this}} and fled, \\
\poemll    the Jordan River\fnote{The Heb. lacks \fbib{River}} ran backwards, \\
\poeml \v{4}the mountains skipped like rams, \\
\poemll    and the hills like lambs. \\
\poeml \v{5}What happened to you, sea, that you fled? \\
\poemll    Jordan, that you ran backwards? \\
\poeml \v{6}Mountains, that you skipped like rams? \\
\poemll    and you hills, that you skipped\fnote{The Heb. lacks \fbib{that you skipped}} like lambs? \\
\poeml \v{7}Tremble then, earth, at the presence of the Lord, \\
\poemll    at the presence of the God of Jacob, \\
\poeml \v{8}who turned the rock into a pool of water, \\
\poemll    the flinty rock into flowing springs.
\end{poetry}
\labelpsalm{115}
\passage{The Impotence of Idols}

\begin{poetry}
\poeml \v{1}Not to us, \divine{Lord}, not to us, \\
\poemll    but to your name be given glory \\
\poemll    on account of your gracious love and faithfulness. \\
\poeml \v{2}Why should the nations ask \\
\poemll    ``Where now is their God?'' \\
\poeml \v{3}when our God is in the heavens \\
\poemll    and he does whatever he desires? \\
\poeml \v{4}Their idols are silver and gold, \\
\poemll    crafted by human hands. \\
\poeml \v{5}They have mouths, but cannot speak; \\
\poemll    they have eyes, but cannot see. \\
\poeml \v{6}They have ears, but cannot hear; \\
\poemll    they have noses, but cannot smell. \\
\poeml \v{7}They have hands, but cannot touch; \\
\poemll    feet, but cannot walk; \\
\poemlll       they cannot even groan with their throats. \\
\poeml \v{8}Those who craft them will become like them, \\
\poemll    as will all those who trust in them. \\
\poeml \v{9}Israel, trust in the \divine{Lord}! \\
\poemll    He is their helper and shield. \\
\poeml \v{10}House of Aaron, trust in the \divine{Lord}! \\
\poemll    He is their helper and shield. \\
\poeml \v{11}You who fear the \divine{Lord}, trust in the \divine{Lord}! \\
\poemll    He is their helper and shield. \\
\poeml \v{12}The \divine{Lord} remembers and blesses us. \\
\poemll    He will indeed bless the house of Israel; \\
\poemlll       he will bless the house of Aaron. \\
\poeml \v{13}He will bless those who fear the \divine{Lord}, \\
\poemll    both the important and the insignificant together. \\
\poeml \v{14}May the \divine{Lord} add to your numbers--- \\
\poemll    to you and to your descendants. \\
\poeml \v{15}May you be blessed by the \divine{Lord}, \\
\poemll    who made the heavens and the earth. \\
\poeml \v{16}The highest heavens\fnote{Lit. \fbib{heaven of heaven}} belong to the \divine{Lord}, \\
\poemll    but he gave the earth to human beings. \\
\poeml \v{17}Neither can the dead praise the \divine{Lord}, \\
\poemll    nor those who go down into the silence of death.\fnote{The Heb. lacks \fbib{of death}} \\
\poeml \v{18}But we will bless the \divine{Lord} \\
\poemll    from now to eternity. \\
\poeml Hallelujah!
\end{poetry}
\labelpsalm{116}
\passage{God, My Deliverer}

\begin{poetry}
\poeml \v{1}I love the \divine{Lord} \\
\poemll    because he has heard my prayer for mercy;\fnote{Lit. \fbib{the voice of my supplication}} \\
\poeml \v{2}for he listens to me whenever I call. \\
\poeml \v{3}The ropes of death were wound around me \\
\poemll    and the anguish of Sheol\fnote{I.e. the realm of the dead} came upon me; \\
\poemlll       I encountered distress and sorrow. \\
\poeml \v{4}Then I called on the name of the \divine{Lord}, \\
\poemll    ``\divine{Lord}, please deliver me!''\fnote{Lit. \fbib{deliver my soul}} \\
\poeml \v{5}The \divine{Lord} is gracious and righteous; \\
\poemll    our God is compassionate; \\
\poeml \v{6}the \divine{Lord} watches over the innocent;\fnote{Or \fbib{naive}} \\
\poemll    I was brought low, and he delivered me. \\
\poeml \v{7}Return to your resting place, my soul, \\
\poemll    for the \divine{Lord} treated you generously. \\
\poeml \v{8}Indeed, you delivered my soul from death, \\
\poemll    my eyes from crying,\fnote{Lit. \fbib{tears}} \\
\poemlll       and my feet from stumbling. \\
\poeml \v{9}I will walk in the \divine{Lord}'s presence \\
\poemll    in the lands of the living. \\
\poeml \v{10}I will continue to believe, even when I say, \\
\poemll    ``I am greatly afflicted'' \\
\poeml \v{11}and speak hastily, \\
\poemll    ``All people are liars!'' \\
\poeml \v{12}What will I return to the \divine{Lord} \\
\poemll    for all his benefits to me? \\
\poeml \v{13}I will raise my cup of deliverance \\
\poemll    and invoke the \divine{Lord}'s name. \\
\poeml \v{14}I will fulfill my vows to the \divine{Lord} \\
\poemll    in the presence of all his people. \\
\poeml \v{15}In the sight of the \divine{Lord}, \\
\poemll    the death of his faithful ones is valued. \\
\poeml \v{16}\divine{Lord}, I am indeed your servant. \\
\poemll    I am your servant. \\
\poeml I am the son of your handmaid. \\
\poemll    You have released my bonds. \\
\poeml \v{17}I will bring you a thanksgiving offering \\
\poemll    and call on the name of the \divine{Lord}! \\
\poeml \v{18}I will fulfill my vows to the \divine{Lord} \\
\poemll    in the presence of all his people, \\
\poeml \v{19}in the courts of the \divine{Lord}'s house, \\
\poemll    in your midst, Jerusalem. \\
\poeml Hallelujah!
\end{poetry}
\labelpsalm{117}
\passage{A Call to Praise the \divine{Lord}}

\begin{poetry}
\poeml \v{1}Praise the \divine{Lord}, all you nations! \\
\poemll    Exalt him, all you peoples! \\
\poeml \v{2}For great is his gracious love toward us, \\
\poemll    and the \divine{Lord}'s faithfulness is eternal. \\
\poeml Hallelujah!
\end{poetry}
\labelpsalm{118}
\passage{Thanksgiving to God}

\begin{poetry}
\poeml \v{1}Give thanks to the \divine{Lord}, \\
\poemll    for he is good; \\
\poemlll       his gracious love is eternal. \\
\poeml \v{2}Let Israel now say, \\
\poemll    ``His gracious love is eternal.'' \\
\poeml \v{3}Let the house of Aaron now say, \\
\poemll    ``His gracious love is eternal.'' \\
\poeml \v{4}Let those who fear the \divine{Lord} now say, \\
\poemll    ``His gracious love is eternal.'' \\
\poeml \v{5}I called on the \divine{Lord} in my distress; \\
\poemll    the \divine{Lord} answered me openly.\fnote{Lit. \fbib{in a wide open place}} \\
\poeml \v{6}The \divine{Lord} is with me. \\
\poemll    I will not be afraid. \\
\poemlll       What can people do to me? \\
\poeml \v{7}With the \divine{Lord} beside me as my helper, \\
\poemll    I will triumph over those who hate me. \\
\poeml \v{8}It is better to take shelter\fnote{Or \fbib{refuge}; LXX DSS 4QPs\textsuperscript{b} 11QPs\textsuperscript{a} read \fbib{to trust}} in the \divine{Lord} \\
\poemll    than to trust in people. \\
\poeml \v{9}It is better to take shelter\fnote{Or \fbib{refuge}} in the \divine{Lord} \\
\poemll    than to trust in princes. \\
\poeml \v{10}All the nations surrounded me; \\
\poemll    but in the name of the \divine{Lord} I will defeat them. \\
\poeml \v{11}They surrounded me, they are around me; \\
\poemll    but in the name of the \divine{Lord} I will defeat them. \\
\poeml \v{12}They surrounded me like bees; \\
\poemll    but they will be extinguished like\fnote{So MT DSS 4QPs\textsuperscript{b}; LXX reads \fbib{bees; they blazed like a fire among}} burning thorns. \\
\poemlll       In the name of the \divine{Lord} I will defeat them. \\
\poeml \v{13}Indeed, you\fnote{I.e. the enemy} oppressed me so much that I nearly fell, \\
\poemll    but the \divine{Lord} helped me. \\
\poeml \v{14}The \divine{Lord} is my strength and protector,\fnote{Or \fbib{might}} \\
\poemll    for he has become my deliverer.\fnote{Or \fbib{salvation}} \\
\poeml \v{15}There's exultation\fnote{Lit. \fbib{sound of}} for deliverance in the tents of the righteous:
\end{poetry}

\begin{poetry}
\poeml ``The right hand of the \divine{Lord} is victorious!\fnote{Lit. \fbib{\divine{Lord} acted valiantly}} \\
\poeml \v{16}The right hand of the \divine{Lord} is exalted! \\
\poemlll       The right hand of the \divine{Lord} is victorious!''\fnote{MT reads \fbib{\divine{Lord} acted valiantly}; LXX DSS 11QPs\textsuperscript{a} read \fbib{\divine{Lord} acted powerfully}} \\
\poeml \v{17}I will not die, but I will live \\
\poemll    to recount the deeds of the \divine{Lord}. \\
\poeml \v{18}The \divine{Lord} will discipline me severely, \\
\poemll    but he won't hand me over to die. \\
\poeml \v{19}Open for me the righteous gates \\
\poemll    so I may enter through them to give thanks to the \divine{Lord}. \\
\poeml \v{20}This is the \divine{Lord}'s gate--- \\
\poemll    The righteous will enter through it. \\
\poeml \v{21}I will praise you because you have answered me \\
\poemll    and have become my deliverer. \\
\poeml \v{22}The stone that the builders rejected \\
\poemll    has become the cornerstone. \\
\poeml \v{23}This is from the \divine{Lord}--- \\
\poemll    it is awesome in our sight. \\
\poeml \v{24}This is the day that the \divine{Lord} has made; \\
\poemll    let's rejoice and be glad in it. \\
\poeml \v{25}Please \divine{Lord}, deliver us! \\
\poemll    Please \divine{Lord}, hurry\fnote{Or \fbib{rush}} and bring success now! \\
\poeml \v{26}Blessed is the one who comes in the name of the \divine{Lord}! \\
\poemll    Let us bless you from the \divine{Lord}'s house. \\
\poeml \v{27}The \divine{Lord} is God---he will be our light! \\
\poemll    Bind the festival sacrifice with ropes \\
\poemlll       to the horn at the altar. \\
\poeml \v{28}You are my God, and I will praise you; \\
\poemll    my God, and I will exalt you. \\
\poeml \v{29}Give thanks to the \divine{Lord}, for he is good \\
\poemll    and his gracious love is eternal.
\end{poetry}
\labelpsalm{119}
\psalminfo{Alef\fnote{T This Psalm is an acrostic in which all verses in each eight-verse section begin with the letter of the Heb. alphabet indicated.}}
\passage{Living in the Law of God}

\begin{poetry}
\poeml \v{1}How blessed are those whose life\fnote{Lit. \fbib{way}} is blameless, \\
\poemll    who walk in the Law of the \divine{Lord}! \\
\poeml \v{2}How blessed are those who observe his decrees, \\
\poemll    who seek him with all of their heart, \\
\poeml \v{3}who practice no evil \\
\poemll    while they walk in his ways. \\
\poeml \v{4}You have commanded concerning your precepts, \\
\poemll    that they be guarded with diligence. \\
\poeml \v{5}Oh, that my ways were steadfast, \\
\poemll    so I may keep your statutes. \\
\poeml \v{6}Then I will not be ashamed, \\
\poemll    since my eyes will be fixed on all of your commands. \\
\poeml \v{7}I will praise you with an upright heart, \\
\poemll    as I learn your righteous decrees. \\
\poeml \v{8}I will keep your statutes; \\
\poemll    do not ever abandon me.
\end{poetry}
\psalminfo{Bet}
\passage{The Benefits of the Word}

\begin{poetry}
\poeml \v{9}How can a young man keep his behavior pure? \\
\poemll    By guarding it in accordance with your word. \\
\poeml \v{10}I have sought you with all of my heart; \\
\poemll    do not let me drift away from your commands. \\
\poeml \v{11}I have stored what you have said\fnote{So MT DSS 4QPs\textsuperscript{h}; LXX Syr read \fbib{stored your oracles}} in my heart, \\
\poemll    so I won't sin against you. \\
\poeml \v{12}Blessed are you, \divine{Lord}! \\
\poemll    Teach me your statutes. \\
\poeml \v{13}I have spoken with my lips \\
\poemll    about all your decrees that you have announced.\fnote{Lit. \fbib{decrees of your mouth}} \\
\poeml \v{14}I find joy in the path of your decrees, \\
\poemll    as if I owned all kinds of riches. \\
\poeml \v{15}I will meditate on your precepts, \\
\poemll    and I will respect your ways. \\
\poeml \v{16}I am delighted with your statutes; \\
\poemll    I will not forget your word.\fnote{So MT; LXX Syr DSS 11QPs\textsuperscript{a} read \fbib{words}}
\end{poetry}
\psalminfo{Gimmel}
\passage{Living and Keeping God's Word}

\begin{poetry}
\poeml \v{17}Deal kindly with your servant \\
\poemll    so I may live and keep your word.\fnote{So MT; LXX DSS 11QPs\textsuperscript{a} read \fbib{words}} \\
\poeml \v{18}Open my eyes \\
\poemll    so that I will observe amazing things from your instruction.\fnote{Or \fbib{Law}} \\
\poeml \v{19}Since I am a stranger on the earth, \\
\poemll    do not hide your commands from me. \\
\poeml \v{20}My soul is consumed with longing \\
\poemll    for your decrees at all times. \\
\poeml \v{21}You rebuke the accursed ones, \\
\poemll    who wander from your commands. \\
\poeml \v{22}Remove scorn and disrespect from me, \\
\poemll    for I observe your decrees. \\
\poeml \v{23}Though nobles take their seat and gossip about me, \\
\poemll    your servant will meditate on your statutes. \\
\poeml \v{24}I take joy in your decrees, \\
\poemll    for they are my counselors.
\end{poetry}
\psalminfo{Daleth}
\passage{Strength Comes from the Word}

\begin{poetry}
\poeml \v{25}My soul clings to the dust; \\
\poemll    revive me according to your word. \\
\poeml \v{26}I have talked about my ways, \\
\poemll    and you have answered me; \\
\poemlll       Teach me your statutes. \\
\poeml \v{27}Help me understand how your precepts function,\fnote{Lit. \fbib{understand the ways of your precepts}} \\
\poemll    and I will meditate on your wondrous acts. \\
\poeml \v{28}I weep because of sorrow; \\
\poemll    fortify me according to your word. \\
\poeml \v{29}Remove false paths from me; \\
\poemll    and graciously give me your instruction.\fnote{Or \fbib{Law}} \\
\poeml \v{30}I have chosen the faithful way; \\
\poemll    I have firmly placed your ordinances before me.\fnote{The Heb. lacks \fbib{before me}} \\
\poeml \v{31}I cling to your decrees; \\
\poemll    \divine{Lord}, do not put me to shame. \\
\poeml \v{32}I eagerly race along the way of your commands, \\
\poemll    for you enable me to do so.\fnote{Lit. \fbib{will enlarge my heart}}
\end{poetry}
\psalminfo{He}
\passage{Instructed by the Word}

\begin{poetry}
\poeml \v{33}Teach me, \divine{Lord}, about the way of your statutes, \\
\poemll    and I will observe them without fail.\fnote{Or \fbib{them to the end}} \\
\poeml \v{34}Give me understanding \\
\poemll    and I will observe your instruction.\fnote{Or \fbib{Law}} \\
\poemlll       I will keep it with all of my heart. \\
\poeml \v{35}Help me live my life by your commands, \\
\poemll    because my joy is in them. \\
\poeml \v{36}Turn my heart to your decrees \\
\poemll    and away from unjust gain. \\
\poeml \v{37}Turn my eyes away from gazing at worthless things, \\
\poemll    and revive me by your ways. \\
\poeml \v{38}Confirm your promise to your servant, \\
\poemll    which is for those who fear you. \\
\poeml \v{39}Turn away the shame that I dread, \\
\poemll    because your ordinances are good. \\
\poeml \v{40}Look, I long for your precepts; \\
\poemll    revive me through your righteousness.
\end{poetry}
\psalminfo{Vav}
\passage{A Song of Praise}

\begin{poetry}
\poeml \v{41}May your gracious love come to me, \divine{Lord}, \\
\poemll    your salvation, just as you said. \\
\poeml \v{42}Then I can answer the one who insults me, \\
\poemll    for I place my trust in your word. \\
\poeml \v{43}Never take your truthful words from me, \\
\poemll    For I wait for\fnote{Or \fbib{place my hope in}} your ordinances. \\
\poeml \v{44}Then I will always keep your Law, \\
\poemll    forever and ever, \\
\poeml \v{45}I will walk in liberty, \\
\poemll    for I seek your precepts. \\
\poeml \v{46}Then I will speak of your decrees before kings \\
\poemll    and not be ashamed. \\
\poeml \v{47}I will take delight in your commands, \\
\poemll    which I love. \\
\poeml \v{48}I will lift up my hands to your commands, \\
\poemll    which I love, \\
\poemlll       and I will meditate on your statutes.
\end{poetry}
\psalminfo{Zayin}
\passage{Remembering What God Has Said}

\begin{poetry}
\poeml \v{49}Remember what you said\fnote{Lit. \fbib{Remember the word}} to your servant, \\
\poemll    by which you caused me to hope. \\
\poeml \v{50}This is what comforts me in my troubles: \\
\poemll    that what you say revives me. \\
\poeml \v{51}Even though the arrogant utterly deride me, \\
\poemll    I do not turn away from your instruction.\fnote{Or \fbib{Law}} \\
\poeml \v{52}I have remembered your ancient ordinances, \divine{Lord}, \\
\poemll    and I take comfort in them. \\
\poeml \v{53}I burn with indignation because of the wicked \\
\poemll    who forsake your instruction.\fnote{Or \fbib{Law}} \\
\poeml \v{54}Your statutes are my songs, \\
\poemll    no matter where I make my home.\fnote{Lit. \fbib{songs in the house of my sojourn}} \\
\poeml \v{55}In the night I remember your name, \divine{Lord}, \\
\poemll    and keep your instruction.\fnote{Or \fbib{Law}} \\
\poeml \v{56}I have made it my personal responsibility \\
\poemll    to keep your precepts.
\end{poetry}
\psalminfo{Cheth}
\passage{Keeping God's Word}

\begin{poetry}
\poeml \v{57}The \divine{Lord} is my inheritance; \\
\poemll    I have given my promise to keep your word. \\
\poeml \v{58}I have sought your favor with all of my heart; \\
\poemll    be gracious to me according to your promise. \\
\poeml \v{59}I examined my lifestyle \\
\poemll    and set my feet in the direction of your decrees. \\
\poeml \v{60}I hurried and did not procrastinate \\
\poemll    to keep your commands. \\
\poeml \v{61}Though the ropes of the wicked have ensnared me, \\
\poemll    I have not forgotten your instruction.\fnote{Or \fbib{Law}} \\
\poeml \v{62}At midnight I will get up to thank you \\
\poemll    for your righteous ordinances. \\
\poeml \v{63}I am united with all who fear you, \\
\poemll    and with everyone who keeps your precepts. \\
\poeml \v{64}\divine{Lord}, the earth overflows with your gracious love! \\
\poemll    Teach me your statutes.
\end{poetry}
\psalminfo{Teth}
\passage{Praise for God's Word}

\begin{poetry}
\poeml \v{65}\divine{Lord}, you have dealt well with your servant, \\
\poemll    according to your word. \\
\poeml \v{66}Teach me both knowledge and appropriate discretion, \\
\poemll    because I believe in your commands. \\
\poeml \v{67}Before I was humbled, I wandered away, \\
\poemll    but now I observe your words. \\
\poeml \v{68}\divine{Lord},\fnote{So LXX Syr DSS 11QPs\textsuperscript{a}; the Heb. lacks \fbib{\divine{Lord}}} you are good\fnote{So MT; LXX reads \fbib{kind}}, and do what is good; \\
\poemll    teach me your statutes. \\
\poeml \v{69}The arrogant have accused me falsely; \\
\poemll    but I will observe your precepts wholeheartedly. \\
\poeml \v{70}Their minds are clogged as with greasy fat, \\
\poemll    but I find joy in your instruction.\fnote{Or \fbib{Law}} \\
\poeml \v{71}It was for my good that I was humbled;\fnote{So MT; LXX reads \fbib{that you humbled me}; DSS 11QPs\textsuperscript{a} reads \fbib{that you afflicted me}} \\
\poemll    so that I would learn your statutes. \\
\poeml \v{72}Instruction\fnote{Or \fbib{Law}} that comes from you\fnote{Lit. \fbib{from your mouth}} is better for me \\
\poemll    than thousands of gold and silver coins.\fnote{Lit. \fbib{pieces}}
\end{poetry}
\psalminfo{Yod}
\passage{Prayer for God's Grace}

\begin{poetry}
\poeml \v{73}Your hands made and formed me; \\
\poemll    give me understanding, \\
\poemlll       that I may learn your commands. \\
\poeml \v{74}May those who fear you see me and be glad, \\
\poemll    for I have hoped in your word. \\
\poeml \v{75}I know, \divine{Lord}, that your decrees are just, \\
\poemll    and that you have rightfully humbled me. \\
\poeml \v{76}May your gracious love comfort me \\
\poemll    in accordance with your promise to your servant. \\
\poeml \v{77}May your mercies come to me that I may live, \\
\poemll    for your instruction\fnote{Or \fbib{Law}} is my delight. \\
\poeml \v{78}May the arrogant become ashamed, \\
\poemll    because they have subverted me with deceit; \\
\poemlll       as for me, I will meditate on your precepts. \\
\poeml \v{79}May those who fear you turn to me, \\
\poemll    along with those who know your decrees. \\
\poeml \v{80}May my heart be blameless with respect to your statutes \\
\poemll    so that I may not become ashamed.
\end{poetry}
\psalminfo{Kaf}
\passage{On Obeying God's Word}

\begin{poetry}
\poeml \v{81}I long for your deliverance; \\
\poemll    I have looked to your word, \\
\poemlll       placing my hope in it. \\
\poeml \v{82}My eyes grow weary \\
\poemll    with respect to what you have promised--- \\
\poemlll       I keep asking, ``When will you comfort me?'' \\
\poeml \v{83}Though I have become like a water skin dried by\fnote{The Heb. lacks \fbib{dried by}} smoke, \\
\poemll    I have not forgotten your statutes. \\
\poeml \v{84}How many days must your servant endure this?\fnote{The Heb. lacks \fbib{this}} \\
\poemll    When will you judge those who persecute me? \\
\poeml \v{85}The arrogant have dug pitfalls for me \\
\poemll    disobeying your instruction.\fnote{Or \fbib{Law}} \\
\poeml \v{86}All of your commands are reliable. \\
\poemll    I am persecuted without cause---help me! \\
\poeml \v{87}Though the arrogant\fnote{Lit. \fbib{they}} nearly destroyed me on earth, \\
\poemll    I did not abandon your precepts. \\
\poeml \v{88}Revive me according to your gracious love; \\
\poemll    and I will keep the decrees that you have proclaimed.
\end{poetry}
\psalminfo{Lamed}
\passage{Pay Attention to God's Word}

\begin{poetry}
\poeml \v{89}Your word is forever, \divine{Lord}; \\
\poemll    it is firmly established in heaven. \\
\poeml \v{90}Your faithfulness continues from generation to generation. \\
\poemll    You established the earth, and it stands firm. \\
\poeml \v{91}To this day they stand by means of your rulings, \\
\poemll    for all things serve you. \\
\poeml \v{92}Had your instruction\fnote{Or \fbib{Law}} not been my pleasure, \\
\poemll    I would have died in my affliction. \\
\poeml \v{93}I will never forget your precepts, \\
\poemll    for you have revived me with them. \\
\poeml \v{94}I am yours, so save me, \\
\poemll    since I have sought your precepts. \\
\poeml \v{95}The wicked lay in wait to destroy me, \\
\poemll    while I ponder your decrees. \\
\poeml \v{96}I have observed that all things have their limit, \\
\poemll    but your commandment is very broad.
\end{poetry}
\psalminfo{Mem}
\passage{Loving God's Word}

\begin{poetry}
\poeml \v{97}How I love your instruction!\fnote{Or \fbib{Law}} \\
\poemll    Every day it is my meditation. \\
\poeml \v{98}Your commands make me wiser than my adversaries, \\
\poemll    since they are always with me. \\
\poeml \v{99}I am more insightful than my teachers, \\
\poemll    because your decrees are my meditations. \\
\poeml \v{100}I have more common sense than the elders, \\
\poemll    for I observe your precepts. \\
\poeml \v{101}I keep away from every evil choice\fnote{Lit. \fbib{way}} \\
\poemll    so that I may keep your word.\fnote{So MT DSS 5QPs; LXX reads \fbib{words}} \\
\poeml \v{102}I do not avoid your judgments, \\
\poemll    for you pointed them out to me. \\
\poeml \v{103}How pleasing is what you have to say to me--- \\
\poemll    tasting better than honey. \\
\poeml \v{104}I obtain understanding from your precepts; \\
\poemll    therefore I hate every false way.
\end{poetry}
\psalminfo{Nun}
\passage{God's Word a Light}

\begin{poetry}
\poeml \v{105}Your word is\fnote{So MT and LXX; the DSS 11QPs\textsuperscript{a} reads \fbib{words are}} a lamp for my feet, \\
\poemll    a light for my pathway. \\
\poeml \v{106}I have given my word and affirmed it, \\
\poemll    to keep your righteous judgments. \\
\poeml \v{107}I am severely afflicted. \\
\poemll    Revive me, \divine{Lord}, according to your word. \\
\poeml \v{108}\divine{Lord}, please accept my voluntary offerings of praise,\fnote{Lit. \fbib{of my mouth}} \\
\poemll    and teach me your judgments. \\
\poeml \v{109}Though I constantly take my life in my hands, \\
\poemll    I do not forget your instruction.\fnote{Or \fbib{Law}} \\
\poeml \v{110}Though the wicked lay a trap for me, \\
\poemll    I haven't wandered away from your precepts. \\
\poeml \v{111}I have inherited your decrees forever, \\
\poemll    because they are the joy of my heart. \\
\poeml \v{112}As a result, I am determined \\
\poemll    to carry out your statutes forever.
\end{poetry}
\psalminfo{Samek}
\passage{Loving God's Law}

\begin{poetry}
\poeml \v{113}I despise the double-minded, \\
\poemll    but I love your instruction.\fnote{Or \fbib{Law}} \\
\poeml \v{114}You are my fortress and shield; \\
\poemll    I hope in your word. \\
\poeml \v{115}Leave me, you who practice evil, \\
\poemll    that I may observe the commands of my God. \\
\poeml \v{116}Sustain me, God,\fnote{The Heb. lacks \fbib{God}} as you have promised, \\
\poemll    and I will live. \\
\poemlll       Do not let me be ashamed of my hope. \\
\poeml \v{117}Support me, that I may be saved, \\
\poemll    and I will carry out your statutes consistently. \\
\poeml \v{118}You reject all who wander from your statutes, \\
\poemll    since their deceitfulness is vain. \\
\poeml \v{119}You remove\fnote{So MT; Hieronymus Aquila Symmachus read \fbib{You consider}; LXX reads \fbib{I considered}; DSS 11QPs\textsuperscript{a} reads \fbib{I consider}} all the wicked of the earth like\fnote{The Heb. lacks \fbib{like}} dross; \\
\poemll    therefore I love your decrees. \\
\poeml \v{120}My flesh trembles out of fear of you, \\
\poemll    and I am in awe of\fnote{Or \fbib{I fear}} your judgments.
\end{poetry}
\psalminfo{Ayin}
\passage{Praying for God's Deliverance}

\begin{poetry}
\poeml \v{121}I have acted with justice and righteousness; \\
\poemll    do not abandon me to my oppressors. \\
\poeml \v{122}Back up your servant in a positive way; \\
\poemll    do not let the arrogant oppress me. \\
\poeml \v{123}My eyes fail as I look\fnote{The Heb. lacks \fbib{as I look}} for your salvation \\
\poemll    and for your righteous promise. \\
\poeml \v{124}Act toward your servant consistent with your gracious love, \\
\poemll    and teach me your statutes. \\
\poeml \v{125}Since I am your servant, give me understanding, \\
\poemll    so I will know your decrees. \\
\poeml \v{126}It is time for the \divine{Lord} to act, \\
\poemll    since they have violated your instruction.\fnote{Or \fbib{Law}} \\
\poeml \v{127}I truly love your commands more than gold, \\
\poemll    including fine gold. \\
\poeml \v{128}I truly consider all of your precepts---all of them---to be just, \\
\poemll    while I despise every false way.
\end{poetry}
\psalminfo{Peyh}
\passage{Living in God's Word}

\begin{poetry}
\poeml \v{129}Your decrees are wonderful--- \\
\poemll    that's why I observe them. \\
\poeml \v{130}The disclosure of your words illuminates, \\
\poemll    providing understanding to the simple. \\
\poeml \v{131}I open my mouth and pant \\
\poemll    as I long for your commands. \\
\poeml \v{132}Turn in my direction and show mercy to me, \\
\poemll    as you have decreed regarding those who love your name. \\
\poeml \v{133}Direct my footsteps by your promise, \\
\poemll    and do not let any kind of iniquity rule over me. \\
\poeml \v{134}Deliver me from human oppression \\
\poemll    and I will keep your precepts. \\
\poeml \v{135}Show favor to\fnote{Lit. \fbib{Make your face shine on}} your servant, \\
\poemll    and teach me your statutes. \\
\poeml \v{136}My eyes shed rivers of tears, \\
\poemll    when others do not obey your instruction.\fnote{Or \fbib{Law}}
\end{poetry}
\psalminfo{Tsade}
\passage{God's Righteous Decrees}

\begin{poetry}
\poeml \v{137}\divine{Lord}, you are righteous, \\
\poemll    and your judgments are right. \\
\poeml \v{138}You have ordered your decrees to us rightly, \\
\poemll    and they are very faithful. \\
\poeml \v{139}My zeal consumes me \\
\poemll    because my enemies forget your words. \\
\poeml \v{140}Your word is very pure, \\
\poemll    and your servant loves it. \\
\poeml \v{141}Though I may be small and despised, \\
\poemll    I do not neglect your precepts. \\
\poeml \v{142}Your righteousness is an eternal righteousness, \\
\poemll    and your instruction\fnote{Or \fbib{Law}} is true. \\
\poeml \v{143}Though trouble and anguish overwhelm me, \\
\poemll    your commands remain my delight. \\
\poeml \v{144}Your righteous decrees are eternal; \\
\poemll    give me understanding, and I will live.
\end{poetry}
\psalminfo{Qof}
\passage{Waiting in Hope}

\begin{poetry}
\poeml \v{145}I have cried out with all of my heart. \\
\poemll    Answer me, \divine{Lord}! \\
\poemlll       I will observe your statutes. \\
\poeml \v{146}I have called out to you, ``Save me, \\
\poemll    so I may keep your decrees.'' \\
\poeml \v{147}I get up before dawn and cry for help; \\
\poemll    I place my hope in your word. \\
\poeml \v{148}I look forward to the night watches, \\
\poemll    when I may meditate on what you have said. \\
\poeml \v{149}Hear my voice according to your gracious love. \\
\poemll    \divine{Lord}, revive me in keeping with your justice. \\
\poeml \v{150}Those who pursue wickedness draw near; \\
\poemll    they remain far from your instruction.\fnote{Or \fbib{Law}} \\
\poeml \v{151}You are near, \divine{Lord}, \\
\poemll    and all of your commands are true. \\
\poeml \v{152}I discovered long ago about your decrees \\
\poemll    that you have confirmed them forever.
\end{poetry}
\psalminfo{Resh}
\passage{God's Word is Truth}

\begin{poetry}
\poeml \v{153}Look on my misery, and rescue me, \\
\poemll    for I do not ignore your instruction.\fnote{Or \fbib{Law}} \\
\poeml \v{154}Defend my case and redeem me; \\
\poemll    revive me according to your promise. \\
\poeml \v{155}Deliverance remains remote from the wicked, \\
\poemll    for they do not seek your statutes. \\
\poeml \v{156}Your mercies are magnificent, \divine{Lord}; \\
\poemll    revive me according to your judgments. \\
\poeml \v{157}Though my persecutors and adversaries are numerous, \\
\poemll    I do not turn aside from your decrees. \\
\poeml \v{158}I watch the treacherous, and despise them, \\
\poemll    because they do not do what you have said. \\
\poeml \v{159}Look how I love your precepts, \divine{Lord}; \\
\poemll    revive me according to your gracious love. \\
\poeml \v{160}The sum\fnote{So MT; LXX reads \fbib{beginning}} of your word\fnote{So MT; LXX Hieronymous DSS 11QPs\textsuperscript{a} read \fbib{words}} is truth, \\
\poemll    and each righteous ordinance of yours is everlasting.
\end{poetry}
\psalminfo{Sin/Shin}
\passage{Loving God's Instruction}

\begin{poetry}
\poeml \v{161}Though nobles persecute me for no reason, \\
\poemll    my heart stands in awe of your words. \\
\poeml \v{162}I find joy at what you have said \\
\poemll    like one who has discovered a great treasure. \\
\poeml \v{163}I despise and hate falsehood, \\
\poemll    but\fnote{So LXX Syr DSS 11QPs\textsuperscript{a}; the Heb. lacks \fbib{but}} I love your instruction.\fnote{Or \fbib{Law}} \\
\poeml \v{164}I praise you seven times a day \\
\poemll    because of your righteous ordinances. \\
\poeml \v{165}Great peace belongs to those who love your instruction,\fnote{Or \fbib{Law}} \\
\poemll    and nothing makes them stumble. \\
\poeml \v{166}I am looking in hope for your deliverance, \divine{Lord}, \\
\poemll    as I carry out your commands. \\
\poeml \v{167}My soul treasures\fnote{Lit. \fbib{guards}} your decrees, \\
\poemll    and I love them deeply. \\
\poeml \v{168}I keep your precepts and your decrees \\
\poemll    because all of my ways are before you.
\end{poetry}
\psalminfo{Tav}
\passage{The Joy of God's Word}

\begin{poetry}
\poeml \v{169}May my cry arise before you, \divine{Lord}; \\
\poemll    give me understanding according to your word. \\
\poeml \v{170}Let my request come before you; \\
\poemll    deliver me, as you have promised. \\
\poeml \v{171}May my lips utter praise, \\
\poemll    for you teach me your statutes. \\
\poeml \v{172}May my tongue sing about your promise, \\
\poemll    for all of your commands are right. \\
\poeml \v{173}May your hand stand ready to assist me, \\
\poemll    for I have chosen your precepts. \\
\poeml \v{174}I am longing for your deliverance, \divine{Lord}, \\
\poemll    and your instruction\fnote{Or \fbib{Law}} is my joy. \\
\poeml \v{175}Let me live, and I will praise you; \\
\poemll    let your ordinances\fnote{So LXX Targ DSS 11QPs\textsuperscript{a} (original); MT reads \fbib{ordinance}} help me. \\
\poeml \v{176}I have wandered away like a lost sheep; \\
\poemll    come find your servant, \\
\poemlll       for I do not forget your commands.
\end{poetry}
\labelpsalm{120}
\psalminfo{A Song of Ascents\fnote{T Or \fbib{Degrees}; and so through Psalm 134}}
\passage{A Prayer for Deliverance}

\begin{poetry}
\poeml \v{1}I cried to the \divine{Lord} in my distress, \\
\poemll    and he responded to me. \\
\poeml \v{2}``\divine{Lord}, deliver me\fnote{Lit. \fbib{my soul}} from lips that lie \\
\poemll    and tongues that deceive.'' \\
\poeml \v{3}What will be given to you, \\
\poemll    and what will be done to you, \\
\poemlll       you treacherous tongue? \\
\poeml \v{4}Like a\fnote{The Heb. lacks \fbib{Like a}} sharp arrow from a warrior, \\
\poemll    along with fiery coals from juniper trees! \\
\poeml \v{5}How terrible for me, \\
\poemll    that I am an alien in Meshech, \\
\poemlll       that I reside among the tents of Kedar! \\
\poeml \v{6}I have resided too long \\
\poemll    with those who hate peace. \\
\poeml \v{7}I am in favor of peace; \\
\poemll    but when I speak, \\
\poemlll       they are in favor of war.
\end{poetry}
\labelpsalm{121}
\psalminfo{A Song of Ascents}
\passage{The Guardian of God's People}

\begin{poetry}
\poeml \v{1}I lift up my eyes toward the mountains--- \\
\poemll    from where will my help come? \\
\poeml \v{2}My help is from the \divine{Lord}, \\
\poemll    maker of heaven and earth. \\
\poeml \v{3}He will never let\fnote{So MT; LXX reads \fbib{Do not let}} your foot slip, \\
\poemll    nor\fnote{So LXX Syr Hieronymous DSS 11QPs\textsuperscript{a}; the Heb. lacks \fbib{nor}} will\fnote{So MT; LXX reads \fbib{nor let}} your guardian become drowsy. \\
\poeml \v{4}Look! The one who is guarding Israel \\
\poemll    never sleeps and does not take naps. \\
\poeml \v{5}The \divine{Lord} is your guardian; \\
\poemll    the \divine{Lord} is your shade at your right side. \\
\poeml \v{6}The sun will not ravage you by day, \\
\poemll    nor the moon by night. \\
\poeml \v{7}The \divine{Lord} will guard you from all evil, \\
\poemll    preserving\fnote{Or \fbib{guarding}} your life. \\
\poeml \v{8}The \divine{Lord} will guard your goings and comings,\fnote{Cf. Deut 28:6} \\
\poemll    from this time on and forever.
\end{poetry}
\labelpsalm{122}
\psalminfo{A Davidic Song of Ascents}
\passage{Up to Jerusalem}

\begin{poetry}
\poeml \v{1}I rejoiced when they kept on asking me, \\
\poemll    ``Let us go to the \divine{Lord}'s Temple.'' \\
\poeml \v{2}Our feet are standing \\
\poemll    inside your gates, Jerusalem. \\
\poeml \v{3}Jerusalem stands built up, \\
\poemll    a city knitted together. \\
\poeml \v{4}To it the tribes ascend--- \\
\poemll    the tribes of the \divine{Lord}--- \\
\poeml as decreed to Israel, \\
\poemll    to give thanks to the name of the \divine{Lord}. \\
\poeml \v{5}For thrones are established there for judgment, \\
\poemll    thrones of the house of David. \\
\poeml \v{6}Pray for peace for Jerusalem: \\
\poemll    ``May those who love you be at peace!\fnote{Or \fbib{you prosper}} \\
\poeml \v{7}May peace be within your ramparts, \\
\poemll    and\fnote{So LXX Syr DSS 11QPs\textsuperscript{a}; the Heb. lacks \fbib{and}} prosperity\fnote{Or \fbib{peacefulness}; LXX reads \fbib{abundance}} within your fortresses.'' \\
\poeml \v{8}For the sake of my relatives and friends \\
\poemll    I will now say, ``May there be peace within you.'' \\
\poeml \v{9}For the sake of the Temple of the \divine{Lord} our God, \\
\poemll    I will seek your welfare.
\end{poetry}
\labelpsalm{123}
\psalminfo{A Song of Ascents}
\passage{A Prayer for Relief}

\begin{poetry}
\poeml \v{1}To you, who sit enthroned in heaven, \\
\poemll    I lift up my eyes. \\
\poeml \v{2}Consider this: as the eyes of a servant focus \\
\poemll    on what his master provides,\fnote{Lit. \fbib{on the hand of his master}} \\
\poeml and as the eyes of a female servant focus\fnote{The Heb. lacks \fbib{focus}} \\
\poemll    on what her mistress provides,\fnote{Lit. \fbib{on the hand of her mistress}} \\
\poeml so our eyes focus on the \divine{Lord} our God, \\
\poemll    until he has mercy on us. \\
\poeml \v{3}Have mercy on us, \divine{Lord}, have mercy, \\
\poemll    for we have had more than enough of contempt. \\
\poeml \v{4}Our lives overflow \\
\poemll    with scorn from those who live at ease, \\
\poemlll       with contempt from those who are proud.
\end{poetry}
\labelpsalm{124}
\psalminfo{A Davidic Song of Ascents}
\passage{God is for Us}

\begin{poetry}
\poeml \v{1}If the \divine{Lord} had not been on our side--- \\
\poemll    let Israel now say--- \\
\poeml \v{2}if the \divine{Lord} had not been on our side, \\
\poemll    when men came against us, \\
\poeml \v{3}then they would have devoured us alive, \\
\poemll    when their anger burned against us. \\
\poeml \v{4}Then the flood waters would have overwhelmed us, \\
\poemll    the torrent would have flooded over us; \\
\poeml \v{5}the swollen waters would have swept us away. \\
\poeml \v{6}Blessed be the \divine{Lord}, \\
\poemll    who did not give us as prey to their teeth. \\
\poeml \v{7}We have escaped like a bird from the hunter's trap. \\
\poemll    The trap has been broken, \\
\poemlll       and we have escaped. \\
\poeml \v{8}Our help is in the name of the \divine{Lord}, \\
\poemll    the maker of heaven and earth.
\end{poetry}
\labelpsalm{125}
\psalminfo{A Song of Ascents}
\passage{God is Secure}

\begin{poetry}
\poeml \v{1}Those who are trusting in the \divine{Lord} \\
\poemll    are like Mount Zion, which cannot be overthrown. \\
\poemlll       They remain forever. \\
\poeml \v{2}Just as mountains encircle Jerusalem, \\
\poemll    so the \divine{Lord} encircles his people, \\
\poemlll       from now to eternity. \\
\poeml \v{3}For evil's scepter will not rest \\
\poemll    on the land that has been allotted to the righteous, \\
\poeml and so the righteous will not direct themselves\fnote{Lit. \fbib{will not set their hands}} to do wrong. \\
\poeml \v{4}\divine{Lord}, do good to those who are good, \\
\poemll    and to those who are upright in heart.\fnote{So LXX DSS 4QPs\textsuperscript{e} 11QPs\textsuperscript{a}; MT reads \fbib{in their hearts}} \\
\poeml \v{5}But for those who choose their own devious paths, \\
\poemll    the \divine{Lord} will lead them away, \\
\poemlll       along with those who practice evil. \\
\poeml Peace be upon Israel.
\end{poetry}
\labelpsalm{126}
\psalminfo{A Song of Ascents}
\passage{The Exiles Restored}

\begin{poetry}
\poeml \v{1}When the \divine{Lord} brought back Zion's exiles,\fnote{Or \fbib{fortunes}} \\
\poemll    we were like dreamers.\fnote{So MT; LXX DSS11QPsa read \fbib{were restored}} \\
\poeml \v{2}Then our mouths were filled with laughter, \\
\poemll    and our tongues formed joyful shouts. \\
\poeml Then it was said among the nations, \\
\poemll    ``The \divine{Lord} has done great things for them.'' \\
\poeml \v{3}The great things that the \divine{Lord} has done for us \\
\poemll    gladden us. \\
\poeml \v{4}Restore our exiles,\fnote{Or \fbib{fortunes}} \divine{Lord}, \\
\poemll    like the streams of the Negev.\fnote{I.e. the southern regions of the Sinai peninsula; cf. Josh 10:40} \\
\poeml \v{5}Those who weep while they plant \\
\poemll    will sing for joy while they harvest. \\
\poeml \v{6}The one who goes out weeping,\fnote{So MT and DSS 11QPs\textsuperscript{a} (corrected); LXX DSS 11QPs\textsuperscript{a} (original) read \fbib{out and weeps}} \\
\poemll    carrying a bag of seeds, \\
\poeml will surely return with a joyful song, \\
\poemll    bearing sheaves from his harvest.\fnote{The Heb. lacks \fbib{harvest}}
\end{poetry}
\labelpsalm{127}
\psalminfo{A Solomonic Song of Ascents}
\passage{God's Blessing in the Family}

\begin{poetry}
\poeml \v{1}Unless the \divine{Lord} builds the house, \\
\poemll    its builders labor uselessly. \\
\poeml Unless the \divine{Lord} guards the city, \\
\poemll    its security forces keep watch uselessly. \\
\poeml \v{2}It is useless to get up early \\
\poemll    and to stay up late,\fnote{Lit. \fbib{delay sitting}} \\
\poeml eating the food of exhausting labor--- \\
\poemll    truly he gives sleep to those he loves. \\
\poeml \v{3}Children\fnote{Lit. \fbib{Sons}} are a gift\fnote{Lit. \fbib{heritage}} from the \divine{Lord}; \\
\poemll    a productive womb, the \divine{Lord}'s\fnote{The Heb. lacks \fbib{\divine{Lord}'s}} reward. \\
\poeml \v{4}As arrows in the hand of a warrior, \\
\poemll    so also are children\fnote{Lit. \fbib{sons}} born during one's\fnote{The Heb. lacks \fbib{born during one's}} youth. \\
\poeml \v{5}How blessed\fnote{Or \fbib{happy}} is the man whose quiver is full of them! \\
\poemll    He\fnote{Lit. \fbib{They}} will not be ashamed \\
\poemlll       as they confront their enemies at the city gate.
\end{poetry}
\labelpsalm{128}
\psalminfo{A Song of Ascents}
\passage{The Blessings of Fearing God}

\begin{poetry}
\poeml \v{1}How blessed\fnote{Or \fbib{happy}} are all who fear the \divine{Lord} \\
\poemll    as they follow in his ways. \\
\poeml \v{2}You will eat from the work of your hands; \\
\poemll    you will be happy, and it will go well for you. \\
\poeml \v{3}Your wife will be like a fruitful vine within your house; \\
\poemll    your children\fnote{Lit. \fbib{sons}} like olive shoots surrounding your table. \\
\poeml \v{4}See how the man will be blessed \\
\poemll    who fears the \divine{Lord}. \\
\poeml \v{5}May the \divine{Lord} bless you from Zion, \\
\poemll    and may you observe the prosperity of Jerusalem \\
\poemlll       every day that you live! \\
\poeml \v{6}And may you see your children's children! \\
\poemll    Peace be on Israel!
\end{poetry}
\labelpsalm{129}
\psalminfo{A Song of Ascents}
\passage{God Defeats Israel's Enemies}

\begin{poetry}
\poeml \v{1}``Since my youth they have often persecuted me,'' \\
\poemll    let Israel repeat it, \\
\poeml \v{2}``Since my youth they have often persecuted me, \\
\poemll    yet they haven't defeated me. \\
\poeml \v{3}Wicked people\fnote{So LXX DSS 11QPs\textsuperscript{a}; MT reads \fbib{The ploughman}} plowed over my back, \\
\poemll    creating long-lasting wounds.''\fnote{Or \fbib{long furrows}; LXX reads \fbib{back; they prolonged their lawlessness}} \\
\poeml \v{4}The \divine{Lord} is righteous--- \\
\poemll    he has cut me free from the cords of the wicked. \\
\poeml \v{5}Let all who hate Zion \\
\poemll    be turned away and be ashamed. \\
\poeml \v{6}May they become like a tuft of grass on a roof top, \\
\poemll    that withers before it takes root--- \\
\poeml \v{7}not enough to fill one's hand \\
\poemll    or to bundle in one's arms. \\
\poeml \v{8}And may those who pass by never tell them, \\
\poemll    ``May the \divine{Lord}'s blessing be upon you. \\
\poemlll       We bless you in the name of the \divine{Lord}.''
\end{poetry}
\labelpsalm{130}
\psalminfo{A Song of Ascents}
\passage{A Prayer for Mercy}

\begin{poetry}
\poeml \v{1}I cry to you from the depths, \divine{Lord}, \\
\poeml \v{2}Lord, listen to my voice; \\
\poeml let your ears pay attention \\
\poemll    to what I ask of you!\fnote{Lit. \fbib{to the voice of my supplications}} \\
\poeml \v{3}\divine{Lord},\fnote{Lit. \fbib{Yah}} if you were to record iniquities, \\
\poemll    Lord, who could remain standing? \\
\poeml \v{4}But with you there is forgiveness, \\
\poemll    so that you may be feared. \\
\poeml \v{5}I wait for the \divine{Lord}; \\
\poemll    my soul waits, \\
\poemlll       and I will hope in his word. \\
\poeml \v{6}My soul looks to the Lord \\
\poemll    more than watchmen look for the morning--- \\
\poemlll       more, indeed, than\fnote{The Heb. lacks \fbib{more indeed, than}} watchmen for the morning. \\
\poeml \v{7}Israel, hope in the \divine{Lord}! \\
\poemll    For with the \divine{Lord} there is gracious love, \\
\poemlll       along with abundant redemption. \\
\poeml \v{8}And he will redeem Israel \\
\poemll    from all its sins.
\end{poetry}
\labelpsalm{131}
\psalminfo{A Davidic Song of Ascents}
\passage{Hope in the \divine{Lord}}

\begin{poetry}
\poeml \v{1}\divine{Lord}, my heart is not arrogant, \\
\poemll    nor do I look haughty. \\
\poeml I do not aspire\fnote{Lit. \fbib{walk}} to great things, \\
\poemll    nor concern myself with things beyond my ability. \\
\poeml \v{2}Instead, I have composed and quieted myself \\
\poemll    like a weaned child with its mother; \\
\poemlll       I am like a weaned child. \\
\poeml \v{3}Place your hope in the \divine{Lord}, Israel, \\
\poemll    both now and forever.
\end{poetry}
\labelpsalm{132}
\psalminfo{A Song of Ascents}
\passage{The \divine{Lord} Lives in Zion}

\begin{poetry}
\poeml \v{1}\divine{Lord}, remember in David's favor \\
\poemll    all of his troubles; \\
\poeml \v{2}how he swore an oath to the \divine{Lord}, \\
\poemll    vowing to the Mighty One of Jacob, \\
\poeml \v{3}``I will not enter\fnote{Lit. \fbib{enter the tent that is}} my house, \\
\poemll    or lie down on\fnote{Lit. \fbib{on the couch that is}} my bed, \\
\poeml \v{4}or let myself go to sleep\fnote{Lit. \fbib{or give sleep to my eyes}} \\
\poemll    or even take a nap,\fnote{Lit. \fbib{or let my eyelids slumber}} \\
\poeml \v{5}until I locate a place for the \divine{Lord}, \\
\poemll    a dwelling place for the Mighty One of Jacob.'' \\
\poeml \v{6}We heard about it\fnote{I.e. the Ark of the Covenant} in Ephrata;\fnote{I.e. the region of Bethlehem} \\
\poemll    we found it in the fields of Jaar.\fnote{Cf. 1Sam 7:1-2; 1Chr 16:5-6} \\
\poeml \v{7}Let's go to his dwelling place \\
\poemll    and worship at his footstool. \\
\poeml \v{8}Arise, \divine{Lord}, \\
\poemll    and go to your resting place, \\
\poemlll       you and the ark of your strength. \\
\poeml \v{9}May your priests be clothed with righteousness \\
\poemll    and may your godly ones shout for joy. \\
\poeml \v{10}For the sake of your servant David, \\
\poemll    don't turn away the face of your anointed one. \\
\poeml \v{11}The \divine{Lord} made an oath to David \\
\poemll    from which he will not retreat: \\
\poeml ``One of your sons \\
\poemll    I will set in place on your throne. \\
\poeml \v{12}If your sons keep my covenant \\
\poemll    and my statutes that I will teach them, \\
\poemlll       then their sons will also sit on your throne forever.'' \\
\poeml \v{13}For the \divine{Lord} has chosen Zion, \\
\poemll    desiring it as his dwelling place. \\
\poeml \v{14}``This is my resting place forever. \\
\poemll    Here I will live, \\
\poemlll       because I desire to do so. \\
\poeml \v{15}I will bless its provisions abundantly; \\
\poemll    I will satiate its poor with food.\fnote{Lit. \fbib{bread}} \\
\poeml \v{16}I will clothe its priests with salvation \\
\poemll    and its godly ones will shout for joy. \\
\poeml \v{17}There I will create a power base\fnote{Lit. \fbib{will cause a horn to sprout}} for David--- \\
\poemll    I have prepared a lamp for my anointed one. \\
\poeml \v{18}I will clothe his enemies with disgrace, \\
\poemll    but on him his crown will shine.''
\end{poetry}
\labelpsalm{133}
\psalminfo{A Davidic Song of Ascents}
\passage{The Significance of Unity}

\begin{poetry}
\poeml \v{1}Look how good and how pleasant it is \\
\poemll    when brothers live together in unity! \\
\poeml \v{2}It is like precious oil on the head, \\
\poemll    descending to the beard--- \\
\poeml even to Aaron's beard--- \\
\poemll    and flowing down to the edge of his robes. \\
\poeml \v{3}It is like the dew of Hermon \\
\poemll    falling on Zion's mountains. \\
\poeml For there the \divine{Lord} commanded his blessing--- \\
\poemll    life everlasting.
\end{poetry}
\labelpsalm{134}
\psalminfo{A Song of Ascents}
\passage{Praise to the Creator}

\begin{poetry}
\poeml \v{1}Now bless the \divine{Lord}, \\
\poemll    all you servants of the \divine{Lord} \\
\poemlll       who serve\fnote{Lit. \fbib{stand}} nightly in the \divine{Lord}'s Temple. \\
\poeml \v{2}Lift up your hands to the Holy Place \\
\poemll    and bless the \divine{Lord}. \\
\poeml \v{3}May the \divine{Lord} who fashions heaven and earth \\
\poemll    bless you from Zion.
\end{poetry}
\labelpsalm{135}
\passage{Praising God for His Graciousness}

\begin{poetry}
\poeml \v{1}Hallelujah! \\
\poemll    Praise the name of the \divine{Lord}! \\
\poeml Give praise, you servants of the \divine{Lord}, \\
\poeml \v{2}you who are standing in the \divine{Lord}'s Temple, \\
\poemlll       in the courtyards of the house of our God. \\
\poeml \v{3}Praise the \divine{Lord}, \\
\poemll    because the \divine{Lord} is good; \\
\poeml Sing to his name, \\
\poemll    for he is gracious. \\
\poeml \v{4}It is Jacob whom the \divine{Lord} chose for himself--- \\
\poemll    Israel as his personal possession. \\
\poeml \v{5}Indeed, I know that the \divine{Lord} is great, \\
\poemll    and that our Lord\fnote{So MT LXX; DSS 11QPs\textsuperscript{a} reads \fbib{God}} surpasses all gods. \\
\poeml \v{6}The \divine{Lord} does whatever pleases him \\
\poemll    in heaven and on earth, \\
\poemlll       in the seas and all its\fnote{So DSS 11QPs\textsuperscript{a}; MT LXX lack \fbib{its}} deep regions. \\
\poeml \v{7}He makes the clouds rise from the ends of the earth, \\
\poemll    fashioning lightning for the rain, \\
\poemlll       bringing the wind from his storehouses. \\
\poeml \v{8}It was the \divine{Lord}\fnote{Lit. \fbib{was he}} who struck down the firstborn of Egypt, \\
\poemll    including both men and animals. \\
\poeml \v{9}He sent signs and wonders among you, Egypt, \\
\poemll    before\fnote{Or \fbib{among}} Pharaoh and all his servants. \\
\poeml \v{10}He struck down many nations, \\
\poemll    killing many kings--- \\
\poeml \v{11}Sihon, king of the Amorites, \\
\poemll    Og, king of Bashan, \\
\poemlll       and every kingdom of Canaan--- \\
\poeml \v{12}and he gave their land as an inheritance, \\
\poemll    an inheritance to his people Israel. \\
\poeml \v{13}Your name, \divine{Lord}, exists forever, \\
\poemll    and your reputation, \divine{Lord}, throughout the ages. \\
\poeml \v{14}For the \divine{Lord} will vindicate his people, \\
\poemll    and he will show compassion on his servants. \\
\poeml \v{15}The idols of the nations are silver and gold, \\
\poemll    worked by\fnote{So MT LXX; DSS 4QPs\textsuperscript{k} reads \fbib{gold, products of}} the hands of human beings. \\
\poeml \v{16}Mouths are attributed to them, \\
\poemll    but they cannot speak; \\
\poeml sight is attributed to them, \\
\poemll    but they cannot see; \\
\poeml \v{17}ears are attributed to them, \\
\poemll    but they do not hear, \\
\poemlll       and there is no breath in their mouths. \\
\poeml \v{18}Those who craft them--- \\
\poemll    and all\fnote{So LXX DSS 11QPs\textsuperscript{a}; the lacks \fbib{and}} who trust in them--- \\
\poemlll       will become like them. \\
\poeml \v{19}House of Israel, bless the \divine{Lord}! \\
\poemll    House of Aaron, bless the \divine{Lord}! \\
\poeml \v{20}House of Levi, bless the \divine{Lord}! \\
\poemll    You who fear the \divine{Lord}, bless the \divine{Lord}! \\
\poeml \v{21}Blessed be the \divine{Lord} from Zion, \\
\poemll    he who lives in Jerusalem. \\
\poeml Hallelujah!
\end{poetry}
\labelpsalm{136}
\passage{God's Gracious Love}

\begin{poetry}
\poeml \v{1}Give thanks to the \divine{Lord}, for he is good, \\
\poemll    for his gracious love is everlasting. \\
\poeml \v{2}Give thanks to the God of gods, \\
\poemll    for his gracious love is everlasting. \\
\poeml \v{3}Give thanks to the Lord of lords, \\
\poemll    for his gracious love is everlasting--- \\
\poeml \v{4}To the one who alone does great and wondrous things, \\
\poemll    for his gracious love is everlasting--- \\
\poeml \v{5}to the one who by wisdom made the heavens, \\
\poemll    for his gracious love is everlasting--- \\
\poeml \v{6}to the one who spread out the earth over the waters, \\
\poemll    for his gracious love is everlasting--- \\
\poeml \v{7}to the one who made the great lights, \\
\poemll    for his gracious love is everlasting--- \\
\poeml \v{8}the sun to illumine\fnote{Lit. \fbib{govern}; cf. Gen 1:16} the day, \\
\poemll    for his gracious love is everlasting--- \\
\poeml \v{9}and the moon and stars to illumine\fnote{Lit. \fbib{govern}; cf. Gen 1:16} the night, \\
\poemll    for his gracious love is everlasting--- \\
\poeml \v{10}to the one who struck the firstborn of Egypt, \\
\poemll    for his gracious love is everlasting--- \\
\poeml \v{11}and brought Israel out from among them, \\
\poemll    for his gracious love is everlasting--- \\
\poeml \v{12}with a strong hand and an active\fnote{Lit. \fbib{outstretched}} arm, \\
\poemll    for his gracious love is everlasting. \\
\poeml \v{13}To the one who split the Reed\fnote{So MT; LXX reads \fbib{Red}} Sea in two \\
\poemll    for his gracious love is everlasting--- \\
\poeml \v{14}and made Israel pass through the middle of it, \\
\poemll    for his gracious love is everlasting--- \\
\poeml \v{15}and cast Pharaoh and his armies into the Reed\fnote{So MT; LXX reads \fbib{Red}} Sea, \\
\poemll    for his gracious love is everlasting. \\
\poeml \v{16}To the one who led his people into the wilderness, \\
\poemll    for his gracious love is everlasting--- \\
\poeml \v{17}to the one who struck down great kings, \\
\poemll    for his gracious love is everlasting--- \\
\poeml \v{18}and killed famous kings, \\
\poemll    for his gracious love is everlasting--- \\
\poeml \v{19}including Sihon king of the Amorites, \\
\poemll    for his gracious love is everlasting--- \\
\poeml \v{20}and Og king of Bashan, \\
\poemll    for his gracious love is everlasting--- \\
\poeml \v{21}and gave their land as an inheritance, \\
\poemll    for his gracious love is everlasting--- \\
\poeml \v{22}to Israel his servant as a possession, \\
\poemll    for his gracious love is everlasting--- \\
\poeml \v{23}He it is who remembered us in our lowly circumstances, \\
\poemll    for his gracious love is everlasting--- \\
\poeml \v{24}and rescued us from our enemies, \\
\poemll    for his gracious love is everlasting. \\
\poeml \v{25}He gives food to all creatures, \\
\poemll    for his gracious love is everlasting. \\
\poeml \v{26}Give thanks to the God of Heaven, \\
\poemll    for his gracious love is everlasting.
\end{poetry}
\labelpsalm{137}
\passage{Remembering Jerusalem}

\begin{poetry}
\poeml \v{1}There we sat down and cried--- \\
\poemll    by the rivers of Babylon--- \\
\poemlll       as we remembered Zion. \\
\poeml \v{2}On the willows there \\
\poemll    we hung our harps, \\
\poeml \v{3}for it was there that our captors \\
\poemll    asked us for songs \\
\poeml and our torturers demanded joy from us, \\
\poemll    ``Sing us one of the songs about Zion!'' \\
\poeml \v{4}How are we to sing the song of the \divine{Lord} \\
\poemll    on foreign soil? \\
\poeml \v{5}If I forget you, Jerusalem, \\
\poemll    may my right hand cease to function.\fnote{Lit. \fbib{remember}} \\
\poeml \v{6}May my tongue stick to the roof of my mouth \\
\poemll    if I don't remember you, \\
\poeml if I don't consider Jerusalem \\
\poemll    to be more important than my highest joy. \\
\poeml \v{7}Remember the day of Jerusalem's fall,\fnote{The Heb. lacks \fbib{fall}} \divine{Lord}, \\
\poemll    because of\fnote{Lit. \fbib{against}} the Edomites, \\
\poeml who kept saying, ``Tear it down! \\
\poemll    Tear it right down to its foundations!'' \\
\poeml \v{8}Daughter of Babylon! You devastator! \\
\poemll    How blessed will be the one who pays you back \\
\poemlll       for what you have done to us. \\
\poeml \v{9}How blessed will be the one who seizes your young children \\
\poemll    and pulverizes them against the cliff!
\end{poetry}
\labelpsalm{138}
\passage{Thanksgiving to God}

\begin{poetry}
\poeml \v{1}\divine{Lord},\fnote{So LXX DSS 11QPs\textsuperscript{a}; MT and Aquilla lack \fbib{\divine{Lord}}} I thank\fnote{So MT; LXX reads \fbib{acknowledge}} you with all of my heart; \\
\poemll    because you heard the words that I spoke,\fnote{So LXX; MT DSS lack this line} \\
\poemlll       I will sing your praise before the heavenly beings.\fnote{Or \fbib{the gods}; LXX reads \fbib{the angels}} \\
\poeml \v{2}I will bow down in worship toward your holy Temple \\
\poemll    and give thanks to your name for your gracious love and truth, \\
\poeml for you have done great things \\
\poemll    to carry out your word \\
\poemlll       consistent with your name. \\
\poeml \v{3}When\fnote{Lit. \fbib{In the day}} I called out, you answered me; \\
\poemll    you strengthened me. \\
\poeml \v{4}\divine{Lord}, all the kings of the earth will give you thanks, \\
\poemll    for they have heard what you have spoken.\fnote{Lit. \fbib{heard the words of your mouth}} \\
\poeml \v{5}They will sing about the ways of the \divine{Lord}, \\
\poemll    for great is the glory of the \divine{Lord}! \\
\poeml \v{6}Though the \divine{Lord} is highly exalted, \\
\poemll    yet he pays attention to those who are lowly regarded, \\
\poemlll       but he is aware of the arrogant from afar. \\
\poeml \v{7}Though I walk straight into trouble, \\
\poemll    you preserve my life, \\
\poeml stretching out your hand \\
\poemll    to fight the vehemence of my enemies, \\
\poemlll       and your right hand delivers me. \\
\poeml \v{8}The \divine{Lord} will complete what his purpose is for me. \\
\poemll    \divine{Lord}, your gracious love is eternal; \\
\poemlll       do not abandon your personal work in me.\fnote{Lit. \fbib{abandon the work of your hand}}
\end{poetry}
\labelpsalm{139}
\psalminfo{To the Music Director: A Davidic Song}
\passage{God's Knowledge and Presence}

\begin{poetry}
\poeml \v{1}\divine{Lord}, you have examined me; \\
\poemll    you have known me. \\
\poeml \v{2}You know when I rest\fnote{Lit. \fbib{know my sitting}} \\
\poemll    and when I am active.\fnote{Lit. \fbib{and my rising}} \\
\poeml You understand what I am thinking \\
\poemll    when I am distant from you.\fnote{Or \fbib{thinking from a distance}} \\
\poeml \v{3}You scrutinize my life and my rest;\fnote{Or \fbib{death}; Lit. \fbib{my path and my lying down}} \\
\poemll    you are familiar with all of my ways. \\
\poeml \v{4}Even before I have formed a word with my tongue, \\
\poemll    you, \divine{Lord}, know it completely! \\
\poeml \v{5}You encircle me from back to front, \\
\poemll    placing your hand upon me. \\
\poeml \v{6}Knowledge like this is too amazing for me. \\
\poemll    It is beyond my reach, \\
\poemlll       and I cannot fathom it.
\passage{The Magnitude of God}
\poeml \v{7}Where can I flee from your spirit? \\
\poemll    Or where will I run from your presence? \\
\poeml \v{8}If I rise to heaven, there you are! \\
\poemll    If I lay down with the dead,\fnote{Lit. \fbib{to Sheol}; i.e. the realm of the dead} there you are! \\
\poeml \v{9}If I take wings with the dawn \\
\poemll    and settle down on the western horizon\fnote{Lit. \fbib{the end of the sea}} \\
\poeml \v{10}your hand will guide me there, too, \\
\poemll    while your right hand keeps a firm grip on me. \\
\poeml \v{11}If I say, ``Darkness will surely conceal me, \\
\poemll    and the light around me will become night,''\fnote{So MT LXX; DSS 11QPs\textsuperscript{a} reads \fbib{And let me say, ``Surely darkness conceals and night has girded me about.''}} \\
\poeml \v{12}even darkness isn't dark to you, \\
\poemll    darkness and light are the same to you.\fnote{The Heb. lacks \fbib{to you}} \\
\poeml \v{13}It was you who formed my internal organs,\fnote{Lit. \fbib{my kidneys}} \\
\poemll    fashioning me within my mother's womb. \\
\poeml \v{14}I praise you, \\
\poemll    because you are fearful and wondrous!\fnote{So DSS 11QPsa Syr Hieronymus; MT LXX read \fbib{because I am fearfully and wonderfully made}} \\
\poeml Your work is wonderful, \\
\poemll    and I am fully aware of it. \\
\poeml \v{15}My frame was not hidden from you \\
\poemll    while I was being crafted in a hidden place, \\
\poemlll       knit together in the depths of the earth. \\
\poeml \v{16}Your eyes looked upon my embryo, \\
\poemll    and everything was recorded in your book. \\
\poeml The days scheduled\fnote{The Heb. lacks \fbib{scheduled}} for my formation were inscribed, \\
\poemll    even though not one of them had come yet.\fnote{The Heb. lacks \fbib{had come yet}} \\
\poeml \v{17}How deep\fnote{Or \fbib{precious}} are your thoughts, God! \\
\poemll    How great is their number! \\
\poeml \v{18}Were I to count them, \\
\poemll    they would number more than the sand. \\
\poemlll       When I awake, I will be with you. \\
\poeml \v{19}God, if only you would execute the wicked, \\
\poemll    so that\fnote{So LXX DSS 11QPsa; MT reads \fbib{and so that}} the men guilty of bloodshed would get away from me, \\
\poeml \v{20}who speak against you with evil motives, \\
\poemll    your enemies who are acting in vain. \\
\poeml \v{21}I hate those who hate you, \divine{Lord}, do I not? \\
\poemll    I loathe those who rebel against you, do I not ? \\
\poeml \v{22}With consummate hatred I hate them; \\
\poemll    I consider them my enemies. \\
\poeml \v{23}Examine me, God, and know my mind, \\
\poemll    test me, and know my thoughts. \\
\poeml \v{24}See if there is any offensive tendency\fnote{Lit. \fbib{way}} in me, \\
\poemll    and lead me in the eternal way.
\end{poetry}
\labelpsalm{140}
\psalminfo{To the Music Director: A Davidic Song}
\passage{A Prayer for Deliverance}

\begin{poetry}
\poeml \v{1}\fnote{V.1 is v. 2 in MT, and so throughout the chapter.}Deliver me, \divine{Lord}, from evil people, \\
\poemll    preserve me from violent men, \\
\poeml \v{2}who craft evil plans in their minds, \\
\poemll    inciting wars every day.\fnote{Lit. \fbib{all day}; LXX DSS 11QPs\textsuperscript{a} read \fbib{all the day}} \\
\poeml \v{3}They sharpen their tongues like a serpent; \\
\poemll    the venom of vipers is on their lips.
\end{poetry}
\interlude{Interlude}

\begin{poetry}
\poeml \v{4}Protect me, \divine{Lord}, from the control of evil people, \\
\poemll    from violent men who have planned to trip me. \\
\poeml \v{5}The arrogant have laid a trap for me; \\
\poemll    they have spread a net with ropes, \\
\poemlll       lining it with snares along the way.
\end{poetry}
\interlude{Interlude}

\begin{poetry}
\poeml \v{6}So I say to the \divine{Lord}, ``You are my God; \\
\poemll    listen to my voice \\
\poemlll       as I plead for mercy, \divine{Lord}. \\
\poeml \v{7}\divine{Lord}, my Lord, my strong deliverer, \\
\poemll    you have protected my head in the time\fnote{Lit. \fbib{day}} of battle. \\
\poeml \v{8}Never grant, \divine{Lord}, the desires of the wicked; \\
\poemll    never condone their plans \\
\poemlll       so they cannot exalt themselves.
\end{poetry}
\interlude{Interlude}

\begin{poetry}
\poeml \v{9}May those who surround me discover \\
\poemll    that the trouble they talk about falls on their own head! \\
\poeml \v{10}May burning coals fall on them; \\
\poemll    may they be cast into fire, \\
\poemlll       and into miry pits, never to rise again. \\
\poeml \v{11}Let not the slanderer\fnote{Lit. \fbib{the man of tongue}} become established in the land. \\
\poemll    May evil quickly hunt down the violent man. \\
\poeml \v{12}I know that the \divine{Lord} will act on behalf of the tormented, \\
\poemll    providing justice for the needy. \\
\poeml \v{13}Surely the righteous will give thanks to your name, \\
\poemll    while the upright live in your presence.
\end{poetry}
\labelpsalm{141}
\psalminfo{A Davidic Song}
\passage{A Prayer for Maturity}

\begin{poetry}
\poeml \v{1}\divine{Lord}, I call to you, \\
\poemll    be quick to listen to me when I cry out! \\
\poeml \v{2}Let my prayer be like incense offered before you, \\
\poemll    and my uplifted hands like the evening sacrifice. \\
\poeml \v{3}\divine{Lord}, set a guard over my mouth; \\
\poemll    keep watch over the door to my lips. \\
\poeml \v{4}Don't let my heart turn toward evil \\
\poemll    or involve itself in wicked activities \\
\poeml with men who practice iniquity. \\
\poemll    Let me not feast on their delicacies. \\
\poeml \v{5}Let one who is righteous strike me; \\
\poemll    It is an act of gracious love. \\
\poeml Let him rebuke me, \\
\poemll    because it is oil for my head; \\
\poemll    do not let my head refuse it. \\
\poeml My prayers continuously will be \\
\poemll    against their wicked activities. \\
\poeml \v{6}When their judges are thrown off the cliff, \\
\poemll    the people\fnote{Lit. \fbib{they}} will hear my words, \\
\poemlll       for they are appropriate. \\
\poeml \v{7}Just as one plows and breaks up the earth, \\
\poemll    our\fnote{So MT LXX; DSS 11QPs\textsuperscript{a} reads \fbib{my}; Syr reads \fbib{their}} bones are scattered \\
\poemlll       near the entrance to the place of the dead.\fnote{Lit. \fbib{to Sheol}; i.e. the realm of the dead} \\
\poeml \v{8}Nevertheless, my eyes are on you, Lord \divine{God}, \\
\poemll    as I seek protection in you. \\
\poemlll       Don't leave me defenseless! \\
\poeml \v{9}Protect me from the trap laid for me \\
\poemll    and from the snares of those who practice evil. \\
\poeml \v{10}Let the wicked fall into their own nets, \\
\poemll    while I come through.
\end{poetry}
\labelpsalm{142}
\psalminfo{A Davidic Song, when he was in the cave.\fnote{T cf. 1Sam 24:3-4} A prayer.}
\passage{A Call to God for Help}

\begin{poetry}
\poeml \v{1}My voice cries out to the \divine{Lord}; \\
\poemll    my voice pleads for mercy to the \divine{Lord}. \\
\poeml \v{2}I pour out my complaint to him, \\
\poemll    telling him all of my troubles. \\
\poeml \v{3}Though my spirit grows faint within me, \\
\poemll    you are aware of my path. \\
\poeml Wherever I go, \\
\poemll    they have hidden a trap for me. \\
\poeml \v{4}I look to my right\fnote{So LXX and DSS 11QPs\textsuperscript{a}; MT reads \fbib{Look to the right}} and observe--- \\
\poemll    no one is concerned about me. \\
\poeml There is nowhere I can go for refuge, \\
\poemll    and no one cares for me. \\
\poeml \v{5}So I cry to you, Lord, \\
\poemll    declaring, ``You are my refuge, \\
\poemlll       my only\fnote{The Heb. lacks \fbib{only}} possession while I am on this earth.''\fnote{Lit. \fbib{possession in the land of the living}} \\
\poeml \v{6}Pay attention to my cry, \\
\poemll    for I have been brought very low. \\
\poeml Deliver me from my tormentors, \\
\poemll    for they are far too strong for me. \\
\poeml \v{7}Break me out of this prison, \\
\poemll    so I can give thanks to your name. \\
\poeml The righteous will surround me, \\
\poemll    for you will deal generously with me.
\end{poetry}
\labelpsalm{143}
\psalminfo{A Davidic Song}
\passage{Longing for God}

\begin{poetry}
\poeml \v{1}\divine{Lord}, hear my prayer; \\
\poemll    pay attention to my request, because you are faithful; \\
\poemlll       answer me in your righteousness. \\
\poeml \v{2}Do not enter into judgment with your servant, \\
\poemll    for no living person is righteous in your sight. \\
\poeml \v{3}For those who oppose me are pursuing my life, \\
\poemll    crushing me to the ground, \\
\poeml making me sit in darkness \\
\poemll    like those who died long ago. \\
\poeml \v{4}As a result, my spirit is desolate within me, \\
\poemll    and my mind within me is appalled. \\
\poeml \v{5}I remember the former times, \\
\poemll    meditating on everything you have done. \\
\poemlll       I think about the work\fnote{So MT; LXX DSS 11QPs\textsuperscript{a} read \fbib{works}} of your hands. \\
\poeml \v{6}I stretch out my hands toward you, \\
\poemll    longing for you like a parched land.
\end{poetry}
\interlude{Interlude}

\begin{poetry}
\poeml \v{7}Answer me quickly, \divine{Lord}; \\
\poemll    my spirit is failing. \\
\poeml Do not hide your face from me; \\
\poemll    otherwise, I will become like those who descend to the Pit,\fnote{I.e. the place of punishment in the afterlife} \\
\poeml \v{8}In the morning let me hear of your gracious love, \\
\poemll    for in you I trust. \\
\poeml Cause me to know the way I should take, \\
\poemll    because I have set my hope on you. \\
\poeml \v{9}Deliver me from my enemies, \divine{Lord}. \\
\poemll    I have taken refuge in you. \\
\poeml \v{10}Teach me to do your will, \\
\poemll    for you are my God. \\
\poemlll       Let your good Spirit lead me on level ground. \\
\poeml \v{11}For the sake of your name, \divine{Lord}, \\
\poemll    preserve my life. \\
\poeml Because you are righteous, \\
\poemll    bring me out of trouble. \\
\poeml \v{12}Because of your gracious love, \\
\poemll    you will cut off my enemies. \\
\poeml You will destroy all who oppose me, \\
\poemll    for I am your servant.
\end{poetry}
\labelpsalm{144}
\psalminfo{Davidic}
\passage{A Song for God's Provision}

\begin{poetry}
\poeml \v{1}Blessed be the \divine{Lord}, my rock, \\
\poemll    who trains my hands for battle \\
\poemlll       and my fingers for warfare, \\
\poeml \v{2}he is my gracious love and my fortress, \\
\poemll    my strong tower and my deliverer, \\
\poeml my shield and the one in whom I find refuge, \\
\poemll    who subdues\fnote{So LXX; the Heb. lacks \fbib{subdues}} peoples\fnote{So DSS 11QPs\textsuperscript{a} Sebir Aquila Syr Targ Hieronymus; cf. Psa 18:48; 2 Sam 22:48; LXX reads \fbib{subdues my people}} under me. \\
\poeml \v{3}\divine{Lord}, what are human beings, \\
\poemll    that you should care about them, \\
\poeml or mortal man, \\
\poemll    that you should think about him? \\
\poeml \v{4}The human person is a mere empty breath; \\
\poemll    his days are like a fading shadow. \\
\poeml \v{5}Bow your heavens, \divine{Lord}, and descend;\fnote{So MT (imperfect verb); LXX, DSS 11QPs\textsuperscript{a} read \fbib{and descend!}; i.e. a verb of command} \\
\poemll    touch the mountains, and they will smolder. \\
\poeml \v{6}Send forth lightning and scatter the enemy,\fnote{Lit. \fbib{scatter them}} \\
\poemll    shoot your arrows and confuse them. \\
\poeml \v{7}Reach down your hand from your high place; \\
\poemll    rescue me and deliver me from mighty waters, \\
\poemlll       from the control of foreigners.\fnote{Lit. \fbib{the hand of the sons of strangers}} \\
\poeml \v{8}Their mouths speak lies, \\
\poemll    and their right hand deceives,\fnote{I.e. they swear to false oaths} \\
\poeml \v{9}God, I will sing a new song to you. \\
\poemll    On a harp of ten strings I will play to you--- \\
\poeml \v{10}to you who gives victory to kings, \\
\poemll    rescuing his servant David from cruel swords. \\
\poeml \v{11}Rescue me and deliver me \\
\poemll    from the control of foreigners,\fnote{Lit. \fbib{the hand of the sons of strangers}} \\
\poeml whose mouths speak lies, \\
\poemll    and whose right hand deceives.\fnote{I.e. they swear to false oaths} \\
\poeml \v{12}May our sons in their youth be like full-grown plants, \\
\poemll    and our daughters like pillars \\
\poemlll       destined to decorate a palace. \\
\poeml \v{13}May our granaries be filled, \\
\poemll    storing produce in abundance; \\
\poeml may our sheep bring forth thousands, \\
\poemll    even tens of thousands in our fields. \\
\poeml \v{14}May our cattle grow heavy with young, \\
\poemll    with no damage or loss. \\
\poeml May there be no cry of anguish in our streets! \\
\poeml \v{15}Happy are the people to whom these things come; \\
\poemll    happy are the people whose God is the \divine{Lord}.
\end{poetry}
\labelpsalm{145}
\psalminfo{A Davidic Psalm\fnote{T In this acrostic psalm each verse begins with a successive letter of the Heb. alphabet, except that v13b, corresponding to the Heb. Letter nun, is missing from MT}}
\passage{Praising God for His Works}

\begin{poetry}
\poeml \v{1}I will speak highly of you, my God and King, \\
\poemll    and I will bless your name forever and ever. \\
\poeml \v{2}I will bless you every day \\
\poemll    and I will praise your name forever and ever. \\
\poeml \v{3}The \divine{Lord} is great, \\
\poemll    and to be praised highly, \\
\poemlll       though his greatness is indescribable. \\
\poeml \v{4}One generation will acclaim your works to another \\
\poemll    and will describe your mighty actions. \\
\poeml \v{5}I\fnote{So MT; LXX Syr DSS 11QPs\textsuperscript{a} read \fbib{They}} will speak about the glorious splendor of your majesty \\
\poemll    as well as\fnote{So LXX DSS 11QPs\textsuperscript{a}; the Heb. lacks \fbib{as}} your awesome actions. \\
\poeml \v{6}People\fnote{Lit. \fbib{They}} will speak about the might of your great deeds, \\
\poemll    and I will announce your greatness. \\
\poeml \v{7}They will extol the fame of your abundant goodness, \\
\poemll    and will sing out loud about your righteousness. \\
\poeml \v{8}Gracious and merciful is the \divine{Lord}, \\
\poemll    slow to become angry, \\
\poemlll       and overflowing with gracious love. \\
\poeml \v{9}The \divine{Lord} is good to everyone \\
\poemll    and his mercies extend to everything he does. \\
\poeml \v{10}\divine{Lord}, everything you have done will praise you, \\
\poemll    and your holy ones will bless you. \\
\poeml \v{11}They will speak about the glory of your kingdom, \\
\poemll    and they will talk about your might, \\
\poeml \v{12}in order to make known your mighty acts to mankind\fnote{Lit. \fbib{the children of the Man}} \\
\poemll    as well as the majestic splendor of your kingdom. \\
\poeml \v{13}Your kingdom is an everlasting kingdom, \\
\poemll    and your authority endures from one generation to another. \\
\poeml 13bGod\fnote{So DSS 11QPs\textsuperscript{a}; LXX Vg Syr read \fbib{The \divine{Lord};} MT lacks this v.} is faithful about everything he says \\
\poemll    and merciful in everything he does. \\
\poeml \v{14}The \divine{Lord} supports everyone who falls \\
\poemll    and raises up those who are bowed down. \\
\poeml \v{15}Everyone's eyes are on you, \\
\poemll    as you give them their food in due time. \\
\poeml \v{16}You\fnote{So MT; LXX DSS 11QPs\textsuperscript{a} read \fbib{You yourself}} open your hand \\
\poemll    and keep on satisfying the desire of every living thing. \\
\poeml \v{17}The \divine{Lord} is righteous in all of his ways \\
\poemll    and graciously loving in all of his activities. \\
\poeml \v{18}The \divine{Lord} remains near to all who call out to him, \\
\poemll    to everyone who calls out to him sincerely.\fnote{Or \fbib{truthfully}} \\
\poeml \v{19}He fulfills the desire of those who fear him, \\
\poemll    hearing their cry and saving them. \\
\poeml \v{20}The \divine{Lord} preserves everyone who loves him, \\
\poemll    but he will destroy all of the wicked. \\
\poeml \v{21}My mouth will praise the \divine{Lord}, \\
\poemll    and all creatures will bless his holy name forever and ever.
\end{poetry}
\labelpsalm{146}
\passage{Praise to God the Help of Israel}

\begin{poetry}
\poeml \v{1}Hallelujah! \\
\poemll    Praise the \divine{Lord}, my soul! \\
\poeml \v{2}I will praise the \divine{Lord} as long as I live, \\
\poemll    singing praises to my God while I exist. \\
\poeml \v{3}Do not look to nobles, \\
\poemll    nor to mere human beings who cannot save. \\
\poeml \v{4}When they stop breathing, \\
\poemll    they return to the ground; \\
\poemlll       on that very day their plans evaporate! \\
\poeml \v{5}Happy is the one whose help is the God of Jacob, \\
\poemll    whose hope is in the \divine{Lord} his God, \\
\poeml \v{6}maker of heaven and earth, \\
\poemll    the seas and everything in them, \\
\poemlll       forever the guardian of truth, \\
\poeml \v{7}who brings justice for the oppressed, \\
\poemll    and who gives food to the hungry. \\
\poeml The \divine{Lord} frees the prisoners; \\
\poeml \v{8}the \divine{Lord} gives sight to the blind. \\
\poeml The \divine{Lord} lifts up those who are weighed down. \\
\poemll    The \divine{Lord} loves the righteous. \\
\poeml \v{9}The \divine{Lord} stands guard over the stranger; \\
\poemll    he supports both widows and orphans, \\
\poemlll       but makes the path of the wicked slippery.\fnote{Or \fbib{treacherous}} \\
\poeml \v{10}The \divine{Lord} will reign forever, \\
\poemll    your God, Zion, for all generations! \\
\poeml Hallelujah!
\end{poetry}
\labelpsalm{147}
\passage{Praise for God's Provision}

\begin{poetry}
\poeml \v{1}Hallelujah! \\
\poemll    It is good to sing praise to our God, \\
\poemlll       and it is fitting to sing glorious praise. \\
\poeml \v{2}The \divine{Lord} rebuilds Jerusalem; \\
\poemll    he gathers together the outcasts of Israel. \\
\poeml \v{3}He heals the brokenhearted, \\
\poemll    binding up their injuries. \\
\poeml \v{4}He keeps track of the number of stars, \\
\poemll    assigning names to all of them. \\
\poeml \v{5}Our Lord is great, \\
\poemll    and rich in power; \\
\poemlll       his understanding has no limitation. \\
\poeml \v{6}The \divine{Lord} supports the afflicted \\
\poemll    while he casts the wicked to the ground. \\
\poeml \v{7}Sing to the \divine{Lord} with thanksgiving, \\
\poemll    and compose music to our God with the lyre. \\
\poeml \v{8}He shields the heavens with clouds, \\
\poemll    preparing rain for the earth \\
\poemlll       and making grass grow on the hills. \\
\poeml \v{9}He gives wild animals their food, \\
\poemll    including the young ravens when they cry. \\
\poeml \v{10}He takes no delight in the strength of a horse, \\
\poemll    and gains no pleasure in the runner's swiftness.\fnote{Lit. \fbib{the legs of a man}} \\
\poeml \v{11}But the \divine{Lord} is pleased with those who fear him, \\
\poemll    with those who depend on his gracious love. \\
\poeml \v{12}Glorify the \divine{Lord}, Jerusalem! \\
\poemll    Praise your God, Zion! \\
\poeml \v{13}For he has strengthened the bars of your gates, \\
\poemll    blessing your children within you. \\
\poeml \v{14}He grants peace within your borders, \\
\poemll    satisfying\fnote{So MT; LXX DSS 4QPs\textsuperscript{d} read \fbib{borders, and satisfies}} you with the finest of wheat. \\
\poeml \v{15}He sends out his command to the earth, \\
\poemll    making\fnote{The Heb. lacks \fbib{making}} his word go forth quickly. \\
\poeml \v{16}He supplies snow like wool, \\
\poemll    scattering frost like ashes. \\
\poeml \v{17}He casts down his ice crystals like bread\fnote{The Heb. lacks \fbib{bread}} fragments. \\
\poemll    Who can endure his freezing cold? \\
\poeml \v{18}He sends out his word \\
\poemll    and melts them. \\
\poeml He makes his wind blow \\
\poemll    and the water flows. \\
\poeml \v{19}He declares his words to Jacob, \\
\poemll    his statutes and decrees to Israel. \\
\poeml \v{20}He has not dealt with any other nation like this; \\
\poemll    they never knew\fnote{So MT; LXX reads \fbib{he did not explain to them}; Syr Targ DSS 11QPs\textsuperscript{a} read \fbib{he has not revealed to them}} his decrees. \\
\poeml Hallelujah!
\end{poetry}
\labelpsalm{148}
\passage{Let All the Earth Praise the \divine{Lord}}

\begin{poetry}
\poeml \v{1}Hallelujah! \\
\poemll    Praise the \divine{Lord} from heaven; \\
\poemlll       praise him in the highest places. \\
\poeml \v{2}Praise him, all his angels; \\
\poemll    praise him, all his armies! \\
\poeml \v{3}Praise him, sun and moon; \\
\poemll    praise him, all you shining stars.\fnote{Lit. \fbib{you stars of light}} \\
\poeml \v{4}Praise him, you heaven of heavens, \\
\poemll    and you waters above the heavens. \\
\poeml \v{5}Let them praise the name of the \divine{Lord}, \\
\poemll    for he himself gave the command that they be created. \\
\poeml \v{6}He set them in place to last forever and ever; \\
\poemll    he gave the command and will not rescind it. \\
\poeml \v{7}Praise the \divine{Lord}, you from the earth, \\
\poemll    you creatures of the sea \\
\poemlll       and all you depths, \\
\poeml \v{8}fire, hail, snow, fog, and wind storm \\
\poemll    that carry out his command,\fnote{Or \fbib{word}} \\
\poeml \v{9}mountains and every hill, \\
\poemll    fruit trees and cedars, \\
\poeml \v{10}living creatures and livestock, \\
\poemll    insects and flying birds, \\
\poeml \v{11}earthly kings and all peoples, \\
\poemll    nobles and all court officials of the earth, \\
\poeml \v{12}young men and young women alike, \\
\poemll    along with older people and children. \\
\poeml \v{13}Let them praise the name of the \divine{Lord}, \\
\poemll    for his name alone is lifted up; \\
\poemlll       his majesty transcends earth and heaven. \\
\poeml \v{14}He has raised up a source of strength\fnote{Lit. \fbib{a horn}} for his people, \\
\poemll    an object of praise for all of his holy ones, \\
\poemlll       that is, for the people of Israel who are near him. \\
\poeml Hallelujah!
\end{poetry}
\labelpsalm{149}
\passage{A Song About Rejoicing in God}

\begin{poetry}
\poeml \v{1}Hallelujah! \\
\poemll    Sing a new song to the \divine{Lord}, \\
\poemlll       praising him where the godly gather together. \\
\poeml \v{2}May Israel rejoice in its Maker, \\
\poemll    and Zion's descendants in their King! \\
\poeml \v{3}May they praise his name with dancing, \\
\poemll    chanting songs to him with tambourines and lyres. \\
\poeml \v{4}For the \divine{Lord} is pleased with his people; \\
\poemll    he beautifies the afflicted with salvation. \\
\poeml \v{5}May those he loves be exalted, \\
\poemll    singing for joy on their couches. \\
\poeml \v{6}Let high praises to God be heard\fnote{The Heb. lacks \fbib{heard}} in their throats, \\
\poemll    while they wield two-edged swords in their hands \\
\poeml \v{7}as they bring retribution to nations \\
\poemll    and punishment to peoples, \\
\poeml \v{8}binding their kings with chains, \\
\poemll    their officials with iron bands, \\
\poeml \v{9}and executing the judgment written against them. \\
\poeml This is honor for all the ones he loves. \\
\poeml Hallelujah!
\end{poetry}
\labelpsalm{150}
\passage{A Psalm of Praise}

\begin{poetry}
\poeml \v{1}Hallelujah! \\
\poeml Praise God in his Holy Place. \\
\poemll    Praise him in his great expanse. \\
\poeml \v{2}Praise him for his mighty works. \\
\poemll    Praise him according to his excellent greatness. \\
\poeml \v{3}Praise him with trumpet sounding. \\
\poemll    Praise him with stringed instrument and harp. \\
\poeml \v{4}Praise him with tambourine and dancing. \\
\poemll    Praise him with stringed and wind instruments. \\
\poeml \v{5}Praise him with loud cymbals. \\
\poemll    Praise him with reverberating cymbals. \\
\poeml \v{6}Let everyone who breathes praise the \divine{Lord}. \\
\poeml Hallelujah!
\end{poetry}
